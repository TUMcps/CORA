\documentclass[crop=true]{standalone}

% tikz
\usepackage{tikzscale}
\usepackage{pgfplots}
\usepackage{mathtools}
\pgfplotsset{compat=1.18}
% \usepgfplotslibrary{external}
% \tikzexternalize[prefix=./]
% \newcommand{\includetikz}[1]{%
%     \tikzsetnextfilename{#1}%
%     \input{#1.tikz}%
% }
% explicitly set line width to avoid different widths in different pdf viewers
\tikzset{every picture/.style={line width=0.5pt}}
% groupplots
\usepgfplotslibrary{groupplots}
\usetikzlibrary{pgfplots.groupplots}
\usetikzlibrary{matrix}
\usetikzlibrary{patterns}

\begin{document}
%
% HOWTO: Convert a Matlab figure to svg via tikz
%
% 1. Create and save Matlab figure      								savefig('<figure-path>/<figure-name>.fig')
% 2. Convert fig to tikz:										        convertToTikz('<figure-path>/<figure-name>.fig')
% 3. Update <figure-path>/<figure-name> in the \input command below
% 4. Create dvi of figure:										        pdflatex -output-format=dvi main
% 5. Convert dvi to svg:										        dvisvgm main.dvi --font-format=woff --optimize -o ./figures/dynamics-systems/feedback_neuralNetContrSys.svg
% 6. Add to website         										    <img src="<figure-path>/<figure-name>.svg"/>
%
% \input{./<figure-path>/<figure-name>.tikz}
% This file was created by matlab2tikz.
%
%The latest updates can be retrieved from
%  http://www.mathworks.com/matlabcentral/fileexchange/22022-matlab2tikz-matlab2tikz
%where you can also make suggestions and rate matlab2tikz.
%
\definecolor{mycolor1}{rgb}{0.00000,0.44700,0.74100}%
%
\begin{tikzpicture}

\begin{axis}[%
width=4cm,
height=4cm,
at={(0.0in,0.0in)},
scale only axis,
xmin=-2.4,
xmax=2.4,
xtick={\empty},
ymin=-2.4,
ymax=2.4,
ytick={\empty},
axis background/.style={fill=white}
]

\addplot[area legend, draw=mycolor1, fill=mycolor1, fill opacity=0.2, forget plot]
table[row sep=crcr] {%
x	y\\
-2	-2\\
-2	0\\
0	2\\
2	2\\
2	0\\
0	-2\\
-2	-2\\
}--cycle;
\end{axis}

\begin{axis}[%
width=7.427in,
height=9.479in,
at={(0in,0in)},
scale only axis,
xmin=0,
xmax=1,
ymin=0,
ymax=1,
axis line style={draw=none},
ticks=none,
axis x line*=bottom,
axis y line*=left
]
\end{axis}
\end{tikzpicture}%

%
% Note: Make sure that there are no empty lines within the document environment for standalone to properly crop the image.
%
\end{document}