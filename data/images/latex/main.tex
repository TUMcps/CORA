\documentclass[crop=true]{standalone}

% tikz
\usepackage{tikzscale}
\usepackage{pgfplots}
\usepackage{mathtools}
\pgfplotsset{compat=1.18}
% \usepgfplotslibrary{external}
% \tikzexternalize[prefix=./]
% \newcommand{\includetikz}[1]{%
%     \tikzsetnextfilename{#1}%
%     \input{#1.tikz}%
% }
% explicitly set line width to avoid different widths in different pdf viewers
\tikzset{every picture/.style={line width=0.5pt}}
% groupplots
\usepgfplotslibrary{groupplots}
\usetikzlibrary{pgfplots.groupplots}
\usetikzlibrary{matrix}
\usetikzlibrary{patterns}

\begin{document}
%
% HOWTO: Convert a Matlab figure to svg via tikz
%
% 1. Create and save Matlab figure      								savefig('<figure-path>/<figure-name>.fig')
% 2. Convert fig to tikz:										        convertToTikz('<figure-path>/<figure-name>.fig')
% 3. Update <figure-path>/<figure-name> in the \input command below
% 4. Create dvi of figure:										        pdflatex -output-format=dvi main
% 5. Convert dvi to svg:										        dvisvgm main.dvi --font-format=woff --optimize -o ./figures/geometric-sets/conHyperplane.svg
% 6. Add to website         										    <img src="<figure-path>/<figure-name>.svg"/>
%
% \input{./<figure-path>/<figure-name>.tikz}

\definecolor{mycolor1}{rgb}{0.27060,0.58820,1.00000}%

\begin{tikzpicture}
\footnotesize

\path [fill, pattern=north east lines, pattern color=black] (0,0) -- (5,0) -- (5,-1) -- (0,-1);
\draw [thick] (0,0) -- (5,0);

\draw [thick,->] (1,0) -- (1,4);
\draw [thick] (0.75,4) -- (1.5,4) node[anchor=west] {$s_0$};

\draw (3,4) node[circle,fill=mycolor1,minimum size=1cm] {};
\draw [thick,->] (3,4) -- (3,5) node[anchor=south] {$v_0$};

\draw [thick] (3.75,2) -- (4.25,2);
\draw [thick,->] (4,2) -- (4,1) node[anchor=north] {$g$};

\end{tikzpicture}%
%
% Note: Make sure that there are no empty lines within the document environment for standalone to properly crop the image.
%
\end{document}