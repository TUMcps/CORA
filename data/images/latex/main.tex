\documentclass[crop=true]{standalone}

% tikz
\usepackage{tikzscale}
\usepackage{pgfplots}
\usepackage{mathtools}
\pgfplotsset{compat=1.18}
% \usepgfplotslibrary{external}
% \tikzexternalize[prefix=./]
% \newcommand{\includetikz}[1]{%
%     \tikzsetnextfilename{#1}%
%     \input{#1.tikz}%
% }
% explicitly set line width to avoid different widths in different pdf viewers
\tikzset{every picture/.style={line width=0.5pt}}
% groupplots
\usepgfplotslibrary{groupplots}
\usetikzlibrary{pgfplots.groupplots}
\usetikzlibrary{matrix}

\begin{document}
%
% HOWTO: Convert a Matlab figure to svg via tikz
%
% 1. Create and save Matlab figure      								savefig('<figure-path>/<figure-name>.fig')
% 2. Convert fig to tikz:										convertToTikz('<figure-path>/<figure-name>.fig')
% 3. Update <figure-path>/<figure-name> in the \input command below
% 4. Create dvi of figure:										pdflatex -output-format=dvi main
% 5. Convert dvi to svg:										dvisvgm main.dvi --font-format=woff --optimize -o ./figures/setBasedAlgorithm.svg
% 6. Add to website         										<img src="<figure-path>/<figure-name>.svg"/>
%
% \input{./<figure-path>/<figure-name>.tikz}
% This file was created by matlab2tikz.
%
\definecolor{mycolor1}{rgb}{0.00000,0.44700,0.74100}%
%
\begin{tikzpicture}
\footnotesize

\begin{axis}[%
width=8cm,
height=4.5cm,
at={(0in,0in)},
scale only axis,
xmin=-3,
xmax=6,
ymin=-4,
ymax=9,
axis background/.style={fill=white}
]
\addplot [color=mycolor1, forget plot]
  table[row sep=crcr]{%
-1	6\\
1	6\\
1	8\\
-1	8\\
-1	6\\
};
\addplot [color=mycolor1, forget plot]
  table[row sep=crcr]{%
0.7627	5.3638\\
2.4746	4.8076\\
2.5746	4.8076\\
3.1309	6.5195\\
3.1309	6.6195\\
1.419	7.1757\\
1.319	7.1757\\
0.7627	5.4638\\
0.7627	5.3638\\
};
\addplot [color=mycolor1, forget plot]
  table[row sep=crcr]{%
2.0946	4.329\\
3.4052	3.3768\\
3.4908	3.349\\
3.5908	3.349\\
4.543	4.6596\\
4.5709	4.7452\\
4.5709	4.8452\\
3.2602	5.7974\\
3.1747	5.8252\\
3.0747	5.8252\\
2.1224	4.5146\\
2.0946	4.429\\
2.0946	4.329\\
};
\addplot [color=mycolor1, forget plot]
  table[row sep=crcr]{%
2.9469	3.0729\\
3.8039	1.8933\\
3.8694	1.8457\\
3.955	1.8179\\
4.055	1.8179\\
5.2345	2.6749\\
5.2821	2.7404\\
5.31	2.826\\
5.31	2.926\\
4.453	4.1056\\
4.3874	4.1532\\
4.3018	4.181\\
4.2018	4.181\\
3.0223	3.324\\
2.9747	3.2585\\
2.9469	3.1729\\
2.9469	3.0729\\
};
\addplot [color=mycolor1, forget plot]
  table[row sep=crcr]{%
3.327	1.7607\\
3.7325	0.5127\\
3.7753	0.4537\\
3.8409	0.4061\\
3.9265	0.3783\\
4.0265	0.3783\\
5.2744	0.7838\\
5.3334	0.8266\\
5.381	0.8922\\
5.4088	0.9778\\
5.4088	1.0778\\
5.0033	2.3257\\
4.9605	2.3847\\
4.895	2.4323\\
4.8094	2.4601\\
4.7094	2.4601\\
3.4614	2.0547\\
3.4024	2.0118\\
3.3548	1.9463\\
3.327	1.8607\\
3.327	1.7607\\
};
\addplot [color=mycolor1, forget plot]
  table[row sep=crcr]{%
3.2874	-0.6492\\
3.3077	-0.7116\\
3.3505	-0.7706\\
3.4161	-0.8182\\
3.5017	-0.846\\
4.7826	-0.846\\
4.845	-0.8257\\
4.904	-0.7829\\
4.9516	-0.7174\\
4.9794	-0.6318\\
4.9794	0.6492\\
4.9592	0.7116\\
4.9163	0.7706\\
4.8508	0.8182\\
4.7652	0.846\\
3.4842	0.846\\
3.4218	0.8257\\
3.3628	0.7829\\
3.3152	0.7174\\
3.2874	0.6318\\
3.2874	-0.6492\\
};
\addplot [color=mycolor1, forget plot]
  table[row sep=crcr]{%
2.5833	-1.579\\
2.6036	-1.6414\\
2.6464	-1.7004\\
2.712	-1.748\\
3.8084	-2.1043\\
3.9675	-2.1043\\
4.0299	-2.084\\
4.0888	-2.0411\\
4.1365	-1.9756\\
4.4927	-0.8792\\
4.4927	-0.7201\\
4.4724	-0.6577\\
4.4296	-0.5987\\
4.3641	-0.5511\\
3.2676	-0.1949\\
3.1086	-0.1949\\
3.0462	-0.2151\\
2.9872	-0.258\\
2.9396	-0.3235\\
2.5833	-1.42\\
2.5833	-1.579\\
};
\addplot [color=mycolor1, forget plot]
  table[row sep=crcr]{%
1.722	-2.1791\\
1.7423	-2.2415\\
1.7852	-2.3005\\
2.6246	-2.9103\\
2.7607	-2.9546\\
2.9198	-2.9546\\
2.9822	-2.9343\\
3.0412	-2.8914\\
3.651	-2.052\\
3.6953	-1.9159\\
3.6953	-1.7568\\
3.675	-1.6944\\
3.6321	-1.6354\\
2.7927	-1.0256\\
2.6566	-0.9813\\
2.4975	-0.9813\\
2.4351	-1.0016\\
2.3762	-1.0445\\
1.7663	-1.8839\\
1.722	-2.02\\
1.722	-2.1791\\
};
\addplot [color=mycolor1, forget plot]
  table[row sep=crcr]{%
0.8179	-2.4532\\
0.8382	-2.5156\\
1.3871	-3.271\\
1.4913	-3.3468\\
1.6275	-3.391\\
1.7865	-3.391\\
1.8489	-3.3707\\
2.6044	-2.8218\\
2.6801	-2.7176\\
2.7244	-2.5815\\
2.7244	-2.4224\\
2.7041	-2.36\\
2.1552	-1.6045\\
2.051	-1.5288\\
1.9148	-1.4846\\
1.7558	-1.4846\\
1.6934	-1.5048\\
0.9379	-2.0537\\
0.8622	-2.158\\
0.8179	-2.2941\\
0.8179	-2.4532\\
};
\addplot [color=mycolor1, forget plot]
  table[row sep=crcr]{%
-0.0321	-2.4363\\
0.2276	-3.2356\\
0.2957	-3.3294\\
0.3999	-3.4052\\
0.5361	-3.4494\\
0.6951	-3.4494\\
1.4945	-3.1897\\
1.5883	-3.1215\\
1.664	-3.0173\\
1.7082	-2.8812\\
1.7082	-2.7221\\
1.4485	-1.9228\\
1.3803	-1.829\\
1.2761	-1.7533\\
1.14	-1.709\\
0.9809	-1.709\\
0.1816	-1.9687\\
0.0878	-2.0369\\
0.0121	-2.1411\\
-0.0321	-2.2773\\
-0.0321	-2.4363\\
};
\addplot [color=mycolor1, forget plot]
  table[row sep=crcr]{%
-0.7551	-2.8828\\
-0.7228	-2.9821\\
-0.6547	-3.0759\\
-0.5505	-3.1516\\
-0.4143	-3.1958\\
0.4421	-3.1958\\
0.5413	-3.1636\\
0.6351	-3.0954\\
0.7109	-2.9912\\
0.7551	-2.8551\\
0.7551	-1.9987\\
0.7228	-1.8994\\
0.6547	-1.8056\\
0.5505	-1.7299\\
0.4143	-1.6857\\
-0.4421	-1.6857\\
-0.5413	-1.7179\\
-0.6351	-1.7861\\
-0.7109	-1.8903\\
-0.7551	-2.0264\\
-0.7551	-2.8828\\
};
\addplot [color=mycolor1, forget plot]
  table[row sep=crcr]{%
-1.4981	-2.4015\\
-1.4658	-2.5007\\
-1.3977	-2.5945\\
-1.2935	-2.6702\\
-0.5604	-2.9084\\
-0.3665	-2.9084\\
-0.2672	-2.8762\\
-0.1734	-2.808\\
-0.0977	-2.7038\\
0.1405	-1.9708\\
0.1405	-1.7768\\
0.1082	-1.6776\\
0.0401	-1.5838\\
-0.0642	-1.5081\\
-0.7972	-1.2699\\
-0.9911	-1.2699\\
-1.0904	-1.3021\\
-1.1842	-1.3703\\
-1.2599	-1.4745\\
-1.4981	-2.2076\\
-1.4981	-2.4015\\
};
\addplot [color=mycolor1, forget plot]
  table[row sep=crcr]{%
-2.0002	-1.7828\\
-1.9679	-1.8821\\
-1.8998	-1.9759\\
-1.3386	-2.3836\\
-1.1726	-2.4375\\
-0.9787	-2.4375\\
-0.8794	-2.4053\\
-0.7856	-2.3371\\
-0.3779	-1.7759\\
-0.3239	-1.61\\
-0.3239	-1.416\\
-0.3562	-1.3168\\
-0.4243	-1.223\\
-0.9855	-0.8153\\
-1.1515	-0.7613\\
-1.3454	-0.7613\\
-1.4447	-0.7936\\
-1.5385	-0.8617\\
-1.9462	-1.4229\\
-2.0002	-1.5889\\
-2.0002	-1.7828\\
};
\addplot [color=mycolor1, forget plot]
  table[row sep=crcr]{%
-2.2579	-1.1137\\
-2.2256	-1.2129\\
-1.8587	-1.718\\
-1.7316	-1.8103\\
-1.5656	-1.8642\\
-1.3717	-1.8642\\
-1.2724	-1.832\\
-0.7674	-1.465\\
-0.675	-1.3379\\
-0.6211	-1.172\\
-0.6211	-0.978\\
-0.6533	-0.8788\\
-1.0203	-0.3737\\
-1.1474	-0.2814\\
-1.3134	-0.2275\\
-1.5073	-0.2275\\
-1.6065	-0.2597\\
-2.1116	-0.6267\\
-2.2039	-0.7538\\
-2.2579	-0.9197\\
-2.2579	-1.1137\\
};
\addplot [color=mycolor1, forget plot]
  table[row sep=crcr]{%
-2.2924	-0.4692\\
-2.1187	-1.0036\\
-2.0356	-1.118\\
-1.9086	-1.2103\\
-1.7426	-1.2642\\
-1.5487	-1.2642\\
-1.0143	-1.0906\\
-0.8999	-1.0075\\
-0.8076	-0.8804\\
-0.7536	-0.7144\\
-0.7536	-0.5205\\
-0.9273	0.0139\\
-1.0104	0.1282\\
-1.1374	0.2206\\
-1.3034	0.2745\\
-1.4973	0.2745\\
-2.0317	0.1009\\
-2.1461	0.0178\\
-2.2384	-0.1093\\
-2.2924	-0.2753\\
-2.2924	-0.4692\\
};
\addplot [color=mycolor1, forget plot]
  table[row sep=crcr]{%
-2.1426	-0.3198\\
-2.1033	-0.4408\\
-2.0202	-0.5551\\
-1.8932	-0.6475\\
-1.7272	-0.7014\\
-1.1215	-0.7014\\
-1.0005	-0.6621\\
-0.8861	-0.579\\
-0.7938	-0.4519\\
-0.7398	-0.2859\\
-0.7398	0.3198\\
-0.7792	0.4408\\
-0.8623	0.5551\\
-0.9893	0.6475\\
-1.1553	0.7014\\
-1.761	0.7014\\
-1.882	0.6621\\
-1.9964	0.579\\
-2.0887	0.4519\\
-2.1426	0.2859\\
-2.1426	-0.3198\\
};
\addplot [color=mycolor1, forget plot]
  table[row sep=crcr]{%
-1.9729	0.1577\\
-1.9336	0.0367\\
-1.8505	-0.0777\\
-1.7234	-0.17\\
-1.205	-0.3385\\
-0.9905	-0.3385\\
-0.8695	-0.2991\\
-0.7551	-0.2161\\
-0.6628	-0.089\\
-0.4943	0.4295\\
-0.4943	0.644\\
-0.5337	0.765\\
-0.6167	0.8793\\
-0.7438	0.9717\\
-1.2623	1.1401\\
-1.4768	1.1401\\
-1.5978	1.1008\\
-1.7121	1.0177\\
-1.8045	0.8906\\
-1.9729	0.3722\\
-1.9729	0.1577\\
};
\addplot [color=mycolor1, forget plot]
  table[row sep=crcr]{%
-1.6949	0.5192\\
-1.6556	0.3982\\
-1.5725	0.2838\\
-1.1755	-0.0046\\
-0.9919	-0.0642\\
-0.7774	-0.0642\\
-0.6564	-0.0249\\
-0.5421	0.0582\\
-0.2537	0.4551\\
-0.194	0.6387\\
-0.194	0.8532\\
-0.2333	0.9742\\
-0.3164	1.0886\\
-0.7134	1.3769\\
-0.897	1.4366\\
-1.1115	1.4366\\
-1.2325	1.3973\\
-1.3468	1.3142\\
-1.6352	0.9173\\
-1.6949	0.7337\\
-1.6949	0.5192\\
};
\addplot [color=mycolor1, forget plot]
  table[row sep=crcr]{%
-1.3563	0.7512\\
-1.317	0.6302\\
-1.0575	0.273\\
-0.9169	0.1709\\
-0.7333	0.1112\\
-0.5188	0.1112\\
-0.3978	0.1505\\
-0.0406	0.4101\\
0.0616	0.5507\\
0.1212	0.7343\\
0.1212	0.9488\\
0.0819	1.0698\\
-0.1776	1.427\\
-0.3182	1.5291\\
-0.5018	1.5888\\
-0.7163	1.5888\\
-0.8373	1.5495\\
-1.1946	1.2899\\
-1.2967	1.1494\\
-1.3563	0.9657\\
-1.3563	0.7512\\
};
\addplot [color=mycolor1, forget plot]
  table[row sep=crcr]{%
-1.002	0.8557\\
-0.8792	0.4778\\
-0.7873	0.3513\\
-0.6467	0.2492\\
-0.4631	0.1895\\
-0.2486	0.1895\\
0.1293	0.3123\\
0.2558	0.4042\\
0.358	0.5448\\
0.4176	0.7284\\
0.4176	0.9429\\
0.2948	1.3208\\
0.2029	1.4474\\
0.0623	1.5495\\
-0.1213	1.6091\\
-0.3358	1.6091\\
-0.7137	1.4863\\
-0.8402	1.3944\\
-0.9424	1.2539\\
-1.002	1.0702\\
-1.002	0.8557\\
};
\addplot [color=mycolor1, forget plot]
  table[row sep=crcr]{%
-0.6697	0.6035\\
-0.6262	0.4696\\
-0.5343	0.3431\\
-0.3937	0.241\\
-0.2101	0.1813\\
0.2475	0.1813\\
0.3814	0.2248\\
0.5079	0.3168\\
0.61	0.4573\\
0.6697	0.6409\\
0.6697	1.0986\\
0.6262	1.2324\\
0.5343	1.3589\\
0.3937	1.4611\\
0.2101	1.5207\\
-0.2475	1.5207\\
-0.3814	1.4772\\
-0.5079	1.3853\\
-0.61	1.2448\\
-0.6697	1.0611\\
-0.6697	0.6035\\
};
\addplot [color=red, forget plot]
  table[row sep=crcr]{%
0.937	6\\
1	6\\
1	6.194\\
0.937	6\\
};
\addplot [color=red, forget plot]
  table[row sep=crcr]{%
2.3619	4.8442\\
2.4746	4.8076\\
2.5746	4.8076\\
2.7683	5.4036\\
2.3619	4.8442\\
};
\addplot [color=red, forget plot]
  table[row sep=crcr]{%
3.2501	3.4895\\
3.4052	3.3768\\
3.4908	3.349\\
3.5908	3.349\\
4.188	4.171\\
3.2501	3.4895\\
};
\addplot [color=red, forget plot]
  table[row sep=crcr]{%
3.6436	2.1139\\
3.8039	1.8933\\
3.8694	1.8457\\
3.955	1.8179\\
4.055	1.8179\\
4.8978	2.4303\\
4.895	2.4323\\
4.8094	2.4601\\
4.7094	2.4601\\
3.6436	2.1139\\
};
\addplot [color=red, forget plot]
  table[row sep=crcr]{%
3.6242	0.846\\
3.7325	0.5127\\
3.7753	0.4537\\
3.8409	0.4061\\
3.9265	0.3783\\
4.0265	0.3783\\
4.9681	0.6842\\
4.9592	0.7116\\
4.9163	0.7706\\
4.8508	0.8182\\
4.7652	0.846\\
3.6242	0.846\\
};
\addplot [color=red, forget plot]
  table[row sep=crcr]{%
3.2874	-0.6492\\
3.3077	-0.7116\\
3.3505	-0.7706\\
3.4161	-0.8182\\
3.5017	-0.846\\
4.4927	-0.846\\
4.4927	-0.7201\\
4.4724	-0.6577\\
4.4296	-0.5987\\
4.3641	-0.5511\\
3.2874	-0.2013\\
3.2874	-0.6492\\
};
\addplot [color=red, forget plot]
  table[row sep=crcr]{%
2.5833	-1.579\\
2.6036	-1.6414\\
2.6464	-1.7004\\
2.712	-1.748\\
3.6504	-2.0529\\
3.651	-2.052\\
3.6953	-1.9159\\
3.6953	-1.7568\\
3.675	-1.6944\\
3.6321	-1.6354\\
2.7927	-1.0256\\
2.7192	-1.0017\\
2.5833	-1.42\\
2.5833	-1.579\\
};
\addplot [color=red, forget plot]
  table[row sep=crcr]{%
1.722	-2.1791\\
1.7423	-2.2415\\
1.7852	-2.3005\\
2.5536	-2.8588\\
2.6044	-2.8218\\
2.6801	-2.7176\\
2.7244	-2.5815\\
2.7244	-2.4224\\
2.7041	-2.36\\
2.1552	-1.6045\\
2.051	-1.5288\\
2.0294	-1.5218\\
1.7663	-1.8839\\
1.722	-2.02\\
1.722	-2.1791\\
};
\addplot [color=red, forget plot]
  table[row sep=crcr]{%
0.8179	-2.4532\\
0.8382	-2.5156\\
1.3598	-3.2334\\
1.4945	-3.1897\\
1.5883	-3.1215\\
1.664	-3.0173\\
1.7082	-2.8812\\
1.7082	-2.7221\\
1.4485	-1.9228\\
1.3803	-1.829\\
1.3138	-1.7806\\
0.9379	-2.0537\\
0.8622	-2.158\\
0.8179	-2.2941\\
0.8179	-2.4532\\
};
\addplot [color=red, forget plot]
  table[row sep=crcr]{%
-0.0321	-2.4363\\
0.2146	-3.1958\\
0.4421	-3.1958\\
0.5413	-3.1636\\
0.6351	-3.0954\\
0.7109	-2.9912\\
0.7551	-2.8551\\
0.7551	-1.9987\\
0.7228	-1.8994\\
0.6602	-1.8132\\
0.1816	-1.9687\\
0.0878	-2.0369\\
0.0121	-2.1411\\
-0.0321	-2.2773\\
-0.0321	-2.4363\\
};
\addplot [color=red, forget plot]
  table[row sep=crcr]{%
-0.7551	-2.8452\\
-0.5604	-2.9084\\
-0.3665	-2.9084\\
-0.2672	-2.8762\\
-0.1734	-2.808\\
-0.0977	-2.7038\\
0.1405	-1.9708\\
0.1405	-1.7768\\
0.1108	-1.6857\\
-0.4421	-1.6857\\
-0.5413	-1.7179\\
-0.6351	-1.7861\\
-0.7109	-1.8903\\
-0.7551	-2.0264\\
-0.7551	-2.8452\\
};
\addplot [color=red, forget plot]
  table[row sep=crcr]{%
-1.4981	-2.2677\\
-1.3386	-2.3836\\
-1.1726	-2.4375\\
-0.9787	-2.4375\\
-0.8794	-2.4053\\
-0.7856	-2.3371\\
-0.3779	-1.7759\\
-0.3239	-1.61\\
-0.3239	-1.4237\\
-0.7972	-1.2699\\
-0.9911	-1.2699\\
-1.0904	-1.3021\\
-1.1842	-1.3703\\
-1.2599	-1.4745\\
-1.4981	-2.2076\\
-1.4981	-2.2677\\
};
\addplot [color=red, forget plot]
  table[row sep=crcr]{%
-1.9854	-1.5435\\
-1.8587	-1.718\\
-1.7316	-1.8103\\
-1.5656	-1.8642\\
-1.3717	-1.8642\\
-1.2724	-1.832\\
-0.7674	-1.465\\
-0.675	-1.3379\\
-0.6211	-1.172\\
-0.6211	-1.08\\
-0.9855	-0.8153\\
-1.1515	-0.7613\\
-1.3454	-0.7613\\
-1.4447	-0.7936\\
-1.5385	-0.8617\\
-1.9462	-1.4229\\
-1.9854	-1.5435\\
};
\addplot [color=red, forget plot]
  table[row sep=crcr]{%
-2.2011	-0.7499\\
-2.1187	-1.0036\\
-2.0356	-1.118\\
-1.9086	-1.2103\\
-1.7426	-1.2642\\
-1.5487	-1.2642\\
-1.0143	-1.0906\\
-0.8999	-1.0075\\
-0.8076	-0.8804\\
-0.7596	-0.7326\\
-1.0203	-0.3737\\
-1.1474	-0.2814\\
-1.3134	-0.2275\\
-1.5073	-0.2275\\
-1.6065	-0.2597\\
-2.1116	-0.6267\\
-2.2011	-0.7499\\
};
\addplot [color=red, forget plot]
  table[row sep=crcr]{%
-2.1426	-0.3198\\
-2.1033	-0.4408\\
-2.0202	-0.5551\\
-1.8932	-0.6475\\
-1.7272	-0.7014\\
-1.1215	-0.7014\\
-1.0005	-0.6621\\
-0.8861	-0.579\\
-0.7938	-0.4519\\
-0.7849	-0.4245\\
-0.9273	0.0139\\
-1.0104	0.1282\\
-1.1374	0.2206\\
-1.3034	0.2745\\
-1.4973	0.2745\\
-2.0317	0.1009\\
-2.1426	0.0203\\
-2.1426	-0.3198\\
};
\addplot [color=red, forget plot]
  table[row sep=crcr]{%
-1.9729	0.1577\\
-1.9336	0.0367\\
-1.8505	-0.0777\\
-1.7234	-0.17\\
-1.205	-0.3385\\
-0.9905	-0.3385\\
-0.8695	-0.2991\\
-0.7551	-0.2161\\
-0.7398	-0.195\\
-0.7398	0.3198\\
-0.7792	0.4408\\
-0.8623	0.5551\\
-0.9893	0.6475\\
-1.1553	0.7014\\
-1.761	0.7014\\
-1.8783	0.6633\\
-1.9729	0.3722\\
-1.9729	0.1577\\
};
\addplot [color=red, forget plot]
  table[row sep=crcr]{%
-1.6949	0.5192\\
-1.6556	0.3982\\
-1.5725	0.2838\\
-1.1755	-0.0046\\
-0.9919	-0.0642\\
-0.7774	-0.0642\\
-0.6564	-0.0249\\
-0.6375	-0.0112\\
-0.4943	0.4295\\
-0.4943	0.644\\
-0.5337	0.765\\
-0.6167	0.8793\\
-0.7438	0.9717\\
-1.2623	1.1401\\
-1.4733	1.1401\\
-1.6352	0.9173\\
-1.6949	0.7337\\
-1.6949	0.5192\\
};
\addplot [color=red, forget plot]
  table[row sep=crcr]{%
-1.3563	0.7512\\
-1.317	0.6302\\
-1.0575	0.273\\
-0.9169	0.1709\\
-0.7333	0.1112\\
-0.5188	0.1112\\
-0.4988	0.1177\\
-0.2537	0.4551\\
-0.194	0.6387\\
-0.194	0.8532\\
-0.2333	0.9742\\
-0.3164	1.0886\\
-0.7134	1.3769\\
-0.897	1.4366\\
-0.9927	1.4366\\
-1.1946	1.2899\\
-1.2967	1.1494\\
-1.3563	0.9657\\
-1.3563	0.7512\\
};
\addplot [color=red, forget plot]
  table[row sep=crcr]{%
-1.002	0.8557\\
-0.8792	0.4778\\
-0.7873	0.3513\\
-0.6467	0.2492\\
-0.4631	0.1895\\
-0.3442	0.1895\\
-0.0406	0.4101\\
0.0616	0.5507\\
0.1212	0.7343\\
0.1212	0.9488\\
0.0819	1.0698\\
-0.1776	1.427\\
-0.3182	1.5291\\
-0.4501	1.572\\
-0.7137	1.4863\\
-0.8402	1.3944\\
-0.9424	1.2539\\
-1.002	1.0702\\
-1.002	0.8557\\
};
\addplot [color=red, forget plot]
  table[row sep=crcr]{%
-0.6697	0.6035\\
-0.6262	0.4696\\
-0.5343	0.3431\\
-0.3937	0.241\\
-0.2419	0.1917\\
0.1293	0.3123\\
0.2558	0.4042\\
0.358	0.5448\\
0.4176	0.7284\\
0.4176	0.9429\\
0.2948	1.3208\\
0.2029	1.4474\\
0.1019	1.5207\\
-0.2475	1.5207\\
-0.3814	1.4772\\
-0.5079	1.3853\\
-0.61	1.2448\\
-0.6697	1.0611\\
-0.6697	0.6035\\
};
\end{axis}
\end{tikzpicture}%
%
% Note: Make sure that there are no empty lines within the document environment for standalone to properly crop the image
%
\end{document}