\documentclass[crop=true]{standalone}

% tikz
\usepackage{tikzscale}
\usepackage{pgfplots}
\usepackage{mathtools}
\pgfplotsset{compat=1.18}
% \usepgfplotslibrary{external}
% \tikzexternalize[prefix=./]
% \newcommand{\includetikz}[1]{%
%     \tikzsetnextfilename{#1}%
%     \input{#1.tikz}%
% }
% explicitly set line width to avoid different widths in different pdf viewers
\tikzset{every picture/.style={line width=0.5pt}}
% groupplots
\usepgfplotslibrary{groupplots}
\usetikzlibrary{pgfplots.groupplots}
\usetikzlibrary{matrix}

\begin{document}
%
% HOWTO: Convert a Matlab figure to svg via tikz
%
% 1. Create and save Matlab figure      								savefig('<figure-path>/<figure-name>.fig')
% 2. Convert fig to tikz:										        convertToTikz('<figure-path>/<figure-name>.fig')
% 3. Update <figure-path>/<figure-name> in the \input command below
% 4. Create dvi of figure:										        pdflatex -output-format=dvi main
% 5. Convert dvi to svg:										        dvisvgm main.dvi --font-format=woff --optimize -o ./figures/geometric-sets/conHyperplane.svg
% 6. Add to website         										    <img src="<figure-path>/<figure-name>.svg"/>
%
% \input{./<figure-path>/<figure-name>.tikz}
% This file was created by matlab2tikz.
%
\definecolor{mycolor1}{rgb}{0.00000,0.44700,0.74100}%
\definecolor{mycolor2}{rgb}{0.85000,0.32500,0.09800}%
\definecolor{mycolor3}{rgb}{0.49400,0.18400,0.55600}%
\definecolor{mycolor4}{rgb}{0.46600,0.67400,0.18800}%
%
\begin{tikzpicture}
\footnotesize

\begin{axis}[%
width=8cm,
height=4.5cm,
at={(0in,0in)},
scale only axis,
xmin=-4,
xmax=3,
ymin=-4,
ymax=3,
axis background/.style={fill=white},
legend style={at={(0.03,0.97)}, anchor=north west, legend cell align=left, align=left, draw=white!15!black}
]
\addplot [color=mycolor1]
table[row sep=crcr]{%
  -1.5	-3.5\\
  0.5	-1.5\\
  -1.5	0.5\\
  -3.5	-1.5\\
  -1.5	-3.5\\
};
\addlegendentry{$\mathcal{S}$}

\addplot [color=mycolor2]
  table[row sep=crcr]{%
0.5	0.5\\
2.5	0.5\\
2.5	2.5\\
0.5	2.5\\
0.5	0.5\\
};
\addlegendentry{$M\mathcal{S}$}

\addplot [color=mycolor3, dashed]
  table[row sep=crcr]{%
2.5	2.5\\
0.5	2.5\\
-3.5	-1.5\\
-1.5	-3.5\\
2.5	0.5\\
2.5	2.5\\
};
\addlegendentry{$\texttt{convHull}(S,M\mathcal{S})$}

\addplot [color=mycolor4]
  table[row sep=crcr]{%
2.6	2.6\\
0.4	2.6\\
-3.6	-1.4\\
-3.6	-1.6\\
-1.6	-3.6\\
-1.4	-3.6\\
2.6	0.4\\
2.6	2.6\\
};
\addlegendentry{$\texttt{convHull}(S,M\mathcal{S}) \oplus \mathcal{I}$}

\end{axis}
\end{tikzpicture}%
%
% Note: Make sure that there are no empty lines within the document environment for standalone to properly crop the image.
%
\end{document}