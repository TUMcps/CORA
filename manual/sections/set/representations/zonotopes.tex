\subsubsubsection{Zonotopes} \label{sec:zonotope}

A zonotope $\mathcal{Z} \subset \mathbb{R}^n$ is defined as
\begin{equation} \label{eq:zonotope}
\mathcal{Z} := \bigg \{ c + \sum_{i=1}^{p} \beta_i g\^{i} ~\bigg|~ \beta_i \in [-1,1] \bigg\},
\end{equation}
where $c \in \mathbb{R}^n$ is the center and $g\^i \in \mathbb{R}^n$ are the generators. The zonotope order $\rho$ is defined as $\rho = \frac{p}{n}$ and represents a dimensionless measure for the representation size.

Zonotopes are represented in CORA by the class \texttt{zonotope}. An object of class \texttt{zonotope} can be constructed as follows:
\begin{equation*}
	\begin{split}
		& \mathcal{Z} = \texttt{zonotope}(c,G), \\
		& \mathcal{Z} = \texttt{zonotope}(Z),
	\end{split}
\end{equation*} 
where $G= [g^{(1)},\dots,g^{(p)}]$, $Z = [c,G]$, and $c,g^{(i)}$ are defined as in \eqref{eq:zonotope}. Let us demonstrate the construction of a zonotope by an example:

\begin{center}
\begin{minipage}[t]{0.35\textwidth}
	\vspace{20pt}
	\footnotesize
	% This file was automatically created from the m-file 
% "m2tex.m" written by USL. 
% The fontencoding in this file is UTF-8. 
%  
% You will need to include the following two packages in 
% your LaTeX-Main-File. 
%  
% \usepackage{color} 
% \usepackage{fancyvrb} 
%  
% It is advised to use the following option for Inputenc 
% \usepackage[utf8]{inputenc} 
%  
  
% definition of matlab colors: 
\definecolor{mblue}{rgb}{0,0,1} 
\definecolor{mgreen}{rgb}{0.13333,0.5451,0.13333} 
\definecolor{mred}{rgb}{0.62745,0.12549,0.94118} 
\definecolor{mgrey}{rgb}{0.5,0.5,0.5} 
\definecolor{mdarkgrey}{rgb}{0.25,0.25,0.25} 
  
\DefineShortVerb[fontfamily=courier,fontseries=m]{\$} 
\DefineShortVerb[fontfamily=courier,fontseries=b]{\#} 
  
\noindent     
 $$\color{mgreen}$% construct zonotope $\color{black}$$\\
 $c = [1;1];$\\
 $G = [1 1 1; 1 -1 0];$\\
 $$\\
 $zono = zonotope(c,G);$\\ 
  
\UndefineShortVerb{\$} 
\UndefineShortVerb{\#}
\end{minipage}
\begin{minipage}[t]{0.3\textwidth}
	\vspace{0pt}
	\centering
	\includetikz{./figures/tikz/set-representations/example_zonotope}
\end{minipage}
\end{center}

A more detailed example for zonotopes is provided in \cref{sec:zonotopeExample} and in the file \textit{examples/contSet/example\_zonotope.m} in the CORA toolbox.

A zonotope can be interpreted as the Minkowski addition of line segments $l\^i = [-1,1]g\^{i}$. The step-by-step construction of a two-dimesional zonotope is visualized in \cref{fig:zonotope}. Zonotopes are a compact representation of sets in high-dimensional space. More importantly, operations required for reachability analysis, such as linear maps (see \cref{sec:mtimes}) and Minkowski addition (see \cref{sec:plus}) can be computed efficiently and exactly, and others, such as convex hull computation (see \cref{sec:convHull}) can be tightly over-approximated \cite{Girard2005}.

\begin{figure}[htb]
	\centering
	\includetikz{./figures/tikz/set-representations/example_zonotope_construction}
	\caption{Step-by-step construction of a zonotope.}
	\label{fig:zonotope}
\end{figure}

In addition to the standard set operations described in \cref{sec:setOperations} and the methods for converting between set operations (see \cref{tab:setConversion}), the class \texttt{zonotope} supports additional methods which are listed in \cref{sec:zonotopeOperations}.