\subsubsubsection{Constrained Zonotopes} \label{sec:conZonotope}

An extension of zonotopes described in \cref{sec:zonotope} are constrained zonotopes, which are introduced in \cite{Scott2016}. A constrained zonotope is defined as a zonotope with additional equality constraints on the factors $\beta_i$:
\begin{equation} \label{eq:constrainedZonotope}
	\mathcal{Z}_c := \Big\{ c + G \beta ~\Big|~ \lVert \beta \rVert_{\infty} \leq 1, A \beta = b \Big\}, 
\end{equation}
where $c \in \mathbb{R}^n$ is the zonotope center, $G \in \mathbb{R}^{n \times p}$ is the zonotope generator matrix and $\beta \in \mathbb{R}^p$ is the vector of zonotope factors. The equality constraints are parametrized by the matrix $A \in \mathbb{R}^{q \times p}$ and the vector $b \in \mathbb{R}^q$. Constrained zonotopes are able to describe arbitrary polytopes, and are therefore a more general set representation than zonotopes. The main advantage compared to a polytope representation using inequality constraints (see \cref{sec:polytopes}) is that constrained zonotopes inherit the excellent scaling properties of zonotopes for increasing state-space dimensions, since constrained zonotopes are also based on a generator representation for sets.

Constrained zonotopes are represented in CORA by the class \texttt{conZonotope}. An object of class \texttt{conZonotope} can be constructed as follows:
\begin{equation*}
	\begin{split}
		& \mathcal{Z}_c = \texttt{conZonotope}(c,G,A,b), \\
		& \mathcal{Z}_c = \texttt{conZonotope}(Z,A,b),
	\end{split}
\end{equation*} 
where $Z = [c,G]$, and $c,G,A,b$ are defined as in \eqref{eq:constrainedZonotope}. Let us demonstrate the construction of a constrained zonotope by an example:

\begin{center}
\begin{minipage}[t]{0.5\textwidth}
	\vspace{20pt}
	\footnotesize
	\input{./MATLABcode/example_setRep_conZonotope}
\end{minipage}
\begin{minipage}[t]{0.3\textwidth}
    \vspace{0pt}
    \centering
    \includetikz{./figures/tikz/set-representations/example_conZonotope}
\end{minipage}
\end{center}

The unconstrained zonotope from this example is visualized in \cref{fig:conZonotope1}, and the equality constraints in \cref{fig:conZonotope2}.

\begin{figure}[h!tb]
\begin{minipage}{0.45\columnwidth}
  \centering
  %\footnotesize
  \includetikz{./figures/tikz/set-representations/example_conZonotope_zonotope}
  \caption{Zonotope (red) and the corresponding constrained zonotope (blue).}
  \label{fig:conZonotope1}
\end{minipage}
\hspace{0.08\columnwidth}
\begin{minipage}{0.45\columnwidth}
  \centering
  %\footnotesize
  \includetikz{./figures/tikz/set-representations/example_conZonotope_constraints}
  \caption{Visualization of the equality constraints of the constrained zonotope.}
  \label{fig:conZonotope2}
\end{minipage}
\end{figure}

A more detailed example for constrained zonotopes is provided in \cref{sec:conZonotopeExample} and in the file \textit{examples/contSet/example\_conZonotope.m} in the CORA toolbox. In addition to the standard set operations described in \cref{sec:setOperations} and the methods for converting between set operations (see \cref{tab:setConversion}), the class \texttt{conZonotope} supports additional methods, which are listed in \cref{sec:conZonotopeOperations}.