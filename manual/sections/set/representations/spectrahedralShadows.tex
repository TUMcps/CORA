\subsubsubsection{Spectrahedral Shadows} \label{sec:spectraShadow}

Spectrahedral shadows can be seen as the semidefinite generalization of polytopes, and can represent a large variety of convex sets. In particular, every convex set representation (e.g., zonotopes, intervals, ellipsoids, polytopes, capsules, zonotope bundles, and constrained zonotopes) implemented in CORA can be represented as a spectrahedral shadow. Formally, a spectrahedral shadow $\mathcal{SPS} \subseteq \Rn$ is defined as
\begin{equation} \label{eq:spectrahedralShadow}
	\mathcal{SPS} := \bigg \{ c + \sum_{i=1}^{p} \beta_i g\^{i} ~\bigg|~ A_{(0)} + \sum_{i=1}^p \beta_i A_{(i)} \succeq 0 \bigg\},
\end{equation}
where $c \in \mathbb{R}^n$ is the center vector, $g\^i \in \mathbb{R}^n$ are the generators, $A_{(i)}$ are symmetric coefficient matrices, and for a matrix $M$ the expression $M \succeq 0$ means that $M$ is positive semidefinite. If the matrix $G$ with columns $g\^i$ is the identity matrix, then $\mathcal{SPS}$ is also called a spectrahedron.

As shown in \cite[Lemma 4.1.5.]{Netzer2011}, non-empty spectrahedral shadows can equivalently be defined as
\begin{equation} \label{eq:spectrahedralShadow_existentialSum}
	\mathcal{SPS} := \bigg \{ x \in \Rn ~\bigg|~ \exists y \in \mathbb{R}^q, A_{(0)} + \sum_{i=1}^p x_i A_{(i)} + \sum_{j=1}^q y_j B_{(j)} \succeq 0 \bigg\},
\end{equation}
where $A_{(i)}$ and $B_{(j)}$ are symmetric coefficient matrices. We call this representation the existential sum representation.

Spectrahedral shadows are represented in CORA by the class \texttt{spectraShadow}. An object of class \texttt{spectraShadow} can be constructed as follows:
\begin{equation*}
	\begin{split}
		& \mathcal{SPS} = \texttt{spectraShadow}(A), \\
		& \mathcal{SPS} = \texttt{spectraShadow}(A,c), \\
		& \mathcal{SPS} = \texttt{spectraShadow}(A,c,G), \\
		& \mathcal{SPS} = \texttt{spectraShadow}(E), \\
	\end{split}
\end{equation*} 
where $G= [g^{(1)},\dots,g^{(p)}]$, with $c, g^{(i)}$ defined as in \eqref{eq:spectrahedralShadow} (if they are not specified, $c$ is chosen to be the origin, and $G$ the identity matrix). The matrix $A$ is the horizontal concatenation of the coefficient matrices $A_{(0)},\dots, A_{(p)}$, i.e., $A = \begin{bmatrix} A_{(0)} & A_{(1)} & \dots & A_{(p)}\end{bmatrix}$. On the other hand, $E$ is a $1\times 2$ cell array where the first element is a matrix $A$ that is the concatenation of the coefficient matrices matrices $A_{(0)},\dots, A_{(p)}$, i.e., $A = \begin{bmatrix} A_{(0)} & A_{(1)} & \dots & A_{(p)}\end{bmatrix}$, and the second element is a matrix $B$ that is the concatenation of the additional coefficient matrices $B_{(0)},\dots, B_{(q)}$, i.e., $B = \begin{bmatrix} B_{(0)} & B_{(1)} & \dots & B_{(q)}\end{bmatrix}$.
Let us demonstrate the construction of a spectrahedral shadow by an example:

\begin{center}
	\begin{minipage}[t]{0.5\textwidth}
		\vspace{20pt}
		\footnotesize
		% You will need to include the following two packages in 
% your LaTeX-Main-File. 
%  
% \usepackage{color} 
% \usepackage{fancyvrb} 
%  
% It is advised to use the following option for Inputenc 
% \usepackage[utf8]{inputenc} 
%  
  
% definition of matlab colors: 
\definecolor{mblue}{rgb}{0,0,1} 
\definecolor{mgreen}{rgb}{0.13333,0.5451,0.13333} 
\definecolor{mred}{rgb}{0.62745,0.12549,0.94118} 
\definecolor{mgrey}{rgb}{0.5,0.5,0.5} 
\definecolor{mdarkgrey}{rgb}{0.25,0.25,0.25} 
  
\DefineShortVerb[fontfamily=courier,fontseries=m]{\$} 
\DefineShortVerb[fontfamily=courier,fontseries=b]{\#} 
  
\noindent      
 $$\color{mgreen}$% create two simple$\color{black}$$\\
 $$\color{mgreen}$% spectrahedral shadows$\color{black}$$\\
 $$\color{mgreen}$% an ellipsoid, using the first$\color{black}$$\\
 $$\color{mgreen}$% type of instantiation$\color{black}$$\\
 $A0 = eye(3);$\\
 $A1 = [0 1 0; 1 0 0; 0 0 0];$\\
 $A2 = [0 0 1; 0 0 0; 1 0 0];$\\
 $A = [A0 A1 A2];$\\
 $c = [-1.5;0];$\\
 $G = [1 0;0 1.5];$\\
 $     $\\
 $SpS_ellipsoid = spectraShadow(A,c,G);$\\
 $     $\\
 $$\color{mgreen}$% a small zonotope, using the$\color{black}$$\\
 $$\color{mgreen}$% second type of instantiation$\color{black}$$\\
 $A0 = diag([-19 1 11 21 1 -9]);$\\
 $A1 = diag([10 10 0 -10 -10 0]);$\\
 $A2 = (10/3)*diag([1 -1 2 -1 1 -2]);$\\
 $A = [A0 A1 A2];$\\
 $$\\
 $B = 0.5774*diag([-1 1 1 1 -1 -1]);$\\
 $$\\
 $ESumRep = {A, B};$\\
 $$\\
 $SpS_zonotope = spectraShadow(ESumRep);$\\
 $$\\
 $$\color{mgreen}$% we can construct more complicated sets$\color{black}$$\\
 $$\color{mgreen}$% based on the two above, such as the$\color{black}$$\\
 $$\color{mgreen}$% convex hull of the two previous$\color{black}$$\\
 $$\color{mgreen}$% spectrahedral shadows$\color{black}$$\\
 $SpS = convHull(SpS_ellipsoid, SpS_zonotope);$\\
  
\UndefineShortVerb{\$} 
\UndefineShortVerb{\#}
	\end{minipage}
	\begin{minipage}[t]{0.3\textwidth}
		\vspace{0pt}
		\centering
		\includetikz{./figures/tikz/set-representations/example_spectraShadow}
	\end{minipage}
\end{center}

A more detailed example for spectrahedral shadows is provided in \cref{sec:spectraShadowExample} and in the file \textit{examples/contSet/example\_spectraShadow.m} in the CORA toolbox.