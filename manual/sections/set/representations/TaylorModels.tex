\subsubsubsection{Taylor Models} \label{sec:taylorModels}

Taylor models \cite{Berz1998, Makino1996, Makino2003, Makino2009} can be used to obtain rigorous bounds of functions that are often tighter than the ones obtained by interval arithmetic. A Taylor model $\mathcal{T}(x)$ is defined as
\begin{equation}
	\mathcal{T}(x) = \{ p(x) + y~|~ y \in \mathcal{I} \},
\end{equation}
where $p:~\R^p \to \Rn$ is a polynomial function and $\mathcal{I} \subset \Rn$ is an interval (see \cref{sec:interval}). For range bounding, the possible values for the variable $x$ are usually restricted by an interval domain $\mathcal{D} \subset \R^p$ (see \eqref{eq:rangeBounding}).

To enclose a nonlinear function with a Taylor model, a Taylor series expansion of the function is computed:
\begin{equation*}
	f(x) \approx f(x^*) + \frac{\partial f}{\partial x} \bigg|_{x^*} (x - x^*) + \frac{\partial^2 f}{\partial x^2} \bigg|_{x^*} (x - x^*)^2 + \dots\ .
\end{equation*}
Let us consider the nonlinear function $f(x) = \cos(x)$ as an example. By computing a second-order Taylor series expansion at the expansion point $x^* = 0$, the function $f(x)$ on the domain $x \in [-1,1]$ can be enclosed by the Taylor model
\begin{equation}
		\mathcal{T}(x) := \big \{ 1 - 0.5 x^2 + y~\big|~ y \in [-0.15,0.15] \big \},
		\label{eq:taylorExample}
\end{equation}
which is visualized in \cref{fig:taylorExample}.

\begin{figure}[h]	
	\centering
	\includetikz{./figures/tikz/set-representations-range/example_taylm}
	\caption{Function $f(x) = \cos(x)$ (black) and the enclosing Taylor model $\mathcal{T}(x)$ in \eqref{eq:taylorExample} (blue).}
	\label{fig:taylorExample}
\end{figure}

Taylor models are represented by the class \texttt{taylm}. An object of class \texttt{taylm} can be constructed as follows:
\begin{equation*}
	\begin{split}
		& \mathcal{T}(x) = \operator{taylm}(\mathcal{D}), \\
		& \mathcal{T}(x) = \operator{taylm}(\mathcal{D},\texttt{maxOrder},\texttt{name},\texttt{optMethod},\texttt{tolerance},\texttt{eps}),
	\end{split}
\end{equation*}
where $\mathcal{D} \subset \R^p$ is the interval domain for the variable $x$. The domain $\mathcal{D}$ is defined by an object of class \operator{interval} (see \cref{sec:interval}). The additional optional parameters are defined as follows:
\begin{itemize}
	\item \texttt{maxOrder}: Maximum polynomial degree of the monomials in the polynomial part of the Taylor model. Monomials with a degree larger than \texttt{maxOrder} are enclosed and added to the interval remainder. Further, $q = $ \texttt{maxOrder} is used for the implementation of the formulas listed in \cite[Appendix~A]{Althoff2018b}. 
	\item \texttt{name:} String or cell array of strings defining the names for the variables. Unique names are important since Taylor models explicitly consider dependencies between the variables.
	\item \texttt{optMethod}: Method used to calculate the bounds of the Taylor model objects. The available methods are 'int' (interval arithmetic, default), 'bnb' (branch and bound algorithm, see \cite[Sec.~2.3.2]{Althoff2018b}), 'bnbAdv' (branch and bound with Taylor model re-expansion) and 'linQuad' (optimization with Linear Dominated Bounder and Quadratic Fast Bounder, see \cite[Sec.~2.3.3]{Althoff2018b})
	\item \texttt{tolerance:} Minimum absolute value of the monomial coefficients in the polynomial part of the Taylor model. Monomials with a coefficient whose absolute value is smaller than \texttt{tolerance} are enclosed and added to the interval remainder.
	\item \texttt{eps:} Termination tolerance $\epsilon$ for the branch and bound algorithm from \cite[Sec.~2.3.2]{Althoff2018b} and for the algorithm based on the Linear Dominated Bounder and the Quadratic Fast Bounder from \cite[Sec.~2.3.3]{Althoff2018b}.
\end{itemize}
CORA also supports to create Taylor models from symbolic functions. A detailed description of this is provided in \cref{sec:createTaylorModels}.

\newpage
Let us demonstrate Taylor models by an example:

\begin{center}
\begin{minipage}[t]{0.50\textwidth}
	\vspace{10pt}
	\footnotesize
	% This file was automatically created from the m-file 
% "m2tex.m" written by USL. 
% The fontencoding in this file is UTF-8. 
%  
% You will need to include the following two packages in 
% your LaTeX-Main-File. 
%  
% \usepackage{color} 
% \usepackage{fancyvrb} 
%  
% It is advised to use the following option for Inputenc 
% \usepackage[utf8]{inputenc} 
%  
  
% definition of matlab colors: 
\definecolor{mblue}{rgb}{0,0,1} 
\definecolor{mgreen}{rgb}{0.13333,0.5451,0.13333} 
\definecolor{mred}{rgb}{0.62745,0.12549,0.94118} 
\definecolor{mgrey}{rgb}{0.5,0.5,0.5} 
\definecolor{mdarkgrey}{rgb}{0.25,0.25,0.25} 
  
\DefineShortVerb[fontfamily=courier,fontseries=m]{\$} 
\DefineShortVerb[fontfamily=courier,fontseries=b]{\#} 
  
\noindent          
 $$\color{mgreen}$% function f(x)$\color{black}$$\\
 $f = @(x) sin(x(1))*x(2) + x(1)^2;$\\
 $$\\
 $$\color{mgreen}$% create Taylor model$\color{black}$$\\
 $D = interval([-1;0],[2;1]);$\\
 $$\\
 $tay = taylm(D,10,$\color{mred}$'x'$\color{black}$,$\color{mred}$'linQuad'$\color{black}$);$\\
 $             $\\
 $$\color{mgreen}$% compute bounds$\color{black}$$\\
 $res = interval(f(tay))$\\ 
  
\UndefineShortVerb{\$} 
\UndefineShortVerb{\#}
\end{minipage}
\begin{minipage}[t]{0.25\textwidth}
	\vspace{10pt}

	\begin{verbatim}	
	Command Window:
	
	res =

  [-0.23256,4.90940]
	\end{verbatim}
\end{minipage}
\end{center}

A more detailed example for Taylor models is provided in \cref{sec:taylorModelExample} and in the file \textit{examples/contSet/example\_taylm.m} in the CORA toolbox. A detailed description of how Taylor models are treated in CORA can be found in \cite{Althoff2018b}. Furthermore, a list of operations that are implemented for the class \texttt{taylm} is provided in \cref{sec:taylorModelOperations}.