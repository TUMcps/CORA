\subsubsubsection{Capsules} \label{sec:capsules}

A capsule $\mathcal{C} \subset \Rn$ is defined as the Minkowski sum (see \cref{sec:plus}) of a line segment $\mathcal{L}$ and a sphere $\mathcal{S}$:
\begin{equation} \label{eq:capsule}
	\mathcal{C} := \mathcal{L} \oplus \mathcal{S}, ~~ \mathcal{L} = \{c + g \alpha~|~ \alpha \in [-1,1]\}, ~ \mathcal{S} = \{ x~|~ ||x||_2 \leq r \},
\end{equation}
where $c,g \in \Rn$ represent the center and the generator of the line segment, respectively, and $r \in \mathbb{R}_{\geq 0}$ is the radius of the sphere.

Capsules are represented in CORA by the class \texttt{capsule}. An object of class \texttt{capsule} can be constructed as follows:
\begin{equation*}
	\begin{split}
		& \mathcal{C} = \texttt{capsule}(c), \\
		& \mathcal{C} = \texttt{capsule}(c,g), \\
		& \mathcal{C} = \texttt{capsule}(c,r), \\
		& \mathcal{C} = \texttt{capsule}(c,g,r), \\
	\end{split}
\end{equation*} 
where $c,g,r$ are defined as in \eqref{eq:capsule}. Let us demonstrate the construction of a capsule by an example:

\begin{center}
\begin{minipage}[t]{0.35\textwidth}
	\vspace{30pt}
	\footnotesize
	% This file was automatically created from the m-file 
% "m2tex.m" written by USL. 
% The fontencoding in this file is UTF-8. 
%  
% You will need to include the following two packages in 
% your LaTeX-Main-File. 
%  
% \usepackage{color} 
% \usepackage{fancyvrb} 
%  
% It is advised to use the following option for Inputenc 
% \usepackage[utf8]{inputenc} 
%  
  
% definition of matlab colors: 
\definecolor{mblue}{rgb}{0,0,1} 
\definecolor{mgreen}{rgb}{0.13333,0.5451,0.13333} 
\definecolor{mred}{rgb}{0.62745,0.12549,0.94118} 
\definecolor{mgrey}{rgb}{0.5,0.5,0.5} 
\definecolor{mdarkgrey}{rgb}{0.25,0.25,0.25} 
  
\DefineShortVerb[fontfamily=courier,fontseries=m]{\$} 
\DefineShortVerb[fontfamily=courier,fontseries=b]{\#} 
  
\noindent      
 $$\color{mgreen}$% construct capsule$\color{black}$$\\
 $c = [1;2];$\\
 $g = [2;1];$\\
 $r = 1;$\\
 $     $\\
 $C = capsule(c,g,r);$\\ 
  
\UndefineShortVerb{\$} 
\UndefineShortVerb{\#}
\end{minipage}
\begin{minipage}[t]{0.3\textwidth}
	\vspace{0pt}
	\centering
	\includetikz{./figures/tikz/set-representations/example_capsule}
\end{minipage}
\end{center}

A more detailed example for capsules is provided in \cref{sec:capsuleExample} and in the file \textit{examples/contSet/example\_capsule.m} in the CORA toolbox.