\subsubsubsection{Affine} \label{sec:affine}

Affine arithmetic uses affine forms, i.e., first-order polynomials consisting of a vector $x\in \mathbb{R}^n$ and noise symbols $\epsilon_i \in [-1,1]$ (see e.g., \cite{deFigueiredo2004}):
\begin{equation*}
 \hat{x} = x_0 + \epsilon_1 x_1 + \epsilon_2 x_2 + \ldots + \epsilon_p x_p.
\end{equation*}
The possible values of $\hat{x}$ lie within a zonotope \cite{Kuehn1998b}.

Affine arithmetic is implemented by the class \texttt{affine}. Since we only consider intervals as inputs and outputs, we realized affine arithmetic as Taylor models of first order. The class \texttt{affine} therefore inherits all methods from the class \texttt{taylm} and does not implement any functionality on its own. The main purpose of the class \texttt{affine} is to provide a convenient and easy-to-use interface for the user. An object of class \texttt{affine} can be constructed as follows:
\begin{equation*}
	\begin{split}
		& \mathcal{A}(x) = \operator{affine}(\mathcal{D}), \\
		& \mathcal{A}(x) = \operator{affine}(\mathcal{D},\texttt{order},\texttt{name},\texttt{optMethod},\texttt{tolerance},\texttt{eps}),
	\end{split}
\end{equation*}
where the input arguments are identical to the ones for the class \texttt{taylm} (see \cref{sec:taylorModels}). Let us demonstrate the class \texttt{affine} by an example:

\begin{center}
\begin{minipage}[t]{0.50\textwidth}
	\vspace{10pt}
	\footnotesize
	% This file was automatically created from the m-file 
% "m2tex.m" written by USL. 
% The fontencoding in this file is UTF-8. 
%  
% You will need to include the following two packages in 
% your LaTeX-Main-File. 
%  
% \usepackage{color} 
% \usepackage{fancyvrb} 
%  
% It is advised to use the following option for Inputenc 
% \usepackage[utf8]{inputenc} 
%  
  
% definition of matlab colors: 
\definecolor{mblue}{rgb}{0,0,1} 
\definecolor{mgreen}{rgb}{0.13333,0.5451,0.13333} 
\definecolor{mred}{rgb}{0.62745,0.12549,0.94118} 
\definecolor{mgrey}{rgb}{0.5,0.5,0.5} 
\definecolor{mdarkgrey}{rgb}{0.25,0.25,0.25} 
  
\DefineShortVerb[fontfamily=courier,fontseries=m]{\$} 
\DefineShortVerb[fontfamily=courier,fontseries=b]{\#} 
  
\noindent          
 $$\color{mgreen}$% function f(x)$\color{black}$$\\
 $f = @(x) sin(x(1)).*x(2) + x(1)^2;$\\
 $$\\
 $$\color{mgreen}$% create affine object$\color{black}$$\\
 $D = interval([-1;0],[2;1]);$\\
 $$\\
 $aff = affine(D);$\\
 $             $\\
 $$\color{mgreen}$% compute bounds$\color{black}$$\\
 $res = interval(f(aff))$\\ 
  
\UndefineShortVerb{\$} 
\UndefineShortVerb{\#}
\end{minipage}
\begin{minipage}[t]{0.25\textwidth}
	\vspace{10pt}

	\begin{verbatim}	
	Command Window:
	
	res =

  [-3.69137,6.74245]
	\end{verbatim}
\end{minipage}
\end{center}

A more detailed example for the class \texttt{affine} is provided in \cref{sec:affineExample} and in the file \textit{examples/contSet/example\_affine.m} in the CORA toolbox.