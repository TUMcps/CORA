\subsubsubsection{Level Sets} \label{sec:levelSet}

A nonlinear level set $\mathcal{LS} \subset \mathbb{R}^n$ is defined as
\begin{align}
	\label{eq:defLevelSet1}
	\mathcal{LS} &= \{ x ~ | ~ f(x) = 0 \}, \\
	\label{eq:defLevelSet2}
	\mathcal{LS} &= \{ x ~ | ~ f(x) < 0 \}, \text{ or} \\
	\label{eq:defLevelSet3}
	\mathcal{LS} &= \{ x ~ | ~ f(x) \leq 0 \},
\end{align}
where $f:~\mathbb{R}^n \to \mathbb{R}$ is a Lipschitz continuous function. Level sets are represented in CORA by the class \texttt{levelSet}. An object of class \texttt{levelSet} can be constructed as follows:
\begin{equation*}
	\mathcal{LS} = \texttt{levelSet}(f(\cdot),\texttt{vars},\texttt{op}),
\end{equation*} 
where
\begin{itemize}
	\item $f:~\mathbb{R}^n \to \mathbb{R}$ is the nonlinear function that defines the level set (see \eqref{eq:defLevelSet1},\eqref{eq:defLevelSet2}, and \eqref{eq:defLevelSet3}). The function is specified as a symbolic MATLAB function.
	\item \texttt{vars} is a vector containing the symbolic variables of the function $f(\cdot)$.
	\item \texttt{op} $\in$ \{'==','$<$','$<=$'\} defines the type of level set (\eqref{eq:defLevelSet1},\eqref{eq:defLevelSet2}, or \eqref{eq:defLevelSet3}, respectively).
\end{itemize}

Let us demonstrate the construction of a level set by an example:

\begin{center}
\begin{minipage}[t]{0.45\textwidth}
	\vspace{20pt}
	\footnotesize
	% This file was automatically created from the m-file 
% "m2tex.m" written by USL. 
% The fontencoding in this file is UTF-8. 
%  
% You will need to include the following two packages in 
% your LaTeX-Main-File. 
%  
% \usepackage{color} 
% \usepackage{fancyvrb} 
%  
% It is advised to use the following option for Inputenc 
% \usepackage[utf8]{inputenc} 
%  
  
% definition of matlab colors: 
\definecolor{mblue}{rgb}{0,0,1} 
\definecolor{mgreen}{rgb}{0.13333,0.5451,0.13333} 
\definecolor{mred}{rgb}{0.62745,0.12549,0.94118} 
\definecolor{mgrey}{rgb}{0.5,0.5,0.5} 
\definecolor{mdarkgrey}{rgb}{0.25,0.25,0.25} 
  
\DefineShortVerb[fontfamily=courier,fontseries=m]{\$} 
\DefineShortVerb[fontfamily=courier,fontseries=b]{\#} 
  
\noindent      
 $$\color{mgreen}$% construct level set$\color{black}$$\\
 $vars = sym($\color{mred}$'x'$\color{black}$,[2,1]);$\\
 $f = 1/vars(1)^2 - vars(2);$\\
 $op = $\color{mred}$'=='$\color{black}$;$\\
 $     $\\
 $ls = levelSet(f,vars,op);$\\ 
  
\UndefineShortVerb{\$} 
\UndefineShortVerb{\#}
\end{minipage}
\begin{minipage}[t]{0.3\textwidth}
	\vspace{0pt}
	\centering
	\includetikz{./figures/tikz/set-representations-aux/example_levelSet}
\end{minipage}
\end{center}

A more detailed example for level sets is provided in \cref{sec:levelSetExample} and in the file \textit{examples/contSet/example\_levelSet.m} in the CORA toolbox.  In addition to the standard set operations described in \cref{sec:setOperations}, the class \texttt{levelSet} supports additional methods, which are listed in \cref{sec:levelSetOperations}.