\subsubsubsection{Polynomial Zonotopes} \label{sec:polynomialZonotopes}

Polynomial zonotopes, which were first introduced in \cite{Althoff2013a}, are a non-convex set representation. In CORA we implemented the sparse representation of polynomial zonotopes described in \cite{Kochdumper2021a}. A polynomial zonotope $\mathcal{PZ} \subset \Rn$ is defined as
\begin{equation}
    \mathcal{PZ} := \bigg\{ c + \sum _{i=1}^h \bigg( \prod _{k=1}^p \alpha _k ^{E_{(k,i)}} \bigg) G_{(\cdot,i)} + \sum _{j=1}^{q} \beta _j G_{I(\cdot,j)} ~ \bigg| ~ \alpha_k, \beta_j \in [-1,1] \bigg\},
  \label{eq:polyZonotope}
\end{equation}
where $c \in \Rn$ is the center, $G \in \mathbb{R}^{n \times h}$ the matrix of dependent generators, $G_I \in \mathbb{R}^{n \times q}$ the matrix of independent generators, and  $E \in \mathbb{N}_{0}^{p \times h}$ the exponent matrix. Since polynomial zonotopes can represent non-convex sets, and since they are closed under quadratic and higher-order maps, they are a good choice for reachability analysis.

Polynomial zonotopes are represented in CORA by the class \texttt{polyZonotope}. An object of class \texttt{polyZonotope} can be constructed as follows:
\begin{equation*}
	\begin{split}
	& \mathcal{PZ} = \texttt{polyZonotope}(c,G,G_I,E), \\
	& \mathcal{PZ} = \texttt{polyZonotope}(c,G,G_I,E,id), \\
	\end{split}
\end{equation*} 
where $c,G,G_I,E$ are defined as in \eqref{eq:polyZonotope}. The vector $id \in \mathbb{N}^p_{> 0}$ stores unambiguous identifiers for the dependent factors $\alpha_k$, which is important for dependency preservation as described in \cite{Kochdumper2020c}. Let us demonstrate the construction of a polynomial zonotope by an example:

\begin{center}
\begin{minipage}[t]{0.55\textwidth}
	\vspace{10pt}
	\footnotesize
	% This file was automatically created from the m-file 
% "m2tex.m" written by USL. 
% The fontencoding in this file is UTF-8. 
%  
% You will need to include the following two packages in 
% your LaTeX-Main-File. 
%  
% \usepackage{color} 
% \usepackage{fancyvrb} 
%  
% It is advised to use the following option for Inputenc 
% \usepackage[utf8]{inputenc} 
%  
  
% definition of matlab colors: 
\definecolor{mblue}{rgb}{0,0,1} 
\definecolor{mgreen}{rgb}{0.13333,0.5451,0.13333} 
\definecolor{mred}{rgb}{0.62745,0.12549,0.94118} 
\definecolor{mgrey}{rgb}{0.5,0.5,0.5} 
\definecolor{mdarkgrey}{rgb}{0.25,0.25,0.25} 
  
\DefineShortVerb[fontfamily=courier,fontseries=m]{\$} 
\DefineShortVerb[fontfamily=courier,fontseries=b]{\#} 
  
\noindent       
 $$\color{mgreen}$% construct polynomial zonotope$\color{black}$$\\
 $c = [4;4];$\\
 $G = [2 1 2; 0 2 2];$\\
 $E = [1 0 3;0 1 1];$\\
 $GI = [1;0];$\\
 $$\\
 $pZ = polyZonotope(c,G,GI,E);$\\
  
\UndefineShortVerb{\$} 
\UndefineShortVerb{\#}
\end{minipage}
\begin{minipage}[t]{0.3\textwidth}
	\vspace{0pt}
	\centering
	\includetikz{./figures/tikz/set-representations/example_polyZonotope}
\end{minipage}
\end{center}

This example defines the polynomial zonotope
\begin{equation*}
	\mathcal{PZ} = \bigg \{ \begin{bmatrix} 4 \\ 4 \end{bmatrix} + \alpha_1 \begin{bmatrix} 2 \\ 0 \end{bmatrix} + \alpha_2 \begin{bmatrix} 1 \\ 2 \end{bmatrix} + \alpha_1^3 \alpha_2 \begin{bmatrix} 2 \\ 2 \end{bmatrix} + \beta_1\begin{bmatrix} 1 \\ 0 \end{bmatrix}~\bigg|~ \alpha_1,\alpha_2,\beta_1 \in [-1,1] \bigg \}.
\end{equation*}
The construction of this polynomial zonotope is visualized in \cref{fig:polyZonotope}: (a) shows the set spanned by the constant offset vector and the first and second dependent generator, (b) shows the addition of the dependent generator with the mixed term $\alpha_1^3 \alpha_2$, (c) shows the addition of the independent generator, and (d) visualizes the final set.

\begin{figure}[htb]	
	\begin{center}
	\includetikz{./figures/tikz/set-representations/example_manual_polyZonotope_construction}
	\caption{Step-by-step construction of a polynomial zonotope.}
	\label{fig:polyZonotope}
	\end{center}
\end{figure}

A more detailed example for polynomial zonotopes is provided in \cref{sec:polyZonotopeExample} and in the file \textit{examples/contSet/example\_polyZonotope.m} in the CORA toolbox.