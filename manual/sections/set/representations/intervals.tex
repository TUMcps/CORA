\subsubsubsection{Intervals} \label{sec:interval}

A real-valued multi-dimensional interval
\begin{equation}
	\mathcal{I} :=\{x \in \mathbb{R}^n ~|~ \underline{x}_i \leq x_i \leq \overline{x}_i ~\forall i = 1,\dots,n \}
	\label{eq:interval}
\end{equation} 
is a connected subset of $\mathbb{R}^n$ and can be specified by a lower bound $\underline{x}\in\mathbb{R}^n$ and upper bound $\overline{x}\in\mathbb{R}^n$.

Intervals are represented in CORA by the class \texttt{interval}. An object of class \texttt{interval} can be constructed as follows:
\begin{equation*}
	\begin{split}
		& \mathcal{I} = \texttt{interval}(\underline{x},\overline{x}) \\
	\end{split}
\end{equation*} 
where $\underline{x},\overline{x}$ are defined as in \eqref{eq:interval}. A detailed description of how intervals are treated in CORA can be found in \cite{Althoff2016a}. Let us demonstrate the construction of an interval by an example:

\begin{center}
\begin{minipage}[t]{0.35\textwidth}
	\vspace{30pt}
	\footnotesize
	\input{./MATLABcode/example_setRep_interval}
\end{minipage}
\begin{minipage}[t]{0.3\textwidth}
	\vspace{0pt}
	\centering
	\includetikz{./figures/tikz/set-representations/example_interval}
\end{minipage}
\end{center}

A more detailed example for intervals is provided in \cref{sec:intervalExample} and in the file \textit{examples/contSet/example\_interval.m} in the CORA toolbox. Intervals can also be used for range bounding as it described in \cref{sec:rangeBounding}. In addition to the standard set operations described in \cref{sec:setOperations} and the methods for converting between set operations (see \cref{tab:setConversion}), the class \texttt{interval} supports additional methods, which are listed in \cref{sec:intervalOperations}.


