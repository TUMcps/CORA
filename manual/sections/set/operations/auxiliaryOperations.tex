\subsubsection{Auxiliary Operations}

In this subsection, we describe useful auxiliary operations.

\subsubsubsection{cubMap}

The method \operator{cubMap} implements the cubic map of a set. Given a set $\mathcal{S} \subset \Rn$, the cubic map is defined as 
\begin{equation*}
	\operator{cubMap}(\mathcal{S},Q) = \bigg \{ x ~ \bigg | ~ x_{(i)} = \sum_{j=1}^n s_{(j)} ~(s^T ~T_{i,j}~ s), ~s \in \mathcal{S},~ i = 1 \dots w \bigg \}, ~ T_{i,j} \in \R^{n \times n},
\end{equation*}
where $x_{(i)}$ is the $i$-th value of the vector $x$. If the corresponding set representation is not closed under cubic maps, \operator{cubMap} returns an over-approximation. If \operator{cubMap} is called with three different sets $\mathcal{S}_1,\mathcal{S}_2,\mathcal{S}_3 \subset \Rn$ as input arguments, the method computes the mixed cubic map:
\begin{equation*}
\begin{split}
	\operator{cubMap}(\mathcal{S}_1,\mathcal{S}_2,\mathcal{S}_3,Q) = \bigg \{ x ~ \bigg | ~ & x_{(i)} = \sum_{j=1}^n s_{1(j)} ~(s_2^T ~T_{i,j}~ s_3), ~s_1 \in \mathcal{S}_1,~s_2 \in \mathcal{S}_2,~s_3 \in \mathcal{S}_3, \\
	& i = 1 \dots w \bigg \}, ~ T_{i,j} \in \R^{n \times n},
\end{split}
\end{equation*}
Let us demonstrate the method $\operator{cubMap}$ by an example:

\begin{center}
\begin{minipage}[t]{0.35\textwidth}
	\vspace{10pt}
	\footnotesize
	% This file was automatically created from the m-file 
% "m2tex.m" written by USL. 
% The fontencoding in this file is UTF-8. 
%  
% You will need to include the following two packages in 
% your LaTeX-Main-File. 
%  
% \usepackage{color} 
% \usepackage{fancyvrb} 
%  
% It is advised to use the following option for Inputenc 
% \usepackage[utf8]{inputenc} 
%  
  
% definition of matlab colors: 
\definecolor{mblue}{rgb}{0,0,1} 
\definecolor{mgreen}{rgb}{0.13333,0.5451,0.13333} 
\definecolor{mred}{rgb}{0.62745,0.12549,0.94118} 
\definecolor{mgrey}{rgb}{0.5,0.5,0.5} 
\definecolor{mdarkgrey}{rgb}{0.25,0.25,0.25} 
  
\DefineShortVerb[fontfamily=courier,fontseries=m]{\$} 
\DefineShortVerb[fontfamily=courier,fontseries=b]{\#} 
  
\noindent            
 $$\color{mgreen}$% set and matrices$\color{black}$$\\
 $S = polyZonotope([0;0],$\color{mblue}$ ...$\color{black}$$\\
 $                 [1 1;1 0],$\color{mblue}$ ...$\color{black}$$\\
 $                 [],eye(2));$\\
 $          $\\
 $T{1,1} = 0.4*[1 2; -1 2];$\\
 $T{1,2} = 0.4*[-3 0; 1 1];$\\
 $T{2,1} = 0.05*[2 0; -2 1];$\\
 $T{2,2} = 0.05*[-3 0; -21 -1];$\\
 $$\\
 $$\color{mgreen}$% cubic map$\color{black}$$\\
 $res = cubMap(S,T);$\\ 
  
\UndefineShortVerb{\$} 
\UndefineShortVerb{\#}
\end{minipage}
\begin{minipage}[t]{0.6\textwidth}
	\vspace{0pt}
	\centering
	\includetikz{./figures/tikz/set-operations-aux/example_cubMap}
\end{minipage}
\end{center}


\subsubsubsection{enclose}

The method \operator{enclose} computes an enclosure of a set and its linear transformation. Given the sets $\mathcal{S}_1,\mathcal{S}_2 \in \Rn$ and the matrix $M \in \R^{n \times n}$, \operator{enclose} computes the set
\begin{equation}
	\operator{enclose}(\mathcal{S}_1,M,\mathcal{S}_2) = \left \{ \lambda s_1 + (1-\lambda) (Ms_1 +s_2) ~|~ s_1 \in \mathcal{S}_1,~s_2 \in \mathcal{S}_2,~\lambda \in [0,1] \right\}.
	\label{eq:enclose}
\end{equation}
If the set as defined in \eqref{eq:enclose} cannot be computed exactly for the corresponding set representation, \operator{enclose} returns an over-approximation. For convenience, the method can also be called with only two input arguments:
\begin{equation*}
	\operator{enclose}(\mathcal{S}_1,\mathcal{S}_3) = \operator{enclose}(\mathcal{S}_1,M,\mathcal{S}_2), ~~~ \mathcal{S}_3 = (M \otimes \mathcal{S}_1) \oplus \mathcal{S}_2.
\end{equation*}
Let us demonstrate the method $\operator{enclose}$ by an example:

\begin{center}
\begin{minipage}[t]{0.37\textwidth}
	\vspace{10pt}
	\footnotesize
	% This file was automatically created from the m-file 
% "m2tex.m" written by USL. 
% The fontencoding in this file is UTF-8. 
%  
% You will need to include the following two packages in 
% your LaTeX-Main-File. 
%  
% \usepackage{color} 
% \usepackage{fancyvrb} 
%  
% It is advised to use the following option for Inputenc 
% \usepackage[utf8]{inputenc} 
%  
  
% definition of matlab colors: 
\definecolor{mblue}{rgb}{0,0,1} 
\definecolor{mgreen}{rgb}{0.13333,0.5451,0.13333} 
\definecolor{mred}{rgb}{0.62745,0.12549,0.94118} 
\definecolor{mgrey}{rgb}{0.5,0.5,0.5} 
\definecolor{mdarkgrey}{rgb}{0.25,0.25,0.25} 
  
\DefineShortVerb[fontfamily=courier,fontseries=m]{\$} 
\DefineShortVerb[fontfamily=courier,fontseries=b]{\#} 
  
\noindent            
 $$\color{mgreen}$% sets S1,S2 and matrix M$\color{black}$$\\
 $S1 = polyZonotope([1.5;1.5],$\color{mblue}$ ...$\color{black}$$\\
 $                  [1 0;0 1],$\color{mblue}$ ...$\color{black}$$\\
 $                  [],eye(2));$\\
 $S2 = [0.5;0.5];$\\
 $M = [-1 0;0 -1];$\\
 $           $\\
 $$\color{mgreen}$% apply method enclose$\color{black}$$\\
 $S3 = M*S1 + S2;$\\
 $$\\
 $res = enclose(S1,M,S2);$\\
 $res = enclose(S1,S3);$\\ 
  
\UndefineShortVerb{\$} 
\UndefineShortVerb{\#}
\end{minipage}
\begin{minipage}[t]{0.6\textwidth}
	\vspace{0pt}
	\centering
	\includetikz{./figures/tikz/set-operations-aux/example_enclose}
\end{minipage}
\end{center}

\vspace{1cm}

\subsubsubsection{enclosePoints}

Given a point cloud $P = [p_1,\dots,p_m]$, $p_i \in \Rn$, the static method \operator{enclosePoints} computes a set $\mathcal{S} \subset \Rn$ that tightly encloses the point cloud:
\begin{equation*}
	\mathcal{S} = \operator{enclosePoints}\big([p_1,\dots,p_m]\big), ~~ \forall i = 1,\dots,m: ~ p_i \in \mathcal{S}.
\end{equation*}

Let us demonstrate the method \operator{enclosePoints} by an example:

\begin{center}
\begin{minipage}[t]{0.50\textwidth}
	\vspace{10pt}
	\footnotesize
	% This file was automatically created from the m-file 
% "m2tex.m" written by USL. 
% The fontencoding in this file is UTF-8. 
%  
% You will need to include the following two packages in 
% your LaTeX-Main-File. 
%  
% \usepackage{color} 
% \usepackage{fancyvrb} 
%  
% It is advised to use the following option for Inputenc 
% \usepackage[utf8]{inputenc} 
%  
  
% definition of matlab colors: 
\definecolor{mblue}{rgb}{0,0,1} 
\definecolor{mgreen}{rgb}{0.13333,0.5451,0.13333} 
\definecolor{mred}{rgb}{0.62745,0.12549,0.94118} 
\definecolor{mgrey}{rgb}{0.5,0.5,0.5} 
\definecolor{mdarkgrey}{rgb}{0.25,0.25,0.25} 
  
\DefineShortVerb[fontfamily=courier,fontseries=m]{\$} 
\DefineShortVerb[fontfamily=courier,fontseries=b]{\#} 
  
\noindent       
 $$\color{mgreen}$% random point cloud$\color{black}$$\\
 $mu = [0 0];$\\
 $sigma = [0.3 0.4; 0.4 1];$\\
 $points = mvnrnd(mu,sigma,100)';$\\
 $             $\\
 $$\color{mgreen}$% compute enclosing set$\color{black}$$\\
 $S = ellipsoid.enclosePoints(points);$\\ 
  
\UndefineShortVerb{\$} 
\UndefineShortVerb{\#}
\end{minipage}
\begin{minipage}[t]{0.3\textwidth}
	\vspace{0pt}
	\centering
	\includetikz{./figures/tikz/set-operations-aux/example_enclosePoints}
\end{minipage}
\end{center}

\vspace{1cm}

\subsubsubsection{generateRandom}

The static method \operator{generateRandom} randomly generates a set of the given set representation. If no input arguments are provided, the method generates a random set of arbitrary dimension. The desired dimension and other specifications for the set can be provided by name-value pairs:
\begin{equation*}
	\mathcal{S} = \operator{generateRandom}(\texttt{'Dimension'},n), ~~~ \mathcal{S} \subset \R^{\texttt{n}}.
\end{equation*}
The additionally supported name-value pairs of the method \operator{generateRandom} for each class are detailed in their respective function description. Let us demonstrate the method \operator{generateRandom} by an example:

\begin{center}
\begin{minipage}[t]{0.40\textwidth}
	\vspace{10pt}
	\footnotesize
	% This file was created by matlab2tikz.
%
\definecolor{mycolor1}{rgb}{0.00000,0.44700,0.74100}%
%
\begin{tikzpicture}
\footnotesize

\begin{axis}[%
width=3cm,
height=3cm,
at={(0in,0in)},
scale only axis,
xmin=-1.75,
xmax=1.25,
xlabel style={font=\color{white!15!black}},
xlabel={$x_{(1)}$},
ymin=-2.98,
ymax=-0.82,
ylabel style={font=\color{white!15!black}},
ylabel={$x_{(2)}$},
axis background/.style={fill=white},
title style={font=\bfseries},
title={$\mathcal{S}$}
]
\addplot [color=mycolor1, forget plot]
  table[row sep=crcr]{%
-1.4113	-2.6153\\
0.8993	-2.6153\\
0.8993	-1.1773\\
-1.4113	-1.1773\\
-1.4113	-2.6153\\
};
\end{axis}
\end{tikzpicture}%
\end{minipage}
\begin{minipage}[t]{0.25\textwidth}
	\vspace{10pt}

	\begin{verbatim}
	Command Window:
		
	S = 
  [-1.4113, 0.8993]
  [-2.6153, -1.1773]
	\end{verbatim}
\end{minipage}
\begin{minipage}[t]{0.3\textwidth}
	\vspace{0pt}
	\centering
	\includetikz{./figures/tikz/set-operations-aux/example_generateRandom}
\end{minipage}
\end{center}

\vspace{1cm}

\subsubsubsection{linComb}
\label{sec:linComb}

The method \operator{linComb} implements the linear combination of two sets. Given two sets $\mathcal{S}_1,\mathcal{S}_2 \subset \Rn$, their linear combination is defined as 
\begin{equation*}
	\operator{linComb}(\mathcal{S}_1,\mathcal{S}_2) = \left \{ \lambda s_1 + (1-\lambda) s_2 ~|~ s_1 \in \mathcal{S}_1,~s_2 \in \mathcal{S}_2,~\lambda \in [0,1] \right\}.
\end{equation*}
Note that for convex sets the linear combination is identical to the convex hull (see \cref{sec:convHull}). For non-convex sets, however, the two operations differ.
Let us demonstrate the method $\operator{linComb}$ by an example:

\begin{center}
\begin{minipage}[t]{0.38\textwidth}
	\vspace{10pt}
	\footnotesize
	\input{./MATLABcode/example_linComb}
\end{minipage}
\begin{minipage}[t]{0.6\textwidth}
	\vspace{0pt}
	\centering
	\includetikz{./figures/tikz/set-operations-aux/example_linComb}
\end{minipage}
\end{center}

\subsubsubsection{randPoint}

The method \operator{randPoint} returns random points located inside a set. Given a set $\mathcal{S} \subset \Rn$, the method \operator{randPoint} generates random points $p = [p_1,\dots,p_N] \in \R^{n \times N}$ with $p_1,\dots, p_N \in \mathcal{S}$:
\begin{equation*}
	\begin{split}
		& p = \operator{randPoint}(\mathcal{S}), \\
		& p = \operator{randPoint}(\mathcal{S},N), \\
		& p = \operator{randPoint}(\mathcal{S},N,\texttt{type}),
   \end{split}
\end{equation*}
where $N \in \mathbb{N}_{> 0}$ is the desired number of points, and \texttt{type} specifies the desired type random points. The setting \texttt{type = 'extreme'} aims to generate points close to or on the boundary of the set, while \texttt{type = 'standard'} generates arbitrary points within the set. In contrast, the setting \texttt{type = 'uniform'} generates uniformly distributed points within the set, potentially at the cost of a longer runtime. The default values are $N = 1$ and \texttt{type = 'standard'}. Let us demonstrate the method \operator{randPoint} by an example:

\begin{center}
\begin{minipage}[t]{0.40\textwidth}
	\vspace{10pt}
	\footnotesize
	% This file was created by matlab2tikz.
%
\definecolor{mycolor1}{rgb}{0.00000,0.44700,0.74100}%
%
\begin{tikzpicture}
\footnotesize

\begin{axis}[%
width=4cm,
height=4cm,
at={(0in,0in)},
scale only axis,
xmin=-2.4,
xmax=2.4,
xlabel style={font=\color{white!15!black}},
xlabel={$x_{(1)}$},
ymin=-2.4,
ymax=2.4,
ylabel style={font=\color{white!15!black}},
ylabel={$x_{(2)}$},
axis background/.style={fill=white},
title style={font=\bfseries},
title={$\mathcal{S}$ and point}
]
\addplot [color=mycolor1, forget plot]
  table[row sep=crcr]{%
-2	-2\\
0	-2\\
2	0\\
2	2\\
0	2\\
-2	0\\
-2	-2\\
};
\addplot[only marks, mark=*, mark options={}, mark size=1.0000pt, draw=black, forget plot] table[row sep=crcr]{%
x	y\\
0.2747	-1.1657\\
};
\end{axis}
\end{tikzpicture}%
\end{minipage}
\begin{minipage}[t]{0.2\textwidth}
	\vspace{10pt}

	\begin{verbatim}
	Command Window:
		
	p =

    0.2747
   -1.1657
	\end{verbatim}
\end{minipage}
\begin{minipage}[t]{0.3\textwidth}
	\vspace{0pt}
	\centering
	\includetikz{./figures/tikz/set-operations-aux/example_randPoint}
\end{minipage}
\end{center}

\vspace{1cm}

\subsubsubsection{reduce}

The method \operator{reduce} encloses a set by another set with a smaller representation size. Given a set $\mathcal{S} \subset \Rn$, the method \operator{reduce} computes
\begin{equation}
	\operator{reduce}(\mathcal{S},\texttt{method},\texttt{order}) = \overline{\mathcal{S}}, ~~~ \mathcal{S} \subseteq \overline{\mathcal{S}},
	\label{eq:reduce}
\end{equation}
where the representation size of $\overline{\mathcal{S}}$ is smaller than the one of $\mathcal{S}$. The parameter \texttt{method} in \eqref{eq:reduce} is a string that specifies the algorithm to be applied, see \cref{tab:zono_reduction}. The parameter \texttt{order} in \eqref{eq:reduce} is a measure for the desired representation size of the resulting set $\overline{\mathcal{S}}$. Currently, the method \operator{reduce} is implemented for the zonotopic set representations \texttt{zonotope} (see \cref{sec:zonotope}), \texttt{conZonotope} (see \cref{sec:conZonotope}), \texttt{polyZonotope} (see \cref{sec:polynomialZonotopes}), and \texttt{probZonotope} (see \cref{sec:probabilisticZonotopes}), where $\texttt{order} = \frac{p}{n}$ is defined as the division of the number of generator vectors $p$ by the system dimension $n$. Let us demonstrate the method \operator{reduce} by an example:

\begin{center}
\begin{minipage}[t]{0.35\textwidth}
	\vspace{10pt}
	\footnotesize
	\input{./MATLABcode/example_reduce}
\end{minipage}
\begin{minipage}[t]{0.6\textwidth}
	\vspace{0pt}
	\centering
	\includetikz{./figures/tikz/set-operations-aux/example_reduce}
\end{minipage}
\end{center}


\begin{table}[h]
	\caption{Reduction techniques for zonotopic set representations.}
	\centering
	\label{tab:zono_reduction}
	\begin{tabular}{lll}
		\toprule
		\textbf{Technique} & \textbf{Primary use} & \textbf{Reference} \\
		\midrule
		\texttt{cluster} & Reduction to low order by clustering generators & \cite[Sec.~III.B]{Kopetzki2017} \\
		\texttt{combastel} & Reduction of high to medium order & \cite[Sec.~3.2]{Combastel2003} \\
		\texttt{constOpt} & Reduction to low order by optimization & \cite[Sec.~III.D]{Kopetzki2017} \\
		\texttt{girard} & Reduction of high to medium order & \cite[Sec. ~.4]{Girard2005} \\
		\texttt{methA} & Reduction to low order by volume minimization (A) & Meth. A, \cite[Sec.~2.5.5]{Althoff2010a} \\
		\texttt{methB} & Reduction to low order by volume minimization (B) & Meth. B, \cite[Sec.~2.5.5]{Althoff2010a} \\
		\texttt{methC} & Reduction to low order by volume minimization (C) & Meth. C, \cite[Sec.~2.5.5]{Althoff2010a} \\
		\texttt{scott} & Reduction to low order  & \cite[Appendix]{Scott2016} \\
		\texttt{pca} & Reduction of high to medium order using PCA & \cite[Sec.~III.A]{Kopetzki2017} \\
		\bottomrule
	\end{tabular}
\end{table}


\subsubsubsection{supportFunc}

The method \operator{supportFunc} computes the support function for a specific direction. Given a set $\mathcal{S} \in \Rn$ and a vector $l \in \Rn$, the support function is defined as
\begin{equation*}
	\operator{supportFunc}(\mathcal{S},l) = \max_{x \in \mathcal{S}} ~l^T~x.
\end{equation*}
The function also supports the computation of the lower bound, which can be calculated using the flag \texttt{'lower'}:
\begin{equation*}
	\operator{supportFunc}(\mathcal{S},l,\texttt{'lower'}) = \min_{x \in \mathcal{S}} ~l^T~x.
\end{equation*}
Additionally, one can return both the lower and upper bounds by using the flag \texttt{'range'}.
Let us demonstrate the method \operator{supportFunc} by an example:

\begin{center}
\begin{minipage}[t]{0.40\textwidth}
	\vspace{10pt}
	\footnotesize
	% This file was created by matlab2tikz.
%
\definecolor{mycolor1}{rgb}{0.00000,0.44700,0.74100}%
\definecolor{mycolor2}{rgb}{0.85000,0.32500,0.09800}%
%
\begin{tikzpicture}
\footnotesize

\begin{axis}[%
width=4cm,
height=4cm,
at={(0in,0in)},
scale only axis,
xmin=-2.5,
xmax=2.5,
xlabel style={font=\color{white!15!black}},
xlabel={$x_{(1)}$},
ymin=-2.5,
ymax=2.5,
ylabel style={font=\color{white!15!black}},
ylabel={$x_{(2)}$},
axis background/.style={fill=white},
title style={font=\bfseries},
title={$\mathcal{S}$ and \texttt{supportFunc}}
]
\addplot [color=mycolor1, forget plot]
  table[row sep=crcr]{%
-2	-2\\
0	-2\\
2	0\\
2	2\\
0	2\\
-2	0\\
-2	-2\\
};
\addplot [color=mycolor2, forget plot]
  table[row sep=crcr]{%
2.5	1.75\\
1	2.5\\
};
\end{axis}
\end{tikzpicture}%
\end{minipage}
\begin{minipage}[t]{0.2\textwidth}
	\vspace{10pt}

	\begin{verbatim}
	Command Window:
		
	res = 6
	\end{verbatim}
\end{minipage}
\begin{minipage}[t]{0.3\textwidth}
	\vspace{0pt}
	\centering
	\includetikz{./figures/tikz/set-operations-aux/example_supportFunc}
\end{minipage}
\end{center}


\subsubsubsection{plot}

The method \operator{plot} visualizes a 2-dimensional projection of the boundary of a set. Given a set $\mathcal{S} \subset \Rn$, the method \operator{plot} supports the following syntax:
\begin{equation*}
	\begin{split}
		&\texttt{han} = \operator{plot}(\mathcal{S}), \\
		&\texttt{han} = \operator{plot}(\mathcal{S},\texttt{dims}), \\
		&\texttt{han} = \operator{plot}(\mathcal{S},\texttt{dims},\texttt{linespec}), \\
		&\texttt{han} = \operator{plot}(\mathcal{S},\texttt{dims},\texttt{namevaluepairs}), \\
	\end{split}
\end{equation*}
where \texttt{han} is a handle to the plotted MATLAB graphics object and the additional input arguments are defined as

\begin{itemize}
	\item \texttt{dims}: Integer vector $\texttt{dims} \in \mathbb{N}_{\leq n}^2$ specifying the dimensions for which the projection is visualized (default value: \texttt{dim = [1 2]}).
	\item \texttt{linespec}: (optional) line specifications, e.g., \texttt{'--*r'}, as supported by MATLAB\footnote{\url{https://de.mathworks.com/help/matlab/ref/linespec.html}}.
	\item \texttt{namevaluepairs}: (optional) further specifications as name-value pairs, e.g.,
	\texttt{'LineWidth',2} and \texttt{'FaceColor',[.5 .5 .5]}, as supported by MATLAB.
	If the plot is not filled, these are the built-in
	Line Properties\footnote{\url{https://de.mathworks.com/help/matlab/ref/matlab.graphics.chart.primitive.line-properties.html}},
	if the plot is filled, they correspond to the Patch Properties\footnote{\url{https://de.mathworks.com/help/matlab/ref/matlab.graphics.primitive.patch-properties.html}}.
\end{itemize}

Let us demonstrate the method  \operator{plot} by an example:

\begin{center}
\begin{minipage}[t]{0.40\textwidth}
	\vspace{10pt}
	\footnotesize
	\input{./MATLABcode/example_plot}
\end{minipage}
\begin{minipage}[t]{0.3\textwidth}
	\vspace{0pt}
	\centering
	\includetikz{./figures/tikz/set-operations-aux/example_plot}
\end{minipage}
\end{center}


\vspace{1cm}

\subsubsubsection{project}

The method \operator{project} projects a set to a lower-dimensional, axis-aligned subspace. Given a set $\mathcal{S} \subset \Rn$ and a vector of subspace indices $\texttt{dims} \in \mathbb{N}_{\leq n}^m$, the method \operator{project} returns
\begin{equation*}
	\operator{project}(\mathcal{S},\texttt{dims}) = \Big \{ [s_{(\texttt{dims}_{(1)})},\dots,s_{(\texttt{dims}_{(m)})}]^T ~\Big |~ s \in \mathcal{S} \Big \} \subset \R^m,
\end{equation*}
where $s_{(i)}$ denotes the i-th entry of vector $s$. Let us demonstrate the method  \operator{project} by an example:

\begin{center}
\begin{minipage}[t]{0.40\textwidth}
	\vspace{10pt}
	\footnotesize
	\input{./MATLABcode/example_project}
\end{minipage}
\begin{minipage}[t]{0.25\textwidth}
	\vspace{10pt}

	\begin{verbatim}	
	Command Window:
	
	res = 
 [1.00000,3.00000]
 [5.00000,7.00000]
 [0.00000,2.00000]
	\end{verbatim}
\end{minipage}
\end{center}

A set can be projected into a higher-dimensional space using the function \operator{projectHighDim},
where the new dimensions are bounded at 0.
On the other hand, the function \operator{lift} lifts the set into a higher-dimensional space with unbounded new dimensions.