\subsubsection{Set Properties}

In this subsection, we describe the methods that compute geometric properties of sets.

\subsubsubsection{center}

The method \operator{center} returns the center of a set. Let us demonstrate the method \operator{center} by an example:

\begin{center}
\begin{minipage}[t]{0.4\textwidth}
	\vspace{10pt}
	\footnotesize
	% This file was automatically created from the m-file 
% "m2tex.m" written by USL. 
% The fontencoding in this file is UTF-8. 
%  
% You will need to include the following two packages in 
% your LaTeX-Main-File. 
%  
% \usepackage{color} 
% \usepackage{fancyvrb} 
%  
% It is advised to use the following option for Inputenc 
% \usepackage[utf8]{inputenc} 
%  
  
% definition of matlab colors: 
\definecolor{mblue}{rgb}{0,0,1} 
\definecolor{mgreen}{rgb}{0.13333,0.5451,0.13333} 
\definecolor{mred}{rgb}{0.62745,0.12549,0.94118} 
\definecolor{mgrey}{rgb}{0.5,0.5,0.5} 
\definecolor{mdarkgrey}{rgb}{0.25,0.25,0.25} 
  
\DefineShortVerb[fontfamily=courier,fontseries=m]{\$} 
\DefineShortVerb[fontfamily=courier,fontseries=b]{\#} 
  
\noindent     
 $$\color{mgreen}$% set S$\color{black}$$\\
 $S = interval([-2;-2],[1;1]);$\\
 $$\\
 $$\color{mgreen}$% compute center$\color{black}$$\\
 $res = center(S)$\\ 
  
\UndefineShortVerb{\$} 
\UndefineShortVerb{\#}
\end{minipage}
\begin{minipage}[t]{0.2\textwidth}
	\vspace{10pt}

	\begin{verbatim}
	Command Window:
		
	res =

   -0.5000
   -0.5000
	\end{verbatim}
\end{minipage}
\begin{minipage}[t]{0.3\textwidth}
	\vspace{0pt}
	\centering
	\includetikz{./figures/tikz/set-properties/example_center}
\end{minipage}
\end{center}


\subsubsubsection{dim}

The method \operator{dim} returns the dimension of the ambient space of a set, that is, the dimension in which a set is defined. Let us demonstrate the method \operator{dim} by an example:

\begin{center}
\begin{minipage}[t]{0.40\textwidth}
	\vspace{10pt}
	\footnotesize
	% This file was automatically created from the m-file 
% "m2tex.m" written by USL. 
% The fontencoding in this file is UTF-8. 
%  
% You will need to include the following two packages in 
% your LaTeX-Main-File. 
%  
% \usepackage{color} 
% \usepackage{fancyvrb} 
%  
% It is advised to use the following option for Inputenc 
% \usepackage[utf8]{inputenc} 
%  
  
% definition of matlab colors: 
\definecolor{mblue}{rgb}{0,0,1} 
\definecolor{mgreen}{rgb}{0.13333,0.5451,0.13333} 
\definecolor{mred}{rgb}{0.62745,0.12549,0.94118} 
\definecolor{mgrey}{rgb}{0.5,0.5,0.5} 
\definecolor{mdarkgrey}{rgb}{0.25,0.25,0.25} 
  
\DefineShortVerb[fontfamily=courier,fontseries=m]{\$} 
\DefineShortVerb[fontfamily=courier,fontseries=b]{\#} 
  
\noindent       
 $$\color{mgreen}$% set S$\color{black}$$\\
 $S = zonotope([0 1 0 2;$\color{mblue}$ ...$\color{black}$$\\
 $              3 1 1 0;$\color{mblue}$ ...$\color{black}$$\\
 $              1 1 0 1]);$\\
 $$\\
 $$\color{mgreen}$% dimension of the set$\color{black}$$\\
 $res = dim(S)$\\ 
  
\UndefineShortVerb{\$} 
\UndefineShortVerb{\#}
\end{minipage}
\begin{minipage}[t]{0.25\textwidth}
	\vspace{10pt}

	\begin{verbatim}	
	Command Window:
	
	res = 3
	\end{verbatim}
\end{minipage}
\end{center}

\subsubsubsection{norm}
The method \operator{norm} returns the maximum norm value of the vector norm for points inside a set $\mathcal{S} \subset \Rn$: 
$$ \texttt{norm}(\mathcal{S},p) = \max_{x \in \mathcal{S}} \left\lVert x\right\rVert_p, ~~ p \in \{1,2,\dots,\infty\},$$
where the $p$-norm $\left\lVert \cdot \right\rVert_p$ is defined as
$$ \left\lVert x\right\rVert_p = \bigg(\sum_{i=1}^n \left| x_i \right|^p \bigg)^{1/p}.$$
Let us demonstrate the method \operator{norm} by an example:

\begin{center}
	\begin{minipage}[t]{0.4\textwidth}
		\vspace{10pt}
		\footnotesize
		% This file was automatically created from the m-file 
% "m2tex.m" written by USL. 
% The fontencoding in this file is UTF-8. 
%  
% You will need to include the following two packages in 
% your LaTeX-Main-File. 
%  
% \usepackage{color} 
% \usepackage{fancyvrb} 
%  
% It is advised to use the following option for Inputenc 
% \usepackage[utf8]{inputenc} 
%  
  
% definition of matlab colors: 
\definecolor{mblue}{rgb}{0,0,1} 
\definecolor{mgreen}{rgb}{0.13333,0.5451,0.13333} 
\definecolor{mred}{rgb}{0.62745,0.12549,0.94118} 
\definecolor{mgrey}{rgb}{0.5,0.5,0.5} 
\definecolor{mdarkgrey}{rgb}{0.25,0.25,0.25} 
  
\DefineShortVerb[fontfamily=courier,fontseries=m]{\$} 
\DefineShortVerb[fontfamily=courier,fontseries=b]{\#} 
  
\noindent      
 $$\color{mgreen}$% set S$\color{black}$$\\
 $S = zonotope([-0.5 1.5 0;$\color{mblue}$ ...$\color{black}$$\\
 $              -0.5 0 1.5]);$\\
 $$\\
 $$\color{mgreen}$% norm of the set$\color{black}$$\\
 $res = norm(S,2)$\\ 
  
\UndefineShortVerb{\$} 
\UndefineShortVerb{\#}
	\end{minipage}
	\begin{minipage}[t]{0.20\textwidth}
		\vspace{10pt}
		
		\begin{verbatim}
		Command Window:
		
		res =
		
		2.8284
		\end{verbatim}
	\end{minipage}
	\begin{minipage}[t]{0.3\textwidth}
		\vspace{0pt}
		\centering
		\includetikz{./figures/tikz/set-properties/example_norm}
	\end{minipage}
\end{center}

\subsubsubsection{vertices}

Given a set $\mathcal{S} \subset \Rn$, the method \operator{vertices} computes the vertices $v_1,\dots,v_q$, $v_i \in \Rn$ of the set:
\begin{equation*}
	[v_1,\dots,v_q] = \operator{vertices}(\mathcal{S}).
\end{equation*}
Please note that the computation of vertices can be computationally demanding for complex-shaped and/or high-dimensional sets.
Let us demonstrate the method \operator{vertices} by an example:

\begin{center}
\begin{minipage}[t]{0.3\textwidth}
	\vspace{10pt}
	\footnotesize
	% This file was automatically created from the m-file 
% "m2tex.m" written by USL. 
% The fontencoding in this file is UTF-8. 
%  
% You will need to include the following two packages in 
% your LaTeX-Main-File. 
%  
% \usepackage{color} 
% \usepackage{fancyvrb} 
%  
% It is advised to use the following option for Inputenc 
% \usepackage[utf8]{inputenc} 
%  
  
% definition of matlab colors: 
\definecolor{mblue}{rgb}{0,0,1} 
\definecolor{mgreen}{rgb}{0.13333,0.5451,0.13333} 
\definecolor{mred}{rgb}{0.62745,0.12549,0.94118} 
\definecolor{mgrey}{rgb}{0.5,0.5,0.5} 
\definecolor{mdarkgrey}{rgb}{0.25,0.25,0.25} 
  
\DefineShortVerb[fontfamily=courier,fontseries=m]{\$} 
\DefineShortVerb[fontfamily=courier,fontseries=b]{\#} 
  
\noindent      
 $$\color{mgreen}$% set S$\color{black}$$\\
 $S = interval([-2;-2],$\color{mblue}$ ...$\color{black}$$\\
 $             [1;1]);$\\
 $             $\\
 $$\color{mgreen}$% compute vertices$\color{black}$$\\
 $V = vertices(S)$\\ 
  
\UndefineShortVerb{\$} 
\UndefineShortVerb{\#}
\end{minipage}
\begin{minipage}[t]{0.32\textwidth}
	\vspace{10pt}

	\begin{verbatim}
	Command Window:
		
	V =

     1     1    -2    -2
     1    -2     1    -2
	\end{verbatim}
\end{minipage}
\begin{minipage}[t]{0.3\textwidth}
	\vspace{0pt}
	\centering
	\includetikz{./figures/tikz/set-properties/example_vertices}
\end{minipage}
\end{center}



\subsubsubsection{volume}

The method \operator{volume} returns the volume of a set. Let us demonstrate the method \operator{volume} by an example:

\begin{center}
\begin{minipage}[t]{0.4\textwidth}
	\vspace{10pt}
	\footnotesize
	% This file was created by matlab2tikz.
%
\definecolor{mycolor1}{rgb}{0.00000,0.44700,0.74100}%
%
\begin{tikzpicture}
\footnotesize

\begin{axis}[%
width=4cm,
height=4cm,
at={(0in,0in)},
scale only axis,
xmin=-2.4,
xmax=2.4,
xlabel style={font=\color{white!15!black}},
xlabel={$x_{(1)}$},
ymin=-2.4,
ymax=2.4,
ylabel style={font=\color{white!15!black}},
ylabel={$x_{(2)}$},
axis background/.style={fill=white},
title style={font=\bfseries},
title={$\mathcal{S}$}
]
\addplot [color=mycolor1, forget plot]
  table[row sep=crcr]{%
-2	-2\\
0	-2\\
2	0\\
2	2\\
0	2\\
-2	0\\
-2	-2\\
};
\end{axis}
\end{tikzpicture}%
\end{minipage}
\begin{minipage}[t]{0.2\textwidth}
	\vspace{10pt}

	\begin{verbatim}
	Command Window:
		
	res = 12
	\end{verbatim}
\end{minipage}
\begin{minipage}[t]{0.3\textwidth}
	\vspace{0pt}
	\centering
	\includetikz{./figures/tikz/set-properties/example_volume}
\end{minipage}
\end{center}
