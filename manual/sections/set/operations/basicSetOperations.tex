\subsubsection{Basic Set Operations}

We first consider basic operations on sets.

\subsubsubsection{mtimes}
\label{sec:mtimes}

The method \operator{mtimes}, which overloads the * operator, implements the linear map of a set. Given a set $\mathcal{S} \subset \R^n$, the linear map is defined as 
\begin{equation*}
	\operator{mtimes}(M,\mathcal{S}) = M \otimes \mathcal{S} = \{ M s ~|~ s \in \mathcal{S} \},~ M \in \mathbb{R}^{w \times n}.
\end{equation*}
It is also possible to consider a set of matrices $\mathcal{M} \subset \R^{w \times n}$ instead of a fixed-value matrix $M \in \R^{w \times n}$ (see \cref{sec:mtimesMatSet}). Let us demonstrate the method $\operator{mtimes}$ by an example:

\begin{center}
\begin{minipage}[t]{0.35\textwidth}
	\vspace{10pt}
	\footnotesize
	\definecolor{mycolor1}{rgb}{0.00000,0.44700,0.74100}%
\definecolor{mycolor2}{rgb}{0.46600,0.67400,0.18800}%
%
\begin{tikzpicture}
\footnotesize
\pgfplotsset{
plotstyle1/.style={color=mycolor1, forget plot},
plotstyle2/.style={color=mycolor2, forget plot}
}
\def\rows{1}
\def\cols{2}
\def\horzsep{1cm}
\def\basepath{./figures/tikz/matSet-operations/}

\begin{groupplot}[%
group style={rows = \rows, columns = \cols, horizontal sep = \horzsep},
scale only axis,
width=1/\cols*\textwidth -\horzsep,
legend style={legend columns=2,legend to name=legendname, legend cell align=left,/tikz/every even column/.append style={column sep=0.5cm}}
]
\nextgroupplot[xmin=-3,xmax=3,ymin=-3,ymax=3,xlabel={$x_{(1)}$},ylabel={$x_{(2)}$},title={$\mathcal{S}$}]
\input{\basepath example_mtimes_11.tikz}
\coordinate (top) at (rel axis cs:0,1);
\nextgroupplot[xmin=-3,xmax=3,ymin=-3,ymax=3,xlabel={$x_{(1)}$},title={$A\otimes\mathcal{S}$}]
\input{\basepath example_mtimes_legends.tikz}
\input{\basepath example_mtimes_12.tikz}
\coordinate (bot) at (rel axis cs:1,0);
\end{groupplot}
\path (top|-current bounding box.south)--coordinate(legendpos)(bot|-current bounding box.south);
\node at([yshift=-6ex]legendpos) {\pgfplotslegendfromname{legendname}};

\end{tikzpicture}%
\end{minipage}
\begin{minipage}[t]{0.6\textwidth}
	\vspace{0pt}
	\centering
	\includetikz{./figures/tikz/set-operations/example_mtimes}
\end{minipage}
\end{center}


\subsubsubsection{plus}
\label{sec:plus}

The method \operator{plus}, which overloads the + operator, implements the Minkowski sum of two sets. Given two sets $\mathcal{S}_1,\mathcal{S}_2 \subset \Rn$, the Minkowski sum is defined as 
\begin{equation*}
	\operator{plus}(\mathcal{S}_1,\mathcal{S}_2) = \mathcal{S}_1 \oplus \mathcal{S}_2 = \{ s_1 + s_2 ~|~ s_1 \in \mathcal{S}_1,~ s_2 \in \mathcal{S}_2 \}.
\end{equation*}
Let us demonstrate the method $\operator{plus}$ by an example:

\begin{center}
\begin{minipage}[t]{0.35\textwidth}
	\vspace{10pt}
	\footnotesize
	\input{./MATLABcode/example_plus}
\end{minipage}
\begin{minipage}[t]{0.6\textwidth}
	\vspace{0pt}
	\centering
	\includetikz{./figures/tikz/set-operations/example_plus}
\end{minipage}
\end{center}



\subsubsubsection{cartProd}
\label{sec:cartProd}

The method \operator{cartProd} implements the Cartesian product of two sets. Given two sets $\mathcal{S}_1 \subset \Rn$ and $\mathcal{S}_2 \subset \R^w$, the Cartesian product is defined as 
\begin{equation*}
	\operator{cartProd}(\mathcal{S}_1,\mathcal{S}_2) = \mathcal{S}_1 \times \mathcal{S}_2 = \{ [s_1 ~ s_2 ]^T ~|~ s_1 \in \mathcal{S}_1,~ s_2 \in \mathcal{S}_2 \}.
\end{equation*}
Let us demonstrate the method $\operator{cartProd}$ by an example:

\begin{center}
\begin{minipage}[t]{0.35\textwidth}
	\vspace{10pt}
	\footnotesize
	% This file was created by matlab2tikz.
%
\definecolor{mycolor1}{rgb}{0.00000,0.44700,0.74100}%
%
\begin{tikzpicture}
\footnotesize

\begin{axis}[%
width=4cm,
height=4cm,
at={(0in,0in)},
scale only axis,
xmin=-2.3,
xmax=1.3,
xlabel style={font=\color{white!15!black}},
xlabel={$x_{(1)}$},
ymin=-1.3,
ymax=2.3,
ylabel style={font=\color{white!15!black}},
ylabel={$x_{(2)}$},
axis background/.style={fill=white},
title style={font=\bfseries},
title={$\mathcal{S}_1\times\mathcal{S}_2$}
]
\addplot [color=mycolor1, forget plot]
  table[row sep=crcr]{%
-2	-1\\
1	-1\\
1	2\\
-2	2\\
-2	-1\\
};
\end{axis}
\end{tikzpicture}%
\end{minipage}
\begin{minipage}[t]{0.3\textwidth}
	\vspace{10pt}
	\begin{verbatim}
		Command Window:	
	
		res = 
 [-2.00000,1.00000]
 [-1.00000,2.00000]
	\end{verbatim}
\end{minipage}
\begin{minipage}[t]{0.3\textwidth}
	\vspace{0pt}
	\centering
	\includetikz{./figures/tikz/set-operations/example_cartProd}
\end{minipage}
\end{center}


\subsubsubsection{convHull}
\label{sec:convHull}

The method \operator{convHull} implements the convex hull of two sets. Given two sets $\mathcal{S}_1,\mathcal{S}_2 \subset \Rn$, the convex hull is defined as 
\begin{equation*}
	\operator{convHull}(\mathcal{S}_1,\mathcal{S}_2) = \left \{ \lambda s_1 + (1-\lambda) s_2 ~|~ s_1,s_2 \in \mathcal{S}_1 \cup \mathcal{S}_2,~\lambda \in [0,1] \right\}.
\end{equation*}
Furthermore, given a single non-convex set $\mathcal{S} \subset \Rn$, $\operator{convHull}(\mathcal{S})$ computes the convex hull of the set. Let us demonstrate the method $\operator{convHull}$ by an example:

\begin{center}
\begin{minipage}[t]{0.35\textwidth}
	\vspace{10pt}
	\footnotesize
	\input{./MATLABcode/example_convHull}
\end{minipage}
\begin{minipage}[t]{0.6\textwidth}
	\vspace{0pt}
	\centering
	\includetikz{./figures/tikz/set-operations/example_convHull}
\end{minipage}
\end{center}



\subsubsubsection{quadMap}
\label{sec:quadMap}

The method \operator{quadMap} implements the quadratic map of a set. Given a set $\mathcal{S} \subset \Rn$, the quadratic map is defined as 
\begin{equation*}
	\operator{quadMap}(\mathcal{S},Q) = \{ x ~|~ x_{(i)} = s^T Q_i s, ~s \in \mathcal{S},~ i = 1 \dots w \}, ~ Q_i \in \R^{n \times n},
\end{equation*}
where $x_{(i)}$ is the $i$-th value of the vector $x$. If \operator{quadMap} is called with two different sets $\mathcal{S}_1,\mathcal{S}_2 \subset \Rn$ as input arguments, the method computes the mixed quadratic map:
\begin{equation*}
	\operator{quadMap}(\mathcal{S}_1,\mathcal{S}_2,Q) = \{ x ~|~ x_{(i)} = s_1^T Q_i s_2, ~s_1 \in \mathcal{S}_1,~s_2 \in \mathcal{S}_2,~ i = 1 \dots w \}, ~ Q_i \in \R^{n \times n}.
\end{equation*}
Let us demonstrate the method $\operator{quadMap}$ by an example:

\begin{center}
\begin{minipage}[t]{0.35\textwidth}
	\vspace{10pt}
	\footnotesize
	\input{./MATLABcode/example_quadMap}
\end{minipage}
\begin{minipage}[t]{0.6\textwidth}
	\vspace{0pt}
	\centering
	\includetikz{./figures/tikz/set-operations/example_quadMap}
\end{minipage}
\end{center}

\vspace{1cm}

\subsubsubsection{and}
\label{sec:and}

The method \operator{and}, which overloads the \& operator, implements the intersection of two sets. Given two sets $\mathcal{S}_1,\mathcal{S}_2 \subset \Rn$, the intersection is defined as 
\begin{equation*}
	\operator{and}(\mathcal{S}_1,\mathcal{S}_2) = \mathcal{S}_1 \cap \mathcal{S}_2 = \{ s ~|~ s \in \mathcal{S}_1,~ s \in \mathcal{S}_2 \}.
\end{equation*}
Let us demonstrate the method $\operator{and}$ by an example:

\begin{center}
\begin{minipage}[t]{0.35\textwidth}
	\vspace{10pt}
	\footnotesize
	\input{./MATLABcode/example_and}
\end{minipage}
\begin{minipage}[t]{0.6\textwidth}
	\vspace{0pt}
	\centering
	\includetikz{./figures/tikz/set-operations/example_and}
\end{minipage}
\end{center}

\subsubsubsection{or}
\label{sec:or}

The method \operator{or}, which overloads the $|$ operator, implements the union of two sets. Given two sets $\mathcal{S}_1,\mathcal{S}_2 \subset \Rn$, their union is defined as 
\begin{equation*}
	\operator{or}(\mathcal{S}_1,\mathcal{S}_2) = \mathcal{S}_1 \cup \mathcal{S}_2 = \{ s ~|~ s \in \mathcal{S}_1 \vee s \in \mathcal{S}_2 \}.
\end{equation*}
Let us demonstrate the method $\operator{or}$ by an example:

\begin{center}
\begin{minipage}[t]{0.35\textwidth}
	\vspace{10pt}
	\footnotesize
	\definecolor{mycolor1}{rgb}{0.00000,0.44700,0.74100}%
\definecolor{mycolor2}{rgb}{0.85000,0.32500,0.09800}%
\definecolor{mycolor3}{rgb}{0.46600,0.67400,0.18800}%
%
\begin{tikzpicture}
\footnotesize
\pgfplotsset{
plotstyle1/.style={color=mycolor1, forget plot},
plotstyle2/.style={color=mycolor2, forget plot},
plotstyle3/.style={color=mycolor3, forget plot}
}
\def\rows{1}
\def\cols{2}
\def\horzsep{1cm}
\def\basepath{./figures/tikz/set-operations/}

\begin{groupplot}[%
group style={rows = \rows, columns = \cols, horizontal sep = \horzsep},
scale only axis,
width=1/\cols*\textwidth -\horzsep,
legend style={legend columns=2,legend to name=legendname, legend cell align=left,/tikz/every even column/.append style={column sep=0.5cm}}
]
\nextgroupplot[xmin=-3,xmax=3,ymin=-3,ymax=3,xlabel={$x_{(1)}$},ylabel={$x_{(2)}$},title={$\mathcal{S}_1$ and $\mathcal{S}_2$}]
\input{\basepath example_or_11.tikz}
\coordinate (top) at (rel axis cs:0,1);
\nextgroupplot[xmin=-3,xmax=3,ymin=-3,ymax=3,xlabel={$x_{(1)}$},title={$\mathcal{S}_1\cup\mathcal{S}_2$}]
\input{\basepath example_or_legends.tikz}
\input{\basepath example_or_12.tikz}
\coordinate (bot) at (rel axis cs:1,0);
\end{groupplot}
\path (top|-current bounding box.south)--coordinate(legendpos)(bot|-current bounding box.south);
\node at([yshift=-6ex]legendpos) {\pgfplotslegendfromname{legendname}};

\end{tikzpicture}%
\end{minipage}
\begin{minipage}[t]{0.6\textwidth}
	\vspace{0pt}
	\centering
	\includetikz{./figures/tikz/set-operations/example_or}
\end{minipage}
\end{center}


\begin{table*}
\centering
	\caption{Relations between set representations and set operations. The shortcuts e (exact computation) and o (over-approximation) are used. The symbol e* indicates that the operation is exact if no independent generators (see \cref{sec:polynomialZonotopes} and \cref{sec:conPolyZono} for details) are used.}
	\label{tab:basicOperations}
	\begin{tabular}{ p{2.5cm} C{1.4cm} C{1.4cm} C{1.4cm} C{1.4cm} C{1.4cm} C{1.4cm} C{1.4cm}}
		 \toprule
		 \textbf{Set Rep.} & \textbf{Lin. Map} & \textbf{Mink. Sum} & \textbf{Cart. Prod.} & \textbf{Conv. Hull} & \textbf{Quad. Map} & \textbf{Inter- section} & \textbf{Union} \\
		 \midrule
		 \operator{interval} & o & e & e & o &  & e & o \\
		 \operator{zonotope} & e & e & e & o & o & o & o  \\
		 \operator{polytope} & e & e & e & e &  & e & o \\
		 \operator{conZonotope} & e & e & e & e & o & e & o \\
		 \operator{zonoBundle} & e & e & e & e & o & e & o \\
		 \operator{ellipsoid} & e  & o & o & o &  & o &  \\
		 \operator{capsule} & e & o & & & & & \\
		 \operator{taylm} & e & e & e & & & & \\
		 \operator{polyZonotope} & e & e & e & e* & e* &  & \\
		 \operator{conPolyZono} & e & e & e & e* & e* & e* & e* \\
		 \bottomrule
	\end{tabular}
\end{table*}

\subsubsubsection{minkDiff}
\label{sec:minkDiff}

The method \operator{minkDiff} implements the Minkowski difference of two sets. Given two sets $\mathcal{S}_1,\mathcal{S}_2 \subset \Rn$, their Minkowski difference is defined as 
\begin{equation*}
	\operator{minkDiff}(\mathcal{S}_1,\mathcal{S}_2) = \{ s \in \R^n ~|~ s \oplus \mathcal{S}_2 \subseteq \mathcal{S}_1 \}.
\end{equation*}
Let us demonstrate the method $\operator{minkDiff}$ by an example:

\begin{center}
	\begin{minipage}[t]{0.35\textwidth}
		\vspace{10pt}
		\footnotesize
		% This file was automatically created from the m-file 
% "m2tex.m" written by USL. 
% The fontencoding in this file is UTF-8. 
%  
% You will need to include the following two packages in 
% your LaTeX-Main-File. 
%  
% \usepackage{color} 
% \usepackage{fancyvrb} 
%  
% It is advised to use the following option for Inputenc 
% \usepackage[utf8]{inputenc} 
%  
  
% definition of matlab colors: 
\definecolor{mblue}{rgb}{0,0,1} 
\definecolor{mgreen}{rgb}{0.13333,0.5451,0.13333} 
\definecolor{mred}{rgb}{0.62745,0.12549,0.94118} 
\definecolor{mgrey}{rgb}{0.5,0.5,0.5} 
\definecolor{mdarkgrey}{rgb}{0.25,0.25,0.25} 
  
\DefineShortVerb[fontfamily=courier,fontseries=m]{\$} 
\DefineShortVerb[fontfamily=courier,fontseries=b]{\#} 
  
\noindent       
 $$\color{mgreen}$% set 1 and set 2$\color{black}$$\\
 $S1 = zonotope($\color{mblue}$ ...$\color{black}$$\\
 $        [0.5 0.5 -0.3 1 0;$\color{mblue}$ ...$\color{black}$$\\
 $         0   0.2  1   0 1]);$\\
 $S2 = interval([-1;-1],[1;1]);$\\
 $$\\
 $$\color{mgreen}$% Minkowski difference$\color{black}$$\\
 $res = minkDiff(S1,S2);$\\ 
  
\UndefineShortVerb{\$} 
\UndefineShortVerb{\#}
	\end{minipage}
	\begin{minipage}[t]{0.6\textwidth}
		\vspace{0pt}
		\centering
		\includetikz{./figures/tikz/set-operations/example_minkDiff}
	\end{minipage}
\end{center}
