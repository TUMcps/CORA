\subsection{Signal Temporal Logic}
\label{sec:temporalLogic}

Signal temporal logic is a common formalism to represent complex specifications that describe the desired behavior of a system. In CORA, signal temporal logic formulas are represented by the class \texttt{stl}. An object of class \texttt{stl} can be constructed as follows:
\begin{equation*}
	\texttt{obj = stl(name,}n\texttt{)},
\end{equation*}
where \texttt{name} is a string specifying the name of the variable and $n$ is the dimension of the variable. The variables constructed with the constructor of the class \texttt{stl} correspond to the system states. These variables can be used to construct predicates with the operators \texttt{+}, \texttt{-}, \texttt{*}, \texttt{<}, \texttt{<=}, \texttt{>}, and \texttt{>=}. In addition, set containment $x \in \mathcal{S}$ can be realized with the function $\texttt{in(}x,\mathcal{S}\texttt{)}$, where $\mathcal{S}$ is a continuous set (see \cref{sec:setRepresentations}). The predicates can then be used as inputs for the signal temporal logic operators in \cref{tab:signalTemporalLogic}. Note that \texttt{stlInterval} objects, and thus the temporal operators, also support open and half-open intervals (see the example below). The resulting signal temporal logic formula can be used to construct a system specification (see \cref{sec:specification}) which is checked during reachability analysis, where CORA implements the approach in \cite{Roehm2016b} to check if the reachable set satisfies temporal logic specifications. Moreover, the incremental verification algorithm for signal temporal logic from \cite{Lercher2024} is implemented in the method \texttt{modelChecking} of the class \texttt{reachSet}.

Let us demonstrate signal temporal logic in CORA by an example:

\begin{center}
\begin{minipage}[t]{0.65\textwidth}
	\vspace{10pt}
	\footnotesize
	% This file was automatically created from the m-file 
% "m2tex.m" written by USL. 
% The fontencoding in this file is UTF-8. 
%  
% You will need to include the following two packages in 
% your LaTeX-Main-File. 
%  
% \usepackage{color} 
% \usepackage{fancyvrb} 
%  
% It is advised to use the following option for Inputenc 
% \usepackage[utf8]{inputenc} 
%  
  
% definition of matlab colors: 
\definecolor{mblue}{rgb}{0,0,1} 
\definecolor{mgreen}{rgb}{0.13333,0.5451,0.13333} 
\definecolor{mred}{rgb}{0.62745,0.12549,0.94118} 
\definecolor{mgrey}{rgb}{0.5,0.5,0.5} 
\definecolor{mdarkgrey}{rgb}{0.25,0.25,0.25} 
  
\DefineShortVerb[fontfamily=courier,fontseries=m]{\$} 
\DefineShortVerb[fontfamily=courier,fontseries=b]{\#} 
  
\noindent     
 $$\color{mgreen}$% create variable$\color{black}$$\\
 $x = stl($\color{mred}$'x'$\color{black}$,2);$\\
 $$\\
 $$\color{mgreen}$% signal temporal logic formula$\color{black}$$\\
 $eq = until(x(1) < 3,x(2) > 5, ...$\\
 $    stlInterval(1,3,true,false))$\\ 
  
\UndefineShortVerb{\$} 
\UndefineShortVerb{\#}
\end{minipage}
\begin{minipage}[t]{0.3\textwidth}
	\vspace{10pt}
	\begin{verbatim}
		Command Window:	
	
eq = 

(x1 < 3 U[1,3) x2 > 5)
  
  
	\end{verbatim}
\end{minipage}  
\end{center}


\begin{table}
\centering
\caption{Operators for signal temporal logic, where $\xi(t)$ is a trace and $\models$ denotes entailment.}
\label{tab:signalTemporalLogic}
\begin{tabular}{ l l l}	
\toprule
\textbf{Operator} & \textbf{CORA} & \textbf{Definition} \\
\midrule
$\phi \wedge \psi$ & \texttt{p \& q} & $\xi(t) \models \phi \wedge \xi(t) \models \psi$\\
$\phi \vee \psi$ & \texttt{p | q} & $\xi(t) \models \phi \vee \xi(t) \models \psi$\\
$\neg \, \phi$ & $\sim$\texttt{p} & $\xi(t) \models \neg \phi$\\
$\phi \Rightarrow \psi$ & \texttt{implies(p,q)} & $\xi(t) \models \phi \Rightarrow \xi(t) \models \psi$\\
$\mathcal{X}_{a} \, \phi$ & \texttt{next(p,a)} & $\xi(t+a) \models \phi$\\
$\mathcal{F}_{[a,b]} \, \phi$ & \texttt{finally(p,stlInterval(a,b))} & $\exists t \in [a,b]: \xi(t) \models \phi$ \\
$\mathcal{G}_{[a,b]} \, \phi$ & \texttt{globally(p,stlInterval(a,b))} & $\forall t \in [a,b]: \xi(t) \models \phi$ \\
$\phi \, \mathcal{U}_{[a,b]} \, \psi$ & \texttt{until(p,q,stlInterval(a,b))} & $\exists t \in [a,b]: \xi(t) \models \psi \wedge \forall t' \in [0,t): \xi(t') \models \phi$ \\
$\phi \, \mathcal{R}_{[a,b]} \, \psi$ \hspace{-10pt} & \texttt{release(p,q,stlInterval(a,b))} \hspace{-10pt} & $\forall t \in [a,b]: \xi(t) \models \psi \vee \exists t' \in [0,t): \xi(t') \models \phi$ \\
\bottomrule
\end{tabular}
\end{table}
