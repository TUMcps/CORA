\subsection{Class \texttt{simResult}}
\label{sec:simResult}

The results of simulations are stored in CORA as objects of the class \texttt{simResult}, which provides several methods to easily visualize the simulated trajectories. An object of class \texttt{simResult} can be constructed as follows:
\begin{equation*}
	\begin{split}
		& \texttt{simRes} = \texttt{simResult}(\texttt{x},\texttt{t}), \\
		& \texttt{simRes} = \texttt{simResult}(\texttt{x},\texttt{t},\texttt{loc}),
	\end{split}
\end{equation*}
with input arguments
\begin{center}
\renewcommand{\arraystretch}{1.3}
\begin{tabular}[t]{l p{13cm} }
	$\bullet$~\texttt{x} & cell array storing the states of the simulated trajectories.\\
	$\bullet$~\texttt{t} & cell array storing the time points for the simulated trajectories. \\
	$\bullet$~\texttt{loc} & cell array storing the indices of the locations for the simulated trajectories (hybrid systems only).
\end{tabular}
\end{center}

Next, we explain the methods of the class \texttt{simResult} in detail.

\subsubsection{\texttt{add}}

The method \texttt{add} combines two \texttt{simResult} objects \texttt{simRes1} and \texttt{simRes2}:
\begin{equation*}
	\texttt{simRes} = \texttt{add}(\texttt{simRes1},\texttt{simRes2}).
\end{equation*}


\subsubsection{\texttt{plot}}

The method \texttt{plot} visualizes a two-dimensional projection of the obtained trajectories:

\begin{equation*}
	\begin{split}
		&\texttt{han} = \operator{plot}(\texttt{simRes}), \\
		&\texttt{han} = \operator{plot}(\texttt{simRes},\texttt{dims}), \\
		&\texttt{han} = \operator{plot}(\texttt{simRes},\texttt{dims},\texttt{linespec}), \\
		&\texttt{han} = \operator{plot}(\texttt{simRes},\texttt{dims},\texttt{namevaluepairs}),
	\end{split}
\end{equation*}
where \texttt{simRes} is an object of class \texttt{simResult}, \texttt{han} is a handle to the plotted MATLAB graphics object, and the additional input arguments are defined as

\begin{itemize}
	\item \texttt{dims}: Integer vector $\texttt{dims} \in \mathbb{N}_{\leq n}^2$ specifying the dimensions for which the projection is visualized (default value: \texttt{dims = [1 2]}).
	\item \texttt{linespec}: (optional) line specifications, e.g., \texttt{'--*r'}, as supported by MATLAB\footnote{\url{https://de.mathworks.com/help/matlab/ref/linespec.html}} (default value: \texttt{linespec = 'b'}).
	\item \texttt{namevaluepairs}: (optional) further specifications as name-value pairs, e.g.,
	\texttt{'LineWidth',2} and \texttt{'MarkerSize',1.5}, as supported by MATLAB.
	They correspond to the Line Properties\footnote{\url{https://de.mathworks.com/help/matlab/ref/matlab.graphics.chart.primitive.line-properties.html}}.
\end{itemize}


\subsubsection{\texttt{plotOverTime}}

The method \texttt{plotOverTime} visualizes a one-dimensional projection of the simulated trajectories over time:

\begin{equation*}
	\begin{split}
		&\texttt{han} = \operator{plotOverTime}(\texttt{simRes}), \\
		&\texttt{han} = \operator{plotOverTime}(\texttt{simRes},\texttt{dims}), \\
		&\texttt{han} = \operator{plotOverTime}(\texttt{simRes},\texttt{dims},\texttt{linespec}), \\
		&\texttt{han} = \operator{plotOverTime}(\texttt{simRes},\texttt{dims},\texttt{namevaluepairs}),
	\end{split}
\end{equation*}
where \texttt{simRes} is an object of class \texttt{simResult}, \texttt{han} is a handle to the plotted MATLAB graphics object, and the additional input arguments are defined as

\begin{itemize}
	\item \texttt{dims}: Integer vector $\texttt{dims} \in \mathbb{N}_{\leq n}$ specifying the dimensions for which the projection is visualized (default value: \texttt{dims = 1}).
	\item \texttt{linespec}: (optional) line specifications, e.g., \texttt{'--*r'}, as supported by MATLAB\footnote{\url{https://de.mathworks.com/help/matlab/ref/linespec.html}}.
	\item \texttt{namevaluepairs}: (optional) further specifications as name-value pairs, e.g.,
	\texttt{'LineWidth',2} and \texttt{'MarkerSize',1.5}, as supported by MATLAB.
	They correspond to the Line Properties\footnote{\url{https://de.mathworks.com/help/matlab/ref/matlab.graphics.primitive.patch-properties.html}}.
\end{itemize}
