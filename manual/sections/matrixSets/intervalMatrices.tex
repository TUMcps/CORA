\subsubsection{Interval Matrices} \label{sec:intervalMatrix}

An interval matrix is a special case of a matrix zonotope and specifies the interval of possible values for each matrix element:
\begin{equation*}
	\mathcal{A}_{[i]}=[\underline{A},\overline{A}], \quad  \forall i,j: \underline{a}_{ij}\leq\overline{a}_{ij}, \quad \underline{A},\overline{A}\in\mathbb{R}^{n \times n}.
\end{equation*}
The matrix $\underline{A}$ is referred to as the \textit{lower bound} and $\overline{A}$ as the \textit{upper bound} of $\mathcal{A}_{[i]}$. 

In CORA, interval matrices are implemented by the class \texttt{intervalMatrix}. An object of class \texttt{intervalMatrix} can be constructed as follows:
\begin{equation*}
	\mathcal{A}_{[i]} = \texttt{intervalMatrix}(C,D),
\end{equation*}
where $C = 0.5 (\overline{A}+\underline{A})$ is the center matrix and $D = 0.5 (\overline{A}-\underline{A})$ is the width matrix.

Let us demonstrate the construction of an \texttt{intervalMatrix} object by an example:

\begin{center}
\begin{minipage}[t]{0.40\textwidth}
	\footnotesize
	% This file was automatically created from the m-file 
% "m2tex.m" written by USL. 
% The fontencoding in this file is UTF-8. 
%  
% You will need to include the following two packages in 
% your LaTeX-Main-File. 
%  
% \usepackage{color} 
% \usepackage{fancyvrb} 
%  
% It is advised to use the following option for Inputenc 
% \usepackage[utf8]{inputenc} 
%  
  
% definition of matlab colors: 
\definecolor{mblue}{rgb}{0,0,1} 
\definecolor{mgreen}{rgb}{0.13333,0.5451,0.13333} 
\definecolor{mred}{rgb}{0.62745,0.12549,0.94118} 
\definecolor{mgrey}{rgb}{0.5,0.5,0.5} 
\definecolor{mdarkgrey}{rgb}{0.25,0.25,0.25} 
  
\DefineShortVerb[fontfamily=courier,fontseries=m]{\$} 
\DefineShortVerb[fontfamily=courier,fontseries=b]{\#} 
  
\noindent        
 $$\color{mgreen}$% center matrix$\color{black}$$\\
 $C = [0 2; 3 1];$\\
 $$\\
 $$\color{mgreen}$% width matrix$\color{black}$$\\
 $D = [1 2; 1 1];$\\
 $$\\
 $$\color{mgreen}$% interval matrix$\color{black}$$\\
 $mi = intervalMatrix(C,D);$\\ 
  
\UndefineShortVerb{\$} 
\UndefineShortVerb{\#}
\end{minipage}
\end{center}

A more detailed example for interval matrices is provided in \cref{sec:intervalMatrixExample} and in the file \textit{examples/matrixSet/example\_intervalMatrix.m} in the CORA toolbox. Furthermore, a list of methods for the class \texttt{intervalMatrix} is provided in \cref{sec:intervalMatrixOperations}.
