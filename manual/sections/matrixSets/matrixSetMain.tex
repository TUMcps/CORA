\section{Matrix Set Representations and Operations} \label{sec:matrixSetRepresentationsAndOperations}
\logToConsole{MATRIX SET REPRESENTATIONS AND OPERATIONS}

Besides vector sets as introduced in the previous section, it is often useful to represent sets of possible matrices. This occurs for instance when a linear system has uncertain parameters as described later in \cref{sec:linearParamSystems}. CORA supports the following matrix set representations:
\begin{itemize}
    \item Matrix polytope (\cref{sec:polytopeMatrix}).
    \item Matrix zonotope (\cref{sec:zonotopeMatrix}, specialization of a matrix polytope).
    \item Interval matrix (\cref{sec:intervalMatrix}, specialization of a matrix zonotope).
\end{itemize}
Note that we use the term \textit{matrix polytope} instead of \textit{polytope matrix}. The reason is that the analogous term \textit{vector polytope} makes sense, while \textit{polytope vector} can be misinterpreted as a vertex of a polytope. We do not use the term \textit{matrix interval} since the term \textit{interval matrix} is already established.

For each matrix set representation, the conversion to all other matrix set computations is implemented. Of course, conversions to specializations are realized in an over-approximative way, while the other direction is computed exactly (see \cref{tab:matSetConversion}). In order to convert a matrix set, it is sufficient to pass the current matrix set object to the class constructor of the target matrix set representation, as demonstrated by the following example:

    {\footnotesize \input{./MATLABcode/example_matSetConversion}}

\begin{table}[htb]
    \centering
    \caption{Matrix set conversions supported by CORA. The row headers represent the original matrix set representation and the column headers the target matrix set representation after conversion. The shortcuts e (exact conversion) and o (over-approximation) are used.}
    \label{tab:matSetConversion}
    \begin{tabular}{ l c c c}
        \toprule
        & \textbf{matPolytope} & \textbf{matZonotope} & \textbf{intervalMatrix} \\
        \midrule
        \textbf{matPolytope} (\cref{sec:polytopeMatrix})    & -                    & o                    & o                       \\
        \textbf{matZonotope} (\cref{sec:zonotopeMatrix})    & e                    & -                    & o                       \\
        \textbf{intervalMatrix} (\cref{sec:intervalMatrix}) & e                    & e                    & -                       \\
        \bottomrule
    \end{tabular}
\end{table}

We first introduce importrant operations for matrix sets in \cref{sec:matSetOperations} before we describe the matrix set representations implemented in detail in \cref{sec:matSetRepresentations}.


\newpage

% matrix set operations
\subsection{Matrix Set Operations} \label{sec:matSetOperations}

This section describes the implemented standard operations for matrix sets.

\subsubsection{mtimes}
\label{sec:mtimesMatSet}

The method \texttt{mtimes}, which overloads the $*$ operator, implements the multiplication of two matrix sets or the multiplication of a matrix set with a vector set, depending on the input arguments. Given two matrix sets $\mathcal{A}_1,\mathcal{A}_2 \subset \R^{n \times n}$ and a vector set $\mathcal{S} \subset \Rn$, the method \texttt{mtimes} computes
\begin{equation*}
	\begin{split}
		& \texttt{mtimes}(\mathcal{A}_1,\mathcal{A}_2) = \mathcal{A}_1 \otimes \mathcal{A}_2 = \{ A_1 \cdot A_2~|~ A_1 \in \mathcal{A}_1,~A_2 \in \mathcal{A}_2 \}, \\
		& \texttt{mtimes}(\mathcal{A}_1,\mathcal{S}) = \mathcal{A}_1 \otimes \mathcal{S} = \{ A_1 \cdot s ~|~ A_1 \in \mathcal{A}_1,~s \in \mathcal{S} \}.
	\end{split}
\end{equation*}  
If the corresponding matrix set representation is not closed under multiplication, \operator{mtimes} returns an over-approximation. Let us demonstrate the method \texttt{mtimes} by an example:

\begin{center}
\begin{minipage}[t]{0.35\textwidth}
	\footnotesize
	\input{./MATLABcode/example_mtimesMatSet}
\end{minipage}
\begin{minipage}[t]{0.6\textwidth}
	\vspace{0pt}
	\centering
	\includetikz{./figures/tikz/matSet-operations/example_mtimes}
\end{minipage}
\end{center}

\vspace{1cm}
\subsubsection{plus} \label{sec:plusMatSet}

The method \texttt{plus}, which overloads the $+$ operator, implements the Minkowski sum of two matrix sets. Given two matrix sets $\mathcal{A}_1,\mathcal{A}_2 \subset \R^{n \times n}$, their Minkowski sum is defined as 
\begin{equation*}
	\texttt{plus}(\mathcal{A}_1,\mathcal{A}_2) = \mathcal{A}_1 \oplus \mathcal{A}_2 = \{ A_1 + A_2 ~|~ A_1 \in \mathcal{A}_1,~ A_2 \in \mathcal{A}_2 \}.
\end{equation*}
If the corresponding matrix set representation is not closed under Minkowski sum, \operator{plus} returns an over-approximation. Let us demonstrate the method \texttt{plus} by an example:

\begin{center}
\begin{minipage}[t]{0.55\textwidth}
	\vspace{10pt}
	\footnotesize
	\input{./MATLABcode/example_plusMatSet}
\end{minipage}
\begin{minipage}[t]{0.43\textwidth}
	\vspace{10pt}
	\begin{verbatim}
		Command Window:	
	
		res = 
  [2.000,4.000] [0.000,6.000]
  [3.000,5.000] [4.000,6.000]
	\end{verbatim}
\end{minipage}
\end{center}

\newpage
\subsubsection{expm} \label{expmMatSet}

Given a matrix set $\mathcal{A} \subset \R^{n \times n}$, the method \texttt{expm} computes a tight enclosure of the matrix exponential
\begin{equation*}
	\texttt{expm}(\mathcal{A}) \supseteq e^{\mathcal{A}} = \sum_{i=0}^\infty \frac{\mathcal{A}^k}{k!} \; .
\end{equation*}
The number of Taylor terms $\eta$ used for the calculation of the matrix exponential (see \cite[Theorem~3.2]{Althoff2010a}) can be specified as an additional input argument:
\begin{equation*}
	\texttt{expm}(\mathcal{A},\eta) \supseteq e^{\mathcal{A}}.
\end{equation*}
The computation of a tight enclosure of the matrix exponential for matrix sets is essential for reachability analysis of linear parametric systems (see \cref{sec:linearParamSystems}). Let us demonstrate the method \texttt{expm} by an example:

\begin{center}
\begin{minipage}[t]{0.35\textwidth}
	\vspace{10pt}
	\footnotesize
	\input{./MATLABcode/example_expmMatSet}
\end{minipage}
\begin{minipage}[t]{0.5\textwidth}
	\vspace{10pt}
	\begin{verbatim}
		Command Window:	
	
		res = 
  [1.00000,1.00000] [-1.21072,1.95859]
  [0.00000,0.00000] [-5.25685,5.44556]
	\end{verbatim}
\end{minipage}
\end{center}

\vspace{1cm}
\subsubsection{vertices} \label{sec:verticesMatSet}

Given a matrix set $\mathcal{A} \subset \R^{n \times n}$, the method \texttt{vertices} computes its vertices $V_1,\dots,V_q$, $V_i \in \R^{n \times n}$:
\begin{equation*}
	\texttt{vertList} = \texttt{vertices}(\mathcal{A}),
\end{equation*}
where \texttt{vertList} is a MATLAB cell array that stores the vertices $V_i$.
Let us demonstrate the method \texttt{vertices} by an example:

\begin{center}
\begin{minipage}[t]{0.35\textwidth}
	\vspace{10pt}
	\footnotesize
	\input{./MATLABcode/example_verticesMatSet}
\end{minipage}
\begin{minipage}[t]{0.3\textwidth}
	\vspace{10pt}
	\begin{verbatim}
		Command Window:	
	
		res{1} = 
   -1.0000   -1.0000
    3.0000    1.0000

	\end{verbatim}
\end{minipage}
\begin{minipage}[t]{0.3\textwidth}
	\vspace{31pt}
	\begin{verbatim}
	
		res{2} = 
    1.0000    3.0000
    3.0000    3.0000

	\end{verbatim}
\end{minipage}
\end{center}


\newpage

\subsection{Matrix Set Representations} \label{sec:matSetRepresentations}

This section describes the different matrix set representations implemented in CORA.

% matrix polytopes
\subsubsection{Matrix Polytopes} \label{sec:polytopeMatrix}

A matrix polytope is defined analogously to a V-polytope (see \cref{sec:polytopes}):
\begin{equation}\label{eq:matrixVertices}
 \mathcal{A}_{[p]}=\bigg \{ \sum_{i=1}^{r} \alpha_i V\^i ~\Big|~ \alpha_i \in\mathbb{R}, \alpha_i \geq 0, \sum_i \alpha_i=1 \bigg \}, \quad V\^i \in \mathbb{R}^{n\times m}.
\end{equation}
The matrices $V\^i$ are also called vertices of the matrix polytope.
When substituting the matrix vertices by vector vertices $v\^{i}\in \mathbb{R}^{n}$, one obtains a V-polytope (see \cref{sec:polytopes}).

Matrix polytopes are implemented in CORA by the class \texttt{matPolytope}.
An object of class \texttt{matPolytope} can be constructed as follows:
\begin{equation*}
	 \mathcal{A}_{[p]} = \texttt{matPolytope}(\texttt{V}),
\end{equation*}
where \texttt{V} is a three-dimensional array ($n\times m\times r$) that stores the vertices $V^{(i)}$, see \eqref{eq:matrixVertices}, of the matrix polytope.

Let us demonstrate the construction of a \texttt{matPolytope} object by an example:

\begin{center}
\begin{minipage}[t]{0.40\textwidth}
	\footnotesize
	\input{./MATLABcode/example_matPolytope}
\end{minipage}
\end{center}

A more detailed example for matrix polytopes is provided in \cref{sec:matrixPolytopeExample} and in the file \textit{examples/matrixSet/example\_matPolytope.m} in the CORA toolbox. Furthermore, a list of methods for the class \texttt{matPolytope} is provided in \cref{sec:matrixPolytopeOperations}.


% matrix zonotopes
\subsubsection{Matrix Zonotopes} \label{sec:zonotopeMatrix}

A matrix zonotope is defined analogously to zonotopes (see \cref{sec:zonotope}):
\begin{equation}\label{eq:matrixZonotope}
	\mathcal{A}_{[z]}=\Big\{ C+\sum_{i=1}^{\kappa} p_i G\^{i} \Big| p_i \in [-1,1] \Big\}, \quad G\^{i}\in\mathbb{R}^{n \times m},
\end{equation}
and is written in short form as $\mathcal{A}_{[z]}=(C, G\^1, \ldots , G\^\kappa)$,
where the first matrix is referred to as the \textit{matrix center} and the other matrices as \textit{matrix generators}.
The order of a matrix zonotope is defined as $\rho = \kappa/n$. When exchanging the matrix generators by vector generators $g\^{i}\in \mathbb{R}^{n\cdot m}$,
one obtains a zonotope (see e.g., \cite{Girard2005}).

Matrix zonotopes are implemented by the class \texttt{matZonotope}. An object of class \texttt{matZonotope} can be constructed as follows:
\begin{equation*}
	\mathcal{A}_{[z]} = \texttt{matZonotope}(C,\texttt{G}),
\end{equation*}
where \texttt{G} is a three-dimensional array ($n\times m\times \kappa$) that stores the generator matrices, see \eqref{eq:matrixZonotope}.

\newpage
Let us demonstrate the construction of a \texttt{matZonotope} object by an example:

\begin{center}
\begin{minipage}[t]{0.40\textwidth}
	\footnotesize
	\input{./MATLABcode/example_matZonotope}
\end{minipage}
\end{center}

A more detailed example for matrix zonotopes is provided in \cref{sec:matrixZonotopeExample} and in the file \textit{examples/matrixSet/example\_matZonotope.m} in the CORA toolbox. Furthermore, a list of methods for the class \texttt{matZonotope} is provided in \cref{sec:matrixZonotopeOperations}.


% interval matrix
\subsubsection{Interval Matrices} \label{sec:intervalMatrix}

An interval matrix is a special case of a matrix zonotope and specifies the interval of possible values for each matrix element:
\begin{equation*}
	\mathcal{A}_{[i]}=[\underline{A},\overline{A}], \quad  \forall i,j: \underline{a}_{ij}\leq\overline{a}_{ij}, \quad \underline{A},\overline{A}\in\mathbb{R}^{n \times n}.
\end{equation*}
The matrix $\underline{A}$ is referred to as the \textit{lower bound} and $\overline{A}$ as the \textit{upper bound} of $\mathcal{A}_{[i]}$. 

In CORA, interval matrices are implemented by the class \texttt{intervalMatrix}. An object of class \texttt{intervalMatrix} can be constructed as follows:
\begin{equation*}
	\mathcal{A}_{[i]} = \texttt{intervalMatrix}(C,D),
\end{equation*}
where $C = 0.5 (\overline{A}+\underline{A})$ is the center matrix and $D = 0.5 (\overline{A}-\underline{A})$ is the width matrix.

Let us demonstrate the construction of an \texttt{intervalMatrix} object by an example:

\begin{center}
\begin{minipage}[t]{0.40\textwidth}
	\footnotesize
	% This file was automatically created from the m-file 
% "m2tex.m" written by USL. 
% The fontencoding in this file is UTF-8. 
%  
% You will need to include the following two packages in 
% your LaTeX-Main-File. 
%  
% \usepackage{color} 
% \usepackage{fancyvrb} 
%  
% It is advised to use the following option for Inputenc 
% \usepackage[utf8]{inputenc} 
%  
  
% definition of matlab colors: 
\definecolor{mblue}{rgb}{0,0,1} 
\definecolor{mgreen}{rgb}{0.13333,0.5451,0.13333} 
\definecolor{mred}{rgb}{0.62745,0.12549,0.94118} 
\definecolor{mgrey}{rgb}{0.5,0.5,0.5} 
\definecolor{mdarkgrey}{rgb}{0.25,0.25,0.25} 
  
\DefineShortVerb[fontfamily=courier,fontseries=m]{\$} 
\DefineShortVerb[fontfamily=courier,fontseries=b]{\#} 
  
\noindent        
 $$\color{mgreen}$% center matrix$\color{black}$$\\
 $C = [0 2; 3 1];$\\
 $$\\
 $$\color{mgreen}$% width matrix$\color{black}$$\\
 $D = [1 2; 1 1];$\\
 $$\\
 $$\color{mgreen}$% interval matrix$\color{black}$$\\
 $mi = intervalMatrix(C,D);$\\ 
  
\UndefineShortVerb{\$} 
\UndefineShortVerb{\#}
\end{minipage}
\end{center}

A more detailed example for interval matrices is provided in \cref{sec:intervalMatrixExample} and in the file \textit{examples/matrixSet/example\_intervalMatrix.m} in the CORA toolbox. Furthermore, a list of methods for the class \texttt{intervalMatrix} is provided in \cref{sec:intervalMatrixOperations}.
