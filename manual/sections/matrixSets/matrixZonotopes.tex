\subsubsection{Matrix Zonotopes} \label{sec:zonotopeMatrix}

A matrix zonotope is defined analogously to zonotopes (see \cref{sec:zonotope}):
\begin{equation}\label{eq:matrixZonotope}
	\mathcal{A}_{[z]}=\Big\{ C+\sum_{i=1}^{\kappa} p_i G\^{i} \Big| p_i \in [-1,1] \Big\}, \quad G\^{i}\in\mathbb{R}^{n \times m},
\end{equation}
and is written in short form as $\mathcal{A}_{[z]}=(C, G\^1, \ldots , G\^\kappa)$,
where the first matrix is referred to as the \textit{matrix center} and the other matrices as \textit{matrix generators}.
The order of a matrix zonotope is defined as $\rho = \kappa/n$. When exchanging the matrix generators by vector generators $g\^{i}\in \mathbb{R}^{n\cdot m}$,
one obtains a zonotope (see e.g., \cite{Girard2005}).

Matrix zonotopes are implemented by the class \texttt{matZonotope}. An object of class \texttt{matZonotope} can be constructed as follows:
\begin{equation*}
	\mathcal{A}_{[z]} = \texttt{matZonotope}(C,\texttt{G}),
\end{equation*}
where \texttt{G} is a three-dimensional array ($n\times m\times \kappa$) that stores the generator matrices, see \eqref{eq:matrixZonotope}.

\newpage
Let us demonstrate the construction of a \texttt{matZonotope} object by an example:

\begin{center}
\begin{minipage}[t]{0.40\textwidth}
	\footnotesize
	% This file was automatically created from the m-file 
% "m2tex.m" written by USL. 
% The fontencoding in this file is UTF-8. 
%  
% You will need to include the following two packages in 
% your LaTeX-Main-File. 
%  
% \usepackage{color} 
% \usepackage{fancyvrb} 
%  
% It is advised to use the following option for Inputenc 
% \usepackage[utf8]{inputenc} 
%  
  
% definition of matlab colors: 
\definecolor{mblue}{rgb}{0,0,1} 
\definecolor{mgreen}{rgb}{0.13333,0.5451,0.13333} 
\definecolor{mred}{rgb}{0.62745,0.12549,0.94118} 
\definecolor{mgrey}{rgb}{0.5,0.5,0.5} 
\definecolor{mdarkgrey}{rgb}{0.25,0.25,0.25} 
  
\DefineShortVerb[fontfamily=courier,fontseries=m]{\$} 
\DefineShortVerb[fontfamily=courier,fontseries=b]{\#} 
  
\noindent         
 $$\color{mgreen}$% matrix center$\color{black}$$\\
 $C = [0 0; 0 0];$\\
 $$\\
 $$\color{mgreen}$% matrix generators$\color{black}$$\\
 $G(:,:,1) = [1 3; -1 2];$\\
 $G(:,:,2) = [2 0; 1 -1];$\\
 $$\\
 $$\color{mgreen}$% matrix zonotope$\color{black}$$\\
 $mz = matZonotope(C,G);$\\ 
  
\UndefineShortVerb{\$} 
\UndefineShortVerb{\#}
\end{minipage}
\end{center}

A more detailed example for matrix zonotopes is provided in \cref{sec:matrixZonotopeExample} and in the file \textit{examples/matrixSet/example\_matZonotope.m} in the CORA toolbox. Furthermore, a list of methods for the class \texttt{matZonotope} is provided in \cref{sec:matrixZonotopeOperations}.
