\subsubsection{Matrix Polytopes} \label{sec:polytopeMatrix}

A matrix polytope is defined analogously to a V-polytope (see \cref{sec:polytopes}):
\begin{equation}\label{eq:matrixVertices}
 \mathcal{A}_{[p]}=\bigg \{ \sum_{i=1}^{r} \alpha_i V\^i ~\Big|~ \alpha_i \in\mathbb{R}, \alpha_i \geq 0, \sum_i \alpha_i=1 \bigg \}, \quad V\^i \in \mathbb{R}^{n\times m}.
\end{equation}
The matrices $V\^i$ are also called vertices of the matrix polytope.
When substituting the matrix vertices by vector vertices $v\^{i}\in \mathbb{R}^{n}$, one obtains a V-polytope (see \cref{sec:polytopes}).

Matrix polytopes are implemented in CORA by the class \texttt{matPolytope}.
An object of class \texttt{matPolytope} can be constructed as follows:
\begin{equation*}
	 \mathcal{A}_{[p]} = \texttt{matPolytope}(\texttt{V}),
\end{equation*}
where \texttt{V} is a three-dimensional array ($n\times m\times r$) that stores the vertices $V^{(i)}$, see \eqref{eq:matrixVertices}, of the matrix polytope.

Let us demonstrate the construction of a \texttt{matPolytope} object by an example:

\begin{center}
\begin{minipage}[t]{0.40\textwidth}
	\footnotesize
	% This file was automatically created from the m-file 
% "m2tex.m" written by USL. 
% The fontencoding in this file is UTF-8. 
%  
% You will need to include the following two packages in 
% your LaTeX-Main-File. 
%  
% \usepackage{color} 
% \usepackage{fancyvrb} 
%  
% It is advised to use the following option for Inputenc 
% \usepackage[utf8]{inputenc} 
%  
  
% definition of matlab colors: 
\definecolor{mblue}{rgb}{0,0,1} 
\definecolor{mgreen}{rgb}{0.13333,0.5451,0.13333} 
\definecolor{mred}{rgb}{0.62745,0.12549,0.94118} 
\definecolor{mgrey}{rgb}{0.5,0.5,0.5} 
\definecolor{mdarkgrey}{rgb}{0.25,0.25,0.25} 
  
\DefineShortVerb[fontfamily=courier,fontseries=m]{\$} 
\DefineShortVerb[fontfamily=courier,fontseries=b]{\#} 
  
\noindent      
 $$\color{mgreen}$% vertices$\color{black}$$\\
 $V(:,:,1) = [1 2; 0 1];$\\
 $V(:,:,2) = [1 3; -1 2];$\\
 $$\\
 $$\color{mgreen}$% matrix polytope$\color{black}$$\\
 $mp = matPolytope(V);$\\ 
  
\UndefineShortVerb{\$} 
\UndefineShortVerb{\#}
\end{minipage}
\end{center}

A more detailed example for matrix polytopes is provided in \cref{sec:matrixPolytopeExample} and in the file \textit{examples/matrixSet/example\_matPolytope.m} in the CORA toolbox. Furthermore, a list of methods for the class \texttt{matPolytope} is provided in \cref{sec:matrixPolytopeOperations}.
