\subsection{Set Representations}
\logToConsole{EXAMPLES: SET REPRESENTATIONS}

We first provide examples for set-based computation using the different set representations in \cref{sec:setRepresentationsAndOperations}.

% Zonotopes -------------------------------------------------------------------

\subsubsection{Zonotopes} \label{sec:zonotopeExample}

The following MATLAB code demonstrates how to perform set-based computations on zonotopes (see \cref{sec:zonotope}):

{\small
	% This file was automatically created from the m-file 
% "m2tex.m" written by USL. 
% The fontencoding in this file is UTF-8. 
%  
% You will need to include the following two packages in 
% your LaTeX-Main-File. 
%  
% \usepackage{color} 
% \usepackage{fancyvrb} 
%  
% It is advised to use the following option for Inputenc 
% \usepackage[utf8]{inputenc} 
%  
  
% definition of matlab colors: 
\definecolor{mblue}{rgb}{0,0,1} 
\definecolor{mgreen}{rgb}{0.13333,0.5451,0.13333} 
\definecolor{mred}{rgb}{0.62745,0.12549,0.94118} 
\definecolor{mgrey}{rgb}{0.5,0.5,0.5} 
\definecolor{mdarkgrey}{rgb}{0.25,0.25,0.25} 
  
\DefineShortVerb[fontfamily=courier,fontseries=m]{\$} 
\DefineShortVerb[fontfamily=courier,fontseries=b]{\#} 
  
\noindent                   
 \hspace*{-1.6em}{\scriptsize 1}$  Z1 = zonotope([1 1 1; 1 -1 1]); $\color{mgreen}$% create zonotope Z1$\color{black}$$\\
 \hspace*{-1.6em}{\scriptsize 2}$  Z2 = zonotope([-1 1 0; 1 0 1]); $\color{mgreen}$% create zonotope Z2$\color{black}$$\\
 \hspace*{-1.6em}{\scriptsize 3}$  A = [0.5 1; 1 0.5]; $\color{mgreen}$% numerical matrix A$\color{black}$$\\
 \hspace*{-1.6em}{\scriptsize 4}$  $\\
 \hspace*{-1.6em}{\scriptsize 5}$  Z3 = Z1 + Z2; $\color{mgreen}$% Minkowski addition$\color{black}$$\\
 \hspace*{-1.6em}{\scriptsize 6}$  Z4 = A*Z3; $\color{mgreen}$% linear map$\color{black}$$\\
 \hspace*{-1.6em}{\scriptsize 7}$  $\\
 \hspace*{-1.6em}{\scriptsize 8}$  figure; hold $\color{mred}$on$\color{black}$$\\
 \hspace*{-1.6em}{\scriptsize 9}$  plot(Z1,[1 2],$\color{mred}$'b'$\color{black}$); $\color{mgreen}$% plot Z1 in blue$\color{black}$$\\
 \hspace*{-2em}{\scriptsize 10}$  plot(Z2,[1 2],$\color{mred}$'g'$\color{black}$); $\color{mgreen}$% plot Z2 in green$\color{black}$$\\
 \hspace*{-2em}{\scriptsize 11}$  plot(Z3,[1 2],$\color{mred}$'r'$\color{black}$); $\color{mgreen}$% plot Z3 in red$\color{black}$$\\
 \hspace*{-2em}{\scriptsize 12}$  plot(Z4,[1 2],$\color{mred}$'k'$\color{black}$); $\color{mgreen}$% plot Z4 in black$\color{black}$$\\
 \hspace*{-2em}{\scriptsize 13}$  $\\
 \hspace*{-2em}{\scriptsize 14}$  P = polytope(Z4) $\color{mgreen}$% convert to and display halfspace representation$\color{black}$$\\
 \hspace*{-2em}{\scriptsize 15}$  I = interval(Z4) $\color{mgreen}$% convert to and display interval$\color{black}$$\\
 \hspace*{-2em}{\scriptsize 16}$  $\\
 \hspace*{-2em}{\scriptsize 17}$  figure; hold $\color{mred}$on$\color{black}$$\\
 \hspace*{-2em}{\scriptsize 18}$  plot(Z4); $\color{mgreen}$% plot Z4$\color{black}$$\\
 \hspace*{-2em}{\scriptsize 19}$  plot(I,[1 2],$\color{mred}$'g'$\color{black}$); $\color{mgreen}$% plot intervalhull in green$\color{black}$$\\ 
  
\UndefineShortVerb{\$} 
\UndefineShortVerb{\#}}

This produces the workspace output
\begin{verbatim}
Normalized, minimal representation polytope in R^2
H: [8x2 double]
K: [8x1 double]
normal: 1
minrep: 1
xCheb: [2x1 double]
RCheb: 1.4142

[ 0.70711     0.70711]      [  6.364]
[ 0.70711    -0.70711]      [ 2.1213]
[ 0.89443    -0.44721]      [ 3.3541]
[ 0.44721    -0.89443]      [ 2.0125]
[-0.70711    -0.70711] x <= [ 2.1213]
[-0.70711     0.70711]      [0.70711]
[-0.89443     0.44721]      [0.67082]
[-0.44721     0.89443]      [ 2.0125]

Intervals: 
[-1.5,5.5]
[-2.5,4.5]
\end{verbatim}

The plots generated in lines 9-12 are shown in \cref{fig:zonotopeExample_1} and the ones generated in lines 18-19 are shown in \cref{fig:zonotopeExample_2}.

\begin{figure}[h!tb]
	\begin{minipage}{0.45\columnwidth}
		\centering
		%\footnotesize
        \includetikz{./figures/tikz/examples/set-representations/example_zonotope_1}
		\caption{Zonotopes generated in lines 9-12 of the zonotope example in \cref{sec:zonotopeExample}.}
		\label{fig:zonotopeExample_1}
	\end{minipage}
	\hspace{0.08\columnwidth}
	\begin{minipage}{0.45\columnwidth}
		\centering
		%\footnotesize
        \includetikz{./figures/tikz/examples/set-representations/example_zonotope_2}
		\caption{Sets generated in lines 18-19 of the zonotope example in \cref{sec:zonotopeExample}.}
		\label{fig:zonotopeExample_2}
	\end{minipage}
\end{figure}









% Intervals -------------------------------------------------------------------

\subsubsection{Intervals}	\label{sec:intervalExample}

The following MATLAB code demonstrates how to perform set-based computations on intervals (see \cref{sec:interval}):

{\small
% This file was automatically created from the m-file 
% "m2tex.m" written by USL. 
% The fontencoding in this file is UTF-8. 
%  
% You will need to include the following two packages in 
% your LaTeX-Main-File. 
%  
% \usepackage{color} 
% \usepackage{fancyvrb} 
%  
% It is advised to use the following option for Inputenc 
% \usepackage[utf8]{inputenc} 
%  
  
% definition of matlab colors: 
\definecolor{mblue}{rgb}{0,0,1} 
\definecolor{mgreen}{rgb}{0.13333,0.5451,0.13333} 
\definecolor{mred}{rgb}{0.62745,0.12549,0.94118} 
\definecolor{mgrey}{rgb}{0.5,0.5,0.5} 
\definecolor{mdarkgrey}{rgb}{0.25,0.25,0.25} 
  
\DefineShortVerb[fontfamily=courier,fontseries=m]{\$} 
\DefineShortVerb[fontfamily=courier,fontseries=b]{\#} 
  
\noindent              
 \hspace*{-1.6em}{\scriptsize 1}$  I1 = interval([0; -1], [3; 1]); $\color{mgreen}$% create interval I1$\color{black}$$\\
 \hspace*{-1.6em}{\scriptsize 2}$  I2 = interval([-1; -1.5], [1; -0.5]); $\color{mgreen}$% create interval I2$\color{black}$$\\
 \hspace*{-1.6em}{\scriptsize 3}$  Z1 = zonotope([1 1 1; 1 -1 1]); $\color{mgreen}$% create zonotope Z1$\color{black}$$\\
 \hspace*{-1.6em}{\scriptsize 4}$  $\\
 \hspace*{-1.6em}{\scriptsize 5}$  r = rad(I1) $\color{mgreen}$% obtain and display radius of I1$\color{black}$$\\
 \hspace*{-1.6em}{\scriptsize 6}$  is_intersecting = isIntersecting(I1, Z1) $\color{mgreen}$% Z1 intersecting I1?$\color{black}$$\\
 \hspace*{-1.6em}{\scriptsize 7}$  I3 = I1 & I2; $\color{mgreen}$% computes the intersection of I1 and I2$\color{black}$$\\
 \hspace*{-1.6em}{\scriptsize 8}$  c = center(I3) $\color{mgreen}$% returns and displays the center of I3$\color{black}$$\\
 \hspace*{-1.6em}{\scriptsize 9}$  $\\
 \hspace*{-2em}{\scriptsize 10}$  figure; hold $\color{mred}$on$\color{black}$$\\
 \hspace*{-2em}{\scriptsize 11}$  plot(I1); $\color{mgreen}$% plot I1$\color{black}$$\\
 \hspace*{-2em}{\scriptsize 12}$  plot(I2); $\color{mgreen}$% plot I2$\color{black}$$\\
 \hspace*{-2em}{\scriptsize 13}$  plot(Z1,[1 2],$\color{mred}$'g'$\color{black}$); $\color{mgreen}$% plot Z1$\color{black}$$\\
 \hspace*{-2em}{\scriptsize 14}$  plot(I3,[1 2],$\color{mred}$'FaceColor'$\color{black}$,[.6 .6 .6]); $\color{mgreen}$% plot I3 $\color{black}$$\\ 
  
\UndefineShortVerb{\$} 
\UndefineShortVerb{\#}}

This produces the workspace output
\begin{verbatim}
r =

     1.5000
     1.0000


is_intersecting =

     1


c =

    0.5000
   -0.7500
\end{verbatim}

The plot generated in lines 11-14 is shown in \cref{fig:intervalExample}.

\begin{figure}[h!tb]
  \centering
  %\footnotesize
  \includetikz{./figures/tikz/examples/set-representations/example_interval}
  \caption{Sets generated in lines 11-14 of the interval example in \cref{sec:intervalExample}.}
  \label{fig:intervalExample}
\end{figure}



% Ellipsoids ------------------------------------------------------------------

\subsubsection{Ellipsoids}	\label{sec:ellipsoidExample}

The following MATLAB code demonstrates how to perform set-based computations on ellipsoids (see \cref{sec:ellipsoids}):


{\small
	% This file was automatically created from the m-file 
% "m2tex.m" written by USL. 
% The fontencoding in this file is UTF-8. 
%  
% You will need to include the following two packages in 
% your LaTeX-Main-File. 
%  
% \usepackage{color} 
% \usepackage{fancyvrb} 
%  
% It is advised to use the following option for Inputenc 
% \usepackage[utf8]{inputenc} 
%  
  
% definition of matlab colors: 
\definecolor{mblue}{rgb}{0,0,1} 
\definecolor{mgreen}{rgb}{0.13333,0.5451,0.13333} 
\definecolor{mred}{rgb}{0.62745,0.12549,0.94118} 
\definecolor{mgrey}{rgb}{0.5,0.5,0.5} 
\definecolor{mdarkgrey}{rgb}{0.25,0.25,0.25} 
  
\DefineShortVerb[fontfamily=courier,fontseries=m]{\$}
\DefineShortVerb[fontfamily=courier,fontseries=b]{\#} 
  
\noindent                           
 \hspace*{-1.6em}{\scriptsize 1}$  E1 = ellipsoid(diag([1/2,2])) $\color{mgreen}$% create ellipsoid E1 and display it$\color{black}$$\\
 \hspace*{-1.6em}{\scriptsize 2}$  A = diag([2,0.5]);$\\
 \hspace*{-1.6em}{\scriptsize 3}$  $\\
 \hspace*{-1.6em}{\scriptsize 4}$  E2 = A*E1 + 0.5; $\color{mgreen}$% linear Map + Minkowski addition$\color{black}$$\\
 \hspace*{-1.6em}{\scriptsize 5}$  E3 = E1 + E2; $\color{mgreen}$% Minkowski addition$\color{black}$$\\
 \hspace*{-1.6em}{\scriptsize 6}$  E4 = E1 & E2; $\color{mgreen}$% intersection$\color{black}$$\\
 \hspace*{-1.6em}{\scriptsize 7}$  $\\
 \hspace*{-1.6em}{\scriptsize 8}$  disp([$\color{mred}$'E1 in E2?: '$\color{black}$,num2str(E2.contains(E1))]);$\\
 \hspace*{-1.6em}{\scriptsize 9}$  disp([$\color{mred}$'E1 in E3?: '$\color{black}$,num2str(E3.contains(E1))]);$\\
 \hspace*{-2em}{\scriptsize 10}$  $\\
 \hspace*{-2em}{\scriptsize 11}$  figure; hold $\color{mred}$on$\color{black}$$\\
 \hspace*{-2em}{\scriptsize 12}$  plot(E1,[1,2],$\color{mred}$'b'$\color{black}$); $\color{mgreen}$% plot E1 in blue$\color{black}$$\\
 \hspace*{-2em}{\scriptsize 13}$  plot(E2,[1,2],$\color{mred}$'g'$\color{black}$); $\color{mgreen}$% plot E2 in green$\color{black}$$\\
 \hspace*{-2em}{\scriptsize 14}$  plot(E3,[1,2],$\color{mred}$'r'$\color{black}$); $\color{mgreen}$% plot E3 in red$\color{black}$$\\
 \hspace*{-2em}{\scriptsize 15}$  plot(E4,[1,2],$\color{mred}$'k'$\color{black}$); $\color{mgreen}$% plot E4 in black$\color{black}$$\\
 \hspace*{-2em}{\scriptsize 16}$  $\\
 \hspace*{-2em}{\scriptsize 17}$  E5 = ellipsoid([0.8,-0.6; -0.6,0.8],[1; -4]); $\color{mgreen}$% create ellipsoid E5$\color{black}$$\\
 \hspace*{-2em}{\scriptsize 18}$  Zo_box = zonotope(E5); $\color{mgreen}$% overapproximate E5 by a parallelotope$\color{black}$$\\
 \hspace*{-2em}{\scriptsize 19}$  Zu_norm = zonotope(E5,10,$\color{mred}$'outer:norm'$\color{black}$); $\color{mgreen}$% overapproximate E5 using zonotope norm$\color{black}$$\\
 \hspace*{-2em}{\scriptsize 20}$  $\\
 \hspace*{-2em}{\scriptsize 21}$  figure; hold $\color{mred}$on$\color{black}$$\\
 \hspace*{-2em}{\scriptsize 22}$  plot(E5); $\color{mgreen}$% plot E5$\color{black}$$\\
 \hspace*{-2em}{\scriptsize 23}$  plot(Zo_box,[1,2],$\color{mred}$'r'$\color{black}$); $\color{mgreen}$% plot overapproximative zonotope Zo_box $\color{black}$$\\
 \hspace*{-2em}{\scriptsize 24}$  plot(Zu_norm,[1,2],$\color{mred}$'m'$\color{black}$);$\color{mgreen}$% plot overapproximative zonotope Zu_norm$\color{black}$$\\
  
\UndefineShortVerb{\$} 
\UndefineShortVerb{\#}}

This produces the workspace output
\begin{verbatim}
E1 =

ellipsoid:
- dimension: 2

q: 
0
0

Q: 
0.5000         0
0         2.0000

dimension: 
2

degenerate: 
0

E1 in E2?: 0
E1 in E3?: 1
\end{verbatim}
The plots generated in lines 12-15 are shown in \cref{fig:ellipsoidExample_1} and the ones generated in lines 22-24 are shown in \cref{fig:ellipsoidExample_2}.

\begin{figure}[h!tb]
	\begin{minipage}{0.45\columnwidth}
		\centering
		%\footnotesize
        \includetikz{./figures/tikz/examples/set-representations/example_ellipsoid_1}
		\caption{Ellipsoids generated in lines 12-15 of the ellipsoid example in \cref{sec:ellipsoidExample}.}
		\label{fig:ellipsoidExample_1}
	\end{minipage}
	\hspace{0.08\columnwidth}
	\begin{minipage}{0.45\columnwidth}
		\centering
		%\footnotesize
        \includetikz{./figures/tikz/examples/set-representations/example_ellipsoid_2}
		\caption{Sets generated in lines 22-24 of the ellipsoid example in \cref{sec:ellipsoidExample}.}
		\label{fig:ellipsoidExample_2}
	\end{minipage}
\end{figure}




% Polytopes -------------------------------------------------------------------

\newpage
\subsubsection{Polytopes}	\label{sec:polytopeExample}

The following MATLAB code demonstrates how to perform set-based computations on polytopes (see \cref{sec:polytopes}):

{\small
% This file was automatically created from the m-file 
% "m2tex.m" written by USL. 
% The fontencoding in this file is UTF-8. 
%  
% You will need to include the following two packages in 
% your LaTeX-Main-File. 
%  
% \usepackage{color} 
% \usepackage{fancyvrb} 
%  
% It is advised to use the following option for Inputenc 
% \usepackage[utf8]{inputenc} 
%  
  
% definition of matlab colors: 
\definecolor{mblue}{rgb}{0,0,1} 
\definecolor{mgreen}{rgb}{0.13333,0.5451,0.13333} 
\definecolor{mred}{rgb}{0.62745,0.12549,0.94118} 
\definecolor{mgrey}{rgb}{0.5,0.5,0.5} 
\definecolor{mdarkgrey}{rgb}{0.25,0.25,0.25} 
  
\DefineShortVerb[fontfamily=courier,fontseries=m]{\$} 
\DefineShortVerb[fontfamily=courier,fontseries=b]{\#} 
  
\noindent                
 \hspace*{-1.6em}{\scriptsize 1}$  Z1 = zonotope([1 1 1; 1 -1 1]); $\color{mgreen}$% create zonotope Z1$\color{black}$$\\
 \hspace*{-1.6em}{\scriptsize 2}$  Z2 = zonotope([-1 1 0; 1 0 1]); $\color{mgreen}$% create zonotope Z2$\color{black}$$\\
 \hspace*{-1.6em}{\scriptsize 3}$  $\\
 \hspace*{-1.6em}{\scriptsize 4}$  P1 = polytope(Z1); $\color{mgreen}$% convert zonotope Z1 to halfspace representation$\color{black}$$\\
 \hspace*{-1.6em}{\scriptsize 5}$  P2 = polytope(Z2); $\color{mgreen}$% convert zonotope Z2 to halfspace representation$\color{black}$$\\
 \hspace*{-1.6em}{\scriptsize 6}$  $\\
 \hspace*{-1.6em}{\scriptsize 7}$  P3 = P1 + P2 $\color{mgreen}$% perform Minkowski addition and display result$\color{black}$$\\
 \hspace*{-1.6em}{\scriptsize 8}$  P4 = P1 & P2; $\color{mgreen}$% compute intersection of P1 and P2$\color{black}$$\\
 \hspace*{-1.6em}{\scriptsize 9}$  $\\
 \hspace*{-2em}{\scriptsize 10}$  V = vertices(P4) $\color{mgreen}$% obtain and display vertices of P4$\color{black}$$\\
 \hspace*{-2em}{\scriptsize 11}$  $\\
 \hspace*{-2em}{\scriptsize 12}$  figure; hold $\color{mred}$on$\color{black}$$\\
 \hspace*{-2em}{\scriptsize 13}$  plot(P1); $\color{mgreen}$% plot P1$\color{black}$$\\
 \hspace*{-2em}{\scriptsize 14}$  plot(P2); $\color{mgreen}$% plot P2$\color{black}$$\\
 \hspace*{-2em}{\scriptsize 15}$  plot(P3,[1 2],$\color{mred}$'g'$\color{black}$); $\color{mgreen}$% plot P3$\color{black}$$\\
 \hspace*{-2em}{\scriptsize 16}$  plot(P4,[1 2],$\color{mred}$'FaceColor'$\color{black}$,[.6 .6 .6]); $\color{mgreen}$% plot P4 $\color{black}$$\\ 
  
\UndefineShortVerb{\$} 
\UndefineShortVerb{\#}}

This produces the workspace output
\begin{verbatim}
P3 =

polytope:
- dimension: 2

Vertex representation: (not computed)
Inequality constraints (A*x <= b):

A =

         0    1.0000
    0.7071    0.7071
    1.0000         0
         0   -1.0000
    0.7071   -0.7071
   -0.7071    0.7071
   -1.0000         0
   -0.7071   -0.7071


b =

    5.0000
    4.2426
    3.0000
    1.0000
    1.4142
    4.2426
    3.0000
    1.4142

Equality constraints (Ae*x = be): (none)

Bounded?                          true
Empty set?                        Unknown
Full-dimensional set?             Unknown
Minimal halfspace representation? Unknown
Minimal vertex representation?    Unknown

V: 
         0   -1.0000         0
         0    1.0000    2.0000
\end{verbatim}

The plot generated in lines 13-16 is shown in \cref{fig:polytopeExample}.

\begin{figure}[h!tb]
  \centering
  %\footnotesize
  \includetikz{./figures/tikz/examples/set-representations/example_polytope}
  \caption{Sets generated in lines 13-16 of the polytope example in \cref{sec:polytopeExample}.}
  \label{fig:polytopeExample}
\end{figure}




% Polynomial Zonotopes --------------------------------------------------------

\newpage
\subsubsection{Polynomial Zonotopes}	\label{sec:polyZonotopeExample}

The following MATLAB code demonstrates how to perform set-based computations on polynomial zonotopes (see \cref{sec:polynomialZonotopes}):

{\small
% This file was automatically created from the m-file 
% "m2tex.m" written by USL. 
% The fontencoding in this file is UTF-8. 
%  
% You will need to include the following two packages in 
% your LaTeX-Main-File. 
%  
% \usepackage{color} 
% \usepackage{fancyvrb} 
%  
% It is advised to use the following option for Inputenc 
% \usepackage[utf8]{inputenc} 
%  
  
% definition of matlab colors: 
\definecolor{mblue}{rgb}{0,0,1} 
\definecolor{mgreen}{rgb}{0.13333,0.5451,0.13333} 
\definecolor{mred}{rgb}{0.62745,0.12549,0.94118} 
\definecolor{mgrey}{rgb}{0.5,0.5,0.5} 
\definecolor{mdarkgrey}{rgb}{0.25,0.25,0.25} 
  
\DefineShortVerb[fontfamily=courier,fontseries=m]{\$} 
\DefineShortVerb[fontfamily=courier,fontseries=b]{\#} 
  
\noindent                     
 \hspace*{-1.6em}{\scriptsize 1}$  $\color{mgreen}$% construct zonotope$\color{black}$$\\
 \hspace*{-1.6em}{\scriptsize 2}$  c = [1;0];$\\
 \hspace*{-1.6em}{\scriptsize 3}$  G = [1 1;1 0];$\\
 \hspace*{-1.6em}{\scriptsize 4}$  Z = zonotope(c,G);$\\
 \hspace*{-1.6em}{\scriptsize 5}$  $\\
 \hspace*{-1.6em}{\scriptsize 6}$  $\color{mgreen}$% compute over-approximation of the quadratic map$\color{black}$$\\
 \hspace*{-1.6em}{\scriptsize 7}$  Q{1} = [0.5 0.5; 0 -0.5];$\\
 \hspace*{-1.6em}{\scriptsize 8}$  Q{2} = [-1 0; 1 1];$\\
 \hspace*{-1.6em}{\scriptsize 9}$  $\\
 \hspace*{-2em}{\scriptsize 10}$  resZono = quadMap(Z,Q);$\\
 \hspace*{-2em}{\scriptsize 11}$  $\\
 \hspace*{-2em}{\scriptsize 12}$  $\color{mgreen}$% convert zonotope to polynomial zonotope$\color{black}$$\\
 \hspace*{-2em}{\scriptsize 13}$  pZ = polyZonotope(Z);$\\
 \hspace*{-2em}{\scriptsize 14}$  $\\
 \hspace*{-2em}{\scriptsize 15}$  $\color{mgreen}$% compute the exact quadratic map$\color{black}$$\\
 \hspace*{-2em}{\scriptsize 16}$  resPolyZono = quadMap(pZ,Q);$\\
 \hspace*{-2em}{\scriptsize 17}$  $\\
 \hspace*{-2em}{\scriptsize 18}$  $\color{mgreen}$% visualization$\color{black}$$\\
 \hspace*{-2em}{\scriptsize 19}$  figure; hold $\color{mred}$on$\color{black}$;$\\
 \hspace*{-2em}{\scriptsize 20}$  plot(resZono,[1,2],$\color{mred}$'r'$\color{black}$);$\\
 \hspace*{-2em}{\scriptsize 21}$  plot(resPolyZono,[1,2],$\color{mred}$'b'$\color{black}$);$\\ 
  
\UndefineShortVerb{\$} 
\UndefineShortVerb{\#}}

The plot generated in lines 19-21 is shown in \cref{fig:polyZonotopeExample}.

\begin{figure}[h!tb]
  \centering
  %\footnotesize
  \includetikz{./figures/tikz/examples/set-representations/example_polyZonotope}
  \caption{Quadratic map calculated with zonotopes (red) and polynomial zonotopes (blue).}
  \label{fig:polyZonotopeExample}
\end{figure}


% Constrained Polynomial Zonotopes --------------------------------------------------------

\newpage
\subsubsection{Constrained Polynomial Zonotopes}	\label{sec:conPolyZonoExample}

The following MATLAB code demonstrates how to perform set-based computations on constrained polynomial zonotopes (see \cref{sec:conPolyZono}):

{\small
% This file was automatically created from the m-file 
% "m2tex.m" written by USL. 
% The fontencoding in this file is UTF-8. 
%  
% You will need to include the following two packages in 
% your LaTeX-Main-File. 
%  
% \usepackage{color} 
% \usepackage{fancyvrb} 
%  
% It is advised to use the following option for Inputenc 
% \usepackage[utf8]{inputenc} 
%  
  
% definition of matlab colors: 
\definecolor{mblue}{rgb}{0,0,1} 
\definecolor{mgreen}{rgb}{0.13333,0.5451,0.13333} 
\definecolor{mred}{rgb}{0.62745,0.12549,0.94118} 
\definecolor{mgrey}{rgb}{0.5,0.5,0.5} 
\definecolor{mdarkgrey}{rgb}{0.25,0.25,0.25} 
  
\DefineShortVerb[fontfamily=courier,fontseries=m]{\$} 
\DefineShortVerb[fontfamily=courier,fontseries=b]{\#} 
  
\noindent                                         
 \hspace*{-1.6em}{\scriptsize 1}$  $\color{mgreen}$% construct zonotope$\color{black}$$\\
 \hspace*{-1.6em}{\scriptsize 2}$  Z = zonotope([0;0],[1 1;0 1]);$\\
 \hspace*{-1.6em}{\scriptsize 3}$  $\\
 \hspace*{-1.6em}{\scriptsize 4}$  $\color{mgreen}$% construct ellipsoid$\color{black}$$\\
 \hspace*{-1.6em}{\scriptsize 5}$  E = ellipsoid([2 1;1 2],[1;1]);$\\
 \hspace*{-1.6em}{\scriptsize 6}$  $\\
 \hspace*{-1.6em}{\scriptsize 7}$  $\color{mgreen}$% convert sets to constrained polynomial zonotopes$\color{black}$$\\
 \hspace*{-1.6em}{\scriptsize 8}$  cPZ1 = conPolyZono(Z);$\\
 \hspace*{-1.6em}{\scriptsize 9}$  cPZ2 = conPolyZono(E);$\\
 \hspace*{-2em}{\scriptsize 10}$  $\\
 \hspace*{-2em}{\scriptsize 11}$  $\color{mgreen}$% compute the Minkowski sum$\color{black}$$\\
 \hspace*{-2em}{\scriptsize 12}$  resSum = cPZ1 + cPZ2;$\\
 \hspace*{-2em}{\scriptsize 13}$  $\\
 \hspace*{-2em}{\scriptsize 14}$  $\color{mgreen}$% compute the intersection$\color{black}$$\\
 \hspace*{-2em}{\scriptsize 15}$  resAnd = cPZ1 & cPZ2;$\\
 \hspace*{-2em}{\scriptsize 16}$  $\\
 \hspace*{-2em}{\scriptsize 17}$  $\color{mgreen}$% compute the union$\color{black}$$\\
 \hspace*{-2em}{\scriptsize 18}$  resOR = cPZ1 | cPZ2;$\\
 \hspace*{-2em}{\scriptsize 19}$  $\\
 \hspace*{-2em}{\scriptsize 20}$  $\color{mgreen}$% construct conPolyZono object$\color{black}$$\\
 \hspace*{-2em}{\scriptsize 21}$  c = [0;0];$\\
 \hspace*{-2em}{\scriptsize 22}$  G = [1 0 1 -1;0 1 1 1];$\\
 \hspace*{-2em}{\scriptsize 23}$  E = [1 0 1 2;0 1 1 0;0 0 1 1];$\\
 \hspace*{-2em}{\scriptsize 24}$  A = [1 -0.5 0.5];$\\
 \hspace*{-2em}{\scriptsize 25}$  b = 0.5;$\\
 \hspace*{-2em}{\scriptsize 26}$  R = [0 1 2;1 0 0;0 1 0];$\\
 \hspace*{-2em}{\scriptsize 27}$   $\\
 \hspace*{-2em}{\scriptsize 28}$  cPZ = conPolyZono(c,G,E,A,b,R);$\\
 \hspace*{-2em}{\scriptsize 29}$  $\\
 \hspace*{-2em}{\scriptsize 30}$  $\color{mgreen}$% compute quadratic map$\color{black}$$\\
 \hspace*{-2em}{\scriptsize 31}$  Q{1} = [0.5 0.5; 0 -0.5];$\\
 \hspace*{-2em}{\scriptsize 32}$  Q{2} = [-1 0; 1 1];$\\
 \hspace*{-2em}{\scriptsize 33}$  $\\
 \hspace*{-2em}{\scriptsize 34}$  res = quadMap(cPZ,Q);$\\
 \hspace*{-2em}{\scriptsize 35}$  $\\
 \hspace*{-2em}{\scriptsize 36}$  $\color{mgreen}$% visualization$\color{black}$$\\
 \hspace*{-2em}{\scriptsize 37}$  figure; hold $\color{mred}$on$\color{black}$$\\
 \hspace*{-2em}{\scriptsize 38}$  plot(cPZ,[1,2],$\color{mred}$'b'$\color{black}$);$\\
 \hspace*{-2em}{\scriptsize 39}$  $\\
 \hspace*{-2em}{\scriptsize 40}$  figure; hold $\color{mred}$on$\color{black}$$\\
 \hspace*{-2em}{\scriptsize 41}$  plot(res,[1,2],$\color{mred}$'r'$\color{black}$,$\color{mred}$'Splits'$\color{black}$,25);$\\ 
  
\UndefineShortVerb{\$} 
\UndefineShortVerb{\#}}

The plot generated in lines 37-41 is shown in \cref{fig:conPolyZonoExample1} and \cref{fig:conPolyZonoExample2}.

\begin{figure}[h!tb]
\begin{minipage}{0.45\columnwidth}
  \centering
  %\footnotesize
  \includetikz{./figures/tikz/examples/set-representations/example_conPolyZono_1}
  \caption{Constrained polynomial zonotope generated in lines 21-28 of the constrained polynomial zonotope example in \cref{sec:conPolyZonoExample}}
  \label{fig:conPolyZonoExample1}
\end{minipage}
\hspace{0.08\columnwidth}
\begin{minipage}{0.45\columnwidth}
  \centering
  %\footnotesize
  \includetikz{./figures/tikz/examples/set-representations/example_conPolyZono_2}
  \caption{Quadratic map computed in lines 31-34 of the constrained polynomial zonotope example in \cref{sec:conPolyZonoExample}.}
  \label{fig:conPolyZonoExample2}
\end{minipage}
\end{figure}





% Capsules --------------------------------------------------------------------

\subsubsection{Capsules}	\label{sec:capsuleExample}

The following MATLAB code demonstrates how to perform set-based computations on capsules (see \cref{sec:capsules}):

{\small
% This file was automatically created from the m-file 
% "m2tex.m" written by USL. 
% The fontencoding in this file is UTF-8. 
%  
% You will need to include the following two packages in 
% your LaTeX-Main-File. 
%  
% \usepackage{color} 
% \usepackage{fancyvrb} 
%  
% It is advised to use the following option for Inputenc 
% \usepackage[utf8]{inputenc} 
%  
  
% definition of matlab colors: 
\definecolor{mblue}{rgb}{0,0,1} 
\definecolor{mgreen}{rgb}{0.13333,0.5451,0.13333} 
\definecolor{mred}{rgb}{0.62745,0.12549,0.94118} 
\definecolor{mgrey}{rgb}{0.5,0.5,0.5} 
\definecolor{mdarkgrey}{rgb}{0.25,0.25,0.25} 
  
\DefineShortVerb[fontfamily=courier,fontseries=m]{\$} 
\DefineShortVerb[fontfamily=courier,fontseries=b]{\#} 
  
\noindent                            
 \hspace*{-1.6em}{\scriptsize 1}$  $\color{mgreen}$% construct a capsule$\color{black}$$\\
 \hspace*{-1.6em}{\scriptsize 2}$  c = [1;2];$\\
 \hspace*{-1.6em}{\scriptsize 3}$  g = [2;1];$\\
 \hspace*{-1.6em}{\scriptsize 4}$  r = 1;$\\
 \hspace*{-1.6em}{\scriptsize 5}$       $\\
 \hspace*{-1.6em}{\scriptsize 6}$  C1 = capsule(c,g,r)$\\
 \hspace*{-1.6em}{\scriptsize 7}$  $\\
 \hspace*{-1.6em}{\scriptsize 8}$  $\color{mgreen}$% linear map of a capsule$\color{black}$$\\
 \hspace*{-1.6em}{\scriptsize 9}$  A = [0.5 0.2; -0.1 0.4];$\\
 \hspace*{-2em}{\scriptsize 10}$  C2 = A * C1;$\\
 \hspace*{-2em}{\scriptsize 11}$  $\\
 \hspace*{-2em}{\scriptsize 12}$  $\color{mgreen}$% shift the center of a capsule$\color{black}$$\\
 \hspace*{-2em}{\scriptsize 13}$  s = [0;1];$\\
 \hspace*{-2em}{\scriptsize 14}$  C3 = C2 + s;$\\
 \hspace*{-2em}{\scriptsize 15}$  $\\
 \hspace*{-2em}{\scriptsize 16}$  $\color{mgreen}$% check capsule-in-capsule containment$\color{black}$$\\
 \hspace*{-2em}{\scriptsize 17}$  res1 = contains(C1,C2);$\\
 \hspace*{-2em}{\scriptsize 18}$  res2 = contains(C1,C3);$\\
 \hspace*{-2em}{\scriptsize 19}$  $\\
 \hspace*{-2em}{\scriptsize 20}$  disp([$\color{mred}$'C2 in C1?: '$\color{black}$,num2str(res1)]);$\\
 \hspace*{-2em}{\scriptsize 21}$  disp([$\color{mred}$'C3 in C1?: '$\color{black}$,num2str(res2)]);$\\
 \hspace*{-2em}{\scriptsize 22}$  $\\
 \hspace*{-2em}{\scriptsize 23}$  $\\
 \hspace*{-2em}{\scriptsize 24}$  $\color{mgreen}$% visualization$\color{black}$$\\
 \hspace*{-2em}{\scriptsize 25}$  figure; hold $\color{mred}$on$\color{black}$$\\
 \hspace*{-2em}{\scriptsize 26}$  plot(C1,[1,2],$\color{mred}$'r'$\color{black}$);$\\
 \hspace*{-2em}{\scriptsize 27}$  plot(C2,[1,2],$\color{mred}$'g'$\color{black}$);$\\
 \hspace*{-2em}{\scriptsize 28}$  plot(C3,[1,2],$\color{mred}$'b'$\color{black}$);$\\ 
  
\UndefineShortVerb{\$} 
\UndefineShortVerb{\#}}

This produces the workspace output
\begin{verbatim}
id: 0
dimension: 2
center: 
     1
     2

generator: 
     2
     1

radius: 
     1

C2 in C1?: 0
C3 in C1?: 1
\end{verbatim}

The plot generated in lines 25-28 is shown in \cref{fig:capsuleExample}.

\begin{figure}[h!tb]
  \centering
  %\footnotesize
  \includetikz{./figures/tikz/examples/set-representations/example_capsule}
  \caption{Capsules generated in lines 6, 10, and 14 of the capsule example in \cref{sec:capsuleExample}.}
  \label{fig:capsuleExample}
\end{figure}




% Zonotope Bundles ------------------------------------------------------------

\newpage
\subsubsection{Zonotope Bundles}	\label{sec:zonoBundleExample}

The following MATLAB code demonstrates how to perform set-based computations on zonotope bundles (see \cref{sec:zonoBundle}):

{\small
% This file was automatically created from the m-file 
% "m2tex.m" written by USL. 
% The fontencoding in this file is UTF-8. 
%  
% You will need to include the following two packages in 
% your LaTeX-Main-File. 
%  
% \usepackage{color} 
% \usepackage{fancyvrb} 
%  
% It is advised to use the following option for Inputenc 
% \usepackage[utf8]{inputenc} 
%  
  
% definition of matlab colors: 
\definecolor{mblue}{rgb}{0,0,1} 
\definecolor{mgreen}{rgb}{0.13333,0.5451,0.13333} 
\definecolor{mred}{rgb}{0.62745,0.12549,0.94118} 
\definecolor{mgrey}{rgb}{0.5,0.5,0.5} 
\definecolor{mdarkgrey}{rgb}{0.25,0.25,0.25} 
  
\DefineShortVerb[fontfamily=courier,fontseries=m]{\$} 
\DefineShortVerb[fontfamily=courier,fontseries=b]{\#} 
  
\noindent         
 \hspace*{-1.6em}{\scriptsize 1}$  Z{1} = zonotope([1 1 1; 1 -1 1]); $\color{mgreen}$% create zonotope Z1;$\color{black}$$\\
 \hspace*{-1.6em}{\scriptsize 2}$  Z{2} = zonotope([-1 1 0; 1 0 1]); $\color{mgreen}$% create zonotope Z2;$\color{black}$$\\
 \hspace*{-1.6em}{\scriptsize 3}$  Zb = zonoBundle(Z); $\color{mgreen}$% instantiate zonotope bundle from Z1, Z2$\color{black}$$\\
 \hspace*{-1.6em}{\scriptsize 4}$  vol = volume(Zb) $\color{mgreen}$% compute and display volume of zonotope bundle$\color{black}$$\\
 \hspace*{-1.6em}{\scriptsize 5}$  $\\
 \hspace*{-1.6em}{\scriptsize 6}$  figure; hold $\color{mred}$on$\color{black}$$\\
 \hspace*{-1.6em}{\scriptsize 7}$  plot(Z{1}); $\color{mgreen}$% plot Z1 $\color{black}$$\\
 \hspace*{-1.6em}{\scriptsize 8}$  plot(Z{2}); $\color{mgreen}$% plot Z2 $\color{black}$$\\
 \hspace*{-1.6em}{\scriptsize 9}$  plot(Zb,[1 2],$\color{mred}$'FaceColor'$\color{black}$,[.675 .675 .675]); $\color{mgreen}$% plot Zb in gray$\color{black}$$\\ 
  
\UndefineShortVerb{\$} 
\UndefineShortVerb{\#}}

This produces the workspace output
\begin{verbatim}
vol =

    1.0000
\end{verbatim}

The plot generated in lines 7-9 is shown in \cref{fig:zonoBundleExample}.

\begin{figure}[h!tb]
  \centering
  %\footnotesize
  \includetikz{./figures/tikz/examples/set-representations/example_zonoBundle}
  \caption{Sets generated in lines 7-9 of the zonotope bundle example in \cref{sec:zonoBundleExample}.}
  \label{fig:zonoBundleExample}
\end{figure}






% Constrained Zonotopes -------------------------------------------------------

\subsubsection{Constrained Zonotopes}	\label{sec:conZonotopeExample}

The following MATLAB code demonstrates how to perform set-based computations on constrained zonotopes (see \cref{sec:conZonotope}):

{\small
% This file was automatically created from the m-file 
% "m2tex.m" written by USL. 
% The fontencoding in this file is UTF-8. 
%  
% You will need to include the following two packages in 
% your LaTeX-Main-File. 
%  
% \usepackage{color} 
% \usepackage{fancyvrb} 
%  
% It is advised to use the following option for Inputenc 
% \usepackage[utf8]{inputenc} 
%  
  
% definition of matlab colors: 
\definecolor{mblue}{rgb}{0,0,1} 
\definecolor{mgreen}{rgb}{0.13333,0.5451,0.13333} 
\definecolor{mred}{rgb}{0.62745,0.12549,0.94118} 
\definecolor{mgrey}{rgb}{0.5,0.5,0.5} 
\definecolor{mdarkgrey}{rgb}{0.25,0.25,0.25} 
  
\DefineShortVerb[fontfamily=courier,fontseries=m]{\$} 
\DefineShortVerb[fontfamily=courier,fontseries=b]{\#} 
  
\noindent                
 \hspace*{-1.6em}{\scriptsize 1}$  Z = [0 1 0 1; 0 1 2 -1]; $\color{mgreen}$% zonotope (center + generators)$\color{black}$$\\
 \hspace*{-1.6em}{\scriptsize 2}$  A = [-2 1 -1]; $\color{mgreen}$% constraints (matrix A)$\color{black}$$\\
 \hspace*{-1.6em}{\scriptsize 2}$  b = 2; $\color{mgreen}$% constraints (vector b)$\color{black}$$\\
 \hspace*{-1.6em}{\scriptsize 3}$  $\\
 \hspace*{-1.6em}{\scriptsize 4}$  cZ = conZonotope(Z,A,b) $\color{mgreen}$% construct conZonotope object$\color{black}$$\\
 \hspace*{-1.6em}{\scriptsize 6}$  $\\
 \hspace*{-1.6em}{\scriptsize 7}$  plotZono(cZ,[1,2]) $\color{mgreen}$% visualize conZonotope object + linear zonotope$\color{black}$$\\
  
\UndefineShortVerb{\$} 
\UndefineShortVerb{\#}
}

This produces the workspace output
\begin{verbatim}
id: 0
dimension: 2
c: 
     0
     0

g_i: 
     1     0     1
     1     2    -1

A: 
    -2     1    -1

b: 
     2
\end{verbatim}

The plot generated in line 9 is shown in \cref{fig:conZonoExample_1}. \cref{fig:conZonoExample_2} displays a visualization of the constraints for the \texttt{conZonotope} object that is shown in \cref{fig:conZonoExample_1}.

\begin{figure}[h!tb]
\begin{minipage}{0.45\columnwidth}
  \centering
  %\footnotesize
  \includetikz{./figures/tikz/examples/set-representations/example_conZonotope_zonotope}
  \caption{Zonotope (red) and the corresponding constrained zonotope (blue) generated in the constrained zonotope example in \cref{sec:conZonotopeExample}}
  \label{fig:conZonoExample_1}
\end{minipage}
\hspace{0.08\columnwidth}
\begin{minipage}{0.45\columnwidth}
  \centering
  %\footnotesize
  \includetikz{./figures/tikz/examples/set-representations/example_conZonotope_constraints}
  \caption{Visualization of the constraints for the \texttt{conZonotope} object generated in the constrained zonotope example in \cref{sec:conZonotopeExample}.}
  \label{fig:conZonoExample_2}
\end{minipage}
\end{figure}

% Spectrahedral Shadows ------------------------------------------------------------

\newpage
\subsubsection{Spectrahedral Shadows}	\label{sec:spectraShadowExample}

The following MATLAB code demonstrates how to perform set-based computations on spectrahedral shadows (see \cref{sec:spectraShadow}):

{\small
	% This file was automatically created from the m-file 
% "m2tex.m" written by USL. 
% The fontencoding in this file is UTF-8. 
%  
% You will need to include the following two packages in 
% your LaTeX-Main-File. 
%  
% \usepackage{color} 
% \usepackage{fancyvrb} 
%  
% It is advised to use the following option for Inputenc 
% \usepackage[utf8]{inputenc} 
%  
  
% definition of matlab colors: 
\definecolor{mblue}{rgb}{0,0,1} 
\definecolor{mgreen}{rgb}{0.13333,0.5451,0.13333} 
\definecolor{mred}{rgb}{0.62745,0.12549,0.94118} 
\definecolor{mgrey}{rgb}{0.5,0.5,0.5} 
\definecolor{mdarkgrey}{rgb}{0.25,0.25,0.25} 
  
\DefineShortVerb[fontfamily=courier,fontseries=m]{\$} 
\DefineShortVerb[fontfamily=courier,fontseries=b]{\#} 
  
\noindent                                 
 \hspace*{-1.6em}{\scriptsize 1}$  $\color{mgreen}$% Any convex set representation implemented in CORA can be represented as a$\color{black}$$\\
 \hspace*{-1.6em}{\scriptsize 2}$  $\color{mgreen}$% spectrahedral shadow. For instance, ellipsoids, capsules, polytopes, and$\color{black}$$\\
 \hspace*{-1.6em}{\scriptsize 3}$  $\color{mgreen}$% zonotopes can be recast as spectrahedral shadows:$\color{black}$$\\
 \hspace*{-1.6em}{\scriptsize 4}$  E = ellipsoid([2 1; 1 2], [3;1]);$\\
 \hspace*{-1.6em}{\scriptsize 5}$  SpS_ellipsoid = spectraShadow(E);$\\
 \hspace*{-1.6em}{\scriptsize 6}$  $\\
 \hspace*{-1.6em}{\scriptsize 7}$  C = capsule([-1;-3], [1;-2], 1);$\\
 \hspace*{-1.6em}{\scriptsize 8}$  SpS_capsule = spectraShadow(C);$\\
 \hspace*{-1.6em}{\scriptsize 9}$  $\\
 \hspace*{-2em}{\scriptsize 10}$  P = polytope([1 0 -1 0 1; 0 1 0 -1 1]', [3; 2; 3; 2; 1]);$\\
 \hspace*{-2em}{\scriptsize 11}$  SpS_polytope = spectraShadow(P);$\\
 \hspace*{-2em}{\scriptsize 12}$  $\\
 \hspace*{-2em}{\scriptsize 13}$  Z = zonotope([3;2], [0.5 1 0; 0 0.5 2.5]);$\\
 \hspace*{-2em}{\scriptsize 14}$  SpS_zonotope = spectraShadow(Z);$\\
 \hspace*{-2em}{\scriptsize 15}$  $\\
 \hspace*{-2em}{\scriptsize 16}$  $\color{mgreen}$% perform Minkowski addition and display result$\color{black}$$\\
 \hspace*{-2em}{\scriptsize 17}$  SpS_addition = SpS_polytope + SpS_ellipsoid; $\\
 \hspace*{-2em}{\scriptsize 18}$  $\color{mgreen}$% compute convex hull of C and Z$\color{black}$$\\
 \hspace*{-2em}{\scriptsize 19}$  SpS_convHull = convHull(SpS_capsule, SpS_zonotope); $\\
 \hspace*{-2em}{\scriptsize 20}$  $\\
 \hspace*{-2em}{\scriptsize 21}$  figure; $\color{mred}$hold on;$\color{black}$$\\
 \hspace*{-2em}{\scriptsize 22}$  $\color{mgreen}$% One can now plot the convex hull of C and Z$\color{black}$$\\
 \hspace*{-2em}{\scriptsize 23}$  plot(SpS_convHull,[1 2],$\color{mred}$'FaceColor'$\color{black}$,colorblind($\color{mred}$'gray'$\color{black}$));$\\
 \hspace*{-2em}{\scriptsize 24}$  plot(C,[1 2],$\color{mred}$'Color'$\color{black}$,colorblind($\color{mred}$'r'$\color{black}$)); $\color{mgreen}$% plot C$\color{black}$$\\
 \hspace*{-2em}{\scriptsize 25}$  plot(Z,[1 2],$\color{mred}$'Color'$\color{black}$,colorblind($\color{mred}$'b'$\color{black}$)); $\color{mgreen}$% plot Z$\color{black}$$\\
  
\UndefineShortVerb{\$} 
\UndefineShortVerb{\#}}

The plot generated in lines 21-25 is shown in \cref{fig:spectraShadowExample}.

\begin{figure}[h!tb]
	\centering
	%\footnotesize
	\includetikz{./figures/tikz/examples/set-representations/example_spectraShadow}
	\caption{Sets generated in lines 21-25 of the spectrahedral shadow example in \cref{sec:spectraShadowExample}.}
	\label{fig:spectraShadowExample}
\end{figure}



% Probabilistic Zonotopes ------------------------------------------------------

\subsubsection{Probabilistic Zonotopes}		\label{sec:probZonotopeExample}

The following MATLAB code demonstrates how to compute with probabilistic zonotopes (see \cref{sec:probabilisticZonotopes}):



{\small
% This file was automatically created from the m-file 
% "m2tex.m" written by USL. 
% The fontencoding in this file is UTF-8. 
%  
% You will need to include the following two packages in 
% your LaTeX-Main-File. 
%  
% \usepackage{color} 
% \usepackage{fancyvrb} 
%  
% It is advised to use the following option for Inputenc 
% \usepackage[utf8]{inputenc} 
%  
  
% definition of matlab colors: 
\definecolor{mblue}{rgb}{0,0,1} 
\definecolor{mgreen}{rgb}{0.13333,0.5451,0.13333} 
\definecolor{mred}{rgb}{0.62745,0.12549,0.94118} 
\definecolor{mgrey}{rgb}{0.5,0.5,0.5} 
\definecolor{mdarkgrey}{rgb}{0.25,0.25,0.25} 
  
\DefineShortVerb[fontfamily=courier,fontseries=m]{\$} 
\DefineShortVerb[fontfamily=courier,fontseries=b]{\#} 
  
\noindent                        
 \hspace*{-1.6em}{\scriptsize 1}$  Z1=[10; 0]; $\color{mgreen}$% uncertain center$\color{black}$$\\
 \hspace*{-1.6em}{\scriptsize 2}$  Z2=[0.6 1.2  ; 0.6 -1.2]; $\color{mgreen}$% generators with normally distributed factors$\color{black}$$\\
 \hspace*{-1.6em}{\scriptsize 3}$  pZ=probZonotope(Z1,Z2); $\color{mgreen}$% probabilistic zonotope$\color{black}$$\\
 \hspace*{-1.6em}{\scriptsize 4}$  $\\
 \hspace*{-1.6em}{\scriptsize 5}$  M=[-1 -1;1 -1]*0.2; $\color{mgreen}$% mapping matrix$\color{black}$$\\
 \hspace*{-1.6em}{\scriptsize 6}$  pZencl = enclose(pZ,M); $\color{mgreen}$% probabilistic enclosure of pZ and M*pZ$\color{black}$$\\
 \hspace*{-1.6em}{\scriptsize 7}$  $\\
 \hspace*{-1.6em}{\scriptsize 8}$  figure $\color{mgreen}$% initialize figure$\color{black}$$\\
 \hspace*{-1.6em}{\scriptsize 9}$  hold $\color{mred}$on$\color{black}$$\\
 \hspace*{-2em}{\scriptsize 10}$  camlight $\color{mred}$headlight$\color{black}$$\\
 \hspace*{-2em}{\scriptsize 11}$      $\\
 \hspace*{-2em}{\scriptsize 12}$  plot(pZ,[1 2],$\color{mred}$'FaceColor'$\color{black}$,[0.2 0.2 0.2],...$\\
 \hspace*{-2em}{\scriptsize 13}$      $\color{mred}$'EdgeColor'$\color{black}$,$\color{mred}$'none'$\color{black}$, $\color{mred}$'FaceLighting'$\color{black}$,$\color{mred}$'phong'$\color{black}$); $\color{mgreen}$% plot pZ $\color{black}$$\\
 \hspace*{-2em}{\scriptsize 14}$      $\\
 \hspace*{-2em}{\scriptsize 15}$  plot(expm(M)*pZ,[1,2],$\color{mred}$'FaceColor'$\color{black}$,[0.5 0.5 0.5],...$\\
 \hspace*{-2em}{\scriptsize 16}$      $\color{mred}$'EdgeColor'$\color{black}$,$\color{mred}$'none'$\color{black}$, $\color{mred}$'FaceLighting'$\color{black}$,$\color{mred}$'phong'$\color{black}$); $\color{mgreen}$% plot expm(M)*pZ$\color{black}$$\\
 \hspace*{-2em}{\scriptsize 17}$  $\\
 \hspace*{-2em}{\scriptsize 18}$  plot(pZencl,[1,2],$\color{mred}$'k'$\color{black}$,$\color{mred}$'FaceAlpha'$\color{black}$,0) $\color{mgreen}$% plot enclosure$\color{black}$$\\
 \hspace*{-2em}{\scriptsize 19}$  $\\
 \hspace*{-2em}{\scriptsize 20}$  campos([-3,-51,1]); $\color{mgreen}$% set camera position$\color{black}$$\\
 \hspace*{-2em}{\scriptsize 21}$  drawnow; $\color{mgreen}$% draw 3D view$\color{black}$$\\
  
\UndefineShortVerb{\$} 
\UndefineShortVerb{\#}}

The plot generated in lines 8-21 is shown in \cref{fig:probZonotopeExample}.

\begin{figure}[h!tb]
  \centering
  %\footnotesize
  \includetikz{./figures/tikz/examples/set-representations/example_probZonotope}
  \caption{Sets generated in lines 10-15 of the probabilistic zonotope example in \cref{sec:probZonotopeExample}.}
  \label{fig:probZonotopeExample}
\end{figure}



% Level Sets ------------------------------------------------------------------

\newpage
\subsubsection{Level Sets}	\label{sec:levelSetExample}

The following MATLAB code demonstrates how to compute with level sets (see \cref{sec:levelSet}):

{\small
	% This file was automatically created from the m-file 
% "m2tex.m" written by USL. 
% The fontencoding in this file is UTF-8. 
%  
% You will need to include the following two packages in 
% your LaTeX-Main-File. 
%  
% \usepackage{color} 
% \usepackage{fancyvrb} 
%  
% It is advised to use the following option for Inputenc 
% \usepackage[utf8]{inputenc} 
%  
  
% definition of matlab colors: 
\definecolor{mblue}{rgb}{0,0,1} 
\definecolor{mgreen}{rgb}{0.13333,0.5451,0.13333} 
\definecolor{mred}{rgb}{0.62745,0.12549,0.94118} 
\definecolor{mgrey}{rgb}{0.5,0.5,0.5} 
\definecolor{mdarkgrey}{rgb}{0.25,0.25,0.25} 
  
\DefineShortVerb[fontfamily=courier,fontseries=m]{\$} 
\DefineShortVerb[fontfamily=courier,fontseries=b]{\#} 
  
\noindent                 
 \hspace*{-1.6em}{\scriptsize 1}$  $\color{mgreen}$% construct level sets$\color{black}$$\\
 \hspace*{-1.6em}{\scriptsize 2}$  syms $\color{mred}$x y$\color{black}$$\\
 \hspace*{-1.6em}{\scriptsize 3}$  eq = sin(x) + y;$\\
 \hspace*{-1.6em}{\scriptsize 4}$  $\\
 \hspace*{-1.6em}{\scriptsize 5}$  ls1 = levelSet(eq,[x;y],$\color{mred}$'=='$\color{black}$);$\\
 \hspace*{-1.6em}{\scriptsize 6}$  ls2 = levelSet(eq,[x;y],$\color{mred}$'<='$\color{black}$);$\\
 \hspace*{-1.6em}{\scriptsize 7}$  $\\
 \hspace*{-1.6em}{\scriptsize 8}$  $\color{mgreen}$% visualize the level sets$\color{black}$$\\
 \hspace*{-1.6em}{\scriptsize 9}$  subplot(1,2,1)$\\
 \hspace*{-2em}{\scriptsize 10}$  xlim([-1.5,1.5]);$\\
 \hspace*{-2em}{\scriptsize 11}$  ylim([-1,1]);$\\
 \hspace*{-2em}{\scriptsize 12}$  plot(ls1,[1,2],$\color{mred}$'b'$\color{black}$);$\\
 \hspace*{-2em}{\scriptsize 13}$  $\\
 \hspace*{-2em}{\scriptsize 14}$  subplot(1,2,2)$\\
 \hspace*{-2em}{\scriptsize 15}$  xlim([-1.5,1.5]);$\\
 \hspace*{-2em}{\scriptsize 16}$  ylim([-1,1]);$\\
 \hspace*{-2em}{\scriptsize 17}$  plot(ls2,[1,2],$\color{mred}$'Color'$\color{black}$,[0.9451 0.5529 0.5686]);$\\
  
\UndefineShortVerb{\$} 
\UndefineShortVerb{\#}}

The generated plot is shown in \cref{fig:levelSet}.

\begin{figure}[h!tb]
	\centering
    \includetikz{./figures/tikz/examples/set-representations/example_manual_example_levelSet}
		\caption{Level sets from the example in \cref{sec:levelSetExample} defined as in \eqref{eq:defLevelSet1} (left) and as in \eqref{eq:defLevelSet3} (rigth).}
		\label{fig:levelSet}
\end{figure}




% Taylor Models ---------------------------------------------------------------

\newpage
\subsubsection{Taylor Models}	\label{sec:taylorModelExample}

The following MATLAB code demonstrates how to compute with Taylor models (see \cref{sec:taylorModels}):

{\small
% This file was automatically created from the m-file 
% "m2tex.m" written by USL. 
% The fontencoding in this file is UTF-8. 
%  
% You will need to include the following two packages in 
% your LaTeX-Main-File. 
%  
% \usepackage{color} 
% \usepackage{fancyvrb} 
%  
% It is advised to use the following option for Inputenc 
% \usepackage[utf8]{inputenc} 
%  
  
% definition of matlab colors: 
\definecolor{mblue}{rgb}{0,0,1} 
\definecolor{mgreen}{rgb}{0.13333,0.5451,0.13333} 
\definecolor{mred}{rgb}{0.62745,0.12549,0.94118} 
\definecolor{mgrey}{rgb}{0.5,0.5,0.5} 
\definecolor{mdarkgrey}{rgb}{0.25,0.25,0.25} 
  
\DefineShortVerb[fontfamily=courier,fontseries=m]{\$} 
\DefineShortVerb[fontfamily=courier,fontseries=b]{\#} 
  
\noindent                             
 \hspace*{-1.6em}{\scriptsize 1}$  a1 = interval(-1, 2); $\color{mgreen}$% generate a scalar interval [-1,2]$\color{black}$$\\
 \hspace*{-1.6em}{\scriptsize 2}$  a2 = interval(2, 3); $\color{mgreen}$% generate a scalar interval [2,3]$\color{black}$$\\
 \hspace*{-1.6em}{\scriptsize 3}$  a3 = interval(-6, -4); $\color{mgreen}$% generate a scalar interval [-6,4]$\color{black}$$\\
 \hspace*{-1.6em}{\scriptsize 4}$  a4 = interval(4, 6); $\color{mgreen}$% generate a scalar interval [4,6]$\color{black}$$\\
 \hspace*{-1.6em}{\scriptsize 5}$  $\\
 \hspace*{-1.6em}{\scriptsize 6}$  b1 = taylm(a1, 6); $\color{mgreen}$% Taylor model with maximum order of 6 and name a1$\color{black}$$\\
 \hspace*{-1.6em}{\scriptsize 7}$  b2 = taylm(a2, 6); $\color{mgreen}$% Taylor model with maximum order of 6 and name a2$\color{black}$$\\
 \hspace*{-1.6em}{\scriptsize 8}$  b3 = taylm(a3, 6); $\color{mgreen}$% Taylor model with maximum order of 6 and name a3$\color{black}$$\\
 \hspace*{-1.6em}{\scriptsize 9}$  b4 = taylm(a4, 6); $\color{mgreen}$% Taylor model with maximum order of 6 and name a4$\color{black}$$\\
 \hspace*{-2em}{\scriptsize 10}$  $\\
 \hspace*{-2em}{\scriptsize 11}$  B1 = [b1; b2] $\color{mgreen}$% generate a row of Taylor models$\color{black}$$\\
 \hspace*{-2em}{\scriptsize 12}$  B2 = [b3; b4] $\color{mgreen}$% generate a row of Taylor models$\color{black}$$\\
 \hspace*{-2em}{\scriptsize 13}$  $\\
 \hspace*{-2em}{\scriptsize 14}$  B1 + B2 $\color{black}$$\color{mgreen}$% addition$\color{black}$$\\
 \hspace*{-2em}{\scriptsize 15}$  B1' * B2 $\color{black}$$\color{mgreen}$% matrix multiplication$\color{black}$$\\
 \hspace*{-2em}{\scriptsize 16}$  B1 .* B2 $\color{black}$$\color{mgreen}$% pointwise multiplication$\color{black}$$\\
 \hspace*{-2em}{\scriptsize 17}$  B1 / 2 $\color{black}$$\color{mgreen}$% division by scalar$\color{black}$$\\
 \hspace*{-2em}{\scriptsize 18}$  B1 ./ B2 $\color{black}$$\color{mgreen}$% pointwise division$\color{black}$$\\
 \hspace*{-2em}{\scriptsize 19}$  B1.^3 $\color{mgreen}$% power function$\color{black}$$\\
 \hspace*{-2em}{\scriptsize 20}$  sin(B1) $\color{mgreen}$% sine function$\color{black}$$\\
 \hspace*{-2em}{\scriptsize 21}$  sin(B1(1,1)) + B1(2,1).^2 - B1' * B2 $\color{black}$ $\color{mgreen}$% combination of functions$\color{black}$$\\
  
\UndefineShortVerb{\$} 
\UndefineShortVerb{\#}}

The resulting workspace output is:
\footnotesize
\begin{Verbatim}
B1 = 
 	 0.5 + 1.5*a1 + [0.00000,0.00000]
 	 2.5 + 0.5*a2 + [0.00000,0.00000]

B2 = 
 	 -5.0 + a3 + [0.00000,0.00000]
 	 5.0 + a4 + [0.00000,0.00000]

B1 + B2 = 
 	 -4.5 + 1.5*a1 + a3 + [0.00000,0.00000]
 	 7.5 + 0.5*a2 + a4 + [0.00000,0.00000]

B1' * B2 = 
 	 10.0 - 7.5*a1 + 2.5*a2 + 0.5*a3 + 2.5*a4 + 1.5*a1*a3 + 0.5*a2*a4 + [0.00000,0.00000]

B1 .* B2 = 
 	 -2.5 - 7.5*a1 + 0.5*a3 + 1.5*a1*a3 + [0.00000,0.00000]
 	 12.5 + 2.5*a2 + 2.5*a4 + 0.5*a2*a4 + [0.00000,0.00000]

B1 / 2 = 
 	 0.25 + 0.75*a1 + [0.00000,0.00000]
 	 1.25 + 0.25*a2 + [0.00000,0.00000]

B1 ./ B2 = 
 	 -0.1 - 0.3*a1 - 0.02*a3 - 0.06*a1*a3 - 0.004*a3^2 - 0.012*a1*a3^2 
	  - 0.0008*a3^3 - 0.0024*a1*a3^3 - 0.00016*a3^4 - 0.00048*a1*a3^4 
	  - 0.000032*a3^5 - 0.000096*a1*a3^5 - 6.4e-6*a3^6 + [-0.00005,0.00005]
	  
 	 0.5 + 0.1*a2 - 0.1*a4 - 0.02*a2*a4 + 0.02*a4^2 + 0.004*a2*a4^2 
	  - 0.004*a4^3 - 0.0008*a2*a4^3 + 0.0008*a4^4 + 0.00016*a2*a4^4 
	  - 0.00016*a4^5 - 0.000032*a2*a4^5 + 0.000032*a4^6 + [-0.00005,0.00005]

B1.^3 = 
 	 0.125 + 1.125*a1 + 3.375*a1^2 + 3.375*a1^3 + [0.00000,0.00000]
 	 15.625 + 9.375*a2 + 1.875*a2^2 + 0.125*a2^3 + [0.00000,0.00000]

sin(B1) = 
 	 0.47943 + 1.3164*a1 - 0.53935*a1^2 - 0.49364*a1^3 + 0.10113*a1^4 
	  + 0.055535*a1^5 - 0.0075847*a1^6 + [-0.00339,0.00339]
	  
 	 0.59847 - 0.40057*a2 - 0.074809*a2^2 + 0.01669*a2^3 + 0.0015585*a2^4 
	  - 0.00020863*a2^5 - 0.000012988*a2^6 + [-0.00000,0.00000]

sin(B1(1,1)) + B1(2,1).^2 - B1' * B2 = 
 	 -3.2706 + 8.8164*a1 - 0.5*a3 - 2.5*a4 - 0.53935*a1^2 + 0.25*a2^2 
	  - 1.5*a1*a3 - 0.5*a2*a4 - 0.49364*a1^3 + 0.10113*a1^4 
	  + 0.055535*a1^5 - 0.0075847*a1^6 + [-0.00339,0.00339]
\end{Verbatim}
\normalsize




% Affine ----------------------------------------------------------------------

\subsubsection{Affine}	\label{sec:affineExample}

The following MATLAB code demonstrates how to use affine arithmetics in CORA (see \cref{sec:affine}):

{\small
% This file was automatically created from the m-file 
% "m2tex.m" written by USL. 
% The fontencoding in this file is UTF-8. 
%  
% You will need to include the following two packages in 
% your LaTeX-Main-File. 
%  
% \usepackage{color} 
% \usepackage{fancyvrb} 
%  
% It is advised to use the following option for Inputenc 
% \usepackage[utf8]{inputenc} 
%  
  
% definition of matlab colors: 
\definecolor{mblue}{rgb}{0,0,1} 
\definecolor{mgreen}{rgb}{0.13333,0.5451,0.13333} 
\definecolor{mred}{rgb}{0.62745,0.12549,0.94118} 
\definecolor{mgrey}{rgb}{0.5,0.5,0.5} 
\definecolor{mdarkgrey}{rgb}{0.25,0.25,0.25} 
  
\DefineShortVerb[fontfamily=courier,fontseries=m]{\$} 
\DefineShortVerb[fontfamily=courier,fontseries=b]{\#} 
  
\noindent              
 \hspace*{-1.6em}{\scriptsize 1}$  $\color{mgreen}$% create affine object$\color{black}$$\\
 \hspace*{-1.6em}{\scriptsize 2}$  I = interval(-1,1);$\\
 \hspace*{-1.6em}{\scriptsize 3}$  aff = affine(I);$\\
 \hspace*{-1.6em}{\scriptsize 4}$  $\\
 \hspace*{-1.6em}{\scriptsize 5}$  $\color{mgreen}$% create taylor model object (for comparison)$\color{black}$$\\
 \hspace*{-1.6em}{\scriptsize 6}$  maxOrder = 1;$\\
 \hspace*{-1.6em}{\scriptsize 7}$  tay = taylm(int,maxOrder,$\color{mred}$'x'$\color{black}$);$\\
 \hspace*{-1.6em}{\scriptsize 8}$  $\\
 \hspace*{-1.6em}{\scriptsize 9}$  $\color{mgreen}$% define function$\color{black}$$\\
 \hspace*{-2em}{\scriptsize 10}$  f = @(x) sin(x) * (x+1);$\\
 \hspace*{-2em}{\scriptsize 11}$  $\\
 \hspace*{-2em}{\scriptsize 12}$  $\color{mgreen}$% evaluate the function with affine arithmetic and taylor model$\color{black}$$\\
 \hspace*{-2em}{\scriptsize 13}$  intAff = interval(f(aff))$\\
 \hspace*{-2em}{\scriptsize 14}$  intTay = interval(f(tay))$\\ 
  
\UndefineShortVerb{\$} 
\UndefineShortVerb{\#}}

The resulting workspace output is:
%\footnotesize
\begin{Verbatim}
intAff = 
 [-1.84147,2.84147]
intTay = 
 [-1.84147,2.84147]
\end{Verbatim}







% Zoo -------------------------------------------------------------------------

\subsubsection{Zoo}		\label{sec:zooExample}

The following MATLAB code demonstrates how to use the class \texttt{zoo} in CORA (see \cref{sec:zoo}):

{\small
% This file was automatically created from the m-file 
% "m2tex.m" written by USL. 
% The fontencoding in this file is UTF-8. 
%  
% You will need to include the following two packages in 
% your LaTeX-Main-File. 
%  
% \usepackage{color} 
% \usepackage{fancyvrb} 
%  
% It is advised to use the following option for Inputenc 
% \usepackage[utf8]{inputenc} 
%  
  
% definition of matlab colors: 
\definecolor{mblue}{rgb}{0,0,1} 
\definecolor{mgreen}{rgb}{0.13333,0.5451,0.13333} 
\definecolor{mred}{rgb}{0.62745,0.12549,0.94118} 
\definecolor{mgrey}{rgb}{0.5,0.5,0.5} 
\definecolor{mdarkgrey}{rgb}{0.25,0.25,0.25} 
  
\DefineShortVerb[fontfamily=courier,fontseries=m]{\$} 
\DefineShortVerb[fontfamily=courier,fontseries=b]{\#} 
  
\noindent                
 \hspace*{-1.6em}{\scriptsize 1}$  $\color{mgreen}$% create zoo object$\color{black}$$\\
 \hspace*{-1.6em}{\scriptsize 2}$  I = interval(-1,1);$\\
 \hspace*{-1.6em}{\scriptsize 3}$  methods = {$\color{mred}$'interval'$\color{black}$,$\color{mred}$'taylm(int)'$\color{black}$};$\\
 \hspace*{-1.6em}{\scriptsize 4}$  maxOrder = 3;$\\
 \hspace*{-1.6em}{\scriptsize 5}$  z = zoo(I,methods,maxOrder);$\\
 \hspace*{-1.6em}{\scriptsize 6}$  $\\
 \hspace*{-1.6em}{\scriptsize 7}$  $\color{mgreen}$% create taylor model object (for comparison)$\color{black}$$\\
 \hspace*{-1.6em}{\scriptsize 8}$  maxOrder = 10;$\\
 \hspace*{-1.6em}{\scriptsize 9}$  tay = taylm(I,maxOrder,$\color{mred}$'x'$\color{black}$);$\\
 \hspace*{-2em}{\scriptsize 10}$  $\\
 \hspace*{-2em}{\scriptsize 11}$  $\color{mgreen}$% define function$\color{black}$$\\
 \hspace*{-2em}{\scriptsize 12}$  f = @(x) sin(x) * (x+1);$\\
 \hspace*{-2em}{\scriptsize 13}$  $\\
 \hspace*{-2em}{\scriptsize 14}$  $\color{mgreen}$% evaluate the function with zoo-object and taylor model$\color{black}$$\\
 \hspace*{-2em}{\scriptsize 15}$  intZoo = interval(f(z))$\\
 \hspace*{-2em}{\scriptsize 16}$  intTay = interval(f(tay))$\\ 
  
\UndefineShortVerb{\$} 
\UndefineShortVerb{\#}}

The resulting workspace output is:
%\footnotesize
\begin{Verbatim}
intZoo = 
 [-1.34206,1.68294]
intTay = 
 [-1.34207,2.18354]
\end{Verbatim}





