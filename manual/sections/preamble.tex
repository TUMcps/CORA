% logging
\newcommand{\logToConsole}[1]{
    \typeout{}
    \typeout{------------------------------------------------------------------------------}
    \typeout{#1}
    \typeout{------------------------------------------------------------------------------}
    \typeout{}
}

\logToConsole{PREAMBLE}

\usepackage[
    pdfauthor={\CORAauthors},
    pdftitle={CORA Manual - \CORAVERSION},
    pdfsubject={CORA - A Tool for Continuous Reachability Analysis},
    pdfkeywords={\CORAkeywords}
]{hyperref}

%% More layout: Get rid of indenting throughout entire document
\setlength{\parindent}{0in}

%Changes from Matthias Althoff:
\usepackage{color}
\usepackage{fancyvrb}
\usepackage{eurosym}
%\usepackage[a4paper,left=2.5cm,right=2.5cm,nohead]{geometry}
\usepackage[a4paper,margin=2.5cm]{geometry}
\usepackage{fancyhdr} %for header
%\usepackage{paralist}
\usepackage{titlesec}
\usepackage{enumitem}
\usepackage{tabularx}
\usepackage{lastpage}
\usepackage{parskip}
\usepackage{graphicx}
\usepackage{subfigure}
\usepackage{amsmath,amssymb,amsfonts}
\usepackage{mathrsfs}
\usepackage{cite}
\usepackage{pifont} %for circled numbers
\usepackage{caption}
\usepackage{framed}
\usepackage{footnote}
\usepackage[colorinlistoftodos,prependcaption,textsize=tiny]{todonotes}
\usepackage{longtable}

%\usepackage{breakurl}
% added by Dmitry Grebenyuk
\usepackage{booktabs}
\usepackage{multirow}
\usepackage{listings}
\usepackage{verbatim}
\usepackage{tcolorbox}
\usepackage{rotating}
\tcbuselibrary{breakable,skins}

% table colums with fixed width
\newcolumntype{C}[1]{>{\centering\arraybackslash}m{#1}}
\newcolumntype{L}[1]{>{\arraybackslash}m{#1}}

% enable use of footnotes in tables
\makesavenoteenv{tabular}
\makesavenoteenv{table}

% argmin command
\DeclareMathOperator*{\argmin}{arg\,min}

% subsubsubsection command ---------

\titleclass{\subsubsubsection}{straight}[\subsection]

\newcounter{subsubsubsection}[subsubsection]
\renewcommand\thesubsubsubsection{\thesubsubsection.\arabic{subsubsubsection}}

\titleformat{\subsubsubsection}
{\normalfont\normalsize\bfseries}{\thesubsubsubsection}{1em}{}
\titlespacing*{\subsubsubsection}
{0pt}{3.25ex plus 1ex minus .2ex}{1.5ex plus .2ex}

\makeatletter
\renewcommand\paragraph{\@startsection{paragraph}{5}{\z@}%
{1\baselineskip plus 0.25\baselineskip minus 0.25\baselineskip}%
{1\baselineskip plus 0.25\baselineskip minus 0.25\baselineskip}{\normalfont  \normalsize\bfseries}}%
\renewcommand\subparagraph{\@startsection{subparagraph}{6}{\parindent}%
{3.25ex \@plus1ex \@minus .2ex}%
{-1em}%
{\normalfont\normalsize\bfseries}}
\def\toclevel@subsubsubsection{4}
\def\toclevel@paragraph{5}
\def\toclevel@paragraph{6}
\def\l@subsubsubsection{\@dottedtocline{4}{7em}{4em}}
\def\l@paragraph{\@dottedtocline{5}{10em}{5em}}
\def\l@subparagraph{\@dottedtocline{6}{14em}{6em}}
\makeatother

\setcounter{secnumdepth}{4}
\setcounter{tocdepth}{4}
% -----------------------------------



\newcommand{\mathpzc}[1]{\underline{#1}}
%Indexing elements----------------------
\renewcommand{\^}[1]{^{(#1)}}
\renewcommand{\th}[1]{#1^\text{th}}
%---------------------------------------
%-Dmitry
\newcommand{\ra}[1]{\renewcommand{\arraystretch}{#1}}

\DeclareMathOperator{\arccosh}{arccosh}
\DeclareMathOperator{\arcsinh}{arcsinh}
\DeclareMathOperator{\arctanh}{arctanh}
\newcommand{\R}{\mathbb{R}}
\newcommand{\Rn}{\mathbb{R}^n}
\newcommand{\operator}[1]{{\normalfont \texttt{#1}}}
\newcolumntype{C}[1]{>{\centering\arraybackslash}m{#1}}

\newenvironment{wide_itemize}{
    \begin{itemize}
        \setlength{\itemsep}{10pt}
        \setlength{\parskip}{0pt}
        \setlength{\parsep}{0pt}
        }{
    \end{itemize}}

\newenvironment{packed_itemize}{
    \begin{itemize}
        \renewcommand{\labelitemi}{$\bullet$}
        \setlength{\itemsep}{2pt}
        \setlength{\parskip}{0pt}
        \setlength{\parsep}{0pt}
        }{
    \end{itemize}}

\newenvironment{first_itemize}{
    \setlist{nolistsep}
    \vspace{-0.2cm}
    \begin{itemize}
        \setlength{\itemsep}{2pt}
        \setlength{\parskip}{0pt}
        \setlength{\parsep}{0pt}
        }{
    \end{itemize}}

\newtcolorbox{nestedlist}{blanker,
    unbreakable,
    left=3mm,right=3mm,top=2mm,bottom=2mm,
    before skip=5pt,after skip=5pt,
    every box/.style={borderline west={1pt}{2pt}{black}},
}

\newcommand{\sep}[1]{{\large\textbf{#1}} \vspace{0.3cm}} %seperator

\newcommand{\brake}{\mbox{} \vspace{0cm} \mbox{}} %seperator

%\renewcommand{\sectionmark}[1]{\markright{\thesection\ #1}}



\pagestyle{fancy} %set differnt page style for header
\setlength{\headheight}{14pt}
\fancyhead[L]{\leftmark}
\fancyhead[C]{}
% \fancyhead[R]{}
\fancyhead[R]{\smash{\includegraphics[height=\headheight]{./logo/CoraLogo_icon}}}
\fancyfoot[L]{}
\fancyfoot[C]{\thepage}
\fancyfoot[R]{}


\fancypagestyle{firststyle}
{
    \addtolength{\headheight}{-1.05cm}
    \addtolength{\headsep}{1.05cm}
}

% tikz
\usepackage{tikzscale}
\usepackage{pgfplots}
\usepackage{mathtools}
\pgfplotsset{compat=1.18}
\usepgfplotslibrary{external}
\tikzexternalize[prefix=./figures/externalize/]
\tcbsetforeverylayer{shield externalize}% required for tcolorbox/nestedlist
\newcommand{\includetikz}[1]{%
    \tikzsetnextfilename{#1}%
    \input{#1.tikz}%
}
% explicitly set line width to avoid different widths in different pdf viewers
\tikzset{every picture/.style={line width=0.5pt}}
% groupplots
\usepgfplotslibrary{groupplots}
\usetikzlibrary{pgfplots.groupplots}
\usetikzlibrary{matrix}
\usetikzlibrary{backgrounds}
% uml
\usetikzlibrary{arrows,shapes,positioning}
\usetikzlibrary{ext.paths.ortho}
\usetikzlibrary{calc}
\usetikzlibrary {arrows.meta}
\usetikzlibrary{patterns}
\usetikzlibrary{plotmarks,bending,positioning,fit} % rl

% checkmark and crossmark
\usepackage{pifont}
\newcommand{\cmark}{\text{\ding{51}}}%
\newcommand{\xmark}{\text{\ding{55}}}%


\newcommand{\textttsmall}[1]{\textcolor{blue}{\small{\texttt{#1}}}}

% neural networks

% neural network
\newcommand{\NN}[0]{\ensuremath{\Phi}}
\newcommand{\numLayers}[0]{\ensuremath{\kappa}}
\newcommand{\numNeurons}[0]{\ensuremath{n}}

% layers
\newcommand{\nnLayerName}[2]{\ensuremath{L_{#1}^\text{#2}}}
\newcommand{\nnLayer}[3]{\ensuremath{\nnLayerName{#1}{#2}\left(#3\right)}}
\newcommand{\nnActFun}{\ensuremath{\phi}}
\newcommand{\actfun}{\nnActFun} % remove?



% point propagation
\newcommand{\nnInput}[0]{\ensuremath{x}}
\newcommand{\nnHidden}[0]{\ensuremath{h}}
\newcommand{\nnOutput}[0]{\ensuremath{y}}
\newcommand{\nnGrad}{\ensuremath{g}}

% set propagation
\newcommand{\nnInputSet}[0]{\mathcal{X}}
\newcommand{\nnHiddenSet}[0]{\mathcal{H}}
\newcommand{\nnHiddenSetExact}[0]{\nnHiddenSet^*}
\newcommand{\nnOutputSet}[0]{\mathcal{Y}}
\newcommand{\nnOutputSetExact}[0]{\nnOutputSet^*}
\newcommand{\nnGradSet}{\mathcal{G}}

% set-based training
\newcommand{\nnTarget}{\ensuremath{t}}
\NewDocumentCommand{\nablaOperator}{e{_^}}{%
    \mathop{}\!% \mathop for good spacing before \nabla
    \nabla
    \IfValueT{#1}{_{\!#1}}% tuck in the subscript
    \IfValueT{#2}{^{#2}}% possible superscript
}


% other
\newcommand{\nnPertRadius}{\epsilon}

% specification
\newcommand{\nnUnsafeSet}[0]{\mathcal{S}}


% cleverref ---

\usepackage{cleveref}

\crefname{section}{Sec.}{Sec.}
\crefname{subsection}{Sec.}{Sec.}
\crefname{subsubsection}{Sec.}{Sec.}
\crefname{subsubsubsection}{Sec.}{Sec.}
\crefname{figure}{Fig.}{Fig.}
\crefname{algorithm}{Alg.}{Alg.}
\crefname{table}{Tab.}{Tab.}
\crefname{example}{Example}{Example}
\crefname{definition}{Def.}{Def.}
\crefname{proposition}{Prop.}{Prop.}
\crefname{theorem}{Thm.}{Thm.}
\crefname{lemma}{Lemma}{Lemmas}
\crefname{appendix}{Appendix}{Appendices}

\Crefname{section}{Sec.}{Sec.}
\Crefname{subsection}{Sec.}{Sec.}
\Crefname{subsubsection}{Sec.}{Sec.}
\Crefname{subsubsubsection}{Sec.}{Sec.}
\Crefname{figure}{Fig.}{Fig.}
\Crefname{algorithm}{Alg.}{Alg.}
\Crefname{table}{Tab.}{Tab.}
\crefname{example}{Example}{Example}
\Crefname{definition}{Def.}{Def.}
\Crefname{proposition}{Prop.}{Prop.}
\Crefname{theorem}{Thm.}{Thm.}
\Crefname{lemma}{Lemma}{Lemmas}
\Crefname{appendix}{Appendix}{Appendices}
