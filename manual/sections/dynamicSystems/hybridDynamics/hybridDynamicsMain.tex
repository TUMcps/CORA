
\subsection{Hybrid Dynamics} \label{sec:hybridDynamics}
\logToConsole{HYBRID DYNAMICS}

Hybrid systems consist of a finite number of state space regions for each of which specific continuous dynamics are defined. We refer to these regions as \textit{locations}. Besides a continuous state $x$, there consequently also exists a discrete state $v$ representing current location. The continuous initial state may take values within continuous sets while only a single initial discrete state is assumed without loss of generality\footnote{In the case of several initial discrete states, the reachability analysis can be performed for each discrete state separately.}. The switching of the continuous dynamics is triggered by \textit{guard sets}. Jumps in the continuous state are considered after the discrete state has changed. One of the most intuitive examples where jumps in the continuous state can occur, is the bouncing ball example (see \cref{fig:bouncingBall}), where the velocity of the ball changes instantaneously when hitting the ground.

\vspace{0.2cm}

In CORA, hybrid systems are modeled by hybrid automata. A hybrid automaton $HA = (L_1,\dots,L_p)$ as considered in CORA is defined by a finite list of locations $(L_1,\dots,L_p)$, where each location $L_i = (f_i(\cdot),\mathcal{S}_i,\mathbf{T}_i)$, $i = 1,\dots,p$ consists of
\begin{itemize}
    \item a differential equation $\dot{x}(t) = f_i(\cdot)$ describing the continuous dynamics,
    \item an invariant set $\mathcal{S}_i \subset \Rn$ describing the region where the differential equation is valid,
    \item a list $\mathbf{T}_i = ( T_1,\dots, T_q)$ of transitions $T_j = ( \mathcal{G}_j,r_j(\cdot),d_j )$, $j = \{1,\dots,q\}$ from the current location to other locations, where $\mathcal{G}_j \subset \Rn$ is a guard set, $r_j: \Rn \to \Rn$ is a reset function, and $d_j \in \{1,\dots,p\}$ is the index of the target mode.
\end{itemize}

\vspace{0.2cm}

The evolution of the hybrid automaton is described informally as follows: Starting from an initial location $v(0) \in \{1,\dots,p\}$ and an initial state $x(0)\in \mathcal{S}_{v(0)}$, the continuous state evolves according to the flow function $\dot{x}(t) = f_{v(0)}(\cdot)$ that is assigned to the location $v(0)$. If the continuous state is within a guard set $\mathcal{G}_j$ of a transition $T_j$, the transition $T_j$ can be taken and has to be taken if the state would otherwise leave the invariant $\mathcal{S}_{v(0)}$. When the transition from the previous location $v(0)$ to the next location $d_j$ is taken, the system state is updated according to the reset function $r_j(\cdot)$. Afterwards, the continuous state evolves according to the flow function of the next location.

\begin{center}
    \begin{minipage}[l]{0.3\columnwidth}
        \begin{center}
            \includetikz{./figures/tikz/hybridDynamics/bouncing_ball}
        \end{center}
    \end{minipage}
    \begin{minipage}[l]{0.6\columnwidth}
        \begin{equation*}
            \footnotesize
            \begin{array}{ll}
                HA            & = (L_1)                                                             \\
                & {\tiny ~}                                                           \\
                L_1           & = (f_1(\cdot),\mathcal{S}_1,(T_1))                                  \\
                & {\tiny ~}                                                           \\
                f_1(x,u)      & = \begin{bmatrix}
                                      x_2 \\ -g
                \end{bmatrix}, ~~ g = 9.81              \\
                & {\tiny ~}          \\
                \mathcal{S}_1              & = \Big \{ [x_1 ~x_2]^T \in \R^2~\Big |~ x_2 \geq 0 \Big \}                                                           \\
                & {\tiny ~}                                      \\
                T_1              & = (\mathcal{G}_1,r_1(\cdot),1)                                                           \\
                & {\tiny ~} \\
                \mathcal{G}_1              & = \Big \{ [x_1 ~x_2]^T \in \R^2~\Big |~ x_1 = 0,~x_2 \leq 0 \Big \}                                                           \\
                & {\tiny ~}                                               \\
                r(x)          & = \begin{bmatrix}
                                      x_1 \\ -\alpha x_2
                \end{bmatrix}, ~~ \alpha = 0.75
            \end{array}
        \end{equation*}
    \end{minipage}
    \captionof{figure}{Example for a hybrid system: bouncing ball.} \label{fig:bouncingBall}
\end{center}

A simple example for a hybrid system is the bouncing ball shown in \cref{fig:bouncingBall}, where the continuous system states are the vertical position $x_1 = s$ and the vertical velocity $x_2 = v$, and $\alpha \in [0,1]$ is the rebound factor that indirectly models the loss of energy during the collision with the ground. We will use the bouncing ball as a running example throughout this section.

\vspace{0.2cm}

Transitions between two locations are modeled in CORA by the class \texttt{transition}. An object of class transition can be constructed as follows:
\begin{align}
    \begin{split}
        \label{eq:transition}
        T &= \texttt{transition}(\mathcal{G},r(\cdot),d), \\
        T &= \texttt{transition}(\mathcal{G},r(\cdot),d,\texttt{label}),
    \end{split}
\end{align}
where
\begin{itemize}
    \item $\mathcal{G} \subset \mathbb{R}^n$ is the guard set. Guard sets can be modeled by all set representations described in \cref{sec:setRepresentations}.
    Most commonly, guard sets are modeled as \texttt{polytope} or \texttt{levelSet} objects.
    The guard set can also be left empty which results in an instantaneous transition, i.e., the guard set is active as soon as the location containing the transition of that guard set is entered. This feature is only advisable to be used in combination with synchronization labels (see below).
    \item $r: \mathbb{R}^n \to \mathbb{R}^n$ is the reset function. CORA supports linear reset functions defined as
    \begin{equation}
        \label{eq:linearReset}
        r(x,u) = A x + B u + c, ~~ A \in \R^{n \times n}, B \in \R^{n \times m}, c \in \mathbb{R}^n,
    \end{equation}
    as well as nonlinear reset functions.
    The reset function is specified as a \texttt{linearReset} or \texttt{nonlinearReset} object.
    For linear reset functions, one initializes a \texttt{linearReset(A,B,c)} object with the matrices $A$, $B$, and the vector $c$ in \eqref{eq:linearReset}.
    For nonlinear reset functions, the \texttt{nonlinearReset(f)} object is initialized with \texttt{f} storing a MATLAB function handle that defines the nonlinear reset function $r(x,u)$.

    \item $d \in \{1,\dots,p\}$ is the index of the target location.
    \item \texttt{label} is the synchronization label (only class \texttt{parallelHybridAutomata}): All transitions with the same synchronization label are executed simulaneously under the condition that the corresponding guard sets of all transitions are triggered. Currently, CORA only allows one transition of the set of transitions with the same synchronization label to have a non-empty guard set.
    Consequently, all transitions trigger if the one guard set is triggered.
\end{itemize}

For the bouncing ball example in \cref{fig:bouncingBall}, the transition $T_1$ can be constructed as follows:

\begin{center}
    \begin{minipage}[t]{0.1\textwidth}
        \vspace{10pt}
    \end{minipage}
    \begin{minipage}[t]{0.8\textwidth}
        \footnotesize
        % This file was automatically created from the m-file 
% "m2tex.m" written by USL. 
% The fontencoding in this file is UTF-8. 
%  
% You will need to include the following two packages in 
% your LaTeX-Main-File. 
%  
% \usepackage{color} 
% \usepackage{fancyvrb} 
%  
% It is advised to use the following option for Inputenc 
% \usepackage[utf8]{inputenc} 
%  
  
% definition of matlab colors: 
\definecolor{mblue}{rgb}{0,0,1} 
\definecolor{mgreen}{rgb}{0.13333,0.5451,0.13333} 
\definecolor{mred}{rgb}{0.62745,0.12549,0.94118} 
\definecolor{mgrey}{rgb}{0.5,0.5,0.5} 
\definecolor{mdarkgrey}{rgb}{0.25,0.25,0.25} 
  
\DefineShortVerb[fontfamily=courier,fontseries=m]{\$} 
\DefineShortVerb[fontfamily=courier,fontseries=b]{\#} 
  
\noindent        
 $$\color{mgreen}$% guard set$\color{black}$$\\
 $guard = polytope([0 1],0,[1 0],0)$\\
 $$\\
 $$\color{mgreen}$% reset function$\color{black}$$\\
 $reset = linearReset([1 0; 0 -0.75], [0; 0], [0;0]);$\\
 $$\\
 $$\color{mgreen}$% transtition object$\color{black}$$\\
 $trans = transition(guard,reset,1);$\\ 
  
\UndefineShortVerb{\$} 
\UndefineShortVerb{\#}
    \end{minipage}
\end{center}

\vspace{1cm}

The locations of a hybrid automaton are modeled in CORA by the class \texttt{location}. An object of class \texttt{location} can be constructed as follows:
\begin{equation}
    \begin{split}
        L &= \texttt{location}(\mathcal{S},\mathbf{T},f(\cdot)), \\
        L &= \texttt{location}(\texttt{name},\mathcal{S},\mathbf{T},f(\cdot)),
    \end{split}
    \label{eq:location}
\end{equation}
where
\begin{itemize}
    \item \texttt{name} is a string that specifies the name of the location.
    \item $\mathcal{S} \subset \mathbb{R}^n$ is the invariant set. Invariant sets can be modeled by all set representations described in \cref{sec:setRepresentations}. Most commonly, guard sets are modeled as \texttt{polytope} or \texttt{levelSet} objects.
    \item $\mathbf{T} = (T_1,\dots,T_j)$ is the list of transitions from the current location to other locations represented as a MATLAB cell array. Transitions are modeled by the class \texttt{transition} (see \eqref{eq:transition}).
    \item $\dot x = f(\cdot)$ is the differential equation that describes the continuous dynamics in the current location. The continous dynamics can be modeled by any of the system classes described in \cref{sec:continuousDynamics}.
\end{itemize}

For the bouncing ball example in \cref{fig:bouncingBall}, the location $L_1$ can be constructed as follows:

\begin{center}
    \begin{minipage}[t]{0.1\textwidth}
        \vspace{10pt}
    \end{minipage}
    \begin{minipage}[t]{0.8\textwidth}
        \footnotesize
        % This file was automatically created from the m-file 
% "m2tex.m" written by USL. 
% The fontencoding in this file is UTF-8. 
%  
% You will need to include the following two packages in 
% your LaTeX-Main-File. 
%  
% \usepackage{color} 
% \usepackage{fancyvrb} 
%  
% It is advised to use the following option for Inputenc 
% \usepackage[utf8]{inputenc} 
%  
  
% definition of matlab colors: 
\definecolor{mblue}{rgb}{0,0,1} 
\definecolor{mgreen}{rgb}{0.13333,0.5451,0.13333} 
\definecolor{mred}{rgb}{0.62745,0.12549,0.94118} 
\definecolor{mgrey}{rgb}{0.5,0.5,0.5} 
\definecolor{mdarkgrey}{rgb}{0.25,0.25,0.25} 
  
\DefineShortVerb[fontfamily=courier,fontseries=m]{\$} 
\DefineShortVerb[fontfamily=courier,fontseries=b]{\#} 
  
\noindent         
 $$\color{mgreen}$% differential equation$\color{black}$$\\
 $sys = linearSys([0 1;0 0],[0;0],[0;-9.81]);$\\
 $$\\
 $$\color{mgreen}$% invariant set$\color{black}$$\\
 $inv = polytope([-1 0],0);$\\
 $$\\
 $$\color{mgreen}$% location object$\color{black}$$\\
 $loc = location(inv,trans,sys);$\\
  
\UndefineShortVerb{\$} 
\UndefineShortVerb{\#}
    \end{minipage}
\end{center}

\vspace{1cm}

% hybrid automata
\subsubsection{Hybrid Automata} \label{sec:hybridAutomaton}

A hybrid automaton is modeled by the class \texttt{hybridAutomaton}. An object of class \texttt{hybridAutomaton} can be constructed as follows:
\begin{equation*}
    HA = \texttt{hybridAutomaton}(\mathbf{L}),
\end{equation*}
where $\mathbf{L} = (L_1,\dots,L_p)$ is a list of location objects represented as a MATLAB cell array. Locations are modeled by the class \texttt{location} (see \eqref{eq:location}).

The hybrid automaton for the bouncing ball example in \cref{fig:bouncingBall} can be constructed as follows:

\begin{center}
    \begin{minipage}[t]{0.1\textwidth}
        \vspace{10pt}
    \end{minipage}
    \begin{minipage}[t]{0.8\textwidth}
        \footnotesize
        % This file was automatically created from the m-file 
% "m2tex.m" written by USL. 
% The fontencoding in this file is UTF-8. 
%  
% You will need to include the following two packages in 
% your LaTeX-Main-File. 
%  
% \usepackage{color} 
% \usepackage{fancyvrb} 
%  
% It is advised to use the following option for Inputenc 
% \usepackage[utf8]{inputenc} 
%  
  
% definition of matlab colors: 
\definecolor{mblue}{rgb}{0,0,1} 
\definecolor{mgreen}{rgb}{0.13333,0.5451,0.13333} 
\definecolor{mred}{rgb}{0.62745,0.12549,0.94118} 
\definecolor{mgrey}{rgb}{0.5,0.5,0.5} 
\definecolor{mdarkgrey}{rgb}{0.25,0.25,0.25} 
  
\DefineShortVerb[fontfamily=courier,fontseries=m]{\$} 
\DefineShortVerb[fontfamily=courier,fontseries=b]{\#} 
  
\noindent     
 $$\color{mgreen}$% list of locations$\color{black}$$\\
 $locs(1) = loc;$\\
 $$\\
 $$\color{mgreen}$% hybrid automaton object$\color{black}$$\\
 $HA = hybridAutomaton(locs);$\\ 
  
\UndefineShortVerb{\$} 
\UndefineShortVerb{\#}
    \end{minipage}
\end{center}

\vspace{1cm}

\subsubsubsection{Operation \texttt{reach}}
\label{sec:hybridAutomatonReach}

For reachability analysis, we consider a set of initial states $\mathcal{X}_0 \subseteq \mathcal{S}_{v(0)}$ and a set of uncertain inputs $\mathcal{U} \subset \mathbb{R}^m$. The set of uncertain inputs can be different for each location of the hybrid automaton. An illustration of a reachable set of a hybrid automaton is provided in \cref{fig_hybridReachIllustration}. To calculate the reachable set inside a single location, CORA uses the reachability algorithms for continuous systems described in \cref{sec:continuousDynamics}. The most challenging part in reachability analysis for hybrid automata is the computation of the intersection between the reachable set and the guard set. CORA supports multiple methods for the calculation of guard intersections, which are listed in \cref{tab:guardIntersection}. For the intersection methods \texttt{polytope}, \texttt{zonoGirard}, \texttt{conZonotope}, and \texttt{nondetGuard} (see \cref{tab:guardIntersection}), the intersection with the guard set is enclosed by one or multiple oriented hyperrectangles. CORA supports the three methods listed in \cref{tab:enclose} to calculate the orientation of these hyperrectangles. The resulting hyperrectangles for the different enclosure methods are visualized in \cref{fig:enclose}. If multiple enclosure methods are specified, the reachable set is enclosed by the intersection of all computed hyperrectangles (see \cref{fig:enclose} (right)).

\begin{figure}[htb]
    \centering
    \includetikz{./figures/tikz/hybridDynamics/hybridReachOverview}
    \caption{Illustration of the reachable set of a hybrid automaton.}
    \label{fig_hybridReachIllustration}
\end{figure}

The settings for reachability analysis are specified as fields of the struct \texttt{options} (see \cref{sec:reach}). For hybrid automata the settings for the involved continuous dynamics objects (see \cref{sec:continuousDynamics})
have to be provided. In addition, the following settings specific to hybrid automata are available:

\begin{center}
    \renewcommand{\arraystretch}{1.3}
    \begin{tabular}[t]{l p{10cm}}
        --~\texttt{.guardIntersect}     & string specifying the method used to calculate the intersections with the guard sets. The available methods are listed in \cref{tab:guardIntersection}.                                                                                                                                                                       \\
        --~\texttt{.enclose}            & cell array storing the strings that describe the methods for enclosing the intersections with the guard sets. The available methods are listed in \cref{tab:enclose}. Required for the guard intersection methods \texttt{polytope}, \texttt{zonoGirard}, \texttt{conZonotope}, and \texttt{nondetGuard}.                     \\
        --~\texttt{.guardOrder}         & upper bound for the zonotope order $\rho$ (see \cref{sec:zonotope}). The zonotope order is reduced to \texttt{guardOrder} before the intersections with the guard sets are calculated in order to decrease the computation time. Required for the guard intersection methods \texttt{conZonotope} and \texttt{hyperplaneMap}. \\
        --~\texttt{.timeStep}           & time step size for one reachability time step. One can choose different time steps for each location by specifying \texttt{timeStep} as a cell array.                                                                                                                                                                             \\
        --~\texttt{.intersectInvariant} & flag with value \texttt{true} or \texttt{false} specifying whether the computed reachable set is intersected with the invariant set to obtain a tighter enclosure. The default value is \texttt{false} (no intersection).
    \end{tabular}
\end{center}
Furthermore, it is possible for hybrid automata to specify the set of uncertain inputs \texttt{params.U}, the time step \texttt{options.timeStep}, and the specification \texttt{spec} (see \cref{sec:reach}) as a MATLAB cell array with as many entries as the hybrid automaton has locations if the values are different for each location.

\begin{table*}[h]
    \centering
    \caption{Methods for enclosing guard intersections.}
    \label{tab:enclose}
    \begin{tabular}{l p{10cm} c}
        \toprule
        \textbf{Method} & \textbf{Description}                                                                                            & \textbf{Reference}                \\
        \midrule
        \texttt{box}    & The intersection is enclosed with an axis-aligned box.                                                          & Sec. V.A.a in \cite{Althoff2011f} \\
        \texttt{pca}    & The orientation of the hyperrectangle is determined using principal component analysis. & Sec. V.A.b in \cite{Althoff2011f} \\
        \texttt{flow}   & The orientation of the hyperrectangle is determined based on the direction of the flow of the dynamic function. & Sec. V.A.d in \cite{Althoff2011f} \\
        \bottomrule
    \end{tabular}
\end{table*}


\begin{table*}
    \centering
    \caption{Guard intersection methods in CORA.}
    \label{tab:guardIntersection}
    \begin{tabular}{l p{10cm} c}
        \toprule
        \textbf{Method}        & \textbf{Description}                                                                                                                                                                                                                                                                                                                                                                                                                                      & \textbf{Reference}     \\
        \midrule
        \texttt{polytope}      & The reachable sets are converted to polytopes and then intersected with the guard sets. Afterwards, the vertices of the sets representing the intersections are calculated. Finally, the vertices are enclosed by oriented hyperrectangles, where the orientation is determined by the methods in \cref{tab:enclose}. & \cite{Althoff2010d} \\
        \texttt{zonoGirard}    & First, suitable template directions are determined using the methods in \cref{tab:enclose}. Then, the algorithm described in \cite{Girard2008} is applied to compute an upper and a lower bound for the projection of the intersection between reachable set and guard set onto each template direction. & \cite{Girard2008} \\
        \texttt{conZonotope}   & Guard intersection computation based on constrained zonotopes (see \cref{sec:conZonotope}). Constrained zonotopes are closed under intersection. To this end, we first convert the reachable sets to constrained zonotopes and then intersect the reachable set with the guard sets. Finally, the union of all intersections is enclosed by oriented hyperrectangles, where the orientation is determined with the methods in \cref{tab:enclose}. & \\
        \texttt{hyperplaneMap} & The continuous dynamics are abstracted by constant flow, which allows to calculate the intersection with a hyperplane using a closed formula (guard mapping). & \cite{Althoff2012a} \\
        \texttt{pancake}       & The dynamics of the system is scaled by the distance to the guard set so that the reachable set is very flat shortly before passing the guard set. It is then often possible to pass the guard set in a single time step. & \cite{Bak2017} \\
        \texttt{nondetGuard}   & Guard intersection approach that works very well for non-deterministic guard sets. We first enclose all reachable sets that intersect the guard set with oriented hyperrectangles, where the orientation is determined using the methods in \cref{tab:enclose}. Afterwards, we compute the intersection of the oriented hyperrectangles with the guard set. & \\
        \texttt{levelSet}      & The intersections between the reachable set and nonlinear guard sets are enclosed by polynomial zonotopes (see \cref{sec:polynomialZonotopes}) & \cite{Kochdumper2020d} \\
        \bottomrule
    \end{tabular}
\end{table*}

\begin{table*}
    \centering
    \caption{Supported combinations of guard sets and guard intersection methods. The shorthand \texttt{polytope} denotes all polytopic set representations, which are \texttt{interval}, \texttt{zonotope}, \texttt{polytope}, \texttt{conZonotope}, and \texttt{zonoBundle}.}
    \label{tab:guardSet}
    \begin{tabular}{ p{4.5cm} C{3cm} C{3cm} C{3cm}}
        \toprule
        \texttt{options.guardIntersect} & \texttt{polytope} & \texttt{levelSet} \\ \midrule
        \texttt{polytope}               & $\cmark$          & $\xmark$          \\
        \texttt{zonoGirard}             & $\xmark$          & $\xmark$          \\
        \texttt{conZonotope}            & $\cmark$          & $\xmark$          \\
        \texttt{hyperplaneMap}          & $\xmark$          & $\xmark$          \\
        \texttt{pancake}                & $\xmark$          & $\xmark$          \\
        \texttt{nondetGuard}            & $\cmark$          & $\xmark$          \\
        \texttt{levelSet}               & $\xmark$          & $\cmark$          \\
        \bottomrule
    \end{tabular}
\end{table*}

\begin{figure}[htb]
    \centering
    \includetikz{./figures/tikz/hybridDynamics/example_manual_guard_intersection}
    \caption{Enclosing hyperrectangles for different methods to obtain the orientation (left) and intersection between the hyperrectangles for all methods (right).}
    \label{fig:enclose}
\end{figure}


% parallel hybrid automata
\subsubsection{Parallel Hybrid Automata} \label{sec:parallelHybridAutomata}

Complex systems can often be modeled as a connection of multiple distinct subcomponents, where each of these subcomponents represents a hybrid automaton. A naive approach to analyze these type of systems would be to construct a flat hybrid automaton from the interconnection of subcomponents (parallel composition, see e.g., \cite[Def.~2.9]{Frehse2005}). This technique, however, requires calculating all possible combinations of subsystem locations, and therefore suffers from the curse of dimensionality: Consider for example a system consisting of 15 subcomponents, where each subcomponent has 10 discrete locations. The flat hybrid automaton for this system would consist of $10^{15}$ discrete locations.

This exponential increase in the number of locations can be avoided if the overall system is modeled as a parallel hybrid automaton. In this case, the system is described by a list of \texttt{hybridAutomaton} objects representing the subcomponents and by connections between these components. The flow function, the invariant set, and the guard sets for a location of the composed system are computed on-demand as soon as a simulated solution or the reachable set enters the corresponding part of the state space. Since usually only a small part of the state space is explored by simulation or reachability analysis, it is possible to significantly reduce the computational costs of the analysis if the system is modeled as a parallel hybrid automaton \cite{Lee2015}. 

Parallel hybrid automata are implemented in CORA by the class \texttt{parallelHybridAutomaton}. An object of class \texttt{parallelHybridAutomaton} can be constructed as follows:
\begin{equation*}
	\texttt{obj} = \texttt{parallelHybridAutomaton}(\texttt{components},\texttt{inputBinds}),
\end{equation*}
with input arguments 
\begin{itemize}
  \item \texttt{components} -- cell array containing all subcomponents of the system. Each subcomponent has to be represented as a \texttt{hybridAutomaton} object (see \cref{sec:hybridAutomaton}). Currently, only hybrid automata for which the continuous dynamics are modeled as a linear system (see \cref{sec:linearSystems}) are supported.
  \item \texttt{inputBinds} -- cell array containing matrices that describe the connections between the subcomponents. Each matrix has two columns: the first column represents the component the signal comes from and the second column the output number, e.g., $[2, 3]$ refers to output 3 of component 2. When an input to a component is also an input to the composed system, we use index 0, e.g., $[0,1]$. For each input of the subcomponent, we specify a new row and the row number corresponds to the input index of the considered component. 
\end{itemize}
  
For better illustration of the required information, we introduce the example presented in \cref{fig:parallelHybridAutomaton} consisting of three components. For the parallel hybrid automaton in this example, the input binds have to be specified as follows:
  {\small % This file was automatically created from the m-file 
% "m2tex.m" written by USL. 
% The fontencoding in this file is UTF-8. 
%  
% You will need to include the following two packages in 
% your LaTeX-Main-File. 
%  
% \usepackage{color} 
% \usepackage{fancyvrb} 
%  
% It is advised to use the following option for Inputenc 
% \usepackage[utf8]{inputenc} 
%  
  
% definition of matlab colors: 
\definecolor{mblue}{rgb}{0,0,1} 
\definecolor{mgreen}{rgb}{0.13333,0.5451,0.13333} 
\definecolor{mred}{rgb}{0.62745,0.12549,0.94118} 
\definecolor{mgrey}{rgb}{0.5,0.5,0.5} 
\definecolor{mdarkgrey}{rgb}{0.25,0.25,0.25} 
  
\DefineShortVerb[fontfamily=courier,fontseries=m]{\$} 
\DefineShortVerb[fontfamily=courier,fontseries=b]{\#} 
  
\noindent                
 $  inputBinds{1} = [[0 2];[0 1];[2 1]]; $\color{mgreen}$% input connections for component 1$\color{black}$$\\
 $  inputBinds{2} = [[0 1];[0 2]]; $\color{mgreen}~~~~~~~~~~~$% input connections for component 2$\color{black}$$\\
 $  inputBinds{3} = [[1 2];[2 2]]; $\color{mgreen}~~~~~~~~~~~$% input connections for component 3$\color{black}$$\\
  
\UndefineShortVerb{\$} 
\UndefineShortVerb{\#}
}
  Let us briefly discuss the solution for component 1, which has three inputs and thus \texttt{inputBinds\{1\}} has three rows: The first input (first row) is the second input of the composed system; the second input is the first input of the composed system; and the third input is the first output of component 2. 

Since the modeling of hybrid automata is tedious and error-prone, we provide a method to read models of parallel hybrid automata using the SpaceEx format \cite{Donze2013}. For modeling and modifying SpaceEx models, one can use the freely-available SpaceEx model editor downloadable from \href{http://spaceex.imag.fr/download-6}{spaceex.imag.fr/download-6}. Details on converting SpaceEx models to models as defined in this section can be found in \cref{sec:loadingModels}.


\begin{figure}[htb]
  \centering	
    \includegraphics[width=0.9\columnwidth]{./figures/parallelHybridAutomaton.eps}
    \caption{Example of a parallel hybrid automaton that consists of three subcomponents.}
    \label{fig:parallelHybridAutomaton}		
\end{figure}



\subsubsubsection{Operation \texttt{reach}}

The settings for reachability analysis are specified as fields of the struct \texttt{options} (see \cref{sec:reach}). For parallel hybrid automata, the settings are identical to the ones for hybrid automata (see \cref{sec:hybridAutomatonReach}).

The initial location \texttt{params.startLoc} and the final location \texttt{params.finalLoc} (see \cref{sec:reach}) are specified as a vector $l \in \mathbb{N}_{\geq 0}^s$, where each entry of the vector represents the index of the location for one of the $s$ subcomponents.

For the set of uncertain inputs specified by \texttt{params.U} (see \cref{sec:reach}), there exist two different cases for parallel hybrid automata:
\begin{enumerate}
	\item The input set is identical for each component and location. In this case, a single set $\mathcal{U} \subset \R^m$ represented as a zonotope (see \cref{sec:zonotope}) is provided.
	\item The input set is different for each component and location. In this case, \texttt{params.U} can be specified as a cell array, where each entry represents the input set for one component. Since each component can have multiple locations, the input set for each component is again a cell array whose entries represent the input sets for all locations. The input set for the overall system is then constructed on demand for each visited location according to
	\begin{equation*}
		\mathcal{U} = \texttt{params.U\{}i_{(1)}\texttt{\}\{}l_{(i_{(1)})} \texttt{\}} \times \dots \times \texttt{params.U\{}i_{(m)}\texttt{\}\{}l_{(i_{(m)})} \texttt{\}} ,
	\end{equation*}
	where the vector $l \in \mathbb{N}_{\geq 0}^s$ stores the index of the current location for all $s$ components, and the vector $i \in \mathbb{N}_{\geq 0}^m$ maps the input sets for the single components to the global input set. The vector $i$ can be specified with an additional setting $\texttt{params.inputCompMap} = i$. 
\end{enumerate}
