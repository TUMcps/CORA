\subsubsection{Nonlinear ARX Models} \label{sec:nonlinearSystems_ARX}

In this section, we consider nonlinear autoregressive models with exogenous input (nonlinear ARX models) defined as
\begin{equation}\label{eq:NARX}
	y[i] = f(y[i-1:i-p],{u}[i:i-p])
\end{equation} 
with
\begin{align*}
	y[i-1:i-p] &=\begin{bmatrix} y[i-1]^\top &... & y[i-p]^\top\end{bmatrix}^\top, \\
	{u}[i:i-p] &= \begin{bmatrix} {u}[i]^\top &...&{u}[i-p]^\top\end{bmatrix}^\top,
\end{align*} 
where $y[i] \in \R^o$ is the system output, $u[i] \in \R^m$ is the system input, and $f:\R^{op} \times \times \R^{m(p+1)} \to \R^o$ is a continuous function. %Lipschitz continuity not required for discrete-time systems
Nonlinear ARX models are implemented in CORA by the class \texttt{nonlinearARX}. An object of class \texttt{nonlinearARX} can be constructed as follows:
\begin{equation*}
	\begin{split}
    	& \texttt{sys} = \texttt{nonlinearARX}(\texttt{fun},\Delta t,o,m,p), \\
     	& \texttt{sys} = \texttt{nonlinearARX}(\texttt{name},\texttt{fun},\Delta t,o,m,p), \\
	\end{split}
\end{equation*} 
where \texttt{name} is a string specifying the name of the system, \texttt{fun} is a MATLAB function handle defining the function $f(y[i-1:i-p],{u}[i:i-p])$ in \eqref{eq:NARX}, $\Delta t$ is the sampling time specifying the time difference between $y[i]$ and $y[i-1]$, $o$ is the dimension of the system output $y[i]$, $m$ is the dimension of the system input $u[i]$, and $p$ is the number of past outputs and inputs that are considered in the function $f$. Let us demonstrate the class \texttt{nonlinearARX} by an example:

\begin{center}
\begin{minipage}[t]{0.48\textwidth}
	\vspace{10pt}
	\begin{equation*}
	\begin{bmatrix} y_1[i] \\ y_2[i] \\ y_3[i] \end{bmatrix} = \begin{bmatrix} y_1[i-1] + u_1[i] + \cos(u_1[i-1]) \\ y_2[i-2] + y_3[i-1] \cos(u_2[i]) \\ y_3[i-1] + u_2[i-1] \sin(y_1[i-2]) \end{bmatrix}
\end{equation*}
\end{minipage}
\begin{minipage}[t]{0.48\textwidth}
	\footnotesize
	% This file was automatically created from the m-file 
% "m2tex.m" written by USL. 
% The fontencoding in this file is UTF-8. 
%  
% You will need to include the following two packages in 
% your LaTeX-Main-File. 
%  
% \usepackage{color} 
% \usepackage{fancyvrb} 
%  
% It is advised to use the following option for Inputenc 
% \usepackage[utf8]{inputenc} 
%  
  
% definition of matlab colors: 
\definecolor{mblue}{rgb}{0,0,1} 
\definecolor{mgreen}{rgb}{0.13333,0.5451,0.13333} 
\definecolor{mred}{rgb}{0.62745,0.12549,0.94118} 
\definecolor{mgrey}{rgb}{0.5,0.5,0.5} 
\definecolor{mdarkgrey}{rgb}{0.25,0.25,0.25} 
  
\DefineShortVerb[fontfamily=courier,fontseries=m]{\$} 
\DefineShortVerb[fontfamily=courier,fontseries=b]{\#} 
  
\noindent          
 $$\color{mgreen}$% equation f(y,u)$\color{black}$$\\
 $f = @(y,u) [y(1) + u(1)*sin(u(3));$\color{mblue}$ ...$\color{black}$$\\
 $            y(5) + y(3)*cos(u(2));$\color{mblue}$ ...$\color{black}$$\\
 $            y(3) + u(4)*sin(y(4))];$\\
 $$\\
 $$\color{mgreen}$% sampling time$\color{black}$$\\
 $dt = 0.25;$\\
 $        $\\
 $$\color{mgreen}$% nonlinear discrete-time system$\color{black}$$\\
 $sys = nonlinearARX(f,dt,3,2,2);$\\ 
  
\UndefineShortVerb{\$} 
\UndefineShortVerb{\#}
\end{minipage}
\end{center}





\subsubsubsection{Operations \texttt{reach}}  

By introducing the state $x[i] = y[i:i-p+1]$ and the input $\tilde{u}[i] = {u}[i:i-p]$, the nonlinear ARX model in \eqref{eq:NARX} can be transformed to an object of the class \texttt{nonlinearSysDT}.
More details on the transformation can be found in \cite{Luetzow2024b}. 
Thus, the reachable set of \texttt{nonlinearARX} objects can be computed analogously to the reachable set of \texttt{nonlinearSysDT} objects leading to the same available settings specified in the struct \texttt{options} (see \cref{sec:nonlinearSystemsDT}).



