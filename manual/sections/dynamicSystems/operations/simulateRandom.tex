\subsubsection{simulateRandom} \label{sec:simulateRandom}

The operation \operator{simulateRandom} simulates a dynamic system for multiple random initial states $x_0 \in \mathcal{X}_0$ and random values for the inputs $u(t) \in \mathcal{U}$ as well as parameters $p \in \mathcal{P}$. The syntax is as follows:
\begin{equation*}
	\texttt{simRes} = \texttt{simulateRandom}(\texttt{sys},\texttt{params},\texttt{options})
\end{equation*}	
with input arguments
\begin{center}
\renewcommand{\arraystretch}{1.3}
\begin{tabular}[t]{l p{13cm} }
	$\bullet$~\texttt{sys} &  dynamic system defined by one of the classes in \cref{sec:continuousDynamics,sec:hybridDynamics}, e.g., \texttt{linearSys}, \texttt{hybridAutomaton}, etc. \\
	$\bullet$~\texttt{params} & struct containing the parameters that define the reachability problem. The parameters are identical to those for the operation \operator{reach} (see \cref{sec:reach}). \\
	$\bullet$~\texttt{options} & struct containing settings for the random simulation \\
		& \begin{tabular}[t]{l p{10cm}}	
		--~\texttt{.points} & number of random initial states (positive integer)\\
	 	--~\texttt{.type} & sampling method: \texttt{'standard'} (default, undefined distribution), \texttt{'gaussian'} (Gaussian distribution), \texttt{'rrt'} (sampling using \textit{rapidly exploring random trees})
	 	\end{tabular} \\
	 	& depending on the sampling method, there are different additional settings \\
		& \texttt{.type = 'standard'}: standard sampling method (undefined distribution) \\
		& \begin{tabular}[t]{l p{10cm}}	
	 	--~\texttt{.fracVert} & percentage of initial states randomly drawn from the vertices of the initial set $\mathcal{X}_0$ (value in $[0,1]$) \\ 	
	 	--~\texttt{.fracInpVert} & percentage of input values randomly drawn from the vertices of the input set $\mathcal{U}$ (value in $[0,1]$) \\
	 	--~\texttt{.nrConstInp} & number piecewise-constant input values within the input signal during simulation (integer $\geq 1$) \\
	 	\end{tabular} \\
	 	& \texttt{.type = 'rrt'}: sampling using \emph{rapidly-exploring random trees} \\
	 	& \begin{tabular}[t]{l p{10cm}}	
	 	--~\texttt{.vertSamp} & flag specifying whether random initial states, inputs, and parameters are sampled from the vertices of the corresponding sets (0 or 1).\\ 	
	 	--~\texttt{.stretchFac} & stretching factor for enlarging the reachable sets during execution of the algorithm (scalar $>$ 1). \\
	 	--~\texttt{.R} & object of class \texttt{reachSet} (see \cref{sec:reachSet}) that stores the reachable set for the corresponding reachability problem.
	 	\end{tabular} \\
	 	& \texttt{.type = 'gaussian'}: sampling from gaussian distribution \\
		& \begin{tabular}[t]{l p{10cm}}	
	 	--~\texttt{.nrConstInp} & number piecewise-constant input values within the input signal during simulation (integer $\geq 1$) \\
	 	--~\texttt{.p\_conf} & probability that sampled value is within the set (value in $(0,1)$)
	 	\end{tabular}
\end{tabular}
\end{center}

and output arguments

\begin{center}
\renewcommand{\arraystretch}{1.3}
\begin{tabular}[t]{l p{13cm} }
	$\bullet$~\texttt{simRes} & object of class \texttt{simResult} (see \cref{sec:simResult}) that stores the simulated trajectories.
\end{tabular}
\end{center}

Let us demonstrate the operation \texttt{simulateRandom} by an example:

\begin{center}
\begin{minipage}[t]{0.58\textwidth}
	\footnotesize
	% This file was automatically created from the m-file 
% "m2tex.m" written by USL. 
% The fontencoding in this file is UTF-8. 
%  
% You will need to include the following two packages in 
% your LaTeX-Main-File. 
%  
% \usepackage{color} 
% \usepackage{fancyvrb} 
%  
% It is advised to use the following option for Inputenc 
% \usepackage[utf8]{inputenc} 
%  
  
% definition of matlab colors: 
\definecolor{mblue}{rgb}{0,0,1} 
\definecolor{mgreen}{rgb}{0.13333,0.5451,0.13333} 
\definecolor{mred}{rgb}{0.62745,0.12549,0.94118} 
\definecolor{mgrey}{rgb}{0.5,0.5,0.5} 
\definecolor{mdarkgrey}{rgb}{0.25,0.25,0.25} 
  
\DefineShortVerb[fontfamily=courier,fontseries=m]{\$} 
\DefineShortVerb[fontfamily=courier,fontseries=b]{\#} 
  
\noindent                
 $$\color{mgreen}$% system dynamics$\color{black}$$\\
 $sys = linearSys([-0.7 -2;2 -0.7],[1;1],[-2;-1]);$\\
 $$\\
 $$\color{mgreen}$% parameter$\color{black}$$\\
 $params.tFinal = 5;$\\
 $params.R0 = zonotope(interval([2;2],[2.5;2.5]));$\\
 $params.U = zonotope(interval(-0.1,0.1));$\\
 $$\\
 $$\color{mgreen}$% simulation settings$\color{black}$$\\
 $options.points = 7;$\\
 $options.fracVert = 0.5;$\\
 $options.fracInpVert = 1;$\\
 $options.nrConstInp = 10;$\\
 $$\\
 $$\color{mgreen}$% random simulation$\color{black}$$\\
 $simRes = simulateRandom(sys,params,options);$\\ 
  
\UndefineShortVerb{\$} 
\UndefineShortVerb{\#}
\end{minipage}
\begin{minipage}[t]{0.4\textwidth}
	\vspace{0pt}
	\centering
	\includetikz{./figures/tikz/contDynamics/example_simulateRandom}
\end{minipage}
\end{center}


% example RRT
%Let us demonstrate the operation \texttt{simulateRRT} by an example:
%
%\begin{center}
%\begin{minipage}[t]{0.58\textwidth}
%	\footnotesize
%	% This file was automatically created from the m-file 
% "m2tex.m" written by USL. 
% The fontencoding in this file is UTF-8. 
%  
% You will need to include the following two packages in 
% your LaTeX-Main-File. 
%  
% \usepackage{color} 
% \usepackage{fancyvrb} 
%  
% It is advised to use the following option for Inputenc 
% \usepackage[utf8]{inputenc} 
%  
  
% definition of matlab colors: 
\definecolor{mblue}{rgb}{0,0,1} 
\definecolor{mgreen}{rgb}{0.13333,0.5451,0.13333} 
\definecolor{mred}{rgb}{0.62745,0.12549,0.94118} 
\definecolor{mgrey}{rgb}{0.5,0.5,0.5} 
\definecolor{mdarkgrey}{rgb}{0.25,0.25,0.25} 
  
\DefineShortVerb[fontfamily=courier,fontseries=m]{\$} 
\DefineShortVerb[fontfamily=courier,fontseries=b]{\#} 
  
\noindent                       
 $$\color{mgreen}$% system dynamics$\color{black}$$\\
 $sys = linearSys([-0.7 -2;2 -0.7],[1;1],[-2;-1]);$\\
 $$\\
 $$\color{mgreen}$% parameter$\color{black}$$\\
 $params.tFinal = 5;$\\
 $params.R0 = zonotope(interval([2;2],[2.5;2.5]));$\\
 $params.U = zonotope(interval(-0.1,0.1));$\\
 $$\\
 $$\color{mgreen}$% reachability settings$\color{black}$$\\
 $options.timeStep = 0.05;$\\
 $options.zonotopeOrder = 10;$\\
 $options.taylorTerms = 5;$\\
 $$\\
 $$\color{mgreen}$% reachability analysis$\color{black}$$\\
 $R = reach(sys,params,options);$\\
 $$\\
 $$\color{mgreen}$% simulation settings$\color{black}$$\\
 $simOptions.points = 20;$\\
 $simOptions.vertSamp = 0;$\\
 $simOptions.stretchFac = 1.5;$\\
 $$\\
 $$\color{mgreen}$% simulation with RRTs$\color{black}$$\\
 $simRes = simulateRRT(sys,R,params,simOptions);$\\ 
  
\UndefineShortVerb{\$} 
\UndefineShortVerb{\#}
%\end{minipage}
%\begin{minipage}[t]{0.4\textwidth}
%	\vspace{30pt}
%	\centering
%	\includegraphics[width=0.9\columnwidth]{./figures/examples/example_simulateRRT.eps}
%\end{minipage}
%\end{center}