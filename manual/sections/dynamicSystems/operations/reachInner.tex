\subsubsection{reachInner} \label{sec:reachInner}

The operation \operator{reach}, which was introduced in \cref{sec:reach}, computes a tight outer-approximation $\mathcal{R}(t) \supseteq \mathcal{R}^e(t)$ of the exact reachable set $\mathcal{R}^e(t)$ as defined in \eqref{eq:reachSet}. For most cases computing an outer-approximation of the reachable set is sufficient. However, sometimes it also required to compute an inner-approximation $\mathcal{R}^i(t) \subseteq \mathcal{R}^e(t)$ of the exact reachable set: Inner-approximations can be used to prove that a system provably violates a given specification, they are required for conformance testing using \textit{reachset conformance} \cite{Roehm2016}, and they are very useful for controller synthesis where one often has to prove that the controlled system is guaranteed to reach a certain goal set.

The operation \operator{reachInner} computes a tight inner-approximation $\mathcal{R}^i(t) \subseteq \mathcal{R}^e(t)$ of the exact reachable set $\mathcal{R}^e(t)$ as defined in \eqref{eq:reachSet}. Currently, CORA only supports the computation of inner-approximations for the reachable set at certain time-points $t_i$, but not for time intervals $\tau_i = [t_i,t_{i+1}]$. Since some approaches compute an inner-approximation based on the outer-approximation of the reachable set, the operation \texttt{reachInner} returns in some cases both an inner-approximation as well as an outer-approximation of the reachable set.

The syntax for the operation \texttt{reachInner} is:
\begin{equation*}
	\begin{split}
		[\texttt{Rin},\texttt{Rout}] &= \texttt{reachInner}(\texttt{sys},\texttt{params},\texttt{options}), \\
		\texttt{Rin} &= \texttt{reachInner}(\texttt{sys},\texttt{params},\texttt{options}),
	\end{split}
\end{equation*}
with input arguments

\begin{center}
\renewcommand{\arraystretch}{1.3}
\begin{tabular}[t]{l p{13cm} }
	$\bullet$~\texttt{sys} & dynamic system defined by any of the classes in \cref{sec:continuousDynamics}, e.g., \texttt{linearSys}, \texttt{nonlinearSys}, etc. \\
	$\bullet$~\texttt{params} & struct containing the parameter that define the reachability problem. The parameters are identical to those for the operation \texttt{reach} (see \cref{sec:reach}). \\
	$\bullet$~\texttt{options} & struct containing algorithm settings for reachability analysis. Since the settings are different for each type of dynamic system, they are documented in \cref{sec:continuousDynamics}.
\end{tabular}
\end{center}

and output arguments

\begin{center}
\renewcommand{\arraystretch}{1.3}
\begin{tabular}[t]{l p{13cm} }
	$\bullet$~\texttt{Rin} & object of class \texttt{reachSet} (see \cref{sec:reachSet}) that stores the inner-approximations $\mathcal{R}^i(t_i)$ of the reachable set at time points $t_i$. \\
	$\bullet$~\texttt{Rout} & object of class \texttt{reachSet} (see \cref{sec:reachSet}) that stores the outer-approximations $\mathcal{R}(t_i)$ of the reachable set at time points $t_i$ (class \texttt{nonlinearSys} only).
\end{tabular}
\end{center}


Let us demonstrate the operation \texttt{reachInner} by an example:

\begin{center}
\begin{minipage}[t]{0.58\textwidth}
	\footnotesize
	% This file was automatically created from the m-file 
% "m2tex.m" written by USL. 
% The fontencoding in this file is UTF-8. 
%  
% You will need to include the following two packages in 
% your LaTeX-Main-File. 
%  
% \usepackage{color} 
% \usepackage{fancyvrb} 
%  
% It is advised to use the following option for Inputenc 
% \usepackage[utf8]{inputenc} 
%  
  
% definition of matlab colors: 
\definecolor{mblue}{rgb}{0,0,1} 
\definecolor{mgreen}{rgb}{0.13333,0.5451,0.13333} 
\definecolor{mred}{rgb}{0.62745,0.12549,0.94118} 
\definecolor{mgrey}{rgb}{0.5,0.5,0.5} 
\definecolor{mdarkgrey}{rgb}{0.25,0.25,0.25} 
  
\DefineShortVerb[fontfamily=courier,fontseries=m]{\$} 
\DefineShortVerb[fontfamily=courier,fontseries=b]{\#} 
  
\noindent                    
 $$\color{mgreen}$% system dynamics$\color{black}$$\\
 $f = @(x,u) [1-2*x(1) + 3/2 * x(1)^2*x(2);$\color{mblue}$ ...$\color{black}$$\\
 $            x(1)-3/2*x(1)^2*x(2)];$\\
 $    $\\
 $sys = nonlinearSys(f); $\\
 $$\\
 $$\color{mgreen}$% parameter$\color{black}$$\\
 $params.tFinal = 1;$\\
 $params.R0 = interval([0.75;0],[1;0.25]);$\\
 $$\\
 $$\color{mgreen}$% reachability settings$\color{black}$$\\
 $options.algInner = $\color{mred}$'scale'$\color{black}$;$\\
 $options.timeStep = 0.001;                           $\\
 $options.taylorTerms = 10;                            $\\
 $options.zonotopeOrder = 50;       $\\
 $options.intermediateOrder = 20;$\\
 $options.errorOrder = 10;$\\
 $$\\
 $$\color{mgreen}$% reachability analysis$\color{black}$$\\
 $[Rin,Rout] = reachInner(sys,params,options);$\\ 
  
\UndefineShortVerb{\$} 
\UndefineShortVerb{\#}
\end{minipage}
\begin{minipage}[t]{0.4\textwidth}
	\vspace{0pt}
	\centering
	\includetikz{./figures/tikz/contDynamics/example_reachInner}
\end{minipage}
\end{center}