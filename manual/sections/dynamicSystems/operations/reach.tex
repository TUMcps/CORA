\subsubsection{reach} \label{sec:reach}

The operation \operator{reach} computes the reachable set of a dynamic system. Let us denote the solution of a dynamic system by $\chi(t;x_0,u(\cdot),p)$, where $t\in\mathbb{R}$ is the time, $x_0 = x(t_0) \in \mathbb{R}^n$ is the initial state, $u(\cdot) \in \mathbb{R}^m$ is the system input, and $p\in \mathbb{R}^p$ is a parameter vector. The reachable set at time $t=t_f$ can be defined for a set of initial states $\mathcal{X}_0 \subset \Rn$, a set of input values $\mathcal{U}(t) \subset \R^m$, and a set of parameter values $\mathcal{P} \subset \R^p$, as
\begin{equation}
  \mathcal{R}^e(t_f) = \Big\{ \chi(t_f;x_0,u(\cdot),p,w) \in \mathbb{R}^n ~\big|~ x_0 \in \mathcal{X}_0, \forall t \in [t_0,t_f]: u(t)\in\mathcal{U}(t), p \in \mathcal{P}, w \in \mathcal{W} \Big\}.
  \label{eq:reachSet}
\end{equation}
Since the exact reachable set $\mathcal{R}^e(t)$ as defined in \eqref{eq:reachSet} cannot be computed in general, the operation \texttt{reach} computes a tight enclosure $\mathcal{R}(t) \supseteq \mathcal{R}^e(t)$.

The syntax for the operation \texttt{reach} is:
\begin{equation*}
\begin{split}
	& \texttt{R} = \texttt{reach}(\texttt{sys},\texttt{params},\texttt{options}) \\,
	& [\texttt{R},\texttt{res}] = \texttt{reach}(\texttt{sys},\texttt{params},\texttt{options},\texttt{spec}),
\end{split}
\end{equation*}
with input arguments

\begin{center}
\renewcommand{\arraystretch}{1.3}
\begin{tabular}[t]{l p{13cm} }
	$\bullet$~\texttt{sys} & dynamic system defined by any of the classes in \cref{sec:continuousDynamics,sec:hybridDynamics}, e.g., \texttt{linearSys}, \texttt{hybridAutomaton}, etc. \\
	$\bullet$~\texttt{params} & struct containing the parameter that define the reachability problem \\
	& \begin{tabular}[t]{l p{10cm}}	
	 	--~\texttt{.tStart} & initial time $t_0$ (default value 0) \\
	 	--~\texttt{.tFinal} & final time $t_f$ \\
	 	--~\texttt{.R0} & initial set $\mathcal{X}_0$ specified by one of the set representations in \cref{sec:basicSetRep}\\
	 	--~\texttt{.U} & input set $\mathcal{U}$ specified as an object of class \texttt{zonotope} (see \cref{sec:zonotope})\\
	 	--~\texttt{.u} & time-dependent center $u_c(t)$ of the time-varying input set $\mathcal{U}(t) := u_c(t) \oplus \mathcal{U}$ specified as a matrix for which the number of columns is identical to the number of reachability steps (optional)\\
	 	--~\texttt{.paramInt} & set of parameter values $\mathcal{P}$ specified as an object of class \texttt{interval} (see \cref{sec:interval}) (class \texttt{nonlinParamSys} only)\\
	 	--~\texttt{.W} & disturbance set $\mathcal{W}$ specified as an object of class \texttt{interval} (see \cref{sec:interval}) or \texttt{zonotope} (see \cref{sec:zonotope} (classes \texttt{linearSys} and \texttt{linearSysDT} only) \\
	 	--~\texttt{.V} & set of sensor noises $\mathcal{V}$ specified as an object of class \texttt{interval} (see \cref{sec:interval})or \texttt{zonotope} (see \cref{sec:zonotope} (classes \texttt{linearSys} and \texttt{linearSysDT} only) \\
	 \end{tabular}
\end{tabular}
\end{center}
\begin{center}
\renewcommand{\arraystretch}{1.3}
\begin{tabular}[t]{l p{13cm} }
	& \begin{tabular}[t]{l p{10cm}}	
	    --~\texttt{.y0guess} & guess for a consistent initial algebraic state (class \texttt{nonlinDASys} only, see \cref{sec:reachDAEsys}). \\
	 	--~\texttt{.startLoc} & index of the initial location (class \texttt{hybridAutomaton} and \texttt{parallelHybridAutomaton} only)\\
	 	--~\texttt{.finalLoc} & index of the final location. Reachability analysis stops as soon as the final location is reached (class \texttt{hybridAutomaton} and \texttt{parallelHybridAutomaton} only, optional)
	 \end{tabular} \\
	$\bullet$~\texttt{options} & struct containing algorithm settings for reachability analysis. Since the settings are different for each type of dynamic system, they are documented in \cref{sec:continuousDynamics} and \cref{sec:hybridDynamics}. \\
	$\bullet$~\texttt{spec} & object of class \texttt{specification} (see \cref{sec:specification}) which represents the specifications the system has to verify. Reachability analysis stops as soon as a specification is violated.
\end{tabular}
\end{center}

and output arguments

\begin{center}
\renewcommand{\arraystretch}{1.3}
\begin{tabular}[t]{l p{13cm} }
	$\bullet$~\texttt{R} & object of class \texttt{reachSet} (see \cref{sec:reachSet}) that stores the reachable set $\mathcal{R}(t_i)$ at time point $t_i$ and the reachable set $\mathcal{R}(\tau_i)$ for time intervals $\tau_i = [t_i,t_{i+1}]$. \\
	$\bullet$~\texttt{res} & Boolean flag that indicates whether the specifications are satisfied (\texttt{res} = 1) or not (\texttt{res} = 0).
\end{tabular}
\end{center}


Let us demonstrate the operation \texttt{reach} by an example:

\begin{center}
\begin{minipage}[t]{0.58\textwidth}
	\footnotesize
	% This file was created by matlab2tikz.
%
\definecolor{mycolor1}{rgb}{0.27060,0.58820,1.00000}%
%
\begin{tikzpicture}
\footnotesize

\begin{axis}[%
width=4cm,
height=4cm,
at={(0in,0in)},
scale only axis,
xmin=-3,
xmax=3,
xlabel style={font=\color{white!15!black}},
xlabel={$x_{(1)}$},
ymin=-3,
ymax=3,
ylabel style={font=\color{white!15!black}},
ylabel={$x_{(2)}$},
axis background/.style={fill=white}
]

\addplot[area legend, draw=mycolor1, fill=mycolor1, forget plot]
table[row sep=crcr] {%
x	y\\
2.0015	1.9698\\
2.0079	1.9698\\
2.4981	1.9939\\
2.5222	2.0037\\
2.5324	2.0139\\
2.5324	2.0207\\
2.5083	2.5109\\
2.4985	2.535\\
2.4983	2.5352\\
2.4976	2.5356\\
2.0968	2.6103\\
2.0905	2.6103\\
1.6003	2.5862\\
1.5762	2.5764\\
1.5659	2.5661\\
1.5659	2.5594\\
1.59	2.0692\\
1.5998	2.0451\\
1.6001	2.0448\\
1.6007	2.0445\\
2.0015	1.9698\\
}--cycle;

\addplot[area legend, draw=mycolor1, fill=mycolor1, forget plot]
table[row sep=crcr] {%
x	y\\
1.6328	2.0264\\
1.6378	2.0269\\
1.6389	2.027\\
2.1076	2.0974\\
2.1298	2.1091\\
2.1398	2.1191\\
2.14	2.1194\\
2.1486	2.1299\\
2.1481	2.135\\
2.148	2.1365\\
2.0776	2.6051\\
2.0658	2.6273\\
2.0656	2.6276\\
2.0649	2.6279\\
2.0642	2.6281\\
1.672	2.6613\\
1.667	2.6608\\
1.6659	2.6607\\
1.1972	2.5903\\
1.175	2.5785\\
1.1648	2.5683\\
1.1561	2.5577\\
1.1566	2.5527\\
1.1568	2.5512\\
1.2272	2.0825\\
1.2389	2.0603\\
1.2392	2.0601\\
1.2399	2.0598\\
1.2405	2.0595\\
1.6328	2.0264\\
}--cycle;

\addplot[area legend, draw=mycolor1, fill=mycolor1, forget plot]
table[row sep=crcr] {%
x	y\\
0.893	2.0392\\
1.2731	2.0452\\
1.2779	2.0461\\
1.2789	2.0464\\
1.7224	2.1592\\
1.7426	2.1726\\
1.7526	2.1826\\
1.7612	2.1932\\
1.7614	2.1934\\
1.7687	2.2044\\
1.7687	2.2045\\
1.7678	2.2093\\
1.7675	2.2107\\
1.6547	2.6541\\
1.6412	2.6744\\
1.641	2.6745\\
1.6403	2.6749\\
1.6389	2.6753\\
1.2588	2.6693\\
1.2541	2.6684\\
1.253	2.6682\\
0.8096	2.5553\\
0.7893	2.5419\\
0.7793	2.5319\\
0.7707	2.5213\\
0.7705	2.5211\\
0.7632	2.5101\\
0.7632	2.51\\
0.7642	2.5053\\
0.7645	2.5038\\
0.8773	2.0604\\
0.8907	2.0402\\
0.891	2.04\\
0.8917	2.0396\\
0.8923	2.0394\\
0.893	2.0392\\
}--cycle;

\addplot[area legend, draw=mycolor1, fill=mycolor1, forget plot]
table[row sep=crcr] {%
x	y\\
0.5611	1.9862\\
0.9257	2.0286\\
0.9267	2.0289\\
0.9312	2.0303\\
1.3464	2.1814\\
1.3645	2.1963\\
1.3745	2.2063\\
1.3831	2.2168\\
1.3904	2.2278\\
1.3906	2.2281\\
1.3965	2.2394\\
1.3951	2.2439\\
1.3947	2.2453\\
1.2436	2.6605\\
1.2287	2.6786\\
1.2285	2.6788\\
1.2278	2.6791\\
1.2264	2.6795\\
1.2258	2.6796\\
0.8611	2.6372\\
0.8602	2.6369\\
0.8557	2.6355\\
0.4405	2.4844\\
0.4223	2.4695\\
0.4123	2.4595\\
0.4037	2.449\\
0.3964	2.438\\
0.3963	2.4377\\
0.3903	2.4265\\
0.3903	2.4264\\
0.3917	2.4219\\
0.3921	2.4206\\
0.5433	2.0053\\
0.5581	1.9872\\
0.5584	1.9871\\
0.5591	1.9867\\
0.5597	1.9865\\
0.5604	1.9863\\
0.5611	1.9862\\
}--cycle;

\addplot[area legend, draw=mycolor1, fill=mycolor1, forget plot]
table[row sep=crcr] {%
x	y\\
0.2473	1.9033\\
0.5935	1.9792\\
0.5977	1.9809\\
0.5986	1.9813\\
0.9829	2.1666\\
1.0089	2.1926\\
1.0176	2.2031\\
1.0249	2.2141\\
1.0308	2.2254\\
1.0309	2.2257\\
1.0355	2.2371\\
1.0338	2.2413\\
1.0333	2.2426\\
0.848	2.6269\\
0.832	2.6429\\
0.8313	2.6432\\
0.831	2.6434\\
0.8304	2.6436\\
0.8297	2.6438\\
0.829	2.6439\\
0.8284	2.6439\\
0.4821	2.5681\\
0.478	2.5663\\
0.4771	2.5659\\
0.0927	2.3807\\
0.0667	2.3547\\
0.0581	2.3441\\
0.0508	2.3331\\
0.0448	2.3218\\
0.0447	2.3216\\
0.0401	2.3102\\
0.0401	2.3101\\
0.0419	2.3059\\
0.0424	2.3046\\
0.2276	1.9203\\
0.2436	1.9043\\
0.2443	1.904\\
0.2446	1.9039\\
0.2453	1.9036\\
0.246	1.9034\\
0.2466	1.9033\\
0.2473	1.9033\\
}--cycle;

\addplot[area legend, draw=mycolor1, fill=mycolor1, forget plot]
table[row sep=crcr] {%
x	y\\
-0.0469	1.7934\\
-0.0462	1.7934\\
0.2791	1.8996\\
0.2829	1.9017\\
0.2838	1.9022\\
0.6352	2.1172\\
0.6452	2.1272\\
0.6539	2.1378\\
0.6677	2.1547\\
0.6749	2.1657\\
0.6809	2.177\\
0.6855	2.1884\\
0.6856	2.1886\\
0.6889	2.2\\
0.6889	2.2001\\
0.6868	2.204\\
0.6862	2.2051\\
0.4712	2.5566\\
0.4543	2.5704\\
0.4529	2.571\\
0.4526	2.5711\\
0.452	2.5712\\
0.4506	2.5714\\
0.45	2.5713\\
0.1247	2.4651\\
0.1208	2.463\\
0.12	2.4625\\
-0.2314	2.2475\\
-0.2414	2.2375\\
-0.2501	2.2269\\
-0.2639	2.21\\
-0.2712	2.199\\
-0.2771	2.1877\\
-0.2817	2.1763\\
-0.2818	2.1761\\
-0.2852	2.1647\\
-0.2852	2.1646\\
-0.2831	2.1607\\
-0.2824	2.1596\\
-0.0674	1.8081\\
-0.0505	1.7943\\
-0.0498	1.794\\
-0.0491	1.7938\\
-0.0489	1.7937\\
-0.0482	1.7935\\
-0.0475	1.7934\\
-0.0469	1.7934\\
}--cycle;

\addplot[area legend, draw=mycolor1, fill=mycolor1, forget plot]
table[row sep=crcr] {%
x	y\\
-0.3189	1.6594\\
-0.3182	1.6594\\
-0.3176	1.6595\\
-0.0153	1.7929\\
-0.0118	1.7953\\
-0.011	1.7958\\
0.3059	2.0363\\
0.3159	2.0463\\
0.3245	2.0569\\
0.3318	2.0679\\
0.3434	2.0855\\
0.3494	2.0967\\
0.354	2.1081\\
0.3573	2.1195\\
0.3574	2.1198\\
0.3595	2.1311\\
0.3595	2.1312\\
0.3571	2.1347\\
0.3564	2.1357\\
0.1159	2.4526\\
0.0983	2.4643\\
0.0977	2.4646\\
0.097	2.4649\\
0.0963	2.465\\
0.096	2.4651\\
0.0954	2.4652\\
0.0941	2.4652\\
0.0935	2.4651\\
-0.2089	2.3316\\
-0.2123	2.3293\\
-0.2131	2.3287\\
-0.53	2.0883\\
-0.54	2.0783\\
-0.5487	2.0677\\
-0.556	2.0567\\
-0.5676	2.0391\\
-0.5735	2.0278\\
-0.5782	2.0165\\
-0.5815	2.0051\\
-0.5816	2.0048\\
-0.5837	1.9935\\
-0.5837	1.9934\\
-0.5813	1.9899\\
-0.5806	1.9888\\
-0.3401	1.6719\\
-0.3225	1.6603\\
-0.3211	1.6597\\
-0.3205	1.6596\\
-0.3202	1.6595\\
-0.3195	1.6594\\
-0.3189	1.6594\\
}--cycle;

\addplot[area legend, draw=mycolor1, fill=mycolor1, forget plot]
table[row sep=crcr] {%
x	y\\
-0.5673	1.5044\\
-0.5666	1.5044\\
-0.566	1.5045\\
-0.5655	1.5047\\
-0.2879	1.662\\
-0.2872	1.6626\\
-0.2841	1.6652\\
-0.0028	1.9268\\
0.0072	1.9368\\
0.0159	1.9474\\
0.0232	1.9584\\
0.0291	1.9697\\
0.0386	1.9877\\
0.0432	1.9991\\
0.0466	2.0105\\
0.0487	2.0217\\
0.0487	2.022\\
0.0497	2.033\\
0.0497	2.0332\\
0.047	2.0363\\
0.0462	2.0372\\
-0.2154	2.3185\\
-0.2334	2.328\\
-0.2341	2.3284\\
-0.2347	2.3286\\
-0.2354	2.3288\\
-0.2361	2.3289\\
-0.2377	2.3289\\
-0.2383	2.3288\\
-0.2388	2.3287\\
-0.5165	2.1713\\
-0.5171	2.1707\\
-0.5202	2.1681\\
-0.8015	1.9065\\
-0.8115	1.8965\\
-0.8202	1.8859\\
-0.8275	1.8749\\
-0.8334	1.8637\\
-0.8429	1.8457\\
-0.8475	1.8343\\
-0.8509	1.8229\\
-0.853	1.8116\\
-0.853	1.8113\\
-0.854	1.8003\\
-0.854	1.8002\\
-0.8513	1.797\\
-0.8505	1.7961\\
-0.5889	1.5148\\
-0.5709	1.5053\\
-0.5702	1.505\\
-0.5696	1.5047\\
-0.5682	1.5045\\
-0.568	1.5044\\
-0.5673	1.5044\\
}--cycle;

\addplot[area legend, draw=mycolor1, fill=mycolor1, forget plot]
table[row sep=crcr] {%
x	y\\
-0.7911	1.3316\\
-0.7904	1.3316\\
-0.7892	1.3318\\
-0.7887	1.332\\
-0.5371	1.51\\
-0.5265	1.5206\\
-0.5238	1.5234\\
-0.2787	1.8019\\
-0.2701	1.8125\\
-0.2628	1.8235\\
-0.2569	1.8347\\
-0.2522	1.8461\\
-0.2449	1.8643\\
-0.2415	1.8757\\
-0.2394	1.887\\
-0.2385	1.898\\
-0.2385	1.8985\\
-0.2386	1.9092\\
-0.2414	1.9119\\
-0.2423	1.9127\\
-0.5207	2.1578\\
-0.5214	2.1581\\
-0.5396	2.1655\\
-0.541	2.1659\\
-0.5416	2.166\\
-0.5433	2.166\\
-0.5439	2.1659\\
-0.5444	2.1658\\
-0.545	2.1656\\
-0.7965	1.9876\\
-0.8099	1.9742\\
-1.0549	1.6957\\
-1.0635	1.6851\\
-1.0708	1.6741\\
-1.0768	1.6629\\
-1.0814	1.6515\\
-1.0888	1.6333\\
-1.0921	1.6219\\
-1.0942	1.6106\\
-1.0952	1.5996\\
-1.0952	1.5991\\
-1.095	1.5884\\
-1.0922	1.5857\\
-1.0914	1.5849\\
-0.8129	1.3398\\
-0.8122	1.3395\\
-0.794	1.3321\\
-0.7933	1.3319\\
-0.7927	1.3317\\
-0.792	1.3316\\
-0.7911	1.3316\\
}--cycle;

\addplot[area legend, draw=mycolor1, fill=mycolor1, forget plot]
table[row sep=crcr] {%
x	y\\
-0.9903	1.1439\\
-0.9889	1.1439\\
-0.9887	1.144\\
-0.9881	1.144\\
-0.9875	1.1442\\
-0.9865	1.1446\\
-0.7619	1.3399\\
-0.7519	1.3499\\
-0.7433	1.3604\\
-0.7428	1.3611\\
-0.7404	1.364\\
-0.5318	1.6552\\
-0.5245	1.6662\\
-0.5186	1.6775\\
-0.514	1.6889\\
-0.5106	1.7002\\
-0.5053	1.7185\\
-0.5032	1.7297\\
-0.5023	1.7408\\
-0.5023	1.741\\
-0.5024	1.7517\\
-0.5024	1.752\\
-0.5036	1.7622\\
-0.5066	1.7646\\
-0.5075	1.7653\\
-0.7986	1.9739\\
-0.8	1.9745\\
-0.8182	1.9798\\
-0.8189	1.98\\
-0.8195	1.9801\\
-0.8209	1.9801\\
-0.8212	1.98\\
-0.8218	1.98\\
-0.8223	1.9798\\
-0.8229	1.9796\\
-0.8233	1.9794\\
-1.0479	1.7841\\
-1.0579	1.7741\\
-1.0665	1.7636\\
-1.067	1.7629\\
-1.0694	1.7599\\
-1.278	1.4688\\
-1.2853	1.4578\\
-1.2912	1.4465\\
-1.2959	1.4351\\
-1.2992	1.4238\\
-1.3045	1.4055\\
-1.3066	1.3943\\
-1.3076	1.3832\\
-1.3076	1.383\\
-1.3074	1.3723\\
-1.3074	1.372\\
-1.3062	1.3618\\
-1.3033	1.3594\\
-1.3024	1.3587\\
-1.0112	1.1501\\
-1.0105	1.1498\\
-1.0099	1.1495\\
-0.9916	1.1442\\
-0.991	1.144\\
-0.9903	1.1439\\
}--cycle;

\addplot[area legend, draw=mycolor1, fill=mycolor1, forget plot]
table[row sep=crcr] {%
x	y\\
-1.1634	0.9445\\
-1.1608	0.9445\\
-1.1605	0.9446\\
-1.16	0.9447\\
-1.1594	0.9449\\
-1.1589	0.9452\\
-1.1585	0.9454\\
-1.1485	0.9554\\
-0.9516	1.1646\\
-0.9429	1.1752\\
-0.9356	1.1862\\
-0.9337	1.1893\\
-0.9332	1.19\\
-0.7609	1.4898\\
-0.7549	1.5011\\
-0.7503	1.5125\\
-0.747	1.5239\\
-0.7449	1.5351\\
-0.7415	1.5532\\
-0.7405	1.5642\\
-0.7405	1.5647\\
-0.7407	1.5754\\
-0.7419	1.5856\\
-0.7419	1.5858\\
-0.744	1.5956\\
-0.745	1.5962\\
-0.748	1.5982\\
-1.0479	1.7705\\
-1.0486	1.7708\\
-1.0492	1.7711\\
-1.0499	1.7712\\
-1.0679	1.7746\\
-1.0708	1.7746\\
-1.0714	1.7744\\
-1.0724	1.774\\
-1.0729	1.7737\\
-1.0829	1.7637\\
-1.2798	1.5545\\
-1.2884	1.5439\\
-1.2957	1.5329\\
-1.2977	1.5298\\
-1.2981	1.5292\\
-1.4705	1.2293\\
-1.4764	1.2181\\
-1.481	1.2067\\
-1.4844	1.1953\\
-1.4865	1.184\\
-1.4899	1.166\\
-1.4908	1.155\\
-1.4908	1.1545\\
-1.4906	1.1438\\
-1.4895	1.1335\\
-1.4894	1.1333\\
-1.4873	1.1236\\
-1.4864	1.123\\
-1.4833	1.121\\
-1.1835	0.9486\\
-1.1828	0.9483\\
-1.1814	0.9479\\
-1.1634	0.9445\\
}--cycle;

\addplot[area legend, draw=mycolor1, fill=mycolor1, forget plot]
table[row sep=crcr] {%
x	y\\
-1.31	0.7363\\
-1.3066	0.7363\\
-1.3064	0.7364\\
-1.3054	0.7368\\
-1.3049	0.7371\\
-1.3045	0.7374\\
-1.2945	0.7474\\
-1.2859	0.758\\
-1.1169	0.978\\
-1.1096	0.989\\
-1.1036	1.0002\\
-1.102	1.0034\\
-1.1017	1.0041\\
-0.965	1.3088\\
-0.9604	1.3202\\
-0.957	1.3316\\
-0.9549	1.3428\\
-0.954	1.3539\\
-0.9525	1.3715\\
-0.9525	1.3828\\
-0.9537	1.3931\\
-0.9558	1.4028\\
-0.9558	1.403\\
-0.9588	1.4122\\
-0.962	1.4138\\
-0.9629	1.4143\\
-0.9636	1.4146\\
-1.2683	1.5513\\
-1.269	1.5515\\
-1.2696	1.5517\\
-1.2873	1.5532\\
-1.2907	1.5532\\
-1.2909	1.5531\\
-1.2919	1.5527\\
-1.2924	1.5524\\
-1.2928	1.5521\\
-1.3028	1.5421\\
-1.3114	1.5315\\
-1.4804	1.3116\\
-1.4877	1.3006\\
-1.4937	1.2893\\
-1.4956	1.2855\\
-1.6323	0.9807\\
-1.6369	0.9694\\
-1.6403	0.958\\
-1.6424	0.9467\\
-1.6433	0.9357\\
-1.6448	0.918\\
-1.6448	0.9067\\
-1.6436	0.8965\\
-1.6415	0.8867\\
-1.6415	0.8865\\
-1.6385	0.8773\\
-1.6353	0.8757\\
-1.6344	0.8753\\
-1.6337	0.8749\\
-1.329	0.7382\\
-1.3276	0.7378\\
-1.31	0.7363\\
}--cycle;

\addplot[area legend, draw=mycolor1, fill=mycolor1, forget plot]
table[row sep=crcr] {%
x	y\\
-1.4484	0.5219\\
-1.4438	0.5219\\
-1.4267	0.5222\\
-1.4265	0.5223\\
-1.426	0.5225\\
-1.4248	0.5234\\
-1.4148	0.5334\\
-1.4062	0.544\\
-1.3989	0.555\\
-1.2577	0.7826\\
-1.2518	0.7939\\
-1.2471	0.8053\\
-1.2469	0.806\\
-1.2456	0.8092\\
-1.1437	1.1151\\
-1.1403	1.1265\\
-1.1382	1.1378\\
-1.1373	1.1488\\
-1.1373	1.1605\\
-1.1375	1.1776\\
-1.1387	1.1878\\
-1.1408	1.1976\\
-1.1438	1.2067\\
-1.1439	1.2069\\
-1.1476	1.2154\\
-1.1483	1.2158\\
-1.1515	1.217\\
-1.1524	1.2174\\
-1.1531	1.2176\\
-1.4591	1.3196\\
-1.4637	1.3196\\
-1.4807	1.3193\\
-1.481	1.3192\\
-1.4815	1.319\\
-1.4827	1.3181\\
-1.4927	1.3081\\
-1.5013	1.2975\\
-1.5086	1.2865\\
-1.6498	1.0588\\
-1.6557	1.0476\\
-1.6603	1.0362\\
-1.6606	1.0355\\
-1.6619	1.0323\\
-1.7638	0.7264\\
-1.7671	0.715\\
-1.7693	0.7037\\
-1.7702	0.6927\\
-1.7702	0.681\\
-1.7699	0.6639\\
-1.7688	0.6536\\
-1.7666	0.6439\\
-1.7637	0.6348\\
-1.7636	0.6345\\
-1.7598	0.626\\
-1.7592	0.6257\\
-1.756	0.6245\\
-1.755	0.6241\\
-1.4484	0.5219\\
}--cycle;

\addplot[area legend, draw=mycolor1, fill=mycolor1, forget plot]
table[row sep=crcr] {%
x	y\\
-1.5437	0.3029\\
-1.5379	0.3029\\
-1.5215	0.3048\\
-1.5213	0.3049\\
-1.5208	0.3052\\
-1.5204	0.3055\\
-1.5101	0.3158\\
-1.5098	0.3162\\
-1.5011	0.3267\\
-1.4938	0.3377\\
-1.4879	0.349\\
-1.3742	0.5813\\
-1.3696	0.5927\\
-1.3662	0.6041\\
-1.3653	0.6073\\
-1.3651	0.608\\
-1.2967	0.9118\\
-1.2946	0.923\\
-1.2936	0.9341\\
-1.2936	0.9462\\
-1.2948	0.9565\\
-1.2967	0.9729\\
-1.2988	0.9826\\
-1.3018	0.9918\\
-1.3056	1.0003\\
-1.3057	1.0005\\
-1.3101	1.0083\\
-1.3107	1.0086\\
-1.3139	1.0095\\
-1.3149	1.0098\\
-1.6186	1.0782\\
-1.6244	1.0782\\
-1.6408	1.0763\\
-1.641	1.0762\\
-1.6415	1.076\\
-1.6522	1.0653\\
-1.6525	1.0649\\
-1.6612	1.0544\\
-1.6685	1.0434\\
-1.6744	1.0321\\
-1.7881	0.7998\\
-1.7927	0.7884\\
-1.7961	0.777\\
-1.797	0.7738\\
-1.7972	0.7731\\
-1.8656	0.4694\\
-1.8677	0.4581\\
-1.8687	0.4471\\
-1.8687	0.4349\\
-1.8675	0.4246\\
-1.8656	0.4082\\
-1.8635	0.3985\\
-1.8605	0.3893\\
-1.8567	0.3808\\
-1.8566	0.3806\\
-1.8522	0.3728\\
-1.8516	0.3725\\
-1.8484	0.3716\\
-1.8474	0.3714\\
-1.5437	0.3029\\
}--cycle;

\addplot[area legend, draw=mycolor1, fill=mycolor1, forget plot]
table[row sep=crcr] {%
x	y\\
-1.6134	0.0833\\
-1.6069	0.0833\\
-1.5913	0.0868\\
-1.5912	0.0869\\
-1.5907	0.0872\\
-1.5804	0.0975\\
-1.5801	0.0979\\
-1.5714	0.1084\\
-1.5641	0.1194\\
-1.5582	0.1307\\
-1.5536	0.1421\\
-1.4667	0.3763\\
-1.4634	0.3877\\
-1.4607	0.4021\\
-1.4606	0.4028\\
-1.4241	0.7012\\
-1.4232	0.7122\\
-1.4232	0.7251\\
-1.4244	0.7353\\
-1.4265	0.7451\\
-1.4299	0.7606\\
-1.4329	0.7698\\
-1.4366	0.7783\\
-1.441	0.7861\\
-1.4411	0.7863\\
-1.4461	0.7934\\
-1.4468	0.7937\\
-1.45	0.7942\\
-1.4509	0.7944\\
-1.7493	0.8309\\
-1.7559	0.8309\\
-1.7714	0.8275\\
-1.7716	0.8274\\
-1.772	0.8271\\
-1.7827	0.8164\\
-1.7913	0.8058\\
-1.7986	0.7948\\
-1.8046	0.7835\\
-1.8092	0.7722\\
-1.896	0.538\\
-1.8994	0.5266\\
-1.9015	0.5153\\
-1.902	0.5122\\
-1.9021	0.5115\\
-1.9386	0.213\\
-1.9396	0.202\\
-1.9396	0.1892\\
-1.9384	0.1789\\
-1.9363	0.1692\\
-1.9329	0.1536\\
-1.9299	0.1444\\
-1.9261	0.1359\\
-1.9217	0.1281\\
-1.9216	0.1279\\
-1.9166	0.1209\\
-1.9159	0.1205\\
-1.9128	0.12\\
-1.9119	0.1198\\
-1.6134	0.0833\\
}--cycle;

\addplot[area legend, draw=mycolor1, fill=mycolor1, forget plot]
table[row sep=crcr] {%
x	y\\
-1.6593	-0.1343\\
-1.6521	-0.1343\\
-1.6375	-0.1295\\
-1.6373	-0.1294\\
-1.6266	-0.1187\\
-1.618	-0.1081\\
-1.6107	-0.0972\\
-1.6048	-0.0859\\
-1.6002	-0.0745\\
-1.5968	-0.0631\\
-1.5359	0.1703\\
-1.5338	0.1815\\
-1.5329	0.1926\\
-1.5327	0.1956\\
-1.5326	0.1963\\
-1.5263	0.4866\\
-1.5263	0.5001\\
-1.5275	0.5104\\
-1.5296	0.5201\\
-1.5326	0.5293\\
-1.5374	0.5439\\
-1.5411	0.5524\\
-1.5456	0.5602\\
-1.5505	0.5673\\
-1.5507	0.5674\\
-1.5562	0.5738\\
-1.5568	0.5741\\
-1.5599	0.5743\\
-1.5608	0.5744\\
-1.8511	0.5807\\
-1.8582	0.5807\\
-1.8729	0.5759\\
-1.8837	0.5651\\
-1.8923	0.5545\\
-1.8996	0.5435\\
-1.9056	0.5322\\
-1.9102	0.5208\\
-1.9135	0.5095\\
-1.9744	0.2761\\
-1.9765	0.2648\\
-1.9774	0.2538\\
-1.9777	0.2507\\
-1.9777	0.25\\
-1.984	-0.0402\\
-1.984	-0.0538\\
-1.9828	-0.064\\
-1.9807	-0.0738\\
-1.9777	-0.0829\\
-1.9729	-0.0975\\
-1.9692	-0.106\\
-1.9648	-0.1139\\
-1.9598	-0.1209\\
-1.9597	-0.1211\\
-1.9542	-0.1274\\
-1.9535	-0.1277\\
-1.9504	-0.128\\
-1.9495	-0.128\\
-1.6593	-0.1343\\
}--cycle;

\addplot[area legend, draw=mycolor1, fill=mycolor1, forget plot]
table[row sep=crcr] {%
x	y\\
-1.9657	-0.3699\\
-1.9571	-0.3699\\
-1.9542	-0.3698\\
-1.6747	-0.3479\\
-1.6611	-0.3419\\
-1.6506	-0.3314\\
-1.6419	-0.3208\\
-1.6347	-0.3098\\
-1.6287	-0.2986\\
-1.6241	-0.2872\\
-1.6207	-0.2758\\
-1.6186	-0.2645\\
-1.5827	-0.0344\\
-1.5817	-0.0234\\
-1.5817	-0.0091\\
-1.5818	-0.0061\\
-1.5818	-0.0055\\
-1.6038	0.274\\
-1.6049	0.2843\\
-1.6071	0.294\\
-1.61	0.3032\\
-1.6138	0.3117\\
-1.6198	0.3253\\
-1.6242	0.3331\\
-1.6292	0.3401\\
-1.6347	0.3465\\
-1.6348	0.3466\\
-1.6407	0.3522\\
-1.6414	0.3525\\
-1.6499	0.3525\\
-1.6529	0.3524\\
-1.9324	0.3304\\
-1.9459	0.3244\\
-1.9463	0.3241\\
-1.9465	0.324\\
-1.9565	0.314\\
-1.9651	0.3034\\
-1.9724	0.2924\\
-1.9783	0.2811\\
-1.9829	0.2697\\
-1.9863	0.2583\\
-1.9884	0.2471\\
-2.0244	0.017\\
-2.0253	0.006\\
-2.0253	-0.0083\\
-2.0252	-0.0113\\
-2.0252	-0.012\\
-2.0033	-0.2914\\
-2.0021	-0.3017\\
-2	-0.3114\\
-1.997	-0.3206\\
-1.9933	-0.3291\\
-1.9873	-0.3427\\
-1.9828	-0.3505\\
-1.9778	-0.3576\\
-1.9724	-0.3639\\
-1.9722	-0.364\\
-1.9664	-0.3696\\
-1.9657	-0.3699\\
}--cycle;

\addplot[area legend, draw=mycolor1, fill=mycolor1, forget plot]
table[row sep=crcr] {%
x	y\\
-1.9547	-0.6037\\
-1.9462	-0.6037\\
-1.9454	-0.6036\\
-1.9425	-0.6033\\
-1.6761	-0.5552\\
-1.6636	-0.5482\\
-1.6633	-0.5478\\
-1.6533	-0.5378\\
-1.6531	-0.5377\\
-1.6445	-0.5271\\
-1.6372	-0.5161\\
-1.6313	-0.5049\\
-1.6267	-0.4935\\
-1.6233	-0.4821\\
-1.6212	-0.4708\\
-1.6203	-0.4598\\
-1.6079	-0.2352\\
-1.6079	-0.2202\\
-1.609	-0.21\\
-1.6091	-0.2094\\
-1.6095	-0.2065\\
-1.6575	0.0599\\
-1.6596	0.0696\\
-1.6626	0.0788\\
-1.6664	0.0873\\
-1.6708	0.0951\\
-1.6779	0.1076\\
-1.6828	0.1147\\
-1.6883	0.121\\
-1.6942	0.1265\\
-1.6943	0.1267\\
-1.7005	0.1314\\
-1.709	0.1314\\
-1.7099	0.1313\\
-1.7127	0.1309\\
-1.9791	0.0829\\
-1.9916	0.0758\\
-1.992	0.0755\\
-2.0021	0.0654\\
-2.0107	0.0548\\
-2.018	0.0438\\
-2.0239	0.0325\\
-2.0286	0.0211\\
-2.0319	0.0098\\
-2.034	-0.0015\\
-2.035	-0.0125\\
-2.0474	-0.2371\\
-2.0474	-0.2521\\
-2.0462	-0.2623\\
-2.0457	-0.2658\\
-1.9977	-0.5322\\
-1.9956	-0.542\\
-1.9926	-0.5511\\
-1.9889	-0.5596\\
-1.9844	-0.5674\\
-1.9774	-0.5799\\
-1.9724	-0.587\\
-1.9669	-0.5933\\
-1.961	-0.5989\\
-1.9609	-0.599\\
-1.9547	-0.6037\\
}--cycle;

\addplot[area legend, draw=mycolor1, fill=mycolor1, forget plot]
table[row sep=crcr] {%
x	y\\
-1.921	-0.8273\\
-1.912	-0.8273\\
-1.9112	-0.8271\\
-1.9085	-0.8265\\
-1.6572	-0.7546\\
-1.6459	-0.7466\\
-1.6359	-0.7366\\
-1.6272	-0.7261\\
-1.6271	-0.7259\\
-1.6198	-0.7149\\
-1.6139	-0.7037\\
-1.6093	-0.6923\\
-1.6059	-0.6809\\
-1.6038	-0.6696\\
-1.6029	-0.6586\\
-1.6029	-0.6429\\
-1.6126	-0.4259\\
-1.6138	-0.4157\\
-1.6159	-0.4059\\
-1.6161	-0.4053\\
-1.6167	-0.4026\\
-1.6885	-0.1513\\
-1.6915	-0.1422\\
-1.6952	-0.1337\\
-1.6996	-0.1259\\
-1.7046	-0.1188\\
-1.7126	-0.1075\\
-1.7181	-0.1011\\
-1.724	-0.0956\\
-1.7302	-0.0908\\
-1.7303	-0.0907\\
-1.7367	-0.0868\\
-1.7457	-0.0868\\
-1.7465	-0.0869\\
-1.7492	-0.0876\\
-2.0005	-0.1594\\
-2.0118	-0.1674\\
-2.0218	-0.1774\\
-2.0305	-0.188\\
-2.0306	-0.1881\\
-2.0378	-0.1991\\
-2.0438	-0.2104\\
-2.0484	-0.2218\\
-2.0517	-0.2331\\
-2.0539	-0.2444\\
-2.0548	-0.2554\\
-2.0548	-0.2712\\
-2.0451	-0.4881\\
-2.0439	-0.4984\\
-2.0418	-0.5081\\
-2.0416	-0.5087\\
-2.041	-0.5114\\
-1.9692	-0.7627\\
-1.9662	-0.7719\\
-1.9625	-0.7804\\
-1.958	-0.7882\\
-1.953	-0.7953\\
-1.945	-0.8066\\
-1.9396	-0.8129\\
-1.9337	-0.8184\\
-1.9275	-0.8232\\
-1.9274	-0.8233\\
-1.921	-0.8273\\
}--cycle;

\addplot[area legend, draw=mycolor1, fill=mycolor1, forget plot]
table[row sep=crcr] {%
x	y\\
-1.867	-1.0388\\
-1.857	-1.0388\\
-1.8563	-1.0385\\
-1.8538	-1.0377\\
-1.6192	-0.9445\\
-1.6091	-0.9357\\
-1.5991	-0.9257\\
-1.5905	-0.9151\\
-1.5832	-0.9041\\
-1.5831	-0.904\\
-1.5772	-0.8927\\
-1.5725	-0.8813\\
-1.5692	-0.8699\\
-1.5671	-0.8586\\
-1.5671	-0.8315\\
-1.5683	-0.8212\\
-1.5985	-0.6138\\
-1.6006	-0.604\\
-1.6036	-0.5949\\
-1.6045	-0.5923\\
-1.6047	-0.5918\\
-1.6979	-0.3572\\
-1.7017	-0.3487\\
-1.7061	-0.3409\\
-1.7111	-0.3338\\
-1.7165	-0.3275\\
-1.7253	-0.3174\\
-1.7312	-0.3119\\
-1.7374	-0.3071\\
-1.7438	-0.3031\\
-1.7439	-0.3031\\
-1.7505	-0.2999\\
-1.7604	-0.2999\\
-1.7612	-0.3001\\
-1.7637	-0.301\\
-1.9982	-0.3942\\
-2.0083	-0.403\\
-2.0183	-0.413\\
-2.027	-0.4236\\
-2.0343	-0.4345\\
-2.0344	-0.4347\\
-2.0403	-0.446\\
-2.0449	-0.4574\\
-2.0483	-0.4687\\
-2.0504	-0.48\\
-2.0504	-0.5072\\
-2.0492	-0.5174\\
-2.0189	-0.7249\\
-2.0168	-0.7346\\
-2.0138	-0.7438\\
-2.013	-0.7463\\
-2.0128	-0.7469\\
-1.9195	-0.9814\\
-1.9158	-0.9899\\
-1.9114	-0.9977\\
-1.9064	-1.0048\\
-1.9009	-1.0111\\
-1.8921	-1.0212\\
-1.8862	-1.0268\\
-1.8801	-1.0315\\
-1.8737	-1.0355\\
-1.8735	-1.0356\\
-1.867	-1.0388\\
}--cycle;

\addplot[area legend, draw=mycolor1, fill=mycolor1, forget plot]
table[row sep=crcr] {%
x	y\\
-1.7959	-1.2368\\
-1.7848	-1.2368\\
-1.7824	-1.2358\\
-1.7817	-1.2354\\
-1.5654	-1.1233\\
-1.5554	-1.1133\\
-1.5465	-1.1038\\
-1.5379	-1.0933\\
-1.5306	-1.0823\\
-1.5246	-1.071\\
-1.5246	-1.0709\\
-1.52	-1.0595\\
-1.5166	-1.0481\\
-1.5145	-1.0368\\
-1.5145	-0.999\\
-1.5166	-0.9893\\
-1.5657	-0.7929\\
-1.5687	-0.7837\\
-1.5724	-0.7752\\
-1.5727	-0.7747\\
-1.5737	-0.7723\\
-1.6859	-0.556\\
-1.6903	-0.5482\\
-1.6953	-0.5411\\
-1.7008	-0.5348\\
-1.7067	-0.5292\\
-1.7161	-0.5204\\
-1.7223	-0.5156\\
-1.7287	-0.5116\\
-1.7352	-0.5084\\
-1.7354	-0.5084\\
-1.7419	-0.5059\\
-1.7531	-0.5059\\
-1.7554	-0.507\\
-1.7561	-0.5073\\
-1.9725	-0.6195\\
-1.9825	-0.6295\\
-1.9914	-0.6389\\
-2	-0.6495\\
-2.0073	-0.6605\\
-2.0132	-0.6717\\
-2.0133	-0.6719\\
-2.0179	-0.6833\\
-2.0213	-0.6947\\
-2.0234	-0.7059\\
-2.0234	-0.7437\\
-2.0213	-0.7535\\
-1.9722	-0.9499\\
-1.9692	-0.959\\
-1.9654	-0.9675\\
-1.9652	-0.968\\
-1.9641	-0.9704\\
-1.8519	-1.1868\\
-1.8475	-1.1946\\
-1.8425	-1.2017\\
-1.8371	-1.208\\
-1.8312	-1.2135\\
-1.8218	-1.2224\\
-1.8156	-1.2271\\
-1.8092	-1.2311\\
-1.8027	-1.2343\\
-1.8025	-1.2344\\
-1.7959	-1.2368\\
}--cycle;

\addplot[area legend, draw=mycolor1, fill=mycolor1, forget plot]
table[row sep=crcr] {%
x	y\\
-1.714	-1.4201\\
-1.6963	-1.4201\\
-1.6935	-1.4185\\
-1.4964	-1.2898\\
-1.4864	-1.2798\\
-1.4778	-1.2693\\
-1.4702	-1.2594\\
-1.4629	-1.2484\\
-1.4569	-1.2371\\
-1.4523	-1.2257\\
-1.4523	-1.2256\\
-1.4489	-1.2142\\
-1.4468	-1.2029\\
-1.4468	-1.1631\\
-1.4489	-1.1533\\
-1.4519	-1.1442\\
-1.518	-0.9602\\
-1.5218	-0.9517\\
-1.5262	-0.9439\\
-1.5274	-0.9417\\
-1.5277	-0.9412\\
-1.6563	-0.7442\\
-1.6613	-0.7371\\
-1.6668	-0.7308\\
-1.6727	-0.7252\\
-1.6789	-0.7205\\
-1.6888	-0.7129\\
-1.6952	-0.7089\\
-1.7017	-0.7057\\
-1.7083	-0.7032\\
-1.7262	-0.7032\\
-1.7268	-0.7036\\
-1.729	-0.7048\\
-1.926	-0.8335\\
-1.936	-0.8435\\
-1.9447	-0.854\\
-1.9523	-0.8639\\
-1.9596	-0.8749\\
-1.9655	-0.8862\\
-1.9701	-0.8976\\
-1.9702	-0.8977\\
-1.9735	-0.9091\\
-1.9756	-0.9204\\
-1.9756	-0.9603\\
-1.9735	-0.97\\
-1.9705	-0.9791\\
-1.9044	-1.1631\\
-1.9007	-1.1716\\
-1.8963	-1.1794\\
-1.895	-1.1816\\
-1.8947	-1.1821\\
-1.7661	-1.3791\\
-1.7611	-1.3862\\
-1.7556	-1.3925\\
-1.7498	-1.3981\\
-1.7436	-1.4028\\
-1.7337	-1.4104\\
-1.7273	-1.4144\\
-1.7208	-1.4176\\
-1.7142	-1.4201\\
-1.714	-1.4201\\
}--cycle;

\addplot[area legend, draw=mycolor1, fill=mycolor1, forget plot]
table[row sep=crcr] {%
x	y\\
-1.6179	-1.5877\\
-1.5936	-1.5877\\
-1.593	-1.5873\\
-1.591	-1.5859\\
-1.4141	-1.4433\\
-1.4041	-1.4333\\
-1.3955	-1.4227\\
-1.3882	-1.4117\\
-1.3818	-1.4015\\
-1.3759	-1.3902\\
-1.3713	-1.3788\\
-1.3679	-1.3674\\
-1.3679	-1.3673\\
-1.3658	-1.356\\
-1.3658	-1.3148\\
-1.3679	-1.305\\
-1.3709	-1.2959\\
-1.3746	-1.2874\\
-1.4559	-1.117\\
-1.4603	-1.1092\\
-1.4653	-1.1021\\
-1.4656	-1.1017\\
-1.467	-1.0997\\
-1.6096	-0.9228\\
-1.6151	-0.9165\\
-1.6209	-0.9109\\
-1.6271	-0.9062\\
-1.6335	-0.9022\\
-1.6438	-0.8958\\
-1.6503	-0.8926\\
-1.6569	-0.8902\\
-1.657	-0.8901\\
-1.6813	-0.8901\\
-1.6819	-0.8906\\
-1.6838	-0.892\\
-1.8608	-1.0346\\
-1.8708	-1.0446\\
-1.8794	-1.0551\\
-1.8867	-1.0661\\
-1.8931	-1.0764\\
-1.899	-1.0876\\
-1.9036	-1.099\\
-1.907	-1.1104\\
-1.907	-1.1106\\
-1.9091	-1.1218\\
-1.9091	-1.1631\\
-1.907	-1.1728\\
-1.904	-1.182\\
-1.9003	-1.1905\\
-1.819	-1.3609\\
-1.8146	-1.3687\\
-1.8096	-1.3758\\
-1.8093	-1.3762\\
-1.8079	-1.3781\\
-1.6653	-1.5551\\
-1.6598	-1.5614\\
-1.6539	-1.5669\\
-1.6478	-1.5717\\
-1.6414	-1.5757\\
-1.6311	-1.582\\
-1.6246	-1.5852\\
-1.618	-1.5877\\
-1.6179	-1.5877\\
}--cycle;

\addplot[area legend, draw=mycolor1, fill=mycolor1, forget plot]
table[row sep=crcr] {%
x	y\\
-1.5094	-1.7388\\
-1.4787	-1.7388\\
-1.4782	-1.7383\\
-1.4765	-1.7368\\
-1.3202	-1.5827\\
-1.3102	-1.5727\\
-1.3016	-1.5622\\
-1.2943	-1.5512\\
-1.2884	-1.5399\\
-1.2832	-1.5295\\
-1.2786	-1.5181\\
-1.2753	-1.5067\\
-1.2732	-1.4954\\
-1.2731	-1.4953\\
-1.2731	-1.4533\\
-1.2753	-1.4436\\
-1.2782	-1.4345\\
-1.282	-1.426\\
-1.2864	-1.4181\\
-1.3809	-1.2623\\
-1.3859	-1.2552\\
-1.3914	-1.2489\\
-1.3929	-1.2471\\
-1.3933	-1.2467\\
-1.5473	-1.0905\\
-1.5532	-1.0849\\
-1.5594	-1.0802\\
-1.5658	-1.0762\\
-1.5723	-1.073\\
-1.5827	-1.0679\\
-1.5893	-1.0654\\
-1.6202	-1.0654\\
-1.6207	-1.0659\\
-1.6224	-1.0674\\
-1.7787	-1.2215\\
-1.7887	-1.2315\\
-1.7973	-1.2421\\
-1.8046	-1.2531\\
-1.8106	-1.2643\\
-1.8157	-1.2748\\
-1.8203	-1.2862\\
-1.8236	-1.2975\\
-1.8257	-1.3088\\
-1.8258	-1.309\\
-1.8258	-1.3509\\
-1.8236	-1.3606\\
-1.8207	-1.3698\\
-1.8169	-1.3783\\
-1.8125	-1.3861\\
-1.718	-1.542\\
-1.713	-1.549\\
-1.7075	-1.5554\\
-1.706	-1.5571\\
-1.7057	-1.5575\\
-1.5516	-1.7137\\
-1.5457	-1.7193\\
-1.5395	-1.724\\
-1.5331	-1.728\\
-1.5266	-1.7312\\
-1.5162	-1.7363\\
-1.5096	-1.7388\\
-1.5094	-1.7388\\
}--cycle;

\addplot[area legend, draw=mycolor1, fill=mycolor1, forget plot]
table[row sep=crcr] {%
x	y\\
-1.3908	-1.8732\\
-1.3538	-1.8732\\
-1.3438	-1.8632\\
-1.3434	-1.8627\\
-1.3419	-1.861\\
-1.2066	-1.6979\\
-1.198	-1.6874\\
-1.1907	-1.6764\\
-1.1847	-1.6651\\
-1.1801	-1.6537\\
-1.1762	-1.6432\\
-1.1729	-1.6318\\
-1.1707	-1.6205\\
-1.1707	-1.5779\\
-1.1729	-1.5682\\
-1.1758	-1.559\\
-1.1796	-1.5505\\
-1.184	-1.5427\\
-1.189	-1.5356\\
-1.2948	-1.395\\
-1.3003	-1.3886\\
-1.3062	-1.3831\\
-1.3065	-1.3828\\
-1.3082	-1.3812\\
-1.4713	-1.246\\
-1.4774	-1.2412\\
-1.4838	-1.2372\\
-1.4904	-1.234\\
-1.4969	-1.2316\\
-1.5075	-1.2277\\
-1.5446	-1.2277\\
-1.5546	-1.2377\\
-1.555	-1.2382\\
-1.5565	-1.2398\\
-1.6918	-1.4029\\
-1.7004	-1.4135\\
-1.7077	-1.4245\\
-1.7137	-1.4357\\
-1.7183	-1.4471\\
-1.7222	-1.4577\\
-1.7255	-1.469\\
-1.7276	-1.4803\\
-1.7276	-1.5229\\
-1.7255	-1.5327\\
-1.7225	-1.5418\\
-1.7188	-1.5503\\
-1.7144	-1.5581\\
-1.7094	-1.5652\\
-1.6036	-1.7059\\
-1.5981	-1.7122\\
-1.5922	-1.7178\\
-1.5919	-1.7181\\
-1.5902	-1.7196\\
-1.4271	-1.8549\\
-1.421	-1.8596\\
-1.4146	-1.8636\\
-1.408	-1.8668\\
-1.4014	-1.8693\\
-1.3909	-1.8732\\
-1.3908	-1.8732\\
}--cycle;

\addplot[area legend, draw=mycolor1, fill=mycolor1, forget plot]
table[row sep=crcr] {%
x	y\\
-1.264	-1.9908\\
-1.2209	-1.9908\\
-1.2109	-1.9808\\
-1.2023	-1.9703\\
-1.201	-1.9685\\
-1.2006	-1.968\\
-1.0863	-1.7983\\
-1.0791	-1.7873\\
-1.0731	-1.776\\
-1.0685	-1.7646\\
-1.0651	-1.7532\\
-1.0624	-1.7428\\
-1.0603	-1.7315\\
-1.0603	-1.6877\\
-1.0624	-1.678\\
-1.0654	-1.6688\\
-1.0691	-1.6603\\
-1.0736	-1.6525\\
-1.0786	-1.6454\\
-1.084	-1.6391\\
-1.1993	-1.5142\\
-1.2051	-1.5086\\
-1.2113	-1.5039\\
-1.2117	-1.5036\\
-1.2134	-1.5023\\
-1.3832	-1.388\\
-1.3896	-1.3841\\
-1.3961	-1.3809\\
-1.4027	-1.3784\\
-1.4132	-1.3757\\
-1.4562	-1.3757\\
-1.4662	-1.3857\\
-1.4749	-1.3963\\
-1.4762	-1.398\\
-1.4765	-1.3985\\
-1.5908	-1.5682\\
-1.5981	-1.5792\\
-1.604	-1.5905\\
-1.6086	-1.6019\\
-1.612	-1.6133\\
-1.6147	-1.6238\\
-1.6168	-1.635\\
-1.6168	-1.6788\\
-1.6147	-1.6885\\
-1.6117	-1.6977\\
-1.608	-1.7062\\
-1.6036	-1.714\\
-1.5986	-1.7211\\
-1.5931	-1.7274\\
-1.4779	-1.8523\\
-1.472	-1.8579\\
-1.4658	-1.8626\\
-1.4654	-1.8629\\
-1.4637	-1.8642\\
-1.294	-1.9785\\
-1.2876	-1.9824\\
-1.281	-1.9857\\
-1.2745	-1.9881\\
-1.264	-1.9908\\
}--cycle;

\addplot[area legend, draw=mycolor1, fill=mycolor1, forget plot]
table[row sep=crcr] {%
x	y\\
-1.1307	-2.0916\\
-1.0819	-2.0916\\
-1.0719	-2.0816\\
-1.0633	-2.071\\
-1.056	-2.0601\\
-1.0557	-2.0595\\
-1.0546	-2.0577\\
-0.9612	-1.8836\\
-0.9553	-1.8724\\
-0.9506	-1.861\\
-0.9473	-1.8496\\
-0.9452	-1.8383\\
-0.9436	-1.828\\
-0.9436	-1.7826\\
-0.9457	-1.7729\\
-0.9487	-1.7637\\
-0.9524	-1.7552\\
-0.9569	-1.7474\\
-0.9618	-1.7403\\
-0.9673	-1.734\\
-0.9732	-1.7285\\
-1.0959	-1.6195\\
-1.1021	-1.6148\\
-1.1085	-1.6108\\
-1.1103	-1.6097\\
-1.1107	-1.6095\\
-1.2848	-1.5161\\
-1.2913	-1.5129\\
-1.2979	-1.5104\\
-1.3083	-1.5088\\
-1.357	-1.5088\\
-1.367	-1.5188\\
-1.3756	-1.5294\\
-1.3829	-1.5404\\
-1.3833	-1.5409\\
-1.3843	-1.5427\\
-1.4777	-1.7168\\
-1.4837	-1.728\\
-1.4883	-1.7394\\
-1.4916	-1.7508\\
-1.4938	-1.7621\\
-1.4954	-1.7724\\
-1.4954	-1.8177\\
-1.4953	-1.8178\\
-1.4932	-1.8276\\
-1.4902	-1.8367\\
-1.4865	-1.8452\\
-1.4821	-1.853\\
-1.4771	-1.8601\\
-1.4716	-1.8664\\
-1.4657	-1.872\\
-1.343	-1.9809\\
-1.3368	-1.9857\\
-1.3304	-1.9896\\
-1.3286	-1.9907\\
-1.3282	-1.991\\
-1.1541	-2.0844\\
-1.1476	-2.0876\\
-1.141	-2.09\\
-1.1307	-2.0916\\
}--cycle;

\addplot[area legend, draw=mycolor1, fill=mycolor1, forget plot]
table[row sep=crcr] {%
x	y\\
-0.9929	-2.1756\\
-0.9388	-2.1756\\
-0.9288	-2.1656\\
-0.9202	-2.155\\
-0.9129	-2.144\\
-0.907	-2.1328\\
-0.9061	-2.1309\\
-0.9058	-2.1304\\
-0.8329	-1.9541\\
-0.8283	-1.9427\\
-0.8249	-1.9314\\
-0.8228	-1.9201\\
-0.8222	-1.91\\
-0.8222	-1.8626\\
-0.8244	-1.8529\\
-0.8244	-1.8528\\
-0.8274	-1.8436\\
-0.8311	-1.8351\\
-0.8355	-1.8273\\
-0.8405	-1.8202\\
-0.846	-1.8139\\
-0.8519	-1.8084\\
-0.8581	-1.8036\\
-0.9865	-1.7108\\
-0.9929	-1.7068\\
-0.9994	-1.7036\\
-1.0013	-1.7027\\
-1.0017	-1.7025\\
-1.1779	-1.6296\\
-1.1845	-1.6271\\
-1.1946	-1.6266\\
-1.2487	-1.6266\\
-1.2587	-1.6366\\
-1.2674	-1.6472\\
-1.2746	-1.6582\\
-1.2806	-1.6694\\
-1.2814	-1.6713\\
-1.2817	-1.6718\\
-1.3547	-1.8481\\
-1.3593	-1.8595\\
-1.3626	-1.8709\\
-1.3647	-1.8821\\
-1.3653	-1.8922\\
-1.3653	-1.9396\\
-1.3632	-1.9493\\
-1.3631	-1.9494\\
-1.3602	-1.9586\\
-1.3564	-1.9671\\
-1.352	-1.9749\\
-1.347	-1.982\\
-1.3415	-1.9883\\
-1.3356	-1.9938\\
-1.3295	-1.9986\\
-1.201	-2.0914\\
-1.1946	-2.0954\\
-1.1881	-2.0986\\
-1.1859	-2.0997\\
-1.0096	-2.1726\\
-1.003	-2.1751\\
-0.9929	-2.1756\\
}--cycle;

\addplot[area legend, draw=mycolor1, fill=mycolor1, forget plot]
table[row sep=crcr] {%
x	y\\
-0.8623	-2.2435\\
-0.806	-2.2435\\
-0.7962	-2.243\\
-0.7961	-2.243\\
-0.7913	-2.2407\\
-0.7813	-2.2307\\
-0.7727	-2.2201\\
-0.7654	-2.2091\\
-0.7594	-2.1979\\
-0.7548	-2.1865\\
-0.7542	-2.1846\\
-0.754	-2.1841\\
-0.7009	-2.0077\\
-0.6975	-1.9963\\
-0.6975	-1.9375\\
-0.698	-1.9278\\
-0.7001	-1.918\\
-0.7031	-1.9089\\
-0.7031	-1.9087\\
-0.7069	-1.9002\\
-0.7113	-1.8924\\
-0.7163	-1.8854\\
-0.7217	-1.879\\
-0.7276	-1.8735\\
-0.7338	-1.8687\\
-0.7402	-1.8648\\
-0.8725	-1.7879\\
-0.8791	-1.7847\\
-0.8857	-1.7823\\
-0.8875	-1.7816\\
-0.8879	-1.7815\\
-1.0643	-1.7284\\
-1.1206	-1.7284\\
-1.1303	-1.7288\\
-1.1304	-1.7289\\
-1.1352	-1.7311\\
-1.1452	-1.7411\\
-1.1539	-1.7517\\
-1.1612	-1.7627\\
-1.1671	-1.774\\
-1.1717	-1.7854\\
-1.1724	-1.7872\\
-1.1726	-1.7877\\
-1.2257	-1.9641\\
-1.229	-1.9755\\
-1.229	-2.0343\\
-1.2286	-2.0441\\
-1.2265	-2.0538\\
-1.2235	-2.063\\
-1.2234	-2.0631\\
-1.2197	-2.0716\\
-1.2153	-2.0794\\
-1.2103	-2.0865\\
-1.2048	-2.0928\\
-1.1989	-2.0983\\
-1.1928	-2.1031\\
-1.1864	-2.1071\\
-1.054	-2.1839\\
-1.0475	-2.1871\\
-1.0409	-2.1896\\
-1.0391	-2.1902\\
-1.0387	-2.1904\\
-0.8623	-2.2435\\
}--cycle;

\addplot[area legend, draw=mycolor1, fill=mycolor1, forget plot]
table[row sep=crcr] {%
x	y\\
-0.7226	-2.2956\\
-0.664	-2.2956\\
-0.6546	-2.2942\\
-0.6498	-2.2919\\
-0.6497	-2.2919\\
-0.6453	-2.2892\\
-0.6353	-2.2792\\
-0.6267	-2.2687\\
-0.6194	-2.2577\\
-0.6135	-2.2464\\
-0.6088	-2.235\\
-0.6055	-2.2236\\
-0.6051	-2.2218\\
-0.6049	-2.2212\\
-0.5709	-2.0467\\
-0.5709	-1.9779\\
-0.5723	-1.9686\\
-0.5752	-1.9595\\
-0.579	-1.9509\\
-0.579	-1.9508\\
-0.5835	-1.943\\
-0.5884	-1.9359\\
-0.5939	-1.9296\\
-0.5998	-1.9241\\
-0.606	-1.9193\\
-0.6124	-1.9153\\
-0.6189	-1.9121\\
-0.7535	-1.8511\\
-0.7601	-1.8487\\
-0.7619	-1.8482\\
-0.7623	-1.8481\\
-0.9369	-1.8141\\
-0.9955	-1.8141\\
-1.0048	-1.8154\\
-1.0096	-1.8177\\
-1.0097	-1.8178\\
-1.0141	-1.8204\\
-1.0241	-1.8304\\
-1.0328	-1.841\\
-1.04	-1.852\\
-1.046	-1.8633\\
-1.0506	-1.8746\\
-1.0539	-1.886\\
-1.0544	-1.8879\\
-1.0545	-1.8884\\
-1.0886	-2.063\\
-1.0886	-2.1317\\
-1.0872	-2.1411\\
-1.0842	-2.1502\\
-1.0805	-2.1587\\
-1.0804	-2.1588\\
-1.076	-2.1666\\
-1.071	-2.1737\\
-1.0655	-2.18\\
-1.0596	-2.1856\\
-1.0535	-2.1903\\
-1.0471	-2.1943\\
-1.0405	-2.1975\\
-0.906	-2.2586\\
-0.8994	-2.261\\
-0.8976	-2.2615\\
-0.8972	-2.2616\\
-0.7226	-2.2956\\
}--cycle;

\addplot[area legend, draw=mycolor1, fill=mycolor1, forget plot]
table[row sep=crcr] {%
x	y\\
-0.5813	-2.332\\
-0.5177	-2.332\\
-0.5088	-2.3298\\
-0.5045	-2.3271\\
-0.5044	-2.327\\
-0.5004	-2.3241\\
-0.4904	-2.3141\\
-0.4818	-2.3035\\
-0.4745	-2.2925\\
-0.4685	-2.2812\\
-0.4639	-2.2698\\
-0.4606	-2.2585\\
-0.4602	-2.2561\\
-0.4444	-2.0851\\
-0.4444	-2.0139\\
-0.4466	-2.0051\\
-0.4496	-1.996\\
-0.4533	-1.9875\\
-0.4577	-1.9796\\
-0.4578	-1.9795\\
-0.4628	-1.9725\\
-0.4683	-1.9661\\
-0.4741	-1.9606\\
-0.4803	-1.9558\\
-0.4867	-1.9519\\
-0.4933	-1.9487\\
-0.4998	-1.9462\\
-0.635	-1.9005\\
-0.6368	-1.9003\\
-0.6372	-1.9002\\
-0.8082	-1.8844\\
-0.8719	-1.8844\\
-0.8807	-1.8866\\
-0.8851	-1.8892\\
-0.8852	-1.8893\\
-0.8891	-1.8923\\
-0.8991	-1.9023\\
-0.9078	-1.9128\\
-0.9151	-1.9238\\
-0.921	-1.9351\\
-0.9256	-1.9465\\
-0.929	-1.9579\\
-0.9292	-1.9597\\
-0.9293	-1.9602\\
-0.9452	-2.1312\\
-0.9452	-2.2024\\
-0.9429	-2.2112\\
-0.94	-2.2204\\
-0.9362	-2.2289\\
-0.9318	-2.2367\\
-0.9317	-2.2368\\
-0.9267	-2.2439\\
-0.9213	-2.2502\\
-0.9154	-2.2557\\
-0.9092	-2.2605\\
-0.9028	-2.2645\\
-0.8963	-2.2677\\
-0.8897	-2.2701\\
-0.7545	-2.3158\\
-0.7527	-2.316\\
-0.7523	-2.3161\\
-0.5813	-2.332\\
}--cycle;

\addplot[area legend, draw=mycolor1, fill=mycolor1, forget plot]
table[row sep=crcr] {%
x	y\\
-0.6145	-2.3545\\
-0.5441	-2.3545\\
-0.3782	-2.3533\\
-0.37	-2.3503\\
-0.3656	-2.3477\\
-0.3616	-2.3447\\
-0.3615	-2.3446\\
-0.358	-2.3414\\
-0.348	-2.3314\\
-0.3394	-2.3208\\
-0.3321	-2.3098\\
-0.3261	-2.2986\\
-0.3215	-2.2872\\
-0.3182	-2.2758\\
-0.3181	-2.274\\
-0.3181	-2.1997\\
-0.3193	-2.0339\\
-0.3223	-2.0248\\
-0.3253	-2.0165\\
-0.329	-2.008\\
-0.3334	-2.0002\\
-0.3384	-1.9931\\
-0.3385	-1.993\\
-0.344	-1.9867\\
-0.3499	-1.9811\\
-0.3561	-1.9764\\
-0.3625	-1.9724\\
-0.369	-1.9692\\
-0.5032	-1.9383\\
-0.5758	-1.9383\\
-0.7417	-1.9395\\
-0.75	-1.9425\\
-0.7543	-1.9451\\
-0.7583	-1.9481\\
-0.7584	-1.9482\\
-0.7619	-1.9514\\
-0.7719	-1.9614\\
-0.7805	-1.972\\
-0.7878	-1.983\\
-0.7938	-1.9942\\
-0.7984	-2.0056\\
-0.8017	-2.017\\
-0.8018	-2.0188\\
-0.8018	-2.093\\
-0.8006	-2.2589\\
-0.7976	-2.268\\
-0.7946	-2.2763\\
-0.7909	-2.2848\\
-0.7865	-2.2926\\
-0.7815	-2.2997\\
-0.7814	-2.2998\\
-0.7759	-2.3061\\
-0.77	-2.3116\\
-0.7639	-2.3164\\
-0.7575	-2.3204\\
-0.7509	-2.3236\\
-0.6167	-2.3544\\
-0.6149	-2.3545\\
-0.6145	-2.3545\\
}--cycle;

\addplot[area legend, draw=mycolor1, fill=mycolor1, forget plot]
table[row sep=crcr] {%
x	y\\
-0.4742	-2.3776\\
-0.3987	-2.3776\\
-0.397	-2.3775\\
-0.2378	-2.3603\\
-0.2302	-2.3567\\
-0.2262	-2.3537\\
-0.2227	-2.3505\\
-0.2126	-2.3404\\
-0.2096	-2.3369\\
-0.2009	-2.3264\\
-0.1936	-2.3154\\
-0.1877	-2.3041\\
-0.1831	-2.2927\\
-0.1797	-2.2813\\
-0.1797	-2.2048\\
-0.1798	-2.2031\\
-0.1798	-2.2026\\
-0.197	-2.0434\\
-0.2	-2.0343\\
-0.2038	-2.0258\\
-0.2074	-2.0181\\
-0.2118	-2.0103\\
-0.2168	-2.0032\\
-0.2223	-1.9969\\
-0.2224	-1.9968\\
-0.2283	-1.9913\\
-0.2344	-1.9865\\
-0.2408	-1.9825\\
-0.2474	-1.9793\\
-0.3793	-1.9626\\
-0.4548	-1.9626\\
-0.4565	-1.9627\\
-0.6157	-1.9799\\
-0.6233	-1.9836\\
-0.6273	-1.9866\\
-0.6308	-1.9898\\
-0.6409	-1.9999\\
-0.644	-2.0033\\
-0.6526	-2.0139\\
-0.6599	-2.0249\\
-0.6658	-2.0361\\
-0.6705	-2.0475\\
-0.6738	-2.0589\\
-0.6738	-2.1354\\
-0.6737	-2.1371\\
-0.6737	-2.1376\\
-0.6565	-2.2968\\
-0.6535	-2.306\\
-0.6498	-2.3145\\
-0.6461	-2.3222\\
-0.6417	-2.33\\
-0.6367	-2.337\\
-0.6312	-2.3434\\
-0.6311	-2.3434\\
-0.6253	-2.349\\
-0.6191	-2.3537\\
-0.6127	-2.3577\\
-0.6062	-2.3609\\
-0.4742	-2.3776\\
}--cycle;

\addplot[area legend, draw=mycolor1, fill=mycolor1, forget plot]
table[row sep=crcr] {%
x	y\\
-0.335	-2.3861\\
-0.2568	-2.3861\\
-0.2552	-2.3858\\
-0.1039	-2.354\\
-0.0968	-2.3497\\
-0.0929	-2.3468\\
-0.0894	-2.3435\\
-0.0794	-2.3335\\
-0.0763	-2.3301\\
-0.0762	-2.33\\
-0.0676	-2.3194\\
-0.0603	-2.3084\\
-0.0544	-2.2972\\
-0.0497	-2.2858\\
-0.0464	-2.2744\\
-0.0464	-2.1942\\
-0.0467	-2.1926\\
-0.0467	-2.1921\\
-0.0786	-2.0408\\
-0.0816	-2.0316\\
-0.0853	-2.0231\\
-0.0897	-2.0153\\
-0.094	-2.0083\\
-0.099	-2.0012\\
-0.1045	-1.9949\\
-0.1103	-1.9894\\
-0.1104	-1.9893\\
-0.1166	-1.9845\\
-0.123	-1.9806\\
-0.1295	-1.9774\\
-0.2579	-1.974\\
-0.3358	-1.974\\
-0.3362	-1.9741\\
-0.3378	-1.9744\\
-0.4891	-2.0062\\
-0.4961	-2.0105\\
-0.5001	-2.0134\\
-0.5036	-2.0167\\
-0.5136	-2.0267\\
-0.5167	-2.0301\\
-0.5167	-2.0302\\
-0.5254	-2.0408\\
-0.5327	-2.0518\\
-0.5386	-2.063\\
-0.5432	-2.0744\\
-0.5466	-2.0858\\
-0.5466	-2.166\\
-0.5463	-2.1676\\
-0.5462	-2.1681\\
-0.5144	-2.3194\\
-0.5114	-2.3285\\
-0.5077	-2.3371\\
-0.5032	-2.3449\\
-0.499	-2.3519\\
-0.494	-2.359\\
-0.4885	-2.3653\\
-0.4826	-2.3708\\
-0.4826	-2.3709\\
-0.4764	-2.3756\\
-0.47	-2.3796\\
-0.4635	-2.3828\\
-0.335	-2.3861\\
}--cycle;

\addplot[area legend, draw=mycolor1, fill=mycolor1, forget plot]
table[row sep=crcr] {%
x	y\\
-0.3242	-2.3901\\
-0.2439	-2.3901\\
-0.1203	-2.3809\\
-0.1199	-2.3808\\
-0.1184	-2.3804\\
0.0239	-2.3352\\
0.0279	-2.3322\\
0.0342	-2.3275\\
0.0377	-2.3242\\
0.0477	-2.3142\\
0.0508	-2.3108\\
0.0594	-2.3002\\
0.0595	-2.3001\\
0.0668	-2.2892\\
0.0727	-2.2779\\
0.0773	-2.2665\\
0.0807	-2.2551\\
0.0807	-2.1712\\
0.0802	-2.1692\\
0.035	-2.0269\\
0.032	-2.0177\\
0.0282	-2.0092\\
0.0238	-2.0014\\
0.0188	-1.9943\\
0.0141	-1.988\\
0.0086	-1.9817\\
0.0027	-1.9761\\
-0.0035	-1.9714\\
-0.0036	-1.9713\\
-0.01	-1.9673\\
-0.0165	-1.9641\\
-0.0968	-1.9641\\
-0.2205	-1.9733\\
-0.2208	-1.9734\\
-0.2223	-1.9738\\
-0.3646	-2.019\\
-0.3686	-2.022\\
-0.3749	-2.0267\\
-0.3784	-2.03\\
-0.3884	-2.04\\
-0.3915	-2.0434\\
-0.4002	-2.054\\
-0.4002	-2.0541\\
-0.4075	-2.0651\\
-0.4134	-2.0763\\
-0.4181	-2.0877\\
-0.4214	-2.0991\\
-0.4214	-2.1831\\
-0.421	-2.1846\\
-0.4209	-2.1851\\
-0.3757	-2.3274\\
-0.3727	-2.3365\\
-0.369	-2.345\\
-0.3645	-2.3528\\
-0.3596	-2.3599\\
-0.3548	-2.3662\\
-0.3493	-2.3725\\
-0.3434	-2.3781\\
-0.3373	-2.3828\\
-0.3372	-2.3829\\
-0.3308	-2.3869\\
-0.3242	-2.3901\\
}--cycle;

\addplot[area legend, draw=mycolor1, fill=mycolor1, forget plot]
table[row sep=crcr] {%
x	y\\
-0.1898	-2.3836\\
-0.1076	-2.3836\\
0.0104	-2.3629\\
0.0107	-2.3628\\
0.0121	-2.3622\\
0.1445	-2.3051\\
0.1485	-2.3021\\
0.152	-2.2989\\
0.1576	-2.2937\\
0.1676	-2.2837\\
0.1707	-2.2802\\
0.1793	-2.2697\\
0.1866	-2.2587\\
0.1866	-2.2586\\
0.1926	-2.2473\\
0.1972	-2.2359\\
0.2005	-2.2245\\
0.2005	-2.1368\\
0.1976	-2.1276\\
0.197	-2.1262\\
0.1969	-2.1258\\
0.1397	-1.9934\\
0.136	-1.9849\\
0.1316	-1.9771\\
0.1266	-1.97\\
0.1211	-1.9637\\
0.1159	-1.9581\\
0.11	-1.9525\\
0.1039	-1.9478\\
0.0975	-1.9438\\
0.0974	-1.9438\\
0.0908	-1.9405\\
0.0086	-1.9405\\
-0.1094	-1.9613\\
-0.1097	-1.9614\\
-0.1111	-1.9619\\
-0.2435	-2.0191\\
-0.2474	-2.0221\\
-0.2509	-2.0253\\
-0.2566	-2.0305\\
-0.2666	-2.0405\\
-0.2696	-2.0439\\
-0.2783	-2.0545\\
-0.2856	-2.0655\\
-0.2856	-2.0656\\
-0.2915	-2.0768\\
-0.2962	-2.0882\\
-0.2995	-2.0996\\
-0.2995	-2.1874\\
-0.2965	-2.1965\\
-0.296	-2.198\\
-0.2958	-2.1984\\
-0.2387	-2.3308\\
-0.2349	-2.3393\\
-0.2305	-2.3471\\
-0.2255	-2.3542\\
-0.2201	-2.3605\\
-0.2149	-2.3661\\
-0.209	-2.3716\\
-0.2028	-2.3764\\
-0.1964	-2.3804\\
-0.1963	-2.3804\\
-0.1898	-2.3836\\
}--cycle;

\addplot[area legend, draw=mycolor1, fill=mycolor1, forget plot]
table[row sep=crcr] {%
x	y\\
-0.0613	-2.3644\\
0.0225	-2.3644\\
0.1338	-2.3331\\
0.1341	-2.333\\
0.1354	-2.3323\\
0.2571	-2.2647\\
0.2611	-2.2617\\
0.2646	-2.2585\\
0.2746	-2.2485\\
0.2776	-2.245\\
0.2825	-2.2395\\
0.2912	-2.2289\\
0.2985	-2.2179\\
0.3044	-2.2067\\
0.3044	-2.2066\\
0.3091	-2.1952\\
0.3124	-2.1838\\
0.3124	-2.0922\\
0.3094	-2.0831\\
0.3057	-2.0746\\
0.305	-2.0732\\
0.3048	-2.0729\\
0.2372	-1.9512\\
0.2328	-1.9434\\
0.2278	-1.9363\\
0.2223	-1.93\\
0.2164	-1.9244\\
0.2109	-1.9195\\
0.2047	-1.9148\\
0.1983	-1.9108\\
0.1918	-1.9076\\
0.108	-1.9076\\
-0.0034	-1.9389\\
-0.0037	-1.939\\
-0.005	-1.9397\\
-0.1267	-2.0073\\
-0.1306	-2.0103\\
-0.1341	-2.0135\\
-0.1441	-2.0235\\
-0.1472	-2.027\\
-0.1521	-2.0325\\
-0.1607	-2.0431\\
-0.168	-2.0541\\
-0.174	-2.0653\\
-0.174	-2.0654\\
-0.1786	-2.0768\\
-0.182	-2.0882\\
-0.182	-2.1798\\
-0.179	-2.1889\\
-0.1753	-2.1974\\
-0.1746	-2.1987\\
-0.1744	-2.1991\\
-0.1067	-2.3208\\
-0.1023	-2.3286\\
-0.0973	-2.3357\\
-0.0919	-2.342\\
-0.086	-2.3476\\
-0.0805	-2.3525\\
-0.0743	-2.3572\\
-0.0679	-2.3612\\
-0.0613	-2.3644\\
}--cycle;

\addplot[area legend, draw=mycolor1, fill=mycolor1, forget plot]
table[row sep=crcr] {%
x	y\\
0.0604	-2.3336\\
0.1452	-2.3336\\
0.2492	-2.2928\\
0.2494	-2.2926\\
0.2506	-2.2919\\
0.361	-2.2151\\
0.365	-2.2122\\
0.3685	-2.2089\\
0.3785	-2.1989\\
0.3815	-2.1955\\
0.3902	-2.1849\\
0.3944	-2.1791\\
0.4017	-2.1681\\
0.4076	-2.1569\\
0.4122	-2.1455\\
0.4122	-2.1454\\
0.4156	-2.134\\
0.4156	-2.0386\\
0.4126	-2.0295\\
0.4089	-2.021\\
0.4044	-2.0132\\
0.4037	-2.012\\
0.4035	-2.0116\\
0.3267	-1.9012\\
0.3217	-1.8941\\
0.3163	-1.8878\\
0.3104	-1.8823\\
0.3042	-1.8775\\
0.2984	-1.8734\\
0.292	-1.8694\\
0.2855	-1.8662\\
0.2854	-1.8661\\
0.2006	-1.8661\\
0.0966	-1.907\\
0.0964	-1.9071\\
0.0952	-1.9079\\
-0.0152	-1.9846\\
-0.0192	-1.9876\\
-0.0227	-1.9908\\
-0.0327	-2.0008\\
-0.0357	-2.0043\\
-0.0444	-2.0148\\
-0.0486	-2.0206\\
-0.0559	-2.0316\\
-0.0618	-2.0429\\
-0.0664	-2.0543\\
-0.0698	-2.0657\\
-0.0698	-2.1611\\
-0.0668	-2.1703\\
-0.0631	-2.1788\\
-0.0586	-2.1866\\
-0.0579	-2.1878\\
-0.0577	-2.1881\\
0.0191	-2.2985\\
0.0241	-2.3056\\
0.0295	-2.3119\\
0.0354	-2.3175\\
0.0416	-2.3222\\
0.0474	-2.3264\\
0.0538	-2.3304\\
0.0603	-2.3336\\
0.0604	-2.3336\\
}--cycle;

\addplot[area legend, draw=mycolor1, fill=mycolor1, forget plot]
table[row sep=crcr] {%
x	y\\
0.1743	-2.2923\\
0.2598	-2.2923\\
0.3557	-2.243\\
0.3597	-2.2401\\
0.3608	-2.2392\\
0.361	-2.239\\
0.4597	-2.1547\\
0.4632	-2.1514\\
0.4732	-2.1414\\
0.4762	-2.138\\
0.4849	-2.1274\\
0.4922	-2.1164\\
0.4956	-2.1105\\
0.5016	-2.0992\\
0.5062	-2.0878\\
0.5095	-2.0764\\
0.5096	-2.0763\\
0.5096	-1.9772\\
0.5066	-1.968\\
0.5028	-1.9595\\
0.4984	-1.9517\\
0.4934	-1.9446\\
0.4926	-1.9436\\
0.4923	-1.9432\\
0.408	-1.8446\\
0.4025	-1.8383\\
0.3966	-1.8327\\
0.3904	-1.828\\
0.384	-1.824\\
0.3781	-1.8205\\
0.3715	-1.8173\\
0.286	-1.8173\\
0.19	-1.8665\\
0.1861	-1.8695\\
0.185	-1.8704\\
0.1847	-1.8705\\
0.0861	-1.9549\\
0.0826	-1.9581\\
0.0726	-1.9681\\
0.0695	-1.9716\\
0.0609	-1.9822\\
0.0536	-1.9932\\
0.0501	-1.9991\\
0.0442	-2.0104\\
0.0396	-2.0218\\
0.0362	-2.0332\\
0.0362	-2.1324\\
0.0392	-2.1415\\
0.0429	-2.15\\
0.0473	-2.1578\\
0.0523	-2.1649\\
0.0532	-2.166\\
0.0534	-2.1663\\
0.1378	-2.265\\
0.1433	-2.2713\\
0.1492	-2.2768\\
0.1553	-2.2816\\
0.1617	-2.2856\\
0.1677	-2.289\\
0.1742	-2.2922\\
0.1743	-2.2923\\
}--cycle;

\addplot[area legend, draw=mycolor1, fill=mycolor1, forget plot]
table[row sep=crcr] {%
x	y\\
0.2798	-2.2415\\
0.3655	-2.2415\\
0.453	-2.185\\
0.4569	-2.182\\
0.4604	-2.1788\\
0.4606	-2.1786\\
0.4616	-2.1777\\
0.4716	-2.1677\\
0.5583	-2.0771\\
0.5613	-2.0737\\
0.57	-2.0631\\
0.5773	-2.0521\\
0.5832	-2.0408\\
0.5859	-2.0348\\
0.5906	-2.0234\\
0.5939	-2.012\\
0.5939	-1.9091\\
0.5909	-1.9\\
0.5872	-1.8915\\
0.5828	-1.8836\\
0.5778	-1.8766\\
0.5723	-1.8702\\
0.5711	-1.869\\
0.4805	-1.7824\\
0.4747	-1.7768\\
0.4685	-1.772\\
0.4621	-1.7681\\
0.4556	-1.7649\\
0.4495	-1.7621\\
0.3636	-1.7621\\
0.2762	-1.8187\\
0.2723	-1.8216\\
0.2687	-1.8249\\
0.2576	-1.836\\
0.1709	-1.9265\\
0.1679	-1.93\\
0.1592	-1.9406\\
0.1519	-1.9516\\
0.146	-1.9628\\
0.1432	-1.9689\\
0.1386	-1.9803\\
0.1353	-1.9917\\
0.1353	-2.0945\\
0.1382	-2.1037\\
0.142	-2.1122\\
0.1464	-2.12\\
0.1514	-2.1271\\
0.1569	-2.1334\\
0.1578	-2.1344\\
0.1581	-2.1347\\
0.2486	-2.2213\\
0.2545	-2.2268\\
0.2607	-2.2316\\
0.2671	-2.2356\\
0.2736	-2.2388\\
0.2797	-2.2415\\
0.2798	-2.2415\\
}--cycle;

\addplot[area legend, draw=mycolor1, fill=mycolor1, forget plot]
table[row sep=crcr] {%
x	y\\
0.3758	-2.1827\\
0.4621	-2.1827\\
0.4661	-2.1797\\
0.5446	-2.1169\\
0.5482	-2.1137\\
0.5582	-2.1037\\
0.5612	-2.1002\\
0.5623	-2.0991\\
0.5709	-2.0885\\
0.6454	-1.9931\\
0.6527	-1.9821\\
0.6586	-1.9709\\
0.6633	-1.9595\\
0.6653	-1.9534\\
0.6687	-1.942\\
0.6687	-1.8356\\
0.6657	-1.8264\\
0.6619	-1.8179\\
0.6575	-1.8101\\
0.6525	-1.803\\
0.647	-1.7967\\
0.6412	-1.7912\\
0.6402	-1.7903\\
0.6399	-1.7901\\
0.5445	-1.7156\\
0.5384	-1.7108\\
0.532	-1.7068\\
0.5254	-1.7036\\
0.5194	-1.7016\\
0.4331	-1.7016\\
0.4291	-1.7045\\
0.3506	-1.7673\\
0.347	-1.7705\\
0.337	-1.7805\\
0.334	-1.784\\
0.3338	-1.7842\\
0.3329	-1.7852\\
0.3243	-1.7957\\
0.2498	-1.8911\\
0.2425	-1.9021\\
0.2366	-1.9133\\
0.2319	-1.9247\\
0.2299	-1.9308\\
0.2265	-1.9422\\
0.2265	-2.0487\\
0.2295	-2.0578\\
0.2333	-2.0663\\
0.2377	-2.0741\\
0.2427	-2.0812\\
0.2482	-2.0875\\
0.254	-2.0931\\
0.255	-2.0939\\
0.2553	-2.0941\\
0.3507	-2.1687\\
0.3568	-2.1734\\
0.3632	-2.1774\\
0.3698	-2.1806\\
0.3758	-2.1827\\
}--cycle;

\addplot[area legend, draw=mycolor1, fill=mycolor1, forget plot]
table[row sep=crcr] {%
x	y\\
0.4622	-2.1168\\
0.5493	-2.1168\\
0.5532	-2.1138\\
0.5567	-2.1106\\
0.6262	-2.0427\\
0.6362	-2.0327\\
0.6392	-2.0293\\
0.6479	-2.0187\\
0.648	-2.0185\\
0.6487	-2.0175\\
0.656	-2.0065\\
0.7184	-1.9077\\
0.7244	-1.8964\\
0.729	-1.885\\
0.7323	-1.8736\\
0.7337	-1.8676\\
0.7337	-1.7577\\
0.7307	-1.7486\\
0.727	-1.7401\\
0.7226	-1.7323\\
0.7176	-1.7252\\
0.7121	-1.7189\\
0.7062	-1.7133\\
0.7	-1.7086\\
0.699	-1.7078\\
0.6987	-1.7076\\
0.5999	-1.6452\\
0.5935	-1.6413\\
0.587	-1.638\\
0.581	-1.6367\\
0.4939	-1.6367\\
0.49	-1.6396\\
0.4865	-1.6429\\
0.417	-1.7107\\
0.407	-1.7207\\
0.404	-1.7242\\
0.3953	-1.7348\\
0.3952	-1.735\\
0.3945	-1.736\\
0.3872	-1.747\\
0.3248	-1.8458\\
0.3188	-1.8571\\
0.3142	-1.8685\\
0.3109	-1.8798\\
0.3095	-1.8859\\
0.3095	-1.9957\\
0.3125	-2.0049\\
0.3162	-2.0134\\
0.3206	-2.0212\\
0.3256	-2.0283\\
0.3311	-2.0346\\
0.337	-2.0401\\
0.3432	-2.0449\\
0.3442	-2.0456\\
0.3445	-2.0458\\
0.4433	-2.1082\\
0.4497	-2.1122\\
0.4562	-2.1154\\
0.4622	-2.1168\\
}--cycle;

\addplot[area legend, draw=mycolor1, fill=mycolor1, forget plot]
table[row sep=crcr] {%
x	y\\
0.5386	-2.0451\\
0.6266	-2.0451\\
0.6306	-2.0421\\
0.6341	-2.0389\\
0.6441	-2.0289\\
0.6472	-2.0255\\
0.7073	-1.9536\\
0.716	-1.943\\
0.7232	-1.932\\
0.7234	-1.9318\\
0.724	-1.9307\\
0.7299	-1.9195\\
0.7803	-1.8185\\
0.785	-1.8071\\
0.7883	-1.7957\\
0.789	-1.7898\\
0.789	-1.6767\\
0.7861	-1.6676\\
0.7823	-1.6591\\
0.7779	-1.6512\\
0.7729	-1.6442\\
0.7674	-1.6378\\
0.7615	-1.6323\\
0.7554	-1.6275\\
0.749	-1.6236\\
0.7479	-1.623\\
0.7476	-1.6228\\
0.6467	-1.5724\\
0.6401	-1.5692\\
0.6342	-1.5684\\
0.5462	-1.5684\\
0.5422	-1.5714\\
0.5387	-1.5747\\
0.5287	-1.5847\\
0.5256	-1.5881\\
0.4655	-1.66\\
0.4568	-1.6706\\
0.4495	-1.6816\\
0.4494	-1.6818\\
0.4488	-1.6828\\
0.4429	-1.6941\\
0.3924	-1.795\\
0.3878	-1.8064\\
0.3845	-1.8178\\
0.3837	-1.8238\\
0.3837	-1.9368\\
0.3867	-1.946\\
0.3905	-1.9545\\
0.3949	-1.9623\\
0.3999	-1.9694\\
0.4054	-1.9757\\
0.4112	-1.9812\\
0.4174	-1.986\\
0.4238	-1.99\\
0.4249	-1.9906\\
0.4252	-1.9907\\
0.5261	-2.0412\\
0.5326	-2.0444\\
0.5386	-2.0451\\
}--cycle;

\addplot[area legend, draw=mycolor1, fill=mycolor1, forget plot]
table[row sep=crcr] {%
x	y\\
0.6047	-1.9688\\
0.694	-1.9688\\
0.698	-1.9658\\
0.7015	-1.9626\\
0.7115	-1.9526\\
0.7146	-1.9492\\
0.7232	-1.9386\\
0.7741	-1.8637\\
0.7814	-1.8527\\
0.7873	-1.8415\\
0.7874	-1.8412\\
0.7879	-1.8402\\
0.7925	-1.8288\\
0.8312	-1.7269\\
0.8346	-1.7155\\
0.8347	-1.7097\\
0.8347	-1.5936\\
0.8317	-1.5845\\
0.828	-1.576\\
0.8235	-1.5682\\
0.8185	-1.5611\\
0.8131	-1.5548\\
0.8072	-1.5492\\
0.801	-1.5445\\
0.7946	-1.5405\\
0.7881	-1.5373\\
0.7878	-1.5372\\
0.7867	-1.5367\\
0.6849	-1.498\\
0.6791	-1.4978\\
0.5898	-1.4978\\
0.5858	-1.5008\\
0.5823	-1.5041\\
0.5723	-1.5141\\
0.5692	-1.5175\\
0.5606	-1.5281\\
0.5097	-1.6029\\
0.5024	-1.6139\\
0.4965	-1.6252\\
0.4964	-1.6254\\
0.4959	-1.6265\\
0.4913	-1.6379\\
0.4526	-1.7397\\
0.4492	-1.7511\\
0.4491	-1.7569\\
0.4491	-1.873\\
0.4521	-1.8822\\
0.4558	-1.8907\\
0.4603	-1.8985\\
0.4652	-1.9056\\
0.4707	-1.9119\\
0.4766	-1.9174\\
0.4828	-1.9222\\
0.4892	-1.9262\\
0.4957	-1.9294\\
0.496	-1.9295\\
0.4971	-1.93\\
0.5989	-1.9687\\
0.6047	-1.9688\\
}--cycle;

\addplot[area legend, draw=mycolor1, fill=mycolor1, forget plot]
table[row sep=crcr] {%
x	y\\
0.6551	-1.8894\\
0.7458	-1.8894\\
0.7514	-1.889\\
0.7554	-1.886\\
0.7589	-1.8828\\
0.7689	-1.8728\\
0.772	-1.8694\\
0.7806	-1.8588\\
0.7879	-1.8478\\
0.8295	-1.7709\\
0.8355	-1.7597\\
0.8401	-1.7483\\
0.8402	-1.7481\\
0.8405	-1.747\\
0.8439	-1.7356\\
0.8712	-1.634\\
0.8712	-1.5152\\
0.8708	-1.5096\\
0.8679	-1.5004\\
0.8678	-1.5004\\
0.8641	-1.4919\\
0.8597	-1.4841\\
0.8547	-1.477\\
0.8492	-1.4707\\
0.8433	-1.4651\\
0.8371	-1.4604\\
0.8307	-1.4564\\
0.8242	-1.4532\\
0.8239	-1.4531\\
0.8228	-1.4527\\
0.7212	-1.4254\\
0.6305	-1.4254\\
0.6249	-1.4258\\
0.6209	-1.4288\\
0.6174	-1.432\\
0.6074	-1.442\\
0.6043	-1.4455\\
0.5957	-1.456\\
0.5884	-1.467\\
0.5467	-1.5439\\
0.5408	-1.5551\\
0.5362	-1.5665\\
0.5361	-1.5667\\
0.5358	-1.5678\\
0.5324	-1.5792\\
0.505	-1.6808\\
0.505	-1.7996\\
0.5055	-1.8052\\
0.5084	-1.8144\\
0.5085	-1.8144\\
0.5122	-1.8229\\
0.5166	-1.8307\\
0.5216	-1.8378\\
0.5271	-1.8441\\
0.533	-1.8497\\
0.5392	-1.8544\\
0.5456	-1.8584\\
0.5521	-1.8616\\
0.5524	-1.8617\\
0.5535	-1.8621\\
0.6551	-1.8894\\
}--cycle;

\addplot[area legend, draw=mycolor1, fill=mycolor1, forget plot]
table[row sep=crcr] {%
x	y\\
0.7011	-1.8078\\
0.7935	-1.8078\\
0.7988	-1.8069\\
0.7989	-1.8068\\
0.8028	-1.8038\\
0.8063	-1.8006\\
0.8163	-1.7906\\
0.8194	-1.7872\\
0.828	-1.7766\\
0.8353	-1.7656\\
0.8413	-1.7543\\
0.8739	-1.6765\\
0.8785	-1.6651\\
0.8818	-1.6537\\
0.8819	-1.6535\\
0.8821	-1.6524\\
0.8986	-1.5522\\
0.8986	-1.4307\\
0.8977	-1.4254\\
0.8947	-1.4163\\
0.891	-1.4078\\
0.8909	-1.4077\\
0.8865	-1.3999\\
0.8815	-1.3928\\
0.8761	-1.3865\\
0.8702	-1.381\\
0.864	-1.3762\\
0.8576	-1.3722\\
0.8511	-1.369\\
0.85	-1.3688\\
0.8497	-1.3687\\
0.7495	-1.3522\\
0.657	-1.3522\\
0.6517	-1.3531\\
0.6517	-1.3532\\
0.6477	-1.3562\\
0.6442	-1.3594\\
0.6342	-1.3694\\
0.6311	-1.3728\\
0.6225	-1.3834\\
0.6152	-1.3944\\
0.6093	-1.4057\\
0.5766	-1.4835\\
0.572	-1.4949\\
0.5687	-1.5063\\
0.5686	-1.5065\\
0.5684	-1.5076\\
0.5519	-1.6078\\
0.5519	-1.7293\\
0.5528	-1.7346\\
0.5558	-1.7437\\
0.5595	-1.7522\\
0.5596	-1.7523\\
0.564	-1.7601\\
0.569	-1.7672\\
0.5745	-1.7735\\
0.5804	-1.779\\
0.5865	-1.7838\\
0.5929	-1.7878\\
0.5995	-1.791\\
0.6005	-1.7912\\
0.6008	-1.7913\\
0.7011	-1.8078\\
}--cycle;

\addplot[area legend, draw=mycolor1, fill=mycolor1, forget plot]
table[row sep=crcr] {%
x	y\\
0.7372	-1.7247\\
0.8314	-1.7247\\
0.8364	-1.7233\\
0.8365	-1.7233\\
0.8405	-1.7203\\
0.844	-1.7171\\
0.854	-1.7071\\
0.857	-1.7036\\
0.8657	-1.6931\\
0.873	-1.6821\\
0.8789	-1.6708\\
0.8835	-1.6594\\
0.9074	-1.5815\\
0.9107	-1.5701\\
0.9107	-1.5699\\
0.9108	-1.5688\\
0.9171	-1.4709\\
0.9171	-1.3471\\
0.9156	-1.3421\\
0.9127	-1.333\\
0.9089	-1.3245\\
0.9045	-1.3167\\
0.9045	-1.3166\\
0.8995	-1.3095\\
0.894	-1.3032\\
0.8881	-1.2977\\
0.8819	-1.2929\\
0.8755	-1.2889\\
0.869	-1.2857\\
0.868	-1.2856\\
0.8677	-1.2856\\
0.7698	-1.2794\\
0.6755	-1.2794\\
0.6705	-1.2808\\
0.6704	-1.2808\\
0.6665	-1.2838\\
0.663	-1.287\\
0.653	-1.297\\
0.6499	-1.3005\\
0.6412	-1.311\\
0.634	-1.322\\
0.628	-1.3333\\
0.6234	-1.3447\\
0.5996	-1.4226\\
0.5962	-1.434\\
0.5962	-1.4342\\
0.5961	-1.4353\\
0.5899	-1.5332\\
0.5899	-1.657\\
0.5913	-1.662\\
0.5943	-1.6711\\
0.598	-1.6796\\
0.6024	-1.6874\\
0.6025	-1.6875\\
0.6075	-1.6946\\
0.6129	-1.7009\\
0.6188	-1.7064\\
0.625	-1.7112\\
0.6314	-1.7152\\
0.6379	-1.7184\\
0.639	-1.7185\\
0.6393	-1.7185\\
0.7372	-1.7247\\
}--cycle;

\addplot[area legend, draw=mycolor1, fill=mycolor1, forget plot]
table[row sep=crcr] {%
x	y\\
0.669	-1.6448\\
0.7652	-1.6448\\
0.8599	-1.6413\\
0.8646	-1.6395\\
0.8685	-1.6365\\
0.8686	-1.6364\\
0.8721	-1.6332\\
0.8821	-1.6232\\
0.8851	-1.6198\\
0.8938	-1.6092\\
0.9011	-1.5982\\
0.907	-1.5869\\
0.9116	-1.5755\\
0.915	-1.5641\\
0.9304	-1.487\\
0.9304	-1.3599\\
0.9269	-1.2652\\
0.9239	-1.2561\\
0.9221	-1.2514\\
0.9183	-1.2429\\
0.9139	-1.2351\\
0.9089	-1.228\\
0.9089	-1.2279\\
0.9034	-1.2216\\
0.8975	-1.2161\\
0.8913	-1.2113\\
0.885	-1.2073\\
0.8784	-1.2041\\
0.7809	-1.2041\\
0.6862	-1.2076\\
0.6815	-1.2094\\
0.6776	-1.2124\\
0.6775	-1.2125\\
0.674	-1.2157\\
0.664	-1.2257\\
0.6609	-1.2291\\
0.6523	-1.2397\\
0.645	-1.2507\\
0.6391	-1.262\\
0.6344	-1.2734\\
0.6311	-1.2847\\
0.6157	-1.3619\\
0.6157	-1.489\\
0.6192	-1.5837\\
0.6222	-1.5928\\
0.624	-1.5975\\
0.6277	-1.606\\
0.6322	-1.6138\\
0.6372	-1.6209\\
0.6427	-1.6273\\
0.6486	-1.6328\\
0.6547	-1.6376\\
0.6611	-1.6415\\
0.6677	-1.6447\\
0.6687	-1.6448\\
0.669	-1.6448\\
}--cycle;

\addplot[area legend, draw=mycolor1, fill=mycolor1, forget plot]
table[row sep=crcr] {%
x	y\\
0.689	-1.571\\
0.7873	-1.571\\
0.7876	-1.5709\\
0.7885	-1.5708\\
0.8792	-1.5584\\
0.8835	-1.5562\\
0.8874	-1.5532\\
0.8909	-1.55\\
0.901	-1.5399\\
0.9041	-1.5365\\
0.9127	-1.5259\\
0.92	-1.5149\\
0.9259	-1.5036\\
0.9305	-1.4922\\
0.9339	-1.4809\\
0.9412	-1.4052\\
0.9412	-1.2773\\
0.9411	-1.2764\\
0.9286	-1.1857\\
0.9257	-1.1766\\
0.9219	-1.1681\\
0.9197	-1.1638\\
0.9153	-1.156\\
0.9103	-1.1489\\
0.9048	-1.1426\\
0.9048	-1.1425\\
0.8989	-1.137\\
0.8927	-1.1322\\
0.8863	-1.1282\\
0.8798	-1.125\\
0.7815	-1.125\\
0.7812	-1.1251\\
0.7802	-1.1251\\
0.6896	-1.1376\\
0.6853	-1.1398\\
0.6813	-1.1428\\
0.6778	-1.146\\
0.6678	-1.156\\
0.6678	-1.1561\\
0.6647	-1.1595\\
0.6561	-1.1701\\
0.6488	-1.1811\\
0.6428	-1.1924\\
0.6382	-1.2037\\
0.6349	-1.2151\\
0.6275	-1.2908\\
0.6275	-1.4187\\
0.6276	-1.4196\\
0.6401	-1.5103\\
0.6431	-1.5194\\
0.6468	-1.5279\\
0.649	-1.5322\\
0.6535	-1.54\\
0.6584	-1.5471\\
0.6639	-1.5534\\
0.664	-1.5535\\
0.6699	-1.559\\
0.676	-1.5638\\
0.6824	-1.5678\\
0.689	-1.571\\
}--cycle;

\addplot[area legend, draw=mycolor1, fill=mycolor1, forget plot]
table[row sep=crcr] {%
x	y\\
0.7021	-1.4978\\
0.8029	-1.4978\\
0.8038	-1.4976\\
0.8897	-1.4769\\
0.8936	-1.4743\\
0.8975	-1.4714\\
0.9011	-1.4681\\
0.9111	-1.4581\\
0.9141	-1.4547\\
0.9228	-1.4441\\
0.9301	-1.4331\\
0.936	-1.4218\\
0.9407	-1.4104\\
0.944	-1.399\\
0.944	-1.2698\\
0.9438	-1.1964\\
0.9437	-1.1962\\
0.9435	-1.1953\\
0.9228	-1.1094\\
0.9199	-1.1003\\
0.9161	-1.0918\\
0.9117	-1.0839\\
0.9092	-1.08\\
0.9042	-1.0729\\
0.8987	-1.0666\\
0.8928	-1.0611\\
0.8928	-1.061\\
0.8866	-1.0563\\
0.8802	-1.0523\\
0.8736	-1.0491\\
0.7729	-1.0491\\
0.772	-1.0493\\
0.6861	-1.07\\
0.6822	-1.0726\\
0.6782	-1.0755\\
0.6747	-1.0788\\
0.6647	-1.0888\\
0.6617	-1.0922\\
0.653	-1.1028\\
0.6457	-1.1138\\
0.6397	-1.1251\\
0.6351	-1.1365\\
0.6318	-1.1479\\
0.6318	-1.2771\\
0.632	-1.3505\\
0.6321	-1.3507\\
0.6322	-1.3516\\
0.653	-1.4375\\
0.6559	-1.4466\\
0.6597	-1.4551\\
0.6641	-1.463\\
0.6666	-1.4669\\
0.6716	-1.474\\
0.6771	-1.4803\\
0.683	-1.4858\\
0.683	-1.4859\\
0.6892	-1.4906\\
0.6956	-1.4946\\
0.7021	-1.4978\\
}--cycle;

\addplot[area legend, draw=mycolor1, fill=mycolor1, forget plot]
table[row sep=crcr] {%
x	y\\
0.7076	-1.4261\\
0.8103	-1.4261\\
0.8105	-1.426\\
0.8114	-1.4257\\
0.8919	-1.3975\\
0.8958	-1.3946\\
0.8994	-1.3918\\
0.9029	-1.3885\\
0.9129	-1.3785\\
0.916	-1.3751\\
0.9246	-1.3645\\
0.9319	-1.3535\\
0.9379	-1.3422\\
0.9425	-1.3308\\
0.9458	-1.3194\\
0.9458	-1.1889\\
0.9385	-1.1184\\
0.9385	-1.1182\\
0.9382	-1.1173\\
0.9352	-1.1082\\
0.907	-1.0277\\
0.9033	-1.0192\\
0.8989	-1.0114\\
0.8939	-1.0043\\
0.8911	-1.0008\\
0.8856	-0.9944\\
0.8797	-0.9889\\
0.8735	-0.9841\\
0.8671	-0.9801\\
0.8606	-0.9769\\
0.7579	-0.9769\\
0.7576	-0.977\\
0.7568	-0.9773\\
0.6763	-1.0054\\
0.6723	-1.0084\\
0.6688	-1.0112\\
0.6653	-1.0145\\
0.6553	-1.0245\\
0.6522	-1.0279\\
0.6436	-1.0385\\
0.6363	-1.0495\\
0.6303	-1.0608\\
0.6257	-1.0722\\
0.6223	-1.0836\\
0.6223	-1.2141\\
0.6297	-1.2846\\
0.6297	-1.2848\\
0.63	-1.2857\\
0.6329	-1.2948\\
0.6611	-1.3753\\
0.6649	-1.3838\\
0.6693	-1.3916\\
0.6743	-1.3987\\
0.6771	-1.4022\\
0.6826	-1.4085\\
0.6885	-1.4141\\
0.6946	-1.4188\\
0.6947	-1.4189\\
0.7011	-1.4229\\
0.7076	-1.4261\\
}--cycle;

\addplot[area legend, draw=mycolor1, fill=mycolor1, forget plot]
table[row sep=crcr] {%
x	y\\
0.7058	-1.3564\\
0.8107	-1.3564\\
0.8109	-1.3563\\
0.8117	-1.356\\
0.8864	-1.3211\\
0.8903	-1.3182\\
0.8938	-1.3149\\
0.897	-1.3119\\
0.907	-1.3019\\
0.91	-1.2984\\
0.9187	-1.2878\\
0.926	-1.2769\\
0.9319	-1.2656\\
0.9319	-1.2655\\
0.9365	-1.2541\\
0.9399	-1.2428\\
0.9399	-1.1112\\
0.9261	-1.0442\\
0.9231	-1.035\\
0.923	-1.0348\\
0.9227	-1.034\\
0.919	-1.0255\\
0.8841	-0.9509\\
0.8797	-0.9431\\
0.8747	-0.936\\
0.8692	-0.9297\\
0.8662	-0.9266\\
0.8603	-0.921\\
0.8541	-0.9163\\
0.8477	-0.9123\\
0.8411	-0.9091\\
0.736	-0.9091\\
0.7352	-0.9095\\
0.6606	-0.9443\\
0.6567	-0.9473\\
0.6531	-0.9505\\
0.64	-0.9636\\
0.637	-0.967\\
0.6283	-0.9776\\
0.621	-0.9886\\
0.6151	-0.9998\\
0.6151	-0.9999\\
0.6104	-1.0113\\
0.6071	-1.0227\\
0.6071	-1.1543\\
0.6209	-1.2213\\
0.6239	-1.2304\\
0.624	-1.2306\\
0.6243	-1.2314\\
0.628	-1.2399\\
0.6629	-1.3145\\
0.6673	-1.3224\\
0.6723	-1.3294\\
0.6778	-1.3358\\
0.6808	-1.3389\\
0.6867	-1.3444\\
0.6929	-1.3492\\
0.6993	-1.3532\\
0.7058	-1.3564\\
}--cycle;

\addplot[area legend, draw=mycolor1, fill=mycolor1, forget plot]
table[row sep=crcr] {%
x	y\\
0.6973	-1.2894\\
0.8044	-1.2894\\
0.8046	-1.2893\\
0.8053	-1.2889\\
0.8737	-1.2482\\
0.8776	-1.2453\\
0.8812	-1.242\\
0.8912	-1.232\\
0.8942	-1.2286\\
0.8969	-1.2253\\
0.9056	-1.2148\\
0.9129	-1.2038\\
0.9188	-1.1925\\
0.9234	-1.1811\\
0.9268	-1.1697\\
0.9268	-1.0373\\
0.9071	-0.9742\\
0.9041	-0.9651\\
0.9003	-0.9566\\
0.9002	-0.9564\\
0.8999	-0.9557\\
0.8954	-0.9479\\
0.8548	-0.8795\\
0.8498	-0.8725\\
0.8443	-0.8661\\
0.8384	-0.8606\\
0.8352	-0.8579\\
0.829	-0.8531\\
0.8226	-0.8491\\
0.8161	-0.8459\\
0.709	-0.8459\\
0.708	-0.8464\\
0.6397	-0.8871\\
0.6357	-0.8901\\
0.6322	-0.8933\\
0.6222	-0.9033\\
0.6191	-0.9068\\
0.6164	-0.91\\
0.6078	-0.9206\\
0.6005	-0.9315\\
0.5946	-0.9428\\
0.5899	-0.9542\\
0.5899	-0.9543\\
0.5866	-0.9656\\
0.5866	-1.098\\
0.6063	-1.1611\\
0.6093	-1.1702\\
0.613	-1.1787\\
0.6135	-1.1797\\
0.6179	-1.1875\\
0.6586	-1.2558\\
0.6636	-1.2629\\
0.6691	-1.2692\\
0.6749	-1.2747\\
0.6782	-1.2774\\
0.6844	-1.2822\\
0.6907	-1.2862\\
0.6973	-1.2894\\
}--cycle;

\addplot[area legend, draw=mycolor1, fill=mycolor1, forget plot]
table[row sep=crcr] {%
x	y\\
0.6826	-1.2257\\
0.7919	-1.2257\\
0.7921	-1.2255\\
0.7927	-1.2251\\
0.8545	-1.1794\\
0.8584	-1.1765\\
0.862	-1.1732\\
0.872	-1.1632\\
0.875	-1.1598\\
0.8837	-1.1492\\
0.8859	-1.1458\\
0.8932	-1.1348\\
0.8992	-1.1236\\
0.9038	-1.1122\\
0.9071	-1.1008\\
0.9072	-1.1007\\
0.9072	-0.9678\\
0.9042	-0.9586\\
0.8791	-0.9\\
0.8754	-0.8915\\
0.871	-0.8837\\
0.8709	-0.8835\\
0.8704	-0.8828\\
0.8654	-0.8757\\
0.8198	-0.814\\
0.8143	-0.8077\\
0.8084	-0.8021\\
0.8022	-0.7974\\
0.7989	-0.7951\\
0.7925	-0.7911\\
0.786	-0.7879\\
0.6767	-0.7879\\
0.6765	-0.788\\
0.6758	-0.7885\\
0.614	-0.8341\\
0.6101	-0.8371\\
0.6066	-0.8404\\
0.5966	-0.8504\\
0.5935	-0.8538\\
0.5849	-0.8644\\
0.5826	-0.8677\\
0.5753	-0.8787\\
0.5693	-0.89\\
0.5647	-0.9014\\
0.5614	-0.9128\\
0.5614	-1.0458\\
0.5643	-1.0549\\
0.5894	-1.1136\\
0.5931	-1.1221\\
0.5975	-1.1299\\
0.5976	-1.1301\\
0.5981	-1.1307\\
0.6031	-1.1378\\
0.6487	-1.1996\\
0.6542	-1.2059\\
0.6601	-1.2114\\
0.6663	-1.2162\\
0.6696	-1.2185\\
0.676	-1.2224\\
0.6826	-1.2256\\
0.6826	-1.2257\\
}--cycle;

\addplot[area legend, draw=mycolor1, fill=mycolor1, forget plot]
table[row sep=crcr] {%
x	y\\
0.6623	-1.1656\\
0.7737	-1.1656\\
0.7776	-1.1627\\
0.7778	-1.1625\\
0.7784	-1.162\\
0.8333	-1.1122\\
0.8369	-1.109\\
0.8469	-1.099\\
0.8499	-1.0955\\
0.8586	-1.0849\\
0.8659	-1.0739\\
0.8677	-1.0705\\
0.8737	-1.0592\\
0.8783	-1.0478\\
0.8816	-1.0364\\
0.8816	-0.9031\\
0.8787	-0.894\\
0.8749	-0.8855\\
0.8452	-0.8315\\
0.8408	-0.8237\\
0.8358	-0.8166\\
0.8353	-0.816\\
0.8352	-0.8159\\
0.8297	-0.8096\\
0.7799	-0.7546\\
0.774	-0.7491\\
0.7678	-0.7443\\
0.7614	-0.7404\\
0.758	-0.7385\\
0.7515	-0.7353\\
0.64	-0.7353\\
0.6361	-0.7383\\
0.6359	-0.7384\\
0.6353	-0.7389\\
0.5804	-0.7887\\
0.5768	-0.7919\\
0.5668	-0.8019\\
0.5638	-0.8054\\
0.5551	-0.816\\
0.5478	-0.827\\
0.546	-0.8304\\
0.54	-0.8417\\
0.5354	-0.8531\\
0.5321	-0.8645\\
0.5321	-0.9978\\
0.535	-1.0069\\
0.5388	-1.0154\\
0.5685	-1.0694\\
0.5729	-1.0772\\
0.5779	-1.0843\\
0.5784	-1.0849\\
0.5785	-1.085\\
0.584	-1.0913\\
0.6338	-1.1463\\
0.6397	-1.1518\\
0.6459	-1.1566\\
0.6523	-1.1605\\
0.6557	-1.1624\\
0.6622	-1.1656\\
0.6623	-1.1656\\
}--cycle;

\addplot[area legend, draw=mycolor1, fill=mycolor1, forget plot]
table[row sep=crcr] {%
x	y\\
0.6369	-1.1097\\
0.7504	-1.1097\\
0.7544	-1.1067\\
0.7579	-1.1035\\
0.7686	-1.0928\\
0.8166	-1.0396\\
0.8196	-1.0362\\
0.8283	-1.0256\\
0.8356	-1.0146\\
0.8415	-1.0034\\
0.843	-0.9999\\
0.8476	-0.9885\\
0.8509	-0.9771\\
0.8509	-0.8437\\
0.848	-0.8345\\
0.8442	-0.826\\
0.8398	-0.8182\\
0.8061	-0.7692\\
0.8011	-0.7621\\
0.7956	-0.7558\\
0.795	-0.7553\\
0.7949	-0.7552\\
0.789	-0.7496\\
0.7359	-0.7017\\
0.7297	-0.6969\\
0.7233	-0.6929\\
0.7168	-0.6897\\
0.7133	-0.6882\\
0.5997	-0.6882\\
0.5958	-0.6912\\
0.5923	-0.6945\\
0.5823	-0.7045\\
0.5821	-0.7046\\
0.5816	-0.7051\\
0.5336	-0.7583\\
0.5305	-0.7618\\
0.5219	-0.7723\\
0.5146	-0.7833\\
0.5087	-0.7946\\
0.5072	-0.7981\\
0.5026	-0.8095\\
0.4992	-0.8209\\
0.4992	-0.9543\\
0.5022	-0.9634\\
0.506	-0.9719\\
0.5104	-0.9798\\
0.5441	-1.0287\\
0.5491	-1.0358\\
0.5546	-1.0421\\
0.5551	-1.0427\\
0.5552	-1.0428\\
0.5611	-1.0483\\
0.6143	-1.0963\\
0.6205	-1.101\\
0.6269	-1.105\\
0.6334	-1.1082\\
0.6369	-1.1097\\
}--cycle;

\addplot[area legend, draw=mycolor1, fill=mycolor1, forget plot]
table[row sep=crcr] {%
x	y\\
0.6071	-1.0584\\
0.7227	-1.0584\\
0.7267	-1.0554\\
0.7302	-1.0522\\
0.7402	-1.0422\\
0.7432	-1.0388\\
0.7519	-1.0282\\
0.752	-1.028\\
0.7525	-1.0274\\
0.7934	-0.9717\\
0.8007	-0.9608\\
0.8067	-0.9495\\
0.8113	-0.9381\\
0.8124	-0.9346\\
0.8157	-0.9232\\
0.8157	-0.7895\\
0.8127	-0.7803\\
0.809	-0.7718\\
0.8046	-0.764\\
0.7996	-0.7569\\
0.7624	-0.7131\\
0.7569	-0.7068\\
0.7511	-0.7013\\
0.7505	-0.7008\\
0.7504	-0.7007\\
0.7442	-0.6959\\
0.6885	-0.655\\
0.6821	-0.651\\
0.6756	-0.6478\\
0.6721	-0.6467\\
0.5565	-0.6467\\
0.5525	-0.6497\\
0.549	-0.6529\\
0.539	-0.6629\\
0.536	-0.6664\\
0.5273	-0.677\\
0.5272	-0.6771\\
0.5267	-0.6777\\
0.4858	-0.7334\\
0.4785	-0.7444\\
0.4725	-0.7556\\
0.4679	-0.767\\
0.4668	-0.7705\\
0.4635	-0.7819\\
0.4635	-0.9157\\
0.4665	-0.9248\\
0.4702	-0.9333\\
0.4746	-0.9411\\
0.4796	-0.9482\\
0.5168	-0.992\\
0.5223	-0.9983\\
0.5281	-1.0039\\
0.5287	-1.0043\\
0.5288	-1.0044\\
0.535	-1.0092\\
0.5907	-1.0502\\
0.5971	-1.0541\\
0.6036	-1.0573\\
0.6071	-1.0584\\
}--cycle;

\addplot[area legend, draw=mycolor1, fill=mycolor1, forget plot]
table[row sep=crcr] {%
x	y\\
0.5736	-1.012\\
0.6911	-1.012\\
0.6951	-1.009\\
0.6986	-1.0058\\
0.7086	-0.9958\\
0.7116	-0.9923\\
0.7203	-0.9818\\
0.7276	-0.9708\\
0.7277	-0.9706\\
0.7281	-0.97\\
0.7621	-0.9125\\
0.768	-0.9013\\
0.7726	-0.8899\\
0.776	-0.8785\\
0.7766	-0.875\\
0.7766	-0.7408\\
0.7737	-0.7317\\
0.7699	-0.7232\\
0.7655	-0.7154\\
0.7605	-0.7083\\
0.755	-0.702\\
0.7151	-0.6635\\
0.7092	-0.6579\\
0.7031	-0.6532\\
0.7025	-0.6528\\
0.7023	-0.6527\\
0.6959	-0.6487\\
0.6385	-0.6147\\
0.632	-0.6115\\
0.6285	-0.6108\\
0.511	-0.6108\\
0.507	-0.6138\\
0.5035	-0.617\\
0.4935	-0.627\\
0.4904	-0.6305\\
0.4818	-0.6411\\
0.4745	-0.6521\\
0.4744	-0.6522\\
0.474	-0.6528\\
0.44	-0.7103\\
0.4341	-0.7215\\
0.4295	-0.7329\\
0.4261	-0.7443\\
0.4254	-0.7478\\
0.4254	-0.882\\
0.4284	-0.8911\\
0.4322	-0.8997\\
0.4366	-0.9075\\
0.4416	-0.9145\\
0.447	-0.9209\\
0.487	-0.9594\\
0.4928	-0.9649\\
0.499	-0.9697\\
0.4996	-0.9701\\
0.4997	-0.9701\\
0.5061	-0.9741\\
0.5636	-1.0081\\
0.5701	-1.0113\\
0.5736	-1.012\\
}--cycle;

\addplot[area legend, draw=mycolor1, fill=mycolor1, forget plot]
table[row sep=crcr] {%
x	y\\
0.5369	-0.9706\\
0.6562	-0.9706\\
0.6602	-0.9677\\
0.6637	-0.9644\\
0.6737	-0.9544\\
0.6768	-0.951\\
0.6854	-0.9404\\
0.6927	-0.9294\\
0.6986	-0.9181\\
0.6987	-0.918\\
0.699	-0.9174\\
0.7261	-0.8589\\
0.7308	-0.8475\\
0.7341	-0.8361\\
0.7344	-0.8327\\
0.7344	-0.6979\\
0.7315	-0.6887\\
0.7277	-0.6802\\
0.7233	-0.6724\\
0.7183	-0.6653\\
0.7128	-0.659\\
0.7069	-0.6534\\
0.6649	-0.6203\\
0.6587	-0.6155\\
0.6523	-0.6116\\
0.6517	-0.6112\\
0.6516	-0.6112\\
0.645	-0.608\\
0.5866	-0.5809\\
0.5832	-0.5805\\
0.4638	-0.5805\\
0.4599	-0.5835\\
0.4563	-0.5868\\
0.4463	-0.5968\\
0.4433	-0.6002\\
0.4346	-0.6108\\
0.4273	-0.6218\\
0.4214	-0.633\\
0.421	-0.6338\\
0.3939	-0.6923\\
0.3893	-0.7037\\
0.3859	-0.7151\\
0.3856	-0.7185\\
0.3856	-0.8533\\
0.3886	-0.8625\\
0.3923	-0.871\\
0.3967	-0.8788\\
0.4017	-0.8859\\
0.4072	-0.8922\\
0.4131	-0.8977\\
0.4551	-0.9309\\
0.4613	-0.9356\\
0.4677	-0.9396\\
0.4685	-0.94\\
0.475	-0.9432\\
0.5335	-0.9703\\
0.5369	-0.9706\\
}--cycle;

\addplot[area legend, draw=mycolor1, fill=mycolor1, forget plot]
table[row sep=crcr] {%
x	y\\
0.4943	-0.9344\\
0.6187	-0.9344\\
0.6226	-0.9314\\
0.6261	-0.9282\\
0.6361	-0.9182\\
0.6392	-0.9148\\
0.6478	-0.9042\\
0.6551	-0.8932\\
0.6611	-0.8819\\
0.6657	-0.8705\\
0.6659	-0.8699\\
0.666	-0.8697\\
0.6864	-0.8109\\
0.6898	-0.7996\\
0.6898	-0.6607\\
0.6868	-0.6515\\
0.683	-0.643\\
0.6786	-0.6352\\
0.6736	-0.6281\\
0.6681	-0.6218\\
0.6623	-0.6162\\
0.6561	-0.6115\\
0.6125	-0.5837\\
0.6061	-0.5797\\
0.5995	-0.5765\\
0.5989	-0.5763\\
0.5988	-0.5762\\
0.54	-0.5558\\
0.4156	-0.5558\\
0.4117	-0.5588\\
0.4082	-0.562\\
0.3982	-0.572\\
0.3951	-0.5755\\
0.3865	-0.586\\
0.3792	-0.597\\
0.3732	-0.6083\\
0.3686	-0.6197\\
0.3684	-0.6203\\
0.3683	-0.6205\\
0.3479	-0.6793\\
0.3445	-0.6907\\
0.3445	-0.8295\\
0.3475	-0.8387\\
0.3513	-0.8472\\
0.3557	-0.8551\\
0.3607	-0.8621\\
0.3662	-0.8685\\
0.3721	-0.874\\
0.3782	-0.8788\\
0.4218	-0.9065\\
0.4282	-0.9105\\
0.4348	-0.9137\\
0.4354	-0.914\\
0.4355	-0.914\\
0.4943	-0.9344\\
}--cycle;

\addplot[area legend, draw=mycolor1, fill=mycolor1, forget plot]
table[row sep=crcr] {%
x	y\\
0.4532	-0.9037\\
0.5758	-0.9037\\
0.579	-0.9034\\
0.583	-0.9004\\
0.5865	-0.8972\\
0.5965	-0.8872\\
0.5996	-0.8837\\
0.6082	-0.8732\\
0.6155	-0.8622\\
0.6214	-0.8509\\
0.6261	-0.8395\\
0.6294	-0.8281\\
0.6295	-0.828\\
0.6296	-0.8273\\
0.6436	-0.7689\\
0.6436	-0.6324\\
0.6432	-0.6292\\
0.6403	-0.6201\\
0.6365	-0.6116\\
0.6321	-0.6037\\
0.6271	-0.5967\\
0.6216	-0.5903\\
0.6157	-0.5848\\
0.6096	-0.58\\
0.6032	-0.5761\\
0.5586	-0.5536\\
0.5521	-0.5504\\
0.5514	-0.5502\\
0.5513	-0.5501\\
0.4928	-0.5362\\
0.3702	-0.5362\\
0.3671	-0.5365\\
0.367	-0.5366\\
0.3631	-0.5395\\
0.3596	-0.5428\\
0.3496	-0.5528\\
0.3465	-0.5562\\
0.3379	-0.5668\\
0.3306	-0.5778\\
0.3246	-0.589\\
0.32	-0.6004\\
0.3167	-0.6118\\
0.3166	-0.612\\
0.3164	-0.6126\\
0.3025	-0.6711\\
0.3025	-0.8075\\
0.3028	-0.8107\\
0.3058	-0.8199\\
0.3095	-0.8284\\
0.3096	-0.8284\\
0.314	-0.8362\\
0.319	-0.8433\\
0.3245	-0.8496\\
0.3303	-0.8551\\
0.3365	-0.8599\\
0.3429	-0.8639\\
0.3875	-0.8864\\
0.394	-0.8896\\
0.3946	-0.8898\\
0.3948	-0.8898\\
0.4532	-0.9037\\
}--cycle;

\addplot[area legend, draw=mycolor1, fill=mycolor1, forget plot]
table[row sep=crcr] {%
x	y\\
0.4108	-0.8782\\
0.5348	-0.8782\\
0.5378	-0.8776\\
0.5379	-0.8776\\
0.5418	-0.8746\\
0.5453	-0.8714\\
0.5553	-0.8614\\
0.5584	-0.8579\\
0.5671	-0.8473\\
0.5743	-0.8364\\
0.5803	-0.8251\\
0.5849	-0.8137\\
0.5882	-0.8023\\
0.5883	-0.8021\\
0.5884	-0.8015\\
0.5961	-0.744\\
0.5961	-0.6065\\
0.5955	-0.6035\\
0.5925	-0.5944\\
0.5888	-0.5859\\
0.5844	-0.5781\\
0.5844	-0.578\\
0.5794	-0.571\\
0.5739	-0.5646\\
0.568	-0.5591\\
0.5618	-0.5543\\
0.5554	-0.5504\\
0.5489	-0.5471\\
0.5039	-0.5298\\
0.5033	-0.5297\\
0.5032	-0.5297\\
0.4457	-0.5219\\
0.3216	-0.5219\\
0.3186	-0.5226\\
0.3146	-0.5255\\
0.3111	-0.5288\\
0.3011	-0.5388\\
0.298	-0.5422\\
0.2894	-0.5528\\
0.2821	-0.5638\\
0.2762	-0.5751\\
0.2715	-0.5864\\
0.2682	-0.5978\\
0.2682	-0.598\\
0.268	-0.5986\\
0.2603	-0.6561\\
0.2603	-0.7936\\
0.2609	-0.7966\\
0.2639	-0.8058\\
0.2676	-0.8143\\
0.272	-0.8221\\
0.2721	-0.8221\\
0.2771	-0.8292\\
0.2825	-0.8355\\
0.2884	-0.8411\\
0.2946	-0.8458\\
0.301	-0.8498\\
0.3075	-0.853\\
0.3525	-0.8703\\
0.3531	-0.8704\\
0.3533	-0.8705\\
0.4108	-0.8782\\
}--cycle;

\addplot[area legend, draw=mycolor1, fill=mycolor1, forget plot]
table[row sep=crcr] {%
x	y\\
0.3675	-0.8578\\
0.4929	-0.8578\\
0.4957	-0.8569\\
0.4997	-0.8539\\
0.5032	-0.8507\\
0.5132	-0.8407\\
0.5163	-0.8372\\
0.5249	-0.8266\\
0.5322	-0.8156\\
0.5382	-0.8044\\
0.5428	-0.793\\
0.5461	-0.7816\\
0.5461	-0.7814\\
0.5462	-0.7808\\
0.5481	-0.7248\\
0.5481	-0.5863\\
0.5472	-0.5834\\
0.5442	-0.5743\\
0.5405	-0.5658\\
0.5361	-0.558\\
0.5311	-0.5509\\
0.5256	-0.5445\\
0.5197	-0.539\\
0.5135	-0.5342\\
0.5071	-0.5303\\
0.5006	-0.5271\\
0.4557	-0.5147\\
0.455	-0.5147\\
0.399	-0.5128\\
0.2736	-0.5128\\
0.2708	-0.5137\\
0.2707	-0.5137\\
0.2668	-0.5167\\
0.2633	-0.5199\\
0.2533	-0.5299\\
0.2502	-0.5333\\
0.2416	-0.5439\\
0.2343	-0.5549\\
0.2283	-0.5662\\
0.2237	-0.5776\\
0.2204	-0.589\\
0.2203	-0.5891\\
0.2203	-0.5897\\
0.2184	-0.6457\\
0.2184	-0.7843\\
0.2193	-0.7871\\
0.2223	-0.7963\\
0.226	-0.8048\\
0.2304	-0.8126\\
0.2354	-0.8197\\
0.2409	-0.826\\
0.2468	-0.8316\\
0.253	-0.8363\\
0.2594	-0.8403\\
0.2659	-0.8435\\
0.3108	-0.8558\\
0.3114	-0.8558\\
0.3115	-0.8559\\
0.3675	-0.8578\\
}--cycle;

\addplot[area legend, draw=mycolor1, fill=mycolor1, forget plot]
table[row sep=crcr] {%
x	y\\
0.2693	-0.8459\\
0.3965	-0.8459\\
0.4505	-0.8423\\
0.4532	-0.8412\\
0.4571	-0.8382\\
0.4607	-0.8349\\
0.4707	-0.8249\\
0.4737	-0.8215\\
0.4824	-0.8109\\
0.4897	-0.7999\\
0.4956	-0.7887\\
0.5002	-0.7773\\
0.5036	-0.7659\\
0.5036	-0.626\\
0.5035	-0.6254\\
0.5	-0.5714\\
0.497	-0.5623\\
0.4959	-0.5596\\
0.4921	-0.5511\\
0.4877	-0.5433\\
0.4827	-0.5362\\
0.4772	-0.5299\\
0.4713	-0.5243\\
0.4652	-0.5196\\
0.4588	-0.5156\\
0.4522	-0.5124\\
0.4079	-0.5049\\
0.2807	-0.5049\\
0.2267	-0.5085\\
0.2241	-0.5096\\
0.2201	-0.5126\\
0.2166	-0.5158\\
0.2066	-0.5258\\
0.2035	-0.5293\\
0.1948	-0.5399\\
0.1876	-0.5509\\
0.1816	-0.5621\\
0.177	-0.5735\\
0.1737	-0.5849\\
0.1737	-0.7254\\
0.1772	-0.7794\\
0.1802	-0.7885\\
0.1813	-0.7911\\
0.1851	-0.7996\\
0.1895	-0.8075\\
0.1945	-0.8145\\
0.2	-0.8209\\
0.2059	-0.8264\\
0.2121	-0.8312\\
0.2185	-0.8352\\
0.225	-0.8384\\
0.2693	-0.8459\\
}--cycle;

\addplot[area legend, draw=mycolor1, fill=mycolor1, forget plot]
table[row sep=crcr] {%
x	y\\
0.2285	-0.8403\\
0.3561	-0.8403\\
0.3567	-0.8402\\
0.4082	-0.8316\\
0.4106	-0.8303\\
0.4146	-0.8273\\
0.4181	-0.8241\\
0.4281	-0.8141\\
0.4281	-0.814\\
0.4312	-0.8106\\
0.4398	-0.8\\
0.4471	-0.789\\
0.4531	-0.7778\\
0.4577	-0.7664\\
0.461	-0.755\\
0.461	-0.6133\\
0.4523	-0.5618\\
0.4493	-0.5527\\
0.4456	-0.5441\\
0.4443	-0.5417\\
0.4398	-0.5339\\
0.4349	-0.5268\\
0.4294	-0.5205\\
0.4235	-0.515\\
0.4235	-0.5149\\
0.4173	-0.5102\\
0.4109	-0.5062\\
0.4044	-0.503\\
0.3611	-0.5001\\
0.2334	-0.5001\\
0.2329	-0.5002\\
0.1813	-0.5088\\
0.1789	-0.5101\\
0.1749	-0.5131\\
0.1714	-0.5163\\
0.1614	-0.5263\\
0.1614	-0.5264\\
0.1583	-0.5298\\
0.1497	-0.5404\\
0.1424	-0.5514\\
0.1365	-0.5626\\
0.1318	-0.574\\
0.1285	-0.5854\\
0.1285	-0.7263\\
0.1286	-0.7269\\
0.1286	-0.7271\\
0.1372	-0.7786\\
0.1402	-0.7877\\
0.1439	-0.7962\\
0.1453	-0.7987\\
0.1497	-0.8065\\
0.1547	-0.8136\\
0.1602	-0.8199\\
0.166	-0.8254\\
0.1661	-0.8254\\
0.1722	-0.8302\\
0.1786	-0.8342\\
0.1852	-0.8374\\
0.2285	-0.8403\\
}--cycle;

\addplot[area legend, draw=mycolor1, fill=mycolor1, forget plot]
table[row sep=crcr] {%
x	y\\
0.1468	-0.8403\\
0.2752	-0.8403\\
0.3171	-0.8389\\
0.3172	-0.8389\\
0.3177	-0.8388\\
0.3664	-0.8255\\
0.3686	-0.824\\
0.3726	-0.821\\
0.3761	-0.8178\\
0.3861	-0.8078\\
0.3892	-0.8043\\
0.3978	-0.7938\\
0.3978	-0.7937\\
0.4051	-0.7827\\
0.4111	-0.7715\\
0.4157	-0.7601\\
0.419	-0.7487\\
0.419	-0.6064\\
0.4189	-0.6058\\
0.4056	-0.5572\\
0.4026	-0.548\\
0.3989	-0.5395\\
0.3945	-0.5317\\
0.393	-0.5295\\
0.388	-0.5224\\
0.3825	-0.5161\\
0.3766	-0.5106\\
0.3704	-0.5058\\
0.364	-0.5018\\
0.3575	-0.4986\\
0.2291	-0.4986\\
0.1872	-0.5\\
0.1871	-0.5\\
0.1865	-0.5001\\
0.1379	-0.5134\\
0.1357	-0.5149\\
0.1317	-0.5178\\
0.1282	-0.5211\\
0.1182	-0.5311\\
0.1151	-0.5345\\
0.1065	-0.5451\\
0.0992	-0.5561\\
0.0932	-0.5674\\
0.0886	-0.5788\\
0.0853	-0.5902\\
0.0853	-0.7325\\
0.0854	-0.733\\
0.0987	-0.7817\\
0.1017	-0.7908\\
0.1054	-0.7993\\
0.1098	-0.8072\\
0.1113	-0.8094\\
0.1163	-0.8164\\
0.1218	-0.8228\\
0.1277	-0.8283\\
0.1339	-0.8331\\
0.1403	-0.8371\\
0.1468	-0.8403\\
}--cycle;

\addplot[area legend, draw=mycolor1, fill=mycolor1, forget plot]
table[row sep=crcr] {%
x	y\\
0.1103	-0.8467\\
0.2394	-0.8467\\
0.2795	-0.8414\\
0.2796	-0.8413\\
0.2801	-0.8412\\
0.3256	-0.8237\\
0.3295	-0.8208\\
0.3315	-0.8191\\
0.335	-0.8159\\
0.345	-0.8059\\
0.3481	-0.8024\\
0.3567	-0.7918\\
0.364	-0.7809\\
0.37	-0.7696\\
0.3746	-0.7582\\
0.3779	-0.7468\\
0.3779	-0.6033\\
0.3749	-0.5942\\
0.3749	-0.594\\
0.3747	-0.5936\\
0.3573	-0.5481\\
0.3536	-0.5396\\
0.3491	-0.5318\\
0.3441	-0.5247\\
0.3425	-0.5227\\
0.337	-0.5164\\
0.3311	-0.5108\\
0.3249	-0.5061\\
0.3185	-0.5021\\
0.312	-0.4989\\
0.1829	-0.4989\\
0.1428	-0.5042\\
0.1427	-0.5043\\
0.1422	-0.5044\\
0.0967	-0.5219\\
0.0927	-0.5248\\
0.0908	-0.5265\\
0.0872	-0.5297\\
0.0772	-0.5397\\
0.0742	-0.5432\\
0.0655	-0.5538\\
0.0582	-0.5647\\
0.0523	-0.576\\
0.0477	-0.5874\\
0.0443	-0.5988\\
0.0443	-0.7423\\
0.0473	-0.7514\\
0.0473	-0.7516\\
0.0475	-0.752\\
0.0649	-0.7975\\
0.0687	-0.806\\
0.0731	-0.8139\\
0.0781	-0.8209\\
0.0798	-0.8229\\
0.0852	-0.8292\\
0.0911	-0.8348\\
0.0973	-0.8395\\
0.1037	-0.8435\\
0.1103	-0.8467\\
}--cycle;

\addplot[area legend, draw=mycolor1, fill=mycolor1, forget plot]
table[row sep=crcr] {%
x	y\\
0.0757	-0.8564\\
0.2054	-0.8564\\
0.2434	-0.8474\\
0.2435	-0.8474\\
0.244	-0.8472\\
0.286	-0.8261\\
0.29	-0.8231\\
0.2935	-0.8199\\
0.3035	-0.8099\\
0.3052	-0.8081\\
0.3083	-0.8046\\
0.3169	-0.794\\
0.3242	-0.783\\
0.3302	-0.7718\\
0.3348	-0.7604\\
0.3381	-0.749\\
0.3381	-0.6043\\
0.3352	-0.5951\\
0.3314	-0.5866\\
0.3314	-0.5865\\
0.3312	-0.586\\
0.31	-0.544\\
0.3056	-0.5362\\
0.3006	-0.5291\\
0.2951	-0.5228\\
0.2933	-0.5211\\
0.2875	-0.5155\\
0.2813	-0.5108\\
0.2749	-0.5068\\
0.2683	-0.5036\\
0.1387	-0.5036\\
0.1006	-0.5126\\
0.1005	-0.5126\\
0.1001	-0.5128\\
0.0581	-0.5339\\
0.0541	-0.5369\\
0.0506	-0.5401\\
0.0388	-0.5519\\
0.0358	-0.5554\\
0.0271	-0.566\\
0.0198	-0.577\\
0.0139	-0.5882\\
0.0093	-0.5996\\
0.0059	-0.611\\
0.0059	-0.7557\\
0.0089	-0.7649\\
0.0126	-0.7734\\
0.0127	-0.7735\\
0.0129	-0.774\\
0.034	-0.816\\
0.0385	-0.8238\\
0.0435	-0.8309\\
0.0489	-0.8372\\
0.0507	-0.8389\\
0.0566	-0.8445\\
0.0628	-0.8492\\
0.0692	-0.8532\\
0.0757	-0.8564\\
}--cycle;

\addplot[area legend, draw=mycolor1, fill=mycolor1, forget plot]
table[row sep=crcr] {%
x	y\\
0.0435	-0.8691\\
0.1736	-0.8691\\
0.2093	-0.8568\\
0.2094	-0.8567\\
0.2098	-0.8565\\
0.2481	-0.8321\\
0.2521	-0.8292\\
0.2556	-0.8259\\
0.2656	-0.8159\\
0.2687	-0.8125\\
0.2773	-0.8019\\
0.2788	-0.8\\
0.2861	-0.789\\
0.292	-0.7778\\
0.2966	-0.7664\\
0.3	-0.755\\
0.3	-0.609\\
0.297	-0.5998\\
0.2933	-0.5913\\
0.2889	-0.5835\\
0.2886	-0.5831\\
0.2885	-0.583\\
0.2642	-0.5446\\
0.2592	-0.5376\\
0.2537	-0.5312\\
0.2478	-0.5257\\
0.2459	-0.5242\\
0.2398	-0.5194\\
0.2334	-0.5155\\
0.2268	-0.5123\\
0.0967	-0.5123\\
0.0611	-0.5246\\
0.061	-0.5246\\
0.0606	-0.5249\\
0.0222	-0.5492\\
0.0183	-0.5522\\
0.0147	-0.5554\\
0.0047	-0.5654\\
0.0017	-0.5689\\
-0.007	-0.5794\\
-0.0085	-0.5813\\
-0.0157	-0.5923\\
-0.0217	-0.6036\\
-0.0263	-0.615\\
-0.0297	-0.6264\\
-0.0297	-0.7724\\
-0.0267	-0.7815\\
-0.0229	-0.79\\
-0.0185	-0.7978\\
-0.0183	-0.7983\\
-0.0182	-0.7984\\
0.0061	-0.8367\\
0.0111	-0.8438\\
0.0166	-0.8501\\
0.0225	-0.8557\\
0.0244	-0.8572\\
0.0306	-0.8619\\
0.037	-0.8659\\
0.0435	-0.8691\\
}--cycle;

\addplot[area legend, draw=mycolor1, fill=mycolor1, forget plot]
table[row sep=crcr] {%
x	y\\
0.0138	-0.8844\\
0.1441	-0.8844\\
0.1772	-0.8691\\
0.1773	-0.8691\\
0.1777	-0.8688\\
0.1816	-0.8658\\
0.2161	-0.8387\\
0.2196	-0.8355\\
0.2296	-0.8255\\
0.2327	-0.822\\
0.2414	-0.8115\\
0.2486	-0.8005\\
0.2499	-0.7985\\
0.2558	-0.7872\\
0.2605	-0.7759\\
0.2638	-0.7645\\
0.2638	-0.6172\\
0.2608	-0.608\\
0.2571	-0.5995\\
0.2527	-0.5917\\
0.2477	-0.5847\\
0.2476	-0.5845\\
0.2473	-0.5842\\
0.2202	-0.5497\\
0.2147	-0.5434\\
0.2089	-0.5378\\
0.2027	-0.5331\\
0.2007	-0.5318\\
0.1943	-0.5278\\
0.1878	-0.5246\\
0.0574	-0.5246\\
0.0243	-0.5399\\
0.0199	-0.5432\\
-0.0146	-0.5703\\
-0.0181	-0.5735\\
-0.0281	-0.5835\\
-0.0312	-0.587\\
-0.0398	-0.5975\\
-0.0471	-0.6085\\
-0.0483	-0.6105\\
-0.0543	-0.6218\\
-0.0589	-0.6331\\
-0.0622	-0.6445\\
-0.0622	-0.7918\\
-0.0593	-0.8009\\
-0.0555	-0.8095\\
-0.0511	-0.8173\\
-0.0461	-0.8243\\
-0.046	-0.8245\\
-0.0458	-0.8248\\
-0.0187	-0.8593\\
-0.0132	-0.8656\\
-0.0073	-0.8712\\
-0.0011	-0.8759\\
0.0008	-0.8772\\
0.0072	-0.8812\\
0.0138	-0.8844\\
}--cycle;

\addplot[area legend, draw=mycolor1, fill=mycolor1, forget plot]
table[row sep=crcr] {%
x	y\\
-0.0133	-0.9019\\
0.1172	-0.9019\\
0.1475	-0.8841\\
0.1515	-0.8811\\
0.1515	-0.881\\
0.1519	-0.8807\\
0.1554	-0.8775\\
0.1859	-0.8481\\
0.1959	-0.8381\\
0.199	-0.8347\\
0.2076	-0.8241\\
0.2149	-0.8131\\
0.2209	-0.8019\\
0.2219	-0.7999\\
0.2265	-0.7885\\
0.2298	-0.7771\\
0.2298	-0.6286\\
0.2269	-0.6194\\
0.2231	-0.6109\\
0.2187	-0.6031\\
0.2137	-0.596\\
0.2082	-0.5897\\
0.2078	-0.5893\\
0.1785	-0.5588\\
0.1726	-0.5532\\
0.1664	-0.5484\\
0.16	-0.5445\\
0.158	-0.5435\\
0.1515	-0.5403\\
0.021	-0.5403\\
-0.0094	-0.5581\\
-0.0133	-0.5611\\
-0.0134	-0.5612\\
-0.0137	-0.5614\\
-0.0172	-0.5647\\
-0.0478	-0.594\\
-0.0578	-0.604\\
-0.0608	-0.6075\\
-0.0695	-0.6181\\
-0.0768	-0.629\\
-0.0827	-0.6403\\
-0.0837	-0.6423\\
-0.0883	-0.6537\\
-0.0917	-0.6651\\
-0.0917	-0.8136\\
-0.0887	-0.8228\\
-0.085	-0.8313\\
-0.0805	-0.8391\\
-0.0755	-0.8461\\
-0.0701	-0.8525\\
-0.0697	-0.8529\\
-0.0403	-0.8834\\
-0.0344	-0.889\\
-0.0283	-0.8937\\
-0.0219	-0.8977\\
-0.0199	-0.8987\\
-0.0133	-0.9019\\
}--cycle;

\addplot[area legend, draw=mycolor1, fill=mycolor1, forget plot]
table[row sep=crcr] {%
x	y\\
-0.0377	-0.9214\\
0.0929	-0.9214\\
0.1203	-0.9013\\
0.1243	-0.8983\\
0.1278	-0.8951\\
0.1378	-0.8851\\
0.1378	-0.885\\
0.1381	-0.8847\\
0.1412	-0.8813\\
0.1677	-0.8501\\
0.1764	-0.8395\\
0.1836	-0.8286\\
0.1896	-0.8173\\
0.1942	-0.8059\\
0.195	-0.8039\\
0.1983	-0.7925\\
0.1983	-0.6428\\
0.1953	-0.6336\\
0.1916	-0.6251\\
0.1872	-0.6173\\
0.1822	-0.6102\\
0.1767	-0.6039\\
0.1708	-0.5984\\
0.1704	-0.598\\
0.1393	-0.5715\\
0.1331	-0.5667\\
0.1267	-0.5628\\
0.1202	-0.5596\\
0.1181	-0.5588\\
-0.0125	-0.5588\\
-0.0399	-0.5789\\
-0.0438	-0.5818\\
-0.0474	-0.5851\\
-0.0574	-0.5951\\
-0.0577	-0.5955\\
-0.0608	-0.5989\\
-0.0873	-0.6301\\
-0.0959	-0.6406\\
-0.1032	-0.6516\\
-0.1092	-0.6629\\
-0.1138	-0.6743\\
-0.1146	-0.6763\\
-0.1179	-0.6877\\
-0.1179	-0.8374\\
-0.1149	-0.8465\\
-0.1112	-0.855\\
-0.1068	-0.8629\\
-0.1018	-0.8699\\
-0.0963	-0.8762\\
-0.0904	-0.8818\\
-0.09	-0.8822\\
-0.0588	-0.9087\\
-0.0527	-0.9134\\
-0.0463	-0.9174\\
-0.0397	-0.9206\\
-0.0377	-0.9214\\
}--cycle;

\addplot[area legend, draw=mycolor1, fill=mycolor1, forget plot]
table[row sep=crcr] {%
x	y\\
-0.0594	-0.9424\\
0.0714	-0.9424\\
0.0754	-0.9395\\
0.0998	-0.9175\\
0.1033	-0.9143\\
0.1133	-0.9043\\
0.1164	-0.9009\\
0.125	-0.8903\\
0.1251	-0.8902\\
0.1253	-0.8899\\
0.1478	-0.8574\\
0.1551	-0.8464\\
0.161	-0.8351\\
0.1656	-0.8237\\
0.169	-0.8124\\
0.1695	-0.8103\\
0.1695	-0.6595\\
0.1666	-0.6503\\
0.1628	-0.6418\\
0.1584	-0.634\\
0.1534	-0.6269\\
0.1479	-0.6206\\
0.142	-0.6151\\
0.1359	-0.6103\\
0.1358	-0.6102\\
0.1354	-0.61\\
0.1029	-0.5875\\
0.0965	-0.5836\\
0.09	-0.5803\\
0.088	-0.5798\\
-0.0428	-0.5798\\
-0.0468	-0.5828\\
-0.0712	-0.6047\\
-0.0747	-0.6079\\
-0.0847	-0.6179\\
-0.0878	-0.6214\\
-0.0964	-0.632\\
-0.0965	-0.632\\
-0.0967	-0.6324\\
-0.1192	-0.6648\\
-0.1265	-0.6758\\
-0.1324	-0.6871\\
-0.137	-0.6985\\
-0.1404	-0.7099\\
-0.1409	-0.7119\\
-0.1409	-0.8627\\
-0.1379	-0.8719\\
-0.1342	-0.8804\\
-0.1298	-0.8882\\
-0.1248	-0.8953\\
-0.1193	-0.9016\\
-0.1134	-0.9072\\
-0.1072	-0.9119\\
-0.1068	-0.9123\\
-0.0743	-0.9347\\
-0.0679	-0.9387\\
-0.0614	-0.9419\\
-0.0594	-0.9424\\
}--cycle;

\addplot[area legend, draw=mycolor1, fill=mycolor1, forget plot]
table[row sep=crcr] {%
x	y\\
-0.0783	-0.9647\\
0.0528	-0.9647\\
0.0568	-0.9618\\
0.0603	-0.9585\\
0.0703	-0.9485\\
0.0917	-0.9251\\
0.0947	-0.9217\\
0.1034	-0.9111\\
0.1106	-0.9001\\
0.1107	-0.9\\
0.1109	-0.8997\\
0.1294	-0.8663\\
0.1353	-0.855\\
0.1399	-0.8436\\
0.1433	-0.8323\\
0.1436	-0.8303\\
0.1436	-0.6783\\
0.1406	-0.6692\\
0.1369	-0.6606\\
0.1325	-0.6528\\
0.1275	-0.6458\\
0.122	-0.6394\\
0.1161	-0.6339\\
0.1099	-0.6291\\
0.1035	-0.6252\\
0.1032	-0.625\\
0.1031	-0.6249\\
0.0697	-0.6064\\
0.0632	-0.6032\\
0.0612	-0.6029\\
-0.0699	-0.6029\\
-0.0739	-0.6059\\
-0.0774	-0.6091\\
-0.0874	-0.6191\\
-0.1087	-0.6425\\
-0.1118	-0.646\\
-0.1205	-0.6566\\
-0.1277	-0.6676\\
-0.1278	-0.6676\\
-0.128	-0.668\\
-0.1464	-0.7013\\
-0.1524	-0.7126\\
-0.157	-0.724\\
-0.1603	-0.7354\\
-0.1607	-0.7374\\
-0.1607	-0.8893\\
-0.1577	-0.8985\\
-0.154	-0.907\\
-0.1495	-0.9148\\
-0.1445	-0.9219\\
-0.1391	-0.9282\\
-0.1332	-0.9337\\
-0.127	-0.9385\\
-0.1206	-0.9425\\
-0.1203	-0.9427\\
-0.1202	-0.9428\\
-0.0868	-0.9612\\
-0.0803	-0.9644\\
-0.0783	-0.9647\\
}--cycle;

\addplot[area legend, draw=mycolor1, fill=mycolor1, forget plot]
table[row sep=crcr] {%
x	y\\
-0.0944	-0.988\\
0.0371	-0.988\\
0.0411	-0.985\\
0.0446	-0.9818\\
0.0546	-0.9718\\
0.0577	-0.9683\\
0.0663	-0.9577\\
0.0846	-0.9332\\
0.0918	-0.9222\\
0.0978	-0.9109\\
0.0978	-0.9108\\
0.098	-0.9105\\
0.1125	-0.8766\\
0.1171	-0.8653\\
0.1205	-0.8539\\
0.1206	-0.8519\\
0.1206	-0.6989\\
0.1176	-0.6898\\
0.1139	-0.6813\\
0.1094	-0.6735\\
0.1044	-0.6664\\
0.099	-0.6601\\
0.0931	-0.6545\\
0.0869	-0.6498\\
0.0805	-0.6458\\
0.074	-0.6426\\
0.0739	-0.6425\\
0.0735	-0.6424\\
0.0397	-0.6278\\
0.0377	-0.6277\\
-0.0938	-0.6277\\
-0.0977	-0.6307\\
-0.1013	-0.6339\\
-0.1113	-0.6439\\
-0.1143	-0.6474\\
-0.123	-0.6579\\
-0.1412	-0.6825\\
-0.1485	-0.6935\\
-0.1544	-0.7048\\
-0.1545	-0.7048\\
-0.1546	-0.7052\\
-0.1691	-0.739\\
-0.1738	-0.7504\\
-0.1771	-0.7618\\
-0.1772	-0.7638\\
-0.1772	-0.9167\\
-0.1742	-0.9259\\
-0.1705	-0.9344\\
-0.1661	-0.9422\\
-0.1611	-0.9493\\
-0.1556	-0.9556\\
-0.1497	-0.9612\\
-0.1435	-0.9659\\
-0.1372	-0.9699\\
-0.1306	-0.9731\\
-0.1305	-0.9732\\
-0.1302	-0.9733\\
-0.0963	-0.9878\\
-0.0944	-0.988\\
}--cycle;

\addplot[area legend, draw=mycolor1, fill=mycolor1, forget plot]
table[row sep=crcr] {%
x	y\\
-0.1096	-1.0119\\
0.0224	-1.0119\\
0.0243	-1.0118\\
0.0282	-1.0088\\
0.0318	-1.0056\\
0.0418	-0.9956\\
0.0448	-0.9921\\
0.0535	-0.9816\\
0.0608	-0.9706\\
0.0759	-0.9452\\
0.0818	-0.934\\
0.0865	-0.9226\\
0.0866	-0.9222\\
0.0866	-0.9221\\
0.0973	-0.8882\\
0.1006	-0.8768\\
0.1006	-0.721\\
0.0976	-0.7118\\
0.0938	-0.7033\\
0.0894	-0.6955\\
0.0844	-0.6884\\
0.0789	-0.6821\\
0.0731	-0.6766\\
0.0669	-0.6718\\
0.0605	-0.6678\\
0.054	-0.6646\\
0.0539	-0.6646\\
0.0535	-0.6645\\
0.0196	-0.6538\\
-0.1124	-0.6538\\
-0.1143	-0.6539\\
-0.1182	-0.6568\\
-0.1218	-0.6601\\
-0.1318	-0.6701\\
-0.1348	-0.6735\\
-0.1435	-0.6841\\
-0.1508	-0.6951\\
-0.1659	-0.7204\\
-0.1718	-0.7317\\
-0.1765	-0.7431\\
-0.1766	-0.7434\\
-0.1766	-0.7435\\
-0.1873	-0.7774\\
-0.1906	-0.7888\\
-0.1906	-0.9446\\
-0.1876	-0.9538\\
-0.1838	-0.9623\\
-0.1794	-0.9701\\
-0.1744	-0.9772\\
-0.1689	-0.9835\\
-0.1631	-0.9891\\
-0.1569	-0.9938\\
-0.1505	-0.9978\\
-0.144	-1.001\\
-0.1439	-1.0011\\
-0.1435	-1.0012\\
-0.1096	-1.0119\\
}--cycle;

\addplot[area legend, draw=mycolor1, fill=mycolor1, forget plot]
table[row sep=crcr] {%
x	y\\
-0.1201	-1.0362\\
0.0124	-1.0362\\
0.0142	-1.0359\\
0.0182	-1.033\\
0.0217	-1.0297\\
0.0317	-1.0197\\
0.0348	-1.0163\\
0.0434	-1.0057\\
0.0507	-0.9947\\
0.0566	-0.9834\\
0.0687	-0.9576\\
0.0734	-0.9462\\
0.0767	-0.9348\\
0.0768	-0.9344\\
0.0838	-0.9008\\
0.0838	-0.746\\
0.0836	-0.7442\\
0.0806	-0.735\\
0.0768	-0.7265\\
0.0724	-0.7187\\
0.0674	-0.7116\\
0.0619	-0.7053\\
0.056	-0.6997\\
0.0499	-0.695\\
0.0435	-0.691\\
0.0369	-0.6878\\
0.0368	-0.6878\\
0.0365	-0.6877\\
0.0029	-0.6807\\
-0.1296	-0.6807\\
-0.1314	-0.6809\\
-0.1354	-0.6839\\
-0.1389	-0.6871\\
-0.1489	-0.6971\\
-0.152	-0.7006\\
-0.1606	-0.7112\\
-0.1679	-0.7222\\
-0.1739	-0.7334\\
-0.186	-0.7592\\
-0.1906	-0.7706\\
-0.1939	-0.782\\
-0.194	-0.7821\\
-0.194	-0.7825\\
-0.201	-0.8161\\
-0.201	-0.9709\\
-0.2008	-0.9727\\
-0.1978	-0.9819\\
-0.1941	-0.9904\\
-0.1896	-0.9982\\
-0.1846	-1.0053\\
-0.1792	-1.0116\\
-0.1733	-1.0171\\
-0.1671	-1.0219\\
-0.1607	-1.0259\\
-0.1542	-1.0291\\
-0.1541	-1.0291\\
-0.1537	-1.0292\\
-0.1201	-1.0362\\
}--cycle;

\addplot[area legend, draw=mycolor1, fill=mycolor1, forget plot]
table[row sep=crcr] {%
x	y\\
-0.128	-1.0605\\
0.0051	-1.0605\\
0.0068	-1.0601\\
0.0108	-1.0571\\
0.0143	-1.0539\\
0.0243	-1.0439\\
0.0274	-1.0404\\
0.0361	-1.0299\\
0.0433	-1.0189\\
0.0493	-1.0076\\
0.0539	-0.9962\\
0.063	-0.9702\\
0.0664	-0.9589\\
0.0664	-0.9584\\
0.0699	-0.9255\\
0.0699	-0.7698\\
0.0695	-0.7681\\
0.0665	-0.759\\
0.0628	-0.7504\\
0.0584	-0.7426\\
0.0534	-0.7355\\
0.0479	-0.7292\\
0.042	-0.7237\\
0.0358	-0.7189\\
0.0294	-0.7149\\
0.0229	-0.7117\\
0.0225	-0.7117\\
-0.0105	-0.7082\\
-0.1436	-0.7082\\
-0.1453	-0.7086\\
-0.1493	-0.7116\\
-0.1528	-0.7148\\
-0.1628	-0.7248\\
-0.1659	-0.7283\\
-0.1745	-0.7388\\
-0.1818	-0.7498\\
-0.1878	-0.7611\\
-0.1924	-0.7725\\
-0.2015	-0.7985\\
-0.2049	-0.8099\\
-0.2049	-0.8103\\
-0.2084	-0.8432\\
-0.2084	-0.9989\\
-0.208	-1.0006\\
-0.205	-1.0098\\
-0.2013	-1.0183\\
-0.1968	-1.0261\\
-0.1918	-1.0332\\
-0.1864	-1.0395\\
-0.1805	-1.045\\
-0.1743	-1.0498\\
-0.1679	-1.0538\\
-0.1614	-1.057\\
-0.1609	-1.057\\
-0.128	-1.0605\\
}--cycle;

\addplot[area legend, draw=mycolor1, fill=mycolor1, forget plot]
table[row sep=crcr] {%
x	y\\
-0.1332	-1.0846\\
0.0005	-1.0846\\
0.0021	-1.084\\
0.0061	-1.0811\\
0.0096	-1.0778\\
0.0196	-1.0678\\
0.0227	-1.0644\\
0.0313	-1.0538\\
0.0386	-1.0428\\
0.0445	-1.0315\\
0.0492	-1.0201\\
0.0525	-1.0088\\
0.0588	-0.9829\\
0.0588	-0.9825\\
0.059	-0.9505\\
0.059	-0.7941\\
0.056	-0.785\\
0.0554	-0.7834\\
0.0517	-0.7749\\
0.0473	-0.767\\
0.0423	-0.76\\
0.0368	-0.7536\\
0.0309	-0.7481\\
0.0247	-0.7433\\
0.0183	-0.7394\\
0.0118	-0.7362\\
0.0117	-0.7361\\
0.0114	-0.7361\\
-0.0206	-0.736\\
-0.1544	-0.736\\
-0.156	-0.7365\\
-0.1599	-0.7395\\
-0.1635	-0.7427\\
-0.1735	-0.7527\\
-0.1765	-0.7562\\
-0.1852	-0.7668\\
-0.1925	-0.7778\\
-0.1984	-0.789\\
-0.203	-0.8004\\
-0.2064	-0.8118\\
-0.2127	-0.8376\\
-0.2127	-0.8381\\
-0.2128	-0.8701\\
-0.2128	-1.0264\\
-0.2099	-1.0356\\
-0.2093	-1.0372\\
-0.2056	-1.0457\\
-0.2011	-1.0535\\
-0.1962	-1.0606\\
-0.1907	-1.0669\\
-0.1848	-1.0725\\
-0.1786	-1.0772\\
-0.1722	-1.0812\\
-0.1657	-1.0844\\
-0.1652	-1.0844\\
-0.1332	-1.0846\\
}--cycle;

\addplot[area legend, draw=mycolor1, fill=mycolor1, forget plot]
table[row sep=crcr] {%
x	y\\
-0.1672	-1.1111\\
-0.0324	-1.1111\\
-0.0016	-1.1082\\
-0.0001	-1.1075\\
0.0038	-1.1045\\
0.0073	-1.1013\\
0.0174	-1.0913\\
0.0204	-1.0878\\
0.0291	-1.0772\\
0.0364	-1.0663\\
0.0423	-1.055\\
0.0469	-1.0436\\
0.0503	-1.0322\\
0.0538	-1.0068\\
0.0538	-0.8494\\
0.0509	-0.8186\\
0.0479	-0.8094\\
0.0441	-0.8009\\
0.0434	-0.7994\\
0.039	-0.7916\\
0.034	-0.7846\\
0.0285	-0.7782\\
0.0227	-0.7727\\
0.0165	-0.7679\\
0.0101	-0.7639\\
0.0036	-0.7607\\
-0.131	-0.7607\\
-0.1313	-0.7608\\
-0.1621	-0.7637\\
-0.1635	-0.7644\\
-0.1675	-0.7674\\
-0.171	-0.7706\\
-0.181	-0.7806\\
-0.1841	-0.784\\
-0.1927	-0.7946\\
-0.2	-0.8056\\
-0.206	-0.8169\\
-0.2106	-0.8283\\
-0.2139	-0.8397\\
-0.2175	-0.8651\\
-0.2175	-1.0225\\
-0.2145	-1.0533\\
-0.2115	-1.0624\\
-0.2078	-1.0709\\
-0.2071	-1.0724\\
-0.2027	-1.0802\\
-0.1977	-1.0873\\
-0.1922	-1.0936\\
-0.1863	-1.0992\\
-0.1801	-1.1039\\
-0.1738	-1.1079\\
-0.1672	-1.1111\\
}--cycle;

\addplot[area legend, draw=mycolor1, fill=mycolor1, forget plot]
table[row sep=crcr] {%
x	y\\
-0.1662	-1.1369\\
-0.031	-1.1369\\
-0.0307	-1.1368\\
-0.0014	-1.131\\
-0	-1.1302\\
0.0039	-1.1273\\
0.0074	-1.124\\
0.0174	-1.114\\
0.0205	-1.1106\\
0.0292	-1.1\\
0.0365	-1.089\\
0.0424	-1.0777\\
0.047	-1.0663\\
0.0504	-1.055\\
0.0513	-1.0302\\
0.0513	-0.8722\\
0.0455	-0.8429\\
0.0425	-0.8338\\
0.0387	-0.8253\\
0.0343	-0.8175\\
0.0335	-0.8161\\
0.0285	-0.809\\
0.0231	-0.8027\\
0.0172	-0.7972\\
0.011	-0.7924\\
0.0046	-0.7884\\
-0.0019	-0.7852\\
-0.1372	-0.7852\\
-0.1375	-0.7853\\
-0.1668	-0.7911\\
-0.1681	-0.7919\\
-0.1721	-0.7948\\
-0.1756	-0.7981\\
-0.1856	-0.8081\\
-0.1887	-0.8115\\
-0.1973	-0.8221\\
-0.2046	-0.8331\\
-0.2105	-0.8444\\
-0.2152	-0.8558\\
-0.2185	-0.8671\\
-0.2195	-0.8919\\
-0.2195	-1.0495\\
-0.2194	-1.0498\\
-0.2194	-1.0499\\
-0.2136	-1.0792\\
-0.2106	-1.0883\\
-0.2069	-1.0968\\
-0.2025	-1.1046\\
-0.2017	-1.106\\
-0.1967	-1.1131\\
-0.1912	-1.1194\\
-0.1853	-1.1249\\
-0.1791	-1.1297\\
-0.1727	-1.1337\\
-0.1662	-1.1369\\
}--cycle;

\addplot[area legend, draw=mycolor1, fill=mycolor1, forget plot]
table[row sep=crcr] {%
x	y\\
-0.1628	-1.1615\\
-0.0269	-1.1615\\
-0.0268	-1.1614\\
-0.0265	-1.1614\\
0.001	-1.153\\
0.0023	-1.1521\\
0.0062	-1.1491\\
0.0097	-1.1459\\
0.0197	-1.1359\\
0.0228	-1.1324\\
0.0314	-1.1219\\
0.0387	-1.1108\\
0.0447	-1.0996\\
0.0493	-1.0882\\
0.0526	-1.0768\\
0.0526	-0.9188\\
0.0512	-0.8949\\
0.0511	-0.8946\\
0.0511	-0.8945\\
0.0427	-0.8669\\
0.0397	-0.8578\\
0.036	-0.8493\\
0.0316	-0.8415\\
0.0266	-0.8344\\
0.0257	-0.8331\\
0.0202	-0.8268\\
0.0143	-0.8213\\
0.0081	-0.8165\\
0.0017	-0.8125\\
-0.0048	-0.8093\\
-0.1407	-0.8093\\
-0.1408	-0.8094\\
-0.141	-0.8094\\
-0.1686	-0.8178\\
-0.1699	-0.8187\\
-0.1738	-0.8217\\
-0.1773	-0.8249\\
-0.1873	-0.8349\\
-0.1904	-0.8384\\
-0.199	-0.8489\\
-0.2063	-0.8599\\
-0.2123	-0.8712\\
-0.2169	-0.8826\\
-0.2202	-0.894\\
-0.2202	-1.052\\
-0.2188	-1.0759\\
-0.2187	-1.0762\\
-0.2187	-1.0763\\
-0.2103	-1.1039\\
-0.2073	-1.113\\
-0.2036	-1.1215\\
-0.1992	-1.1293\\
-0.1942	-1.1364\\
-0.1933	-1.1376\\
-0.1878	-1.144\\
-0.1819	-1.1495\\
-0.1757	-1.1543\\
-0.1693	-1.1583\\
-0.1628	-1.1615\\
}--cycle;

\addplot[area legend, draw=mycolor1, fill=mycolor1, forget plot]
table[row sep=crcr] {%
x	y\\
-0.1572	-1.1847\\
-0.0206	-1.1847\\
-0.0205	-1.1846\\
-0.0202	-1.1845\\
0.0055	-1.1738\\
0.0094	-1.1709\\
0.0105	-1.1699\\
0.014	-1.1666\\
0.024	-1.1566\\
0.0271	-1.1532\\
0.0357	-1.1426\\
0.043	-1.1316\\
0.049	-1.1204\\
0.0536	-1.109\\
0.0569	-1.0976\\
0.0569	-0.9391\\
0.0532	-0.9163\\
0.0502	-0.9072\\
0.0502	-0.9071\\
0.0501	-0.9068\\
0.0394	-0.8812\\
0.0357	-0.8726\\
0.0313	-0.8648\\
0.0263	-0.8578\\
0.0208	-0.8514\\
0.0198	-0.8503\\
0.0139	-0.8448\\
0.0077	-0.84\\
0.0013	-0.8361\\
-0.0052	-0.8329\\
-0.1419	-0.8329\\
-0.1421	-0.833\\
-0.1678	-0.8437\\
-0.1718	-0.8467\\
-0.1729	-0.8476\\
-0.1764	-0.8509\\
-0.1864	-0.8609\\
-0.1895	-0.8643\\
-0.1981	-0.8749\\
-0.2054	-0.8859\\
-0.2113	-0.8972\\
-0.216	-0.9086\\
-0.2193	-0.92\\
-0.2193	-1.0784\\
-0.2156	-1.1012\\
-0.2126	-1.1104\\
-0.2125	-1.1107\\
-0.2018	-1.1364\\
-0.198	-1.1449\\
-0.1936	-1.1527\\
-0.1886	-1.1598\\
-0.1831	-1.1661\\
-0.1822	-1.1672\\
-0.1763	-1.1727\\
-0.1701	-1.1775\\
-0.1637	-1.1815\\
-0.1572	-1.1847\\
}--cycle;

\addplot[area legend, draw=mycolor1, fill=mycolor1, forget plot]
table[row sep=crcr] {%
x	y\\
-0.1496	-1.2064\\
-0.0122	-1.2064\\
-0.0122	-1.2063\\
-0.0119	-1.2062\\
0.0117	-1.1934\\
0.0157	-1.1905\\
0.0192	-1.1872\\
0.0302	-1.1762\\
0.0332	-1.1727\\
0.0419	-1.1622\\
0.0492	-1.1512\\
0.0551	-1.1399\\
0.0597	-1.1285\\
0.0631	-1.1171\\
0.0631	-0.9584\\
0.0573	-0.9368\\
0.0543	-0.9277\\
0.0506	-0.9192\\
0.0505	-0.9189\\
0.0504	-0.9189\\
0.0377	-0.8952\\
0.0332	-0.8874\\
0.0282	-0.8803\\
0.0228	-0.874\\
0.0169	-0.8685\\
0.0158	-0.8675\\
0.0097	-0.8627\\
0.0033	-0.8588\\
-0.0033	-0.8556\\
-0.1407	-0.8556\\
-0.1409	-0.8557\\
-0.1646	-0.8685\\
-0.1685	-0.8714\\
-0.172	-0.8747\\
-0.183	-0.8857\\
-0.1861	-0.8892\\
-0.1947	-0.8998\\
-0.202	-0.9107\\
-0.2079	-0.922\\
-0.2126	-0.9334\\
-0.2159	-0.9448\\
-0.2159	-1.1035\\
-0.2101	-1.1251\\
-0.2072	-1.1343\\
-0.2034	-1.1428\\
-0.2033	-1.143\\
-0.2033	-1.1431\\
-0.1905	-1.1667\\
-0.1861	-1.1745\\
-0.1811	-1.1816\\
-0.1756	-1.1879\\
-0.1697	-1.1935\\
-0.1687	-1.1944\\
-0.1625	-1.1992\\
-0.1561	-1.2032\\
-0.1496	-1.2064\\
}--cycle;

\addplot[area legend, draw=mycolor1, fill=mycolor1, forget plot]
table[row sep=crcr] {%
x	y\\
-0.1402	-1.2264\\
-0.0021	-1.2264\\
-0.0018	-1.2262\\
0.0196	-1.2117\\
0.0236	-1.2087\\
0.0271	-1.2055\\
0.0371	-1.1955\\
0.0402	-1.192\\
0.0488	-1.1815\\
0.0496	-1.1804\\
0.0569	-1.1694\\
0.0629	-1.1581\\
0.0675	-1.1467\\
0.0708	-1.1353\\
0.0708	-0.9764\\
0.0679	-0.9672\\
0.0602	-0.947\\
0.0565	-0.9385\\
0.0521	-0.9307\\
0.0519	-0.9305\\
0.0519	-0.9304\\
0.0374	-0.909\\
0.0324	-0.9019\\
0.0269	-0.8956\\
0.021	-0.89\\
0.0148	-0.8853\\
0.0137	-0.8844\\
0.0073	-0.8805\\
0.0008	-0.8773\\
-0.1373	-0.8772\\
-0.1373	-0.8773\\
-0.1376	-0.8774\\
-0.1591	-0.892\\
-0.163	-0.895\\
-0.1665	-0.8982\\
-0.1765	-0.9082\\
-0.1796	-0.9116\\
-0.1882	-0.9222\\
-0.1891	-0.9233\\
-0.1963	-0.9343\\
-0.2023	-0.9456\\
-0.2069	-0.957\\
-0.2103	-0.9684\\
-0.2103	-1.1273\\
-0.2073	-1.1365\\
-0.1997	-1.1566\\
-0.1959	-1.1651\\
-0.1915	-1.1729\\
-0.1914	-1.1732\\
-0.1913	-1.1732\\
-0.1768	-1.1947\\
-0.1718	-1.2018\\
-0.1663	-1.2081\\
-0.1604	-1.2136\\
-0.1542	-1.2184\\
-0.1531	-1.2192\\
-0.1467	-1.2232\\
-0.1402	-1.2264\\
}--cycle;

\addplot[area legend, draw=mycolor1, fill=mycolor1, forget plot]
table[row sep=crcr] {%
x	y\\
-0.1293	-1.2447\\
0.0095	-1.2447\\
0.0134	-1.2417\\
0.0135	-1.2417\\
0.0137	-1.2415\\
0.033	-1.2255\\
0.0365	-1.2222\\
0.0465	-1.2122\\
0.0495	-1.2088\\
0.0582	-1.1982\\
0.0655	-1.1872\\
0.0661	-1.1861\\
0.0721	-1.1748\\
0.0767	-1.1634\\
0.0801	-1.152\\
0.0801	-0.9929\\
0.0771	-0.9838\\
0.0733	-0.9753\\
0.0641	-0.9566\\
0.0596	-0.9488\\
0.0547	-0.9418\\
0.0545	-0.9415\\
0.0384	-0.9223\\
0.0329	-0.9159\\
0.0271	-0.9104\\
0.0209	-0.9056\\
0.0145	-0.9017\\
0.0133	-0.901\\
0.0068	-0.8978\\
-0.132	-0.8978\\
-0.1359	-0.9007\\
-0.1362	-0.901\\
-0.1554	-0.917\\
-0.159	-0.9202\\
-0.169	-0.9302\\
-0.172	-0.9337\\
-0.1807	-0.9443\\
-0.188	-0.9552\\
-0.1886	-0.9564\\
-0.1946	-0.9676\\
-0.1992	-0.979\\
-0.2025	-0.9904\\
-0.2025	-1.1495\\
-0.1996	-1.1587\\
-0.1958	-1.1672\\
-0.1866	-1.1858\\
-0.1821	-1.1936\\
-0.1771	-1.2007\\
-0.177	-1.2009\\
-0.177	-1.201\\
-0.1609	-1.2202\\
-0.1554	-1.2265\\
-0.1495	-1.2321\\
-0.1434	-1.2368\\
-0.137	-1.2408\\
-0.1358	-1.2415\\
-0.1293	-1.2447\\
}--cycle;

\addplot[area legend, draw=mycolor1, fill=mycolor1, forget plot]
table[row sep=crcr] {%
x	y\\
-0.1171	-1.2611\\
0.0224	-1.2611\\
0.0264	-1.2582\\
0.0299	-1.2549\\
0.0401	-1.2447\\
0.0571	-1.2274\\
0.0601	-1.224\\
0.0688	-1.2134\\
0.0761	-1.2024\\
0.082	-1.1912\\
0.0826	-1.19\\
0.0872	-1.1786\\
0.0905	-1.1672\\
0.0905	-1.0081\\
0.0875	-0.9989\\
0.0838	-0.9904\\
0.0794	-0.9826\\
0.0687	-0.9656\\
0.0637	-0.9585\\
0.0582	-0.9522\\
0.058	-0.952\\
0.0407	-0.935\\
0.0348	-0.9295\\
0.0287	-0.9247\\
0.0223	-0.9207\\
0.0157	-0.9175\\
0.0146	-0.917\\
-0.1249	-0.917\\
-0.1289	-0.92\\
-0.1324	-0.9232\\
-0.1426	-0.9334\\
-0.1596	-0.9507\\
-0.1626	-0.9541\\
-0.1713	-0.9647\\
-0.1786	-0.9757\\
-0.1845	-0.987\\
-0.185	-0.9881\\
-0.1897	-0.9995\\
-0.193	-1.0109\\
-0.193	-1.1701\\
-0.19	-1.1792\\
-0.1863	-1.1877\\
-0.1819	-1.1955\\
-0.1712	-1.2125\\
-0.1662	-1.2196\\
-0.1607	-1.2259\\
-0.1605	-1.2261\\
-0.1605	-1.2262\\
-0.1432	-1.2431\\
-0.1373	-1.2487\\
-0.1312	-1.2534\\
-0.1248	-1.2574\\
-0.1182	-1.2606\\
-0.1171	-1.2611\\
}--cycle;

\addplot[area legend, draw=mycolor1, fill=mycolor1, forget plot]
table[row sep=crcr] {%
x	y\\
-0.1037	-1.2757\\
0.0364	-1.2757\\
0.0404	-1.2728\\
0.0439	-1.2695\\
0.0539	-1.2595\\
0.057	-1.2561\\
0.057	-1.256\\
0.0572	-1.2558\\
0.0658	-1.2453\\
0.0804	-1.227\\
0.0877	-1.216\\
0.0937	-1.2048\\
0.0983	-1.1934\\
0.0987	-1.1922\\
0.102	-1.1808\\
0.102	-1.0216\\
0.0991	-1.0124\\
0.0953	-1.0039\\
0.0909	-0.9961\\
0.0859	-0.989\\
0.074	-0.9737\\
0.0685	-0.9674\\
0.0626	-0.9619\\
0.0624	-0.9617\\
0.0442	-0.9471\\
0.038	-0.9423\\
0.0316	-0.9383\\
0.0251	-0.9351\\
0.0239	-0.9347\\
-0.1163	-0.9347\\
-0.1202	-0.9377\\
-0.1237	-0.9409\\
-0.1337	-0.9509\\
-0.1368	-0.9544\\
-0.1369	-0.9544\\
-0.137	-0.9546\\
-0.1457	-0.9652\\
-0.1603	-0.9834\\
-0.1676	-0.9944\\
-0.1735	-1.0057\\
-0.1781	-1.0171\\
-0.1785	-1.0182\\
-0.1819	-1.0296\\
-0.1819	-1.1889\\
-0.1789	-1.198\\
-0.1752	-1.2065\\
-0.1707	-1.2143\\
-0.1657	-1.2214\\
-0.1538	-1.2367\\
-0.1483	-1.243\\
-0.1425	-1.2486\\
-0.1423	-1.2487\\
-0.1422	-1.2488\\
-0.124	-1.2634\\
-0.1178	-1.2681\\
-0.1114	-1.2721\\
-0.1049	-1.2753\\
-0.1037	-1.2757\\
}--cycle;

\addplot[area legend, draw=mycolor1, fill=mycolor1, forget plot]
table[row sep=crcr] {%
x	y\\
-0.0895	-1.2885\\
0.0513	-1.2885\\
0.0553	-1.2855\\
0.0588	-1.2823\\
0.0688	-1.2723\\
0.0718	-1.2688\\
0.0805	-1.2582\\
0.0807	-1.258\\
0.088	-1.247\\
0.1002	-1.2281\\
0.1062	-1.2168\\
0.1108	-1.2054\\
0.1141	-1.194\\
0.1144	-1.1929\\
0.1144	-1.0335\\
0.1114	-1.0243\\
0.1077	-1.0158\\
0.1033	-1.008\\
0.0983	-1.001\\
0.0928	-0.9946\\
0.0799	-0.9811\\
0.074	-0.9755\\
0.0678	-0.9708\\
0.0676	-0.9706\\
0.0487	-0.9583\\
0.0423	-0.9543\\
0.0357	-0.9511\\
0.0346	-0.9509\\
-0.1062	-0.9509\\
-0.1102	-0.9538\\
-0.1137	-0.9571\\
-0.1237	-0.9671\\
-0.1268	-0.9705\\
-0.1354	-0.9811\\
-0.1355	-0.9811\\
-0.1356	-0.9813\\
-0.1429	-0.9923\\
-0.1552	-1.0112\\
-0.1611	-1.0225\\
-0.1658	-1.0339\\
-0.1691	-1.0453\\
-0.1694	-1.0465\\
-0.1694	-1.2058\\
-0.1664	-1.215\\
-0.1627	-1.2235\\
-0.1582	-1.2313\\
-0.1532	-1.2384\\
-0.1478	-1.2447\\
-0.1348	-1.2583\\
-0.129	-1.2638\\
-0.1228	-1.2686\\
-0.1226	-1.2687\\
-0.1225	-1.2687\\
-0.1036	-1.281\\
-0.0972	-1.285\\
-0.0907	-1.2882\\
-0.0895	-1.2885\\
}--cycle;

\addplot[area legend, draw=mycolor1, fill=mycolor1, forget plot]
table[row sep=crcr] {%
x	y\\
-0.0746	-1.2993\\
0.0668	-1.2993\\
0.0708	-1.2964\\
0.0743	-1.2931\\
0.0843	-1.2831\\
0.0874	-1.2797\\
0.096	-1.2691\\
0.1033	-1.2581\\
0.1034	-1.2579\\
0.1094	-1.2466\\
0.1193	-1.2272\\
0.124	-1.2158\\
0.1273	-1.2044\\
0.1275	-1.2033\\
0.1275	-1.0438\\
0.1245	-1.0346\\
0.1207	-1.0261\\
0.1163	-1.0183\\
0.1113	-1.0112\\
0.1058	-1.0049\\
0.1	-0.9993\\
0.0862	-0.9876\\
0.0801	-0.9828\\
0.0737	-0.9788\\
0.0735	-0.9787\\
0.0734	-0.9787\\
0.0541	-0.9687\\
0.0475	-0.9655\\
0.0464	-0.9653\\
-0.0951	-0.9653\\
-0.099	-0.9683\\
-0.1025	-0.9715\\
-0.1125	-0.9815\\
-0.1156	-0.985\\
-0.1243	-0.9956\\
-0.1315	-1.0066\\
-0.1316	-1.0066\\
-0.1317	-1.0068\\
-0.1376	-1.0181\\
-0.1476	-1.0374\\
-0.1522	-1.0488\\
-0.1556	-1.0602\\
-0.1557	-1.0614\\
-0.1557	-1.2209\\
-0.1528	-1.2301\\
-0.149	-1.2386\\
-0.1446	-1.2464\\
-0.1396	-1.2535\\
-0.1341	-1.2598\\
-0.1282	-1.2653\\
-0.1145	-1.2771\\
-0.1083	-1.2819\\
-0.1019	-1.2859\\
-0.1017	-1.286\\
-0.0823	-1.296\\
-0.0758	-1.2992\\
-0.0746	-1.2993\\
}--cycle;

\addplot[area legend, draw=mycolor1, fill=mycolor1, forget plot]
table[row sep=crcr] {%
x	y\\
-0.0593	-1.3083\\
0.0827	-1.3083\\
0.0867	-1.3054\\
0.0902	-1.3021\\
0.1002	-1.2921\\
0.1033	-1.2887\\
0.1119	-1.2781\\
0.1192	-1.2671\\
0.1251	-1.2559\\
0.1252	-1.2558\\
0.1253	-1.2556\\
0.1299	-1.2442\\
0.1376	-1.2246\\
0.1409	-1.2132\\
0.141	-1.2121\\
0.141	-1.0523\\
0.138	-1.0432\\
0.1343	-1.0347\\
0.1298	-1.0269\\
0.1249	-1.0198\\
0.1194	-1.0135\\
0.1135	-1.0079\\
0.1073	-1.0032\\
0.093	-0.9932\\
0.0866	-0.9892\\
0.0801	-0.986\\
0.08	-0.986\\
0.0798	-0.9859\\
0.0602	-0.9782\\
0.0591	-0.9781\\
-0.0829	-0.9781\\
-0.0869	-0.9811\\
-0.0904	-0.9843\\
-0.1004	-0.9943\\
-0.1035	-0.9978\\
-0.1121	-1.0083\\
-0.1194	-1.0193\\
-0.1253	-1.0306\\
-0.1254	-1.0307\\
-0.1254	-1.0309\\
-0.1301	-1.0422\\
-0.1378	-1.0618\\
-0.1411	-1.0732\\
-0.1412	-1.0743\\
-0.1412	-1.2341\\
-0.1382	-1.2433\\
-0.1344	-1.2518\\
-0.13	-1.2596\\
-0.125	-1.2667\\
-0.1195	-1.273\\
-0.1137	-1.2785\\
-0.1075	-1.2833\\
-0.0932	-1.2933\\
-0.0868	-1.2973\\
-0.0802	-1.3005\\
-0.08	-1.3006\\
-0.0604	-1.3083\\
-0.0593	-1.3083\\
}--cycle;

\addplot[area legend, draw=mycolor1, fill=mycolor1, forget plot]
table[row sep=crcr] {%
x	y\\
-0.0448	-1.3156\\
0.0978	-1.3156\\
0.0989	-1.3155\\
0.1028	-1.3125\\
0.1064	-1.3093\\
0.1164	-1.2993\\
0.1194	-1.2958\\
0.1281	-1.2853\\
0.1354	-1.2743\\
0.1413	-1.263\\
0.1459	-1.2516\\
0.146	-1.2514\\
0.1494	-1.24\\
0.1549	-1.2204\\
0.1549	-1.0604\\
0.1548	-1.0593\\
0.1518	-1.0501\\
0.1481	-1.0416\\
0.1437	-1.0338\\
0.1387	-1.0267\\
0.1332	-1.0204\\
0.1273	-1.0149\\
0.1211	-1.0101\\
0.1147	-1.0061\\
0.1	-0.9979\\
0.0935	-0.9947\\
0.0934	-0.9947\\
0.0932	-0.9946\\
0.0737	-0.9891\\
-0.0689	-0.9891\\
-0.07	-0.9892\\
-0.0739	-0.9921\\
-0.0775	-0.9954\\
-0.0875	-1.0054\\
-0.0905	-1.0088\\
-0.0992	-1.0194\\
-0.1064	-1.0304\\
-0.1124	-1.0416\\
-0.117	-1.053\\
-0.117	-1.0531\\
-0.1171	-1.0533\\
-0.1204	-1.0647\\
-0.126	-1.0842\\
-0.126	-1.2443\\
-0.1259	-1.2454\\
-0.1229	-1.2545\\
-0.1192	-1.263\\
-0.1148	-1.2709\\
-0.1098	-1.2779\\
-0.1043	-1.2843\\
-0.0984	-1.2898\\
-0.0922	-1.2946\\
-0.0858	-1.2985\\
-0.0711	-1.3068\\
-0.0646	-1.31\\
-0.0643	-1.31\\
-0.0448	-1.3156\\
}--cycle;

\addplot[area legend, draw=mycolor1, fill=mycolor1, forget plot]
table[row sep=crcr] {%
x	y\\
-0.0291	-1.3211\\
0.114	-1.3211\\
0.1151	-1.3209\\
0.119	-1.3179\\
0.1225	-1.3147\\
0.1325	-1.3047\\
0.1356	-1.3012\\
0.1443	-1.2907\\
0.1515	-1.2797\\
0.1575	-1.2684\\
0.1621	-1.257\\
0.1654	-1.2456\\
0.1655	-1.2456\\
0.1655	-1.2454\\
0.1689	-1.226\\
0.1689	-1.0657\\
0.1688	-1.0646\\
0.1658	-1.0555\\
0.162	-1.047\\
0.1576	-1.0392\\
0.1526	-1.0321\\
0.1471	-1.0258\\
0.1413	-1.0202\\
0.1351	-1.0155\\
0.1287	-1.0115\\
0.1222	-1.0083\\
0.1072	-1.0018\\
0.107	-1.0017\\
0.0877	-0.9983\\
-0.0554	-0.9983\\
-0.0564	-0.9985\\
-0.0604	-1.0015\\
-0.0639	-1.0047\\
-0.0739	-1.0147\\
-0.077	-1.0181\\
-0.0856	-1.0287\\
-0.0929	-1.0397\\
-0.0989	-1.051\\
-0.1035	-1.0624\\
-0.1068	-1.0737\\
-0.1068	-1.0738\\
-0.1069	-1.074\\
-0.1103	-1.0933\\
-0.1103	-1.2537\\
-0.1102	-1.2547\\
-0.1072	-1.2639\\
-0.1034	-1.2724\\
-0.099	-1.2802\\
-0.094	-1.2873\\
-0.0885	-1.2936\\
-0.0827	-1.2991\\
-0.0765	-1.3039\\
-0.0701	-1.3079\\
-0.0636	-1.3111\\
-0.0486	-1.3176\\
-0.0484	-1.3176\\
-0.0291	-1.3211\\
}--cycle;

\addplot[area legend, draw=mycolor1, fill=mycolor1, forget plot]
table[row sep=crcr] {%
x	y\\
-0.0134	-1.3248\\
0.1301	-1.3248\\
0.1311	-1.3245\\
0.1351	-1.3216\\
0.1386	-1.3183\\
0.1486	-1.3083\\
0.1516	-1.3049\\
0.1603	-1.2943\\
0.1676	-1.2833\\
0.1735	-1.2721\\
0.1781	-1.2607\\
0.1815	-1.2493\\
0.1815	-1.249\\
0.1829	-1.2301\\
0.1829	-1.0694\\
0.1827	-1.0684\\
0.1797	-1.0593\\
0.176	-1.0508\\
0.1715	-1.043\\
0.1665	-1.0359\\
0.1611	-1.0295\\
0.1552	-1.024\\
0.149	-1.0192\\
0.1426	-1.0153\\
0.1361	-1.0121\\
0.1211	-1.0073\\
0.1209	-1.0073\\
0.1209	-1.0072\\
0.102	-1.0058\\
-0.0416	-1.0058\\
-0.0425	-1.0061\\
-0.0465	-1.0091\\
-0.05	-1.0123\\
-0.06	-1.0223\\
-0.0631	-1.0257\\
-0.0717	-1.0363\\
-0.079	-1.0473\\
-0.085	-1.0586\\
-0.0896	-1.07\\
-0.0929	-1.0813\\
-0.0929	-1.0814\\
-0.093	-1.0816\\
-0.0944	-1.1005\\
-0.0944	-1.2612\\
-0.0941	-1.2622\\
-0.0911	-1.2713\\
-0.0874	-1.2799\\
-0.083	-1.2877\\
-0.078	-1.2948\\
-0.0725	-1.3011\\
-0.0666	-1.3066\\
-0.0604	-1.3114\\
-0.054	-1.3154\\
-0.0475	-1.3186\\
-0.0325	-1.3233\\
-0.0323	-1.3234\\
-0.0134	-1.3248\\
}--cycle;

\addplot[area legend, draw=mycolor1, fill=mycolor1, forget plot]
table[row sep=crcr] {%
x	y\\
-0.0163	-1.3273\\
0.1276	-1.3273\\
0.1459	-1.3269\\
0.1468	-1.3265\\
0.1508	-1.3236\\
0.1543	-1.3203\\
0.1643	-1.3103\\
0.1674	-1.3069\\
0.176	-1.2963\\
0.1833	-1.2853\\
0.1892	-1.2741\\
0.1939	-1.2627\\
0.1972	-1.2513\\
0.1972	-1.0899\\
0.1968	-1.0716\\
0.1938	-1.0625\\
0.1934	-1.0616\\
0.1897	-1.0531\\
0.1853	-1.0453\\
0.1803	-1.0382\\
0.1748	-1.0318\\
0.1689	-1.0263\\
0.1627	-1.0215\\
0.1563	-1.0176\\
0.1498	-1.0144\\
0.135	-1.0112\\
-0.0092	-1.0112\\
-0.0275	-1.0117\\
-0.0284	-1.012\\
-0.0324	-1.015\\
-0.0359	-1.0182\\
-0.0459	-1.0282\\
-0.049	-1.0317\\
-0.0576	-1.0422\\
-0.0649	-1.0532\\
-0.0708	-1.0645\\
-0.0754	-1.0759\\
-0.0788	-1.0873\\
-0.0788	-1.2486\\
-0.0783	-1.2669\\
-0.0754	-1.2761\\
-0.075	-1.277\\
-0.0713	-1.2855\\
-0.0669	-1.2933\\
-0.0619	-1.3004\\
-0.0564	-1.3067\\
-0.0505	-1.3122\\
-0.0443	-1.317\\
-0.0379	-1.321\\
-0.0314	-1.3242\\
-0.0165	-1.3273\\
-0.0163	-1.3273\\
}--cycle;

\addplot[area legend, draw=mycolor1, fill=mycolor1, forget plot]
table[row sep=crcr] {%
x	y\\
-0.0008	-1.3296\\
0.1437	-1.3296\\
0.1613	-1.3274\\
0.1621	-1.327\\
0.1661	-1.324\\
0.1696	-1.3208\\
0.1796	-1.3108\\
0.1826	-1.3073\\
0.1913	-1.2968\\
0.1986	-1.2858\\
0.2045	-1.2745\\
0.2091	-1.2631\\
0.2125	-1.2517\\
0.2125	-1.09\\
0.2103	-1.0725\\
0.2073	-1.0633\\
0.2035	-1.0548\\
0.2031	-1.054\\
0.1987	-1.0461\\
0.1937	-1.0391\\
0.1882	-1.0328\\
0.1824	-1.0272\\
0.1762	-1.0224\\
0.1698	-1.0185\\
0.1632	-1.0153\\
0.1487	-1.0137\\
0.0041	-1.0137\\
-0.0134	-1.0159\\
-0.0142	-1.0163\\
-0.0182	-1.0193\\
-0.0217	-1.0225\\
-0.0317	-1.0325\\
-0.0348	-1.0359\\
-0.0434	-1.0465\\
-0.0507	-1.0575\\
-0.0566	-1.0688\\
-0.0613	-1.0802\\
-0.0646	-1.0916\\
-0.0646	-1.2533\\
-0.0624	-1.2708\\
-0.0594	-1.28\\
-0.0557	-1.2885\\
-0.0553	-1.2893\\
-0.0508	-1.2971\\
-0.0458	-1.3042\\
-0.0404	-1.3105\\
-0.0345	-1.3161\\
-0.0283	-1.3208\\
-0.0219	-1.3248\\
-0.0154	-1.328\\
-0.0008	-1.3296\\
}--cycle;

\addplot[area legend, draw=mycolor1, fill=mycolor1, forget plot]
table[row sep=crcr] {%
x	y\\
0.0146	-1.3303\\
0.1594	-1.3303\\
0.176	-1.3265\\
0.1768	-1.326\\
0.1807	-1.323\\
0.1842	-1.3198\\
0.1942	-1.3098\\
0.1973	-1.3063\\
0.206	-1.2958\\
0.2132	-1.2848\\
0.2192	-1.2735\\
0.2238	-1.2621\\
0.2272	-1.2507\\
0.2272	-1.0888\\
0.2271	-1.0887\\
0.2271	-1.0886\\
0.2233	-1.0719\\
0.2203	-1.0628\\
0.2166	-1.0543\\
0.2122	-1.0465\\
0.2117	-1.0457\\
0.2067	-1.0386\\
0.2012	-1.0323\\
0.1953	-1.0268\\
0.1892	-1.022\\
0.1828	-1.018\\
0.1762	-1.0148\\
0.1621	-1.0147\\
0.0172	-1.0147\\
0.0006	-1.0185\\
-0.0001	-1.019\\
-0.0041	-1.022\\
-0.0076	-1.0252\\
-0.0176	-1.0352\\
-0.0207	-1.0387\\
-0.0293	-1.0492\\
-0.0366	-1.0602\\
-0.0426	-1.0715\\
-0.0472	-1.0829\\
-0.0505	-1.0943\\
-0.0505	-1.2564\\
-0.0467	-1.2731\\
-0.0437	-1.2822\\
-0.04	-1.2907\\
-0.0355	-1.2985\\
-0.0351	-1.2993\\
-0.0301	-1.3064\\
-0.0246	-1.3127\\
-0.0187	-1.3182\\
-0.0125	-1.323\\
-0.0061	-1.327\\
0.0004	-1.3302\\
0.0146	-1.3303\\
}--cycle;

\addplot[area legend, draw=mycolor1, fill=mycolor1, forget plot]
table[row sep=crcr] {%
x	y\\
0.0158	-1.3307\\
0.1606	-1.3307\\
0.1742	-1.3295\\
0.1744	-1.3294\\
0.19	-1.3242\\
0.194	-1.3212\\
0.1947	-1.3207\\
0.1982	-1.3174\\
0.2082	-1.3074\\
0.2113	-1.304\\
0.2199	-1.2934\\
0.2272	-1.2824\\
0.2332	-1.2711\\
0.2378	-1.2598\\
0.2411	-1.2484\\
0.2411	-1.0858\\
0.2381	-1.0766\\
0.2328	-1.061\\
0.2291	-1.0525\\
0.2247	-1.0447\\
0.2197	-1.0377\\
0.2191	-1.037\\
0.2137	-1.0306\\
0.2078	-1.0251\\
0.2016	-1.0203\\
0.1952	-1.0164\\
0.1887	-1.0132\\
0.0438	-1.0132\\
0.0302	-1.0144\\
0.03	-1.0144\\
0.03	-1.0145\\
0.0144	-1.0197\\
0.0104	-1.0227\\
0.0097	-1.0232\\
0.0062	-1.0264\\
-0.0038	-1.0364\\
-0.0068	-1.0399\\
-0.0155	-1.0505\\
-0.0228	-1.0615\\
-0.0287	-1.0727\\
-0.0333	-1.0841\\
-0.0367	-1.0955\\
-0.0367	-1.2579\\
-0.0366	-1.2581\\
-0.0337	-1.2672\\
-0.0284	-1.2828\\
-0.0247	-1.2913\\
-0.0202	-1.2991\\
-0.0152	-1.3062\\
-0.0147	-1.3069\\
-0.0092	-1.3132\\
-0.0034	-1.3188\\
0.0028	-1.3235\\
0.0092	-1.3275\\
0.0158	-1.3307\\
}--cycle;

\addplot[area legend, draw=mycolor1, fill=mycolor1, forget plot]
table[row sep=crcr] {%
x	y\\
0.0306	-1.3298\\
0.1756	-1.3298\\
0.1886	-1.3272\\
0.1888	-1.3272\\
0.2033	-1.3206\\
0.2072	-1.3177\\
0.2108	-1.3144\\
0.2214	-1.3038\\
0.2244	-1.3004\\
0.2331	-1.2898\\
0.2404	-1.2788\\
0.2463	-1.2676\\
0.2509	-1.2562\\
0.2543	-1.2448\\
0.2543	-1.082\\
0.2513	-1.0729\\
0.2513	-1.0728\\
0.2512	-1.0727\\
0.2475	-1.0641\\
0.2409	-1.0497\\
0.2365	-1.0419\\
0.2315	-1.0348\\
0.226	-1.0285\\
0.2254	-1.0278\\
0.2196	-1.0223\\
0.2134	-1.0175\\
0.207	-1.0136\\
0.2005	-1.0104\\
0.0554	-1.0104\\
0.0424	-1.0129\\
0.0422	-1.0129\\
0.0277	-1.0195\\
0.0238	-1.0225\\
0.0203	-1.0257\\
0.0097	-1.0363\\
0.0066	-1.0397\\
-0.002	-1.0503\\
-0.0093	-1.0613\\
-0.0153	-1.0725\\
-0.0199	-1.0839\\
-0.0232	-1.0953\\
-0.0232	-1.2581\\
-0.0203	-1.2673\\
-0.0202	-1.2673\\
-0.0202	-1.2675\\
-0.0164	-1.276\\
-0.0099	-1.2904\\
-0.0055	-1.2983\\
-0.0005	-1.3053\\
0.005	-1.3117\\
0.0056	-1.3123\\
0.0115	-1.3178\\
0.0176	-1.3226\\
0.024	-1.3265\\
0.0306	-1.3298\\
}--cycle;

\addplot[area legend, draw=mycolor1, fill=mycolor1, forget plot]
table[row sep=crcr] {%
x	y\\
0.0447	-1.3274\\
0.1899	-1.3274\\
0.2021	-1.3237\\
0.2023	-1.3237\\
0.2156	-1.316\\
0.2196	-1.313\\
0.2231	-1.3098\\
0.2331	-1.2998\\
0.2361	-1.2963\\
0.2367	-1.2957\\
0.2453	-1.2851\\
0.2526	-1.2741\\
0.2585	-1.2629\\
0.2632	-1.2515\\
0.2665	-1.2401\\
0.2665	-1.0769\\
0.2635	-1.0677\\
0.2598	-1.0592\\
0.2597	-1.059\\
0.2553	-1.0512\\
0.2476	-1.0379\\
0.2426	-1.0309\\
0.2371	-1.0246\\
0.2312	-1.019\\
0.2306	-1.0185\\
0.2244	-1.0137\\
0.2181	-1.0097\\
0.2115	-1.0065\\
0.0663	-1.0065\\
0.0541	-1.0102\\
0.054	-1.0102\\
0.0539	-1.0103\\
0.0406	-1.018\\
0.0366	-1.0209\\
0.0331	-1.0242\\
0.0231	-1.0342\\
0.0201	-1.0376\\
0.0195	-1.0382\\
0.0109	-1.0488\\
0.0036	-1.0598\\
-0.0023	-1.0711\\
-0.007	-1.0825\\
-0.0103	-1.0939\\
-0.0103	-1.257\\
-0.0073	-1.2662\\
-0.0036	-1.2747\\
-0.0035	-1.2749\\
0.0009	-1.2827\\
0.0086	-1.296\\
0.0136	-1.3031\\
0.0191	-1.3094\\
0.025	-1.3149\\
0.0256	-1.3155\\
0.0318	-1.3202\\
0.0382	-1.3242\\
0.0447	-1.3274\\
}--cycle;

\addplot[area legend, draw=black, fill=mycolor1, forget plot]
table[row sep=crcr] {%
x	y\\
2	2\\
2.5	2\\
2.5	2.5\\
2	2.5\\
2	2\\
}--cycle;

\addplot[area legend, draw=black, fill=mycolor1, forget plot]
table[row sep=crcr] {%
x	y\\
1.6284	2.0553\\
2.1088	2.1035\\
2.1188	2.1135\\
2.0706	2.5939\\
2.07	2.5942\\
1.5896	2.546\\
1.5796	2.536\\
1.6278	2.0556\\
1.6284	2.0553\\
}--cycle;

\addplot[area legend, draw=black, fill=mycolor1, forget plot]
table[row sep=crcr] {%
x	y\\
1.2661	2.0725\\
1.723	2.1651\\
1.733	2.1751\\
1.7417	2.1857\\
1.649	2.6427\\
1.6484	2.643\\
1.6477	2.6432\\
1.1908	2.5506\\
1.1808	2.5406\\
1.1721	2.53\\
1.2648	2.0731\\
1.2654	2.0728\\
1.2661	2.0725\\
}--cycle;

\addplot[area legend, draw=black, fill=mycolor1, forget plot]
table[row sep=crcr] {%
x	y\\
0.9163	2.0542\\
1.3464	2.1872\\
1.3564	2.1972\\
1.365	2.2078\\
1.3723	2.2188\\
1.2393	2.6489\\
1.2379	2.6495\\
1.2373	2.6497\\
0.8072	2.5166\\
0.7972	2.5066\\
0.7886	2.496\\
0.7813	2.4851\\
0.7813	2.485\\
0.9143	2.0549\\
0.915	2.0546\\
0.9157	2.0544\\
0.9163	2.0542\\
}--cycle;

\addplot[area legend, draw=black, fill=mycolor1, forget plot]
table[row sep=crcr] {%
x	y\\
0.582	2.0028\\
0.9824	2.1721\\
0.9924	2.1821\\
1.0011	2.1927\\
1.0083	2.2037\\
1.0143	2.2149\\
1.0143	2.215\\
0.845	2.6154\\
0.8436	2.616\\
0.843	2.6161\\
0.8423	2.6162\\
0.4419	2.447\\
0.4319	2.437\\
0.4233	2.4264\\
0.416	2.4154\\
0.4101	2.4041\\
0.5793	2.0037\\
0.5807	2.0031\\
0.5814	2.0029\\
0.582	2.0028\\
}--cycle;

\addplot[area legend, draw=black, fill=mycolor1, forget plot]
table[row sep=crcr] {%
x	y\\
0.2658	1.9213\\
0.6342	2.1225\\
0.6442	2.1325\\
0.6528	2.1431\\
0.6601	2.1541\\
0.666	2.1653\\
0.6707	2.1767\\
0.6707	2.1768\\
0.4694	2.5452\\
0.4687	2.5455\\
0.4681	2.5458\\
0.4667	2.546\\
0.4661	2.5461\\
0.0977	2.3449\\
0.0877	2.3349\\
0.0791	2.3243\\
0.0718	2.3133\\
0.0658	2.302\\
0.0612	2.2906\\
0.2625	1.9222\\
0.2631	1.9218\\
0.2645	1.9214\\
0.2652	1.9213\\
0.2658	1.9213\\
}--cycle;

\addplot[area legend, draw=black, fill=mycolor1, forget plot]
table[row sep=crcr] {%
x	y\\
-0.0308	1.8124\\
-0.0302	1.8125\\
0.3043	2.0413\\
0.3143	2.0513\\
0.323	2.0619\\
0.3303	2.0729\\
0.3362	2.0841\\
0.3408	2.0955\\
0.3442	2.1069\\
0.3442	2.1071\\
0.1153	2.4416\\
0.1146	2.4419\\
0.114	2.4421\\
0.1133	2.4423\\
0.1126	2.4424\\
0.1113	2.4424\\
-0.2232	2.2136\\
-0.2332	2.2036\\
-0.2418	2.193\\
-0.2491	2.182\\
-0.255	2.1707\\
-0.2597	2.1593\\
-0.263	2.1479\\
-0.263	2.1478\\
-0.0341	1.8133\\
-0.0335	1.813\\
-0.0328	1.8127\\
-0.0314	1.8125\\
-0.0308	1.8124\\
}--cycle;

\addplot[area legend, draw=black, fill=mycolor1, forget plot]
table[row sep=crcr] {%
x	y\\
-0.3053	1.6792\\
-0.3046	1.6793\\
-0.304	1.6794\\
-0.0047	1.9315\\
0.0053	1.9415\\
0.0139	1.952\\
0.0212	1.963\\
0.0272	1.9743\\
0.0318	1.9857\\
0.0351	1.9971\\
0.0372	2.0083\\
0.0372	2.0085\\
-0.2149	2.3078\\
-0.2156	2.3082\\
-0.2163	2.3084\\
-0.2169	2.3086\\
-0.2176	2.3087\\
-0.2189	2.3087\\
-0.2195	2.3086\\
-0.5188	2.0565\\
-0.5288	2.0465\\
-0.5375	2.0359\\
-0.5447	2.0249\\
-0.5507	2.0136\\
-0.5553	2.0022\\
-0.5586	1.9909\\
-0.5608	1.9796\\
-0.5608	1.9795\\
-0.5607	1.9794\\
-0.3086	1.6801\\
-0.3079	1.6798\\
-0.3073	1.6795\\
-0.3059	1.6793\\
-0.3053	1.6792\\
}--cycle;

\addplot[area legend, draw=black, fill=mycolor1, forget plot]
table[row sep=crcr] {%
x	y\\
-0.5561	1.5248\\
-0.5555	1.5248\\
-0.5549	1.5249\\
-0.5543	1.5251\\
-0.5443	1.5351\\
-0.281	1.8062\\
-0.2724	1.8167\\
-0.2651	1.8277\\
-0.2592	1.839\\
-0.2545	1.8504\\
-0.2512	1.8618\\
-0.2491	1.873\\
-0.2481	1.8841\\
-0.2481	1.8842\\
-0.5193	2.1475\\
-0.5199	2.1478\\
-0.5206	2.1481\\
-0.5213	2.1483\\
-0.522	2.1484\\
-0.5233	2.1484\\
-0.5239	2.1483\\
-0.5244	2.1481\\
-0.5344	2.1381\\
-0.7977	1.867\\
-0.8063	1.8565\\
-0.8136	1.8455\\
-0.8196	1.8342\\
-0.8242	1.8228\\
-0.8275	1.8114\\
-0.8297	1.8002\\
-0.8306	1.7891\\
-0.8306	1.789\\
-0.5595	1.5257\\
-0.5581	1.5251\\
-0.5575	1.5249\\
-0.5568	1.5248\\
-0.5561	1.5248\\
}--cycle;

\addplot[area legend, draw=black, fill=mycolor1, forget plot]
table[row sep=crcr] {%
x	y\\
-0.7823	1.3522\\
-0.7816	1.3523\\
-0.781	1.3524\\
-0.7805	1.3525\\
-0.7799	1.3527\\
-0.7699	1.3627\\
-0.7613	1.3733\\
-0.5345	1.6591\\
-0.5272	1.6701\\
-0.5212	1.6814\\
-0.5166	1.6928\\
-0.5133	1.7042\\
-0.5112	1.7154\\
-0.5102	1.7264\\
-0.5102	1.7266\\
-0.5104	1.7373\\
-0.7962	1.9642\\
-0.7969	1.9645\\
-0.7976	1.9647\\
-0.7982	1.9649\\
-0.7989	1.965\\
-0.8002	1.965\\
-0.8014	1.9648\\
-0.8019	1.9646\\
-0.8119	1.9546\\
-0.8206	1.944\\
-1.0474	1.6582\\
-1.0547	1.6472\\
-1.0606	1.6359\\
-1.0653	1.6245\\
-1.0686	1.6131\\
-1.0707	1.6019\\
-1.0717	1.5908\\
-1.0717	1.5906\\
-1.0715	1.5799\\
-0.7857	1.3531\\
-0.785	1.3528\\
-0.7836	1.3524\\
-0.783	1.3523\\
-0.7823	1.3522\\
}--cycle;

\addplot[area legend, draw=black, fill=mycolor1, forget plot]
table[row sep=crcr] {%
x	y\\
-0.9836	1.1647\\
-0.9823	1.1647\\
-0.9811	1.1649\\
-0.9801	1.1653\\
-0.9701	1.1753\\
-0.9614	1.1859\\
-0.9541	1.1969\\
-0.7638	1.4934\\
-0.7578	1.5046\\
-0.7532	1.516\\
-0.7499	1.5274\\
-0.7478	1.5387\\
-0.7468	1.5497\\
-0.7468	1.55\\
-0.747	1.5607\\
-0.7481	1.5709\\
-1.0446	1.7613\\
-1.046	1.7619\\
-1.0467	1.7621\\
-1.0473	1.7622\\
-1.0487	1.7622\\
-1.0493	1.7621\\
-1.0499	1.7619\\
-1.0509	1.7615\\
-1.0609	1.7515\\
-1.0695	1.7409\\
-1.0768	1.7299\\
-1.2672	1.4334\\
-1.2731	1.4222\\
-1.2777	1.4108\\
-1.2811	1.3994\\
-1.2832	1.3881\\
-1.2841	1.3771\\
-1.2841	1.3768\\
-1.284	1.3661\\
-1.2828	1.3559\\
-0.9863	1.1655\\
-0.9856	1.1652\\
-0.985	1.1649\\
-0.9836	1.1647\\
}--cycle;

\addplot[area legend, draw=black, fill=mycolor1, forget plot]
table[row sep=crcr] {%
x	y\\
-1.159	0.9651\\
-1.1563	0.9651\\
-1.1558	0.9652\\
-1.1552	0.9654\\
-1.1548	0.9657\\
-1.1543	0.9659\\
-1.1443	0.9759\\
-1.1357	0.9865\\
-1.1284	0.9975\\
-1.1224	1.0088\\
-0.9681	1.312\\
-0.9635	1.3234\\
-0.9601	1.3348\\
-0.958	1.346\\
-0.9571	1.3571\\
-0.9571	1.3575\\
-0.9572	1.3682\\
-0.9584	1.3785\\
-0.9605	1.3882\\
-1.2638	1.5425\\
-1.2644	1.5429\\
-1.2658	1.5433\\
-1.2684	1.5433\\
-1.269	1.5431\\
-1.27	1.5427\\
-1.2705	1.5424\\
-1.2805	1.5324\\
-1.2891	1.5219\\
-1.2964	1.5109\\
-1.3023	1.4996\\
-1.4567	1.1964\\
-1.4613	1.185\\
-1.4646	1.1736\\
-1.4667	1.1623\\
-1.4677	1.1513\\
-1.4677	1.1508\\
-1.4675	1.1402\\
-1.4664	1.1299\\
-1.4642	1.1202\\
-1.161	0.9658\\
-1.1603	0.9655\\
-1.1597	0.9653\\
-1.159	0.9651\\
}--cycle;

\addplot[area legend, draw=black, fill=mycolor1, forget plot]
table[row sep=crcr] {%
x	y\\
-1.3078	0.7565\\
-1.3044	0.7565\\
-1.3038	0.7567\\
-1.3034	0.7569\\
-1.3029	0.7572\\
-1.3025	0.7575\\
-1.2925	0.7675\\
-1.2839	0.7781\\
-1.2766	0.7891\\
-1.2706	0.8004\\
-1.266	0.8117\\
-1.147	1.1179\\
-1.1436	1.1293\\
-1.1415	1.1406\\
-1.1406	1.1516\\
-1.1406	1.163\\
-1.1417	1.1732\\
-1.1439	1.1829\\
-1.1468	1.1921\\
-1.1475	1.1924\\
-1.4537	1.3115\\
-1.4551	1.3119\\
-1.4584	1.3119\\
-1.459	1.3117\\
-1.4595	1.3114\\
-1.4599	1.3112\\
-1.4603	1.3109\\
-1.4703	1.3009\\
-1.479	1.2903\\
-1.4863	1.2793\\
-1.4922	1.268\\
-1.4968	1.2566\\
-1.6159	0.9504\\
-1.6192	0.9391\\
-1.6213	0.9278\\
-1.6223	0.9168\\
-1.6223	0.9054\\
-1.6211	0.8952\\
-1.619	0.8854\\
-1.616	0.8763\\
-1.6153	0.876\\
-1.3091	0.7569\\
-1.3084	0.7567\\
-1.3078	0.7565\\
}--cycle;

\addplot[area legend, draw=black, fill=mycolor1, forget plot]
table[row sep=crcr] {%
x	y\\
-1.4311	0.5418\\
-1.4265	0.5418\\
-1.426	0.542\\
-1.4248	0.5429\\
-1.4148	0.5529\\
-1.4062	0.5635\\
-1.3989	0.5745\\
-1.3929	0.5858\\
-1.3883	0.5972\\
-1.385	0.6086\\
-1.3001	0.9142\\
-1.298	0.9255\\
-1.297	0.9365\\
-1.297	0.9482\\
-1.2982	0.9585\\
-1.3004	0.9682\\
-1.3033	0.9773\\
-1.3071	0.9859\\
-1.3077	0.9862\\
-1.3084	0.9864\\
-1.6141	1.0713\\
-1.6187	1.0713\\
-1.6192	1.071\\
-1.6196	1.0708\\
-1.62	1.0705\\
-1.6304	1.0601\\
-1.639	1.0496\\
-1.6463	1.0386\\
-1.6523	1.0273\\
-1.6569	1.0159\\
-1.6602	1.0045\\
-1.7451	0.6989\\
-1.7472	0.6876\\
-1.7481	0.6766\\
-1.7481	0.6649\\
-1.747	0.6546\\
-1.7448	0.6449\\
-1.7419	0.6357\\
-1.7381	0.6272\\
-1.7374	0.6269\\
-1.7368	0.6266\\
-1.4311	0.5418\\
}--cycle;

\addplot[area legend, draw=black, fill=mycolor1, forget plot]
table[row sep=crcr] {%
x	y\\
-1.529	0.3237\\
-1.5232	0.3237\\
-1.5227	0.3239\\
-1.5223	0.3242\\
-1.5116	0.3349\\
-1.503	0.3455\\
-1.4957	0.3565\\
-1.4898	0.3678\\
-1.4851	0.3791\\
-1.4818	0.3905\\
-1.4797	0.4018\\
-1.4276	0.7037\\
-1.4267	0.7147\\
-1.4267	0.7269\\
-1.4279	0.7371\\
-1.43	0.7469\\
-1.433	0.756\\
-1.4367	0.7645\\
-1.4411	0.7723\\
-1.4418	0.7727\\
-1.7437	0.8247\\
-1.7494	0.8247\\
-1.7499	0.8244\\
-1.7507	0.8238\\
-1.7607	0.8138\\
-1.761	0.8134\\
-1.7696	0.8029\\
-1.7769	0.7919\\
-1.7828	0.7806\\
-1.7875	0.7692\\
-1.7908	0.7578\\
-1.7929	0.7466\\
-1.845	0.4447\\
-1.8459	0.4337\\
-1.8459	0.4215\\
-1.8447	0.4112\\
-1.8426	0.4015\\
-1.8396	0.3924\\
-1.8359	0.3839\\
-1.8315	0.376\\
-1.8308	0.3757\\
-1.529	0.3237\\
}--cycle;

\addplot[area legend, draw=black, fill=mycolor1, forget plot]
table[row sep=crcr] {%
x	y\\
-1.6013	0.1047\\
-1.5948	0.1047\\
-1.5944	0.105\\
-1.5837	0.1157\\
-1.575	0.1263\\
-1.5678	0.1373\\
-1.5618	0.1485\\
-1.5572	0.1599\\
-1.5538	0.1713\\
-1.5517	0.1826\\
-1.5508	0.1936\\
-1.5299	0.4887\\
-1.5299	0.5015\\
-1.531	0.5118\\
-1.5332	0.5215\\
-1.5361	0.5306\\
-1.5399	0.5391\\
-1.5443	0.547\\
-1.5493	0.554\\
-1.55	0.5544\\
-1.845	0.5753\\
-1.8515	0.5753\\
-1.8519	0.575\\
-1.8626	0.5643\\
-1.8713	0.5537\\
-1.8785	0.5427\\
-1.8845	0.5315\\
-1.8891	0.5201\\
-1.8924	0.5087\\
-1.8946	0.4974\\
-1.8955	0.4864\\
-1.9164	0.1913\\
-1.9164	0.1785\\
-1.9152	0.1682\\
-1.9131	0.1585\\
-1.9101	0.1494\\
-1.9064	0.1408\\
-1.902	0.133\\
-1.897	0.126\\
-1.8963	0.1256\\
-1.6013	0.1047\\
}--cycle;

\addplot[area legend, draw=black, fill=mycolor1, forget plot]
table[row sep=crcr] {%
x	y\\
-1.9351	-0.121\\
-1.928	-0.121\\
-1.6425	-0.1126\\
-1.6421	-0.1123\\
-1.6321	-0.1023\\
-1.6318	-0.1019\\
-1.6232	-0.0913\\
-1.6159	-0.0804\\
-1.61	-0.0691\\
-1.6053	-0.0577\\
-1.602	-0.0463\\
-1.5999	-0.035\\
-1.5989	-0.024\\
-1.5989	-0.0105\\
-1.6073	0.275\\
-1.6085	0.2853\\
-1.6106	0.295\\
-1.6136	0.3041\\
-1.6173	0.3127\\
-1.6217	0.3205\\
-1.6267	0.3275\\
-1.6322	0.3339\\
-1.6329	0.3342\\
-1.64	0.3342\\
-1.9255	0.3258\\
-1.9259	0.3255\\
-1.9362	0.3152\\
-1.9448	0.3046\\
-1.9521	0.2936\\
-1.9581	0.2823\\
-1.9627	0.2709\\
-1.966	0.2595\\
-1.9681	0.2483\\
-1.9691	0.2372\\
-1.9691	0.2237\\
-1.9607	-0.0618\\
-1.9596	-0.072\\
-1.9574	-0.0818\\
-1.9545	-0.0909\\
-1.9507	-0.0994\\
-1.9463	-0.1072\\
-1.9413	-0.1143\\
-1.9358	-0.1206\\
-1.9351	-0.121\\
}--cycle;

\addplot[area legend, draw=black, fill=mycolor1, forget plot]
table[row sep=crcr] {%
x	y\\
-1.9487	-0.3616\\
-1.941	-0.3616\\
-1.6675	-0.3261\\
-1.6672	-0.3257\\
-1.6572	-0.3157\\
-1.6485	-0.3052\\
-1.6413	-0.2942\\
-1.6353	-0.2829\\
-1.6307	-0.2715\\
-1.6273	-0.2601\\
-1.6252	-0.2489\\
-1.6243	-0.2378\\
-1.6243	-0.2235\\
-1.6255	-0.2133\\
-1.661	0.0602\\
-1.6631	0.0699\\
-1.6661	0.0791\\
-1.6698	0.0876\\
-1.6743	0.0954\\
-1.6793	0.1025\\
-1.6847	0.1088\\
-1.6906	0.1143\\
-1.6913	0.1147\\
-1.699	0.1147\\
-1.9724	0.0791\\
-1.9728	0.0788\\
-1.9828	0.0688\\
-1.9914	0.0582\\
-1.9987	0.0472\\
-2.0047	0.036\\
-2.0093	0.0246\\
-2.0126	0.0132\\
-2.0147	0.0019\\
-2.0157	-0.0091\\
-2.0157	-0.0234\\
-2.0145	-0.0337\\
-1.979	-0.3071\\
-1.9769	-0.3169\\
-1.9739	-0.326\\
-1.9701	-0.3345\\
-1.9657	-0.3423\\
-1.9607	-0.3494\\
-1.9552	-0.3557\\
-1.9494	-0.3613\\
-1.9487	-0.3616\\
}--cycle;

\addplot[area legend, draw=black, fill=mycolor1, forget plot]
table[row sep=crcr] {%
x	y\\
-1.9392	-0.5941\\
-1.9307	-0.5941\\
-1.6714	-0.5336\\
-1.671	-0.5333\\
-1.661	-0.5233\\
-1.6524	-0.5127\\
-1.6451	-0.5017\\
-1.6391	-0.4905\\
-1.6345	-0.4791\\
-1.6312	-0.4677\\
-1.629	-0.4564\\
-1.6281	-0.4454\\
-1.6281	-0.4304\\
-1.6293	-0.4201\\
-1.6314	-0.4104\\
-1.6919	-0.1511\\
-1.6949	-0.1419\\
-1.6986	-0.1334\\
-1.703	-0.1256\\
-1.708	-0.1185\\
-1.7135	-0.1122\\
-1.7194	-0.1067\\
-1.7256	-0.1019\\
-1.7341	-0.1019\\
-1.9934	-0.1624\\
-1.9938	-0.1627\\
-2.0038	-0.1727\\
-2.0124	-0.1833\\
-2.0197	-0.1943\\
-2.0256	-0.2056\\
-2.0302	-0.217\\
-2.0336	-0.2283\\
-2.0357	-0.2396\\
-2.0366	-0.2506\\
-2.0366	-0.2656\\
-2.0355	-0.2759\\
-2.0333	-0.2856\\
-1.9728	-0.545\\
-1.9699	-0.5541\\
-1.9661	-0.5626\\
-1.9617	-0.5704\\
-1.9567	-0.5775\\
-1.9512	-0.5838\\
-1.9454	-0.5894\\
-1.9392	-0.5941\\
}--cycle;

\addplot[area legend, draw=black, fill=mycolor1, forget plot]
table[row sep=crcr] {%
x	y\\
-1.907	-0.8166\\
-1.898	-0.8166\\
-1.6547	-0.7334\\
-1.6447	-0.7234\\
-1.636	-0.7129\\
-1.6288	-0.7019\\
-1.6228	-0.6906\\
-1.6182	-0.6792\\
-1.6149	-0.6678\\
-1.6127	-0.6566\\
-1.6118	-0.6455\\
-1.6118	-0.6298\\
-1.613	-0.6195\\
-1.6151	-0.6098\\
-1.6181	-0.6006\\
-1.7012	-0.3573\\
-1.705	-0.3488\\
-1.7094	-0.341\\
-1.7144	-0.3339\\
-1.7198	-0.3276\\
-1.7257	-0.3221\\
-1.7319	-0.3173\\
-1.7383	-0.3133\\
-1.7472	-0.3133\\
-1.9906	-0.3965\\
-2.0006	-0.4065\\
-2.0092	-0.417\\
-2.0165	-0.428\\
-2.0224	-0.4393\\
-2.0271	-0.4507\\
-2.0304	-0.4621\\
-2.0325	-0.4733\\
-2.0335	-0.4844\\
-2.0335	-0.5001\\
-2.0323	-0.5104\\
-2.0302	-0.5201\\
-2.0272	-0.5292\\
-1.9441	-0.7726\\
-1.9403	-0.7811\\
-1.9359	-0.7889\\
-1.9309	-0.796\\
-1.9254	-0.8023\\
-1.9195	-0.8078\\
-1.9134	-0.8126\\
-1.907	-0.8166\\
}--cycle;

\addplot[area legend, draw=black, fill=mycolor1, forget plot]
table[row sep=crcr] {%
x	y\\
-1.8546	-1.0272\\
-1.8446	-1.0272\\
-1.6189	-0.9238\\
-1.6089	-0.9138\\
-1.6002	-0.9033\\
-1.5929	-0.8923\\
-1.587	-0.881\\
-1.5824	-0.8696\\
-1.579	-0.8582\\
-1.5769	-0.847\\
-1.5769	-0.8198\\
-1.5781	-0.8095\\
-1.5802	-0.7998\\
-1.5832	-0.7907\\
-1.5869	-0.7822\\
-1.6903	-0.5564\\
-1.6947	-0.5486\\
-1.6997	-0.5415\\
-1.7052	-0.5352\\
-1.711	-0.5296\\
-1.7172	-0.5249\\
-1.7236	-0.5209\\
-1.7301	-0.5177\\
-1.7401	-0.5177\\
-1.9659	-0.621\\
-1.9759	-0.631\\
-1.9845	-0.6416\\
-1.9918	-0.6526\\
-1.9977	-0.6638\\
-2.0024	-0.6752\\
-2.0057	-0.6866\\
-2.0078	-0.6979\\
-2.0078	-0.725\\
-2.0066	-0.7353\\
-2.0045	-0.745\\
-2.0015	-0.7542\\
-1.9978	-0.7627\\
-1.8945	-0.9885\\
-1.89	-0.9963\\
-1.885	-1.0034\\
-1.8796	-1.0097\\
-1.8737	-1.0152\\
-1.8675	-1.02\\
-1.8611	-1.0239\\
-1.8546	-1.0272\\
}--cycle;

\addplot[area legend, draw=black, fill=mycolor1, forget plot]
table[row sep=crcr] {%
x	y\\
-1.7851	-1.2244\\
-1.774	-1.2244\\
-1.567	-1.1034\\
-1.557	-1.0934\\
-1.5484	-1.0828\\
-1.5411	-1.0718\\
-1.5352	-1.0606\\
-1.5305	-1.0492\\
-1.5272	-1.0378\\
-1.5251	-1.0265\\
-1.5251	-0.9887\\
-1.5272	-0.979\\
-1.5302	-0.9699\\
-1.5339	-0.9614\\
-1.5383	-0.9535\\
-1.6594	-0.7466\\
-1.6644	-0.7395\\
-1.6699	-0.7332\\
-1.6757	-0.7276\\
-1.6819	-0.7229\\
-1.6883	-0.7189\\
-1.6948	-0.7157\\
-1.7014	-0.7133\\
-1.7126	-0.7133\\
-1.9195	-0.8343\\
-1.9295	-0.8443\\
-1.9382	-0.8549\\
-1.9454	-0.8659\\
-1.9514	-0.8771\\
-1.956	-0.8885\\
-1.9594	-0.8999\\
-1.9615	-0.9112\\
-1.9615	-0.9489\\
-1.9593	-0.9587\\
-1.9564	-0.9678\\
-1.9526	-0.9763\\
-1.9482	-0.9842\\
-1.8272	-1.1911\\
-1.8222	-1.1982\\
-1.8167	-1.2045\\
-1.8108	-1.2101\\
-1.8046	-1.2148\\
-1.7982	-1.2188\\
-1.7917	-1.222\\
-1.7851	-1.2244\\
}--cycle;

\addplot[area legend, draw=black, fill=mycolor1, forget plot]
table[row sep=crcr] {%
x	y\\
-1.7048	-1.4072\\
-1.6871	-1.4072\\
-1.4999	-1.2709\\
-1.4899	-1.2609\\
-1.4813	-1.2504\\
-1.474	-1.2394\\
-1.468	-1.2281\\
-1.4634	-1.2167\\
-1.4601	-1.2053\\
-1.458	-1.1941\\
-1.458	-1.1542\\
-1.4601	-1.1445\\
-1.4631	-1.1353\\
-1.4668	-1.1268\\
-1.4712	-1.119\\
-1.4762	-1.1119\\
-1.6125	-0.9247\\
-1.6179	-0.9184\\
-1.6238	-0.9129\\
-1.63	-0.9081\\
-1.6364	-0.9041\\
-1.6429	-0.9009\\
-1.6495	-0.8985\\
-1.6673	-0.8985\\
-1.8544	-1.0347\\
-1.8644	-1.0447\\
-1.8731	-1.0553\\
-1.8804	-1.0663\\
-1.8863	-1.0776\\
-1.8909	-1.089\\
-1.8943	-1.1003\\
-1.8964	-1.1116\\
-1.8964	-1.1515\\
-1.8943	-1.1612\\
-1.8913	-1.1704\\
-1.8875	-1.1789\\
-1.8831	-1.1867\\
-1.8781	-1.1938\\
-1.7419	-1.3809\\
-1.7364	-1.3872\\
-1.7305	-1.3928\\
-1.7243	-1.3975\\
-1.7179	-1.4015\\
-1.7114	-1.4047\\
-1.7048	-1.4072\\
}--cycle;

\addplot[area legend, draw=black, fill=mycolor1, forget plot]
table[row sep=crcr] {%
x	y\\
-1.6103	-1.5744\\
-1.586	-1.5744\\
-1.4193	-1.4254\\
-1.4093	-1.4154\\
-1.4006	-1.4049\\
-1.3934	-1.3939\\
-1.3874	-1.3826\\
-1.3828	-1.3712\\
-1.3795	-1.3598\\
-1.3773	-1.3486\\
-1.3773	-1.3073\\
-1.3795	-1.2976\\
-1.3824	-1.2884\\
-1.3862	-1.2799\\
-1.3906	-1.2721\\
-1.3956	-1.265\\
-1.4011	-1.2587\\
-1.55	-1.092\\
-1.5559	-1.0865\\
-1.5621	-1.0817\\
-1.5685	-1.0778\\
-1.575	-1.0745\\
-1.5816	-1.0721\\
-1.6059	-1.0721\\
-1.7726	-1.221\\
-1.7826	-1.231\\
-1.7912	-1.2416\\
-1.7985	-1.2526\\
-1.8044	-1.2639\\
-1.8091	-1.2753\\
-1.8124	-1.2866\\
-1.8145	-1.2979\\
-1.8145	-1.3391\\
-1.8124	-1.3489\\
-1.8094	-1.358\\
-1.8057	-1.3665\\
-1.8013	-1.3743\\
-1.7963	-1.3814\\
-1.7908	-1.3877\\
-1.6418	-1.5544\\
-1.636	-1.56\\
-1.6298	-1.5647\\
-1.6234	-1.5687\\
-1.6169	-1.5719\\
-1.6103	-1.5744\\
}--cycle;

\addplot[area legend, draw=black, fill=mycolor1, forget plot]
table[row sep=crcr] {%
x	y\\
-1.5034	-1.7253\\
-1.4727	-1.7253\\
-1.4627	-1.7153\\
-1.3169	-1.5561\\
-1.3083	-1.5455\\
-1.301	-1.5345\\
-1.2951	-1.5233\\
-1.2904	-1.5119\\
-1.2871	-1.5005\\
-1.285	-1.4892\\
-1.285	-1.4473\\
-1.2871	-1.4376\\
-1.2901	-1.4284\\
-1.2938	-1.4199\\
-1.2982	-1.4121\\
-1.3032	-1.405\\
-1.3087	-1.3987\\
-1.3146	-1.3932\\
-1.4738	-1.2474\\
-1.48	-1.2426\\
-1.4863	-1.2386\\
-1.4929	-1.2354\\
-1.4995	-1.233\\
-1.5302	-1.233\\
-1.5402	-1.243\\
-1.686	-1.4021\\
-1.6946	-1.4127\\
-1.7019	-1.4237\\
-1.7078	-1.435\\
-1.7125	-1.4464\\
-1.7158	-1.4578\\
-1.7179	-1.469\\
-1.7179	-1.5109\\
-1.7158	-1.5207\\
-1.7128	-1.5298\\
-1.7091	-1.5383\\
-1.7047	-1.5461\\
-1.6997	-1.5532\\
-1.6942	-1.5595\\
-1.6883	-1.5651\\
-1.5291	-1.7109\\
-1.5229	-1.7156\\
-1.5166	-1.7196\\
-1.51	-1.7228\\
-1.5034	-1.7253\\
}--cycle;

\addplot[area legend, draw=black, fill=mycolor1, forget plot]
table[row sep=crcr] {%
x	y\\
-1.3863	-1.8596\\
-1.3494	-1.8596\\
-1.3394	-1.8496\\
-1.3307	-1.839\\
-1.206	-1.672\\
-1.1987	-1.6611\\
-1.1928	-1.6498\\
-1.1881	-1.6384\\
-1.1848	-1.627\\
-1.1827	-1.6157\\
-1.1827	-1.5733\\
-1.1848	-1.5635\\
-1.1878	-1.5544\\
-1.1915	-1.5459\\
-1.196	-1.538\\
-1.2009	-1.531\\
-1.2064	-1.5247\\
-1.2123	-1.5191\\
-1.2185	-1.5143\\
-1.3855	-1.3896\\
-1.3919	-1.3856\\
-1.3984	-1.3824\\
-1.405	-1.38\\
-1.4419	-1.38\\
-1.4519	-1.39\\
-1.4606	-1.4006\\
-1.5853	-1.5675\\
-1.5926	-1.5785\\
-1.5985	-1.5898\\
-1.6031	-1.6012\\
-1.6065	-1.6126\\
-1.6086	-1.6238\\
-1.6086	-1.6663\\
-1.6065	-1.6761\\
-1.6035	-1.6852\\
-1.5998	-1.6937\\
-1.5953	-1.7015\\
-1.5903	-1.7086\\
-1.5849	-1.7149\\
-1.579	-1.7205\\
-1.5728	-1.7252\\
-1.4058	-1.85\\
-1.3994	-1.8539\\
-1.3929	-1.8571\\
-1.3863	-1.8596\\
}--cycle;

\addplot[area legend, draw=black, fill=mycolor1, forget plot]
table[row sep=crcr] {%
x	y\\
-1.2609	-1.9774\\
-1.2179	-1.9774\\
-1.2079	-1.9674\\
-1.1993	-1.9568\\
-1.192	-1.9458\\
-1.0883	-1.7733\\
-1.0823	-1.7621\\
-1.0777	-1.7507\\
-1.0744	-1.7393\\
-1.0722	-1.728\\
-1.0722	-1.6844\\
-1.0744	-1.6747\\
-1.0773	-1.6655\\
-1.0811	-1.657\\
-1.0855	-1.6492\\
-1.0905	-1.6421\\
-1.096	-1.6358\\
-1.1019	-1.6303\\
-1.108	-1.6255\\
-1.1144	-1.6215\\
-1.2869	-1.5178\\
-1.2934	-1.5146\\
-1.3	-1.5121\\
-1.3429	-1.5121\\
-1.3529	-1.5221\\
-1.3616	-1.5327\\
-1.3689	-1.5437\\
-1.4726	-1.7161\\
-1.4785	-1.7274\\
-1.4832	-1.7388\\
-1.4865	-1.7502\\
-1.4886	-1.7615\\
-1.4886	-1.8051\\
-1.4865	-1.8148\\
-1.4835	-1.824\\
-1.4798	-1.8325\\
-1.4754	-1.8403\\
-1.4704	-1.8474\\
-1.4649	-1.8537\\
-1.459	-1.8592\\
-1.4528	-1.864\\
-1.4464	-1.868\\
-1.274	-1.9717\\
-1.2674	-1.9749\\
-1.2609	-1.9774\\
}--cycle;

\addplot[area legend, draw=black, fill=mycolor1, forget plot]
table[row sep=crcr] {%
x	y\\
-1.129	-2.0784\\
-1.0804	-2.0784\\
-1.0704	-2.0684\\
-1.0617	-2.0578\\
-1.0545	-2.0468\\
-1.0485	-2.0356\\
-0.9655	-1.8599\\
-0.9608	-1.8485\\
-0.9575	-1.8371\\
-0.9554	-1.8258\\
-0.9554	-1.7806\\
-0.9575	-1.7708\\
-0.9605	-1.7617\\
-0.9642	-1.7532\\
-0.9686	-1.7454\\
-0.9736	-1.7383\\
-0.9791	-1.732\\
-0.985	-1.7264\\
-0.9912	-1.7217\\
-0.9976	-1.7177\\
-1.0041	-1.7145\\
-1.1798	-1.6314\\
-1.1864	-1.629\\
-1.235	-1.629\\
-1.245	-1.639\\
-1.2536	-1.6495\\
-1.2609	-1.6605\\
-1.2669	-1.6718\\
-1.3499	-1.8475\\
-1.3545	-1.8589\\
-1.3579	-1.8703\\
-1.36	-1.8815\\
-1.36	-1.9268\\
-1.3579	-1.9365\\
-1.3549	-1.9457\\
-1.3512	-1.9542\\
-1.3467	-1.962\\
-1.3417	-1.9691\\
-1.3363	-1.9754\\
-1.3304	-1.9809\\
-1.3242	-1.9857\\
-1.3178	-1.9897\\
-1.3113	-1.9929\\
-1.1356	-2.0759\\
-1.129	-2.0784\\
}--cycle;

\addplot[area legend, draw=black, fill=mycolor1, forget plot]
table[row sep=crcr] {%
x	y\\
-0.9926	-2.1627\\
-0.9386	-2.1627\\
-0.9286	-2.1527\\
-0.92	-2.1422\\
-0.9127	-2.1312\\
-0.9067	-2.1199\\
-0.9021	-2.1085\\
-0.8393	-1.9317\\
-0.8359	-1.9203\\
-0.8338	-1.909\\
-0.8338	-1.8617\\
-0.8359	-1.852\\
-0.8389	-1.8428\\
-0.8426	-1.8343\\
-0.8471	-1.8265\\
-0.8521	-1.8194\\
-0.8575	-1.8131\\
-0.8634	-1.8076\\
-0.8696	-1.8028\\
-0.876	-1.7988\\
-0.8825	-1.7956\\
-0.8891	-1.7932\\
-1.0659	-1.7303\\
-1.1199	-1.7303\\
-1.1299	-1.7403\\
-1.1385	-1.7509\\
-1.1458	-1.7619\\
-1.1518	-1.7731\\
-1.1564	-1.7845\\
-1.2193	-1.9613\\
-1.2226	-1.9727\\
-1.2247	-1.984\\
-1.2247	-2.0313\\
-1.2226	-2.0411\\
-1.2196	-2.0502\\
-1.2159	-2.0587\\
-1.2115	-2.0665\\
-1.2065	-2.0736\\
-1.201	-2.0799\\
-1.1951	-2.0855\\
-1.1889	-2.0902\\
-1.1825	-2.0942\\
-1.176	-2.0974\\
-1.1694	-2.0999\\
-0.9926	-2.1627\\
}--cycle;

\addplot[area legend, draw=black, fill=mycolor1, forget plot]
table[row sep=crcr] {%
x	y\\
-0.8534	-2.2306\\
-0.7971	-2.2306\\
-0.7923	-2.2283\\
-0.7823	-2.2183\\
-0.7737	-2.2077\\
-0.7664	-2.1968\\
-0.7605	-2.1855\\
-0.7559	-2.1741\\
-0.7525	-2.1627\\
-0.7092	-1.9868\\
-0.7092	-1.928\\
-0.7113	-1.9182\\
-0.7143	-1.9091\\
-0.718	-1.9006\\
-0.7224	-1.8927\\
-0.7274	-1.8857\\
-0.7329	-1.8794\\
-0.7388	-1.8738\\
-0.745	-1.869\\
-0.7513	-1.8651\\
-0.7579	-1.8619\\
-0.7645	-1.8594\\
-0.9404	-1.8161\\
-0.9967	-1.8161\\
-1.0015	-1.8183\\
-1.0115	-1.8283\\
-1.0201	-1.8389\\
-1.0274	-1.8499\\
-1.0334	-1.8612\\
-1.038	-1.8726\\
-1.0413	-1.884\\
-1.0847	-2.0599\\
-1.0847	-2.1187\\
-1.0826	-2.1284\\
-1.0796	-2.1376\\
-1.0758	-2.1461\\
-1.0714	-2.1539\\
-1.0664	-2.161\\
-1.0609	-2.1673\\
-1.0551	-2.1729\\
-1.0489	-2.1776\\
-1.0425	-2.1816\\
-1.036	-2.1848\\
-1.0294	-2.1872\\
-0.8534	-2.2306\\
}--cycle;

\addplot[area legend, draw=black, fill=mycolor1, forget plot]
table[row sep=crcr] {%
x	y\\
-0.7153	-2.2824\\
-0.6567	-2.2824\\
-0.6519	-2.2801\\
-0.6475	-2.2774\\
-0.6375	-2.2674\\
-0.6289	-2.2569\\
-0.6216	-2.2459\\
-0.6156	-2.2346\\
-0.611	-2.2232\\
-0.6077	-2.2118\\
-0.583	-2.0386\\
-0.583	-1.9699\\
-0.586	-1.9607\\
-0.5897	-1.9522\\
-0.5941	-1.9444\\
-0.5991	-1.9373\\
-0.6046	-1.931\\
-0.6105	-1.9255\\
-0.6167	-1.9207\\
-0.6231	-1.9167\\
-0.6296	-1.9135\\
-0.6362	-1.9111\\
-0.8094	-1.8864\\
-0.868	-1.8864\\
-0.8728	-1.8887\\
-0.8772	-1.8913\\
-0.8872	-1.9013\\
-0.8958	-1.9119\\
-0.9031	-1.9229\\
-0.9091	-1.9341\\
-0.9137	-1.9455\\
-0.917	-1.9569\\
-0.9417	-2.1301\\
-0.9417	-2.1989\\
-0.9388	-2.208\\
-0.935	-2.2165\\
-0.9306	-2.2243\\
-0.9256	-2.2314\\
-0.9201	-2.2377\\
-0.9142	-2.2433\\
-0.9081	-2.248\\
-0.9017	-2.252\\
-0.8951	-2.2552\\
-0.8885	-2.2577\\
-0.7153	-2.2824\\
}--cycle;

\addplot[area legend, draw=black, fill=mycolor1, forget plot]
table[row sep=crcr] {%
x	y\\
-0.5756	-2.3186\\
-0.512	-2.3186\\
-0.5076	-2.3159\\
-0.5036	-2.3129\\
-0.4936	-2.3029\\
-0.485	-2.2924\\
-0.4777	-2.2814\\
-0.4718	-2.2701\\
-0.4671	-2.2587\\
-0.4638	-2.2473\\
-0.4568	-2.0785\\
-0.4568	-2.0074\\
-0.4598	-1.9982\\
-0.4635	-1.9897\\
-0.4679	-1.9819\\
-0.4729	-1.9748\\
-0.4784	-1.9685\\
-0.4843	-1.963\\
-0.4905	-1.9582\\
-0.4968	-1.9542\\
-0.5034	-1.951\\
-0.51	-1.9486\\
-0.6788	-1.9415\\
-0.7424	-1.9415\\
-0.7468	-1.9442\\
-0.7508	-1.9472\\
-0.7608	-1.9572\\
-0.7694	-1.9677\\
-0.7767	-1.9787\\
-0.7826	-1.99\\
-0.7872	-2.0014\\
-0.7906	-2.0128\\
-0.7976	-2.1816\\
-0.7976	-2.2527\\
-0.7946	-2.2619\\
-0.7909	-2.2704\\
-0.7865	-2.2782\\
-0.7815	-2.2853\\
-0.776	-2.2916\\
-0.7701	-2.2972\\
-0.7639	-2.3019\\
-0.7575	-2.3059\\
-0.751	-2.3091\\
-0.7444	-2.3115\\
-0.5756	-2.3186\\
}--cycle;

\addplot[area legend, draw=black, fill=mycolor1, forget plot]
table[row sep=crcr] {%
x	y\\
-0.6074	-2.3494\\
-0.5369	-2.3494\\
-0.3741	-2.3399\\
-0.3697	-2.3372\\
-0.3657	-2.3342\\
-0.3622	-2.331\\
-0.3522	-2.321\\
-0.3436	-2.3104\\
-0.3363	-2.2994\\
-0.3303	-2.2882\\
-0.3257	-2.2768\\
-0.3224	-2.2654\\
-0.3224	-2.1917\\
-0.3319	-2.0288\\
-0.3349	-2.0196\\
-0.3386	-2.0111\\
-0.343	-2.0033\\
-0.348	-1.9962\\
-0.3535	-1.9899\\
-0.3594	-1.9844\\
-0.3656	-1.9796\\
-0.372	-1.9756\\
-0.3785	-1.9724\\
-0.4489	-1.9724\\
-0.6118	-1.982\\
-0.6162	-1.9846\\
-0.6201	-1.9876\\
-0.6237	-1.9908\\
-0.6337	-2.0008\\
-0.6423	-2.0114\\
-0.6496	-2.0224\\
-0.6555	-2.0336\\
-0.6602	-2.045\\
-0.6635	-2.0564\\
-0.6635	-2.1302\\
-0.654	-2.293\\
-0.651	-2.3022\\
-0.6473	-2.3107\\
-0.6428	-2.3185\\
-0.6378	-2.3256\\
-0.6324	-2.3319\\
-0.6265	-2.3374\\
-0.6203	-2.3422\\
-0.6139	-2.3462\\
-0.6074	-2.3494\\
}--cycle;

\addplot[area legend, draw=black, fill=mycolor1, forget plot]
table[row sep=crcr] {%
x	y\\
-0.4657	-2.3719\\
-0.3907	-2.3719\\
-0.2351	-2.347\\
-0.2311	-2.3441\\
-0.2276	-2.3408\\
-0.2176	-2.3308\\
-0.2146	-2.3274\\
-0.2059	-2.3168\\
-0.1986	-2.3058\\
-0.1927	-2.2945\\
-0.1881	-2.2832\\
-0.1847	-2.2718\\
-0.1847	-2.1953\\
-0.2096	-2.0397\\
-0.2126	-2.0306\\
-0.2163	-2.0221\\
-0.2207	-2.0142\\
-0.2257	-2.0072\\
-0.2312	-2.0008\\
-0.2371	-1.9953\\
-0.2433	-1.9905\\
-0.2496	-1.9866\\
-0.2562	-1.9834\\
-0.3312	-1.9834\\
-0.4868	-2.0082\\
-0.4908	-2.0112\\
-0.4943	-2.0144\\
-0.5043	-2.0244\\
-0.5073	-2.0279\\
-0.516	-2.0384\\
-0.5233	-2.0494\\
-0.5292	-2.0607\\
-0.5338	-2.0721\\
-0.5372	-2.0835\\
-0.5372	-2.16\\
-0.5123	-2.3155\\
-0.5093	-2.3247\\
-0.5056	-2.3332\\
-0.5012	-2.341\\
-0.4962	-2.3481\\
-0.4907	-2.3544\\
-0.4848	-2.3599\\
-0.4787	-2.3647\\
-0.4723	-2.3687\\
-0.4657	-2.3719\\
}--cycle;

\addplot[area legend, draw=black, fill=mycolor1, forget plot]
table[row sep=crcr] {%
x	y\\
-0.3275	-2.3798\\
-0.2496	-2.3798\\
-0.1025	-2.3409\\
-0.0986	-2.338\\
-0.095	-2.3347\\
-0.085	-2.3247\\
-0.082	-2.3213\\
-0.0733	-2.3107\\
-0.066	-2.2997\\
-0.0601	-2.2885\\
-0.0555	-2.2771\\
-0.0521	-2.2657\\
-0.0521	-2.1855\\
-0.091	-2.0384\\
-0.094	-2.0293\\
-0.0977	-2.0208\\
-0.1021	-2.013\\
-0.1071	-2.0059\\
-0.1126	-1.9996\\
-0.1185	-1.994\\
-0.1247	-1.9893\\
-0.1311	-1.9853\\
-0.1376	-1.9821\\
-0.2155	-1.9821\\
-0.3626	-2.021\\
-0.3665	-2.0239\\
-0.37	-2.0272\\
-0.38	-2.0372\\
-0.3831	-2.0406\\
-0.3918	-2.0512\\
-0.399	-2.0622\\
-0.405	-2.0734\\
-0.4096	-2.0848\\
-0.4129	-2.0962\\
-0.4129	-2.1764\\
-0.3741	-2.3235\\
-0.3711	-2.3326\\
-0.3674	-2.3411\\
-0.3629	-2.3489\\
-0.3579	-2.356\\
-0.3525	-2.3623\\
-0.3466	-2.3679\\
-0.3404	-2.3726\\
-0.334	-2.3766\\
-0.3275	-2.3798\\
}--cycle;

\addplot[area legend, draw=black, fill=mycolor1, forget plot]
table[row sep=crcr] {%
x	y\\
-0.1939	-2.3741\\
-0.1136	-2.3741\\
0.0239	-2.3226\\
0.0279	-2.3196\\
0.0314	-2.3164\\
0.0414	-2.3064\\
0.0445	-2.3029\\
0.0531	-2.2923\\
0.0604	-2.2813\\
0.0664	-2.2701\\
0.071	-2.2587\\
0.0743	-2.2473\\
0.0743	-2.1633\\
0.0713	-2.1542\\
0.0198	-2.0166\\
0.0161	-2.0081\\
0.0117	-2.0003\\
0.0067	-1.9933\\
0.0012	-1.9869\\
-0.0047	-1.9814\\
-0.0109	-1.9766\\
-0.0173	-1.9727\\
-0.0238	-1.9695\\
-0.1041	-1.9695\\
-0.2416	-2.021\\
-0.2456	-2.0239\\
-0.2491	-2.0272\\
-0.2591	-2.0372\\
-0.2622	-2.0406\\
-0.2708	-2.0512\\
-0.2781	-2.0622\\
-0.2841	-2.0735\\
-0.2887	-2.0848\\
-0.292	-2.0962\\
-0.292	-2.1802\\
-0.289	-2.1893\\
-0.2375	-2.3269\\
-0.2338	-2.3354\\
-0.2294	-2.3432\\
-0.2244	-2.3503\\
-0.2189	-2.3566\\
-0.213	-2.3621\\
-0.2068	-2.3669\\
-0.2004	-2.3709\\
-0.1939	-2.3741\\
}--cycle;

\addplot[area legend, draw=black, fill=mycolor1, forget plot]
table[row sep=crcr] {%
x	y\\
-0.0661	-2.3557\\
0.0161	-2.3557\\
0.1433	-2.2929\\
0.1473	-2.2899\\
0.1508	-2.2867\\
0.1608	-2.2767\\
0.1639	-2.2733\\
0.1725	-2.2627\\
0.1798	-2.2517\\
0.1857	-2.2404\\
0.1903	-2.229\\
0.1937	-2.2177\\
0.1937	-2.1299\\
0.1907	-2.1207\\
0.187	-2.1122\\
0.1242	-1.9851\\
0.1198	-1.9772\\
0.1148	-1.9702\\
0.1093	-1.9638\\
0.1034	-1.9583\\
0.0973	-1.9535\\
0.0909	-1.9496\\
0.0843	-1.9464\\
0.0021	-1.9464\\
-0.1251	-2.0091\\
-0.129	-2.0121\\
-0.1326	-2.0153\\
-0.1426	-2.0253\\
-0.1456	-2.0288\\
-0.1543	-2.0393\\
-0.1616	-2.0503\\
-0.1675	-2.0616\\
-0.1721	-2.073\\
-0.1755	-2.0844\\
-0.1755	-2.1721\\
-0.1725	-2.1813\\
-0.1687	-2.1898\\
-0.106	-2.317\\
-0.1016	-2.3248\\
-0.0966	-2.3319\\
-0.0911	-2.3382\\
-0.0852	-2.3437\\
-0.079	-2.3485\\
-0.0726	-2.3525\\
-0.0661	-2.3557\\
}--cycle;

\addplot[area legend, draw=black, fill=mycolor1, forget plot]
table[row sep=crcr] {%
x	y\\
0.0549	-2.3257\\
0.1386	-2.3257\\
0.2548	-2.2531\\
0.2587	-2.2501\\
0.2623	-2.2469\\
0.2723	-2.2369\\
0.2753	-2.2334\\
0.284	-2.2229\\
0.2912	-2.2119\\
0.2972	-2.2006\\
0.3018	-2.1892\\
0.3052	-2.1778\\
0.3052	-2.0863\\
0.3022	-2.0771\\
0.2984	-2.0686\\
0.294	-2.0608\\
0.2215	-1.9447\\
0.2165	-1.9376\\
0.211	-1.9313\\
0.2051	-1.9257\\
0.1989	-1.921\\
0.1925	-1.917\\
0.186	-1.9138\\
0.1023	-1.9138\\
-0.0139	-1.9863\\
-0.0178	-1.9893\\
-0.0214	-1.9925\\
-0.0314	-2.0025\\
-0.0344	-2.006\\
-0.0431	-2.0166\\
-0.0504	-2.0275\\
-0.0563	-2.0388\\
-0.0609	-2.0502\\
-0.0643	-2.0616\\
-0.0643	-2.1532\\
-0.0613	-2.1623\\
-0.0575	-2.1708\\
-0.0531	-2.1786\\
0.0194	-2.2948\\
0.0244	-2.3019\\
0.0299	-2.3082\\
0.0358	-2.3137\\
0.042	-2.3185\\
0.0484	-2.3225\\
0.0549	-2.3257\\
}--cycle;

\addplot[area legend, draw=black, fill=mycolor1, forget plot]
table[row sep=crcr] {%
x	y\\
0.1683	-2.2852\\
0.2531	-2.2852\\
0.257	-2.2822\\
0.3616	-2.2013\\
0.3651	-2.198\\
0.3751	-2.188\\
0.3782	-2.1846\\
0.3869	-2.174\\
0.3941	-2.163\\
0.4001	-2.1518\\
0.4047	-2.1404\\
0.408	-2.129\\
0.408	-2.0336\\
0.4051	-2.0245\\
0.4013	-2.0159\\
0.3969	-2.0081\\
0.3919	-2.0011\\
0.311	-1.8965\\
0.3055	-1.8901\\
0.2996	-1.8846\\
0.2935	-1.8798\\
0.2871	-1.8759\\
0.2805	-1.8727\\
0.1957	-1.8727\\
0.1918	-1.8756\\
0.0872	-1.9565\\
0.0836	-1.9598\\
0.0736	-1.9698\\
0.0706	-1.9732\\
0.0619	-1.9838\\
0.0546	-1.9948\\
0.0487	-2.0061\\
0.0441	-2.0174\\
0.0407	-2.0288\\
0.0407	-2.1242\\
0.0437	-2.1334\\
0.0475	-2.1419\\
0.0519	-2.1497\\
0.0569	-2.1568\\
0.1378	-2.2614\\
0.1433	-2.2677\\
0.1491	-2.2732\\
0.1553	-2.278\\
0.1617	-2.2819\\
0.1683	-2.2852\\
}--cycle;

\addplot[area legend, draw=black, fill=mycolor1, forget plot]
table[row sep=crcr] {%
x	y\\
0.2733	-2.2353\\
0.3587	-2.2353\\
0.3627	-2.2323\\
0.3662	-2.2291\\
0.4589	-2.1413\\
0.4689	-2.1313\\
0.472	-2.1278\\
0.4806	-2.1173\\
0.4879	-2.1063\\
0.4939	-2.095\\
0.4985	-2.0836\\
0.5018	-2.0722\\
0.5018	-1.9731\\
0.4988	-1.9639\\
0.4951	-1.9554\\
0.4907	-1.9476\\
0.4857	-1.9405\\
0.4802	-1.9342\\
0.3924	-1.8415\\
0.3865	-1.836\\
0.3803	-1.8312\\
0.3739	-1.8273\\
0.3674	-1.8241\\
0.2819	-1.8241\\
0.278	-1.827\\
0.2744	-1.8303\\
0.1817	-1.9181\\
0.1717	-1.9281\\
0.1687	-1.9315\\
0.16	-1.9421\\
0.1527	-1.9531\\
0.1468	-1.9644\\
0.1422	-1.9757\\
0.1388	-1.9871\\
0.1388	-2.0863\\
0.1418	-2.0954\\
0.1456	-2.1039\\
0.15	-2.1117\\
0.155	-2.1188\\
0.1605	-2.1251\\
0.2483	-2.2178\\
0.2542	-2.2234\\
0.2603	-2.2281\\
0.2667	-2.2321\\
0.2733	-2.2353\\
}--cycle;

\addplot[area legend, draw=black, fill=mycolor1, forget plot]
table[row sep=crcr] {%
x	y\\
0.3693	-2.1773\\
0.4551	-2.1773\\
0.4591	-2.1743\\
0.4626	-2.1711\\
0.4726	-2.1611\\
0.4757	-2.1576\\
0.5563	-2.0643\\
0.5649	-2.0538\\
0.5722	-2.0428\\
0.5781	-2.0315\\
0.5827	-2.0201\\
0.5861	-2.0087\\
0.5861	-1.9059\\
0.5831	-1.8968\\
0.5794	-1.8883\\
0.575	-1.8804\\
0.57	-1.8734\\
0.5645	-1.867\\
0.5586	-1.8615\\
0.4653	-1.7809\\
0.4591	-1.7762\\
0.4527	-1.7722\\
0.4462	-1.769\\
0.3604	-1.769\\
0.3564	-1.7719\\
0.3529	-1.7752\\
0.3429	-1.7852\\
0.3399	-1.7886\\
0.2593	-1.8819\\
0.2506	-1.8925\\
0.2433	-1.9035\\
0.2374	-1.9148\\
0.2328	-1.9262\\
0.2294	-1.9375\\
0.2294	-2.0404\\
0.2324	-2.0495\\
0.2362	-2.058\\
0.2406	-2.0658\\
0.2456	-2.0729\\
0.251	-2.0792\\
0.2569	-2.0848\\
0.3502	-2.1654\\
0.3564	-2.1701\\
0.3628	-2.1741\\
0.3693	-2.1773\\
}--cycle;

\addplot[area legend, draw=black, fill=mycolor1, forget plot]
table[row sep=crcr] {%
x	y\\
0.4558	-2.1123\\
0.542	-2.1123\\
0.546	-2.1093\\
0.5495	-2.1061\\
0.5595	-2.0961\\
0.5626	-2.0926\\
0.5712	-2.0821\\
0.6396	-1.9846\\
0.6469	-1.9737\\
0.6529	-1.9624\\
0.6575	-1.951\\
0.6608	-1.9396\\
0.6608	-1.8332\\
0.6579	-1.8241\\
0.6541	-1.8156\\
0.6497	-1.8078\\
0.6447	-1.8007\\
0.6392	-1.7944\\
0.6333	-1.7888\\
0.6272	-1.7841\\
0.5297	-1.7156\\
0.5233	-1.7117\\
0.5168	-1.7085\\
0.4306	-1.7085\\
0.4266	-1.7114\\
0.4231	-1.7147\\
0.4131	-1.7247\\
0.4101	-1.7281\\
0.4014	-1.7387\\
0.333	-1.8361\\
0.3257	-1.8471\\
0.3198	-1.8584\\
0.3151	-1.8698\\
0.3118	-1.8811\\
0.3118	-1.9875\\
0.3148	-1.9967\\
0.3185	-2.0052\\
0.3229	-2.013\\
0.3279	-2.0201\\
0.3334	-2.0264\\
0.3393	-2.0319\\
0.3455	-2.0367\\
0.4429	-2.1051\\
0.4493	-2.1091\\
0.4558	-2.1123\\
}--cycle;

\addplot[area legend, draw=black, fill=mycolor1, forget plot]
table[row sep=crcr] {%
x	y\\
0.5323	-2.0415\\
0.6192	-2.0415\\
0.6232	-2.0385\\
0.6267	-2.0353\\
0.6367	-2.0253\\
0.6398	-2.0218\\
0.6484	-2.0113\\
0.6557	-2.0003\\
0.7121	-1.9001\\
0.718	-1.8888\\
0.7226	-1.8774\\
0.726	-1.866\\
0.726	-1.7562\\
0.723	-1.7471\\
0.7192	-1.7386\\
0.7148	-1.7308\\
0.7098	-1.7237\\
0.7044	-1.7174\\
0.6985	-1.7118\\
0.6923	-1.7071\\
0.6859	-1.7031\\
0.5857	-1.6467\\
0.5792	-1.6435\\
0.4922	-1.6435\\
0.4883	-1.6465\\
0.4848	-1.6497\\
0.4748	-1.6597\\
0.4717	-1.6632\\
0.463	-1.6738\\
0.4558	-1.6848\\
0.3994	-1.785\\
0.3935	-1.7962\\
0.3888	-1.8076\\
0.3855	-1.819\\
0.3855	-1.9288\\
0.3885	-1.9379\\
0.3922	-1.9464\\
0.3966	-1.9542\\
0.4016	-1.9613\\
0.4071	-1.9676\\
0.413	-1.9732\\
0.4192	-1.9779\\
0.4256	-1.9819\\
0.5258	-2.0383\\
0.5323	-2.0415\\
}--cycle;

\addplot[area legend, draw=black, fill=mycolor1, forget plot]
table[row sep=crcr] {%
x	y\\
0.5986	-1.966\\
0.6866	-1.966\\
0.6905	-1.963\\
0.6941	-1.9598\\
0.7041	-1.9498\\
0.7071	-1.9464\\
0.7158	-1.9358\\
0.7231	-1.9248\\
0.729	-1.9135\\
0.7735	-1.8118\\
0.7781	-1.8004\\
0.7815	-1.7891\\
0.7815	-1.676\\
0.7785	-1.6669\\
0.7747	-1.6584\\
0.7703	-1.6506\\
0.7653	-1.6435\\
0.7598	-1.6372\\
0.754	-1.6316\\
0.7478	-1.6269\\
0.7414	-1.6229\\
0.7349	-1.6197\\
0.6332	-1.5752\\
0.5452	-1.5752\\
0.5412	-1.5782\\
0.5377	-1.5814\\
0.5277	-1.5914\\
0.5246	-1.5949\\
0.516	-1.6054\\
0.5087	-1.6164\\
0.5028	-1.6277\\
0.4583	-1.7294\\
0.4537	-1.7408\\
0.4503	-1.7522\\
0.4503	-1.8652\\
0.4533	-1.8743\\
0.457	-1.8828\\
0.4615	-1.8906\\
0.4665	-1.8977\\
0.4719	-1.904\\
0.4778	-1.9096\\
0.484	-1.9143\\
0.4904	-1.9183\\
0.4969	-1.9215\\
0.5986	-1.966\\
}--cycle;

\addplot[area legend, draw=black, fill=mycolor1, forget plot]
table[row sep=crcr] {%
x	y\\
0.6548	-1.887\\
0.744	-1.887\\
0.748	-1.8841\\
0.7515	-1.8808\\
0.7615	-1.8708\\
0.7646	-1.8674\\
0.7732	-1.8568\\
0.7805	-1.8458\\
0.7864	-1.8346\\
0.791	-1.8232\\
0.824	-1.7212\\
0.8273	-1.7098\\
0.8273	-1.5937\\
0.8244	-1.5846\\
0.8206	-1.5761\\
0.8162	-1.5683\\
0.8112	-1.5612\\
0.8057	-1.5549\\
0.7998	-1.5493\\
0.7937	-1.5446\\
0.7873	-1.5406\\
0.7807	-1.5374\\
0.6787	-1.5044\\
0.5895	-1.5044\\
0.5855	-1.5074\\
0.582	-1.5106\\
0.572	-1.5206\\
0.5689	-1.5241\\
0.5603	-1.5347\\
0.553	-1.5457\\
0.5471	-1.5569\\
0.5425	-1.5683\\
0.5095	-1.6703\\
0.5062	-1.6817\\
0.5062	-1.7977\\
0.5091	-1.8069\\
0.5129	-1.8154\\
0.5173	-1.8232\\
0.5223	-1.8303\\
0.5278	-1.8366\\
0.5337	-1.8421\\
0.5398	-1.8469\\
0.5462	-1.8509\\
0.5528	-1.8541\\
0.6548	-1.887\\
}--cycle;

\addplot[area legend, draw=black, fill=mycolor1, forget plot]
table[row sep=crcr] {%
x	y\\
0.7008	-1.8056\\
0.7916	-1.8056\\
0.7955	-1.8027\\
0.799	-1.7994\\
0.809	-1.7894\\
0.8121	-1.786\\
0.8207	-1.7754\\
0.828	-1.7644\\
0.834	-1.7531\\
0.8386	-1.7418\\
0.8419	-1.7304\\
0.8638	-1.6292\\
0.8638	-1.5103\\
0.8608	-1.5012\\
0.857	-1.4927\\
0.8526	-1.4849\\
0.8476	-1.4778\\
0.8421	-1.4715\\
0.8363	-1.4659\\
0.8301	-1.4612\\
0.8237	-1.4572\\
0.8172	-1.454\\
0.716	-1.4322\\
0.6252	-1.4322\\
0.6213	-1.4351\\
0.6178	-1.4384\\
0.6078	-1.4484\\
0.6047	-1.4518\\
0.5961	-1.4624\\
0.5888	-1.4734\\
0.5828	-1.4846\\
0.5782	-1.496\\
0.5749	-1.5074\\
0.553	-1.6086\\
0.553	-1.7275\\
0.556	-1.7366\\
0.5598	-1.7451\\
0.5642	-1.7529\\
0.5692	-1.76\\
0.5747	-1.7663\\
0.5805	-1.7719\\
0.5867	-1.7766\\
0.5931	-1.7806\\
0.5996	-1.7838\\
0.7008	-1.8056\\
}--cycle;

\addplot[area legend, draw=black, fill=mycolor1, forget plot]
table[row sep=crcr] {%
x	y\\
0.737	-1.7228\\
0.8294	-1.7228\\
0.8333	-1.7199\\
0.8369	-1.7166\\
0.8469	-1.7066\\
0.8499	-1.7032\\
0.8586	-1.6926\\
0.8659	-1.6816\\
0.8718	-1.6704\\
0.8764	-1.659\\
0.8798	-1.6476\\
0.891	-1.5483\\
0.891	-1.4268\\
0.888	-1.4177\\
0.8843	-1.4092\\
0.8798	-1.4013\\
0.8748	-1.3943\\
0.8694	-1.3879\\
0.8635	-1.3824\\
0.8573	-1.3776\\
0.8509	-1.3737\\
0.8444	-1.3705\\
0.7451	-1.3593\\
0.6526	-1.3593\\
0.6487	-1.3622\\
0.6452	-1.3655\\
0.6352	-1.3755\\
0.6321	-1.3789\\
0.6235	-1.3895\\
0.6162	-1.4005\\
0.6102	-1.4117\\
0.6056	-1.4231\\
0.6023	-1.4345\\
0.5911	-1.5338\\
0.5911	-1.6553\\
0.594	-1.6644\\
0.5978	-1.6729\\
0.6022	-1.6807\\
0.6072	-1.6878\\
0.6127	-1.6941\\
0.6185	-1.6997\\
0.6247	-1.7044\\
0.6311	-1.7084\\
0.6376	-1.7116\\
0.737	-1.7228\\
}--cycle;

\addplot[area legend, draw=black, fill=mycolor1, forget plot]
table[row sep=crcr] {%
x	y\\
0.7635	-1.6397\\
0.8577	-1.6397\\
0.8617	-1.6367\\
0.8652	-1.6334\\
0.8752	-1.6234\\
0.8783	-1.62\\
0.8869	-1.6094\\
0.8942	-1.5984\\
0.9001	-1.5872\\
0.9048	-1.5758\\
0.9081	-1.5644\\
0.9093	-1.4679\\
0.9093	-1.3441\\
0.9063	-1.335\\
0.9026	-1.3265\\
0.8982	-1.3186\\
0.8932	-1.3116\\
0.8877	-1.3052\\
0.8818	-1.2997\\
0.8756	-1.2949\\
0.8692	-1.291\\
0.8627	-1.2878\\
0.7662	-1.2866\\
0.6719	-1.2866\\
0.668	-1.2895\\
0.6645	-1.2928\\
0.6545	-1.3028\\
0.6514	-1.3062\\
0.6428	-1.3168\\
0.6355	-1.3278\\
0.6295	-1.339\\
0.6249	-1.3504\\
0.6216	-1.3618\\
0.6204	-1.4583\\
0.6204	-1.5821\\
0.6233	-1.5913\\
0.6271	-1.5998\\
0.6315	-1.6076\\
0.6365	-1.6147\\
0.642	-1.621\\
0.6479	-1.6265\\
0.654	-1.6313\\
0.6604	-1.6353\\
0.667	-1.6385\\
0.7635	-1.6397\\
}--cycle;

\addplot[area legend, draw=black, fill=mycolor1, forget plot]
table[row sep=crcr] {%
x	y\\
0.6879	-1.5652\\
0.7841	-1.5652\\
0.8769	-1.557\\
0.8809	-1.554\\
0.8844	-1.5508\\
0.8944	-1.5408\\
0.8975	-1.5374\\
0.9061	-1.5268\\
0.9134	-1.5158\\
0.9194	-1.5045\\
0.924	-1.4931\\
0.9273	-1.4817\\
0.9273	-1.3559\\
0.9192	-1.263\\
0.9162	-1.2539\\
0.9124	-1.2454\\
0.908	-1.2376\\
0.903	-1.2305\\
0.8976	-1.2242\\
0.8917	-1.2186\\
0.8855	-1.2139\\
0.8791	-1.2099\\
0.8726	-1.2067\\
0.7763	-1.2067\\
0.6835	-1.2149\\
0.6795	-1.2178\\
0.676	-1.2211\\
0.666	-1.2311\\
0.663	-1.2345\\
0.6543	-1.2451\\
0.647	-1.2561\\
0.6411	-1.2673\\
0.6365	-1.2787\\
0.6331	-1.2901\\
0.6331	-1.416\\
0.6413	-1.5088\\
0.6442	-1.518\\
0.648	-1.5265\\
0.6524	-1.5343\\
0.6574	-1.5414\\
0.6629	-1.5477\\
0.6688	-1.5532\\
0.6749	-1.558\\
0.6813	-1.562\\
0.6879	-1.5652\\
}--cycle;

\addplot[area legend, draw=black, fill=mycolor1, forget plot]
table[row sep=crcr] {%
x	y\\
0.7007	-1.4925\\
0.799	-1.4925\\
0.8874	-1.4758\\
0.8914	-1.4728\\
0.8949	-1.4696\\
0.9049	-1.4596\\
0.908	-1.4561\\
0.9166	-1.4455\\
0.9239	-1.4345\\
0.9298	-1.4233\\
0.9345	-1.4119\\
0.9378	-1.4005\\
0.9378	-1.2728\\
0.921	-1.1844\\
0.918	-1.1753\\
0.9143	-1.1667\\
0.9099	-1.1589\\
0.9049	-1.1519\\
0.8994	-1.1455\\
0.8935	-1.14\\
0.8873	-1.1352\\
0.8809	-1.1313\\
0.8744	-1.1281\\
0.7761	-1.1281\\
0.6877	-1.1448\\
0.6837	-1.1478\\
0.6802	-1.151\\
0.6702	-1.161\\
0.6671	-1.1645\\
0.6585	-1.1751\\
0.6512	-1.1861\\
0.6453	-1.1973\\
0.6407	-1.2087\\
0.6373	-1.2201\\
0.6373	-1.3478\\
0.6541	-1.4362\\
0.6571	-1.4453\\
0.6608	-1.4539\\
0.6652	-1.4617\\
0.6702	-1.4687\\
0.6757	-1.4751\\
0.6816	-1.4806\\
0.6878	-1.4854\\
0.6942	-1.4893\\
0.7007	-1.4925\\
}--cycle;

\addplot[area legend, draw=black, fill=mycolor1, forget plot]
table[row sep=crcr] {%
x	y\\
0.7059	-1.4213\\
0.8063	-1.4213\\
0.8896	-1.3967\\
0.8936	-1.3937\\
0.8971	-1.3905\\
0.9071	-1.3805\\
0.9102	-1.377\\
0.9188	-1.3665\\
0.9261	-1.3555\\
0.9321	-1.3442\\
0.9367	-1.3328\\
0.94	-1.3214\\
0.94	-1.1922\\
0.9154	-1.1089\\
0.9124	-1.0997\\
0.9086	-1.0912\\
0.9042	-1.0834\\
0.8992	-1.0763\\
0.8938	-1.07\\
0.8879	-1.0645\\
0.8817	-1.0597\\
0.8753	-1.0557\\
0.8688	-1.0525\\
0.7683	-1.0525\\
0.685	-1.0772\\
0.681	-1.0801\\
0.6775	-1.0834\\
0.6675	-1.0934\\
0.6644	-1.0968\\
0.6558	-1.1074\\
0.6485	-1.1184\\
0.6426	-1.1296\\
0.638	-1.141\\
0.6346	-1.1524\\
0.6346	-1.2817\\
0.6593	-1.365\\
0.6622	-1.3741\\
0.666	-1.3826\\
0.6704	-1.3905\\
0.6754	-1.3975\\
0.6809	-1.4039\\
0.6867	-1.4094\\
0.6929	-1.4142\\
0.6993	-1.4181\\
0.7059	-1.4213\\
}--cycle;

\addplot[area legend, draw=black, fill=mycolor1, forget plot]
table[row sep=crcr] {%
x	y\\
0.7038	-1.3522\\
0.8065	-1.3522\\
0.8841	-1.3205\\
0.8881	-1.3175\\
0.8916	-1.3143\\
0.9016	-1.3043\\
0.9047	-1.3009\\
0.9133	-1.2903\\
0.9206	-1.2793\\
0.9265	-1.268\\
0.9312	-1.2566\\
0.9345	-1.2453\\
0.9345	-1.1147\\
0.9315	-1.1055\\
0.8998	-1.0279\\
0.8961	-1.0194\\
0.8917	-1.0116\\
0.8867	-1.0045\\
0.8812	-0.9982\\
0.8753	-0.9926\\
0.8691	-0.9879\\
0.8627	-0.9839\\
0.8562	-0.9807\\
0.7536	-0.9807\\
0.6759	-1.0124\\
0.6719	-1.0154\\
0.6684	-1.0186\\
0.6584	-1.0286\\
0.6553	-1.032\\
0.6467	-1.0426\\
0.6394	-1.0536\\
0.6335	-1.0649\\
0.6288	-1.0763\\
0.6255	-1.0877\\
0.6255	-1.2182\\
0.6285	-1.2274\\
0.6602	-1.305\\
0.6639	-1.3135\\
0.6684	-1.3213\\
0.6733	-1.3284\\
0.6788	-1.3347\\
0.6847	-1.3403\\
0.6909	-1.345\\
0.6973	-1.349\\
0.7038	-1.3522\\
}--cycle;

\addplot[area legend, draw=black, fill=mycolor1, forget plot]
table[row sep=crcr] {%
x	y\\
0.6951	-1.2858\\
0.7999	-1.2858\\
0.8715	-1.2479\\
0.8755	-1.2449\\
0.879	-1.2417\\
0.889	-1.2317\\
0.892	-1.2282\\
0.9007	-1.2176\\
0.908	-1.2066\\
0.9139	-1.1954\\
0.9185	-1.184\\
0.9219	-1.1726\\
0.9219	-1.041\\
0.9189	-1.0318\\
0.9152	-1.0233\\
0.8772	-0.9518\\
0.8728	-0.944\\
0.8678	-0.9369\\
0.8623	-0.9306\\
0.8564	-0.925\\
0.8502	-0.9203\\
0.8438	-0.9163\\
0.8373	-0.9131\\
0.7325	-0.9131\\
0.6609	-0.951\\
0.6569	-0.954\\
0.6534	-0.9572\\
0.6434	-0.9672\\
0.6404	-0.9707\\
0.6317	-0.9813\\
0.6244	-0.9923\\
0.6185	-1.0035\\
0.6139	-1.0149\\
0.6105	-1.0263\\
0.6105	-1.1579\\
0.6135	-1.1671\\
0.6172	-1.1756\\
0.6552	-1.2471\\
0.6596	-1.2549\\
0.6646	-1.262\\
0.6701	-1.2683\\
0.676	-1.2739\\
0.6821	-1.2786\\
0.6885	-1.2826\\
0.6951	-1.2858\\
}--cycle;

\addplot[area legend, draw=black, fill=mycolor1, forget plot]
table[row sep=crcr] {%
x	y\\
0.6802	-1.2226\\
0.7872	-1.2226\\
0.8523	-1.1793\\
0.8563	-1.1763\\
0.8598	-1.1731\\
0.8698	-1.1631\\
0.8729	-1.1596\\
0.8815	-1.1491\\
0.8888	-1.1381\\
0.8947	-1.1268\\
0.8994	-1.1154\\
0.9027	-1.104\\
0.9027	-0.9716\\
0.8997	-0.9625\\
0.896	-0.954\\
0.8916	-0.9462\\
0.8482	-0.881\\
0.8432	-0.874\\
0.8377	-0.8676\\
0.8319	-0.8621\\
0.8257	-0.8573\\
0.8193	-0.8534\\
0.8127	-0.8502\\
0.7057	-0.8502\\
0.6406	-0.8935\\
0.6366	-0.8965\\
0.6331	-0.8997\\
0.6231	-0.9097\\
0.6201	-0.9132\\
0.6114	-0.9238\\
0.6041	-0.9347\\
0.5982	-0.946\\
0.5936	-0.9574\\
0.5902	-0.9688\\
0.5902	-1.1012\\
0.5932	-1.1103\\
0.5969	-1.1188\\
0.6014	-1.1267\\
0.6447	-1.1918\\
0.6497	-1.1988\\
0.6552	-1.2052\\
0.6611	-1.2107\\
0.6673	-1.2155\\
0.6737	-1.2194\\
0.6802	-1.2226\\
}--cycle;

\addplot[area legend, draw=black, fill=mycolor1, forget plot]
table[row sep=crcr] {%
x	y\\
0.6597	-1.1632\\
0.7689	-1.1632\\
0.7729	-1.1602\\
0.8313	-1.1123\\
0.8348	-1.109\\
0.8448	-1.099\\
0.8479	-1.0956\\
0.8565	-1.085\\
0.8638	-1.074\\
0.8697	-1.0627\\
0.8744	-1.0514\\
0.8777	-1.04\\
0.8777	-0.907\\
0.8747	-0.8979\\
0.871	-0.8894\\
0.8666	-0.8815\\
0.8616	-0.8745\\
0.8136	-0.8161\\
0.8081	-0.8098\\
0.8023	-0.8042\\
0.7961	-0.7995\\
0.7897	-0.7955\\
0.7832	-0.7923\\
0.6739	-0.7923\\
0.67	-0.7953\\
0.6116	-0.8432\\
0.6081	-0.8464\\
0.5981	-0.8564\\
0.595	-0.8599\\
0.5864	-0.8705\\
0.5791	-0.8815\\
0.5731	-0.8927\\
0.5685	-0.9041\\
0.5652	-0.9155\\
0.5652	-1.0485\\
0.5681	-1.0576\\
0.5719	-1.0661\\
0.5763	-1.0739\\
0.5813	-1.081\\
0.6292	-1.1394\\
0.6347	-1.1457\\
0.6406	-1.1512\\
0.6468	-1.156\\
0.6532	-1.16\\
0.6597	-1.1632\\
}--cycle;

\addplot[area legend, draw=black, fill=mycolor1, forget plot]
table[row sep=crcr] {%
x	y\\
0.6343	-1.1078\\
0.7456	-1.1078\\
0.7496	-1.1048\\
0.7531	-1.1016\\
0.7631	-1.0916\\
0.8146	-1.0399\\
0.8176	-1.0364\\
0.8263	-1.0259\\
0.8336	-1.0149\\
0.8395	-1.0036\\
0.8441	-0.9922\\
0.8475	-0.9808\\
0.8475	-0.8476\\
0.8445	-0.8384\\
0.8408	-0.8299\\
0.8363	-0.8221\\
0.8314	-0.815\\
0.8259	-0.8087\\
0.7742	-0.7572\\
0.7683	-0.7517\\
0.7621	-0.7469\\
0.7557	-0.743\\
0.7492	-0.7398\\
0.6378	-0.7398\\
0.6338	-0.7427\\
0.6303	-0.746\\
0.6203	-0.756\\
0.5689	-0.8077\\
0.5658	-0.8111\\
0.5572	-0.8217\\
0.5499	-0.8327\\
0.5439	-0.8439\\
0.5393	-0.8553\\
0.536	-0.8667\\
0.536	-1\\
0.5389	-1.0091\\
0.5427	-1.0176\\
0.5471	-1.0255\\
0.5521	-1.0325\\
0.5576	-1.0388\\
0.6093	-1.0903\\
0.6151	-1.0959\\
0.6213	-1.1006\\
0.6277	-1.1046\\
0.6343	-1.1078\\
}--cycle;

\addplot[area legend, draw=black, fill=mycolor1, forget plot]
table[row sep=crcr] {%
x	y\\
0.6044	-1.0568\\
0.7179	-1.0568\\
0.7219	-1.0539\\
0.7254	-1.0506\\
0.7354	-1.0406\\
0.7385	-1.0372\\
0.7471	-1.0266\\
0.7916	-0.972\\
0.7989	-0.961\\
0.8048	-0.9497\\
0.8094	-0.9383\\
0.8128	-0.927\\
0.8128	-0.7936\\
0.8098	-0.7844\\
0.806	-0.7759\\
0.8016	-0.7681\\
0.7966	-0.761\\
0.7911	-0.7547\\
0.7853	-0.7492\\
0.7306	-0.7047\\
0.7245	-0.6999\\
0.7181	-0.696\\
0.7115	-0.6928\\
0.598	-0.6928\\
0.5941	-0.6957\\
0.5906	-0.699\\
0.5806	-0.709\\
0.5775	-0.7124\\
0.5689	-0.723\\
0.5244	-0.7776\\
0.5171	-0.7886\\
0.5112	-0.7999\\
0.5065	-0.8113\\
0.5032	-0.8227\\
0.5032	-0.956\\
0.5062	-0.9652\\
0.5099	-0.9737\\
0.5143	-0.9815\\
0.5193	-0.9886\\
0.5248	-0.9949\\
0.5307	-1.0004\\
0.5853	-1.0449\\
0.5915	-1.0497\\
0.5979	-1.0536\\
0.6044	-1.0568\\
}--cycle;

\addplot[area legend, draw=black, fill=mycolor1, forget plot]
table[row sep=crcr] {%
x	y\\
0.5708	-1.0108\\
0.6864	-1.0108\\
0.6903	-1.0078\\
0.6938	-1.0045\\
0.7038	-0.9945\\
0.7069	-0.9911\\
0.7156	-0.9805\\
0.7228	-0.9695\\
0.7603	-0.9128\\
0.7662	-0.9015\\
0.7709	-0.8901\\
0.7742	-0.8787\\
0.7742	-0.745\\
0.7712	-0.7359\\
0.7675	-0.7274\\
0.7631	-0.7196\\
0.7581	-0.7125\\
0.7526	-0.7062\\
0.7467	-0.7006\\
0.7405	-0.6959\\
0.6838	-0.6584\\
0.6774	-0.6544\\
0.6708	-0.6512\\
0.5553	-0.6512\\
0.5513	-0.6542\\
0.5478	-0.6575\\
0.5378	-0.6675\\
0.5348	-0.6709\\
0.5261	-0.6815\\
0.5188	-0.6925\\
0.4814	-0.7492\\
0.4754	-0.7605\\
0.4708	-0.7719\\
0.4675	-0.7833\\
0.4675	-0.917\\
0.4705	-0.9261\\
0.4742	-0.9346\\
0.4786	-0.9424\\
0.4836	-0.9495\\
0.4891	-0.9558\\
0.495	-0.9614\\
0.5012	-0.9661\\
0.5579	-1.0036\\
0.5643	-1.0075\\
0.5708	-1.0108\\
}--cycle;

\addplot[area legend, draw=black, fill=mycolor1, forget plot]
table[row sep=crcr] {%
x	y\\
0.5341	-0.9697\\
0.6516	-0.9697\\
0.6556	-0.9667\\
0.6591	-0.9635\\
0.6691	-0.9535\\
0.6721	-0.9501\\
0.6808	-0.9395\\
0.6881	-0.9285\\
0.694	-0.9172\\
0.7245	-0.8591\\
0.7291	-0.8477\\
0.7325	-0.8363\\
0.7325	-0.7021\\
0.7295	-0.693\\
0.7258	-0.6845\\
0.7213	-0.6767\\
0.7163	-0.6696\\
0.7109	-0.6633\\
0.705	-0.6577\\
0.6988	-0.653\\
0.6924	-0.649\\
0.6343	-0.6185\\
0.6277	-0.6153\\
0.5103	-0.6153\\
0.5063	-0.6183\\
0.5028	-0.6215\\
0.4928	-0.6315\\
0.4897	-0.6349\\
0.4811	-0.6455\\
0.4738	-0.6565\\
0.4679	-0.6678\\
0.4373	-0.7259\\
0.4327	-0.7373\\
0.4294	-0.7487\\
0.4294	-0.8829\\
0.4324	-0.892\\
0.4361	-0.9005\\
0.4405	-0.9083\\
0.4455	-0.9154\\
0.451	-0.9217\\
0.4569	-0.9273\\
0.4631	-0.932\\
0.4694	-0.936\\
0.5276	-0.9665\\
0.5341	-0.9697\\
}--cycle;

\addplot[area legend, draw=black, fill=mycolor1, forget plot]
table[row sep=crcr] {%
x	y\\
0.4949	-0.9338\\
0.6142	-0.9338\\
0.6181	-0.9308\\
0.6217	-0.9276\\
0.6317	-0.9176\\
0.6347	-0.9142\\
0.6434	-0.9036\\
0.6507	-0.8926\\
0.6566	-0.8813\\
0.6612	-0.8699\\
0.6849	-0.8111\\
0.6883	-0.7997\\
0.6883	-0.6649\\
0.6853	-0.6558\\
0.6815	-0.6473\\
0.6771	-0.6395\\
0.6721	-0.6324\\
0.6667	-0.6261\\
0.6608	-0.6205\\
0.6546	-0.6158\\
0.6482	-0.6118\\
0.6417	-0.6086\\
0.5829	-0.5849\\
0.4636	-0.5849\\
0.4596	-0.5879\\
0.4561	-0.5911\\
0.4461	-0.6011\\
0.443	-0.6045\\
0.4344	-0.6151\\
0.4271	-0.6261\\
0.4212	-0.6374\\
0.4165	-0.6488\\
0.3928	-0.7076\\
0.3895	-0.719\\
0.3895	-0.8538\\
0.3925	-0.8629\\
0.3962	-0.8714\\
0.4006	-0.8792\\
0.4056	-0.8863\\
0.4111	-0.8926\\
0.417	-0.8982\\
0.4232	-0.9029\\
0.4295	-0.9069\\
0.4361	-0.9101\\
0.4949	-0.9338\\
}--cycle;

\addplot[area legend, draw=black, fill=mycolor1, forget plot]
table[row sep=crcr] {%
x	y\\
0.4537	-0.9031\\
0.5747	-0.9031\\
0.5787	-0.9001\\
0.5822	-0.8969\\
0.5922	-0.8869\\
0.5953	-0.8834\\
0.6039	-0.8729\\
0.6112	-0.8619\\
0.6171	-0.8506\\
0.6218	-0.8392\\
0.6251	-0.8278\\
0.6422	-0.769\\
0.6422	-0.6335\\
0.6392	-0.6243\\
0.6355	-0.6158\\
0.6311	-0.608\\
0.6261	-0.6009\\
0.6206	-0.5946\\
0.6147	-0.5891\\
0.6085	-0.5843\\
0.6021	-0.5803\\
0.5956	-0.5771\\
0.5368	-0.56\\
0.4158	-0.56\\
0.4119	-0.563\\
0.4083	-0.5662\\
0.3983	-0.5762\\
0.3953	-0.5797\\
0.3866	-0.5902\\
0.3793	-0.6012\\
0.3734	-0.6125\\
0.3688	-0.6239\\
0.3654	-0.6353\\
0.3483	-0.6941\\
0.3483	-0.8296\\
0.3513	-0.8388\\
0.3551	-0.8473\\
0.3595	-0.8551\\
0.3645	-0.8622\\
0.3699	-0.8685\\
0.3758	-0.874\\
0.382	-0.8788\\
0.3884	-0.8828\\
0.3949	-0.886\\
0.4537	-0.9031\\
}--cycle;

\addplot[area legend, draw=black, fill=mycolor1, forget plot]
table[row sep=crcr] {%
x	y\\
0.4112	-0.8775\\
0.5338	-0.8775\\
0.5378	-0.8746\\
0.5413	-0.8713\\
0.5513	-0.8613\\
0.5543	-0.8579\\
0.563	-0.8473\\
0.5703	-0.8363\\
0.5762	-0.8251\\
0.5808	-0.8137\\
0.5842	-0.8023\\
0.5949	-0.7441\\
0.5949	-0.6077\\
0.592	-0.5985\\
0.5882	-0.59\\
0.5838	-0.5822\\
0.5788	-0.5751\\
0.5733	-0.5688\\
0.5675	-0.5633\\
0.5613	-0.5585\\
0.5549	-0.5545\\
0.5483	-0.5513\\
0.4902	-0.5406\\
0.3676	-0.5406\\
0.3637	-0.5435\\
0.3601	-0.5468\\
0.3501	-0.5568\\
0.3471	-0.5602\\
0.3384	-0.5708\\
0.3311	-0.5818\\
0.3252	-0.593\\
0.3206	-0.6044\\
0.3172	-0.6158\\
0.3065	-0.674\\
0.3065	-0.8104\\
0.3094	-0.8196\\
0.3132	-0.8281\\
0.3176	-0.8359\\
0.3226	-0.843\\
0.3281	-0.8493\\
0.334	-0.8548\\
0.3401	-0.8596\\
0.3465	-0.8636\\
0.3531	-0.8668\\
0.4112	-0.8775\\
}--cycle;

\addplot[area legend, draw=black, fill=mycolor1, forget plot]
table[row sep=crcr] {%
x	y\\
0.3679	-0.8571\\
0.4919	-0.8571\\
0.4959	-0.8541\\
0.4994	-0.8509\\
0.5094	-0.8409\\
0.5125	-0.8374\\
0.5211	-0.8269\\
0.5284	-0.8159\\
0.5343	-0.8046\\
0.539	-0.7932\\
0.5423	-0.7818\\
0.547	-0.7249\\
0.547	-0.5875\\
0.5441	-0.5783\\
0.5403	-0.5698\\
0.5359	-0.562\\
0.5309	-0.5549\\
0.5254	-0.5486\\
0.5196	-0.5431\\
0.5134	-0.5383\\
0.507	-0.5343\\
0.5004	-0.5311\\
0.4436	-0.5264\\
0.3195	-0.5264\\
0.3155	-0.5293\\
0.312	-0.5326\\
0.302	-0.5426\\
0.299	-0.546\\
0.2903	-0.5566\\
0.283	-0.5676\\
0.2771	-0.5789\\
0.2725	-0.5902\\
0.2691	-0.6016\\
0.2644	-0.6585\\
0.2644	-0.796\\
0.2674	-0.8051\\
0.2711	-0.8137\\
0.2755	-0.8215\\
0.2805	-0.8285\\
0.286	-0.8349\\
0.2919	-0.8404\\
0.2981	-0.8452\\
0.3045	-0.8491\\
0.311	-0.8523\\
0.3679	-0.8571\\
}--cycle;

\addplot[area legend, draw=black, fill=mycolor1, forget plot]
table[row sep=crcr] {%
x	y\\
0.2692	-0.8425\\
0.3945	-0.8425\\
0.4496	-0.8416\\
0.4536	-0.8386\\
0.4571	-0.8354\\
0.4671	-0.8254\\
0.4702	-0.822\\
0.4788	-0.8114\\
0.4861	-0.8004\\
0.492	-0.7891\\
0.4967	-0.7777\\
0.5	-0.7663\\
0.5	-0.6278\\
0.4991	-0.5727\\
0.4961	-0.5635\\
0.4924	-0.555\\
0.4879	-0.5472\\
0.4829	-0.5401\\
0.4775	-0.5338\\
0.4716	-0.5283\\
0.4654	-0.5235\\
0.459	-0.5195\\
0.4525	-0.5163\\
0.3271	-0.5163\\
0.272	-0.5172\\
0.268	-0.5202\\
0.2645	-0.5235\\
0.2545	-0.5335\\
0.2515	-0.5369\\
0.2428	-0.5475\\
0.2355	-0.5585\\
0.2296	-0.5697\\
0.225	-0.5811\\
0.2216	-0.5925\\
0.2216	-0.7311\\
0.2226	-0.7862\\
0.2255	-0.7953\\
0.2293	-0.8038\\
0.2337	-0.8117\\
0.2387	-0.8187\\
0.2442	-0.8251\\
0.25	-0.8306\\
0.2562	-0.8354\\
0.2626	-0.8393\\
0.2692	-0.8425\\
}--cycle;

\addplot[area legend, draw=black, fill=mycolor1, forget plot]
table[row sep=crcr] {%
x	y\\
0.228	-0.8371\\
0.3545	-0.8371\\
0.4074	-0.8309\\
0.4114	-0.828\\
0.4149	-0.8247\\
0.4249	-0.8147\\
0.4279	-0.8113\\
0.4366	-0.8007\\
0.4439	-0.7897\\
0.4498	-0.7784\\
0.4544	-0.7671\\
0.4578	-0.7557\\
0.4578	-0.6159\\
0.4516	-0.5631\\
0.4486	-0.5539\\
0.4449	-0.5454\\
0.4404	-0.5376\\
0.4354	-0.5305\\
0.43	-0.5242\\
0.4241	-0.5187\\
0.4179	-0.5139\\
0.4115	-0.5099\\
0.405	-0.5067\\
0.2785	-0.5067\\
0.2256	-0.5129\\
0.2216	-0.5159\\
0.2181	-0.5191\\
0.2081	-0.5291\\
0.205	-0.5326\\
0.1964	-0.5432\\
0.1891	-0.5542\\
0.1832	-0.5654\\
0.1786	-0.5768\\
0.1752	-0.5882\\
0.1752	-0.7279\\
0.1814	-0.7808\\
0.1844	-0.7899\\
0.1881	-0.7984\\
0.1926	-0.8063\\
0.1975	-0.8133\\
0.203	-0.8196\\
0.2089	-0.8252\\
0.2151	-0.83\\
0.2215	-0.8339\\
0.228	-0.8371\\
}--cycle;

\addplot[area legend, draw=black, fill=mycolor1, forget plot]
table[row sep=crcr] {%
x	y\\
0.188	-0.8359\\
0.3155	-0.8359\\
0.3657	-0.8248\\
0.3696	-0.8219\\
0.3732	-0.8186\\
0.3832	-0.8086\\
0.3862	-0.8052\\
0.3949	-0.7946\\
0.4022	-0.7836\\
0.4081	-0.7724\\
0.4127	-0.761\\
0.4161	-0.7496\\
0.4161	-0.6087\\
0.405	-0.5585\\
0.402	-0.5493\\
0.3983	-0.5408\\
0.3939	-0.533\\
0.3889	-0.5259\\
0.3834	-0.5196\\
0.3775	-0.5141\\
0.3713	-0.5093\\
0.3649	-0.5053\\
0.3584	-0.5021\\
0.2309	-0.5021\\
0.1807	-0.5132\\
0.1767	-0.5161\\
0.1732	-0.5194\\
0.1632	-0.5294\\
0.1601	-0.5328\\
0.1515	-0.5434\\
0.1442	-0.5544\\
0.1383	-0.5656\\
0.1337	-0.577\\
0.1303	-0.5884\\
0.1303	-0.7294\\
0.1414	-0.7796\\
0.1443	-0.7887\\
0.1481	-0.7972\\
0.1525	-0.805\\
0.1575	-0.8121\\
0.163	-0.8184\\
0.1689	-0.824\\
0.175	-0.8287\\
0.1814	-0.8327\\
0.188	-0.8359\\
}--cycle;

\addplot[area legend, draw=black, fill=mycolor1, forget plot]
table[row sep=crcr] {%
x	y\\
0.1494	-0.8386\\
0.2778	-0.8386\\
0.3249	-0.8231\\
0.3289	-0.8201\\
0.3324	-0.8169\\
0.3424	-0.8069\\
0.3455	-0.8035\\
0.3541	-0.7929\\
0.3614	-0.7819\\
0.3673	-0.7706\\
0.3719	-0.7592\\
0.3753	-0.7478\\
0.3753	-0.6057\\
0.3723	-0.5965\\
0.3569	-0.5494\\
0.3531	-0.5409\\
0.3487	-0.533\\
0.3437	-0.526\\
0.3382	-0.5196\\
0.3323	-0.5141\\
0.3262	-0.5093\\
0.3198	-0.5054\\
0.3132	-0.5022\\
0.1848	-0.5022\\
0.1377	-0.5176\\
0.1337	-0.5206\\
0.1302	-0.5238\\
0.1202	-0.5338\\
0.1171	-0.5373\\
0.1085	-0.5479\\
0.1012	-0.5588\\
0.0953	-0.5701\\
0.0907	-0.5815\\
0.0873	-0.5929\\
0.0873	-0.7351\\
0.0903	-0.7442\\
0.1057	-0.7914\\
0.1095	-0.7999\\
0.1139	-0.8077\\
0.1189	-0.8148\\
0.1244	-0.8211\\
0.1303	-0.8266\\
0.1364	-0.8314\\
0.1428	-0.8354\\
0.1494	-0.8386\\
}--cycle;

\addplot[area legend, draw=black, fill=mycolor1, forget plot]
table[row sep=crcr] {%
x	y\\
0.1125	-0.8448\\
0.2416	-0.8448\\
0.2855	-0.8254\\
0.2894	-0.8225\\
0.2929	-0.8192\\
0.3029	-0.8092\\
0.306	-0.8058\\
0.3146	-0.7952\\
0.3219	-0.7842\\
0.3279	-0.773\\
0.3325	-0.7616\\
0.3358	-0.7502\\
0.3358	-0.6067\\
0.3329	-0.5976\\
0.3291	-0.5891\\
0.3097	-0.5453\\
0.3053	-0.5375\\
0.3003	-0.5304\\
0.2948	-0.5241\\
0.2889	-0.5185\\
0.2828	-0.5138\\
0.2764	-0.5098\\
0.2698	-0.5066\\
0.1407	-0.5066\\
0.0969	-0.526\\
0.093	-0.529\\
0.0894	-0.5322\\
0.0794	-0.5422\\
0.0764	-0.5456\\
0.0677	-0.5562\\
0.0604	-0.5672\\
0.0545	-0.5785\\
0.0499	-0.5899\\
0.0465	-0.6012\\
0.0465	-0.7447\\
0.0495	-0.7538\\
0.0533	-0.7623\\
0.0727	-0.8062\\
0.0771	-0.814\\
0.0821	-0.821\\
0.0876	-0.8274\\
0.0934	-0.8329\\
0.0996	-0.8377\\
0.106	-0.8416\\
0.1125	-0.8448\\
}--cycle;

\addplot[area legend, draw=black, fill=mycolor1, forget plot]
table[row sep=crcr] {%
x	y\\
0.0778	-0.8544\\
0.2074	-0.8544\\
0.2476	-0.8316\\
0.2516	-0.8286\\
0.2551	-0.8254\\
0.2651	-0.8154\\
0.2682	-0.8119\\
0.2768	-0.8013\\
0.2841	-0.7903\\
0.2901	-0.7791\\
0.2947	-0.7677\\
0.298	-0.7563\\
0.298	-0.6116\\
0.295	-0.6024\\
0.2913	-0.5939\\
0.2869	-0.5861\\
0.264	-0.5459\\
0.259	-0.5388\\
0.2535	-0.5325\\
0.2477	-0.527\\
0.2415	-0.5222\\
0.2351	-0.5182\\
0.2286	-0.515\\
0.0989	-0.515\\
0.0587	-0.5379\\
0.0547	-0.5408\\
0.0512	-0.5441\\
0.0412	-0.5541\\
0.0381	-0.5575\\
0.0295	-0.5681\\
0.0222	-0.5791\\
0.0163	-0.5904\\
0.0116	-0.6017\\
0.0083	-0.6131\\
0.0083	-0.7578\\
0.0113	-0.767\\
0.015	-0.7755\\
0.0194	-0.7833\\
0.0423	-0.8235\\
0.0473	-0.8306\\
0.0528	-0.8369\\
0.0587	-0.8425\\
0.0648	-0.8472\\
0.0712	-0.8512\\
0.0778	-0.8544\\
}--cycle;

\addplot[area legend, draw=black, fill=mycolor1, forget plot]
table[row sep=crcr] {%
x	y\\
0.0453	-0.867\\
0.1753	-0.867\\
0.2118	-0.8411\\
0.2157	-0.8382\\
0.2193	-0.8349\\
0.2293	-0.8249\\
0.2323	-0.8215\\
0.241	-0.8109\\
0.2483	-0.7999\\
0.2542	-0.7887\\
0.2588	-0.7773\\
0.2622	-0.7659\\
0.2622	-0.6199\\
0.2592	-0.6107\\
0.2554	-0.6022\\
0.251	-0.5944\\
0.246	-0.5873\\
0.2202	-0.5509\\
0.2147	-0.5446\\
0.2088	-0.539\\
0.2026	-0.5343\\
0.1962	-0.5303\\
0.1897	-0.5271\\
0.0596	-0.5271\\
0.0232	-0.5529\\
0.0192	-0.5559\\
0.0157	-0.5591\\
0.0057	-0.5691\\
0.0027	-0.5726\\
-0.006	-0.5832\\
-0.0133	-0.5942\\
-0.0192	-0.6054\\
-0.0238	-0.6168\\
-0.0272	-0.6282\\
-0.0272	-0.7742\\
-0.0242	-0.7833\\
-0.0205	-0.7918\\
-0.016	-0.7996\\
-0.011	-0.8067\\
0.0148	-0.8432\\
0.0203	-0.8495\\
0.0262	-0.855\\
0.0323	-0.8598\\
0.0387	-0.8638\\
0.0453	-0.867\\
}--cycle;

\addplot[area legend, draw=black, fill=mycolor1, forget plot]
table[row sep=crcr] {%
x	y\\
0.0153	-0.8822\\
0.1456	-0.8822\\
0.1496	-0.8792\\
0.1821	-0.8508\\
0.1856	-0.8476\\
0.1956	-0.8376\\
0.1987	-0.8342\\
0.2073	-0.8236\\
0.2146	-0.8126\\
0.2205	-0.8013\\
0.2252	-0.7899\\
0.2285	-0.7786\\
0.2285	-0.6313\\
0.2255	-0.6222\\
0.2218	-0.6137\\
0.2174	-0.6059\\
0.2124	-0.5988\\
0.2069	-0.5925\\
0.1785	-0.5599\\
0.1727	-0.5544\\
0.1665	-0.5496\\
0.1601	-0.5457\\
0.1536	-0.5424\\
0.0232	-0.5424\\
0.0193	-0.5454\\
-0.0133	-0.5738\\
-0.0168	-0.577\\
-0.0268	-0.587\\
-0.0299	-0.5904\\
-0.0385	-0.601\\
-0.0458	-0.612\\
-0.0517	-0.6233\\
-0.0564	-0.6347\\
-0.0597	-0.6461\\
-0.0597	-0.7933\\
-0.0567	-0.8024\\
-0.053	-0.8109\\
-0.0486	-0.8188\\
-0.0436	-0.8258\\
-0.0381	-0.8322\\
-0.0097	-0.8647\\
-0.0039	-0.8702\\
0.0023	-0.875\\
0.0087	-0.879\\
0.0153	-0.8822\\
}--cycle;

\addplot[area legend, draw=black, fill=mycolor1, forget plot]
table[row sep=crcr] {%
x	y\\
-0.0121	-0.8997\\
0.1184	-0.8997\\
0.1223	-0.8967\\
0.1259	-0.8934\\
0.1359	-0.8834\\
0.1644	-0.8531\\
0.1674	-0.8496\\
0.1761	-0.8391\\
0.1834	-0.8281\\
0.1893	-0.8168\\
0.1939	-0.8054\\
0.1973	-0.794\\
0.1973	-0.6456\\
0.1943	-0.6364\\
0.1906	-0.6279\\
0.1861	-0.6201\\
0.1811	-0.613\\
0.1757	-0.6067\\
0.1698	-0.6011\\
0.1394	-0.5726\\
0.1332	-0.5679\\
0.1268	-0.5639\\
0.1203	-0.5607\\
-0.0102	-0.5607\\
-0.0141	-0.5637\\
-0.0177	-0.5669\\
-0.0277	-0.5769\\
-0.0562	-0.6073\\
-0.0593	-0.6107\\
-0.0679	-0.6213\\
-0.0752	-0.6323\\
-0.0811	-0.6435\\
-0.0857	-0.6549\\
-0.0891	-0.6663\\
-0.0891	-0.8148\\
-0.0861	-0.8239\\
-0.0824	-0.8324\\
-0.0779	-0.8403\\
-0.073	-0.8473\\
-0.0675	-0.8537\\
-0.0616	-0.8592\\
-0.0312	-0.8877\\
-0.025	-0.8925\\
-0.0186	-0.8965\\
-0.0121	-0.8997\\
}--cycle;

\addplot[area legend, draw=black, fill=mycolor1, forget plot]
table[row sep=crcr] {%
x	y\\
-0.0368	-0.9191\\
0.0938	-0.9191\\
0.0978	-0.9161\\
0.1013	-0.9129\\
0.1113	-0.9029\\
0.1144	-0.8994\\
0.123	-0.8889\\
0.1475	-0.8569\\
0.1548	-0.846\\
0.1607	-0.8347\\
0.1653	-0.8233\\
0.1687	-0.8119\\
0.1687	-0.6622\\
0.1657	-0.6531\\
0.162	-0.6446\\
0.1575	-0.6368\\
0.1526	-0.6297\\
0.1471	-0.6234\\
0.1412	-0.6178\\
0.135	-0.6131\\
0.1031	-0.5886\\
0.0967	-0.5846\\
0.0902	-0.5814\\
-0.0404	-0.5814\\
-0.0444	-0.5844\\
-0.0479	-0.5876\\
-0.0579	-0.5976\\
-0.061	-0.6011\\
-0.0696	-0.6116\\
-0.0941	-0.6436\\
-0.1014	-0.6546\\
-0.1073	-0.6658\\
-0.1119	-0.6772\\
-0.1153	-0.6886\\
-0.1153	-0.8383\\
-0.1123	-0.8474\\
-0.1086	-0.8559\\
-0.1042	-0.8637\\
-0.0992	-0.8708\\
-0.0937	-0.8771\\
-0.0878	-0.8827\\
-0.0816	-0.8874\\
-0.0497	-0.9119\\
-0.0433	-0.9159\\
-0.0368	-0.9191\\
}--cycle;

\addplot[area legend, draw=black, fill=mycolor1, forget plot]
table[row sep=crcr] {%
x	y\\
-0.0587	-0.9401\\
0.0721	-0.9401\\
0.0761	-0.9372\\
0.0796	-0.9339\\
0.0896	-0.9239\\
0.0927	-0.9205\\
0.1013	-0.9099\\
0.1086	-0.8989\\
0.129	-0.8659\\
0.135	-0.8546\\
0.1396	-0.8432\\
0.1429	-0.8318\\
0.1429	-0.681\\
0.14	-0.6719\\
0.1362	-0.6634\\
0.1318	-0.6556\\
0.1268	-0.6485\\
0.1213	-0.6422\\
0.1154	-0.6366\\
0.1093	-0.6319\\
0.1029	-0.6279\\
0.0698	-0.6074\\
0.0633	-0.6042\\
-0.0675	-0.6042\\
-0.0715	-0.6072\\
-0.075	-0.6104\\
-0.085	-0.6204\\
-0.088	-0.6239\\
-0.0967	-0.6345\\
-0.104	-0.6455\\
-0.1244	-0.6785\\
-0.1304	-0.6897\\
-0.135	-0.7011\\
-0.1383	-0.7125\\
-0.1383	-0.8634\\
-0.1353	-0.8725\\
-0.1316	-0.881\\
-0.1272	-0.8888\\
-0.1222	-0.8959\\
-0.1167	-0.9022\\
-0.1108	-0.9078\\
-0.1046	-0.9125\\
-0.0982	-0.9165\\
-0.0652	-0.9369\\
-0.0587	-0.9401\\
}--cycle;

\addplot[area legend, draw=black, fill=mycolor1, forget plot]
table[row sep=crcr] {%
x	y\\
-0.0778	-0.9625\\
0.0533	-0.9625\\
0.0572	-0.9595\\
0.0608	-0.9563\\
0.0708	-0.9463\\
0.0738	-0.9428\\
0.0825	-0.9322\\
0.0898	-0.9213\\
0.0957	-0.91\\
0.1121	-0.8763\\
0.1168	-0.8649\\
0.1201	-0.8535\\
0.1201	-0.7016\\
0.1171	-0.6924\\
0.1134	-0.6839\\
0.109	-0.6761\\
0.104	-0.669\\
0.0985	-0.6627\\
0.0926	-0.6572\\
0.0864	-0.6524\\
0.08	-0.6484\\
0.0735	-0.6452\\
0.0398	-0.6288\\
-0.0913	-0.6288\\
-0.0953	-0.6317\\
-0.0988	-0.635\\
-0.1088	-0.645\\
-0.1118	-0.6484\\
-0.1205	-0.659\\
-0.1278	-0.67\\
-0.1337	-0.6812\\
-0.1502	-0.715\\
-0.1548	-0.7264\\
-0.1581	-0.7377\\
-0.1581	-0.8897\\
-0.1552	-0.8988\\
-0.1514	-0.9073\\
-0.147	-0.9151\\
-0.142	-0.9222\\
-0.1365	-0.9285\\
-0.1306	-0.9341\\
-0.1245	-0.9388\\
-0.1181	-0.9428\\
-0.1115	-0.946\\
-0.0778	-0.9625\\
}--cycle;

\addplot[area legend, draw=black, fill=mycolor1, forget plot]
table[row sep=crcr] {%
x	y\\
-0.0942	-0.9857\\
0.0373	-0.9857\\
0.0413	-0.9828\\
0.0448	-0.9795\\
0.0548	-0.9695\\
0.0579	-0.9661\\
0.0665	-0.9555\\
0.0738	-0.9445\\
0.0797	-0.9333\\
0.0844	-0.9219\\
0.0969	-0.8879\\
0.1003	-0.8765\\
0.1003	-0.7235\\
0.0973	-0.7144\\
0.0935	-0.7059\\
0.0891	-0.6981\\
0.0841	-0.691\\
0.0786	-0.6847\\
0.0728	-0.6791\\
0.0666	-0.6744\\
0.0602	-0.6704\\
0.0537	-0.6672\\
0.0197	-0.6546\\
-0.1118	-0.6546\\
-0.1158	-0.6576\\
-0.1193	-0.6608\\
-0.1293	-0.6708\\
-0.1323	-0.6743\\
-0.141	-0.6849\\
-0.1483	-0.6958\\
-0.1542	-0.7071\\
-0.1588	-0.7185\\
-0.1714	-0.7525\\
-0.1747	-0.7639\\
-0.1747	-0.9168\\
-0.1718	-0.926\\
-0.168	-0.9345\\
-0.1636	-0.9423\\
-0.1586	-0.9494\\
-0.1531	-0.9557\\
-0.1472	-0.9612\\
-0.1411	-0.966\\
-0.1347	-0.97\\
-0.1281	-0.9732\\
-0.0942	-0.9857\\
}--cycle;

\addplot[area legend, draw=black, fill=mycolor1, forget plot]
table[row sep=crcr] {%
x	y\\
-0.1077	-1.0096\\
0.0242	-1.0096\\
0.0282	-1.0067\\
0.0317	-1.0034\\
0.0417	-0.9934\\
0.0448	-0.99\\
0.0534	-0.9794\\
0.0607	-0.9684\\
0.0667	-0.9572\\
0.0713	-0.9458\\
0.0746	-0.9344\\
0.0834	-0.9005\\
0.0834	-0.7466\\
0.0804	-0.7374\\
0.0767	-0.7289\\
0.0723	-0.7211\\
0.0673	-0.714\\
0.0618	-0.7077\\
0.0559	-0.7022\\
0.0497	-0.6974\\
0.0433	-0.6934\\
0.0368	-0.6902\\
0.003	-0.6814\\
-0.129	-0.6814\\
-0.133	-0.6844\\
-0.1365	-0.6877\\
-0.1465	-0.6977\\
-0.1495	-0.7011\\
-0.1582	-0.7117\\
-0.1655	-0.7227\\
-0.1714	-0.7339\\
-0.176	-0.7453\\
-0.1794	-0.7567\\
-0.1882	-0.7906\\
-0.1882	-0.9445\\
-0.1852	-0.9537\\
-0.1814	-0.9622\\
-0.177	-0.97\\
-0.172	-0.977\\
-0.1666	-0.9834\\
-0.1607	-0.9889\\
-0.1545	-0.9937\\
-0.1481	-0.9976\\
-0.1416	-1.0008\\
-0.1077	-1.0096\\
}--cycle;

\addplot[area legend, draw=black, fill=mycolor1, forget plot]
table[row sep=crcr] {%
x	y\\
-0.1185	-1.0339\\
0.014	-1.0339\\
0.0179	-1.0309\\
0.0214	-1.0277\\
0.0314	-1.0177\\
0.0345	-1.0142\\
0.0432	-1.0036\\
0.0504	-0.9926\\
0.0564	-0.9814\\
0.061	-0.97\\
0.0644	-0.9586\\
0.0695	-0.9252\\
0.0695	-0.7704\\
0.0666	-0.7612\\
0.0628	-0.7527\\
0.0584	-0.7449\\
0.0534	-0.7378\\
0.0479	-0.7315\\
0.042	-0.726\\
0.0359	-0.7212\\
0.0295	-0.7173\\
0.0229	-0.714\\
-0.0104	-0.7089\\
-0.1429	-0.7089\\
-0.1469	-0.7118\\
-0.1504	-0.7151\\
-0.1604	-0.7251\\
-0.1635	-0.7285\\
-0.1721	-0.7391\\
-0.1794	-0.7501\\
-0.1854	-0.7613\\
-0.19	-0.7727\\
-0.1933	-0.7841\\
-0.1985	-0.8175\\
-0.1985	-0.9723\\
-0.1955	-0.9815\\
-0.1918	-0.99\\
-0.1874	-0.9978\\
-0.1824	-1.0049\\
-0.1769	-1.0112\\
-0.171	-1.0167\\
-0.1648	-1.0215\\
-0.1584	-1.0255\\
-0.1519	-1.0287\\
-0.1185	-1.0339\\
}--cycle;

\addplot[area legend, draw=black, fill=mycolor1, forget plot]
table[row sep=crcr] {%
x	y\\
-0.1267	-1.0581\\
0.0064	-1.0581\\
0.0104	-1.0552\\
0.0139	-1.0519\\
0.0239	-1.0419\\
0.027	-1.0385\\
0.0356	-1.0279\\
0.0429	-1.0169\\
0.0488	-1.0057\\
0.0535	-0.9943\\
0.0568	-0.9829\\
0.0586	-0.9503\\
0.0586	-0.7947\\
0.0556	-0.7855\\
0.0519	-0.777\\
0.0474	-0.7692\\
0.0424	-0.7621\\
0.037	-0.7558\\
0.0311	-0.7503\\
0.0249	-0.7455\\
0.0185	-0.7415\\
0.012	-0.7383\\
-0.0206	-0.7366\\
-0.1537	-0.7366\\
-0.1576	-0.7395\\
-0.1612	-0.7428\\
-0.1712	-0.7528\\
-0.1742	-0.7562\\
-0.1829	-0.7668\\
-0.1902	-0.7778\\
-0.1961	-0.789\\
-0.2007	-0.8004\\
-0.2041	-0.8118\\
-0.2058	-0.8444\\
-0.2058	-1\\
-0.2028	-1.0092\\
-0.1991	-1.0177\\
-0.1947	-1.0255\\
-0.1897	-1.0326\\
-0.1842	-1.0389\\
-0.1783	-1.0444\\
-0.1721	-1.0492\\
-0.1657	-1.0532\\
-0.1592	-1.0564\\
-0.1267	-1.0581\\
}--cycle;

\addplot[area legend, draw=black, fill=mycolor1, forget plot]
table[row sep=crcr] {%
x	y\\
-0.1637	-1.0836\\
-0.0299	-1.0836\\
0.0015	-1.0822\\
0.0055	-1.0792\\
0.009	-1.076\\
0.019	-1.066\\
0.0221	-1.0625\\
0.0307	-1.052\\
0.038	-1.041\\
0.0439	-1.0297\\
0.0486	-1.0183\\
0.0519	-1.0069\\
0.0519	-0.8505\\
0.0505	-0.8191\\
0.0475	-0.8099\\
0.0437	-0.8014\\
0.0393	-0.7936\\
0.0343	-0.7865\\
0.0288	-0.7802\\
0.023	-0.7747\\
0.0168	-0.7699\\
0.0104	-0.7659\\
0.0039	-0.7627\\
-0.1299	-0.7627\\
-0.1613	-0.7642\\
-0.1653	-0.7672\\
-0.1688	-0.7704\\
-0.1788	-0.7804\\
-0.1819	-0.7838\\
-0.1905	-0.7944\\
-0.1978	-0.8054\\
-0.2037	-0.8167\\
-0.2084	-0.8281\\
-0.2117	-0.8394\\
-0.2117	-0.9958\\
-0.2103	-1.0273\\
-0.2073	-1.0364\\
-0.2035	-1.0449\\
-0.1991	-1.0527\\
-0.1941	-1.0598\\
-0.1886	-1.0661\\
-0.1828	-1.0717\\
-0.1766	-1.0764\\
-0.1702	-1.0804\\
-0.1637	-1.0836\\
}--cycle;

\addplot[area legend, draw=black, fill=mycolor1, forget plot]
table[row sep=crcr] {%
x	y\\
-0.1654	-1.1102\\
-0.0309	-1.1102\\
-0.0009	-1.1058\\
0.0031	-1.1028\\
0.0066	-1.0995\\
0.0166	-1.0895\\
0.0197	-1.0861\\
0.0283	-1.0755\\
0.0356	-1.0645\\
0.0415	-1.0533\\
0.0462	-1.0419\\
0.0495	-1.0305\\
0.0495	-0.8735\\
0.0451	-0.8434\\
0.0421	-0.8342\\
0.0384	-0.8257\\
0.0339	-0.8179\\
0.029	-0.8108\\
0.0235	-0.8045\\
0.0176	-0.799\\
0.0114	-0.7942\\
0.005	-0.7903\\
-0.0015	-0.787\\
-0.1359	-0.787\\
-0.166	-0.7915\\
-0.17	-0.7944\\
-0.1735	-0.7977\\
-0.1835	-0.8077\\
-0.1866	-0.8111\\
-0.1952	-0.8217\\
-0.2025	-0.8327\\
-0.2084	-0.8439\\
-0.213	-0.8553\\
-0.2164	-0.8667\\
-0.2164	-1.0238\\
-0.212	-1.0538\\
-0.209	-1.063\\
-0.2052	-1.0715\\
-0.2008	-1.0793\\
-0.1958	-1.0864\\
-0.1903	-1.0927\\
-0.1845	-1.0982\\
-0.1783	-1.103\\
-0.1719	-1.107\\
-0.1654	-1.1102\\
}--cycle;

\addplot[area legend, draw=black, fill=mycolor1, forget plot]
table[row sep=crcr] {%
x	y\\
-0.1645	-1.1358\\
-0.0294	-1.1358\\
-0.0009	-1.1286\\
0.0031	-1.1257\\
0.0066	-1.1224\\
0.0166	-1.1124\\
0.0196	-1.109\\
0.0283	-1.0984\\
0.0356	-1.0874\\
0.0415	-1.0762\\
0.0461	-1.0648\\
0.0495	-1.0534\\
0.0495	-0.8958\\
0.0423	-0.8673\\
0.0393	-0.8582\\
0.0356	-0.8497\\
0.0312	-0.8419\\
0.0262	-0.8348\\
0.0207	-0.8285\\
0.0148	-0.8229\\
0.0087	-0.8182\\
0.0023	-0.8142\\
-0.0043	-0.811\\
-0.1394	-0.811\\
-0.1679	-0.8181\\
-0.1718	-0.8211\\
-0.1753	-0.8243\\
-0.1853	-0.8343\\
-0.1884	-0.8378\\
-0.1971	-0.8484\\
-0.2043	-0.8593\\
-0.2103	-0.8706\\
-0.2149	-0.882\\
-0.2182	-0.8934\\
-0.2182	-1.051\\
-0.2111	-1.0794\\
-0.2081	-1.0886\\
-0.2044	-1.0971\\
-0.2	-1.1049\\
-0.195	-1.112\\
-0.1895	-1.1183\\
-0.1836	-1.1239\\
-0.1774	-1.1286\\
-0.171	-1.1326\\
-0.1645	-1.1358\\
}--cycle;

\addplot[area legend, draw=black, fill=mycolor1, forget plot]
table[row sep=crcr] {%
x	y\\
-0.1612	-1.1603\\
-0.0254	-1.1603\\
0.0013	-1.1506\\
0.0052	-1.1477\\
0.0087	-1.1444\\
0.0187	-1.1344\\
0.0218	-1.131\\
0.0304	-1.1204\\
0.0377	-1.1094\\
0.0437	-1.0982\\
0.0483	-1.0868\\
0.0516	-1.0754\\
0.0516	-0.9173\\
0.0487	-0.9082\\
0.039	-0.8815\\
0.0353	-0.873\\
0.0309	-0.8652\\
0.0259	-0.8581\\
0.0204	-0.8518\\
0.0145	-0.8463\\
0.0084	-0.8415\\
0.002	-0.8375\\
-0.0046	-0.8343\\
-0.1404	-0.8343\\
-0.1671	-0.8439\\
-0.171	-0.8469\\
-0.1746	-0.8501\\
-0.1846	-0.8601\\
-0.1876	-0.8636\\
-0.1963	-0.8742\\
-0.2036	-0.8851\\
-0.2095	-0.8964\\
-0.2141	-0.9078\\
-0.2175	-0.9192\\
-0.2175	-1.0772\\
-0.2145	-1.0864\\
-0.2049	-1.1131\\
-0.2011	-1.1216\\
-0.1967	-1.1294\\
-0.1917	-1.1365\\
-0.1862	-1.1428\\
-0.1804	-1.1483\\
-0.1742	-1.1531\\
-0.1678	-1.157\\
-0.1612	-1.1603\\
}--cycle;

\addplot[area legend, draw=black, fill=mycolor1, forget plot]
table[row sep=crcr] {%
x	y\\
-0.1558	-1.1834\\
-0.0192	-1.1834\\
0.0055	-1.1716\\
0.0094	-1.1686\\
0.0129	-1.1654\\
0.0229	-1.1554\\
0.026	-1.1519\\
0.0346	-1.1413\\
0.0419	-1.1303\\
0.0479	-1.1191\\
0.0525	-1.1077\\
0.0558	-1.0963\\
0.0558	-0.9379\\
0.0529	-0.9287\\
0.0491	-0.9202\\
0.0373	-0.8955\\
0.0329	-0.8877\\
0.0279	-0.8806\\
0.0224	-0.8743\\
0.0165	-0.8688\\
0.0104	-0.864\\
0.004	-0.86\\
-0.0026	-0.8568\\
-0.1391	-0.8568\\
-0.1638	-0.8686\\
-0.1678	-0.8716\\
-0.1713	-0.8748\\
-0.1813	-0.8848\\
-0.1844	-0.8883\\
-0.193	-0.8989\\
-0.2003	-0.9099\\
-0.2063	-0.9211\\
-0.2109	-0.9325\\
-0.2142	-0.9439\\
-0.2142	-1.1023\\
-0.2112	-1.1115\\
-0.2075	-1.12\\
-0.1957	-1.1447\\
-0.1913	-1.1525\\
-0.1863	-1.1596\\
-0.1808	-1.1659\\
-0.1749	-1.1714\\
-0.1687	-1.1762\\
-0.1623	-1.1802\\
-0.1558	-1.1834\\
}--cycle;

\addplot[area legend, draw=black, fill=mycolor1, forget plot]
table[row sep=crcr] {%
x	y\\
-0.1484	-1.205\\
-0.0111	-1.205\\
0.0115	-1.1913\\
0.0155	-1.1883\\
0.019	-1.1851\\
0.029	-1.1751\\
0.032	-1.1716\\
0.0407	-1.161\\
0.048	-1.15\\
0.0539	-1.1388\\
0.0585	-1.1274\\
0.0619	-1.116\\
0.0619	-0.9573\\
0.0589	-0.9481\\
0.0552	-0.9396\\
0.0507	-0.9318\\
0.037	-0.9092\\
0.032	-0.9021\\
0.0265	-0.8958\\
0.0207	-0.8903\\
0.0145	-0.8855\\
0.0081	-0.8815\\
0.0016	-0.8783\\
-0.1357	-0.8783\\
-0.1583	-0.8921\\
-0.1623	-0.895\\
-0.1658	-0.8983\\
-0.1758	-0.9083\\
-0.1789	-0.9117\\
-0.1875	-0.9223\\
-0.1948	-0.9333\\
-0.2007	-0.9445\\
-0.2054	-0.9559\\
-0.2087	-0.9673\\
-0.2087	-1.1261\\
-0.2057	-1.1352\\
-0.202	-1.1437\\
-0.1976	-1.1515\\
-0.1838	-1.1741\\
-0.1789	-1.1812\\
-0.1734	-1.1875\\
-0.1675	-1.193\\
-0.1613	-1.1978\\
-0.1549	-1.2018\\
-0.1484	-1.205\\
}--cycle;

\addplot[area legend, draw=black, fill=mycolor1, forget plot]
table[row sep=crcr] {%
x	y\\
-0.1392	-1.225\\
-0.0012	-1.225\\
0.0028	-1.222\\
0.0232	-1.2066\\
0.0267	-1.2034\\
0.0367	-1.1934\\
0.0398	-1.19\\
0.0484	-1.1794\\
0.0557	-1.1684\\
0.0616	-1.1571\\
0.0662	-1.1457\\
0.0696	-1.1343\\
0.0696	-0.9754\\
0.0666	-0.9662\\
0.0629	-0.9577\\
0.0585	-0.9499\\
0.0535	-0.9429\\
0.0381	-0.9225\\
0.0326	-0.9162\\
0.0267	-0.9106\\
0.0206	-0.9058\\
0.0142	-0.9019\\
0.0076	-0.8987\\
-0.1304	-0.8987\\
-0.1344	-0.9016\\
-0.1547	-0.917\\
-0.1583	-0.9202\\
-0.1683	-0.9302\\
-0.1713	-0.9337\\
-0.18	-0.9443\\
-0.1873	-0.9553\\
-0.1932	-0.9665\\
-0.1978	-0.9779\\
-0.2012	-0.9893\\
-0.2012	-1.1482\\
-0.1982	-1.1574\\
-0.1944	-1.1659\\
-0.19	-1.1737\\
-0.185	-1.1808\\
-0.1697	-1.2012\\
-0.1642	-1.2075\\
-0.1583	-1.213\\
-0.1521	-1.2178\\
-0.1457	-1.2218\\
-0.1392	-1.225\\
}--cycle;

\addplot[area legend, draw=black, fill=mycolor1, forget plot]
table[row sep=crcr] {%
x	y\\
-0.1285	-1.2432\\
0.0103	-1.2432\\
0.0142	-1.2402\\
0.0178	-1.237\\
0.0359	-1.2203\\
0.0459	-1.2103\\
0.0489	-1.2068\\
0.0576	-1.1963\\
0.0649	-1.1853\\
0.0708	-1.174\\
0.0754	-1.1626\\
0.0788	-1.1512\\
0.0788	-0.9921\\
0.0758	-0.983\\
0.072	-0.9745\\
0.0676	-0.9667\\
0.0626	-0.9596\\
0.0572	-0.9533\\
0.0404	-0.9352\\
0.0345	-0.9296\\
0.0284	-0.9249\\
0.022	-0.9209\\
0.0154	-0.9177\\
-0.1233	-0.9177\\
-0.1273	-0.9207\\
-0.1308	-0.9239\\
-0.1489	-0.9406\\
-0.1589	-0.9506\\
-0.162	-0.9541\\
-0.1706	-0.9646\\
-0.1779	-0.9756\\
-0.1838	-0.9869\\
-0.1885	-0.9983\\
-0.1918	-1.0097\\
-0.1918	-1.1688\\
-0.1888	-1.1779\\
-0.1851	-1.1864\\
-0.1807	-1.1942\\
-0.1757	-1.2013\\
-0.1702	-1.2076\\
-0.1535	-1.2257\\
-0.1476	-1.2313\\
-0.1414	-1.236\\
-0.135	-1.24\\
-0.1285	-1.2432\\
}--cycle;

\addplot[area legend, draw=black, fill=mycolor1, forget plot]
table[row sep=crcr] {%
x	y\\
-0.1164	-1.2596\\
0.0231	-1.2596\\
0.027	-1.2567\\
0.0305	-1.2534\\
0.0405	-1.2434\\
0.0436	-1.24\\
0.0594	-1.2222\\
0.068	-1.2116\\
0.0753	-1.2006\\
0.0812	-1.1893\\
0.0859	-1.178\\
0.0892	-1.1666\\
0.0892	-1.0074\\
0.0862	-0.9983\\
0.0825	-0.9898\\
0.0781	-0.9819\\
0.0731	-0.9749\\
0.0676	-0.9685\\
0.0617	-0.963\\
0.0439	-0.9472\\
0.0377	-0.9425\\
0.0313	-0.9385\\
0.0248	-0.9353\\
-0.1147	-0.9353\\
-0.1186	-0.9383\\
-0.1222	-0.9415\\
-0.1322	-0.9515\\
-0.1352	-0.9549\\
-0.151	-0.9728\\
-0.1596	-0.9833\\
-0.1669	-0.9943\\
-0.1729	-1.0056\\
-0.1775	-1.017\\
-0.1808	-1.0284\\
-0.1808	-1.1875\\
-0.1779	-1.1967\\
-0.1741	-1.2052\\
-0.1697	-1.213\\
-0.1647	-1.2201\\
-0.1592	-1.2264\\
-0.1533	-1.2319\\
-0.1355	-1.2477\\
-0.1293	-1.2525\\
-0.1229	-1.2564\\
-0.1164	-1.2596\\
}--cycle;

\addplot[area legend, draw=black, fill=mycolor1, forget plot]
table[row sep=crcr] {%
x	y\\
-0.1033	-1.2742\\
0.0369	-1.2742\\
0.0409	-1.2713\\
0.0444	-1.268\\
0.0544	-1.258\\
0.0574	-1.2546\\
0.0661	-1.244\\
0.0795	-1.2254\\
0.0868	-1.2144\\
0.0928	-1.2031\\
0.0974	-1.1917\\
0.1007	-1.1803\\
0.1007	-1.0211\\
0.0977	-1.012\\
0.094	-1.0034\\
0.0896	-0.9956\\
0.0846	-0.9886\\
0.0791	-0.9822\\
0.0732	-0.9767\\
0.067	-0.9719\\
0.0484	-0.9585\\
0.042	-0.9545\\
0.0355	-0.9513\\
-0.1047	-0.9513\\
-0.1086	-0.9543\\
-0.1122	-0.9575\\
-0.1222	-0.9675\\
-0.1252	-0.971\\
-0.1339	-0.9815\\
-0.1473	-1.0002\\
-0.1546	-1.0112\\
-0.1605	-1.0224\\
-0.1651	-1.0338\\
-0.1685	-1.0452\\
-0.1685	-1.2044\\
-0.1655	-1.2136\\
-0.1618	-1.2221\\
-0.1574	-1.2299\\
-0.1524	-1.237\\
-0.1469	-1.2433\\
-0.141	-1.2488\\
-0.1348	-1.2536\\
-0.1162	-1.267\\
-0.1098	-1.271\\
-0.1033	-1.2742\\
}--cycle;

\addplot[area legend, draw=black, fill=mycolor1, forget plot]
table[row sep=crcr] {%
x	y\\
-0.0892	-1.287\\
0.0516	-1.287\\
0.0556	-1.284\\
0.0591	-1.2808\\
0.0691	-1.2708\\
0.0721	-1.2673\\
0.0808	-1.2568\\
0.0881	-1.2458\\
0.0992	-1.2266\\
0.1051	-1.2153\\
0.1098	-1.2039\\
0.1131	-1.1925\\
0.1131	-1.0332\\
0.1101	-1.024\\
0.1064	-1.0155\\
0.102	-1.0077\\
0.097	-1.0006\\
0.0915	-0.9943\\
0.0856	-0.9887\\
0.0794	-0.984\\
0.073	-0.98\\
0.0538	-0.9689\\
0.0473	-0.9657\\
-0.0935	-0.9657\\
-0.0975	-0.9687\\
-0.101	-0.9719\\
-0.111	-0.9819\\
-0.1141	-0.9853\\
-0.1227	-0.9959\\
-0.13	-1.0069\\
-0.1411	-1.0261\\
-0.147	-1.0374\\
-0.1517	-1.0488\\
-0.155	-1.0602\\
-0.155	-1.2195\\
-0.152	-1.2287\\
-0.1483	-1.2372\\
-0.1439	-1.245\\
-0.1389	-1.2521\\
-0.1334	-1.2584\\
-0.1275	-1.2639\\
-0.1213	-1.2687\\
-0.1149	-1.2727\\
-0.0957	-1.2838\\
-0.0892	-1.287\\
}--cycle;

\addplot[area legend, draw=black, fill=mycolor1, forget plot]
table[row sep=crcr] {%
x	y\\
-0.0745	-1.2979\\
0.067	-1.2979\\
0.0709	-1.2949\\
0.0744	-1.2917\\
0.0844	-1.2817\\
0.0875	-1.2782\\
0.0961	-1.2676\\
0.1034	-1.2567\\
0.1094	-1.2454\\
0.1182	-1.2259\\
0.1228	-1.2145\\
0.1262	-1.2031\\
0.1262	-1.0435\\
0.1232	-1.0344\\
0.1194	-1.0259\\
0.115	-1.0181\\
0.11	-1.011\\
0.1045	-1.0047\\
0.0987	-0.9991\\
0.0925	-0.9944\\
0.0861	-0.9904\\
0.0796	-0.9872\\
0.06	-0.9784\\
-0.0814	-0.9784\\
-0.0854	-0.9813\\
-0.0889	-0.9846\\
-0.0989	-0.9946\\
-0.1019	-0.998\\
-0.1106	-1.0086\\
-0.1179	-1.0196\\
-0.1238	-1.0308\\
-0.1326	-1.0504\\
-0.1373	-1.0618\\
-0.1406	-1.0731\\
-0.1406	-1.2327\\
-0.1376	-1.2418\\
-0.1339	-1.2503\\
-0.1295	-1.2582\\
-0.1245	-1.2652\\
-0.119	-1.2716\\
-0.1131	-1.2771\\
-0.1069	-1.2819\\
-0.1005	-1.2858\\
-0.094	-1.289\\
-0.0745	-1.2979\\
}--cycle;

\addplot[area legend, draw=black, fill=mycolor1, forget plot]
table[row sep=crcr] {%
x	y\\
-0.0593	-1.3069\\
0.0827	-1.3069\\
0.0867	-1.3039\\
0.0902	-1.3007\\
0.1002	-1.2907\\
0.1033	-1.2873\\
0.1119	-1.2767\\
0.1192	-1.2657\\
0.1251	-1.2544\\
0.1298	-1.243\\
0.1364	-1.2234\\
0.1397	-1.212\\
0.1397	-1.0523\\
0.1367	-1.0431\\
0.133	-1.0346\\
0.1286	-1.0268\\
0.1236	-1.0197\\
0.1181	-1.0134\\
0.1122	-1.0078\\
0.106	-1.0031\\
0.0996	-0.9991\\
0.0931	-0.9959\\
0.0735	-0.9893\\
-0.0685	-0.9893\\
-0.0725	-0.9923\\
-0.076	-0.9955\\
-0.086	-1.0055\\
-0.0891	-1.009\\
-0.0977	-1.0195\\
-0.105	-1.0305\\
-0.1109	-1.0418\\
-0.1156	-1.0532\\
-0.1222	-1.0728\\
-0.1255	-1.0842\\
-0.1255	-1.244\\
-0.1225	-1.2531\\
-0.1188	-1.2616\\
-0.1144	-1.2694\\
-0.1094	-1.2765\\
-0.1039	-1.2828\\
-0.098	-1.2884\\
-0.0918	-1.2931\\
-0.0854	-1.2971\\
-0.0789	-1.3003\\
-0.0593	-1.3069\\
}--cycle;

\addplot[area legend, draw=black, fill=mycolor1, forget plot]
table[row sep=crcr] {%
x	y\\
-0.0438	-1.3141\\
0.0987	-1.3141\\
0.1027	-1.3112\\
0.1062	-1.3079\\
0.1162	-1.2979\\
0.1193	-1.2945\\
0.1279	-1.2839\\
0.1352	-1.2729\\
0.1412	-1.2617\\
0.1458	-1.2503\\
0.1491	-1.2389\\
0.1536	-1.2194\\
0.1536	-1.0593\\
0.1506	-1.0502\\
0.1469	-1.0417\\
0.1424	-1.0339\\
0.1374	-1.0268\\
0.132	-1.0205\\
0.1261	-1.0149\\
0.1199	-1.0102\\
0.1135	-1.0062\\
0.107	-1.003\\
0.0875	-0.9985\\
-0.0551	-0.9985\\
-0.059	-1.0015\\
-0.0625	-1.0047\\
-0.0725	-1.0147\\
-0.0756	-1.0182\\
-0.0843	-1.0288\\
-0.0915	-1.0397\\
-0.0975	-1.051\\
-0.1021	-1.0624\\
-0.1054	-1.0738\\
-0.1099	-1.0933\\
-0.1099	-1.2533\\
-0.1069	-1.2625\\
-0.1032	-1.271\\
-0.0988	-1.2788\\
-0.0938	-1.2859\\
-0.0883	-1.2922\\
-0.0824	-1.2977\\
-0.0762	-1.3025\\
-0.0698	-1.3065\\
-0.0633	-1.3097\\
-0.0438	-1.3141\\
}--cycle;

\addplot[area legend, draw=black, fill=mycolor1, forget plot]
table[row sep=crcr] {%
x	y\\
-0.0283	-1.3196\\
0.1148	-1.3196\\
0.1188	-1.3166\\
0.1223	-1.3134\\
0.1323	-1.3034\\
0.1353	-1.2999\\
0.144	-1.2894\\
0.1513	-1.2784\\
0.1572	-1.2671\\
0.1618	-1.2557\\
0.1652	-1.2443\\
0.1676	-1.2252\\
0.1676	-1.0648\\
0.1646	-1.0556\\
0.1609	-1.0471\\
0.1564	-1.0393\\
0.1514	-1.0322\\
0.146	-1.0259\\
0.1401	-1.0204\\
0.1339	-1.0156\\
0.1275	-1.0116\\
0.121	-1.0084\\
0.1018	-1.006\\
-0.0412	-1.006\\
-0.0452	-1.009\\
-0.0487	-1.0122\\
-0.0587	-1.0222\\
-0.0618	-1.0257\\
-0.0704	-1.0363\\
-0.0777	-1.0473\\
-0.0836	-1.0585\\
-0.0883	-1.0699\\
-0.0916	-1.0813\\
-0.094	-1.1004\\
-0.094	-1.2608\\
-0.091	-1.27\\
-0.0873	-1.2785\\
-0.0829	-1.2863\\
-0.0779	-1.2934\\
-0.0724	-1.2997\\
-0.0665	-1.3052\\
-0.0603	-1.31\\
-0.0539	-1.314\\
-0.0474	-1.3172\\
-0.0283	-1.3196\\
}--cycle;

\addplot[area legend, draw=black, fill=mycolor1, forget plot]
table[row sep=crcr] {%
x	y\\
-0.0128	-1.3233\\
0.1307	-1.3233\\
0.1347	-1.3203\\
0.1382	-1.3171\\
0.1482	-1.3071\\
0.1513	-1.3037\\
0.1599	-1.2931\\
0.1672	-1.2821\\
0.1731	-1.2708\\
0.1778	-1.2594\\
0.1811	-1.2481\\
0.1816	-1.2294\\
0.1816	-1.0687\\
0.1786	-1.0595\\
0.1748	-1.051\\
0.1704	-1.0432\\
0.1654	-1.0361\\
0.1599	-1.0298\\
0.1541	-1.0243\\
0.1479	-1.0195\\
0.1415	-1.0155\\
0.135	-1.0123\\
0.1163	-1.0119\\
-0.0272	-1.0119\\
-0.0311	-1.0149\\
-0.0347	-1.0181\\
-0.0447	-1.0281\\
-0.0477	-1.0315\\
-0.0564	-1.0421\\
-0.0637	-1.0531\\
-0.0696	-1.0644\\
-0.0742	-1.0758\\
-0.0776	-1.0871\\
-0.078	-1.1058\\
-0.078	-1.2665\\
-0.075	-1.2757\\
-0.0713	-1.2842\\
-0.0669	-1.292\\
-0.0619	-1.2991\\
-0.0564	-1.3054\\
-0.0505	-1.3109\\
-0.0444	-1.3157\\
-0.038	-1.3196\\
-0.0314	-1.3229\\
-0.0128	-1.3233\\
}--cycle;

\addplot[area legend, draw=black, fill=mycolor1, forget plot]
table[row sep=crcr] {%
x	y\\
-0.0155	-1.3268\\
0.1284	-1.3268\\
0.1464	-1.3254\\
0.1503	-1.3224\\
0.1538	-1.3192\\
0.1638	-1.3092\\
0.1669	-1.3058\\
0.1755	-1.2952\\
0.1828	-1.2842\\
0.1888	-1.2729\\
0.1934	-1.2615\\
0.1967	-1.2501\\
0.1967	-1.089\\
0.1954	-1.0711\\
0.1924	-1.0619\\
0.1887	-1.0534\\
0.1842	-1.0456\\
0.1792	-1.0385\\
0.1738	-1.0322\\
0.1679	-1.0267\\
0.1617	-1.0219\\
0.1553	-1.0179\\
0.1488	-1.0147\\
0.0048	-1.0147\\
-0.0131	-1.0161\\
-0.0171	-1.0191\\
-0.0206	-1.0223\\
-0.0306	-1.0323\\
-0.0336	-1.0357\\
-0.0423	-1.0463\\
-0.0496	-1.0573\\
-0.0555	-1.0686\\
-0.0601	-1.08\\
-0.0635	-1.0914\\
-0.0635	-1.2525\\
-0.0621	-1.2704\\
-0.0591	-1.2796\\
-0.0554	-1.2881\\
-0.051	-1.2959\\
-0.046	-1.303\\
-0.0405	-1.3093\\
-0.0346	-1.3148\\
-0.0285	-1.3196\\
-0.0221	-1.3236\\
-0.0155	-1.3268\\
}--cycle;

\addplot[area legend, draw=black, fill=mycolor1, forget plot]
table[row sep=crcr] {%
x	y\\
0.0001	-1.329\\
0.1444	-1.329\\
0.1615	-1.3259\\
0.1655	-1.323\\
0.169	-1.3197\\
0.179	-1.3097\\
0.1821	-1.3063\\
0.1907	-1.2957\\
0.198	-1.2847\\
0.2039	-1.2735\\
0.2086	-1.2621\\
0.2119	-1.2507\\
0.2119	-1.0892\\
0.2089	-1.0721\\
0.2059	-1.0629\\
0.2022	-1.0544\\
0.1977	-1.0466\\
0.1927	-1.0395\\
0.1873	-1.0332\\
0.1814	-1.0277\\
0.1752	-1.0229\\
0.1688	-1.0189\\
0.1623	-1.0157\\
0.018	-1.0157\\
0.0009	-1.0187\\
-0.0031	-1.0217\\
-0.0066	-1.025\\
-0.0166	-1.035\\
-0.0197	-1.0384\\
-0.0283	-1.049\\
-0.0356	-1.06\\
-0.0415	-1.0712\\
-0.0462	-1.0826\\
-0.0495	-1.094\\
-0.0495	-1.2555\\
-0.0465	-1.2726\\
-0.0435	-1.2818\\
-0.0397	-1.2903\\
-0.0353	-1.2981\\
-0.0303	-1.3052\\
-0.0249	-1.3115\\
-0.019	-1.317\\
-0.0128	-1.3218\\
-0.0064	-1.3258\\
0.0001	-1.329\\
}--cycle;

\addplot[area legend, draw=black, fill=mycolor1, forget plot]
table[row sep=crcr] {%
x	y\\
0.0154	-1.3296\\
0.16	-1.3296\\
0.1761	-1.325\\
0.1801	-1.3221\\
0.1836	-1.3188\\
0.1936	-1.3088\\
0.1967	-1.3054\\
0.2053	-1.2948\\
0.2126	-1.2838\\
0.2185	-1.2726\\
0.2232	-1.2612\\
0.2265	-1.2498\\
0.2265	-1.0878\\
0.2219	-1.0717\\
0.219	-1.0626\\
0.2152	-1.054\\
0.2108	-1.0462\\
0.2058	-1.0392\\
0.2003	-1.0328\\
0.1945	-1.0273\\
0.1883	-1.0225\\
0.1819	-1.0186\\
0.1753	-1.0154\\
0.0307	-1.0154\\
0.0146	-1.0199\\
0.0106	-1.0229\\
0.0071	-1.0261\\
-0.0029	-1.0361\\
-0.0059	-1.0396\\
-0.0146	-1.0501\\
-0.0219	-1.0611\\
-0.0278	-1.0724\\
-0.0324	-1.0838\\
-0.0358	-1.0952\\
-0.0358	-1.2571\\
-0.0312	-1.2732\\
-0.0282	-1.2824\\
-0.0245	-1.2909\\
-0.0201	-1.2987\\
-0.0151	-1.3058\\
-0.0096	-1.3121\\
-0.0037	-1.3177\\
0.0025	-1.3224\\
0.0089	-1.3264\\
0.0154	-1.3296\\
}--cycle;

\addplot[area legend, draw=black, fill=mycolor1, forget plot]
table[row sep=crcr] {%
x	y\\
0.0301	-1.3287\\
0.175	-1.3287\\
0.19	-1.3228\\
0.194	-1.3198\\
0.1975	-1.3166\\
0.2075	-1.3066\\
0.2106	-1.3031\\
0.2192	-1.2926\\
0.2265	-1.2816\\
0.2324	-1.2703\\
0.2371	-1.2589\\
0.2404	-1.2475\\
0.2404	-1.0852\\
0.2374	-1.076\\
0.2315	-1.061\\
0.2277	-1.0524\\
0.2233	-1.0446\\
0.2183	-1.0376\\
0.2129	-1.0312\\
0.207	-1.0257\\
0.2008	-1.0209\\
0.1944	-1.017\\
0.1879	-1.0138\\
0.043	-1.0138\\
0.0279	-1.0197\\
0.024	-1.0227\\
0.0205	-1.0259\\
0.0105	-1.0359\\
0.0074	-1.0393\\
-0.0012	-1.0499\\
-0.0085	-1.0609\\
-0.0145	-1.0722\\
-0.0191	-1.0836\\
-0.0224	-1.095\\
-0.0224	-1.2573\\
-0.0195	-1.2664\\
-0.0135	-1.2815\\
-0.0098	-1.29\\
-0.0054	-1.2978\\
-0.0004	-1.3049\\
0.0051	-1.3112\\
0.011	-1.3168\\
0.0172	-1.3215\\
0.0236	-1.3255\\
0.0301	-1.3287\\
}--cycle;

\addplot[area legend, draw=black, fill=mycolor1, forget plot]
table[row sep=crcr] {%
x	y\\
0.0442	-1.3264\\
0.1892	-1.3264\\
0.2031	-1.3193\\
0.2071	-1.3163\\
0.2106	-1.3131\\
0.2206	-1.3031\\
0.2237	-1.2996\\
0.2323	-1.2891\\
0.2396	-1.2781\\
0.2455	-1.2668\\
0.2502	-1.2554\\
0.2535	-1.244\\
0.2535	-1.0813\\
0.2505	-1.0721\\
0.2468	-1.0636\\
0.2396	-1.0497\\
0.2352	-1.0419\\
0.2302	-1.0348\\
0.2247	-1.0285\\
0.2189	-1.023\\
0.2127	-1.0182\\
0.2063	-1.0142\\
0.1997	-1.011\\
0.0547	-1.011\\
0.0408	-1.0182\\
0.0368	-1.0211\\
0.0333	-1.0244\\
0.0233	-1.0344\\
0.0202	-1.0378\\
0.0116	-1.0484\\
0.0043	-1.0594\\
-0.0016	-1.0706\\
-0.0063	-1.082\\
-0.0096	-1.0934\\
-0.0096	-1.2562\\
-0.0066	-1.2653\\
-0.0029	-1.2739\\
0.0043	-1.2878\\
0.0087	-1.2956\\
0.0137	-1.3026\\
0.0192	-1.309\\
0.025	-1.3145\\
0.0312	-1.3193\\
0.0376	-1.3232\\
0.0442	-1.3264\\
}--cycle;

\addplot[area legend, draw=black, fill=mycolor1, forget plot]
table[row sep=crcr] {%
x	y\\
0.0574	-1.3229\\
0.2027	-1.3229\\
0.2153	-1.3147\\
0.2193	-1.3117\\
0.2228	-1.3085\\
0.2328	-1.2985\\
0.2359	-1.295\\
0.2445	-1.2845\\
0.2518	-1.2735\\
0.2577	-1.2622\\
0.2624	-1.2508\\
0.2657	-1.2394\\
0.2657	-1.0762\\
0.2627	-1.0671\\
0.259	-1.0586\\
0.2546	-1.0508\\
0.2464	-1.0381\\
0.2414	-1.031\\
0.2359	-1.0247\\
0.23	-1.0192\\
0.2238	-1.0144\\
0.2174	-1.0104\\
0.2109	-1.0072\\
0.0657	-1.0072\\
0.053	-1.0154\\
0.049	-1.0184\\
0.0455	-1.0217\\
0.0355	-1.0317\\
0.0325	-1.0351\\
0.0238	-1.0457\\
0.0165	-1.0567\\
0.0106	-1.0679\\
0.006	-1.0793\\
0.0026	-1.0907\\
0.0026	-1.2539\\
0.0056	-1.263\\
0.0093	-1.2716\\
0.0138	-1.2794\\
0.022	-1.292\\
0.027	-1.2991\\
0.0325	-1.3054\\
0.0383	-1.311\\
0.0445	-1.3157\\
0.0509	-1.3197\\
0.0574	-1.3229\\
}--cycle;

\addplot[area legend, draw=black, fill=white, forget plot]
table[row sep=crcr] {%
x	y\\
2	2\\
2.5	2\\
2.5	2.5\\
2	2.5\\
2	2\\
}--cycle;
\end{axis}
\end{tikzpicture}%
\end{minipage}
\begin{minipage}[t]{0.4\textwidth}
	\vspace{0pt}
	\centering
	\includetikz{./figures/tikz/contDynamics/example_reach}
\end{minipage}
\end{center}
