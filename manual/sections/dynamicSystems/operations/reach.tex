\subsubsection{reach} \label{sec:reach}

The operation \operator{reach} computes the reachable set of a dynamic system. Let us denote the solution of a dynamic system by $\chi(t;x_0,u(\cdot),p)$, where $t\in\mathbb{R}$ is the time, $x_0 = x(t_0) \in \mathbb{R}^n$ is the initial state, $u(\cdot) \in \mathbb{R}^m$ is the system input, and $p\in \mathbb{R}^p$ is a parameter vector. The reachable set at time $t=t_f$ can be defined for a set of initial states $\mathcal{X}_0 \subset \Rn$, a set of input values $\mathcal{U}(t) \subset \R^m$, and a set of parameter values $\mathcal{P} \subset \R^p$, as
\begin{equation}
  \mathcal{R}^e(t_f) = \Big\{ \chi(t_f;x_0,u(\cdot),p,w) \in \mathbb{R}^n ~\big|~ x_0 \in \mathcal{X}_0, \forall t \in [t_0,t_f]: u(t)\in\mathcal{U}(t), p \in \mathcal{P}, w \in \mathcal{W} \Big\}.
  \label{eq:reachSet}
\end{equation}
Since the exact reachable set $\mathcal{R}^e(t)$ as defined in \eqref{eq:reachSet} cannot be computed in general, the operation \texttt{reach} computes a tight enclosure $\mathcal{R}(t) \supseteq \mathcal{R}^e(t)$.

The syntax for the operation \texttt{reach} is:
\begin{equation*}
\begin{split}
	& \texttt{R} = \texttt{reach}(\texttt{sys},\texttt{params},\texttt{options}) \\,
	& [\texttt{R},\texttt{res}] = \texttt{reach}(\texttt{sys},\texttt{params},\texttt{options},\texttt{spec}),
\end{split}
\end{equation*}
with input arguments

\begin{center}
\renewcommand{\arraystretch}{1.3}
\begin{tabular}[t]{l p{13cm} }
	$\bullet$~\texttt{sys} & dynamic system defined by any of the classes in \cref{sec:continuousDynamics,sec:hybridDynamics}, e.g., \texttt{linearSys}, \texttt{hybridAutomaton}, etc. \\
	$\bullet$~\texttt{params} & struct containing the parameter that define the reachability problem \\
	& \begin{tabular}[t]{l p{10cm}}	
	 	--~\texttt{.tStart} & initial time $t_0$ (default value 0) \\
	 	--~\texttt{.tFinal} & final time $t_f$ \\
	 	--~\texttt{.R0} & initial set $\mathcal{X}_0$ specified by one of the set representations in \cref{sec:basicSetRep}\\
	 	--~\texttt{.U} & input set $\mathcal{U}$ specified as an object of class \texttt{zonotope} (see \cref{sec:zonotope})\\
	 	--~\texttt{.u} & time-dependent center $u_c(t)$ of the time-varying input set $\mathcal{U}(t) := u_c(t) \oplus \mathcal{U}$ specified as a matrix for which the number of columns is identical to the number of reachability steps (optional)\\
	 	--~\texttt{.paramInt} & set of parameter values $\mathcal{P}$ specified as an object of class \texttt{interval} (see \cref{sec:interval}) (class \texttt{nonlinParamSys} only)\\
	 	--~\texttt{.W} & disturbance set $\mathcal{W}$ specified as an object of class \texttt{interval} (see \cref{sec:interval}) or \texttt{zonotope} (see \cref{sec:zonotope} (classes \texttt{linearSys} and \texttt{linearSysDT} only) \\
	 	--~\texttt{.V} & set of sensor noises $\mathcal{V}$ specified as an object of class \texttt{interval} (see \cref{sec:interval})or \texttt{zonotope} (see \cref{sec:zonotope} (classes \texttt{linearSys} and \texttt{linearSysDT} only) \\
	 \end{tabular}
\end{tabular}
\end{center}
\begin{center}
\renewcommand{\arraystretch}{1.3}
\begin{tabular}[t]{l p{13cm} }
	& \begin{tabular}[t]{l p{10cm}}	
	    --~\texttt{.y0guess} & guess for a consistent initial algebraic state (class \texttt{nonlinDASys} only, see \cref{sec:reachDAEsys}). \\
	 	--~\texttt{.startLoc} & index of the initial location (class \texttt{hybridAutomaton} and \texttt{parallelHybridAutomaton} only)\\
	 	--~\texttt{.finalLoc} & index of the final location. Reachability analysis stops as soon as the final location is reached (class \texttt{hybridAutomaton} and \texttt{parallelHybridAutomaton} only, optional)
	 \end{tabular} \\
	$\bullet$~\texttt{options} & struct containing algorithm settings for reachability analysis. Since the settings are different for each type of dynamic system, they are documented in \cref{sec:continuousDynamics} and \cref{sec:hybridDynamics}. \\
	$\bullet$~\texttt{spec} & object of class \texttt{specification} (see \cref{sec:specification}) which represents the specifications the system has to verify. Reachability analysis stops as soon as a specification is violated.
\end{tabular}
\end{center}

and output arguments

\begin{center}
\renewcommand{\arraystretch}{1.3}
\begin{tabular}[t]{l p{13cm} }
	$\bullet$~\texttt{R} & object of class \texttt{reachSet} (see \cref{sec:reachSet}) that stores the reachable set $\mathcal{R}(t_i)$ at time point $t_i$ and the reachable set $\mathcal{R}(\tau_i)$ for time intervals $\tau_i = [t_i,t_{i+1}]$. \\
	$\bullet$~\texttt{res} & Boolean flag that indicates whether the specifications are satisfied (\texttt{res} = 1) or not (\texttt{res} = 0).
\end{tabular}
\end{center}


Let us demonstrate the operation \texttt{reach} by an example:

\begin{center}
\begin{minipage}[t]{0.58\textwidth}
	\footnotesize
	% This file was automatically created from the m-file 
% "m2tex.m" written by USL. 
% The fontencoding in this file is UTF-8. 
%  
% You will need to include the following two packages in 
% your LaTeX-Main-File. 
%  
% \usepackage{color} 
% \usepackage{fancyvrb} 
%  
% It is advised to use the following option for Inputenc 
% \usepackage[utf8]{inputenc} 
%  
  
% definition of matlab colors: 
\definecolor{mblue}{rgb}{0,0,1} 
\definecolor{mgreen}{rgb}{0.13333,0.5451,0.13333} 
\definecolor{mred}{rgb}{0.62745,0.12549,0.94118} 
\definecolor{mgrey}{rgb}{0.5,0.5,0.5} 
\definecolor{mdarkgrey}{rgb}{0.25,0.25,0.25} 
  
\DefineShortVerb[fontfamily=courier,fontseries=m]{\$} 
\DefineShortVerb[fontfamily=courier,fontseries=b]{\#} 
  
\noindent               
 $$\color{mgreen}$% system dynamics$\color{black}$$\\
 $sys = linearSys([-0.7 -2;2 -0.7],[1;1],[-2;-1]);$\\
 $$\\
 $$\color{mgreen}$% parameter$\color{black}$$\\
 $params.tFinal = 5;$\\
 $params.R0 = zonotope(interval([2;2],[2.5;2.5]));$\\
 $params.U = zonotope(interval(-0.1,0.1));$\\
 $$\\
 $$\color{mgreen}$% reachability settings$\color{black}$$\\
 $options.timeStep = 0.05;$\\
 $options.zonotopeOrder = 10;$\\
 $options.taylorTerms = 5;$\\
 $$\\
 $$\color{mgreen}$% reachability analysis$\color{black}$$\\
 $R = reach(sys,params,options);$\\ 
  
\UndefineShortVerb{\$} 
\UndefineShortVerb{\#}
\end{minipage}
\begin{minipage}[t]{0.4\textwidth}
	\vspace{0pt}
	\centering
	\includetikz{./figures/tikz/contDynamics/example_reach}
\end{minipage}
\end{center}
