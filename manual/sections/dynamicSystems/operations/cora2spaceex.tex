\subsubsection{cora2spaceex} \label{sec:cora2spaceex}

The operation \texttt{cora2spaceex} convertes a dynamical system represented as a CORA object to a SpaceEx model \cite{Donze2013}. The syntax is as follows:
\begin{equation*}
	\texttt{cora2spaceex}(\texttt{sys},\texttt{fileName}),
\end{equation*} 
with the input arguments
\begin{center}
\renewcommand{\arraystretch}{1.3}
\begin{tabular}[t]{l p{13cm} }
	$\bullet$~\texttt{sys} &  dynamic system represented as an object of class \texttt{linearSys} (see \cref{sec:linearSystems}), \texttt{nonlinearSys} (see \cref{sec:nonlinearSystems}), or \texttt{hybridAutomaton} (see \cref{sec:hybridAutomaton}). \\
	$\bullet$~\texttt{fileName} & name of the converted SpaceEx file.
\end{tabular}
\end{center}

Let us demonstrate the operation \texttt{cora2spaceex} by an example:

\begin{center}
\begin{minipage}[t]{0.45\textwidth}
	\footnotesize
	% This file was automatically created from the m-file 
% "m2tex.m" written by USL. 
% The fontencoding in this file is UTF-8. 
%  
% You will need to include the following two packages in 
% your LaTeX-Main-File. 
%  
% \usepackage{color} 
% \usepackage{fancyvrb} 
%  
% It is advised to use the following option for Inputenc 
% \usepackage[utf8]{inputenc} 
%  
  
% definition of matlab colors: 
\definecolor{mblue}{rgb}{0,0,1} 
\definecolor{mgreen}{rgb}{0.13333,0.5451,0.13333} 
\definecolor{mred}{rgb}{0.62745,0.12549,0.94118} 
\definecolor{mgrey}{rgb}{0.5,0.5,0.5} 
\definecolor{mdarkgrey}{rgb}{0.25,0.25,0.25} 
  
\DefineShortVerb[fontfamily=courier,fontseries=m]{\$} 
\DefineShortVerb[fontfamily=courier,fontseries=b]{\#} 
  
\noindent        
 $$\color{mgreen}$% nonlinear system$\color{black}$$\\
 $f = @(x,u) [x(2);$\color{mblue}$ ...$\color{black}$$\\
 $            (1-x(1)^2)*x(2)-x(1)];$\\
 $ $\\
 $sys = nonlinearSys(f);$\\
 $$\\
 $$\color{mgreen}$% convert to SpaceEx model$\color{black}$$\\
 $cora2spaceex(sys,$\color{mred}$'vanDerPol'$\color{black}$);$\\ 
  
\UndefineShortVerb{\$} 
\UndefineShortVerb{\#}
\end{minipage}
\begin{minipage}[t]{0.5\textwidth}
	\vspace{0pt}
	\centering
	\footnotesize
	\begin{verbatim}
		<?xml version="1.0" encoding="utf-8"?>
<sspaceex math="spaceex" version="2.0">
   <component id="model">
      <param name="x1" type="real"/>
      <param name="x2" type="real"/>
      <location id="1">
         <invariant/>
         <flow>
            x1' == x2 &amp; 
            x2' == - x1 - x2*(x1^2 - 1)
         </flow>
      </location>
   </component>
</sspaceex>
	\end{verbatim}
\end{minipage}
\end{center}
