\subsubsection{observe} \label{sec:observe}

The operation \operator{observe} performs guaranteed state estimation to obtain the set of possible states from inputs and outputs. Since measurements are typically obtained at discrete points in time, we only discuss the discrete-time case subsequently. To formalize the problem of set-based state estimation, we introduce the operator to receive the next state as $\chi(x_k,u_k,w_k)$. Our goal is to obtain the set of states $\mathcal{S}_{k}$ at time step $k$ enclosing the true state from a set of initial states $\mathcal{S}_0 \subset \mathbb{R}^n$, which we define inductively:
\begin{equation*}
\mathcal{S}_{k} = \Big\{x_k = \chi\Big(x_{k-1},u_{k-1},w_{k-1}\Big) \Big|  x_{k-1} \in \mathcal{S}_{k-1}, w_{k-1} \in \mathcal{W}, v_{k} \in \mathcal{V}, y_{k} = C x_k + v_{k} \Big\}.
\end{equation*}
A reachability problem is a special case, which does not require to check the consistency with the measurement to obtain the reachable set as
\begin{equation*}
\mathcal{R}_{k} = \Big\{x_k =\chi\Big(x_{k-1},u_{k-1},w_{k-1}\Big) \Big| x_{k-1} \in \mathcal{S}_{k-1}, w_{k-1} \in \mathcal{W} \Big\}.
\end{equation*}
We aim at computing an over-approximation of $\mathcal{S}_{k}$ that minimizes various cost functions as described in \cite{Althoff2021c, Althoff2021d}. This goal is pursued differently for the strip-based, set-propagation, and interval observers \cite{Althoff2021c, Althoff2021d}.

The syntax for the operation \texttt{observe} is:
\begin{equation*}
\begin{split}
	& \texttt{R} = \texttt{observe}(\texttt{sys},\texttt{params},\texttt{options}),
\end{split}
\end{equation*}
with input arguments

\begin{center}
\renewcommand{\arraystretch}{1.3}
\begin{tabular}[t]{l p{13cm} }
	$\bullet$~\texttt{sys} & dynamic system defined by one of the classes \texttt{linearSysDT} (see \cref{sec:linearSysDT}) or \texttt{nonlinearSysDT} (see \cref{sec:nonlinearSystemsDT}) \\
	$\bullet$~\texttt{params} & struct containing the parameter that define the observation problem \\
	& \begin{tabular}[t]{l p{10cm}}	
	 	--~\texttt{.tStart} & initial time $t_0$ (default value 0) \\
	 	--~\texttt{.tFinal} & final time $t_f$ \\
	 	--~\texttt{.R0} & initial set $\mathcal{X}_0$ specified by one of the set representations in \cref{sec:basicSetRep}\\
	 	--~\texttt{.W} & disturbance $\mathcal{W}$ specified as an object of class \texttt{zonotope} (see \cref{sec:zonotope}) or \texttt{ellipsoid} (see \cref{sec:ellipsoids}) \\
	 	--~\texttt{.V} & set of sensor noises $\mathcal{V}$ specified as an object of class \texttt{zonotope} (see \cref{sec:zonotope}) or \texttt{ellipsoid} (see \cref{sec:ellipsoids}) \\
	 	--~\texttt{.u} & time-dependent input $u(t)$ to the system, specified as a matrix for which the number of columns is identical to the number of measurements \\
	 	--~\texttt{.y} & time-dependent output $y(t)$ to the system, specified as a matrix for which the number of columns is identical to the number of measurements 
	 \end{tabular}
\end{tabular}
\end{center}
\begin{center}
\renewcommand{\arraystretch}{1.3}
\begin{tabular}[t]{l p{13cm} }
	$\bullet$~\texttt{options} & struct containing algorithm settings for set-based observation. Since the settings are different for each type of dynamic system, they are documented in \cref{sec:continuousDynamics} and \cref{sec:hybridDynamics}.
\end{tabular}
\end{center}

and output arguments

\begin{center}
\renewcommand{\arraystretch}{1.3}
\begin{tabular}[t]{l p{13cm} }
	$\bullet$~\texttt{R} & object of class \texttt{reachSet} (see \cref{sec:reachSet}) that stores the observed set $\mathcal{R}(t_i)$ for all time points.
\end{tabular}
\end{center}


Let us demonstrate the operation \texttt{observe} by an example:

\begin{center}
\begin{minipage}[t]{\textwidth}
	\footnotesize
	% This file was automatically created from the m-file 
% "m2tex.m" written by USL. 
% The fontencoding in this file is UTF-8. 
%  
% You will need to include the following two packages in 
% your LaTeX-Main-File. 
%  
% \usepackage{color} 
% \usepackage{fancyvrb} 
%  
% It is advised to use the following option for Inputenc 
% \usepackage[utf8]{inputenc} 
%  
  
% definition of matlab colors: 
\definecolor{mblue}{rgb}{0,0,1} 
\definecolor{mgreen}{rgb}{0.13333,0.5451,0.13333} 
\definecolor{mred}{rgb}{0.62745,0.12549,0.94118} 
\definecolor{mgrey}{rgb}{0.5,0.5,0.5} 
\definecolor{mdarkgrey}{rgb}{0.25,0.25,0.25} 
  
\DefineShortVerb[fontfamily=courier,fontseries=m]{\$} 
\DefineShortVerb[fontfamily=courier,fontseries=b]{\#} 
  
\noindent                                                                      
 $$\color{mgreen}#%% Parameters#\color{black}$$\\
 $params.tFinal = 20; $\color{mgreen}$%final time$\color{black}$$\\
 $params.R0 = zonotope(zeros(2,1),3*eye(2)); $\color{mgreen}$%initial set$\color{black}$$\\
 $params.V = 0.2*zonotope([0,1]); $\color{mgreen}$% sensor noise set$\color{black}$$\\
 $params.W = 0.02*[-6; 1]*zonotope([0,1]); $\color{mgreen}$% disturbance set$\color{black}$$\\
 $params.u = zeros(2,1); $\color{mgreen}$% input vector$\color{black}$$\\
 $params.y = [0.79, 5.00, 4.35, 1.86, -0.11, -1.13, -1.17, -0.76,$\color{mblue}$ ...$\color{black}$$\\
 $    -0.12, 0.72, 0.29, 0.19, 0.09, -0.21, 0.05, -0.00, -0.16, 0.01,$\color{mblue}$ ...$\color{black}$$\\
 $    -0.08, 0.13]; $\color{mgreen}$%measurement vector$\color{black}$$\\
 $$\\
 $$\\
 $$\color{mgreen}#%% Algorithmic Settings#\color{black}$$\\
 $options.zonotopeOrder = 20; $\color{mgreen}$% zonotope order$\color{black}$$\\
 $options.timeStep = 1; $\color{mgreen}$% step size$\color{black}$$\\
 $options.alg = $\color{mred}$'FRad-C'$\color{black}$; $\color{mgreen}$% observer approach$\color{black}$$\\
 $$\\
 $$\color{mgreen}#%% System Dynamics#\color{black}$$\\
 $reactor = linearSysDT($\color{mred}$'reactor'$\color{black}$,[0 -0.5; 1 1], 1, zeros(2,1), [-2 1], options.timeStep); $\\
 $$\\
 $$\color{mgreen}$% observe$\color{black}$$\\
 $EstSet = observe(reactor,params,options);$\\
  
\UndefineShortVerb{\$} 
\UndefineShortVerb{\#}

\end{minipage}
\begin{minipage}[t]{0.75\textwidth}
	\vspace{0pt}
	\centering
	\includetikz{./figures/tikz/contDynamics/example_observe}
\end{minipage}
\end{center}
