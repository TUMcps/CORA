\subsubsection{computeGO} \label{sec:computeGO}

The operation \operator{computeGO} computes the parameters of a general output (GO) model, which provides a linear approximation of the output of the system along a reference trajectory, using a first-order Taylor series expansion ~\cite{Luetzow2024b}. 
For the reference trajectory defined by the initial state $\bar{x}[1]$ and the inputs $\bar{u}[i]$, $i=1,...,k$, the output of the GO model for the state ${x}[1]$ and the inputs ${u}[i]$ is given as
\begin{align}
	y_{GO}[k] &= \bar{y}[k] + \bar{C}[k] (x[1]-\bar{x}[1]) + \sum_{i=1}^k \bar{D}_i[k] (u[i]-\bar{u}[i]) %+ \sum_{i=1}^k \bar{E}_i[k] l[i]
	, \label{eq:GOmodel_y}
\end{align} 
We can extend the GO model to also provide a linear approximation of the system state, which is given as:
\begin{align}	
	x_{GO}[k+1] &= \bar{x}[k+1] + \bar{A}[k] (x[1]-\bar{x}[1]) + \sum_{i=1}^k \bar{B}_i[k] (u[i]-\bar{u}[i]). %+ \sum_{i=1}^k \bar{F}_i[k] l[i]
	\label{eq:GOmodel_x} 
\end{align}

With the set $\mathcal{L}[i]=\mathcal{L}_x[i] \times \mathcal{L}_y[i]$, which contains higher-order Taylor terms and the Lagrange remainder for the Taylor series expansion at time step $i$, we can compute the reachable set of states $\mathcal{X}$ and the reachable set of outputs $\mathcal{Y}$ of the system for the uncertain initial state $x[1] \in \mathcal{X}[1]$ and uncertain inputs $u[i] \in \mathcal{U}[i]$ with the GO model as
\begin{align}
	\mathcal{X}[k+1] &= \bar{x}[k+1] \oplus \bar{A}[k] (\mathcal{X}[1]-\bar{x}[1]) \oplus \bigoplus_{i=1}^k \bar{B}_i[k] (\mathcal{U}[i]-\bar{u}[i]) \oplus \bigoplus_{i=1}^k \bar{F}_i[k] \mathcal{L}[i] \label{eq:GOmodel_xL} \\
	\mathcal{Y}[k] &= \bar{y}[k] \oplus \bar{C}[k] (\mathcal{X}[1]-\bar{x}[1]) \oplus \bigoplus_{i=1}^k \bar{D}_i[k] (\mathcal{U}[i]-\bar{u}[i]) \oplus \bigoplus_{i=1}^k \bar{E}_i[k] \mathcal{L}[i]. \label{eq:GOmodel_yL}
\end{align} 

The syntax for the operation \texttt{computeGO} is:
\begin{equation*}
\begin{split}
    & \texttt{p\_GO} = \texttt{computeGO}(\texttt{sys},\texttt{x0\_ref},\texttt{u\_ref},\texttt{n\_k})
\end{split}
\end{equation*}
with input arguments

\begin{center}
\renewcommand{\arraystretch}{1.3}
\begin{tabular}[t]{l p{13cm} }
	$\bullet$~\texttt{sys} & dynamic system defined by one of the classes \texttt{linearSysDT} (see \cref{sec:linearSysDT}), \texttt{linearARX} (see \cref{sec:linearARX}), \texttt{nonlinearSysDT} (see \cref{sec:nonlinearSystemsDT}), or \texttt{nonlinearARX} (see \cref{sec:nonlinearSystems_ARX}). \\
	$\bullet$~\texttt{x0\_ref} & initial state of the reference trajectory, equivalent to $\bar{x}[1]$ in~\eqref{eq:GOmodel_x} and \eqref{eq:GOmodel_y}. \\
	$\bullet$~\texttt{u\_ref} & inputs for the reference trajectory, equivalent to $\bar{u}[\cdot]$ in~\eqref{eq:GOmodel_x} and \eqref{eq:GOmodel_y}. \\
	$\bullet$~\texttt{n\_k} & time horizon. 
\end{tabular}
\end{center}

and output arguments

\begin{center}
\renewcommand{\arraystretch}{1.3}
\begin{tabular}[t]{l p{13cm} }
	$\bullet$~\texttt{p\_GO} & parameters of the GO model with $k=1,...,n\_k$ and $i=1,...,k$. \\
	& \begin{tabular}[t]{l p{10cm}}
		--~\texttt{.A\{k\}} & matrix that describes the influence of the initial state $x[1]$ on the state $x[k+1]$, equivalent to $\bar{A}[k]$ in \eqref{eq:GOmodel_x} and \eqref{eq:GOmodel_xL}.\\
		--~\texttt{.B\{k,i\}} & matrix that describes the influence of the input $u[i]$ on the state $x[k+1]$, equivalent to $\bar{B}_i[k]$ in \eqref{eq:GOmodel_x} and \eqref{eq:GOmodel_xL}.\\
		--~\texttt{.F\{k,i\}} & matrix that describes the influence of the linearization error $l[i]$ on the state $x[k+1]$, equivalent to $\bar{F}_i[k]$ in \eqref{eq:GOmodel_xL}.\\
		--~\texttt{.C\{k\}} & matrix that describes the influence of the initial state $x[1]$ on the output $y[k]$, equivalent to $\bar{C}[k]$ in \eqref{eq:GOmodel_y} and \eqref{eq:GOmodel_yL}.\\
		--~\texttt{.D\{k,i\}} & matrix that describes the influence of the input $u[i]$ on the output $y[k]$, equivalent to $\bar{D}_i[k]$ in \eqref{eq:GOmodel_y} and \eqref{eq:GOmodel_yL}.\\
		--~\texttt{.E\{k,i\}} & matrix that describes the influence of the linearization error $l[i]$ on the output $y[k]$, equivalent to $\bar{E}_i[k]$ in \eqref{eq:GOmodel_yL}.\\
		--~\texttt{.x(:,k)} & reference state, equivalent to $\bar{x}[k]$. \\
		--~\texttt{.u(:,k)} & reference input, equivalent to $\bar{u}[k]$.\\
		--~\texttt{.y(:,k)} & reference output, equivalent to $\bar{y}[k]$ .
	\end{tabular}
\end{tabular}
\end{center}


Let us demonstrate the operation \texttt{computeGO} by an example:


\begin{center}
	\begin{minipage}[t]{\textwidth}
		\footnotesize
		% This file was automatically created from the m-file 
% "m2tex.m" written by USL. 
% The fontencoding in this file is UTF-8. 
%  
% You will need to include the following two packages in 
% your LaTeX-Main-File. 
%  
% \usepackage{color} 
% \usepackage{fancyvrb} 
%  
% It is advised to use the following option for Inputenc 
% \usepackage[utf8]{inputenc} 
%  
  
% definition of matlab colors: 
\definecolor{mblue}{rgb}{0,0,1} 
\definecolor{mgreen}{rgb}{0.13333,0.5451,0.13333} 
\definecolor{mred}{rgb}{0.62745,0.12549,0.94118} 
\definecolor{mgrey}{rgb}{0.5,0.5,0.5} 
\definecolor{mdarkgrey}{rgb}{0.25,0.25,0.25} 
  
\DefineShortVerb[fontfamily=courier,fontseries=m]{\$} 
\DefineShortVerb[fontfamily=courier,fontseries=b]{\#} 
  
\noindent                                                                      
$$\color{mgreen}#%% Reference trajectory#\color{black}$$\\
$x0_ref = [0; 0]; $\color{mgreen}$%initial state of reference trajectory$\color{black}$$\\
$u_ref = zeros(2,10); $\color{mgreen}$%inputs for reference trajectory$\color{black}$$\\
$n_k = 5; $\color{mgreen}$%time horizon$\color{black}$$\\
$$\\
$$\color{mgreen}#%% System Dynamics#\color{black}$$\\
$reactor = linearSysDT($\color{mred}$'reactor'$\color{black}$,[0 -0.5; 1 1],  eye(2), zeros(2,1), [-2 1], 0.1); $\\
$$\\
$$\color{mgreen}$% conformance synthesis$\color{black}$$\\
$p_G0 = computeGO(reactor,x0_ref,u_ref,n_k);$\\
  
\UndefineShortVerb{\$} 
\UndefineShortVerb{\#}

	\end{minipage}
\end{center}
