\subsubsection{verify} \label{sec:verify}

The operation \operator{verify} automatically verifies given specifications for the dynamic system and parameters.
Currently, this operation is primarily supported for dynamic systems of class \texttt{linearSys}.

The syntax for the operation \texttt{verify} is:
\begin{equation*}
\begin{split}
	& \texttt{res} = \texttt{verify}(\texttt{sys},\texttt{params},\texttt{options},\texttt{spec}), \\
\end{split}
\end{equation*}
with input arguments

\begin{center}
\renewcommand{\arraystretch}{1.3}
\begin{tabular}[t]{l p{13cm} }
	$\bullet$~\texttt{sys} & dynamic system defined by any of the classes in \cref{sec:continuousDynamics,sec:hybridDynamics}, e.g., \texttt{linearSys}, \texttt{hybridAutomaton}, etc. \\
	$\bullet$~\texttt{params} & struct containing the parameter that define the reachability problem \\
	$\bullet$~\texttt{options} & struct containing algorithm settings for the verification. \\
	$\bullet$~\texttt{spec} & object of class \texttt{specification} (see \cref{sec:specification}) or class \texttt{stl} (see \cref{sec:temporalLogic}), which represents the specifications the system has to fulfill.
\end{tabular}
\end{center}

and output arguments

\begin{center}
\renewcommand{\arraystretch}{1.3}
\begin{tabular}[t]{l p{13cm} }
	$\bullet$~\texttt{res} & Boolean flag that indicates whether the specifications are satisfied (\texttt{res} = 1) or not (\texttt{res} = 0).
\end{tabular}
\end{center}

Let us demonstrate the operation \texttt{verify} by an example:

\begin{center}
\begin{minipage}[t]{0.6\textwidth}
	\footnotesize
	% This file was automatically created from the m-file 
% "m2tex.m" written by USL. 
% The fontencoding in this file is UTF-8. 
%  
% You will need to include the following two packages in 
% your LaTeX-Main-File. 
%  
% \usepackage{color} 
% \usepackage{fancyvrb} 
%  
% It is advised to use the following option for Inputenc 
% \usepackage[utf8]{inputenc} 
%  
  
% definition of matlab colors: 
\definecolor{mblue}{rgb}{0,0,1} 
\definecolor{mgreen}{rgb}{0.13333,0.5451,0.13333} 
\definecolor{mred}{rgb}{0.62745,0.12549,0.94118} 
\definecolor{mgrey}{rgb}{0.5,0.5,0.5} 
\definecolor{mdarkgrey}{rgb}{0.25,0.25,0.25} 
  
\DefineShortVerb[fontfamily=courier,fontseries=m]{\$} 
\DefineShortVerb[fontfamily=courier,fontseries=b]{\#} 
  
\noindent                 
 $$\color{mgreen}$% specification$\color{black}$$\\
 $x = stl($\color{mred}$'x'$\color{black}$,2);$\\
 $spec = until(x(2) > 0.4,x(1) < 0,interval(0,2));$\\
 $$\\
 $$\color{mgreen}$% dynamic system$\color{black}$$\\
 $sys = linearSys([0 -1; 1 0],[0; 0]);$\\
 $$\\
 $$\color{mgreen}$% reachability parameters$\color{black}$$\\
 $params.R0 = zonotope(interval([0.5;0.5],[1;1]));$\\
 $$\\
 $$\color{mgreen}$% algorithm settings$\color{black}$$\\
 $options.verifyAlg = $\color{mred}$'stl:seidl'$\color{black}$;$\\
 $options.taylorTerms = 10;$\\
 $options.zonotopeOrder = 10;$\\
 $$\\
 $$\color{mgreen}$% verify$\color{black}$$\\
 $verify(sys,params,options,spec)$\\
  
\UndefineShortVerb{\$} 
\UndefineShortVerb{\#}
\end{minipage}
\end{center}
