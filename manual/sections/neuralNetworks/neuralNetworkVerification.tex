\subsection{Formal Verification of Neural Networks}
\label{sec:neuralNetworkVerification}

CORA enables the formal verification of neural networks.
This refers to the problem of given a neural networks $\NN\colon\R^{\numNeurons_0}\to\R^{\numNeurons_\numLayers}$,
an input set $\nnInputSet\subset\R^{\numNeurons_0}$,
and some unsafe set $\nnUnsafeSet\subset\R^{\numNeurons_\numLayers}$ in the output space of the neural network,
we want to verify that it holds
\begin{equation}
    \NN(\nnInputSet) \cap \nnUnsafeSet = \emptyset.
\end{equation}
As it is in general not feasible to compute the exact output set $\nnOutputSetExact=\NN(\nnInputSet)$~\cite{Katz2017},
CORA usually computes an enclosure of the output set $\nnOutputSet\supseteq\nnOutputSetExact$.
This is done using reachability analysis.

CORA primarily implements the approaches described in~\cite{Kochdumper2023b,ladner2023a} for the reachability analysis on neural networks.
The class \texttt{neuralNetwork} provides the function \texttt{evaluate}, that can be used to compute the output of a neural network for a single point or a set of possible inputs:
\begin{center}
    \texttt{res = nn.evaluate(x,options,idxLayer)}, \\
    \texttt{res = nn.evaluate(X,options,idxLayer)},
\end{center}
where $\texttt{x}\in\R^{\numNeurons_0\times N}$ is an array of $N$ input points and $\texttt{X}\subset\R^{\numNeurons_0}$ is an input set.
Currently, CORA supports zonotopes, polynomial zonotopes, interval, Taylor models, and constraint zonotopes as input sets to the neural network.
A few settings are available to control the set propagation through the neural network.
These are stored in the algorithm options struct for neural networks \texttt{options.nn}.
We list the most important settings below, further information can be found in the documentation of the \texttt{evaluate} function:

\begin{center}
    \renewcommand{\arraystretch}{1.3}
    \begin{longtable}[t]{l p{11cm}}
        --~\texttt{.poly\_method} & string describing the approximation method of nonlinear functions: \texttt{'regression'}, \texttt{'singh'}, and \texttt{'taylor'}. \\
        --~\texttt{.num\_generators} & maximal number of generators of the polynomial zonotope. If this number is exceeded, an order reduction method is executed.
    \end{longtable}
\end{center}
The following code shows an exemplary set propagation through the previously defined neural network:

\begin{center}
    \begin{minipage}[t]{0.9\textwidth}
    {\footnotesize  % This file was automatically created from the m-file 
% "m2tex.m" written by USL. 
% The fontencoding in this file is UTF-8. 
%  
% You will need to include the following two packages in 
% your LaTeX-Main-File. 
%  
% \usepackage{color} 
% \usepackage{fancyvrb} 
%  
% It is advised to use the following option for Inputenc 
% \usepackage[utf8]{inputenc} 
%  
  
% definition of matlab colors: 
\definecolor{mblue}{rgb}{0,0,1} 
\definecolor{mgreen}{rgb}{0.13333,0.5451,0.13333} 
\definecolor{mred}{rgb}{0.62745,0.12549,0.94118} 
\definecolor{mgrey}{rgb}{0.5,0.5,0.5} 
\definecolor{mdarkgrey}{rgb}{0.25,0.25,0.25} 
  
\DefineShortVerb[fontfamily=courier,fontseries=m]{\$} 
\DefineShortVerb[fontfamily=courier,fontseries=b]{\#} 
  
\noindent        
 $$\color{mgreen}$% input set$\color{black}$$\\
 $c = [4;4];$\\
 $G = [2 1 2; 0 2 2];$\\
 $E = [1 0 3;0 1 1];$\\
 $GI = [];$\\
 $X = polyZonotope(c,G,GI,E);$\\
 $$\\
 $$\color{mgreen}$% settings$\color{black}$$\\
 $options = struct;$\\
 $options.nn.poly_method = $\color{mred}$'regression'$\color{black}$;$\\
 $options.nn.num_generators = 1000;$\\
 $$\\
 $$\color{mgreen}$% evaluation$\color{black}$$\\
 $Y = nn.evaluate(X, options); $ \\
  
\UndefineShortVerb{\$} 
\UndefineShortVerb{\#}
}
    \end{minipage}
\end{center}

CORA can also automatically verify given specifications using the \texttt{verify} function:

\begin{center}
    \texttt{[res,x\_,y\_] = nn.verify(x,r,A,b,varargin)},
\end{center}

where \texttt{x, r} is the center of the input and the perturbation radius $\nnPertRadius$ of the input set $\nnInputSet$,
and \texttt{A, b} describe the specification $A\tilde{y}+b\leq0$ with $\tilde{y}=\NN(\tilde{x}),\,\tilde{x}\in\nnInputSet$.
The output variable \texttt{res} is \texttt{true} if the specification is fulfilled $\forall \tilde{x}\in\nnInputSet$,
\texttt{false} if a counterexample was found, and is empty otherwise.
In case a counterexample was found, \texttt{x\_, y\_} contains the respective point of the input set and output space, respectively.

Further examples can be found at \texttt{./cora/examples/nn}.
