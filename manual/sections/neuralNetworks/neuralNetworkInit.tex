\subsection{Neural Networks in CORA}
\label{sec:neuralNetworkInit}

Neural networks are represented in CORA as objects of the class \texttt{neuralNetwork}.
They can be initialized directly in CORA or imported from a file using various formats, e.g., ONNX.
Currently, CORA mainly supports feed-forward neural networks (see \cref{fig:neuralNetwork}) with various activation functions.

\begin{figure}[htb]
  \centering	
    \footnotesize
  	\includetikz{./figures/tikz/neuralNetworks/neural_network}
    \caption{Feed-forward neural network with $\numLayers = 2$ layers.}
    \label{fig:neuralNetwork}
\end{figure}

An object of class \texttt{neuralNetwork} can be constructed as follows:
\begin{center}
	\texttt{nn = neuralNetwork(layers)},
\end{center}
where \texttt{layers} is a cell array of length $\numLayers$ filled with objects of type \texttt{nnLayer}.
For a network~$\NN\colon\R^{\numNeurons_0}\to\R^{\numNeurons_{\numLayers}}$ with input~$\nnInput\in\R^{\numNeurons_0}$ and output~$\nnOutput\in\R^{\numNeurons_\numLayers}$,
each layer $k\in[\numLayers]$ computes $\nnHidden_k = \nnLayer{k}{}{\nnHidden_k-1}\in\R^{\numNeurons_k}$,
where $\nnHidden_0 = \nnInput$ and $\nnOutput = \nnHidden_\numLayers$ and thus obtaining $\nnOutput=\NN(\nnInput)$.
Currently, the following layers are supported in CORA:

\begin{center}
	\renewcommand{\arraystretch}{1.3}
	\begin{tabular}[t]{l p{11cm} }
		$\bullet$~\textbf{\texttt{nnLinearLayer(W, b)}} & Linear layer computing $\nnLayer{k}{LIN}{\nnHidden_{k-1}}=W_k\nnHidden_{k-1} + b_k$ with weight matrix $W \in \R^{\numNeurons_k \times \numNeurons_{k-1}}$ and bias $b\in\R^{\numNeurons_k}$ (\cref{fig:neuralNetwork}).\\
		$\bullet$~\textbf{\texttt{nnActivationLayer()}} & Abstract activation layer computing $\nnLayer{k}{ACT}{\nnHidden_{k-1}}=\nnActFun(\nnHidden_{k-1})$, where $\nnActFun$ is the corresponding element-wise activation function. \\
		$\bullet$~\textbf{\texttt{nnLeakyReLULayer()}} & \texttt{nnActivationLayer()} with $\nnActFun(\nnHidden_{k-1}) = \max(\alpha \nnHidden_{k-1}, \nnHidden_{k-1}),\ \alpha\in\R$ with default $\alpha=0.01$. \\
		$\bullet$~\textbf{\texttt{nnReLULayer()}} & \texttt{nnLeakyReLULayer()} with $\alpha=0$, thus $\nnActFun(\nnHidden_{k-1}) = \max(0, \nnHidden_{k-1})$. \\
		$\bullet$~\textbf{\texttt{nnSigmoidLayer()}} & \texttt{nnActivationLayer()} with $\nnActFun(\nnHidden_{k-1}) = \text{sigmoid}(\nnHidden_{k-1})$. \\
		$\bullet$~\textbf{\texttt{nnTanhLayer()}} & \texttt{nnActivationLayer()} with $\nnActFun(\nnHidden_{k-1}) = \tanh(\nnHidden_{k-1})$. \\
	\end{tabular}
\end{center}

Let us demonstrate the \texttt{neuralNetwork} class by the following example, which constructs a neural network with the same structure as the one shown in \cref{fig:neuralNetwork}:

\begin{center}
\begin{minipage}[t]{0.9\textwidth}
{\footnotesize  % This file was automatically created from the m-file 
% "m2tex.m" written by USL. 
% The fontencoding in this file is UTF-8. 
%  
% You will need to include the following two packages in 
% your LaTeX-Main-File. 
%  
% \usepackage{color} 
% \usepackage{fancyvrb} 
%  
% It is advised to use the following option for Inputenc 
% \usepackage[utf8]{inputenc} 
%  
  
% definition of matlab colors: 
\definecolor{mblue}{rgb}{0,0,1} 
\definecolor{mgreen}{rgb}{0.13333,0.5451,0.13333} 
\definecolor{mred}{rgb}{0.62745,0.12549,0.94118} 
\definecolor{mgrey}{rgb}{0.5,0.5,0.5} 
\definecolor{mdarkgrey}{rgb}{0.25,0.25,0.25} 
  
\DefineShortVerb[fontfamily=courier,fontseries=m]{\$} 
\DefineShortVerb[fontfamily=courier,fontseries=b]{\#} 
  
\noindent        
 $$\color{mgreen}$% init weight and bias$\color{black}$$\\
 $W1 = rand(3,2); b1 = rand(3,1);$\\
 $W2 = rand(2,3); b2 = rand(2,1);$\\
 $$\\
 $$\color{mgreen}$% neural network$\color{black}$$\\
 $nn = neuralNetwork({ $ \\
 $    nnLinearLayer(W1, b1);$ \\
 $    nnReLULayer();$ \\
 $    nnLinearLayer(W2, b2);$ \\
 $    nnReLULayer();$ \\
 $})$ \\
  
\UndefineShortVerb{\$} 
\UndefineShortVerb{\#}
}
\end{minipage}
\end{center}

While the example initializes a random neural network, one can also import neural networks into CORA from various common neural network formats.
CORA currently supports ONNX, NNet, YML, Sherlock, and the conversion for neural networks coming from the Deep Learning Toolbox.
Please note that additional toolboxes are required for the ONNX import (see \cref{sec:installation}).

Please visit \texttt{./cora/nn} for further details.