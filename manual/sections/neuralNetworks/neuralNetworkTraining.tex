\subsection{Training Verifiably Robust Neural Networks}
\label{sec:neuralNetworkTraining}

CORA also enables the training of verifiably robust neural networks both in the supervised~\cite{koller2024a}
and in the reinforcement learning~\cite{Wendl2024a} setting.
This is realized by integrating reachability analysis into the training process.
In particular, we propagate zonotopes through a neural network and compute a set-based gradient based on the computed output set~$\nnOutputSet$, target~$\nnTarget$, and loss function~$E$.
This is done efficiently in CORA by propagating the sets batch-wise through the network and every computation can be done on a GPU.

The set-based loss function is defined as
\begin{equation}
    E_\text{set}(\nnTarget,\nnOutputSet) = (1-\tau) E_\text{point}(\nnTarget,\nnOutputSet) + \tau E_\text{volume}(\nnOutputSet),
\end{equation}
where $E_\text{point}$ refers to the standard (point-based) loss, e.g., mean squared error or cross-entropy loss,
and $E_\text{volume}$ to a volume loss to reduce the size of the output set.
The parameter $\tau\in[0,1]$ is a hyperparameter to weight the two terms.
The process of this set-based training is schematically shown in \cref{fig:set-training}.
The resulting networks are verifiably more robust against input perturbations as these were considered during the training process.

\begin{figure}[htb]
    \centering
    \footnotesize
    \includetikz{./figures/tikz/neuralNetworks/set-training-new}
    \caption{Set-based training of a neural network to make it verifiably robust against input perturbations.}
    \label{fig:set-training}
\end{figure}

In the supervised setting, this is done via the function \texttt{train}:
\begin{center}
    \texttt{[loss,trainTime] = nn.train(trainX, trainT, valX, valT, options, verbose)},
\end{center}
where \texttt{trainX, trainT, valX, valT} are the training and validation dataset, respectively,
algorithm options for training are stored in \texttt{options.nn.train},
and \texttt{verbose} is a logical flag for verbose outputs to the command window during training.
The training options specify all hyperparameters for training,
e.g., the number of epochs, the optimizer, the batch size, and the loss.
See the code below for an example and further details can be found in the documentation of the \texttt{train} function.

\begin{center}
    \begin{minipage}[t]{0.9\textwidth}
    {\footnotesize  % This file was automatically created from the m-file 
% "m2tex.m" written by USL. 
% The fontencoding in this file is UTF-8. 
%  
% You will need to include the following two packages in 
% your LaTeX-Main-File. 
%  
% \usepackage{color} 
% \usepackage{fancyvrb} 
%  
% It is advised to use the following option for Inputenc 
% \usepackage[utf8]{inputenc} 
%  
  
% definition of matlab colors: 
\definecolor{mblue}{rgb}{0,0,1} 
\definecolor{mgreen}{rgb}{0.13333,0.5451,0.13333} 
\definecolor{mred}{rgb}{0.62745,0.12549,0.94118} 
\definecolor{mgrey}{rgb}{0.5,0.5,0.5} 
\definecolor{mdarkgrey}{rgb}{0.25,0.25,0.25} 
  
\DefineShortVerb[fontfamily=courier,fontseries=m]{\$} 
\DefineShortVerb[fontfamily=courier,fontseries=b]{\#} 
  
\noindent                                                                                                                   

 $$\\
 $$\color{mgreen}$% Initialize neural network $\color{black}$$\\
 $nn = neuralNetwork(layers);$\\
 $nn.initWeights();$\\
 $$\\
 $$\color{mgreen}$% Train neural network $\color{black}$$\\
 $options.nn = struct(...$\\
 $    $\color{mred}$'use_approx_error'$\color{black}$,true,...$\\
 $    $\color{mred}$'poly_method'$\color{black}$,$\color{mred}$'bounds'$\color{black}$,...$\\
 $    $\color{mred}$'train'$\color{black}$,struct($\color{mblue}$ ...$\color{black}$$\\
 $        $\color{mred}$'optim'$\color{black}$,nnSGDOptimizer(lr,beta),...$\\
 $        $\color{mred}$'max_epoch'$\color{black}$,maxEpoch,...$\\
 $        $\color{mred}$'mini_batch_size'$\color{black}$,miniBatchSize,...$\\
 $        $\color{mred}$'loss'$\color{black}$,$\color{mred}$'mse'$\color{black}$,...$\\
 $        $\color{mred}$'method'$\color{black}$,$\color{mred}$'set'$\color{black}$,...$\\
 $        $\color{mred}$'noise'$\color{black}$,0.5,...$\\
 $        $\color{mred}$'input_space_inf'$\color{black}$,xl,...$\\
 $        $\color{mred}$'input_space_sup'$\color{black}$,xu,...$\\
 $        $\color{mred}$'tau'$\color{black}$,0.5,...$\\
 $        $\color{mred}$'volume_heuristic'$\color{black}$,$\color{mred}$'f-radius'$\color{black}$,...$\\
 $        $\color{mred}$'zonotope_weight_update'$\color{black}$,$\color{mred}$'sum'$\color{black}$,...$\\
 $        $\color{mred}$'num_approx_err'$\color{black}$,inf,...$\\
 $        $\color{mred}$'init_gens'$\color{black}$,$\color{mred}$'l_inf'$\color{black}$,...$\\
 $        $\color{mred}$'num_init_gens'$\color{black}$,inf...$\\
 $    )...$\\
 $);$\\
 $loss = nn.train(trainX,trainY,valX,valY,options,true);$\\
 $$\\

\UndefineShortVerb{\$} 
\UndefineShortVerb{\#}}
    \end{minipage}
\end{center}

Please visit \texttt{./cora/nn} for details and \texttt{./cora/examples/nn} for examples.

CORA also enables the training of verifiably robust agents in an actor-critic settings,
which is depicted in \cref{fig:rl-DDPG}:
After defining a control environment specifying the dynamic of the model,
and actor $\mu_\phi$, critic $Q_\theta$ networks with parameters $\phi,\,\theta$, respectively,
one can also use set-based training in reinforcement learning using CORA.

\begin{figure}[htb]
    \centering
    \footnotesize
    \includetikz{./figures/tikz/neuralNetworks/rl-DDPG}
    \caption{Illustration of the structure of the deep deterministic policy gradient algorithm: \ding{192} and \ding{193} show the components that are augmented through set-based training.}
    \label{fig:rl-DDPG}
\end{figure}

The respective classes in CORA are \texttt{ctrlEnvironment}, \texttt{actor}, and \texttt{critic},
where both, agents based on the DDPG (\texttt{DDPGagent}) and TD3 (\texttt{TD3agent}) algorithm can be trained.
The options for the agent and critic are as described above for the supervised settings,
which are stored in \texttt{options.rl.actor.nn.train} and \texttt{options.rl.critic.nn.train}, respectively.
Additional options are available for the environment, such as initial set and time steps, which are stored in \texttt{options.rl.env}.
Forther details can be found in the documentation of the respective classes.
Please note that either the actor and critic can be trained set-based (\ding{192} in \cref{fig:rl-DDPG}), or only the actor (\ding{193} in \cref{fig:rl-DDPG}).

Please visit \texttt{./cora/nn/rl} for details and \texttt{./cora/examples/nn} for examples.
