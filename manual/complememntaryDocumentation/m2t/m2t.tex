\documentclass{article}

% packages
\usepackage[margin=1in]{geometry}

% math packages not required
%\usepackage{etoolbox}
%\usepackage{mathtools}
\usepackage{amsmath}
%\usepackage{amsfonts}

% MATLAB code
\usepackage{fancyvrb}

% links
\usepackage[hidelinks]{hyperref}

% enumerate and itemize using costum spacing
\usepackage{enumitem}

\usepackage{setspace}
\setstretch{1.44}%  default

% TikZ / PGF Plots
\usepackage{tikz}
\usepackage{pgfplots}

% libraries
\usetikzlibrary{matrix}
\usepgfplotslibrary{groupplots}
\usetikzlibrary{pgfplots.groupplots}
\usetikzlibrary{calc}

\usepackage{siunitx}

\usepackage{xcolor}
\usepackage{listings}
\lstset
{
	language=[LaTeX]TeX,
	breaklines=true,
	basicstyle=\tt,
	keywordstyle=\color{blue},
	identifierstyle=\color{black},
}

% macros
\definecolor{darkblue}{RGB}{16,52,166}
\newcommand{\linkto}[2]{\href{#1}{\textcolor{darkblue}{#2}}}

\usepackage{cleveref}


\begin{document}
\begin{center}
\huge
\textbf{Introduction to matlab2tikz} \\
\smallskip \small
Mark Wetzlinger, \linkto{mailto:m.wetzlinger@tum.de}{m.wetzlinger@tum.de}, last update: \today
\end{center}

\bigskip

Exporting figures is a common task when working with toolboxes such as CORA and AROC.
Preferably, one wants to use vector graphics for LaTeX documents to achieve high quality.
To this end, the \emph{matlab2tikz} toolbox has been written to enable the export of MATLAB figures to compilable \emph{TikZ/pgfplots} files.
TikZ/pgfplots are LaTeX packages enabling the creation of figures via code commands.
This introduction shows how to export figures created in CORA/AROC for the usage in LaTeX documents.

Acknowledgments: Many thanks to Victor Ga{\ss}mann for his exploratory work and helpful suggestions.

\section{Setup}
\label{sec:setup}

For the setup in MATLAB, the following toolboxes are required on your MATLAB path:
%
\begin{itemize}[itemsep=0pt]
	\item \linkto{https://tumcps.github.io/CORA/}{CORA}, and required toolboxes (see CORA manual, Section~1.3 \emph{Installation})
	\item \linkto{https://tumcps.github.io/AROC/}{AROC} (optional)
	\item \linkto{https://github.com/matlab2tikz/matlab2tikz}{matlab2tikz}
\end{itemize}
%
For the setup in LaTeX, the following packages and libraries are required:
%
\begin{itemize}[itemsep=0pt]
	\item \verb|\usepackage{tikz}|
	\item \verb|\usepackage{pgfplots}|
	\item \verb|\usepackage{calc}| (not required, but strongly recommended)
	\item If group plots are used (see \cref{sec:groupplots}): \\
			\verb|\usepgfplotslibrary{groupplots}| \\
			\verb|\usetikzlibrary{pgfplots.groupplots}| \\
			\verb|\usetikzlibrary{matrix}|
\end{itemize}
%
For further reading, consider the extensive documentation (for \linkto{https://ctan.ebinger.cc/tex-archive/graphics/pgf/base/doc/pgfmanual.pdf}{TikZ} and for \linkto{https://ctan.space-pro.be/tex-archive/graphics/pgf/contrib/pgfplots/doc/pgfplots.pdf}{pgfplots}) or Google.


\section{Main steps}
\label{sec:mainsteps}

The general workflow is comprised of the following main steps:
%
\begin{enumerate}[itemsep=0pt]
	\item Generation of the figure(s) in MATLAB.
	\item Call of the function \verb|fig2tikz|.
	\item Generation of the TikZ/pgfplots file using \verb|matlab2tikz|.
	\item Post-processing of the generated file and integration in LaTeX.
\end{enumerate}

% Step 1
\subsection{Figure Generation}
\label{ssec:figuregeneration}

Figures generated in CORA/AROC are usually based on either individual sets (\verb|contSet| objects) or the classes \verb|reachSet| or \verb|simResult|.
The corresponding plot functions overload the MATLAB built-in functions with some additional input arguments, such as the projected dimension of the set or the desired coloring---this is documented in the CORA manual or the respective function headers in CORA.
For \verb|reachSet| objects, the name-value pair \verb|'Unify',true|false| (default: \verb|false|) has been implemented which unifies all projected sets to a single set in order to keep the resulting file size of the exported figure small.
Note that much of the figure formatting can already be made in MATLAB:
%
\begin{itemize}[itemsep=0pt]
	\item The font of the axis labels; using MATLAB's name-value pair \verb|'interpreter','latex'|, mathtext (\$ ... \$) can be directly copied into the corresponding TikZ/pgfplots file.
	\item Legend entries as specied in MATLAB using the \verb|legend| command, including options such as the name-value pairs \verb|'Location',...| and \verb|'Orientation',...|, among others.
\end{itemize}
%
The following graphics objects (ordered alphabetically) are supported by \verb|matlab2tikz|:
\verb|'area'|, \verb|'bar'|, \verb|'contour'|, \verb|'errorbar'|, \verb|'histogram'|, \verb|'hggroup'|, \verb|'hgtransform'|, \verb|'image'|, \verb|'light'|, \verb|'line'|, \linebreak \verb|'matlab.graphics.primitive.Group'|, \verb|'patch'|, \verb|'quiver'|, \verb|'rectangle'|, \verb|'scatter'|, \verb|'stair'|, \verb|'stem'|, \verb|'surface'|, \verb|'text'|, \verb|''|.
The plotted graphics objects can be listed using the command
%
\begin{verbatim}
  fig.CurrentAxes.Children;
\end{verbatim}
%
where \verb|fig| is a \verb|1x1 Figure| object;

% Step 2
\subsection{Function \texttt{fig2tikz}}
\label{ssec:fig2tikz}

The aforementioned name-value pair \verb|'Unify',true| constructs a \verb|polygon| object which is a wrapper class for the MATLAB built-in class \verb|polyshape| extended by various convenient operations, e.g. \verb|plot|, \verb|and| (intersection), \verb|in| (containment check), \verb|plus| (Minkowski sum).
As this is a costum class, \verb|matlab2tikz| cannot convert it.
Thus, the function \verb|fig2tikz| has been implemented in order to convert the plotted \verb|polygon| object into a line or patch which can be processed.
The function reads as follows:
%
\begin{verbatim}
  fig = fig2tikz(fig,precision);
\end{verbatim}
%
Both input arguments are optional.
If provided, the input argument \verb|fig| has to be a \verb|1x1 Figure| object (default: \verb|gcf|, i.e., the currently selected figure), whereas \verb|precision| has to be a value between 0 and 1 (default: \verb|1e-5|).
Furthermore, the command \verb|cleanfigure| is executed as well, which iterates of all graphics objects within the figure and reduces the precision to limit the size of the resulting file.
Note: We recommend the usage of \verb|fig2tikz| also if all objects are known to be convertible by \verb|matlab2tikz|, as the representation of \verb|Line| objects is also simplified.

% Step 3
\subsection{Function \texttt{matlab2tikz}}
\label{ssec:matlab2tikz}

Next, the conversion to a TikZ/pgfplots file can be made.
To this end, the function
%
\begin{verbatim}
  matlab2tikz(filename.tikz);
\end{verbatim}
%
has to be called, which generates a file called \verb|filename| in the current directory, where the file extension \verb|.tikz| is recommended as a naming convention, but can in principle be chosen arbitrarily.
This file can then be moved to the LaTeX code directory for integration into the document.

% Step 4
\subsection{Post-processing and integration into LaTeX}
\label{ssec:postprocessing}

Finally, the file has to be integrated into the LaTeX document using
%
\begin{verbatim}
  \input{filename.tikz}.
\end{verbatim}
In written reports, this is typically placed inside of a figure environment, while for presentations (LaTeX beamer), it can be inserted directly.
The file itself can be opened within the LaTeX editor and altered as needed.
In general, there is a header, the \verb|\begin{tikzpicture} ... \end{tikzpicture}| environment, and a \verb|\begin{axis} ... \end{axis}| environment per subplot, including the command \verb|\addplot| for each plotted object.
Here are some of the most useful tips for further polishing:
%
\begin{itemize}[itemsep=0pt]
	\item Sometimes, there is a second but empty \verb|\begin{axis} ... \end{axis}| environment at the very bottom, which can be deleted.
	\item By default, \verb|matlab2tikz| creates legend entries using the command \verb|\addlegendentry{text}| after each plotted object. If no legend is desired, one can simply add a \verb|\legend{}| command just before \verb|\end{axis}|. If only some objects should have labels, the search-and-replace function can be used to identify all undesired legend entries.
	\item The position defined in the optional parameter \verb|at| within the \verb|axis|-environment, which is only relevant if there is another \verb|axis|-environment and even then, only the relative position to one another is relevant.
	\item The size of the figure can be easily adjusted using the optional parameters \verb|width| and \verb|height| within the \verb|axis|-environment. The same goes for the axis limits and labels.
	\item Alternatively, one can scale the entire figure except the font sizes using the optional parameter
		\begin{verbatim}
		  scale=\scaleFactor
		\end{verbatim}
		in the \verb|axis|-environment. This assumes that \verb|scaleFactor| is a defined TeX ``variable''.
		By writing
		\begin{verbatim}
		  \def \scaleFactor{fac}.
		\end{verbatim}
		in the \verb|.tex|-file just before \verb|\input{filename.tikz}|, this variable can be defined.
	\item For small or large numbers, there is automated scaling by powers of 10. In order to prevent this, use the following optional parameters:
		\begin{verbatim}
		  yticklabel style={/pgf/number format/fixed,/pgf/number format/precision=N},
		  scaled y ticks=false
		\end{verbatim}
		where \verb|N| is a large enough integer.
	\item In order to push the label of the y-axis closer to the plot, use the optional parameter
		\begin{verbatim}
		  ylabel near ticks
		\end{verbatim}
		%
		(analogously for the x-axis).
	\item To write the title in boldface, use the optional parameter
		\begin{verbatim}
		  title style={font=\textbackslash bfseries}
		\end{verbatim}
	\item If you wish to have the label of the y-axis not rotated by 90 degrees, use the optional parameter
		\begin{verbatim}
		  ylabel style={rotate=-90}
		\end{verbatim}
	\item If you have macros for your colors, you have to replace the default colors \verb|mycolor1|, \verb|mycolor2|, etc. by these macros, again using the search-and-replace function of your LaTeX editor. Otherwise, altering the colors in your document would not alter those in the TikZ/pgfplots figures.
	\item If desired, you can also get rid of many elements, such as axis labels, ticks, etc. (see the \linkto{https://ftp.agdsn.de/pub/mirrors/latex/dante/graphics/pgf/contrib/pgfplots/doc/pgfplots.pdf}{pgfplots manual} for details).
	\item Within the \verb|tikzpicture|-environment, usual TikZ-commands can be written, e.g., for nodes or additional lines. If you want to add a \verb|\node| at a certain position in the plot (e.g. to add a descriptive label), this can be done by, e.g., adding
		\begin{verbatim}
		  \node[inner sep=0,pin={[anchor=center,inner sep=0,pin distance=1cm]180:$text$}]
		    at (axis cs:20,0) {}
		\end{verbatim}
	somewhere inside the \verb|axis|-environment.
	The given example draws a \SI{1}{\centi\meter} line from the point \verb|(20,0)| in the plot (with 0 seperation from that point) to the left 180 degrees, and places \verb|$text$| there.
\end{itemize}


\section{More plots in one figure}
\label{sec:groupplots}

There are instances with more than one plot in the same figure.
For this, there are two options:
%
\begin{enumerate}
	\item Using MATLAB's \verb|subplot| command: Each subplot is exported within its own \verb|axis|-environment. Thus, positioning via the optional parameter \verb|at| has to be done manually, depending also on the size (parameters \verb|width| and \verb|height|) of the individual \verb|axis|-environments.
	\item Using the pfgplots feature \verb|groupplots| (requires \verb|\usepgfplotslibrary{groupplots}| in the preamble): The figure is constructed by one high-level file, grouping all individual subplots and defining axis settings, such as labels, titles etc., and the actual plot data. We recommend to put the plot data for each subplot into a separate file, i.e., only the \verb|\textbackslash addplot| commands that make up each subplot as everything else is now handled by the high-level group file, which is defined as
		\begin{verbatim}
		  \begin{groupplot}[$\bullet$]
		\end{verbatim}
		with the optional parameter
		\begin{verbatim}
		  group style = {group size = m by n, horizontal sep = x.ycm, vertical sep = x.ycm}
		\end{verbatim}
		specifying an \verb|n|-by-\verb|m| grid (\verb|n| rows, \verb|m| columns). The individual subplots are added by
		\begin{verbatim}
		  \nextgroupplot[$\bullet$]
		\end{verbatim}
		with optional parameter \verb|title=nameofsubplot|, and
		\begin{verbatim}
		  \input{nameofsubplot.tikz}
		\end{verbatim}
		is the file containing all plots for that specific subplot.
	One advantage over the previous option---in addition to perfect alignment of all suplots within the grid---is the ability to write a single legend for multiple plots, i.e., using the TikZ library \verb|matrix| and the command
		\begin{verbatim}
		  \matrix[$\bullet$] at (x,y) {
		    11 & ... & 1m \\
		    ... \\
		    n1 & ... & nm};
		\end{verbatim}
		where the optional parameters are
		\begin{verbatim}
		  matrix of nodes, nodes = {text width = x.ycm, align=left|center|right}, draw
		\end{verbatim}
	If one labels a specific plot, e.g.
		\begin{verbatim}
		  \addplot [...]{...};
		  \label{ref:plot};
		\end{verbatim}
	the label symbol can be obtained. Within the legend matrix as shown above, write \verb|\ref{ref:plot}|.
	The corresponding entry in the \verb|\matrix|-command can then be replaced by 
		\begin{verbatim}
		  \ref{ref:plot} description
		\end{verbatim}
		where \verb|description| is the text after the symbol.
\end{enumerate}


\pagebreak

\section{Examples}
\label{sec:examples}

\subsection{Simple plot}
\label{ssec:ex_simpleplot}

The following MATLAB code generates two data graphs over time:

\bigskip
{\small \setstretch{1} % This file was automatically created from the m-file 
% "m2tex.m" written by USL. 
% The fontencoding in this file is UTF-8. 
%  
% You will need to include the following two packages in 
% your LaTeX-Main-File. 
%  
% \usepackage{color} 
% \usepackage{fancyvrb} 
%  
% It is advised to use the following option for Inputenc 
% \usepackage[utf8]{inputenc} 
%  
  
% definition of matlab colors: 
\definecolor{mblue}{rgb}{0,0,1} 
\definecolor{mgreen}{rgb}{0.13333,0.5451,0.13333} 
\definecolor{mred}{rgb}{0.62745,0.12549,0.94118} 
\definecolor{mgrey}{rgb}{0.5,0.5,0.5} 
\definecolor{mdarkgrey}{rgb}{0.25,0.25,0.25} 
  
\DefineShortVerb[fontfamily=courier,fontseries=m]{\$} 
\DefineShortVerb[fontfamily=courier,fontseries=b]{\#} 
  
\noindent                   
 $close $\color{mred}$all; clear;$\color{black}$$\\
 $$\\
 $$\color{mgreen}$% equidistant time step vector$\color{black}$$\\
 $numPoints = 50;$\\
 $timeStepVector = linspace(0,1,numPoints)';$\\
 $$\\
 $$\color{mgreen}$% generate random data$\color{black}$$\\
 $data1 = 0.001*rand(1,numPoints);$\\
 $data2 = 0.001*randn(1,numPoints);$\\
 $$\\
 $$\color{mgreen}$% plot data$\color{black}$$\\
 $figure; $\color{mred}$hold on; box on;$\color{black}$$\\
 $h1 = plot(timeStepVector,data1);$\\
 $h2 = plot(timeStepVector,data2);$\\
 $xlabel($\color{mred}$'t'$\color{black}$);$\\
 $ylabel($\color{mred}$'xi'$\color{black}$);$\\
 $legend([h1,h2],$\color{mred}$'Data1'$\color{black}$,$\color{mred}$'Data2'$\color{black}$,$\color{mred}$'Location'$\color{black}$,$\color{mred}$'northwest'$\color{black}$);$\\
 $$\\
 $matlab2tikz($\color{mred}$'simpleplot.tikz'$\color{black}$);$\\ 
  
\UndefineShortVerb{\$} 
\UndefineShortVerb{\#}}

\noindent
The resulting \verb|.tikz|-file looks as follows (omitting most lines of data):

\bigskip
{\small \setstretch{1} \begin{verbatim}
% This file was created by matlab2tikz.
%
%The latest updates can be retrieved from
%  http://www.mathworks.com/matlabcentral/fileexchange/22022-matlab2tikz-matlab2tikz
%where you can also make suggestions and rate matlab2tikz.
%
\definecolor{mycolor1}{rgb}{0.00000,0.44700,0.74100}%
\definecolor{mycolor2}{rgb}{0.85000,0.32500,0.09800}%
%
\begin{tikzpicture}

\begin{axis}[%
width=4.521in,
height=3.566in,
at={(0.758in,0.481in)},
scale only axis,
xmin=0,
xmax=1,
xlabel style={font=\color{white!15!black}},
xlabel={t},
ymin=-0.003,
ymax=0.003,
ylabel style={font=\color{white!15!black}},
ylabel={xi},
axis background/.style={fill=white},
legend style={at={(0.03,0.97)}, anchor=north west, legend cell align=left, align=left, draw=white!15!black}
]
\addplot [color=mycolor1]
  table[row sep=crcr]{%
0	4.9625262475548e-05\\
...
1	0.000105905694944612\\
};
\addlegendentry{Data1}

\addplot [color=mycolor2]
  table[row sep=crcr]{%
0	0.000938552313581848\\
...
1	-0.00112902046539328\\
};
\addlegendentry{Data2}

\end{axis}
\end{tikzpicture}%
\end{verbatim}}

\noindent
In order to obtain a clean figure, we perform the following steps:
\begin{itemize}[itemsep=0pt]
	\item Rewrite the labels: \verb|xlabel={t}| $\to$ \verb|xlabel={$t$}| and \verb|xlabel={xi}| $\to$ \verb|xlabel={$\xi$}|.
	\item Resize the figure: \verb|width=4.521in| $\to$ \verb|width=5in| and \verb|height=3.566in| $\to$ \verb|height=3in|.
	\item Display the ticks of the y-axis using decimal numbers by adding
		\begin{verbatim}
		  yticklabel style={/pgf/number format/fixed,/pgf/number format/precision=3},
		  scaled y ticks=false
		\end{verbatim}
\end{itemize}
%
Then, we obtain (again, omitting most lines of data)

\bigskip
{\small \setstretch{1} \input{example1/simpleplotvrb_after}}

\noindent
Finally, the integration in LaTeX produces the follwing figure:

\bigskip

\begin{center}
	% This file was created by matlab2tikz.
%
%The latest updates can be retrieved from
%  http://www.mathworks.com/matlabcentral/fileexchange/22022-matlab2tikz-matlab2tikz
%where you can also make suggestions and rate matlab2tikz.
%
\definecolor{mycolor1}{rgb}{0.00000,0.44700,0.74100}%
\definecolor{mycolor2}{rgb}{0.85000,0.32500,0.09800}%
%
\begin{tikzpicture}

\begin{axis}[%
width=5in,
height=3in,
at={(0.758in,0.481in)},
scale only axis,
xmin=0,
xmax=1,
xlabel style={font=\color{white!15!black}},
xlabel={$t$},
ymin=-0.003,
ymax=0.003,
ylabel style={font=\color{white!15!black}},
ylabel={$\xi$},
scaled y ticks=false,
yticklabel style={/pgf/number format/fixed,/pgf/number format/precision=3},
axis background/.style={fill=white},
legend style={at={(0.03,0.97)}, anchor=north west, legend cell align=left, align=left, draw=white!15!black}
]
\addplot [color=mycolor1]
  table[row sep=crcr]{%
0	4.9625262475548e-05\\
0.0204081632653061	0.000813968936273827\\
0.0408163265306122	0.00088743467688427\\
0.0612244897959184	1.20272204613291e-05\\
0.0816326530612245	0.000249766475514849\\
0.102040816326531	5.34283138529675e-05\\
0.122448979591837	0.000351800031589419\\
0.142857142857143	0.00037378550949381\\
0.163265306122449	0.00070016422749783\\
0.183673469387755	0.00044146308600365\\
0.204081632653061	0.000879065736718516\\
0.224489795918367	0.000203433150546306\\
0.244897959183673	0.00010605311712161\\
0.26530612244898	0.000235275916987647\\
0.285714285714286	0.000610486903214436\\
0.306122448979592	0.000595810436727333\\
0.326530612244898	0.000535021169998628\\
0.346938775510204	0.00065201147218538\\
0.36734693877551	0.000758678423303028\\
0.387755102040816	0.000305872113526422\\
0.408163265306122	0.000294189884673659\\
0.428571428571429	0.000518502778618377\\
0.448979591836735	0.000720280639206392\\
0.469387755102041	0.000911393500297169\\
0.489795918367347	0.000323807783558135\\
0.510204081632653	0.000225032000946106\\
0.530612244897959	0.000116121744965461\\
0.551020408163265	0.000324164880375058\\
0.571428571428571	4.06352611084866e-05\\
0.591836734693878	0.000329770953906751\\
0.612244897959184	0.000108666089301287\\
0.63265306122449	0.000585019982536578\\
0.653061224489796	0.000886082580564025\\
0.673469387755102	0.000898562971412446\\
0.693877551020408	0.000535417420985506\\
0.714285714285714	0.000755166866592859\\
0.73469387755102	0.000416722121958315\\
0.755102040816326	2.41277313744011e-05\\
0.775510204081633	0.000200170880401206\\
0.795918367346939	0.000226596439837587\\
0.816326530612245	0.000558167746218104\\
0.836734693877551	7.71384072318998e-05\\
0.857142857142857	0.000419830604499171\\
0.877551020408163	0.000577641191556322\\
0.897959183673469	0.000841025382662212\\
0.918367346938776	0.000674350236522445\\
0.938775510204082	0.000811149753544197\\
0.959183673469388	7.58535867763351e-05\\
0.979591836734694	0.000571052826934493\\
1	0.000105905694944612\\
};
\addlegendentry{Data1}

\addplot [color=mycolor2]
  table[row sep=crcr]{%
0	0.000938552313581848\\
0.0204081632653061	-0.000513515941013523\\
0.0408163265306122	0.000771146320125706\\
0.0612244897959184	0.00275199989977046\\
0.0816326530612245	-0.00082920242111243\\
0.102040816326531	-0.000338427189396113\\
0.122448979591837	0.0017620256456283\\
0.142857142857143	0.00142778331890313\\
0.163265306122449	0.000439767622201758\\
0.183673469387755	-0.000418033003544596\\
0.204081632653061	-0.000786976614714477\\
0.224489795918367	-0.00101703638129472\\
0.244897959183673	-0.000822009275315497\\
0.26530612244898	0.000280479580883242\\
0.285714285714286	0.000286136709598171\\
0.306122448979592	0.00166468195153087\\
0.326530612244898	4.86351525466826e-05\\
0.346938775510204	0.000530306158927802\\
0.36734693877551	-0.000692422061279985\\
0.387755102040816	0.000161705098556844\\
0.408163265306122	-0.00124369493554393\\
0.428571428571429	0.000721027678616853\\
0.448979591836735	-0.00147488731983908\\
0.469387755102041	0.00012539819495857\\
0.489795918367347	0.00103180381625194\\
0.510204081632653	0.000392912934399512\\
0.530612244897959	0.000592386134005938\\
0.551020408163265	-0.000484552255197653\\
0.571428571428571	-0.00269260848954724\\
0.591836734693878	-0.000482045201398429\\
0.612244897959184	0.000395621723490385\\
0.63265306122449	-0.00153842348368708\\
0.653061224489796	-0.00081963693097691\\
0.673469387755102	0.00165024218735551\\
0.693877551020408	-0.000109672711060352\\
0.714285714285714	7.1440670262146e-05\\
0.73469387755102	-0.00150433822122832\\
0.755102040816326	-4.7037930692383e-05\\
0.775510204081633	0.000949445676999243\\
0.795918367346939	-0.00176691308525196\\
0.816326530612245	-0.00039625728741674\\
0.836734693877551	-0.000313862795051354\\
0.857142857142857	0.000383013665188139\\
0.877551020408163	0.00134681294970769\\
0.897959183673469	0.00171399034234155\\
0.918367346938776	0.000522113947699392\\
0.938775510204082	0.00154716954123225\\
0.959183673469388	-0.000614487094685593\\
0.979591836734694	-0.000255684360740088\\
1	-0.00112902046539328\\
};
\addlegendentry{Data2}

\end{axis}
\end{tikzpicture}%
\end{center}



\pagebreak

\subsection{Group plot}
\label{ssec:ex_groupplot}

For the group plot, we omit the generation of the plot and only show the \verb|.tikz|-files and the corresponding result.
In this example, we have a 1-by-3 grid, where the file for the first plot just contains plots generated using the \verb|\addplot|-command.
More specifically, there are:
%
\begin{itemize}[itemsep=0pt]
	\item 1 plot using the \verb|IntReachStyle|
	\item 1 plot using the \verb|X0Style|
	\item 5 plots using the \verb|TpReachStyle|
	\item 5 plots using the \verb|TpReachStyle|
	\item 7 plots using the \verb|SimStyle|
	\item 1 plot using the \verb|ShiftX0Style|
	\item 1 plot using the \verb|FinalReachStyle|
\end{itemize}

The first file called \verb|groupplot1.tikz| looks like this:

\bigskip
{\small \setstretch{1} \begin{verbatim}
\addplot[IntReachStyle]
table[row sep=crcr] {%
-0.199430396792808	-0.021422662084432\\
...
-0.18670849781896	-0.02142452486451\\
};\label{p:IntReach}


\addplot[X0Style]
table[row sep=crcr] {%
-0.2	-0.02\\
0.2	-0.02\\
0.2	0.02\\
-0.2	0.02\\
-0.2	-0.02\\
};\label{p:X0}

\addplot [TpReachStyle]
  table[row sep=crcr]{%
-0.0170878647697879	0.0408154613219366\\
...
-0.0170878647697879	0.0408154613219366\\
};\label{p:TpReach}

\addplot [TpReachStyle]
  table[row sep=crcr]{%
0.0211710807744047	-0.0962760070090079\\
...
0.0211710807744047	-0.0962760070090079\\
};

\addplot [TpReachStyle]
  table[row sep=crcr]{%
0.0145081289453033	-0.0547169712108607\\
...
0.0145081289453033	-0.0547169712108607\\
};

\addplot [TpReachStyle]
  table[row sep=crcr]{%
-0.020671372484484	0.00445757184640437\\
...
-0.020671372484484	0.00445757184640437\\
};

\addplot [TpReachStyle]
  table[row sep=crcr]{%
-0.0104699169489793	0.0145418111234907\\
...
-0.0104699169489793	0.0145418111234907\\
};

\addplot [SimStyle]
  table[row sep=crcr]{%
0.2	0.02\\
...
-0.00599003013886154	-0.00657151054038141\\
};\label{p:Sim}

\addplot [SimStyle]
  table[row sep=crcr]{%
-0.2	-0.02\\
...
0.00141889771152248	-0.00029946251214133\\
};

\addplot [SimStyle]
  table[row sep=crcr]{%
-0.2	-0.02\\
...
0.00220404813719008	-0.00614176712919964\\
};


\addplot [SimStyle]
  table[row sep=crcr]{%
-0.2	0.02\\
...
0.000381567659612808	-0.00113027079952846\\
};

\addplot [SimStyle]
  table[row sep=crcr]{%
-0.0859567657365433	0.0131092869379305\\
...
-0.000107164742825155	7.18617741371358e-05\\
};

\addplot [SimStyle]
  table[row sep=crcr]{%
0.0707484373753676	-0.0116958786248008\\
...
-0.000648470068926488	-0.00036666906454888\\
};


\addplot [SimStyle]
  table[row sep=crcr]{%
-0.132093173589405	-0.0140937689139305\\
...
0.0025841168809046	-0.00343873838172712\\
};

\addplot [ShiftX0Style]
  table[row sep=crcr]{%
-0.2	-0.02\\
0.2	-0.02\\
0.2	0.02\\
-0.2	0.02\\
-0.2	-0.02\\
};\label{p:ShiftX0}

\addplot [FinalReachStyle]
  table[row sep=crcr]{%
-0.0113937453780825	0.014365845328073\\
...
-0.0113937453780825	0.014365845328073\\
};\label{p:FinalReach}
\end{verbatim}}

\noindent
The optional parameters for the \verb|\addplot| commands are defined in the main file (see below), which provides the grid and legend for the three individual files \verb|groupplot1.tikz|, \verb|groupplot2.tikz|, and \verb|groupplot3.tikz|.
It looks like this:

\bigskip
{\small \setstretch{1} \input{example2/groupplotmastervrb}}

\noindent
Finally, the resulting plot:

\bigskip
\begin{center}
	\pgfplotsset{
	X0Style/.style={
		 draw=black, 
		 fill=white,
		 area legend
	},
	ShiftX0Style/.style={
		draw=green,
		area legend
	},
	IntReachStyle/.style={
		draw=none, 
		fill=white!70!black,
		area legend
	},
	TpReachStyle/.style={
		draw=blue,
		area legend
	},
	FinalReachStyle/.style={
		draw=red,
		area legend
	},
	SimStyle/.style={
		draw=black
	}
}

\begin{tikzpicture}
	\begin{groupplot}[
		group style =  {
			group size = 3 by 1,
			horizontal sep = 1.4cm,
			x descriptions at=edge bottom
		},
		width = 0.33\textwidth
		]
 		\nextgroupplot[title={$[\beta,~\Psi]$}]
		\addplot[IntReachStyle]
table[row sep=crcr] {%
-0.199430396792808	-0.021422662084432\\
-0.199651372155972	-0.021359605144234\\
-0.202752531162759	-0.0188031804218166\\
-0.203330891320793	0.0213433920636021\\
-0.201749919005058	0.0214226987505193\\
-0.19283586870089	0.02146134873723\\
-0.192868571274838	0.0224355704771157\\
-0.191948594820236	0.0225944756826477\\
-0.190275032462258	0.0227291113635534\\
-0.18289065020717	0.0228293846022\\
-0.182916969792435	0.0234298866451675\\
-0.181958232648747	0.0237237233519117\\
-0.1797601964483	0.024035930032476\\
-0.17345241470223	0.02419113727961\\
-0.173447776150526	0.0243427645414239\\
-0.172684647361068	0.0247411570102978\\
-0.169655179946455	0.0253653465773509\\
-0.16398366048363	0.02556444377803\\
-0.163723789665947	0.0257205776157008\\
-0.160160615221566	0.0267043306182108\\
-0.15428522271502	0.02697569853965\\
-0.151250365418045	0.0280490587315484\\
-0.14516862521819	0.02840611606878\\
-0.143209120527962	0.0293059716288549\\
-0.142875621822876	0.0293999500060329\\
-0.13678966336393	0.02983983965714\\
-0.135185773103479	0.0307212302027988\\
-0.12902700686439	0.03127471861129\\
-0.127791480803117	0.0320789680377844\\
-0.12181234517277	0.03270873182139\\
-0.120850704539555	0.033437490744471\\
-0.11509026489603	0.0341410157315999\\
-0.114550432505027	0.0346693620706335\\
-0.114313742553615	0.0347999768769859\\
-0.10880644790162	0.03557090331949\\
-0.108411496720427	0.0360279583478268\\
-0.108172144360963	0.0361623952830073\\
-0.10292388693516	0.0369983774909301\\
-0.102635470737989	0.0373900816856748\\
-0.102397766956445	0.0375245879948259\\
-0.0974131009553401	0.03842207480148\\
-0.0972017880609745	0.0387545961783906\\
-0.096967480164673	0.0388861351327725\\
-0.0922522203491901	0.0398430022509601\\
-0.0920933132949742	0.0401181938193015\\
-0.091863329849738	0.0402469260321763\\
-0.0874086824690201	0.04126222131138\\
-0.0872900351141208	0.0414802991201806\\
-0.087065901344268	0.0416066632676967\\
-0.0828642235343901	0.04267771098071\\
-0.0827731549427382	0.0428416892769244\\
-0.0825514266226366	0.0429664105051239\\
-0.0786228343944701	0.0440820166729101\\
-0.0783046243979681	0.0443253896251274\\
-0.0746515334990501	0.04547725070276\\
-0.0743091141158871	0.0456830783733077\\
-0.0709221440683801	0.04686555521566\\
-0.0674012709535701	0.04825813688601\\
-0.0639579598197601	0.04972868570267\\
-0.0610694741683206	0.0510785075526728\\
-0.0607357555136601	0.0511683914052\\
-0.0581603786727949	0.0524896524533518\\
-0.0577491360937901	0.0525861159436\\
-0.0554202242895607	0.0538957323391914\\
-0.0549528630727001	0.05399529936631\\
-0.0528385215625225	0.0552965800602231\\
-0.0523287114102501	0.05539860469577\\
-0.05040758651669	0.0566912758599791\\
-0.0498652612467701	0.0567962448538901\\
-0.0481147762132431	0.058082295278324\\
-0.0475531991705301	0.05818836506267\\
-0.0459543928594141	0.059468235116962\\
-0.0453864365301601	0.0595750187203401\\
-0.0439183608351439	0.0608495966108951\\
-0.0433505382701401	0.06095676281085\\
-0.0432309179531501	0.06107052588752\\
-0.0432076452673604	0.0616651889102299\\
-0.0429524311179801	0.06221142771002\\
-0.0430540058627829	0.0630467896566407\\
-0.0430605028925905	0.0639063542347654\\
-0.0426869226460801	0.0646755499959599\\
-0.042766393367089	0.0661022841875409\\
-0.0423193758525101	0.0669395162666999\\
-0.0423641772407519	0.0681712372446805\\
-0.0418495068398601	0.06905539474245\\
-0.0418690024390651	0.070120679707293\\
-0.0412892981783101	0.07104629390136\\
-0.0412689852600708	0.0720433037626475\\
-0.0406491029989501	0.0729129774651099\\
-0.0406225919181248	0.0737712252449386\\
-0.0399445774482701	0.0746652801065999\\
-0.0399153583751447	0.0753919523536707\\
-0.0391889906642501	0.07629935579954\\
-0.0391572961357706	0.0769201562473181\\
-0.0383929797927601	0.07782697151801\\
-0.0383612974705161	0.0783570004437787\\
-0.0375629420106901	0.07925416419874\\
-0.0375331066129625	0.079707353108962\\
-0.0367061000990501	0.08059086941354\\
-0.0366768690295307	0.0809775098533534\\
-0.0363520543827501	0.0813354405669099\\
-0.0385160522311132	0.0899842202848225\\
-0.0389794575031872	0.0923323357553222\\
-0.0392180100018911	0.0944543299515426\\
-0.0392423829249247	0.0959265913137908\\
-0.0391527948003915	0.0973862507101987\\
-0.0390631099779961	0.0977156044060388\\
-0.0389196790119801	0.0979235387285299\\
-0.0388083620283209	0.0984983275019024\\
-0.0386648738115359	0.0987395138589379\\
-0.0384121611119201	0.0988967747836599\\
-0.0381794481143187	0.0993924664394686\\
-0.0378855280469335	0.0995567122221007\\
-0.0375778086418401	0.0996263490068899\\
-0.037414056268036	0.0998401088971303\\
-0.0371041684826682	0.100005745469481\\
-0.0346119675548981	0.100428734191414\\
-0.0324370000258199	0.100567970815274\\
-0.0297063946003347	0.100569378182816\\
-0.0277169410984893	0.100560419719862\\
-0.0269048580455987	0.100443965849779\\
-0.0253686158800306	0.0999262770760703\\
-0.0244360648766701	0.09945583787327\\
-0.023872338889452	0.0992704384524003\\
-0.0224730511882301	0.09848893544447\\
-0.0219094495448697	0.0982617572991\\
-0.0203360855643401	0.09724324608697\\
-0.0196041983496784	0.0969140869325427\\
-0.0173250450717125	0.0952128400399805\\
-0.0165455968021703	0.0943328328624853\\
-0.0158579710091957	0.0931199049541342\\
-0.0154291365581256	0.0921723204925155\\
-0.00896371901234014	0.0740429197579699\\
-0.00745346892176579	0.0713046186535064\\
-0.00487347825904014	0.0657703693836899\\
-0.00436792249312873	0.0648261505227891\\
-0.00392136795950014	0.0637614006326099\\
-0.00333718155427724	0.062662694148157\\
-0.00197429348323014	0.0594302024384099\\
-0.00129808354844376	0.0581388994851078\\
0.00208512235008986	0.0496069765691199\\
0.00331959589178986	0.0484841432964999\\
0.00435029077070218	0.0477243307436767\\
0.00635825029910986	0.0457889050990599\\
0.00697544331642118	0.0452769989873664\\
0.00924436351161449	0.0429903804746361\\
0.0105885498160199	0.0415215944929799\\
0.0112275530517071	0.0409573893527722\\
0.0127887742300648	0.0391204348188622\\
0.0150818091476599	0.0361400065219399\\
0.0157439106376898	0.0354828058931402\\
0.0168721480574807	0.0338672969586874\\
0.0179871732786299	0.0318785706891099\\
0.018536156108972	0.0312200045334977\\
0.0193786716550399	0.0294125069252199\\
0.0297761161726299	0.02871185863636\\
0.0420427492423499	0.0280710461352199\\
0.0585659590574299	0.02739629976446\\
0.0810929371186399	0.02669304911949\\
0.11250168344417	0.02596398564371\\
0.127372625515353	0.0257162884154778\\
0.12799548273671	0.02555073120133\\
0.137357045272116	0.0254425179324229\\
0.137757131152348	0.0253190881643199\\
0.13813188590946	0.02508050779709\\
0.147973556779847	0.0250078279930988\\
0.148428599986715	0.024854772280433\\
0.14894048240981	0.02444310048925\\
0.159429314683085	0.024398202951172\\
0.159906364401677	0.0242247459908247\\
0.16059890780668	0.02362778903037\\
0.171781286079813	0.0236038139633464\\
0.172197082595669	0.0234536771972103\\
0.17316677563085	0.02262495029913\\
0.185090870755401	0.0226154268284707\\
0.185406364320304	0.0225140025256288\\
0.18670849781896	0.02142452486451\\
0.199546024954691	0.0214085355143214\\
0.202752531162759	0.0188031804218167\\
0.203330891320793	-0.021343392063602\\
0.201749919005058	-0.0214226987505192\\
0.19283586870089	-0.02146134873723\\
0.192868571274838	-0.0224355704771157\\
0.191948594820236	-0.0225944756826477\\
0.190275032462258	-0.0227291113635534\\
0.18289065020717	-0.0228293846022\\
0.182916969792436	-0.0234298866451676\\
0.181958232648747	-0.0237237233519118\\
0.179760196448301	-0.0240359300324761\\
0.17345241470223	-0.02419113727961\\
0.173447776150526	-0.0243427645414239\\
0.172684647361068	-0.0247411570102978\\
0.169655179946455	-0.0253653465773509\\
0.16398366048363	-0.02556444377803\\
0.163723789665947	-0.0257205776157008\\
0.160160615221566	-0.0267043306182108\\
0.15428522271502	-0.02697569853965\\
0.151250365418045	-0.0280490587315483\\
0.14516862521819	-0.02840611606878\\
0.143209120527962	-0.0293059716288549\\
0.142875621822876	-0.0293999500060329\\
0.13678966336393	-0.02983983965714\\
0.135185773103479	-0.0307212302027988\\
0.12902700686439	-0.03127471861129\\
0.127791480803117	-0.0320789680377844\\
0.12181234517277	-0.03270873182139\\
0.120850704539555	-0.0334374907444709\\
0.11509026489603	-0.0341410157316001\\
0.114550432505027	-0.0346693620706336\\
0.114313742553615	-0.034799976876986\\
0.10880644790162	-0.03557090331949\\
0.108411496720427	-0.0360279583478268\\
0.108172144360963	-0.0361623952830073\\
0.10292388693516	-0.0369983774909299\\
0.102635470737989	-0.0373900816856747\\
0.102397766956445	-0.0375245879948258\\
0.0974131009553399	-0.03842207480148\\
0.0972017880609744	-0.0387545961783907\\
0.0969674801646729	-0.0388861351327726\\
0.0922522203491899	-0.0398430022509599\\
0.0920933132949743	-0.0401181938193014\\
0.091863329849738	-0.0402469260321762\\
0.0874086824690199	-0.04126222131138\\
0.0872900351141208	-0.0414802991201807\\
0.087065901344268	-0.0416066632676967\\
0.0828642235343899	-0.04267771098071\\
0.0827731549427383	-0.0428416892769244\\
0.0825514266226368	-0.0429664105051239\\
0.0786228343944699	-0.0440820166729099\\
0.0783046243979681	-0.0443253896251273\\
0.0746515334990499	-0.04547725070276\\
0.0743091141158872	-0.0456830783733077\\
0.0709221440683799	-0.04686555521566\\
0.0674012709535699	-0.04825813688601\\
0.0639579598197599	-0.04972868570267\\
0.0610694741683206	-0.0510785075526728\\
0.0607357555136599	-0.0511683914052\\
0.0581603786727948	-0.0524896524533518\\
0.0577491360937899	-0.0525861159436\\
0.0554202242895607	-0.0538957323391913\\
0.0549528630726999	-0.05399529936631\\
0.0528385215625225	-0.0552965800602231\\
0.0523287114102499	-0.05539860469577\\
0.05040758651669	-0.0566912758599792\\
0.0498652612467699	-0.05679624485389\\
0.0481147762132431	-0.0580822952783239\\
0.0475531991705299	-0.05818836506267\\
0.0459543928594141	-0.0594682351169621\\
0.0453864365301599	-0.0595750187203399\\
0.0439183608351438	-0.060849596610895\\
0.0433505382701399	-0.06095676281085\\
0.0432309179531499	-0.06107052588752\\
0.0432076452673604	-0.0616651889102299\\
0.0429524311179799	-0.06221142771002\\
0.0430540058627829	-0.0630467896566408\\
0.0430605028925905	-0.0639063542347654\\
0.0426869226460799	-0.0646755499959601\\
0.042766393367089	-0.0661022841875411\\
0.0423193758525199	-0.0669395162667001\\
0.0423641772407518	-0.0681712372446806\\
0.0418495068398599	-0.06905539474245\\
0.0418690024390651	-0.0701206797072931\\
0.0412892981783099	-0.07104629390137\\
0.0412689852600708	-0.0720433037626474\\
0.0406491029989499	-0.0729129774651101\\
0.0406225919181247	-0.0737712252449388\\
0.0399445774482699	-0.0746652801066001\\
0.0399153583751447	-0.0753919523536709\\
0.0391889906642499	-0.07629935579954\\
0.0391572961357706	-0.076920156247318\\
0.0383929797927599	-0.07782697151801\\
0.038361297470516	-0.0783570004437787\\
0.0375629420106899	-0.07925416419874\\
0.0375331066129625	-0.079707353108962\\
0.0367061000990499	-0.08059086941354\\
0.0366768690295307	-0.0809775098533534\\
0.0363520543827499	-0.0813354405669101\\
0.0385160522311131	-0.0899842202848225\\
0.0388905827189994	-0.0917395247751045\\
0.0392389107500477	-0.094819499906212\\
0.0391879893065836	-0.0970511373699857\\
0.0390985153849403	-0.0976180719517812\\
0.0389196790119799	-0.0979235387285301\\
0.0387738887688285	-0.0985791740840918\\
0.0386484116772253	-0.0987546480493044\\
0.0384121611119199	-0.0988967747836601\\
0.0382110544348959	-0.0993605310593013\\
0.0380395479597759	-0.0994938398040715\\
0.0375778086418399	-0.0996263490068901\\
0.0372692683341773	-0.0999419931543032\\
0.0346288894094	-0.100426968867612\\
0.0318775179367199	-0.100569378182816\\
0.0277167144559069	-0.100560398112513\\
0.0268195213951697	-0.100421923768651\\
0.0253412172016172	-0.0999156009314402\\
0.0244360648766699	-0.09945583787327\\
0.0238548825343988	-0.0992621321202661\\
0.0224730511882299	-0.09848893544447\\
0.0219670715486264	-0.0982925531694716\\
0.0203360855643399	-0.09724324608697\\
0.0196190418366719	-0.0969237874234678\\
0.0173250450717124	-0.0952128400399804\\
0.0165455968021702	-0.0943328328624852\\
0.0158579710091956	-0.0931199049541341\\
0.0154291365581255	-0.0921723204925154\\
0.00896371901233986	-0.0740429197579701\\
0.00745346892176575	-0.0713046186535064\\
0.00487347825903986	-0.0657703693836901\\
0.00436792249312869	-0.0648261505227892\\
0.00392136795949986	-0.0637614006326101\\
0.00333718155427716	-0.0626626941481569\\
0.00197429348322986	-0.0594302024384101\\
0.00129808354844372	-0.0581388994851076\\
-0.00208512235009014	-0.0496069765691201\\
-0.00331959589179014	-0.0484841432965001\\
-0.00435029077070224	-0.0477243307436766\\
-0.00635825029911014	-0.0457889050990601\\
-0.00697544331642125	-0.0452769989873664\\
-0.00924436351161456	-0.0429903804746361\\
-0.0105885498160201	-0.0415215944929801\\
-0.0112275530517071	-0.0409573893527722\\
-0.0127887742300649	-0.0391204348188623\\
-0.0150818091476601	-0.0361400065219401\\
-0.0157439106376899	-0.0354828058931402\\
-0.0168721480574807	-0.0338672969586874\\
-0.0179871732786301	-0.0318785706891101\\
-0.018536156108972	-0.0312200045334977\\
-0.0193786716550501	-0.0294125069252001\\
-0.0297761161723401	-0.02871185863638\\
-0.0420427492437701	-0.0280710461351401\\
-0.0585659590565001	-0.02739629976449\\
-0.0810929371186401	-0.02669304911949\\
-0.11250168344417	-0.02596398564371\\
-0.127372625515353	-0.0257162884154777\\
-0.12799548273671	-0.02555073120133\\
-0.137357045272116	-0.0254425179324229\\
-0.137757131152348	-0.0253190881643199\\
-0.13813188590946	-0.02508050779709\\
-0.147973556779846	-0.0250078279930988\\
-0.148428599986715	-0.024854772280433\\
-0.14894048240981	-0.02444310048925\\
-0.159429314683085	-0.024398202951172\\
-0.159906364401678	-0.0242247459908247\\
-0.16059890780668	-0.02362778903037\\
-0.171781286079812	-0.0236038139633464\\
-0.172197082595669	-0.0234536771972103\\
-0.17316677563085	-0.02262495029913\\
-0.185090870755401	-0.0226154268284706\\
-0.185406364320304	-0.0225140025256288\\
-0.18670849781896	-0.02142452486451\\
};\label{p:IntReach}


\addplot[X0Style]
table[row sep=crcr] {%
-0.2	-0.02\\
0.2	-0.02\\
0.2	0.02\\
-0.2	0.02\\
-0.2	-0.02\\
};\label{p:X0}

\addplot [TpReachStyle]
  table[row sep=crcr]{%
-0.0170878647697879	0.0408154613219366\\
-0.0129650364111293	0.036694223628136\\
0.0100596527932564	0.0145224534096278\\
0.0104948709321938	0.0141752921550822\\
0.018428523921857	0.00663652676822175\\
0.0185959077931067	0.00648746748811813\\
0.0269552759130155	-0.00160816333778914\\
0.0405779765303283	-0.0238552802301441\\
0.0408319778734259	-0.0246686692167406\\
0.0408564479121546	-0.0249416413967758\\
0.0409059474685525	-0.0607924458115057\\
0.0406917223905363	-0.0610201190855763\\
0.0403339793952603	-0.0610872230306263\\
0.039294079134582	-0.0610429213674003\\
0.0371876893314005	-0.0604452026769996\\
0.0364040363211038	-0.0598847305180133\\
0.0282526636471288	-0.0518224914476665\\
0.016920024167325	-0.0406403800366983\\
0.0167080296273655	-0.0403633001267324\\
-0.0102040693470943	-0.0144209764807479\\
-0.0104948709321938	-0.0141752921550822\\
-0.018428523921857	-0.00663652676822177\\
-0.0185959077931067	-0.00648746748811815\\
-0.0269552759130155	0.00160816333778912\\
-0.0405779765303283	0.0238552802301441\\
-0.0408319778734259	0.0246686692167406\\
-0.0408564479121546	0.0249416413967758\\
-0.0409059474685525	0.0607924458115057\\
-0.0406917223905363	0.0610201190855762\\
-0.0403339793952603	0.0610872230306263\\
-0.039294079134582	0.0610429213674003\\
-0.0371876893314005	0.0604452026769996\\
-0.0364040363211038	0.0598847305180133\\
-0.0282526636471288	0.0518224914476664\\
-0.0170878647697879	0.0408154613219366\\
};\label{p:TpReach}

\addplot [TpReachStyle]
  table[row sep=crcr]{%
0.0211710807744047	-0.0962760070090079\\
0.0199557856706441	-0.0961641284432295\\
0.0193817882224041	-0.0958503291228253\\
0.0177375150135271	-0.0936911220134742\\
0.0169278266997651	-0.0920869143622527\\
0.0168341426752977	-0.0918268286471429\\
0.00939151776732161	-0.0696576099882894\\
0.00480008786102178	-0.049146119695059\\
0.00477180699460837	-0.048968571794237\\
-0.00137459055078326	-0.021647745043401\\
-0.00906748592568987	0.0114379671139868\\
-0.00993330191633668	0.0150915392409336\\
-0.0146835304721824	0.0353111141780283\\
-0.0147644259762568	0.0356131419209203\\
-0.0156984199819879	0.0400024014506986\\
-0.0199057333615782	0.0717229060422008\\
-0.0224997616319643	0.0920184461500022\\
-0.0222825985033807	0.0958723748781984\\
-0.0219252260985353	0.0962077749676211\\
-0.0215605963942131	0.0962760070090079\\
-0.0199557856706441	0.0961641284432295\\
-0.0193817882224041	0.0958503291228253\\
-0.0177375150135271	0.0936911220134742\\
-0.0169278266997651	0.0920869143622527\\
-0.0168341426752977	0.0918268286471429\\
-0.00939151776732162	0.0696576099882894\\
-0.00480008786102178	0.049146119695059\\
-0.00477180699459939	0.0489685717942485\\
0.00137459055078326	0.0216477450434011\\
0.00906748592568987	-0.0114379671139869\\
0.00993330191633668	-0.0150915392409336\\
0.0146835304721823	-0.0353111141780283\\
0.0147644259762568	-0.0356131419209203\\
0.0156984199819879	-0.0400024014506986\\
0.0199057333615782	-0.0717229060422008\\
0.0224997616319643	-0.0920184461500022\\
0.0222825985033807	-0.0958723748781984\\
0.0219252260985353	-0.0962077749676211\\
0.021560596394213	-0.0962760070090079\\
0.0211710807744047	-0.0962760070090079\\
};

\addplot [TpReachStyle]
  table[row sep=crcr]{%
0.0145081289453033	-0.0547169712108607\\
0.00835867427597456	-0.0546018356219568\\
0.00676782475748918	-0.0543712497204673\\
0.00526559965329237	-0.0538303089300391\\
0.00314342087881338	-0.0526551953988953\\
0.00256711883595403	-0.0521749292250125\\
-0.00024404228499382	-0.0476272734906987\\
-0.000719457824063421	-0.0462749121083937\\
-0.002043578336085	-0.044025453009317\\
-0.00342570181708143	-0.0399423998339486\\
-0.00360360955299442	-0.0388471314253409\\
-0.00434429874844165	-0.0365340591940804\\
-0.00449969477719854	-0.0355612384683164\\
-0.00524455354579562	-0.0332362220135147\\
-0.00539994957455255	-0.0322634012877494\\
-0.00611211526671564	-0.0301231897830337\\
-0.00677571485077499	-0.0269235988868962\\
-0.00736637932369492	-0.0231864185111186\\
-0.00785248937051224	-0.021112857115347\\
-0.0082698001459567	-0.018500379983952\\
-0.00880791490224962	-0.0168182004749027\\
-0.0101212639790118	-0.00870774192636078\\
-0.0112180820668806	-0.00215668544488175\\
-0.0112672350007131	-0.00189486917262355\\
-0.0121283856244308	0.00277765320578402\\
-0.0123068412690164	0.00348750797684629\\
-0.0136976758681463	0.011534680898672\\
-0.014146066503633	0.0137500591588546\\
-0.0147869036137398	0.0170347157892959\\
-0.0149351569601953	0.0178339882054026\\
-0.019107385099676	0.0445957364768031\\
-0.0191980279861879	0.0468353798670943\\
-0.0190464820597313	0.0481751005547648\\
-0.01766092421657	0.0523653500903538\\
-0.016552082394101	0.0545433558344185\\
-0.0161861016701111	0.0546851977692969\\
-0.0157927071149775	0.0547169712108598\\
-0.00837184933080559	0.054602871989047\\
-0.00676782475748941	0.0543712497204665\\
-0.0052655996532926	0.0538303089300382\\
-0.00314342087881361	0.0526551953988945\\
-0.00256711883595425	0.0521749292250117\\
0.000244042284993695	0.0476272734906984\\
0.000719457824061881	0.0462749121083966\\
0.00204357833608498	0.0440254530093172\\
0.00342570181708143	0.0399423998339488\\
0.00360360955299422	0.0388471314253436\\
0.00434429874844168	0.0365340591940806\\
0.00449969477719822	0.0355612384683179\\
0.00524455354579567	0.0332362220135148\\
0.00539994957455256	0.0322634012877518\\
0.00611211526671571	0.0301231897830338\\
0.00677571485077506	0.0269235988868964\\
0.00736637932369556	0.0231864185111167\\
0.0078524893705123	0.0211128571153471\\
0.00826980014595756	0.0185003799839497\\
0.00880791490224963	0.0168182004749027\\
0.0101212639790118	0.00870774192636081\\
0.0112180820668806	0.00215668544488175\\
0.0112672350007131	0.00189486917262355\\
0.0121283856244309	-0.00277765320578405\\
0.0123068412690165	-0.00348750797684631\\
0.0136976758681462	-0.0115346808986722\\
0.014146066503633	-0.0137500591588547\\
0.0147869036137397	-0.0170347157892961\\
0.0149351569601951	-0.0178339882054029\\
0.0191073850996757	-0.0445957364768038\\
0.0191980279861877	-0.046835379867095\\
0.019046482059731	-0.0481751005547656\\
0.0176609242165697	-0.0523653500903545\\
0.0165520823941008	-0.0545433558344193\\
0.0161861016701109	-0.0546851977692977\\
0.0157927071149773	-0.0547169712108607\\
0.0145081289453033	-0.0547169712108607\\
};

\addplot [TpReachStyle]
  table[row sep=crcr]{%
-0.020671372484484	0.00445757184640437\\
-0.0206977046069848	0.00666455800271201\\
-0.0204486326292603	0.00900176514238726\\
-0.0203592989092755	0.009479025825029\\
-0.0197235288866211	0.0115943382915005\\
-0.0192732762506744	0.0121747756681902\\
-0.0177021745986918	0.0131904235908428\\
-0.0156614706498244	0.014439585950779\\
-0.0145992699213736	0.0144653689677396\\
-0.0109147570098023	0.0143727267095666\\
-0.00963216110936736	0.0142968382382806\\
-0.00738172614987465	0.0146743431841518\\
-0.00327978627516377	0.0147543328820243\\
-0.00317242253642987	0.0148140675384423\\
0.000378164755681101	0.0163583001208372\\
0.00314219885704939	0.0168598224877329\\
0.00775667934383429	0.0196656481952701\\
0.00823671487365039	0.0197512612595761\\
0.0127314645678778	0.019658619001403\\
0.0142355660969587	0.0193832351019276\\
0.015237931331903	0.0189789411721488\\
0.0163935027974552	0.0182567139047209\\
0.0170799828381867	0.0175131420002346\\
0.0174650495201677	0.0167574454696576\\
0.0176136651189028	0.016243979057984\\
0.0197540430639837	0.00571103219462301\\
0.0203296905239243	0.00104034418595138\\
0.0206713724844828	-0.00445757184640412\\
0.0206977046069836	-0.00666455800271175\\
0.0204486326292597	-0.00900176514238701\\
0.0203592989092749	-0.00947902582502875\\
0.019723528886621	-0.0115943382915002\\
0.0192732762506743	-0.01217477566819\\
0.0177021745986918	-0.0131904235908426\\
0.0156614706498245	-0.0144395859507788\\
0.0145992699213736	-0.0144653689677395\\
0.0109147570098023	-0.0143727267095664\\
0.00963216110936795	-0.0142968382382806\\
0.00738172614987523	-0.0146743431841518\\
0.00327978627516294	-0.0147543328820242\\
0.0031724225364294	-0.0148140675384423\\
-0.000378164755681535	-0.0163583001208372\\
-0.00314219885705182	-0.0168598224877332\\
-0.00775667934383534	-0.0196656481952702\\
-0.00823671487365144	-0.0197512612595762\\
-0.0127314645678788	-0.0196586190014031\\
-0.0142355660969597	-0.0193832351019276\\
-0.015237931331904	-0.0189789411721489\\
-0.0163935027974563	-0.018256713904721\\
-0.0170799828381878	-0.0175131420002347\\
-0.0174650495201687	-0.0167574454696577\\
-0.0176136651189038	-0.0162439790579841\\
-0.0197540430639848	-0.00571103219462292\\
-0.0203296905239254	-0.00104034418595125\\
-0.020671372484484	0.00445757184640437\\
};

\addplot [TpReachStyle]
  table[row sep=crcr]{%
-0.0104699169489793	0.0145418111234907\\
-0.00998833707329662	0.0156053519294782\\
-0.00918563176544218	0.0166051693246536\\
-0.00857937091427875	0.0168700595337231\\
-0.00821937376578262	0.0169038375219609\\
-0.00594903052909413	0.0169030455969059\\
-0.00531791147540251	0.0175171327758799\\
-0.00475139247057453	0.0177377385969104\\
-0.00440107583370008	0.0177697741620566\\
0.00220407585271354	0.0176750583065952\\
0.00461627372218052	0.0174830350160681\\
0.00689660267576561	0.0170487551891247\\
0.00801870563664557	0.0165427127615921\\
0.0104925072743563	0.01513814368302\\
0.0115473709669105	0.0143850633472015\\
0.0121197025613337	0.0136685259778834\\
0.012586773526134	0.0126901747549447\\
0.0129729774337928	0.0111307293789581\\
0.0136109994412173	0.0071785578373456\\
0.0141688195512072	0.00239952056766692\\
0.0142741586336806	0.000488978264324477\\
0.0142274425302096	-0.00169550612281954\\
0.0137947097778219	-0.00597124048982658\\
0.0135147996898603	-0.00767033903471889\\
0.0131828759793231	-0.00974182449046222\\
0.012974312317997	-0.0102969770251974\\
0.012210559221963	-0.0118346240154673\\
0.0117702176424425	-0.0124742358216133\\
0.0115389197119787	-0.0129105507546966\\
0.0113359207389907	-0.0132700922823798\\
0.0104632091943765	-0.0145531800020349\\
0.00998833707329665	-0.0156053519294781\\
0.00918563176544223	-0.0166051693246536\\
0.0085793709142788	-0.0168700595337231\\
0.00821937376578266	-0.0169038375219609\\
0.00594903052909351	-0.0169030455969059\\
0.00531791147540252	-0.01751713277588\\
0.00475139247057455	-0.0177377385969105\\
0.0044010758337001	-0.0177697741620567\\
-0.00220407585273561	-0.0176750583065941\\
-0.00461627372218052	-0.0174830350160681\\
-0.00689660267576563	-0.0170487551891247\\
-0.00801870563664559	-0.016542712761592\\
-0.0104925072743566	-0.0151381436830195\\
-0.0115473709669108	-0.0143850633472011\\
-0.012119702561334	-0.0136685259778829\\
-0.0125867735261342	-0.0126901747549443\\
-0.0129729774337931	-0.0111307293789576\\
-0.0136109994412176	-0.00717855783734513\\
-0.0141688195512075	-0.00239952056766645\\
-0.0142741586336809	-0.000488978264324001\\
-0.0142274425302099	0.00169550612282001\\
-0.0137947097778222	0.00597124048982706\\
-0.0135147996898606	0.007670339034719\\
-0.0131828759793233	0.00974182449046269\\
-0.0129743123179973	0.0102969770251978\\
-0.0122105592219633	0.0118346240154678\\
-0.0117702176424428	0.0124742358216138\\
-0.011538919711979	0.0129105507546971\\
-0.0113359207389909	0.0132700922823803\\
-0.0104699169489793	0.0145418111234907\\
};

\addplot [SimStyle]
  table[row sep=crcr]{%
0.2	0.02\\
0.184504437274484	0.0211379532903241\\
0.170230565925215	0.0219655004113037\\
0.157085679894193	0.0225020655443608\\
0.15611879941742	0.0225318136090499\\
0.144431766037449	0.0227717868769342\\
0.133340255415055	0.0227676861526314\\
0.123748550202776	0.0225460659561534\\
0.122103946262969	0.022484900016944\\
0.113493461953779	0.0220390520620302\\
0.105528355313415	0.0214138269809736\\
0.0981629908007207	0.0206193706742824\\
0.09583372401959	0.0203186077649033\\
0.0892028312631938	0.0193132321724889\\
0.0830777265041199	0.0181600177913777\\
0.0774224067065235	0.0168672559339209\\
0.0756358405398259	0.0164066852439069\\
0.0705554645472224	0.0149399089544947\\
0.065870736174944	0.0133511006943459\\
0.061553241607652	0.0116470601938578\\
0.0601910622993143	0.0110545562967837\\
0.0561653805432445	0.00905535311083058\\
0.0524624286976606	0.00695119082082521\\
0.0490601562389473	0.00474702226792759\\
0.0488102921916602	0.00457346718925117\\
0.0457087677101608	0.00226696626723066\\
0.0428663212167187	-0.000129723224473299\\
0.0402644780932982	-0.00261211838312381\\
0.0400737973597733	-0.00280649717147058\\
0.0376222190516661	-0.0054802633592046\\
0.0353169912074395	-0.008336949522743\\
0.0334470673583023	-0.0109550993855926\\
0.031413127519466	-0.0141849884199366\\
0.0295928056953356	-0.0175010819105371\\
0.02848858503443	-0.0197571060216771\\
0.0268891292355999	-0.0234379145611364\\
0.0253960561274738	-0.027439114423747\\
0.0248380091055156	-0.0291122277080189\\
0.0259327947297885	-0.0330857074301912\\
0.0267825131820254	-0.0370718747722044\\
0.0269680964611641	-0.0381413398189492\\
0.0274957076473724	-0.042065254234326\\
0.0277836577020621	-0.0459724797459314\\
0.0277916784184058	-0.0461622097904082\\
0.0278366603547043	-0.0500307615302302\\
0.0276988257356682	-0.0533525099681669\\
0.0273088503300613	-0.0574906637071647\\
0.0269823633003392	-0.0598582039315144\\
0.0261993091603675	-0.064215850958527\\
0.0258605325547589	-0.0657994474119522\\
0.0242460818327494	-0.0713064317979495\\
0.0226487050338754	-0.0764739933952038\\
0.0210926872389353	-0.0813500599643188\\
0.0195996567206727	-0.0859779060010487\\
0.0181865752984767	-0.0903956717119723\\
0.0221179745501109	-0.0928155560234349\\
0.0231862971354754	-0.0933792396106567\\
0.024619970442357	-0.0939456481690125\\
0.0253503071719835	-0.0940006124024861\\
0.0254809749946184	-0.0939622198866847\\
0.0257893453782083	-0.093673552442733\\
0.0258893141492803	-0.0931069925439332\\
0.0258263951769754	-0.0925293312708456\\
0.0254569450144572	-0.0911642285865321\\
0.0248090279648397	-0.0894182723585229\\
0.0229357736331361	-0.0850494809161688\\
0.0228794008081424	-0.0849223060319484\\
0.0203037353854707	-0.0793074357014832\\
0.0174606278600343	-0.0727947908470568\\
0.0145171311484733	-0.0655567989962566\\
0.011665763283855	-0.0579402708919472\\
0.0115937147552871	-0.0577387694454356\\
0.00891156122463177	-0.0498850227874442\\
0.00877408953805509	-0.0494619065086442\\
0.00380199778497881	-0.0452354233395142\\
3.07843704884792e-05	-0.0417710160777071\\
-0.00344956958538872	-0.0382914319344492\\
-0.00601801101412991	-0.0354719128303316\\
-0.0085338498432084	-0.0323915407158757\\
-0.0100199923657499	-0.0303404966117005\\
-0.0116552025360175	-0.027742028631453\\
-0.0124836730240206	-0.0261872987725677\\
-0.01342283543326	-0.0240161140440293\\
-0.0138064616342387	-0.0228523230740502\\
-0.0138118443450936	-0.0201023078301185\\
-0.0135976440988036	-0.0178361489954385\\
-0.0131935792769318	-0.015993984044212\\
-0.0131543053295961	-0.0158629515582752\\
-0.0125581746059304	-0.0143338134834675\\
-0.011797990083687	-0.0130603412887924\\
-0.0110634189166769	-0.0120077205090019\\
-0.0103407418044177	-0.0110336674104641\\
-0.0096503263998089	-0.0101297967937769\\
-0.00899947282044303	-0.00929151891702357\\
-0.00838907024386648	-0.00851602720354161\\
-0.0083266274227165	-0.00800908929429062\\
-0.00826268320656062	-0.00788447636207881\\
-0.00782417152688311	-0.00742183362738408\\
-0.00724402309468319	-0.00706947940362596\\
-0.00661592073757819	-0.00679292921607152\\
-0.00599003013886154	-0.00657151054038141\\
};\label{p:Sim}

\addplot [SimStyle]
  table[row sep=crcr]{%
-0.2	-0.02\\
-0.180740347890004	-0.02159639276769\\
-0.163360547001997	-0.022807040319923\\
-0.159735590376631	-0.0230267774071845\\
-0.145225140540076	-0.0237721088916303\\
-0.132067722109496	-0.0242290742036585\\
-0.127724039471873	-0.0243255454328071\\
-0.116541789526169	-0.0244265524667791\\
-0.106077419160304	-0.0242932542510011\\
-0.102332230403176	-0.024181896280334\\
-0.0935203097755443	-0.0237626683011398\\
-0.0857050488866968	-0.0231725908336232\\
-0.0822437180688317	-0.0228332742549556\\
-0.0754482386357184	-0.0220018834070228\\
-0.069271153302153	-0.0210207810786425\\
-0.0663977668068387	-0.0204773414636376\\
-0.0612300442471412	-0.0193619288454336\\
-0.056534017120289	-0.0181239328365546\\
-0.0536446651759862	-0.0172350904688259\\
-0.0496439285878666	-0.0158129153667712\\
-0.0460116239672371	-0.0142910838332208\\
-0.0437786461585389	-0.0132248100117117\\
-0.0404942263316123	-0.0114384611800022\\
-0.0375354452587854	-0.00955996010306753\\
-0.0361685385471439	-0.00858838688899208\\
-0.033641849259444	-0.00658473951473282\\
-0.0312354714501678	-0.00437321904264187\\
-0.0303213390571039	-0.00343942030726377\\
-0.0282724418385687	-0.00111899023826664\\
-0.0263454341686406	0.00141387648832178\\
-0.0258508795230962	0.00213043914344727\\
-0.0248374696157529	0.00737769498834243\\
-0.0247794184192791	0.00764573105125074\\
-0.0235663742341731	0.0127214747769381\\
-0.0222626596329418	0.0174136556613239\\
-0.0209151330328866	0.0217742722569989\\
-0.019563340007342	0.0258501349238076\\
-0.0182579018847593	0.0296893955273668\\
-0.01701570926012	0.0333299736030457\\
-0.0158421513037969	0.0367976883519609\\
-0.0147421091315218	0.0401159309011176\\
-0.0137189191481447	0.0433054547660656\\
-0.0186995036052553	0.0455295607986266\\
-0.0205767412339555	0.0461483237236064\\
-0.0210684242313124	0.0462127026388854\\
-0.0215316544755535	0.0460926121854498\\
-0.0216595081162226	0.0457758530297074\\
-0.0215156610548692	0.0451432912631791\\
-0.0210438211114712	0.0441641946131918\\
-0.0196406767874205	0.0418464929230589\\
-0.017671927521669	0.0393133507988802\\
-0.0173886969760489	0.0389202341712799\\
-0.0151644696469542	0.0355989295941447\\
-0.0128074547434091	0.031649605269484\\
-0.0105182132803055	0.0273864535172922\\
-0.00827014284676797	0.0227580291330297\\
-0.00439568182142644	0.0195994289125459\\
-0.00298358870869789	0.0183622483785293\\
-0.000108280370654967	0.0156355150198337\\
0.000804088607669784	0.0146918396828658\\
0.00289186054999158	0.0123114379260028\\
0.00342395405749482	0.0116356983720306\\
0.00486427927170743	0.00956725126841096\\
0.00514906367123019	0.00909659338980964\\
0.00607787468120638	0.00728217922168217\\
0.00619989911962293	0.00699107726874154\\
0.00672594106742788	0.00540921994809118\\
0.00676368456619003	0.0052497388279813\\
0.00695727936343424	0.00384627528488021\\
0.00689335356116341	0.002632422388333\\
0.00665053705515833	0.00166419503762361\\
0.00629214146295126	0.00087437312324104\\
0.00570778456537235	0.000266387341390645\\
0.00509605554339668	-0.000164420042838342\\
0.00448851195108185	-0.000454633243854502\\
0.0039063322621222	-0.000634290479397448\\
0.00336290584714838	-0.000727859041596057\\
0.00293228352311733	-0.000743134681448654\\
0.00250748578562004	-0.000690369419980019\\
0.00210760679380109	-0.000589504052777634\\
0.00174316862409393	-0.000455497391834925\\
0.00141889771152248	-0.00029946251214133\\
};

\addplot [SimStyle]
  table[row sep=crcr]{%
-0.2	-0.02\\
-0.18681828710919	-0.0185358599118461\\
-0.174580155200929	-0.016948949045486\\
-0.164790821505639	-0.0154969893135984\\
-0.154131945407785	-0.0137025243171885\\
-0.144238238375703	-0.0118064426737364\\
-0.136325873941236	-0.0101053119194847\\
-0.127712493650289	-0.00803752012203798\\
-0.11971927849962	-0.00588569368903941\\
-0.113328203081399	-0.00397850490813531\\
-0.106372421647156	-0.0016847345685998\\
-0.0999190137754278	0.00067847850902486\\
-0.0947602936816095	0.00275594476411767\\
-0.0891865239564354	0.00521809065920278\\
-0.0839771554949942	0.00775549721995536\\
-0.0797795814795132	0.00999230095733872\\
-0.075011383816288	0.0128131240762657\\
-0.0705998519185842	0.015681313790403\\
-0.0678445371139977	0.0176185004093674\\
-0.0639723796379429	0.0205600324974747\\
-0.0601645205703455	0.0237425430781443\\
-0.0581594945009553	0.0255520243658507\\
-0.054822184902201	0.0288014004370341\\
-0.0517563628394674	0.0320901675572857\\
-0.0503186226364022	0.0337485329732642\\
-0.0474598559285983	0.0373017958941954\\
-0.0447036183750397	0.0411048864623007\\
-0.0439841386856952	0.0421683377299742\\
-0.041548728221643	0.046020651098766\\
-0.0392225066795287	0.0501251757771033\\
-0.038876623968711	0.0507766405928237\\
-0.0393924489177484	0.0539246451602406\\
-0.039673717565564	0.0570482907386492\\
-0.0397237249960305	0.0587056048131924\\
-0.0396315631635852	0.0616526924201682\\
-0.0393025695194099	0.0645291170695548\\
-0.0391741685059424	0.0652862992312782\\
-0.0385592496714353	0.0680082873692002\\
-0.0377260303572008	0.0706254079268567\\
-0.0376800658827463	0.0707499901296137\\
-0.0366348151977665	0.0732369217680476\\
-0.0355781697310856	0.0752873451460079\\
-0.0341316895929903	0.0776318507125896\\
-0.0331169988344119	0.0790559210751002\\
-0.0315779482119124	0.0810720298280382\\
-0.0306161247062667	0.0821673831902751\\
-0.0286458428740607	0.0841295877017951\\
-0.0279828516618214	0.0847219353978008\\
-0.0256704968059321	0.0865806261569113\\
-0.0253434184166959	0.0868210806264861\\
-0.0227813983542808	0.0885468743214043\\
-0.0203499556622882	0.0899660565427937\\
-0.022470175315924	0.0902808907419578\\
-0.0240011249713552	0.0902887001130735\\
-0.0249352032427808	0.0901143437974552\\
-0.0257177787064574	0.0897492841175282\\
-0.0262602771294826	0.0892203656772415\\
-0.0265967576264358	0.0885343872298638\\
-0.0266649510606864	0.0882813044268317\\
-0.0267612291229466	0.0872411844783013\\
-0.0266108712612863	0.0858268398564386\\
-0.0263927130844496	0.0848281201034833\\
-0.0256683541866343	0.0824633932502317\\
-0.0247492709506939	0.0800339847860868\\
-0.0226942933007248	0.0752249435973262\\
-0.0221982079091224	0.0741162223124409\\
-0.0198442874595057	0.0699704883338422\\
-0.018468146034947	0.0673547324296049\\
-0.0157689197051647	0.0618389401560824\\
-0.0148799125222776	0.0599178779322694\\
-0.0119806368918813	0.0532970917213043\\
-0.0113713078960575	0.0518388484840476\\
-0.00836304813348185	0.044279261132429\\
-0.00793821069868594	0.0431640227767622\\
-0.00501139882418564	0.0351240456580628\\
-0.00459976674552515	0.0339432864397254\\
-3.33937399406847e-05	0.0297663245978794\\
0.00392997811304724	0.0258449638865559\\
0.00441102527854492	0.0253450955827293\\
0.00774378406572387	0.0216923241767596\\
0.0105174715493371	0.0183134340471464\\
0.012705734994071	0.0153034530546302\\
0.0144184654577732	0.0125921338498913\\
0.0157766746917442	0.0100383372020849\\
0.0166721570373789	0.00796382283455341\\
0.0173971562441207	0.0057251465811895\\
0.0177179882179041	0.00424575141868194\\
0.0178386631109421	0.0023075480517753\\
0.0177959705921829	0.00126682450581947\\
0.0175412128101739	-0.000386079802993877\\
0.0173452409824046	-0.00110689481760576\\
0.0167821259465125	-0.00250290103496445\\
0.0165292875919164	-0.00296866864672485\\
0.0157008941923377	-0.00414361382546016\\
0.0154765004427555	-0.00439842305017801\\
0.0144390324131272	-0.00534869105873684\\
0.0142852804867377	-0.00546402912138585\\
0.0124434735190499	-0.0061882388398817\\
0.010640647060982	-0.00666609378867755\\
0.00903945196295655	-0.00692046861061502\\
0.00760525058276515	-0.00701955347488872\\
0.00633898434132171	-0.00700531689563047\\
0.00523478506831737	-0.00691040001572116\\
0.00428244218695839	-0.00675988751980672\\
0.00346921876454812	-0.00657279327316815\\
0.00278116525452671	-0.0063632971922899\\
0.00220404813719008	-0.00614176712919964\\
};


\addplot [SimStyle]
  table[row sep=crcr]{%
-0.2	0.02\\
-0.176147417046664	0.0229017012419008\\
-0.170622572987395	0.0235396727716698\\
-0.148949061923664	0.0259025448629765\\
-0.145388964683755	0.0262675270472939\\
-0.125759196693916	0.0281455229616753\\
-0.123741359640033	0.0283245056496065\\
-0.106484201715058	0.0297308077566857\\
-0.105190444505777	0.0298264309064845\\
-0.0899433586223195	0.0308322040103285\\
-0.0893088707530692	0.0308687542489582\\
-0.0774973879832374	0.0314466503690857\\
-0.0759318092662082	0.0315053690779944\\
-0.0652163335909934	0.0317789024106682\\
-0.0644113817265012	0.0317896209397046\\
-0.0547571201139982	0.0317933655049038\\
-0.0545141167727823	0.0317902147029676\\
-0.0460285703666218	0.031561843897585\\
-0.0387659207158114	0.031148371659887\\
-0.033151572706703	0.0306120175613392\\
-0.0327365730721066	0.0305612107094681\\
-0.0283902493212762	0.0299052492606157\\
-0.0276854586882744	0.0297740993637303\\
-0.0240975808863102	0.0289710337708014\\
-0.0233941095763838	0.0287832022763518\\
-0.0202290637814515	0.0277872560841906\\
-0.019710638901225	0.0275975588913096\\
-0.0169754436050629	0.0264480983157328\\
-0.0165261574022192	0.0262326253023646\\
-0.0142820110583781	0.0246082606604376\\
-0.0120128976561081	0.0227044265493102\\
-0.0098465611403307	0.0206031295303027\\
-0.00784913839535864	0.0183610655853599\\
-0.00604905195569089	0.0160178776550687\\
-0.00354618458087091	0.0140697397950768\\
-0.00324626043208695	0.0138119585506371\\
-0.00139034139254157	0.0120388991192458\\
-0.00126475010141819	0.0119045612750908\\
9.32417958121046e-05	0.0102669833660056\\
0.00107772093171482	0.00872966977245196\\
0.00172090057443183	0.00737282902829256\\
0.00180508170439225	0.00606486591342562\\
0.00197050038123631	0.00475778556748194\\
0.00215196165080456	0.00347427923464894\\
0.00231590722662581	0.00222599080957131\\
0.0024474393518234	0.00101785064740512\\
0.00327150419931918	-2.54533171763216e-05\\
0.00371562565927536	-0.000815649521596157\\
0.00388929426095905	-0.00139771038395656\\
0.00387390435923257	-0.00180788483012845\\
0.00372983732789761	-0.00207559826790468\\
0.00341496983123704	-0.00224032974749094\\
0.00311024220543224	-0.00232799059452024\\
0.00281146664974979	-0.0023448109532431\\
0.00251869420213502	-0.00229788069053163\\
0.00223382604939168	-0.00219436609698764\\
0.00172059485810849	-0.00208536466416875\\
0.00133182177856103	-0.00201010979873345\\
0.00103704955351053	-0.00195978812357822\\
0.000813332129577404	-0.00192774043984512\\
0.000643325776016418	-0.00190892155550448\\
0.000834719235962067	-0.00184099493632595\\
0.000820342721427503	-0.00169926112277441\\
0.000704399652341703	-0.00152030341968279\\
0.000546971974035998	-0.00132623498600165\\
0.000381567659612808	-0.00113027079952846\\
};

\addplot [SimStyle]
  table[row sep=crcr]{%
-0.0859567657365433	0.0131092869379305\\
-0.0692295671942092	0.0108967069146765\\
-0.0558188387060543	0.00911943958406275\\
-0.0450390002410243	0.00767745463597952\\
-0.0363566530851049	0.00649650297953956\\
-0.0293532057213505	0.005520578631141\\
-0.0241575821317658	0.00461889824936836\\
-0.0197049002763694	0.00372454414620338\\
-0.0159462744073983	0.002869599143075\\
-0.0128093310776865	0.00206970807282117\\
-0.0102142215154049	0.00133073402441641\\
-0.00789058459959059	0.000683658022707034\\
-0.00602069140340085	0.000154674764267226\\
-0.00453351689360179	-0.000264810959367992\\
-0.00336379115817602	-0.00058514569733098\\
-0.00245395454960509	-0.000817037772914073\\
-0.0012650999125013	-0.000882565450953554\\
-0.000596238557640155	-0.000756416528623974\\
-0.000251168867977034	-0.000504739793464951\\
-0.000104089198883078	-0.000171410187233448\\
-7.46752669904954e-05	0.000214302968966545\\
-0.000327258323892091	0.000595588127912425\\
-0.000530337662476399	0.000927345994709367\\
-0.000681266594614732	0.00120867197797836\\
-0.000783533708963904	0.00144126284926178\\
-0.000843476573764296	0.00162819321717747\\
-0.000866968260943263	0.00177297794068527\\
-0.000836889912147684	0.00195149487986232\\
-0.000656129917912834	0.00200115438861345\\
-0.000475981236850151	0.00196588937132422\\
-0.000327438026838892	0.00191649888320407\\
-0.000149604901994504	0.00189357114022733\\
-0.000241524417733952	0.0018150300997616\\
-0.000179610909554734	0.00164311012479998\\
-0.000329114592346982	0.0014899583088747\\
-0.000487361175973935	0.00137350392546422\\
-0.000549095255967663	0.00109756832997346\\
-0.000455725700445664	0.000769693018379244\\
-0.000292158052208891	0.000421363801205929\\
-0.000107164742825155	7.18617741371358e-05\\
};

\addplot [SimStyle]
  table[row sep=crcr]{%
0.0707484373753676	-0.0116958786248008\\
0.060138789883928	-0.0129786138001858\\
0.0509770898043276	-0.0139790374867442\\
0.0431171628770767	-0.0147702167882559\\
0.0364069001211562	-0.0154046349501493\\
0.0306994063469284	-0.015920551624289\\
0.0256564876475489	-0.0163896208518153\\
0.0215206316487446	-0.0168489447851872\\
0.0181038079451538	-0.0172854103941239\\
0.0152648431318456	-0.0176933890209845\\
0.012895625723888	-0.0180715683347151\\
0.0111193806739369	-0.0183911607197976\\
0.00966354447514024	-0.0186089671543269\\
0.00842885781330971	-0.0187157198897585\\
0.00735497345779001	-0.018712153215043\\
0.00640492831754466	-0.018604334805094\\
0.00555563304728246	-0.018400858874194\\
0.00479210929019032	-0.0181111989185447\\
0.00410404574312932	-0.0177447766572604\\
0.00348377691660137	-0.0173104681405833\\
0.00292512518498707	-0.0168163722273265\\
0.00287989329009497	-0.0161905960542573\\
0.00264085163341084	-0.0153902221559524\\
0.00229306755716582	-0.0144559133820815\\
0.00189224968315933	-0.0134178788808401\\
0.00147432336491289	-0.0122988459245891\\
0.00144354483850175	-0.01125055152336\\
0.00141663049563363	-0.0110533487091997\\
0.00113069700654991	-0.00966889775665783\\
0.000735733345554379	-0.00819631496296008\\
0.000303940214516338	-0.00666974628287993\\
-0.000122568677560428	-0.00511212989858692\\
-0.0017394274226479	-0.0037415575804107\\
-0.00270539877213249	-0.00270367631191319\\
-0.00322378178419122	-0.00192484108504286\\
-0.00343735697962097	-0.00134987060446171\\
-0.0034467274514996	-0.000936792385886967\\
-0.00300625641838841	-0.000595686062761863\\
-0.00271038548301848	-0.000278469842723589\\
-0.00248929372410575	-2.72282764761378e-06\\
-0.00230407908092696	0.000223737556295164\\
-0.00213425844889831	0.000398850204849949\\
-0.00203742410653447	0.000510980047234835\\
-0.00185559716620586	0.000567362954374692\\
-0.00164012120380083	0.000588829432213958\\
-0.00142023467918381	0.000588576388299358\\
-0.00121166925904143	0.000574948723421925\\
-0.00135200557154821	0.000493829117389585\\
-0.00127693025531382	0.000325262874128054\\
-0.00109622799135189	0.000110157282422793\\
-0.000874513639129576	-0.000125808302904887\\
-0.000648470068926488	-0.00036666906454888\\
};


\addplot [SimStyle]
  table[row sep=crcr]{%
-0.132093173589405	-0.0140937689139305\\
-0.121033448455714	-0.0140516438868069\\
-0.111278152607465	-0.0138003353283447\\
-0.10555193042063	-0.013541504789125\\
-0.0971305206841511	-0.0129832074111849\\
-0.0894381783738476	-0.0122538461325443\\
-0.0846850088926051	-0.0116784551500104\\
-0.0780760020286499	-0.0106898931270353\\
-0.0720455449915741	-0.00956000541267771\\
-0.0683227569584518	-0.00873306694776133\\
-0.0631514772574991	-0.0073889224259916\\
-0.0584387218160472	-0.00592789980050681\\
-0.0555325657501347	-0.00489293315140046\\
-0.0515002369697813	-0.00325464578516815\\
-0.0478307280951001	-0.00151962480223108\\
-0.0455707681489558	-0.000312487581835386\\
-0.0424862919538696	0.00152342430733576\\
-0.0396128517071107	0.00348484479764188\\
-0.0377848117424497	0.00488422997758231\\
-0.0351938508093015	0.00711199564244669\\
-0.0328615017659655	0.00941573713006805\\
-0.0317848270138003	0.0105943522080699\\
-0.0296792289421882	0.0131544011099509\\
-0.0276976387289801	0.0159419269540148\\
-0.0271889014786407	0.0167294110054769\\
-0.0254954414561769	0.0196087939711214\\
-0.0239230173019026	0.0227187853814408\\
-0.0236933305045416	0.0232160594754798\\
-0.0222152881823779	0.0267413706641144\\
-0.0210570674046259	0.0299934977898034\\
-0.0220969166647881	0.0328242947772387\\
-0.0228755855832205	0.0356096566559359\\
-0.0230401030015023	0.0363475344741849\\
-0.02348603229319	0.0390201693081191\\
-0.0236896216367191	0.041626366048327\\
-0.0236935506520745	0.0417514987993557\\
-0.0236570799645855	0.0442737637845517\\
-0.0234542995106167	0.0463944755909158\\
-0.0229937668927153	0.0489768732844719\\
-0.0226334860487351	0.0504240722391054\\
-0.0217719809048553	0.0531148163910601\\
-0.0214524220663435	0.0539568750517026\\
-0.0195631220591338	0.0571733976679822\\
-0.0179326588837073	0.0601834008357629\\
-0.0164772108643358	0.0629906461365194\\
-0.0151511854757993	0.06560847532657\\
-0.0139305950632633	0.0680546201304883\\
-0.0168146637031513	0.0690916653115207\\
-0.0183752596877549	0.0694668247425795\\
-0.0195543942774344	0.0695340965202073\\
-0.0202811458111255	0.0693324087333702\\
-0.0204886326287488	0.0691878283526052\\
-0.0208018548464089	0.0687444102375692\\
-0.0209357491640565	0.0680901509939514\\
-0.0209159917040242	0.0674993829255746\\
-0.0206574447226141	0.0662004422101202\\
-0.0201533994263563	0.064632192084005\\
-0.0188993228840475	0.0615145045042709\\
-0.0185785081634856	0.0607753077993236\\
-0.0178973310350108	0.0583765127678136\\
-0.0170495845849146	0.0559801402277391\\
-0.0150266863284983	0.0509345533773941\\
-0.0147774504954392	0.0503375753156501\\
-0.0121241466739652	0.0440229129598379\\
-0.00932174869061109	0.0371657705045347\\
-0.00651771230248738	0.0298644411487541\\
-0.00206145313289999	0.0263210850579772\\
0.00158457542314744	0.0231577593200231\\
0.00476394002102359	0.020107547902814\\
0.00690857478848961	0.0178139589493708\\
0.00902006154271429	0.0152507280195218\\
0.0102241610214711	0.0135626672264962\\
0.0114926193921452	0.0114461149707412\\
0.0120998015804876	0.0101918468575713\\
0.012732456234261	0.00845615221605522\\
0.0129556134075588	0.00753375831368294\\
0.0125610527170099	0.00535572202877377\\
0.0120803326736612	0.00351219049409437\\
0.0114961814848114	0.00198174124795492\\
0.0108201855717316	0.00073500041610508\\
0.0100759938528226	-0.000259815240948369\\
0.00899458066419057	-0.00105162931380703\\
0.00798820509888837	-0.00168610432383992\\
0.00705821815986418	-0.0021878288162594\\
0.00620543337988386	-0.00257840379114396\\
0.0054294739556103	-0.00287645634270434\\
0.00472498506017136	-0.00309659426717943\\
0.00409094578112693	-0.00325190086251706\\
0.00352540834469206	-0.00335449997975035\\
0.00302469755219262	-0.00341415886966578\\
0.0025841168809046	-0.00343873838172712\\
};

\addplot [ShiftX0Style]
  table[row sep=crcr]{%
-0.2	-0.02\\
0.2	-0.02\\
0.2	0.02\\
-0.2	0.02\\
-0.2	-0.02\\
};\label{p:ShiftX0}

\addplot [FinalReachStyle]
  table[row sep=crcr]{%
-0.0113937453780825	0.014365845328073\\
-0.0110382385381966	0.0150986141716483\\
-0.0100874350871123	0.0163242170636218\\
-0.00876947840612268	0.0172279587317393\\
-0.00819276931600349	0.0173184354583759\\
-0.00593187682730112	0.017317659474375\\
-0.00503152905519551	0.0179542446741269\\
-0.00445481996507631	0.0180447214007634\\
0.00263693735523159	0.0179307350069322\\
0.00442533230774687	0.0177216142474598\\
0.00690968361624233	0.0171841321318443\\
0.00827790788963038	0.0165317940286661\\
0.0111199685343711	0.0148962816878184\\
0.0117410377082597	0.0142381234439431\\
0.0124230269744333	0.0131003430973164\\
0.0127861637452152	0.0120040317154173\\
0.0131988252489762	0.00988267781665972\\
0.0138069022150374	0.00572480881783076\\
0.0142307756186794	0.00162472033798636\\
0.0142733131405985	-0.00111989323951281\\
0.013982507650325	-0.0055519980063076\\
0.0138361084174179	-0.00653225912883549\\
0.0136855804923009	-0.00721947799519804\\
0.0133715620125582	-0.00995253915716377\\
0.0130920259760394	-0.0109867382669174\\
0.0120549324821447	-0.0137179466224012\\
0.0115520907045045	-0.0141857599516117\\
0.0113937453780823	-0.0143658453280725\\
0.0110382385381964	-0.0150986141716478\\
0.0100874350871123	-0.0163242170636217\\
0.00876947840612271	-0.0172279587317393\\
0.00819276931600351	-0.0173184354583758\\
0.00593187682730094	-0.0173176594743758\\
0.00503152905519551	-0.017954244674127\\
0.00445481996507631	-0.0180447214007635\\
-0.00263693735523159	-0.0179307350069323\\
-0.00442533230774687	-0.0177216142474599\\
-0.00690968361624235	-0.0171841321318443\\
-0.0082779078896304	-0.0165317940286661\\
-0.0111199685343713	-0.014896281687818\\
-0.0117410377082599	-0.0142381234439428\\
-0.0124230269744335	-0.013100343097316\\
-0.0127861637452154	-0.0120040317154169\\
-0.0131988252489764	-0.00988267781665933\\
-0.0138069022150376	-0.00572480881783037\\
-0.0142307756186796	-0.00162472033798598\\
-0.0142733131405987	0.00111989323951319\\
-0.0139825076503252	0.00555199800630798\\
-0.0138361084174181	0.00653225912883587\\
-0.0136855804923016	0.00721947799519217\\
-0.0133715620125584	0.00995253915716417\\
-0.0130920259760396	0.0109867382669178\\
-0.0120549324821449	0.0137179466224016\\
-0.0115520907045047	0.0141857599516122\\
-0.0113937453780825	0.014365845328073\\
};\label{p:FinalReach}
		\coordinate (top) at (rel axis cs:0,1);
		\nextgroupplot[title={$[\dot{\Psi}, ~ v]$}]
		\addplot[IntReachStyle]
table[row sep=crcr] {%
0.614635897069773	19.2793175756258\\
0.6093613001547	19.2793175756258\\
0.609361300154708	19.2749982841882\\
0.5722294718053	19.2749982841882\\
0.572229471805337	19.2706853939321\\
0.5374811874691	19.2706853939321\\
0.537481187469156	19.2663755310615\\
0.5050517342558	19.2663755310615\\
0.505051734255869	19.262069011804\\
0.4748271566327	19.2620690118041\\
0.47482715663273	19.257765709593\\
0.4467115203329	19.257765709593\\
0.446711520332926	19.2534655916282\\
0.4206135679977	19.2534655916283\\
0.420613567997702	19.2491685668717\\
0.3962926677186	19.2491685672252\\
0.396292667718679	19.2448734842487\\
0.3734515528489	19.2448734842488\\
0.373451552848955	19.2405657371264\\
0.3521540597606	19.2405657371264\\
0.352154059760681	19.2362440539033\\
0.3322751337285	19.2362440539033\\
0.332275133728576	19.2319154643496\\
0.3136491310381	19.2319154643497\\
0.313649131038128	19.2275719752127\\
0.2963413716162	19.2275719752128\\
0.296341371616253	19.2232135160623\\
0.2802726435443	19.2232135160623\\
0.280272643544371	19.2188436360438\\
0.2651970965658	19.2188436360438\\
0.265197096565826	19.2144641940662\\
0.2510018967903	19.2144641940663\\
0.251001896790398	19.2100776256262\\
0.2377151263197	19.2100776256263\\
0.237715126319759	19.2056859577181\\
0.2252397694854	19.2056859577182\\
0.225239769485403	19.201287472152\\
0.2135785112149	19.201287472152\\
0.213578511214927	19.1968844265782\\
0.2026134863251	19.1968844265782\\
0.202613486325124	19.1924755539982\\
0.1923254326731	19.1924755539982\\
0.192325432673186	19.1880641111756\\
0.1826126751929	19.1880641111756\\
0.18261267519295	19.183650573507\\
0.1734824441258	19.1836505735071\\
0.173482444125892	19.1792350531396\\
0.164928744297	19.1792350531397\\
0.164928744297011	19.1748200361116\\
0.1569105984152	19.1748200361116\\
0.156910598415268	19.1704063624161\\
0.1493467603566	19.1704063624162\\
0.149346760356603	19.16599261966\\
0.1422556252469	19.16599261966\\
0.142255625246953	19.1615780003885\\
0.1355107388692	19.1615780003886\\
0.13551073886921	19.1571593208933\\
0.1291495581915	19.1571593208933\\
0.129149558191512	19.1527416110479\\
0.1231719775553	19.1527416110479\\
0.123171977555356	19.1483248821344\\
0.1172750076668	19.1483248821345\\
0.117275007666882	19.1480818204204\\
-0.117275007666889	19.1480818204206\\
-0.1172750076668	19.1483248821347\\
-0.123171977555365	19.1483248821346\\
-0.1231719775553	19.1527416110481\\
-0.12914955819152	19.152741611048\\
-0.1291495581915	19.1571593208934\\
-0.135510738869218	19.1571593208934\\
-0.1355107388692	19.1615780003887\\
-0.142255625246961	19.1615780003887\\
-0.1422556252469	19.1659926196602\\
-0.149346760356611	19.1659926196601\\
-0.1493467603566	19.1704063624163\\
-0.156910598415276	19.1704063624163\\
-0.1569105984152	19.1748200361117\\
-0.164928744297018	19.1748200361117\\
-0.164928744297	19.1792350531398\\
-0.173482444125899	19.1792350531397\\
-0.1734824441258	19.1836505735072\\
-0.182612675192957	19.1836505735071\\
-0.1826126751929	19.1880641111757\\
-0.192325432673192	19.1880641111757\\
-0.1923254326731	19.1924755539983\\
-0.202613486325132	19.1924755539982\\
-0.2026134863251	19.1968844265782\\
-0.213578511214935	19.1968844265782\\
-0.2135785112149	19.201287472152\\
-0.22523976948541	19.201287472152\\
-0.2252397694854	19.2056859577182\\
-0.237715126319765	19.2056859577181\\
-0.2377151263197	19.2100776256263\\
-0.251001896790403	19.2100776256262\\
-0.2510018967904	19.2144641940663\\
-0.26519709656583	19.2144641940662\\
-0.2651970965658	19.2188436360438\\
-0.280272643544375	19.2188436360438\\
-0.2802726435443	19.2232135160623\\
-0.296341371616255	19.2232135160623\\
-0.2963413716162	19.2275719752128\\
-0.313649131038129	19.2275719752127\\
-0.3136491310381	19.2319154643496\\
-0.332275133728577	19.2319154643495\\
-0.3322751337285	19.2362440539033\\
-0.352154059760681	19.2362440539032\\
-0.3521540597606	19.2405657371264\\
-0.373451552848954	19.2405657371263\\
-0.3734515528489	19.2448734842487\\
-0.396292667718678	19.2448734842486\\
-0.3962926677186	19.249168578746\\
-0.4206135679977	19.2491685790995\\
-0.4206135679976	19.2534655916282\\
-0.446711520332921	19.2534655916281\\
-0.4467115203329	19.2577657095929\\
-0.474827156632724	19.2577657095929\\
-0.4748271566327	19.2620690118039\\
-0.505051734255863	19.2620690118039\\
-0.5050517342558	19.2663755310613\\
-0.53748118746915	19.2663755310613\\
-0.5374811874691	19.2706853939319\\
-0.572229471805328	19.2706853939319\\
-0.5722294718053	19.274998284188\\
-0.609361300154701	19.274998284188\\
-0.6093613001547	19.2793175756256\\
-0.614635897069764	19.2793175756256\\
-0.614635897069774	20.7207211876469\\
-0.6093613001547	20.7207211876468\\
-0.609361300154709	20.7250412345763\\
-0.5722294718053	20.7250412345762\\
-0.572229471805336	20.7293549220391\\
-0.5374811874691	20.7293549220391\\
-0.537481187469158	20.7336656018952\\
-0.5050517342558	20.7336656018951\\
-0.50505173425587	20.7379729560281\\
-0.4748271566327	20.737972956028\\
-0.47482715663273	20.7422771102506\\
-0.4467115203329	20.7422771102505\\
-0.446711520332925	20.746578095753\\
-0.4206135679977	20.746578095753\\
-0.420613567997702	20.7508760027325\\
-0.3962926677186	20.7508760027325\\
-0.39629266771868	20.755171982622\\
-0.3734515528489	20.755171982622\\
-0.373451552848955	20.7594806413913\\
-0.3521540597606	20.7594806413913\\
-0.352154059760681	20.7638032514944\\
-0.3322751337285	20.7638032514944\\
-0.332275133728576	20.7681327830784\\
-0.3136491310381	20.7681327830784\\
-0.313649131038128	20.7724772292716\\
-0.2963413716162	20.7724772292716\\
-0.296341371616252	20.7768366593954\\
-0.2802726435443	20.7768366593954\\
-0.280272643544371	20.7812075233935\\
-0.2651970965658	20.7812075233934\\
-0.265197096565825	20.7855879620693\\
-0.2510018967903	20.7855879620692\\
-0.251001896790397	20.7899755371973\\
-0.2377151263197	20.7899755371973\\
-0.237715126319759	20.7943682221022\\
-0.2252397694854	20.7943682221022\\
-0.225239769485403	20.7987677332138\\
-0.2135785112149	20.7987677332138\\
-0.213578511214927	20.8031718125111\\
-0.2026134863251	20.803171812511\\
-0.202613486325123	20.8075817277601\\
-0.1923254326731	20.80758172776\\
-0.192325432673185	20.8119942207892\\
-0.1826126751929	20.8119942207891\\
-0.18261267519295	20.8164088168619\\
-0.1734824441258	20.8164088168618\\
-0.173482444125891	20.8208254028399\\
-0.164928744297	20.8208254028399\\
-0.16492874429701	20.825241492546\\
-0.1569105984152	20.8252414925459\\
-0.156910598415268	20.8296562462709\\
-0.1493467603566	20.8296562462709\\
-0.149346760356602	20.8340710755343\\
-0.1422556252469	20.8340710755342\\
-0.142255625246952	20.8384867884061\\
-0.1355107388692	20.838486788406\\
-0.135510738869209	20.8429065678649\\
-0.1291495581915	20.8429065678648\\
-0.129149558191511	20.8473253828468\\
-0.1231719775553	20.8473253828468\\
-0.123171977555356	20.8517432221125\\
-0.1172750076668	20.8517432221125\\
-0.117275007666881	20.8519873985973\\
0.11727500766689	20.8519873985971\\
0.1172750076668	20.8517432221123\\
0.123171977555366	20.8517432221123\\
0.1231719775553	20.8473253828467\\
0.129149558191521	20.8473253828467\\
0.1291495581915	20.8429065678646\\
0.135510738869219	20.8429065678647\\
0.1355107388692	20.8384867884059\\
0.142255625246962	20.8384867884059\\
0.1422556252469	20.8340710755341\\
0.149346760356612	20.8340710755341\\
0.1493467603566	20.8296562462707\\
0.156910598415277	20.8296562462708\\
0.1569105984152	20.8252414925458\\
0.164928744297019	20.8252414925459\\
0.164928744297	20.8208254028398\\
0.1734824441259	20.8208254028399\\
0.1734824441258	20.8164088168618\\
0.182612675192958	20.8164088168618\\
0.1826126751929	20.811994220789\\
0.192325432673193	20.8119942207891\\
0.1923254326731	20.80758172776\\
0.202613486325132	20.80758172776\\
0.2026134863251	20.8031718125109\\
0.213578511214935	20.803171812511\\
0.2135785112149	20.7987677332138\\
0.22523976948541	20.7987677332138\\
0.2252397694854	20.7943682221022\\
0.237715126319766	20.7943682221022\\
0.2377151263197	20.7899755371973\\
0.251001896790403	20.7899755371973\\
0.2510018967904	20.7855879620692\\
0.265197096565831	20.7855879620693\\
0.2651970965658	20.7812075233934\\
0.280272643544375	20.7812075233935\\
0.2802726435443	20.7768366593954\\
0.296341371616255	20.7768366593955\\
0.2963413716162	20.7724772292716\\
0.313649131038129	20.7724772292717\\
0.3136491310381	20.7681327830785\\
0.332275133728577	20.7681327830785\\
0.3322751337285	20.7638032514944\\
0.352154059760681	20.7638032514945\\
0.3521540597606	20.7594806413913\\
0.373451552848954	20.7594806413914\\
0.3734515528489	20.7551719826221\\
0.396292667718677	20.7551719826221\\
0.3962926677186	20.7508760027326\\
0.420613567997698	20.7508760027327\\
0.4206135679976	20.7465780957532\\
0.446711520332921	20.7465780957532\\
0.4467115203329	20.7422771102507\\
0.474827156632724	20.7422771102507\\
0.4748271566327	20.7379729560282\\
0.505051734255863	20.7379729560282\\
0.5050517342558	20.7336656018953\\
0.537481187469149	20.7336656018953\\
0.5374811874691	20.7293549220392\\
0.572229471805329	20.7293549220393\\
0.5722294718053	20.7250412345764\\
0.6093613001547	20.7250412345765\\
0.6093613001547	20.720721187647\\
0.614635897069764	20.7207211876471\\
};



\addplot[X0Style]
table[row sep=crcr] {%
-0.2	19.8\\
0.2	19.8\\
0.2	20.2\\
-0.2	20.2\\
-0.2	19.8\\
};


\addplot [TpReachStyle]
  table[row sep=crcr]{%
-0.5124873793722	20.1330757085947\\
-0.513097548038093	20.216935066521\\
0.513097548038093	20.216935066521\\
0.513097548038093	20.1830757018111\\
0.509018122389911	19.883075737071\\
0.507640044142832	19.7830757482197\\
-0.507640044142832	19.7830757482197\\
-0.511095551773927	20.0330757200888\\
-0.5124873793722	20.1330757085947\\
};


\addplot [TpReachStyle]
  table[row sep=crcr]{%
-0.244887376548593	20.1230601230726\\
-0.245389763804685	20.1730602028063\\
-0.245389763804685	20.2269636184881\\
0.190517633333467	20.226963714322\\
0.245389763804685	20.2269636184881\\
0.245389763804685	20.1730602028063\\
0.243464883413726	20.0230599719269\\
0.239970661159678	19.7730595898309\\
-0.239970661159678	19.7730595898309\\
-0.241847754132671	19.9230598201663\\
-0.244887376548593	20.1230601230726\\
};


\addplot [TpReachStyle]
  table[row sep=crcr]{%
0.453969082506685	19.2801544964527\\
-0.584177831827482	19.28015452217\\
-0.584177831827482	19.3451510087822\\
-0.57571970875658	19.5808108364029\\
-0.575161134401306	19.5881283701089\\
-0.570447832133873	19.7069189172477\\
-0.566292727325727	19.8164706640239\\
-0.565298789556206	19.8287128293018\\
-0.561171397453581	19.9343005778346\\
-0.559846040242302	19.9501702728264\\
-0.555744041156615	20.0521304916453\\
-0.554097362378204	20.0713271953364\\
-0.550018555618582	20.1699604054561\\
-0.548146134955402	20.1912407844085\\
-0.544317190127881	20.2877902588674\\
-0.535193966372649	20.3958137898874\\
-0.531780315490003	20.3958137937492\\
-0.522169767267282	20.5038373220357\\
-0.518444245287803	20.5038373263126\\
-0.513951780445804	20.5520501289961\\
-0.51007979399391	20.6020545359134\\
-0.509292374049405	20.6020545361771\\
-0.508378622154389	20.6118608543946\\
-0.504455170205805	20.6118608588186\\
-0.493966721905942	20.7198843869635\\
0.493966721905931	20.7198843869637\\
0.504455170196096	20.6118608589195\\
0.508378622154378	20.6118608543948\\
0.509292374049995	20.6020545361717\\
0.510079793993899	20.6020545359136\\
0.513951780445595	20.5520501289986\\
0.518444245292795	20.5038373262595\\
0.522169767267275	20.5038373220359\\
0.531780315493926	20.3958137937063\\
0.535193966372638	20.3958137898875\\
0.544317190127874	20.2877902588675\\
0.548146134955488	20.1912407844086\\
0.550018555618575	20.1699604054563\\
0.554097362378197	20.0713271953369\\
0.555744041156608	20.0521304916455\\
0.559846040242366	19.9501702728264\\
0.561171397453574	19.9343005778348\\
0.565298789556305	19.8287128293006\\
0.566292727325717	19.816470664024\\
0.575161134401295	19.588128370109\\
0.575719708756573	19.5808108364031\\
0.584177831827475	19.3451510087824\\
0.584177831827475	19.2801545221702\\
0.453969082506685	19.2801544964527\\
};


\addplot [TpReachStyle]
  table[row sep=crcr]{%
0.109385743202086	19.1491590112734\\
-0.109385743202076	19.1491590112736\\
-0.109004366012186	19.3578119035726\\
-0.107377008846669	19.7582542590821\\
-0.107377008846676	19.7592256816575\\
-0.105953834046318	20.038580142637\\
-0.105953834046325	20.0871994807332\\
-0.104322560572218	20.3184972155258\\
-0.103832535233877	20.3234004971637\\
-0.103238399302509	20.3588946521767\\
-0.102561863578075	20.3943888071896\\
-0.101878507594186	20.4298829622026\\
-0.101126643289206	20.4629254530973\\
-0.100426380123498	20.4984196088678\\
-0.0997704497343044	20.5492092765216\\
-0.0957570848059426	20.7444271406622\\
-0.0934851009587447	20.8509096130035\\
0.0934851009587412	20.8509096130034\\
0.0934851009587412	20.8331625342798\\
0.10228742827789	20.4249796828373\\
0.102561863578003	20.389485528479\\
0.104571009972982	20.2778917478016\\
0.107986015828683	19.6221522338228\\
0.108794121184722	19.4089033798879\\
0.109385743202083	19.2256417384474\\
0.109385743202086	19.1491590112734\\
};


\addplot [TpReachStyle]
  table[row sep=crcr]{%
-0.0854787349530923	20.0458141644213\\
0.0911437714688006	20.0458126730563\\
0.0911437714688006	19.9542946300296\\
-0.0911437714688006	19.9542946300296\\
-0.0911437714688006	20.0458126730563\\
-0.0854787349530923	20.0458141644213\\
};


\addplot [SimStyle]
  table[row sep=crcr]{%
0.199999999999999	19.8\\
0.0578870749244444	19.7983136817669\\
-0.0587685674413656	19.7966579281003\\
-0.154760013559763	19.7950503313263\\
-0.233897882617772	19.7935014269623\\
-0.299236269463673	19.792017233145\\
-0.347672015911431	19.7903153272626\\
-0.389237872697901	19.7886971610575\\
-0.424662175684542	19.7871623453125\\
-0.454703553913685	19.7857096510396\\
-0.480090725312792	19.7843373615556\\
-0.424619701487959	19.7830432856859\\
-0.378941765460251	19.7818249105706\\
-0.341299763351888	19.7806795398775\\
-0.310267054206545	19.779604545277\\
-0.284679659196513	19.7785973415882\\
-0.266447405213206	19.7812046993289\\
-0.25070928425145	19.7836965470208\\
-0.237258354194427	19.7860771486957\\
-0.225844012148062	19.7883505620814\\
-0.216207806281503	19.790520658192\\
-0.178163727342994	19.8149805181548\\
-0.141784408574715	19.8394399489616\\
-0.107035949905232	19.8638989163323\\
-0.085754932091298	19.8794634582372\\
-0.0535717563696707	19.9039216782849\\
-0.0228762838690315	19.9283795101234\\
0.00637398204580109	19.9528369262602\\
0.0242557587751016	19.9684005079132\\
0.0510967328247709	19.9927202251536\\
0.0767791673600087	20.0171766066984\\
0.101195087113283	20.0416325986811\\
0.116223878426556	20.057332304705\\
0.138660348428122	20.0817876571325\\
0.159950517062878	20.1062426664303\\
0.181925811177734	20.1329204280032\\
0.192443015971321	20.1462591268498\\
0.212563888184697	20.1729362168501\\
0.231511709911608	20.1996129112848\\
0.249348354452781	20.2262891570335\\
0.255057383397091	20.2351811297267\\
0.272326844316108	20.2635222972038\\
0.288492839288523	20.2918721939694\\
0.303622920191359	20.320230629467\\
0.304745733701477	20.3224123957479\\
0.318811365595145	20.3507799498217\\
0.332906374763422	20.3813390237871\\
0.34508515357523	20.4097238224238\\
0.357266526900982	20.4403011820678\\
0.369300014180489	20.473073037098\\
0.377506215800317	20.4971123880983\\
0.387890607584207	20.5299023735751\\
0.398002833616587	20.5648897496802\\
0.4032895486116	20.5845751739126\\
0.412520307975569	20.6217686318026\\
0.421294425269799	20.6611638090514\\
0.423565446052059	20.6721093738031\\
0.374625996543614	20.6816827969166\\
0.34772344728092	20.6874024558393\\
0.308043633882768	20.6965506510309\\
0.284048810511106	20.7025674702301\\
0.249584926655711	20.7119825192671\\
0.230666974670825	20.7176087859116\\
0.201810041834506	20.7269487175594\\
0.185982071715205	20.7325307158338\\
0.161857917109597	20.7417981924648\\
0.148636518502464	20.7473375065035\\
0.126509870074724	20.7581337470382\\
0.105675885858403	20.7690188036963\\
0.0875804917084899	20.7794387315919\\
0.0866784881210485	20.7799892260679\\
0.0712444003105865	20.7899887746576\\
0.069723341982197	20.7910417617045\\
0.0565701374108585	20.8007776484434\\
0.0548156908998898	20.8021733042119\\
0.0505738085208236	20.7197435536346\\
0.046898545517216	20.6373792598232\\
0.0435253161971687	20.555077919986\\
0.0403251792256398	20.472837098631\\
0.0372409949281689	20.3906544342713\\
0.0324398550098657	20.3577350777186\\
0.0280124447017869	20.3207128255109\\
0.0267292905512555	20.3083748637803\\
0.0228482926243601	20.2652028394099\\
0.0200092054168124	20.2261567427078\\
0.0166859272037811	20.1686393497448\\
0.0155054909312504	20.1439976991741\\
0.0125232711361747	20.0680511643833\\
0.0123172325200542	20.0618954098619\\
0.00992831305543973	19.9798476165348\\
};


\addplot [SimStyle]
  table[row sep=crcr]{%
-0.199999999999999	20.2\\
-0.105534439753285	20.1980328081328\\
-0.0267141226955943	20.1961194953031\\
0.0391071129244907	20.1942734776808\\
0.0941135010215284	20.1925024052721\\
0.140111698834797	20.1908102260601\\
0.182632852659943	20.1908829572533\\
0.217219905413462	20.1909512230982\\
0.245480947507492	20.191017127949\\
0.2686558249363	20.1910821686034\\
0.287712958071189	20.1911474109796\\
0.264297800995841	20.1934288291127\\
0.243745216378983	20.1956006488594\\
0.225907112039206	20.1976671039783\\
0.210544861718507	20.1996323078715\\
0.197385277611385	20.2015002343985\\
0.186771830260636	20.1993751923742\\
0.177501640185227	20.1973563752026\\
0.169464906869305	20.1954400522293\\
0.162532240105232	20.1936225962791\\
0.156571301486352	20.1919004845649\\
0.134983979833869	20.1715600199057\\
0.114643569642432	20.1512241150012\\
0.0933780907326245	20.1286339490299\\
0.0733944359367342	20.1060492783509\\
0.069541322547181	20.10153299777\\
0.0509707872840437	20.0789548593928\\
0.0317996628240564	20.0541251448755\\
0.0138250764463059	20.0293018852409\\
0.00144167556853603	20.0112526141456\\
-0.0149127568820298	19.9861055522145\\
-0.0314265039497279	19.9590448616923\\
-0.0469114691342689	19.931991544778\\
-0.0529009825975635	19.9210560463364\\
-0.0682333985434411	19.8917596715969\\
-0.083661821601396	19.8602191358441\\
-0.0970710429051991	19.8309401059941\\
-0.111552912984251	19.7971667786088\\
-0.125941156026059	19.7611537361689\\
-0.133589845148542	19.7409016940896\\
-0.154487195858341	19.6689783879159\\
-0.158770816203269	19.6537158980785\\
-0.17532777861582	19.5926452850798\\
-0.182112220830351	19.566462180722\\
-0.196699963022681	19.5075287443324\\
-0.203379246827364	19.4791429462635\\
-0.217047936892605	19.4179817369562\\
-0.222605018202337	19.3917604927378\\
-0.235352236167643	19.3283699488874\\
-0.239935511331584	19.3043170390361\\
-0.209915562809275	19.2966997365831\\
-0.200673587486463	19.2941463228757\\
-0.175925921529657	19.2867017592576\\
-0.167294852835997	19.2838627584275\\
-0.146736537115949	19.2764694855192\\
-0.139129653694081	19.2734703127046\\
-0.121570770748452	19.265869917475\\
-0.115475552623099	19.2629728736155\\
-0.100765646658171	19.2552987429065\\
-0.0956641904891917	19.252374246387\\
-0.0832668378448496	19.2440623038649\\
-0.0789747673186909	19.2409018332174\\
-0.0683002973945506	19.2322614241528\\
-0.0650512193821626	19.229374583274\\
-0.0555834808690889	19.2201146934558\\
-0.0534207707773398	19.2177945639296\\
-0.0452705232778357	19.2082027492397\\
-0.0436894649918855	19.2061637834787\\
-0.0364743812645649	19.1959467436684\\
-0.0355308546244864	19.1944841890035\\
-0.0290903101643067	19.2438085946071\\
-0.0256350019158234	19.2734245671882\\
-0.0201153784301056	19.3267728121658\\
-0.0177495910185286	19.3524768092499\\
-0.0129411441652501	19.411836955267\\
-0.0115211634507837	19.4316368559389\\
-0.00730360783368766	19.4990047160513\\
-0.00664768885665623	19.5109007740545\\
-0.00312362096250141	19.5843090667284\\
-0.00287322107751464	19.5902647529719\\
0.000436288604632296	19.6554993845869\\
0.00112748332247747	19.6715909226694\\
0.00378282079886461	19.7461660489775\\
0.00398405295637261	19.7529161131307\\
0.00597875295917305	19.8342403441187\\
0.00733002503951141	19.9155636320372\\
0.00820601258329035	19.9968859908285\\
};

\addplot [SimStyle]
  table[row sep=crcr]{%
0.199999999999999	20.2\\
0.248766916768268	20.2019313220067\\
0.289125170122762	20.2037169357797\\
0.322496512883315	20.2053770785383\\
0.350070226771201	20.2069269602361\\
0.372837634647937	20.2083782771633\\
0.389380335411513	20.209561295907\\
0.403601782980559	20.2106715208239\\
0.415718156195869	20.2117148121934\\
0.425972479336103	20.2126960720194\\
0.434607496553976	20.2136195179078\\
0.360623324692931	20.2128743303522\\
0.299365489918014	20.2121585396241\\
0.248586283258621	20.2114715459278\\
0.206457222938678	20.2108127137107\\
0.171483267686931	20.2101813566712\\
0.140710994147188	20.207748341622\\
0.115585505823997	20.2054330560455\\
0.0950094086603066	20.203231172789\\
0.0781235541306344	20.2011384652063\\
0.0642472265155831	20.1991508097888\\
0.0345575228723547	20.1774085180798\\
0.00644329844069347	20.1556605951215\\
-0.0202158057852344	20.1339070259901\\
-0.0455274892935549	20.1121478015354\\
-0.0698744813516399	20.0900958555976\\
-0.0949116765846618	20.0661482842357\\
-0.118604775426689	20.0421941603305\\
-0.134783327085632	20.025056265433\\
-0.156389455537134	20.0010910286946\\
-0.178695025142446	19.9749399408049\\
-0.19977679257747	19.9487813593992\\
-0.208220952559806	19.9378797424291\\
-0.227723653723544	19.9117106861752\\
-0.247707091750087	19.8833527199368\\
-0.266580746245996	19.8549862912606\\
-0.269390720888005	19.8506214754995\\
-0.28841816235915	19.8200623231911\\
-0.306377308862277	19.7894936990165\\
-0.320969849698194	19.76328449782\\
-0.355141935608369	19.6719722675159\\
-0.383407796031012	19.5944540707617\\
-0.388249840321865	19.5807834168267\\
-0.410211792653865	19.5170223099909\\
-0.419257487284888	19.4897135747524\\
-0.438104419325274	19.4305799618255\\
-0.44780456029374	19.3987584258148\\
-0.46502318718457	19.3396971134806\\
-0.473859673325382	19.3079137973167\\
-0.417818203339351	19.2988937548247\\
-0.388441570829304	19.2937487657557\\
-0.342630926517572	19.2850231527352\\
-0.316857562630396	19.2796691840899\\
-0.278236128804625	19.2709112180816\\
-0.257080912442575	19.2656714257791\\
-0.224889487691996	19.2569628988217\\
-0.207269724857731	19.2517521766265\\
-0.180470418587628	19.2430911492895\\
-0.165806518115417	19.2379083754191\\
-0.143065491989386	19.2283582171162\\
-0.132981036704489	19.2237745375084\\
-0.113816186168247	19.2143442879821\\
-0.105167653476276	19.2097287605621\\
-0.0883083166086713	19.1999473500537\\
-0.0816748622465333	19.195767755725\\
-0.0674563956195229	19.1860437958397\\
-0.0618650695204828	19.1818883591446\\
-0.0494812183047948	19.1718750882087\\
-0.0451718557905956	19.16808751172\\
-0.0375662375750458	19.205887023579\\
-0.0304681373869187	19.2457588763851\\
-0.0294263212432568	19.2520518675091\\
-0.0230134147742938	19.2939875216595\\
-0.0174374045177395	19.3358929248278\\
-0.0121636223585035	19.3819545909736\\
-0.00844294837095205	19.4196151931781\\
-0.00405864053416494	19.4718832237506\\
-0.00181356099008312	19.5032230131067\\
0.00176312402708234	19.5628970347006\\
0.00296663182424339	19.5867205675336\\
0.00581691093062986	19.655525905446\\
0.00631718884780241	19.6701118913566\\
0.00848459293530013	19.7492388019934\\
0.00857656998643819	19.7534008795406\\
0.0100154661170109	19.8365912939727\\
0.0108483688132992	19.9196867693607\\
0.0112435241787026	20.0026908184112\\
};

\addplot [SimStyle]
  table[row sep=crcr]{%
0.199999999999999	20.2\\
0.155394940370368	20.1998364377046\\
0.118559763559894	20.1996486457684\\
0.0880992236458908	20.1994481327787\\
0.062885546565095	20.199242812842\\
0.0420012216778218	20.1990380607414\\
0.0223810113192222	20.1970389050764\\
0.00661311687010624	20.1951367401187\\
-0.00610057619261184	20.1933296709834\\
-0.0163729116835825	20.1916152440279\\
-0.0246808993233714	20.1899906684343\\
-0.034249074369253	20.1884529657781\\
-0.0444854500769694	20.186999062474\\
-0.054523727693649	20.185625876559\\
-0.0639087211032709	20.1843303854292\\
-0.0724326156654698	20.183109628375\\
-0.0890206657969799	20.1858602197949\\
-0.100689529161599	20.1884837718615\\
-0.108974415732831	20.1909847788098\\
-0.114906063679513	20.1933676115086\\
-0.119181879943071	20.1956365180806\\
-0.106007510920634	20.2011680763412\\
-0.102175755681159	20.2030005690623\\
-0.0914742214678164	20.2088272223105\\
-0.0891268265093963	20.2102751363101\\
-0.0806012923940713	20.2162116249842\\
-0.0790067725602874	20.2174635082107\\
-0.0717956029771791	20.2238619744622\\
-0.0710864178547119	20.2245688618601\\
-0.0651201760533091	20.2312448530062\\
-0.0651300659051088	20.2321764372134\\
-0.0656301076706391	20.2347089725881\\
-0.0650730786195588	20.2379143324644\\
-0.0648988199417033	20.238442450358\\
-0.0633582759539131	20.2420211280134\\
-0.0614316859287598	20.2456978052972\\
-0.0593757326375339	20.249468853432\\
-0.0454206393212324	20.2495396286048\\
-0.0339716436497746	20.2495576326878\\
-0.0245349554187655	20.2495247774155\\
-0.0167265263292258	20.2494429144232\\
-0.0102441649069149	20.2493138366125\\
-0.00626425319495993	20.2502481645141\\
-0.00255484159889008	20.2510829584761\\
0.000813095928847929	20.2518219115351\\
0.00381987101301817	20.2524685968737\\
0.00647462814026412	20.253026470842\\
0.00452233977881278	20.2261811335038\\
0.00307600511998984	20.1994495524433\\
0.00201123554118254	20.1728275839445\\
0.00123527077671781	20.1463112159129\\
0.000678269015946142	20.1198965643992\\
0.00366699956386185	20.1068251096875\\
0.00562552576388242	20.0952695269394\\
0.00758734411477846	20.0786516087174\\
0.00825028114434545	20.0706518807798\\
0.00936916211844263	20.0491188516847\\
0.00946571234581484	20.0460432542625\\
0.00983378233325638	20.0214432890598\\
0.00969973558187931	19.9968516391897\\
};

\addplot [SimStyle]
  table[row sep=crcr]{%
-0.123605423721042	19.9770119848812\\
-0.0987809623535192	19.9750598410925\\
-0.0797702417760995	19.9731997482796\\
-0.0650408734199317	19.9714303461052\\
-0.053516850096738	19.9697496398102\\
-0.044430023842942	19.9681552322695\\
-0.0453015122448193	19.969174557999\\
-0.0442104122740545	19.9702621289388\\
-0.0414519110752138	19.9714822808554\\
-0.0384916928740786	19.9724904603109\\
-0.0354058314888874	19.9734514681311\\
-0.0293420654833092	19.9755943139099\\
-0.0236298147535834	19.977631755906\\
-0.0184060931326435	19.9795677093745\\
-0.0137172280252997	19.9814059480824\\
-0.00955814589703152	19.9831501149144\\
0.00218477518363613	19.9832863009719\\
0.00988847007522509	19.9834115844945\\
0.0149197991216141	19.983526419737\\
0.0181732924826292	19.9836312500175\\
0.0202369436092376	19.9837265045655\\
0.0178485571346449	19.9962600239774\\
0.0153212527515478	20.0089045045336\\
0.0128264593755461	20.0216558866008\\
0.010458845584818	20.034510241492\\
0.00826536916666143	20.0474637675374\\
0.00624739651011552	20.0644123062814\\
0.00443126274309336	20.0812567894437\\
0.00280495674703474	20.0980010484722\\
0.00135264473210484	20.1146487909738\\
5.72152802185144e-05	20.1312036043338\\
-0.000551759016332198	20.1316975472308\\
-0.000905739446135811	20.1322634679376\\
-0.00111363792936459	20.1328987471711\\
-0.00124148873344865	20.1336008496796\\
-0.00132843529183191	20.1343673216839\\
-0.000509133438107767	20.1344414898888\\
-7.65902905435212e-05	20.1346132669741\\
0.000132248723424766	20.1352352449493\\
-0.000418252914975881	20.1332158569269\\
-0.00225377721386977	20.1210185197912\\
-0.00360676067797527	20.1063712378959\\
-0.00435994467433432	20.0917361303247\\
-0.00473873436816774	20.0771126587347\\
-0.00488466272330257	20.0625003041259\\
-0.00864895023886447	20.0549131908391\\
-0.0117547810069638	20.0465483138485\\
-0.0142500177899407	20.0370147485051\\
-0.0154237201590561	20.0306743139333\\
-0.0168248964337288	20.0188181947366\\
-0.017107274275844	20.0148753178106\\
-0.0175677202228712	19.9995407551145\\
-0.0172870655103807	19.9834908894425\\
};

\addplot [SimStyle]
  table[row sep=crcr]{%
-0.0727581095398939	19.8535243941425\\
-0.0563637446664096	19.8554725366755\\
-0.0442797311561876	19.8573188114803\\
-0.0352703780700416	19.8590684957473\\
-0.0284867936506465	19.8607261763088\\
-0.0233383440344213	19.8622959477507\\
-0.0233609929176701	19.8608743658231\\
-0.0224662479625266	19.8595181423535\\
-0.0211359639266036	19.8582250889879\\
-0.0196525310555664	19.8569930270212\\
-0.0181751649299962	19.8558198014213\\
-0.01356981251935	19.8537109221418\\
-0.00813976688820972	19.8517065726088\\
-0.00254539005195653	19.849802913815\\
0.00284923431441797	19.8479962220494\\
0.00785966586057896	19.8462828884201\\
0.0124082372559116	19.8446594185862\\
0.0164788523457204	19.8431224316776\\
0.0200892262872365	19.8416686582629\\
0.0232740777892104	19.8402949375474\\
0.0260751613887287	19.8389982140639\\
0.032623531651236	19.8481564456997\\
0.0360456066833166	19.8536499426503\\
0.0417594055408586	19.864267672059\\
0.0436564838565268	19.8682940490282\\
0.048467511175339	19.8800039450673\\
0.0495250951526458	19.8829306794858\\
0.0537857483988198	19.8964630376604\\
0.0540906557750844	19.8975599842479\\
0.0576696363150973	19.912182113112\\
0.0624297286524609	19.9206718283194\\
0.0662593610767033	19.9291299306779\\
0.0699908793375101	19.9396589710625\\
0.0717688886396672	19.9459535689581\\
0.0744256658211455	19.9580755991578\\
0.0752092949126677	19.9626574956896\\
0.0771025295030299	19.9775922668019\\
0.0772593989277759	19.9792460510689\\
0.0783760110601008	19.9957234443487\\
0.0595039194805089	19.9983988348702\\
0.0448915942518546	20.0010914458342\\
0.0334474755519487	20.0038006344074\\
0.0243963888188006	20.0065257740824\\
0.017179130617496	20.0092662569549\\
0.0167497640697896	20.0104186758209\\
0.0149216203570823	20.0118546628087\\
0.0125911694152983	20.0132802874799\\
0.0100397036286353	20.0147891552443\\
0.00748280911468058	20.0163781255006\\
0.00406705230573223	20.0131058039392\\
0.00181015592780653	20.0097081452705\\
0.000444213065222243	20.0064336701689\\
-0.000401412380725219	20.0031974280873\\
-0.000918389647303286	19.9999979009924\\
-0.00576721924268497	19.9969330719634\\
-0.00666477676533717	19.9961828283718\\
-0.00949997170537031	19.992992437338\\
-0.00985794545352547	19.9924342971635\\
-0.011391662371306	19.9889324732185\\
-0.0120232859295939	19.9851266130846\\
-0.0119831839612878	19.9815625470575\\
};

\addplot [SimStyle]
  table[row sep=crcr]{%
-0.00956811269301738	20.1632409666028\\
0.0624791321953069	20.1636302526873\\
0.121930071444087	20.163978034154\\
0.171051784859323	20.1642929030073\\
0.211679682122686	20.1645809309761\\
0.245307310921262	20.1648464953824\\
0.27347849072105	20.166599198498\\
0.296799062757184	20.1682593956892\\
0.316097725749685	20.1698318256701\\
0.332062226912079	20.1713207893711\\
0.34526314422304	20.1727302585011\\
0.29216875365352	20.1702100481457\\
0.249791168155944	20.1678106641959\\
0.215728012594596	20.165527827123\\
0.188197107535117	20.1633573808987\\
0.16585375131887	20.1612952815052\\
0.155689595717021	20.1609345515513\\
0.145349004277023	20.1605944091656\\
0.135490878864466	20.1602743043635\\
0.126442244861394	20.1599736305556\\
0.118332556933776	20.1596917473713\\
0.0815854136960965	20.1444036843319\\
0.0473141542253757	20.1291293844014\\
0.0257443581763397	20.1189540115921\\
-0.00416170278140271	20.1040360225385\\
-0.0327306273001078	20.0887971393378\\
-0.0513035316799311	20.0783117173567\\
-0.0767824467630298	20.0630947760362\\
-0.102136489580566	20.0468774777038\\
-0.115624941908429	20.0377615325012\\
-0.138322015496144	20.0215664882523\\
-0.159483685877131	20.0053854483317\\
-0.169524134200554	19.9973000865839\\
-0.189747513055824	19.9801300188917\\
-0.208555471302969	19.9629751377082\\
-0.214874986736195	19.9569240334953\\
-0.231740027567252	19.9438196413769\\
-0.248408735374202	19.9297167995135\\
-0.262641540449824	19.9166300193307\\
-0.277742007793432	19.9015401965786\\
-0.291581587675005	19.8864612712651\\
-0.300171641673689	19.8764145859498\\
-0.313732962440536	19.8593460162349\\
-0.326114587255411	19.8422908371022\\
-0.330225622619789	19.8362744936194\\
-0.341833481143556	19.8182352198727\\
-0.353006794331531	19.7992093765776\\
-0.354676735814184	19.7962067314351\\
-0.365725987270174	19.7751991527044\\
-0.374825021162067	19.7562084705306\\
-0.298722815207626	19.7529120120056\\
-0.237894603482559	19.7494847661464\\
-0.18899023857519	19.7459313434441\\
-0.149481377130872	19.7422562234229\\
-0.117435440367966	19.738463770187\\
-0.100413497058707	19.7327678500267\\
-0.0841259771844207	19.7270528229977\\
-0.0691682308491117	19.7213193062614\\
-0.0557731994205106	19.7155679508861\\
-0.0439690934715209	19.7097994103664\\
-0.0368356939507883	19.7320183373682\\
-0.0354402420350368	19.7367394378626\\
-0.0292197445962081	19.7597094109321\\
-0.0282123381229056	19.7637695342225\\
-0.0227000204249208	19.7881708799915\\
-0.0221400600584651	19.7908863740228\\
-0.0171882242317807	19.8174057299622\\
-0.017072477022495	19.8180867313115\\
-0.0128645183666123	19.8453674783313\\
-0.0092739743467547	19.87662510419\\
-0.00635725662286646	19.9077611639183\\
-0.00398367529565391	19.9387800898343\\
-0.00204741814701492	19.969686169409\\
-0.000462968062667102	20.0004835500295\\
};

\addplot [ShiftX0Style]
  table[row sep=crcr]{%
-0.199999999999999	19.8\\
0.199999999999999	19.8\\
0.199999999999999	20.2\\
-0.199999999999999	20.2\\
-0.199999999999999	19.8\\
};

\addplot [FinalReachStyle]
  table[row sep=crcr]{%
-0.0845176673418031	20.0458141310414\\
-0.0828734165661302	20.0458141432426\\
0.0828734165661373	20.04581418091\\
0.0911476555709569	20.0458139186147\\
0.0911476555709569	19.9542946300296\\
-0.0911476555709569	19.9542946300296\\
-0.0911476555709569	20.0458139186147\\
-0.0845176673418031	20.0458141310414\\
};
		\nextgroupplot[title={$[s_x, ~ s_y]$}]
		\addplot[IntReachStyle]
table[row sep=crcr] {%
-0.205142440506246	-0.22942013817656\\
-0.205142440506246	0.22942013817656\\
-0.0765252754958128	0.22942013817655\\
-0.0765252754958214	0.256689116355481\\
0.0524665873522043	0.25668911635548\\
0.0524665873522018	0.282218831347794\\
0.181791990746625	0.282218831347791\\
0.181791990746623	0.306101820042297\\
0.311413263267674	0.306101820042291\\
0.31141326326767	0.328424277817509\\
0.441296529136066	0.3284242778175\\
0.441296529136073	0.349266240066345\\
0.571413092191185	0.349266240066301\\
0.57141309219119	0.368700891812\\
0.701737440653689	0.3687008918119\\
0.701737440653767	0.38679688651788\\
0.832246964380199	0.3867968865178\\
0.832246964380239	0.403618574939322\\
0.962921417716683	0.4036185749393\\
0.962921417716748	0.419228390712391\\
1.09374104752026	0.4192283907123\\
1.0937410475203	0.43370432067702\\
1.22468967173714	0.433704320677\\
1.22468967173719	0.447075369494517\\
1.35575257633582	0.4470753694945\\
1.35575257633589	0.45938959916708\\
1.48691656562142	0.459389599167\\
1.48691656562146	0.470691604423883\\
1.61816943313435	0.4706916044238\\
1.61816943313441	0.481022547533543\\
1.74950178707323	0.4810225475335\\
1.74950178707326	0.490545286383977\\
1.88090225275708	0.4905452863839\\
1.88090225275711	0.499455409999742\\
2.01236182302429	0.499455409999701\\
2.01236182302434	0.507566155711707\\
2.14386454453052	0.5075661557117\\
2.14386454453055	0.514906679721179\\
2.27540193915057	0.5149066797211\\
2.27540193915067	0.521504329641869\\
2.4069822552981	0.5215043296418\\
2.40698225529817	0.527383441546892\\
2.53860113807108	0.527383441546801\\
2.53860113807112	0.532565480874429\\
2.67025386398464	0.5325654808744\\
2.67025386398466	0.537071865414505\\
2.80193720017832	0.5370718654145\\
2.80193720017836	0.540920498629945\\
2.93364765679031	0.5409204986299\\
2.93364765679032	0.544128151479125\\
3.06538189835743	0.5441281514791\\
3.06538189835752	0.546708474031161\\
3.59250780647815	0.551013306341246\\
4.52407836228096	0.550355389759896\\
5.19816523188387	0.543779563187205\\
5.19816523188386	0.5414810619253\\
5.33297826365616	0.541481061925345\\
5.33297826365608	0.5388488931548\\
5.467790008012	0.53884889315489\\
5.46779000801196	0.5358884043662\\
5.60260045317976	0.535888404366275\\
5.60260045317974	0.5325951854724\\
5.73740961423613	0.532595185472475\\
5.73740961423609	0.5289734814601\\
5.87221744797942	0.528973481460177\\
5.87221744797934	0.5250256980157\\
6.00702421497373	0.525025698015719\\
6.0070242149737	0.5207765191022\\
6.14183005273344	0.52077651910223\\
6.14183005273338	0.5162362427743\\
6.2766358584522	0.516236242774389\\
6.27663585845213	0.5113774387492\\
6.41144176452292	0.51137743874924\\
6.41144176452288	0.5062035147565\\
6.5462479956684	0.506203514756538\\
6.54624799566836	0.5007151422092\\
6.68106371833977	0.500715142209282\\
6.68106371833973	0.4949177377172\\
6.8158856893327	0.494917737717293\\
6.81588568933267	0.4888157390887\\
6.95070660114119	0.488815739088753\\
6.95070660114115	0.4824143539311\\
7.08552636185294	0.48241435393112\\
7.08552636185289	0.4757190214154\\
7.22034480607583	0.475719021415476\\
7.22034480607577	0.4687312698371\\
7.35516167213399	0.468731269837183\\
7.35516167213397	0.4614490866955\\
7.48997678350338	0.461449086695565\\
7.48997678350336	0.4538793186664\\
7.62478992641138	0.453879318666453\\
7.62478992641128	0.4460392898715\\
7.75973255321841	0.4460392898715\\
7.75973255321833	0.4379467027638\\
7.89475373743591	0.437946702763832\\
7.89475373743585	0.4295784920851\\
8.02978184983095	0.429578492085146\\
8.02978184983091	0.4209422829766\\
8.16481697906386	0.420942282976613\\
8.1648169790638	0.4120404143364\\
8.29985891429985	0.412040414336453\\
8.29985891429982	0.4029582851962\\
8.43489374146929	0.402958285196243\\
8.43489374146922	0.393909799829\\
8.56976118519248	0.393909799829046\\
8.56976118519248	0.3851554669817\\
8.7044345934884	0.385155466981741\\
8.70443459348831	0.3766641306315\\
8.83892532452606	0.376664130631557\\
8.83892532452601	0.3684037557339\\
8.97322005105046	0.368403755733956\\
8.97322005105037	0.3603493408836\\
9.10731825562438	0.360349340883671\\
9.10731825562434	0.3524804074407\\
9.24121976765472	0.352480407440792\\
9.2412197676547	0.3447884631273\\
9.37492480367664	0.344788463127341\\
9.37492480367664	0.3372634543409\\
9.50844976734273	0.337263454340961\\
9.50844976734266	0.3299007286481\\
9.64179501651338	0.329900728648114\\
9.6417950165133	0.3226903974798\\
9.77494246004704	0.322690397479891\\
9.77494246004696	0.3156299572565\\
9.90789125252236	0.315629957256527\\
9.9078912525223	0.30872354005\\
10.0406417989773	0.308723540050055\\
10.0406417989772	0.3019709680304\\
10.173194110679	0.301970968030434\\
10.1731941106789	0.2953871906056\\
10.3055538094145	0.295387190605617\\
10.3055538094144	0.2889783534115\\
10.4377998405443	0.288978353411559\\
10.4377998405443	0.2827355302779\\
10.5698474683484	0.282735530277986\\
10.5698474683484	0.2766776703577\\
10.7016960455029	0.276677670357766\\
10.7016960455028	0.2708161981609\\
10.8333446958157	0.270816198160947\\
10.8333446958156	0.2651669435815\\
10.9647931247328	0.265166943581588\\
10.9647931247328	0.2597288829018\\
11.0960417499303	0.259728882901881\\
11.0960417499302	0.2545137558574\\
11.2270908237596	0.254513755857491\\
11.2270908237595	0.2495341168817\\
11.3579349431471	0.249534116881735\\
11.3579349431471	0.2447959103368\\
11.4885773387152	0.244795910336815\\
11.4885773387152	0.2403107386395\\
11.6190176940316	0.24031073863954\\
11.6190176940315	0.2361114139376\\
11.7492555796834	0.236111413937644\\
11.7492555796833	0.2323186543204\\
11.8792892708225	0.232318654320414\\
11.8792892708224	0.2288351793564\\
12.0091192238001	0.22883517935649\\
12.0091192238001	0.2256762503042\\
12.1387442014986	0.225676250304234\\
12.1387442014985	0.2228181576234\\
12.2681637620555	0.222818157623491\\
12.2681637620547	0.220213801664\\
12.6554880520276	0.21504635075783\\
12.6554880520275	0.21237389193\\
12.7845171125012	0.21237389193009\\
12.7845171125009	0.209691014121\\
12.9135193238681	0.209691014121607\\
12.9135193238673	0.207017996597\\
13.0424933610068	0.20701799659762\\
13.0424933610059	0.2043985949\\
13.1714382992161	0.204398594900895\\
13.1714382992157	0.201823619155\\
13.3003546865516	0.201823619155937\\
13.3003546865508	0.199277811587\\
13.4292441805699	0.199277811587969\\
13.4292441805695	0.196827342358\\
13.5581043904892	0.196827342358277\\
13.5581043904891	0.194455474902\\
13.6869338594645	0.194455474902828\\
13.6869338594642	0.192156449097\\
14.2019383761933	0.184140644381\\
14.9734962461362	0.175507138917\\
15.8722235315223	0.170654777829\\
16.6784912928164	0.171111281581\\
17.63589898264	0.176098510428\\
18.9883565530237	0.189265535929\\
19.7911770042043	0.200809865746543\\
20.0497752553378	0.200809865746547\\
20.0497752553378	-0.200809865746544\\
19.7911770042043	-0.200809865746548\\
18.7193055406567	-0.186203908099\\
17.4996657888395	-0.175146318792\\
16.5410037211559	-0.17071857608\\
15.7439281221084	-0.171038563717\\
14.9734962461362	-0.175507138917\\
14.0732347866118	-0.18599235846\\
13.8157318865866	-0.19000027753\\
13.8157318865867	-0.192156449097857\\
13.6869338594642	-0.192156449097\\
13.6869338594644	-0.194455474902825\\
13.5581043904891	-0.194455474902\\
13.5581043904892	-0.196827342358274\\
13.4292441805695	-0.196827342358\\
13.4292441805698	-0.199277811587966\\
13.3003546865508	-0.199277811587\\
13.3003546865516	-0.201823619155935\\
13.1714382992157	-0.201823619155\\
13.1714382992161	-0.204398594900893\\
13.0424933610059	-0.2043985949\\
13.0424933610067	-0.207017996597618\\
12.9135193238673	-0.207017996597\\
12.9135193238681	-0.209691014121606\\
12.7845171125009	-0.209691014121\\
12.7845171125012	-0.212373891930088\\
12.6554880520275	-0.21237389193\\
12.6554880520275	-0.215046350757828\\
12.2681637620547	-0.220213801664\\
12.2681637620554	-0.22281815762349\\
12.1387442014985	-0.2228181576234\\
12.1387442014986	-0.225676250304233\\
12.0091192238001	-0.2256762503042\\
12.0091192238001	-0.228835179356488\\
11.8792892708224	-0.2288351793564\\
11.8792892708225	-0.232318654320412\\
11.7492555796833	-0.2323186543204\\
11.7492555796834	-0.236111413937642\\
11.6190176940315	-0.2361114139376\\
11.6190176940315	-0.240310738639539\\
11.4885773387152	-0.2403107386395\\
11.4885773387152	-0.244795910336813\\
11.3579349431471	-0.2447959103368\\
11.3579349431471	-0.249534116881734\\
11.2270908237595	-0.2495341168817\\
11.2270908237596	-0.25451375585749\\
11.0960417499302	-0.2545137558574\\
11.0960417499303	-0.25972888290188\\
10.9647931247327	-0.2597288829018\\
10.9647931247328	-0.265166943581586\\
10.8333446958156	-0.2651669435815\\
10.8333446958157	-0.270816198160946\\
10.7016960455028	-0.2708161981609\\
10.7016960455029	-0.276677670357764\\
10.5698474683483	-0.2766776703577\\
10.5698474683484	-0.282735530277985\\
10.4377998405443	-0.2827355302779\\
10.4377998405443	-0.288978353411557\\
10.3055538094144	-0.2889783534115\\
10.3055538094145	-0.295387190605615\\
10.1731941106789	-0.2953871906056\\
10.173194110679	-0.301970968030433\\
10.0406417989772	-0.3019709680304\\
10.0406417989773	-0.308723540050053\\
9.9078912525223	-0.30872354005\\
9.90789125252236	-0.315629957256527\\
9.77494246004696	-0.3156299572565\\
9.77494246004704	-0.322690397479889\\
9.6417950165133	-0.3226903974798\\
9.64179501651337	-0.329900728648114\\
9.50844976734266	-0.3299007286481\\
9.50844976734273	-0.33726345434096\\
9.37492480367654	-0.3372634543409\\
9.37492480367664	-0.344788463127339\\
9.2412197676547	-0.3447884631273\\
9.24121976765472	-0.352480407440791\\
9.10731825562434	-0.3524804074407\\
9.10731825562438	-0.36034934088367\\
8.97322005105037	-0.3603493408836\\
8.97322005105046	-0.368403755733955\\
8.83892532452601	-0.3684037557339\\
8.83892532452606	-0.376664130631555\\
8.70443459348831	-0.3766641306315\\
8.7044345934884	-0.38515546698174\\
8.56976118519248	-0.3851554669817\\
8.56976118519248	-0.393909799829046\\
8.43489374146922	-0.393909799829\\
8.43489374146929	-0.40295828519624\\
8.29985891429982	-0.4029582851962\\
8.29985891429985	-0.412040414336452\\
8.1648169790638	-0.4120404143364\\
8.16481697906386	-0.420942282976614\\
8.02978184983091	-0.4209422829766\\
8.02978184983095	-0.429578492085145\\
7.89475373743585	-0.4295784920851\\
7.89475373743591	-0.437946702763831\\
7.75973255321833	-0.4379467027638\\
7.75973255321841	-0.446039289871499\\
7.62478992641128	-0.4460392898714\\
7.62478992641138	-0.453879318666453\\
7.48997678350336	-0.4538793186664\\
7.48997678350338	-0.461449086695566\\
7.35516167213397	-0.4614490866955\\
7.35516167213399	-0.468731269837182\\
7.22034480607577	-0.4687312698371\\
7.22034480607583	-0.475719021415475\\
7.08552636185289	-0.4757190214154\\
7.08552636185294	-0.482414353931119\\
6.95070660114115	-0.4824143539311\\
6.95070660114119	-0.488815739088752\\
6.81588568933267	-0.4888157390887\\
6.8158856893327	-0.494917737717293\\
6.68106371833973	-0.4949177377172\\
6.68106371833977	-0.500715142209282\\
6.54624799566836	-0.5007151422092\\
6.5462479956684	-0.506203514756538\\
6.41144176452288	-0.5062035147565\\
6.41144176452292	-0.511377438749241\\
6.27663585845213	-0.5113774387492\\
6.2766358584522	-0.516236242774389\\
6.14183005273338	-0.5162362427743\\
6.14183005273344	-0.520776519102231\\
6.0070242149737	-0.5207765191022\\
6.00702421497373	-0.525025698015719\\
5.87221744797934	-0.5250256980157\\
5.87221744797942	-0.528973481460176\\
5.73740961423609	-0.5289734814601\\
5.73740961423613	-0.532595185472475\\
5.60260045317974	-0.5325951854724\\
5.60260045317976	-0.535888404366274\\
5.46779000801196	-0.5358884043662\\
5.467790008012	-0.538848893154891\\
5.33297826365608	-0.5388488931548\\
5.33297826365616	-0.541481061925344\\
4.92853526432924	-0.547384981160039\\
4.254431460749	-0.551013306341247\\
3.32891176271427	-0.55004111810509\\
2.93364765679032	-0.544128151479125\\
2.93364765679031	-0.5409204986299\\
2.80193720017836	-0.540920498629945\\
2.80193720017832	-0.5370718654145\\
2.67025386398466	-0.537071865414504\\
2.67025386398464	-0.5325654808744\\
2.53860113807112	-0.532565480874428\\
2.53860113807108	-0.5273834415468\\
2.40698225529817	-0.527383441546891\\
2.4069822552981	-0.5215043296418\\
2.27540193915067	-0.521504329641868\\
2.27540193915057	-0.5149066797211\\
2.14386454453055	-0.514906679721179\\
2.14386454453052	-0.5075661557117\\
2.01236182302434	-0.507566155711705\\
2.01236182302429	-0.499455409999699\\
1.88090225275711	-0.499455409999742\\
1.88090225275708	-0.4905452863839\\
1.74950178707326	-0.490545286383977\\
1.74950178707323	-0.4810225475335\\
1.61816943313441	-0.481022547533543\\
1.61816943313435	-0.4706916044238\\
1.48691656562146	-0.470691604423883\\
1.48691656562142	-0.459389599167\\
1.35575257633589	-0.45938959916708\\
1.35575257633582	-0.4470753694945\\
1.22468967173719	-0.447075369494517\\
1.22468967173714	-0.433704320677\\
1.0937410475203	-0.433704320677019\\
1.09374104752026	-0.4192283907123\\
0.962921417716748	-0.419228390712391\\
0.962921417716683	-0.4036185749393\\
0.832246964380239	-0.403618574939321\\
0.832246964380199	-0.3867968865178\\
0.701737440653767	-0.386796886517879\\
0.701737440653689	-0.3687008918119\\
0.57141309219119	-0.368700891811998\\
0.571413092191185	-0.349266240066299\\
0.441296529136073	-0.349266240066345\\
0.441296529136066	-0.3284242778175\\
0.31141326326767	-0.328424277817508\\
0.311413263267674	-0.306101820042289\\
0.181791990746623	-0.306101820042296\\
0.181791990746625	-0.282218831347789\\
0.0524665873522018	-0.282218831347793\\
0.0524665873522043	-0.256689116355479\\
-0.0765252754958214	-0.256689116355481\\
-0.0765252754958128	-0.22942013817655\\
};



\addplot[X0Style]
table[row sep=crcr] {%
-0.2	-0.2\\
0.2	-0.2\\
0.2	0.2\\
-0.2	0.2\\
-0.2	-0.2\\
}--cycle;


\addplot [TpReachStyle]
  table[row sep=crcr]{%
4.22512462883437	0.548765464139156\\
4.23402515176142	0.470944756700224\\
4.23458727470669	0.470944756700233\\
4.241050421076	0.393124049261306\\
4.24162852718036	0.393124049261309\\
4.24565429699194	0.315303341822382\\
4.24624916507208	0.315273750838376\\
4.24783662660028	0.237492629331316\\
4.24844685193174	0.237484748146722\\
4.248665250563	0.159734689131466\\
4.24916969223886	0.15970539133632\\
4.24922964671677	0.0139940390686375\\
4.24916969223886	-0.15970539133632\\
4.248665250563	-0.159734689131464\\
4.24844685193174	-0.237484748146722\\
4.24783662660029	-0.237492629331319\\
4.24624838625538	-0.315303341822386\\
4.24565429699194	-0.315303341822371\\
4.24381508965503	-0.351032631260604\\
4.24162852718036	-0.393124049261309\\
4.241050421076	-0.393124049261304\\
4.24095166518209	-0.394622891726727\\
4.23458727470669	-0.470944756700233\\
4.23402515176141	-0.470944756700225\\
4.23388864609043	-0.472443599165651\\
4.22512462883437	-0.548765464139156\\
3.80878884312147	-0.548765464139156\\
3.72907164106301	-0.537890241178455\\
3.72739385551984	-0.320476389303423\\
3.72617614719122	-0.174428896801618\\
3.72662697433504	-0.160812259436534\\
3.74073181030729	-0.02107642474025\\
3.74214070810278	9.76996261670138e-15\\
3.74073181030729	0.0210764247402357\\
3.73639235622297	0.0623038573224681\\
3.73370144484054	0.0937878882272258\\
3.73158046396598	0.111226812168694\\
3.72649462296154	0.163170659788811\\
3.72617614719122	0.190040034589956\\
3.72907164106301	0.537890241178455\\
3.80878884312147	0.548765464139156\\
4.22512462883437	0.548765464139156\\
};


\addplot [TpReachStyle]
  table[row sep=crcr]{%
8.26735996322568	0.400675764052398\\
8.27722382281538	0.346285596655878\\
8.277858315935	0.346285596655884\\
8.28512017948499	0.291895429259364\\
8.28576817412056	0.291895429259368\\
8.29042804163082	0.237505261862845\\
8.29108953778236	0.237505261862854\\
8.29314718398899	0.183126318726588\\
8.29382240692039	0.183116381840852\\
8.29428378973578	0.131372153643147\\
8.29518679468398	0.13136064352687\\
8.29540242141507	-0.0672792581641808\\
8.29518679468398	-0.13136064352687\\
8.29428378973578	-0.131372153643147\\
8.29382240692039	-0.183116381840852\\
8.29314718398899	-0.183126318726577\\
8.2931312573911	-0.183919895821836\\
8.29108953778236	-0.237505261862854\\
8.29042804163082	-0.237505261862845\\
8.29038876635043	-0.238310063218352\\
8.28576817412056	-0.291895429259368\\
8.28512017948499	-0.291895429259363\\
8.285057780786	-0.292700230614866\\
8.277858315935	-0.346285596655884\\
8.27722382281538	-0.34628559665588\\
8.27713830069779	-0.347090398011382\\
8.26735996322568	-0.400675764052398\\
7.83909014832515	-0.400675764052398\\
7.75920872100221	-0.395453697570787\\
7.75919578954952	-0.394124026111671\\
7.67937502535693	-0.389051769767503\\
7.67910313355755	-0.323429254447346\\
7.6801558676136	-0.22627712545366\\
7.67984032256139	-0.140828040772263\\
7.68021371400858	-0.111816525050104\\
7.68528897153538	-0.0665415278095907\\
7.68662333559098	-0.0592462556056432\\
7.69021094551096	-0.0215442562791939\\
7.69361071030871	3.01980662698043e-14\\
7.69021094551096	0.0215442562791939\\
7.68662333559098	0.0592462556056432\\
7.68528897153538	0.0665415278095907\\
7.68014828719296	0.113096077682252\\
7.67984032256139	0.178254980892275\\
7.6801558676136	0.226277125453668\\
7.67910313355755	0.360857457512415\\
7.67937502535693	0.389051769767503\\
7.75919578954951	0.394124026111667\\
7.75920872100221	0.395453697570787\\
7.83909014832515	0.400675764052398\\
8.26735996322568	0.400675764052398\\
};


\addplot [TpReachStyle]
  table[row sep=crcr]{%
12.1369132026955	0.214005115348195\\
12.1751763268828	0.215629381444019\\
12.2224861906251	0.215629381444019\\
12.2336814824655	0.191810643000201\\
12.2346011019883	0.191810643000203\\
12.242909700005	0.167991904568586\\
12.2438747585059	0.167991904568591\\
12.2492966627006	0.144173166147338\\
12.2503071601778	0.144173166147338\\
12.2528419023901	0.120362290987416\\
12.2539016126011	0.12032909602816\\
12.2548176062892	0.0875175539568751\\
12.2558343940871	0.0813790537208519\\
12.2559689113227	0.072814030483384\\
12.2567469574376	0.0727224845360812\\
12.256811851997	0.0706841466168129\\
12.2567271238543	-0.0727531427706527\\
12.2559689113227	-0.0728140304833875\\
12.2558133413684	-0.0824374094810985\\
12.2548176062891	-0.0875175539569266\\
12.2538983070039	-0.120354427797398\\
12.2528419023901	-0.12036229098741\\
12.2528014068108	-0.121042455295829\\
12.2503071601777	-0.14417316614734\\
12.2492966627006	-0.144173166147338\\
12.2492098522124	-0.144861193717082\\
12.2438747585059	-0.167991904568593\\
12.242909700005	-0.167991904568582\\
12.2427770427683	-0.168679932148693\\
12.2346011019883	-0.191810643000204\\
12.2336814824655	-0.191810643000201\\
12.2335029784784	-0.19249867059251\\
12.222486190625	-0.215629381444021\\
12.1751763268828	-0.215629381444021\\
12.1369132026955	-0.214005115348199\\
12.1211438250671	-0.212424984013921\\
12.0986147968227	-0.211452713286663\\
12.0826837964017	-0.209840742335212\\
12.0603163909499	-0.208859194610911\\
12.0442223180169	-0.207214245384042\\
12.0220179850771	-0.206223272817256\\
12.0057614767687	-0.204544391930211\\
11.9837195792043	-0.203543765044103\\
11.9673015395747	-0.201830003086661\\
11.9454211733315	-0.200819488429852\\
11.9301787757075	-0.199210895012891\\
11.9071288451242	-0.198244767310005\\
11.884899825508	-0.196000596234022\\
11.8688243615859	-0.195231897231672\\
11.8295998062009	-0.190722120811593\\
11.8295781957395	-0.189289857212472\\
11.7903877498025	-0.184714709436257\\
11.7903661393411	-0.183271225187752\\
11.751175693404	-0.178621324584075\\
11.7511540829427	-0.177352564502401\\
11.7315796159923	-0.175028158406342\\
11.7315532552122	-0.174265560894181\\
11.7119735877931	-0.171920495061272\\
11.7119448867219	-0.171077781110448\\
11.7120887197427	-0.0700920382308929\\
11.7159624407906	-0.033870391421555\\
11.7167645075263	-0.0311871220895217\\
11.7175686929098	-3.71258579434652e-13\\
11.7167645075263	0.0311871220895181\\
11.7159624407906	0.0338703914215568\\
11.7120887197427	0.0700920382308929\\
11.711944886722	0.0834882858107644\\
11.7119942286674	0.172194001702477\\
11.7315650961408	0.174616178656095\\
11.7316002568666	0.175294494688186\\
11.7511540829427	0.177352564502398\\
11.7512089168128	0.178943608167945\\
11.7903661393411	0.183271225187749\\
11.7904209732112	0.185014886575049\\
11.8295781957396	0.189289857212469\\
11.8296330296097	0.191000548299803\\
11.8688243615859	0.195231897231668\\
11.8848998255081	0.196000596234029\\
11.9071288451242	0.198244767310001\\
11.9301787757075	0.199210895012888\\
11.9454211733315	0.200819488429849\\
11.9673015395747	0.201830003086659\\
11.9837195792043	0.203543765044101\\
12.0057614767687	0.204544391930208\\
12.0220179850771	0.206223272817255\\
12.0442223180169	0.207214245384044\\
12.0603163909499	0.208859194610909\\
12.0826837964019	0.209840742335214\\
12.0986147968227	0.211452713286661\\
12.1211438250671	0.212424984013911\\
12.1369132026955	0.214005115348195\\
};


\addplot [TpReachStyle]
  table[row sep=crcr]{%
16.0018199787357	0.150957572130014\\
16.0151097003961	0.15448761354622\\
16.0798203082292	0.15448761354622\\
16.0919554756853	0.143569938229611\\
16.0929894646735	0.143569938229618\\
16.1013985515174	0.132567233912187\\
16.1025550724492	0.132567233912194\\
16.1078566196697	0.121564529594767\\
16.1092359104995	0.121564529594771\\
16.1118854896769	0.110561864762943\\
16.1130319788244	0.11056186476295\\
16.1136942331649	0.103752361182224\\
16.1146374728393	0.102299273163958\\
16.1158184491449	0.0850097999836947\\
16.1157083680781	-0.0888226393782467\\
16.1146374728393	-0.102299273163961\\
16.1136942331649	-0.103752361182263\\
16.1130319788244	-0.110561864762953\\
16.1118854896769	-0.110561864762939\\
16.1109168841851	-0.11319363352484\\
16.1092359104995	-0.121564529594771\\
16.1078014132961	-0.121658585493634\\
16.1074326193344	-0.122660624441483\\
16.1025550724492	-0.132567233912198\\
16.1013985515174	-0.132567233912191\\
16.1007551955735	-0.13366332875891\\
16.0929894646735	-0.143569938229625\\
16.0919554756853	-0.143569938229614\\
16.0910850440127	-0.144581004075512\\
16.0798203082291	-0.154487613546227\\
16.0151097003961	-0.154487613546227\\
16.0018199787356	-0.150957572130025\\
15.9787131898334	-0.141860072771042\\
15.9751422011385	-0.140895754348588\\
15.9656086999247	-0.136756661086842\\
15.9618068147809	-0.135805432089807\\
15.9475772832002	-0.130204051481329\\
15.9475397250948	-0.129336438707654\\
15.9466615787113	-0.128837218434512\\
15.9260129453307	-0.119321245128223\\
15.9218101925305	-0.11835135645676\\
15.9147130661939	-0.115310158421469\\
15.9146461792437	-0.113463227608786\\
15.9056746399617	-0.109485972579378\\
15.905661454389	-0.108915028159984\\
15.901340244324	-0.106994918180327\\
15.9012922646163	-0.105748997943152\\
15.8974010234338	-0.103979186541633\\
15.8923357511631	-0.102639167083108\\
15.8923100623582	-0.101473472422384\\
15.8886066127414	-0.0999892120214945\\
15.8836758632365	-0.098643919568282\\
15.8623657252293	-0.0986439195682927\\
15.8512696357057	-0.0959173212187956\\
15.8512696357057	0.0959173212187885\\
15.8623657252292	0.0986439195682856\\
15.8836758632365	0.0986439195682749\\
15.8923100623582	0.101473472422377\\
15.8923357511631	0.102639167083101\\
15.8974010234338	0.103979186541615\\
15.9013001408689	0.106030112639036\\
15.9013402443241	0.10699491818032\\
15.9056614543891	0.108915028159988\\
15.9056746399617	0.109485972579371\\
15.9066339161342	0.109732749806927\\
15.9066532455475	0.110271970988091\\
15.9146461792437	0.113463227608776\\
15.9147130661939	0.115310158421458\\
15.9218101925305	0.118351356456753\\
15.9260129453307	0.119321245128226\\
15.9475772832003	0.130204051481318\\
15.9618068147809	0.1358054320898\\
15.9656086999247	0.136756661086824\\
15.9751422011386	0.140895754348577\\
15.9787131898335	0.141860072771042\\
15.9906107872998	0.146506200571391\\
16.0018199787357	0.150957572130014\\
};

\addplot [TpReachStyle]
  table[row sep=crcr]{%
20.0390954900158	0.158445888919772\\
20.0405973869385	0.150023502253383\\
20.0431974897838	0.143098763948871\\
20.0443490755433	0.142984964277758\\
20.0449351223934	0.135692501658809\\
20.0451481267041	0.134093925512111\\
20.0455912654446	-0.125079653900375\\
20.0443490755433	-0.142984964277751\\
20.0431974897838	-0.143098763948871\\
20.0427563183353	-0.145164925199531\\
20.0418578679987	-0.146616200620002\\
20.0404449531604	-0.15052958999236\\
20.0390954900158	-0.158445888919776\\
20.0355531712	-0.164234629897152\\
20.0350255244048	-0.166753216194305\\
20.029908155453	-0.173450475853482\\
20.0282563204224	-0.179989892753209\\
20.0172567009325	-0.190523617831072\\
20.0055136414357	-0.199796577208851\\
19.9246887655753	-0.199796577208851\\
19.9246887655753	0.19979657720884\\
20.0055136414357	0.199796577208843\\
20.0172567009325	0.190523617831065\\
20.0282563204224	0.179989892753202\\
20.029908155453	0.173450475853485\\
20.0350255244048	0.166753216194298\\
20.0355531712	0.164234629897155\\
20.0390954900158	0.158445888919772\\
};


\addplot [SimStyle]
  table[row sep=crcr]{%
-0.199999999999999	0.199999999999999\\
-0.112951495376642	0.219003885489183\\
-0.0257223213130189	0.237152204646403\\
0.0713915446406865	0.256357684596015\\
0.168689052622849	0.274598859756239\\
0.188168994096639	0.278135334833085\\
0.281961906364479	0.294643539044806\\
0.379598046287541	0.310948398904102\\
0.487151245026716	0.327923857708999\\
0.578973536745767	0.341639576089264\\
0.686782316106093	0.356875609034063\\
0.794708864296044	0.371235368466731\\
0.912566891022124	0.385948298824243\\
0.971538524272894	0.392946897639721\\
1.08955895145743	0.406256968710288\\
1.20767255525604	0.418686556641401\\
1.33572193887683	0.431204666244874\\
1.36528429468707	0.433958402699535\\
1.49343697229444	0.445326352585571\\
1.62166210353068	0.455807190197465\\
1.75982139855518	0.466146459718221\\
1.89804311493275	0.475557863135442\\
2.03631764835687	0.48409087374365\\
2.15487511845084	0.490730806214071\\
2.30311255676882	0.498186312948633\\
2.45138823463144	0.504735476056094\\
2.55025661866948	0.508613562866906\\
2.70369759434566	0.513881057053766\\
2.86193559790006	0.518383837390076\\
2.94583597188361	0.520398454491755\\
3.10410005563111	0.523512581079864\\
3.26237429217612	0.525750340570461\\
3.34151366052996	0.526547383507875\\
3.50968649733692	0.527542574068075\\
3.67785729768754	0.527605094968646\\
3.7372102744616	0.52740806708859\\
3.96472125278046	0.525993321469024\\
4.13287147377276	0.524366214676004\\
4.36035230970687	0.521357437596787\\
4.52847609785297	0.518527855507529\\
4.7559154285269	0.513872976851907\\
4.92400430711298	0.509820123969032\\
5.15139081629177	0.503510791470877\\
5.31943666960616	0.498240717060831\\
5.54675986746429	0.490298555193245\\
5.71475537872836	0.483835392787498\\
5.9420170505512	0.474280407294103\\
6.1099780088172	0.466620275976865\\
6.33719662125031	0.455469469238761\\
6.5051226294709	0.446659757099141\\
6.74216622494132	0.433422786470668\\
6.90017569146265	0.424090277003344\\
7.13715878505653	0.40934973265216\\
7.29512586505014	0.399039466827247\\
7.54191321397926	0.382175990392128\\
7.68996371490801	0.371626567678703\\
8.08565965176334	0.343422940777206\\
8.48318099509778	0.315819017878155\\
8.88252384270377	0.288778612306409\\
9.28369259126241	0.262388098522493\\
9.6665024328533	0.238071771734994\\
9.68669637308773	0.236815836913539\\
10.0408247831908	0.215269131599808\\
10.0915281643411	0.212263383042689\\
10.4167052606631	0.193514383486928\\
10.498182530947	0.188969065117501\\
10.7941532455167	0.173020289009809\\
10.9066709104916	0.16720192831804\\
11.1834415999161	0.15350691303469\\
11.3170014008569	0.147228014106084\\
11.5847061574977	0.135328594779253\\
11.7291780642122	0.129303692995265\\
12.1424199537264	0.112209067731165\\
12.5559599785982	0.0949527242453527\\
12.9698141104032	0.077981954735165\\
13.3839914900296	0.0615921527978642\\
13.7984958079943	0.0459647650839763\\
14.1718010107703	0.032742779670329\\
14.2132921352864	0.0313296088620447\\
14.5972048929443	0.0187628780969185\\
14.6283423868634	0.0177828289743971\\
15.0228720990182	0.00584950915891724\\
15.0436430311096	0.00524544630726353\\
15.4591911557755	-0.00635792660033374\\
15.874984298314	-0.0171027951379159\\
16.2900841345213	-0.0270499389458934\\
16.703553159248	-0.0362367117333626\\
17.1153904478884	-0.0447090103195542\\
17.5255951949722	-0.0525150536012262\\
17.934166766345	-0.0597025932473301\\
18.3411012420722	-0.0664359467776414\\
18.7463964683102	-0.0728008713255299\\
19.1500538106012	-0.07876721248563\\
19.552074031555	-0.0843386078857264\\
19.9524576301466	-0.0895358745366863\\
};


\addplot [SimStyle]
  table[row sep=crcr]{%
0.199999999999999	0.199999999999999\\
0.288797085087563	0.18056682845414\\
0.387654820860089	0.159893925381077\\
0.48669895217963	0.140145707157004\\
0.585912444815726	0.121279658467291\\
0.595842501895181	0.119440001159241\\
0.695224534901275	0.101497820964575\\
0.804704438509585	0.0826838610123417\\
0.914335579392212	0.0647899150015121\\
0.994153751069295	0.0523281393162236\\
1.108816145133	0.0352199889077767\\
1.22879637109224	0.0182705659043627\\
1.34889759187627	0.00222562959914541\\
1.39415670297026	-0.00359204758440157\\
1.51840809327923	-0.018947839755473\\
1.64878378484101	-0.0341327554149764\\
1.77925606931596	-0.0484302039034894\\
1.79532612312617	-0.0501314547836458\\
1.93594100011626	-0.0644808491673849\\
2.0766422646095	-0.0779133037714814\\
2.19730427734721	-0.0887339180029798\\
2.34820504165715	-0.101394347632311\\
2.49918052778746	-0.113133203251468\\
2.5998677481731	-0.120471908364131\\
2.76102176667125	-0.131441766429845\\
2.92223513356208	-0.141505781568558\\
3.00286161091827	-0.146212721157337\\
3.17423174466047	-0.155525651897428\\
3.34564863676903	-0.163939411959642\\
3.4061586199019	-0.16670234389678\\
3.58771596331322	-0.174367767335909\\
3.76930943999063	-0.181130825369859\\
3.80966786296569	-0.182515101087272\\
4.00138910298452	-0.188519380497528\\
4.19313714481746	-0.193607827631983\\
4.21332249411218	-0.194091597984166\\
4.44523054502147	-0.199148449746946\\
4.61708366789751	-0.202330231973981\\
4.85938453342526	-0.206018444129722\\
5.02093409527514	-0.207970225107207\\
5.26327886189837	-0.21015621869978\\
5.42485398068167	-0.21113002710505\\
5.67733130678986	-0.211897364153401\\
5.82882580688639	-0.211926324830241\\
6.09142797343333	-0.211230796447904\\
6.23283418306621	-0.210473500738065\\
6.49543608361363	-0.208376835622772\\
6.63682637612479	-0.206885206272936\\
6.90948507950905	-0.203312412260367\\
7.0407541269671	-0.201272763490763\\
7.32346186492201	-0.196194675495249\\
7.44461128382139	-0.193738258648672\\
7.72726659579176	-0.187371233038569\\
7.84839289394031	-0.184375918222695\\
8.14108504487211	-0.17649491396763\\
8.25209501016174	-0.173272694278502\\
8.604620405944	-0.163309811849579\\
8.65487007703861	-0.161954472790942\\
9.05586297656744	-0.151557388020095\\
9.45506554549936	-0.141753794950052\\
9.85247499960236	-0.132387354564589\\
10.248091579325	-0.123416453008758\\
10.6419433207005	-0.114786260925396\\
11.034058688564	-0.106549399343724\\
11.4244393733905	-0.0988746542137839\\
11.8130862453987	-0.0919209766468931\\
12.19999926343	-0.0858411231702476\\
12.5859406062101	-0.0800484376377746\\
12.9716745890306	-0.0740722115366417\\
13.3572028718835	-0.0681670040979014\\
13.7425259538299	-0.0625108561675631\\
14.1276432508024	-0.0572214656702599\\
14.5125454757154	-0.0523673809177332\\
14.8972233269103	-0.0479836032529981\\
15.2816752846458	-0.0440822058638588\\
15.6658996549442	-0.0406563526929276\\
16.0498947069876	-0.0376862472098551\\
16.4345653114495	-0.0351637813897696\\
16.8208184089006	-0.0330624679614786\\
17.2086554802346	-0.0313285497832752\\
17.5980780947763	-0.0299103397355758\\
17.9890878859693	-0.0287594298068363\\
18.3817053048226	-0.0278148342333431\\
18.7759495885383	-0.0270276005046313\\
19.1718206067566	-0.0263698335297669\\
19.5693182591279	-0.025814393763671\\
19.9684424697192	-0.0253359996770435\\
};


\addplot [SimStyle]
  table[row sep=crcr]{%
0.199999999999999	-0.199999999999999\\
0.288803652969605	-0.219412425108992\\
0.377786668731179	-0.237995260666438\\
0.476850272186788	-0.257704689760345\\
0.57610084425847	-0.276462879120658\\
0.595971967321454	-0.280103554182581\\
0.695425983499018	-0.297767445690159\\
0.79503370093035	-0.314556775165833\\
0.894782024480818	-0.330503787043881\\
0.99465895746879	-0.345639190812104\\
1.10465903759306	-0.361385574050558\\
1.21478811751459	-0.376222172917966\\
1.32503407004955	-0.390183931414967\\
1.39524634323701	-0.398628396405556\\
1.51570240099112	-0.412334927901892\\
1.63626445230538	-0.425102637276286\\
1.75692143415536	-0.436968269847863\\
1.79715976425449	-0.440729052443306\\
1.91689897008796	-0.451357555282595\\
2.04781262748861	-0.462044611187459\\
2.17880197540715	-0.47180139276249\\
2.19999051259	-0.473294197477017\\
2.3310559450416	-0.482018878932099\\
2.4722660862839	-0.490466154098819\\
2.60344046937549	-0.497460536003988\\
2.74475217468924	-0.504110367474784\\
2.88610547431135	-0.509876174303425\\
3.00729350476225	-0.514137381264479\\
3.15880885438501	-0.518608183053555\\
3.31035155122337	-0.522158041619534\\
3.41139212905364	-0.52402775189077\\
3.56296692884012	-0.526107667353489\\
3.72465986963586	-0.527393790043277\\
3.8156159461431	-0.527705036802121\\
3.97731873352495	-0.527543845002398\\
4.13902114963535	-0.526489765330513\\
4.21987072915589	-0.525635336171177\\
4.45230218642202	-0.522400138610223\\
4.62408667790143	-0.519399702348085\\
4.85648073580349	-0.514509922631902\\
5.02823329258623	-0.510283246994632\\
5.26057866987561	-0.503744558938479\\
5.43229150650966	-0.498314597956103\\
5.66457863719747	-0.490178585746829\\
5.83624549754715	-0.48359584941236\\
6.0785628014633	-0.473505358143925\\
6.24008352022992	-0.466276425694513\\
6.49241332676367	-0.454223764991632\\
6.6437813410774	-0.446568991075665\\
6.90609928056912	-0.432580000199138\\
7.04731945341318	-0.424683084707382\\
7.31961781448424	-0.40876810767314\\
7.45069893849935	-0.40079689569956\\
7.74305202384215	-0.382332586719926\\
7.85392374456971	-0.375093394250214\\
8.15624447485396	-0.354731765599649\\
8.25700001931779	-0.347752697791226\\
8.53871016052556	-0.328580368112313\\
8.659200233798	-0.320643234879263\\
8.97110311719494	-0.300736692257775\\
9.05976710509114	-0.295233730805947\\
9.3989467260602	-0.274774101197206\\
9.45867704185997	-0.271265740445273\\
9.79643907675012	-0.251956181556363\\
9.85591877296257	-0.248650133285356\\
10.1823843545741	-0.231028222884156\\
10.2514882595945	-0.227415966284859\\
10.6060324509612	-0.209448357500076\\
10.6453381746422	-0.207518542754542\\
10.9591526763755	-0.192626634241368\\
11.0374313248709	-0.189066026929449\\
11.3206062385619	-0.176755181943893\\
11.4277777583728	-0.172344268065668\\
11.6902768972252	-0.162164641501246\\
11.8163825336491	-0.157607862895393\\
12.0680420872696	-0.149206092986404\\
12.2032461319362	-0.145092610319388\\
12.589092567802	-0.133636245255957\\
12.9746600699805	-0.122303105795041\\
13.3599624183751	-0.111519586175923\\
13.74500767389	-0.101560945499557\\
14.1105658395823	-0.0930113308470872\\
14.1297994239121	-0.0925874981152788\\
14.4758932032287	-0.0854377497949059\\
14.5143350697329	-0.0846993165062671\\
14.8685643152381	-0.0784107686365907\\
14.898610028363	-0.0779195839201634\\
15.2634278601675	-0.072464647431115\\
15.2826221940799	-0.0722031014609925\\
15.6567790456534	-0.0675886438106765\\
15.6663694846567	-0.0674821110427644\\
16.0498501257338	-0.0636739755916373\\
16.4147913170971	-0.0609238532225831\\
16.434040894744	-0.0608034402300319\\
16.8199155413446	-0.0588541045890416\\
17.2074689968361	-0.0576846633799235\\
17.5966971854434	-0.0571682832772495\\
17.9875967753183	-0.0571927339199654\\
18.3801649790323	-0.0576599037481778\\
18.7743994002927	-0.058484926842322\\
19.1702979202617	-0.0595950850684943\\
19.5678586158992	-0.0609286029585014\\
19.9670797030901	-0.0624334174125316\\
};


\addplot [SimStyle]
  table[row sep=crcr]{%
-0.199999999999999	0.199999999999999\\
-0.100553646650344	0.182355216141985\\
-0.000960037236698241	0.165562531269714\\
0.108747850254488	0.148035808356031\\
0.198619037470845	0.134403890253751\\
0.308307851739432	0.118612097092772\\
0.418395039026123	0.103656614111774\\
0.538615812006345	0.0882976037961143\\
0.599047999227484	0.0809447613022982\\
0.719440354004995	0.0669985397847768\\
0.839931567604648	0.0539370097775773\\
0.970562704000621	0.0407362903465973\\
1.00072168316885	0.0378248804143517\\
1.13146308788208	0.025770510707428\\
1.26228198672128	0.0145932150200245\\
1.40323958434799	0.00348398839263098\\
1.53975947391882	-0.00640426186528487\\
1.69092176387423	-0.0164127350056731\\
1.80632032089216	-0.0234257051786031\\
1.96766067746111	-0.0323830568277224\\
2.12904305675349	-0.0404207664588121\\
2.20974752451851	-0.0441136998070206\\
2.38126899122457	-0.0512806050274648\\
2.56290960038107	-0.0579204807382254\\
2.61336973506499	-0.0596015251115638\\
2.80274134392741	-0.06531402977377\\
2.99571606114065	-0.0702261129126782\\
3.01709960097936	-0.0707172497238489\\
3.22299184257522	-0.0749365124945349\\
3.42087887704469	-0.0781754296257056\\
3.64296423726891	-0.080944741825359\\
3.82466986268582	-0.0825905430283527\\
4.07703426933672	-0.084053419995481\\
4.22844903142612	-0.084523652317305\\
4.51107998034895	-0.0846885907303658\\
4.63220310851218	-0.0845055376828938\\
4.95518412027761	-0.0833781053383618\\
5.03592610974658	-0.0829666299816445\\
5.40934073753359	-0.0804985093932125\\
5.43961638149426	-0.0802621723303822\\
5.8432747999401	-0.0766902006389465\\
6.2469443844332	-0.0726395231157468\\
6.65066563495879	-0.0683870846225894\\
7.05443718993034	-0.0640464237012601\\
7.45825796170355	-0.0597416075103965\\
7.86212690794978	-0.0555959763652893\\
8.26609191644252	-0.051427324183468\\
8.67020194882057	-0.0471278766843426\\
9.07445710531207	-0.0428750598983925\\
9.47885695024689	-0.0387922392484512\\
9.88340059313745	-0.0349642699638402\\
10.2880514627387	-0.0315358103692418\\
10.6927745801669	-0.0285835344306022\\
11.0975713111206	-0.0260838448255498\\
11.502443089722	-0.0240258126075155\\
11.9073913377262	-0.0224047383148012\\
12.3123792121784	-0.0210401014279853\\
12.7173683474558	-0.0197925075057483\\
13.1223577770127	-0.0186995028647843\\
13.527346479532	-0.0177766296285675\\
13.9323334223703	-0.0170241674466922\\
14.3373288068855	-0.0164544997588578\\
14.7423420870937	-0.0160609566428285\\
15.1473712133923	-0.0158104934252776\\
15.552414245161	-0.0156735244423736\\
15.9574693373193	-0.0156229112333222\\
16.3622612115572	-0.0156922304437224\\
16.7665172743005	-0.0159057013169779\\
17.170239734384	-0.0162308975021617\\
17.5734307383179	-0.0166431147759596\\
17.9760923628692	-0.017123463053089\\
18.3782437539934	-0.0175742928474456\\
18.7799027768917	-0.017949845512284\\
19.1810695705699	-0.0182884227469984\\
19.5817442947608	-0.0186075353662716\\
19.9819271136552	-0.0189132793732227\\
};

\addplot [SimStyle]
  table[row sep=crcr]{%
-0.042635397452969	0.130629591617105\\
0.105428459345745	0.120259819212446\\
0.264906300499167	0.110010265584741\\
0.356029609282672	0.104550993627441\\
0.52555104993548	0.0951052884415908\\
0.705086220056003	0.0860303420225002\\
0.754963521966488	0.0836666163143853\\
0.944518420865492	0.0752613365740586\\
1.14407958962886	0.0673247648005351\\
1.15405830785135	0.0669507257515214\\
1.36362319333338	0.0595603388797521\\
1.55324527174828	0.0535832456153749\\
1.79278281862714	0.0468907042170379\\
1.95248083605133	0.0429064977756362\\
2.24195377276209	0.0364780545422683\\
2.35176199273621	0.034272127481767\\
2.66124236292118	0.0286716682163686\\
2.75109671561875	0.0272026179355436\\
3.10054970643566	0.0220812769119725\\
3.15047387350587	0.0214201917846317\\
3.53989981593776	0.0167886277420024\\
3.54988547631337	0.0166811208842255\\
3.94932583171274	0.012776515600212\\
4.34880359191457	0.00957485158724936\\
4.7483275258686	0.00697343650164228\\
5.1478940305073	0.00485099665676714\\
5.5475001209936	0.00310969834497143\\
5.94714324181021	0.00167201004959594\\
6.34680617383909	0.000602975652299165\\
6.74647256252153	-8.80969315630864e-05\\
7.14614173682325	-0.000504453644246894\\
7.54581327432748	-0.000707689723252969\\
7.94548686379983	-0.000733283662544437\\
8.34528653208477	-0.000651639986426034\\
8.7453379734642	-0.000518610580392931\\
9.14564335949397	-0.000333625000685345\\
9.54620478171754	-9.62100945649524e-05\\
9.9470242549839	0.000193044641889628\\
10.3481430485832	0.000532432249535475\\
10.7495997235884	0.0009188185335951\\
11.1513922370199	0.00134751511482634\\
11.5535186239051	0.00181284232803236\\
11.9559769946286	0.00230841211193678\\
12.3586055415259	0.00283343631043209\\
12.7612446479859	0.0033908464642316\\
13.1638957291586	0.0039775494858425\\
13.5665601513819	0.00458909445135802\\
13.9692392325784	0.005220614907536\\
14.3719266130927	0.00588111176457673\\
14.774616407829	0.00657227977479735\\
15.1773105504825	0.00728264503888809\\
15.5800108977996	0.0080047605747211\\
15.9827192335404	0.00873369477190877\\
16.3852855767985	0.00943174717896511\\
16.7875589282183	0.0100818255783039\\
17.1895395047751	0.0107012376108671\\
17.5912275296291	0.0112985094811435\\
17.9926232240476	0.0118771480146229\\
18.3937133162624	0.0123309341670712\\
18.794485313412	0.0126137267672028\\
19.1949406500243	0.0127840506787713\\
19.5950807595962	0.0128716652030541\\
19.99490703874	0.0128900857893335\\
};


\addplot [SimStyle]
  table[row sep=crcr]{%
0.0685851557912116	0.0283964301849622\\
0.231036632353817	0.0375810392529949\\
0.409494238773064	0.0467451136803092\\
0.46511637470989	0.0494134842259832\\
0.653559710648175	0.0578262970557546\\
0.842049950175575	0.0653265539075036\\
0.861893461432711	0.0660657281643289\\
1.06035127630663	0.0729583712174566\\
1.25884580557567	0.0789932043985502\\
1.47722540645207	0.0847206281957327\\
1.655923088306	0.0887528336279821\\
1.89421419304665	0.093290776575973\\
2.0530898545064	0.0958248345199344\\
2.30134193539677	0.0990430031025369\\
2.45029263930327	0.100566827443156\\
2.7283300337214	0.102674807357705\\
2.84748665018215	0.103307864034743\\
3.14537046252636	0.104250320413456\\
3.24466229708934	0.104376028510735\\
3.57231437916564	0.104188622000176\\
3.64181406030251	0.104038147038096\\
3.99922779150596	0.10271367047514\\
4.03893903949393	0.102513171774749\\
4.43602630657133	0.100025853377101\\
4.83306711170254	0.0967940798617448\\
5.2300635615145	0.0929665472049379\\
5.62701804085042	0.0886623603393879\\
6.02393305223914	0.0839801976611376\\
6.42081112667182	0.0790040491725463\\
6.81765476933076	0.0738068448613731\\
7.2144664238665	0.0684527910939963\\
7.6112484477454	0.0629989198034053\\
8.00800309512178	0.057496159688192\\
8.40489289390629	0.052099386439707\\
8.80207861841892	0.0469235235365701\\
9.19956001819323	0.0419708617532351\\
9.59733700427465	0.0372573331572603\\
9.99540954662013	0.0328064287895664\\
10.3938020512809	0.0287350538595277\\
10.7925366196035	0.0251131022733553\\
11.1916101376619	0.0219206373549383\\
11.5910197908401	0.0191584484989882\\
11.9907629140677	0.0168384511459401\\
12.3906982944543	0.0146822995221285\\
12.7906871950175	0.0124961877991545\\
13.1907304892738	0.0103797867916882\\
13.5908287766372	0.00838987113906242\\
13.9909824593153	0.00655433959006047\\
14.3911769515635	0.00496326789043522\\
14.7913983422671	0.00364896065899245\\
15.1916480522319	0.00255536116275223\\
15.5919275673441	0.00164272531733189\\
15.9922383824917	0.000880991856121227\\
16.3925317816707	0.000226442253648429\\
16.7927586218829	-0.000337169400669524\\
17.1929197007449	-0.000804887525788445\\
17.5930157718583	-0.00118082934187669\\
17.9930475573943	-0.00147364451524368\\
18.3930091346347	-0.00177844627687307\\
18.7928950306688	-0.00214319884434744\\
19.1927065775798	-0.00253174356979713\\
19.5924450431585	-0.00292900408013352\\
19.9921116356161	-0.00333152790798152\\
};


\addplot [SimStyle]
  table[row sep=crcr]{%
0.020870010686334	-0.186824042900049\\
0.125882620359658	-0.201874385978485\\
0.235775672633068	-0.21677346090139\\
0.355781436644534	-0.232101995249305\\
0.42060612075468	-0.239991878434108\\
0.540782634752908	-0.253927865848137\\
0.671081715330722	-0.268070651248763\\
0.801480306933829	-0.281271968788843\\
0.821549709587455	-0.283222537711065\\
0.952048928718153	-0.295397933393691\\
1.092671761065	-0.3075706844146\\
1.22331989966326	-0.318045646035845\\
1.36408397175101	-0.32847894386736\\
1.51496924895465	-0.338730528717647\\
1.62565702902765	-0.34566679196789\\
1.77664104411381	-0.35436809689061\\
1.93774188872033	-0.362729139773677\\
2.02838146661864	-0.367031771434529\\
2.18590921641469	-0.373847345694681\\
2.35719626395717	-0.380340215063619\\
2.43138228571149	-0.382866399177519\\
2.60272243238947	-0.388064005824081\\
2.78417391208534	-0.392632664510934\\
2.83458222214771	-0.393736359949894\\
3.01606785545874	-0.397131892701143\\
3.19757420956559	-0.399648047527172\\
3.23791128788399	-0.400090840321166\\
3.42952174145638	-0.401630452368504\\
3.62114408393713	-0.402260427231369\\
3.64131536306514	-0.402274936306728\\
3.83294567641352	-0.401930162164955\\
4.03466487066533	-0.400640984063287\\
4.04475083473464	-0.400552006934511\\
4.33722586012534	-0.397489757798176\\
4.44815496014713	-0.396083282092231\\
4.73049431684023	-0.391862752577037\\
4.85148511086436	-0.389765458596841\\
5.12368704965659	-0.384401907052037\\
5.25473319038119	-0.381499361376179\\
5.52687539758218	-0.374807257537864\\
5.65789165257849	-0.371267787323532\\
5.94004582771553	-0.362954091821525\\
6.06095421616824	-0.359106914049772\\
6.32289600719953	-0.350141775644225\\
6.4639276688655	-0.344956391425772\\
6.74596128798736	-0.333874278168185\\
6.86682059686527	-0.32884616502221\\
7.1588666638744	-0.316038068656294\\
7.2696313839157	-0.310945255005198\\
7.57168507482885	-0.296429426594429\\
7.67235925162912	-0.291393591274101\\
7.98441622289062	-0.275182481067382\\
8.07500408526391	-0.270311071529026\\
8.41694786237539	-0.252254704967417\\
8.4772344892687	-0.249159752639923\\
8.86119514189421	-0.229980379958281\\
8.87870735843879	-0.229126028560955\\
9.27941289410787	-0.210025602956499\\
9.67934823328569	-0.191800203487077\\
10.0785147352433	-0.174481845186779\\
10.4072550677513	-0.161046269486523\\
10.4769219944032	-0.158312754093807\\
10.805040860806	-0.14598262895116\\
10.8745758730424	-0.143487740549322\\
11.1921563423281	-0.132644855091883\\
11.2714761364448	-0.13008346710561\\
11.5785558587521	-0.120755278840338\\
11.6676233240418	-0.11823214909225\\
11.9543589958702	-0.110701668490329\\
12.0630170860066	-0.108094384292404\\
12.4579929101662	-0.0985388471707118\\
12.8528953551148	-0.0887325745214511\\
13.2477329679264	-0.0791304707905347\\
13.6425098906628	-0.0700197598496324\\
14.037226763539	-0.0615684498472575\\
14.412136682279	-0.0543530856368974\\
14.4318664039623	-0.0539983492329412\\
14.8264092919029	-0.0473946385867627\\
15.2208513932359	-0.0416674261087948\\
15.6151894194344	-0.0367375331445174\\
16.0094206263491	-0.0325306780154868\\
16.4038703253695	-0.029036183569989\\
16.7988652562233	-0.0262300980831185\\
17.194405504781	-0.0240270944275238\\
17.5904915239894	-0.0223501804653274\\
17.9871240385033	-0.0211299209809681\\
18.3843432983554	-0.0203042148017083\\
18.782187052077	-0.0198176776207362\\
19.1806525999208	-0.0196202931897247\\
19.5797374391018	-0.0196673018641498\\
19.9794392312689	-0.0199189294674937\\
};


\addplot [ShiftX0Style]
  table[row sep=crcr]{%
19.8	-0.199999999999999\\
20.2	-0.199999999999999\\
20.2	0.199999999999999\\
19.8	0.199999999999999\\
19.8	-0.199999999999999\\
};


\addplot [FinalReachStyle]
  table[row sep=crcr]{%
20.0456951281523	0.16073179767719\\
20.0456951281523	-0.160731797677187\\
20.0424156958993	-0.161021568313057\\
20.0418863310997	-0.170762394234313\\
20.03787382736	-0.18286014302749\\
20.0193807222556	-0.195236149741824\\
20.0094284269358	-0.199796577208847\\
19.9246887655753	-0.199796577208851\\
19.9246887655753	0.19979657720884\\
20.0094284269358	0.199796577208843\\
20.0193807222556	0.19523614974182\\
20.03787382736	0.182860143027483\\
20.0418863310997	0.170762394234313\\
20.0422724033283	0.167933949188399\\
20.0424198720401	0.160731797677176\\
20.0456951281523	0.16073179767719\\
};
		\coordinate (bot) at (rel axis cs:1,0);
	\end{groupplot}
	% legend
	\path (top|-current bounding box.south)--
	coordinate(legendpos)
	(bot|-current bounding box.south);
	\matrix[
	matrix of nodes,
	nodes = {text width=2.7cm,align=left},
	inner sep=0.2em,
	draw
	]at([yshift=-5ex]legendpos)
	{
		\ref{p:IntReach} $\mathcal{R}_{\text{cl}}([0;t_f])$
		& \ref{p:X0} $\mathcal{X}^{(0)}$
		& \ref{p:TpReach} $\mathcal{R}_{\text{cl}}(i\frac{t_f}{N})$ \\
		\ref{p:ShiftX0} $\langle x_{\text{f}}, G_{\mathcal{X}^{(0)}} \rangle_Z$
		& \ref{p:FinalReach} $\mathcal{R}_{\text{cl}}(t_f)$
		& \ref{p:Sim} Simulation \\
	};
	
\end{tikzpicture}
\end{center}

\end{document}