% Note that the a4paper option is mainly intended so that authors in
% countries using A4 can easily print to A4 and see how their papers will
% look in print - the typesetting of the document will not typically be
% affected with changes in paper size (but the bottom and side margins will).
% Use the testflow package mentioned above to verify correct handling of
% both paper sizes by the user's LaTeX system.
%
% Also note that the "draftcls" or "draftclsnofoot", not "draft", option
% should be used if it is desired that the figures are to be displayed in
% draft mode.
%
\documentclass{amsproc}

% Sample optional packages to use
\usepackage{amsthm}
\usepackage{graphicx}
\usepackage{amscd}
\usepackage{amssymb}
\usepackage{epsf}

\usepackage{amsmath,amsfonts}
\usepackage{mathrsfs} %for \mathscr
\usepackage{psfrag}
\usepackage{algorithm}  %for algorithm environment
\usepackage{algpseudocode}
\usepackage{subfigure}
\usepackage{pifont}
\usepackage{color}
%\usepackage[amsmath,hyperref,thmmarks]{ntheorem}  %better package for proof environment
\usepackage{cite}
\usepackage{url}

\usepackage{chngcntr} %for changing the paragraph counter

\newtheorem{definition}{Definition}[section]
\newtheorem{lemma}{Lemma}[section]
\newtheorem{theorem}{Theorem}[section]
\newtheorem{proposition}{Proposition}[section]
\newtheorem{corollary}{Corollary}[section]
\newtheorem{property}{Property}[section]
\newtheorem{specification}{Specification}[section]
\newtheorem{remark}{Remark}[section]

%Indexing elements----------------------
\renewcommand{\^}[1]{^{(#1)}}
\renewcommand{\th}[1]{#1^\text{th}}
%---------------------------------------


\counterwithout{paragraph}{subsubsection} % removes paragraph from the subsubsections
\counterwithin{paragraph}{section} % makes paragraph depend on section



% *** ALIGNMENT PACKAGES ***
%
%\usepackage{array}
% Frank Mittelbach's and David Carlisle's array.sty patches and improves
% the standard LaTeX2e array and tabular environments to provide better
% appearance and additional user controls. As the default LaTeX2e table
% generation code is lacking to the point of almost being broken with
% respect to the quality of the end results, all users are strongly
% advised to use an enhanced (at the very least that provided by array.sty)
% set of table tools. array.sty is already installed on most systems. The
% latest version and documentation can be obtained at:
% http://www.ctan.org/tex-archive/macros/latex/required/tools/


% correct bad hyphenation here
\hyphenation{op-tical net-works semi-conduc-tor}


\begin{document}


\title[Input Error Bounds in Reachability Analysis]{Input Error Bounds in Reachability Analysis}


%    author one information
\author{Matthias~Althoff}
\address{Department of Computer Science, Technische Universit\"at M\"unchen, 85748 Garching, Germany}
\email{althoff@tum.de}



\begin{abstract}
 This brief presents two different approaches for computing truncation errors in the computation of input solutions of linear systems for reachability analysis.
\end{abstract}


% make the title area
\maketitle




% % Note that keywords are not normally used for peerreview papers.
% \begin{IEEEkeywords}
% Reachability analysis, linear systems, Krylov subspace, uncertain inputs, large-scale systems.
% \end{IEEEkeywords}

%\IEEEpeerreviewmaketitle


\section{Truncation Error of Exponential Matrix} 

We wish to over-approximate the exponential matrix $e^{At}$ by a finite Taylor expansion of $e^{At}$ of $\eta^{\text{th}}$ order with error matrix $E$ 

\begin{equation}\label{eq:taylorSeries}
\begin{split}
	e^{A \, t} =& \sum_{i=0}^{\eta} \frac{1}{i!}(A \, t)^i + E(t).
\end{split}
\end{equation}
Two enclosures for $E(t)$ have been proposed. The first one is presented in \cite[eq.~(3.3)]{Althoff2010a}:
\begin{gather}\label{eq:matrixExponentialRemainder_thesis}
 E(t) \in \mathcal{E}(t)  = \mathbf{[-1,1]}^{n \times n} \Phi(t), \\
 \Phi(t) = \frac{(\|A\|_{\infty}t)^{\eta+1}}{(\eta+1)!}\frac{1}{1-\epsilon}, \quad \epsilon=\frac{\|A\|_{\infty}t}{\eta+2}\overset{!}{<}1. \notag
\end{gather}
The second enclosure is presented in \cite[Prop.~2]{Althoff2011a}:
\begin{equation}\label{eq:matrixExponentialRemainder_hscc}
 \mathcal{E}(t)= [-W(t),W(t)], \quad W(t)=e^{|A|t} - \sum_{i=0}^{\eta} \frac{t^i}{i!}|A|^i.
\end{equation}
Since the approach in \cite[Prop.~2]{Althoff2011a} converges much faster, while not adding much computation time, only \eqref{eq:matrixExponentialRemainder_hscc} is implemented in CORA. There might exist cases, where \eqref{eq:matrixExponentialRemainder_thesis} has a better compromise of accuracy and computation time, but this is rarely the case and in those cases, the difference is marginal.

\section{Truncation Error of Integrated Exponential Matrix}

We also require the input solution for constant input $u_{\mathtt{const}}$:
\begin{equation*}
  x_{p,\mathtt{const}}(t) :=  \int_{0}^{t} e^{A (t - \tau)} \mathtt{d}\tau \, u_{\mathtt{const}}.
\end{equation*}
When $A$ is invertiable, the solution can be obtained as presented in \cite[eq.~(3.6)]{Althoff2010a}:
\begin{equation} \label{eq:constInputSolution}
  x_{p,\mathtt{const}}(t) =  A^{-1} (e^{A t} - I) u_{\mathtt{const}}.
\end{equation}
Since $A$ is not always invertible, one can insert \eqref{eq:taylorSeries} into \eqref{eq:constInputSolution}:
\begin{equation} \label{eq:constInputSolution_Taylor}
  x_{p,\mathtt{const}}(t) =  \sum_{i=0}^{\eta} \frac{1}{(i+1)!}A^i \, t^{i+1} + \hat{E}(t),
\end{equation}
where (see \cite[Appendix~A.1]{Althoff2010a})
\begin{gather}\label{eq:matrixExponentialRemainderIntegrated_thesis}
 \hat{E}(t) \in \hat{\mathcal{E}}(t)  = \mathbf{[-1,1]}^{n \times n} \hat{\Phi}(t), \\
 \hat{\Phi}(t) = \Phi(t) \, t. \notag
\end{gather}
The result for $\hat{\mathcal{E}}(t)$ can be generalized:
\begin{proposition}
 For any monotonic $\mathcal{E}(t)$, i.e., $\forall \underline{t}, \overline{t}, \underline{t} \leq \overline{t}: \mathcal{E}(\underline{t}) \subseteq \mathcal{E}(\overline{t})$, we have that
 \begin{equation*}
  \hat{E}(t) = \int_{0}^{t} E(\tau) \mathtt{d}\tau \in \mathcal{E}(t) \, t.
 \end{equation*}
\end{proposition}
\begin{proof}
 Since $\forall \tau < t: \mathcal{E}(\tau) \subseteq \mathcal{E}(t)$ it directly follows that 
 \begin{equation*}
  \hat{E}(t) = \int_{0}^{t} E(\tau) \mathtt{d}\tau \in \int_{0}^{t} \mathcal{E}(\tau) \mathtt{d}\tau \subseteq \int_{0}^{t} \mathcal{E}(t) \mathtt{d}\tau = \mathcal{E}(t) \, t = \hat{\mathcal{E}}(t).
 \end{equation*}
\end{proof}

A further option is to rewrite the integration of the exponential matrix as a different exponential matrix:
\begin{proposition}
 \begin{equation*}
 x_\mathtt{int}(t) = \int_0^\delta e^{A (\delta - t)} \mathtt{d}t \, u_{\mathtt{const}}.
\end{equation*}
is the solution of
\begin{equation} \label{eq:ODEinputSol}
 \begin{bmatrix} \dot{x}_\mathtt{int} \\ \dot{x} \end{bmatrix} = \underbrace{\begin{bmatrix} 0 & I \\ 0 & A \end{bmatrix}}_{=:A_\mathtt{int}} \begin{bmatrix} x_\mathtt{int} \\ x \end{bmatrix}, x(0) = \underbrace{\begin{bmatrix} 0 \\ u_{\mathtt{const}} \end{bmatrix}}_{=: u_\mathtt{int}}
\end{equation}
\end{proposition}
\begin{proof}
 Eq.~\eqref{eq:ODEinputSol} can be interpreted as two multivariate differential equations: The solution of $\dot{x} = A x$ is $x(t) = e^{A \delta} u$ and the one of $\dot{x}_\mathtt{int}(\delta) = x$ is $\int_0^\delta x(t) \mathtt{d}t = \int_0^\delta e^{A \delta} u \mathtt{d}t \overset{u=\mathtt{const}}{=} \int_0^\delta e^{A \delta} \mathtt{d}t \, u$.
\end{proof}
The disadvantage of this technique is that the exponential matrix has to be computed with twice as many dimensions.


\section{Truncation Error of Time Interval Solutions of Inputs}

We consider the computation of $\tilde{\mathcal{F}}$ as defined in \cite[eq.~(3.9)]{Althoff2010a}:
\begin{equation*}
  t\in[0,r] : \tilde{\mathcal{F}} = A^{-1}(e^{A \, t}-I) - \frac{t}{r}A^{-1}(e^{A \, r} - I) = A^{-1}\underbrace{\left[e^{At}-I - \frac{t}{r}(e^{Ar}-I)\right]}_{\in\mathcal{F}}.
\end{equation*}
Since $A^{-1}$ does not always exist, we derive the error differently. Inserting \eqref{eq:taylorSeries} results in 
\begin{equation*}
\begin{split}
  t\in[0,r] : \tilde{\mathcal{F}} &= \sum_{i=0}^{\eta} \frac{1}{(i+1)!}A^i \, t^{i+1} + \hat{E}(t) - \frac{t}{r}(\frac{1}{(i+1)!}A^i \, r^{i+1} + \hat{E}(r)) \\
  &= \sum_{i=2}^{\eta} (t^{i+1} - tr^{i} )\frac{1}{(i+1)!}A^i  + \hat{E}(t) - \frac{t}{r}\hat{E}(r) 
\end{split}
\end{equation*}
As shown in \cite[Prop.~3.1]{Althoff2010a}, one can bound $\{t^{i} - tr^{i-1}|t\in[0,r]\}$ by $[(i^{\frac{-i}{i-1}}-i^{\frac{-1}{i-1}})r^i,0]$ so that we can bound $\{t^{i+1} - tr^{i}|t\in[0,r]\}$ by $[((i+1)^{\frac{-i-1}{i}}-(i+1)^{\frac{-1}{i}})r^{i+1},0]$. We also need the following proposition that applies to symmetric bounds:
\begin{proposition}
 For any symmetric $\mathcal{E}(t)=[-W,W]$, which is also monotonically increasing, we have that
 \begin{equation*}
  \Big\{\hat{E}(t) - \frac{t}{r}\hat{E}(r) \Big| t\in[0,r], \hat{E}(t) \in \hat{\mathcal{E}}(t) \Big\} \subseteq \hat{\mathcal{E}}(r).
 \end{equation*}
\end{proposition}
\begin{proof}
We first bound $\hat{E}(t) - \frac{t}{r} \hat{E}(r)$ for $t\in[0,r]$. Using \eqref{eq:matrixExponentialRemainder_hscc}{eq:matrixExponentialRemainder} we have that (see \cite[Lemma III.3]{Althoff2020c})
\begin{equation*}
 \begin{split}
  & \hat{E}(t) - \frac{t}{r} \hat{E}(r) \\
  \leq & \bigg|\hat{E}(t) - \frac{t}{r} \hat{E}(r)\bigg| \\
  = & \bigg|\int_0^t E(\tau) \mathtt{d}\tau - \frac{t}{r} \int_0^r E(\tau) \mathtt{d}\tau \bigg| \\
  \overset{\eqref{eq:matrixExponentialRemainder_hscc}}{=} & \bigg|\sum_{i=\eta+1}^{\infty} \frac{t^{i+1}}{(i+1)!}|A|^i - \frac{t}{r}\sum_{i=\eta+1}^{\infty} \frac{r^{i+1}}{(i+1)!}|A|^i\bigg| \\
  = & \bigg|\sum_{i=\eta+1}^{\infty} (t^{i+1} - tr^{i})\frac{|A|^i}{(i+1)!} \bigg| \\
  = & \sum_{i=\eta+1}^{\infty} \Big|t^{i+1} - tr^{i}\Big|\frac{|A|^i}{(i+1)!}  \\
  \leq & \sum_{i=\eta+1}^{\infty} r^{i+1} \frac{|A|^i}{(i+1)!} = \hat{E}(r).
 \end{split}
\end{equation*}
Thus, $\forall t\in[0,r]: \, \hat{E}(t) - \frac{t}{r} \hat{E}(r) \leq \hat{E}(r)$, which completes the proof.
\end{proof}

\bibliography{althoff_own,althoff_other}
\bibliographystyle{plain}


\end{document}


