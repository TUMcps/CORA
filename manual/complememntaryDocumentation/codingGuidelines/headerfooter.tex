\subheading{Header and footer}

A CORA function has the following header and footer: \bigskip

{\small \setstretch{1} % This file was automatically created from the m-file 
% "m2tex.m" written by USL. 
% The fontencoding in this file is UTF-8. 
%  
% You will need to include the following two packages in 
% your LaTeX-Main-File. 
%  
% \usepackage{color} 
% \usepackage{fancyvrb} 
%  
% It is advised to use the following option for Inputenc 
% \usepackage[utf8]{inputenc} 
%  
  
% definition of matlab colors: 
\definecolor{mblue}{rgb}{0,0,1} 
\definecolor{mgreen}{rgb}{0.13333,0.5451,0.13333} 
\definecolor{mred}{rgb}{0.62745,0.12549,0.94118} 
\definecolor{mgrey}{rgb}{0.5,0.5,0.5} 
\definecolor{mdarkgrey}{rgb}{0.25,0.25,0.25} 
  
\DefineShortVerb[fontfamily=courier,fontseries=m]{\$} 
\DefineShortVerb[fontfamily=courier,fontseries=b]{\#} 
  
\noindent                                             
 \hspace*{-1.6em}{\scriptsize 1}$  $\color{mblue}$function$\color{black}$ [argout1,argout2] = fun(argin1,argin2,varargin)$\\
 \hspace*{-1.6em}{\scriptsize 2}$  $\color{mgreen}$% fun - <text1>$\color{black}$$\\
 \hspace*{-1.6em}{\scriptsize 3}$  $\color{mgreen}$%$\color{black}$$\\
 \hspace*{-1.6em}{\scriptsize 4}$  $\color{mgreen}$% Description:$\color{black}$$\\
 \hspace*{-1.6em}{\scriptsize 5}$  $\color{mgreen}$%    <text2>$\color{black}$$\\
 \hspace*{-1.6em}{\scriptsize 6}$  $\color{mgreen}$%$\color{black}$$\\
 \hspace*{-1.6em}{\scriptsize 7}$  $\color{mgreen}$% Syntax:$\color{black}$$\\
 \hspace*{-1.6em}{\scriptsize 8}$  $\color{mgreen}$%    argout1 = fun(argin1,argin2)$\color{black}$$\\
 \hspace*{-1.6em}{\scriptsize 9}$  $\color{mgreen}$%    argout1 = fun(argin1,argin2,argin3)$\color{black}$$\\
 \hspace*{-2em}{\scriptsize 10}$  $\color{mgreen}$%    [argout1,argout2] = fun(argin1,argin2)$\color{black}$$\\
 \hspace*{-2em}{\scriptsize 11}$  $\color{mgreen}$%$\color{black}$$\\
 \hspace*{-2em}{\scriptsize 12}$  $\color{mgreen}$% Inputs:$\color{black}$$\\
 \hspace*{-2em}{\scriptsize 13}$  $\color{mgreen}$%    argin1 - <text_argin1>$\color{black}$$\\
 \hspace*{-2em}{\scriptsize 14}$  $\color{mgreen}$%    argin2 - <text_argin1>$\color{black}$$\\
 \hspace*{-2em}{\scriptsize 15}$  $\color{mgreen}$%    argin3 - (optional) <text_argin3>$\color{black}$$\\
 \hspace*{-2em}{\scriptsize 16}$  $\color{mgreen}$%$\color{black}$$\\
 \hspace*{-2em}{\scriptsize 17}$  $\color{mgreen}$% Outputs:$\color{black}$$\\
 \hspace*{-2em}{\scriptsize 18}$  $\color{mgreen}$%    argout1 - <text_argout1>$\color{black}$$\\
 \hspace*{-2em}{\scriptsize 19}$  $\color{mgreen}$%    argout2 - <text_argout2>$\color{black}$$\\
 \hspace*{-2em}{\scriptsize 20}$  $\color{mgreen}$%$\color{black}$$\\
 \hspace*{-2em}{\scriptsize 21}$  $\color{mgreen}$% Example: $\color{black}$$\\
 \hspace*{-2em}{\scriptsize 22}$  $\color{mgreen}$%    argin1 = 1;$\color{black}$$\\
 \hspace*{-2em}{\scriptsize 23}$  $\color{mgreen}$%    argin2 = 2;$\color{black}$$\\
 \hspace*{-2em}{\scriptsize 24}$  $\color{mgreen}$%    argout = fun(argin1,argin2);$\color{black}$$\\
 \hspace*{-2em}{\scriptsize 25}$  $\color{mgreen}$%$\color{black}$$\\
 \hspace*{-2em}{\scriptsize 26}$  $\color{mgreen}$% References:$\color{black}$$\\
 \hspace*{-2em}{\scriptsize 27}$  $\color{mgreen}$%    [1] <authors1>, "<papertitle1>", <proceedings/journal1>.$\color{black}$$\\
 \hspace*{-2em}{\scriptsize 28}$  $\color{mgreen}$%    [2] <authors2>, "<papertitle2>", <proceedings/journal2>.$\color{black}$$\\
 \hspace*{-2em}{\scriptsize 29}$  $\color{mgreen}$%$\color{black}$$\\
 \hspace*{-2em}{\scriptsize 30}$  $\color{mgreen}$% Other m-files required: none$\color{black}$$\\
 \hspace*{-2em}{\scriptsize 31}$  $\color{mgreen}$% Subfunctions: none$\color{black}$$\\
 \hspace*{-2em}{\scriptsize 32}$  $\color{mgreen}$% MAT-files required: none$\color{black}$$\\
 \hspace*{-2em}{\scriptsize 33}$  $\color{mgreen}$%$\color{black}$$\\
 \hspace*{-2em}{\scriptsize 34}$  $\color{mgreen}$% See also: <otherfun1>, <otherfun2>$\color{black}$$\\
 \hspace*{-2em}{\scriptsize 35}$  $\\
 \hspace*{-2em}{\scriptsize 36}$  $\color{mgreen}$% Author:        <author1>, <author2>$\color{black}$$\\
 \hspace*{-2em}{\scriptsize 37}$  $\color{mgreen}$% Written:       DD-Month-YYYY$\color{black}$$\\
 \hspace*{-2em}{\scriptsize 38}$  $\color{mgreen}$% Last update:   --- or DD-Month-YYYY (<initials>, <explanation1>)$\color{black}$$\\
 \hspace*{-2em}{\scriptsize 39}$  $\color{mgreen}$% Last revision: --- or DD-Month-YYYY (<initials>, <explanation1>)$\color{black}$$\\
 \hspace*{-2em}{\scriptsize 40}$  $\\
 \hspace*{-2em}{\scriptsize 41}$  $\color{mgreen}$%------------- BEGIN CODE --------------$\color{black}$$\\
 \hspace*{-2em}{\scriptsize 42}$  $\\
 \hspace*{-2em}{\scriptsize 43}$  $\color{mgreen}$% code$\color{black}$$\\
 \hspace*{-2em}{\scriptsize 44}$  $\\
 \hspace*{-2em}{\scriptsize 45}$  $\color{mgreen}$%------------- END OF CODE --------------$\color{black}$$\\ 
  
\UndefineShortVerb{\$} 
\UndefineShortVerb{\#}}

Let us now look at the respective parts in detail:
\begin{itemize}
	\item Function text \mcomment{<text1>}:
	This text describes in simple words what the function does.
	Do \textbf{not} use mathematical expressions, object names, or programming variables here.
	\item Description \mcomment{<text2>} (optional):
	Mathematical description of the function.
	\item Function syntax \mcomment{Syntax}:
	Here, we give some examples on how to call the function properly.
	Not all variations have to be included (apart from object constructors).
	\item Input arguments \mcomment{Inputs}:
	The input arguments are described in more detail one after another.
	If an input argument is optional, write \mcomment{(optional)} before the description.
	\item Output arguments \mcomment{Outputs}:
	Same as for input arguments.
	\item Example call \mcomment{Example}:
	Here, we provide an example function call.
	The input arguments are defined and the function output is provided by calling the function.
	For some methods, one can also plot the result.
	Do not use randomized functions such as \verb|randPoint| or \verb|generateRandom| here.
	\item Resources \mcomment{References}:
	List all resources which are required for the function.
	There is no bibliographic style to follow, each reference should be described exhaustively enough so that one can find it.
	Using \mcomment{[1], [2], ...}, one can refer to them in the function text or in the code itself.
	\item Additional information: ...
	\item Related functions \mcomment{See also}:
	List functions \mcomment{<otherfun1>,<otherfun2>,...} that have similar content to this function.
	In general, these other functions are helpful to understand this function.
	\item Empty line: This empty line (without \mcomment{\&}!) is important as the command \verb|help fun| only display the first connect sequence of comments on the command window.
	Since the authors and the changelog are irrelevant to the user, they are hidden by this extra empty line.
	\item Authorship and changelog:
	The author(s) \mcomment{<author1>,<author2>} are mentioned in order of influence on the function.
	One can be considered as an author if a substantial influence on the function has been made.
	Notably, this includes simple syntax changes.
	The date after \mcomment{Written} is the date on which the function was either written or committed for the first time.
	If functions change name, the original date is used.
	The fields \mcomment{Last update} and \mcomment{Last revision} can either be left empty using \mcomment{---} or filled with a date and the initials of the author of the changes with an explanation of the made changes.
	Updates are usually smaller changes compared to revisions.	
	Finally, please ensure vertical alignment.
	\item Blocks for begin and end code:
	Please copy these from existing functions.
\end{itemize}