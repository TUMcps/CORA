\subheading{Input and output arguments}

The naming convention from \tabref{tab:contSet} also applies to input and output arguments.
(Currently, many functions still use the generic \verb|obj|.)

If a function \verb|fun| within a class \verb|class| does not take an object from that same class as one of their input arguments, it must be declared in the constructor as follows:
\begin{Verbatim}
methods (static=true)
	argout = fun(argin);
end
\end{Verbatim}
Then, these functions can be called using \verb|class.fun|.
Examples include \verb|generateRandom| or \verb|enclosePoints|.

\smallskip

Name-value pairs are a convenient method to integrate an unordered and incomplete list of optional input arguments.
MATLAB uses these name-value pairs generically for functions such as \verb|plot|.
Name-value pairs are comprised of a name (char-array) and a value (any single variable).
As the name suggests, they always come in a multiple of $2$.
Name-value pairs are not checked in the input argument validation via \verb|inputArgsCheck| (see below).
The variable name convention for the entire list of name-value pairs is \verb|NVpairs|.
The function \verb|readNameValuePair| goes through the list in search of a desired name and outputs the corresponding value as well as the redacted list.
Furthermore, one can define a check function and a default value which should be used in case the name-value pair is not provided.
Moreover, the function \verb|checkNameValuePairs| checks whether there are any name-value pairs that do not belong to the calling function.
In that case, an error is thrown.

\smallskip

For optional input arguments, we consistently use \verb|varargin|.
This has to do with the input argument check, see below.

\smallskip

To ensure correctness of input arguments, there is a standardized way to check input arguments comprised of a two-step process:
\begin{enumerate}
	\item Setting default values:
		All variables (except name-value pairs) defined via \verb|varargin| are first processed via the function \verb|setDefaultValues|.
		This function takes in the entire list of variable input arguments.
		All unprovided optional variables are overwritten by default values.
		Thus, all optional variables are defined after calling \verb|setDefaultValues|.
	\item The function \verb|inputArgsCheck| checks all input arguments for correctness.
	In case an input argument is considered 'wrong', an error is thrown.
	There are two different checks, attribute checks and string checks.
	Attribute checks correspond to \verb|isXYZ| checks, all of which need to be true.
	String checks correspond to \verb|any(strcmp(argin,{...}))| checks, where \verb|{...}| is a list of suitable strings.
\end{enumerate}

\smallskip

In general, aim to call constructors as little as possible.
Notably, output arguments in classes can overwrite any input arguments.