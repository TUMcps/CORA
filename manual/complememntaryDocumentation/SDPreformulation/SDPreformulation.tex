\documentclass{article}

\usepackage{amsmath,amssymb}
\usepackage{natbib}
\usepackage{bibentry}
\usepackage{accents}
\usepackage{mathtools}
\usepackage{etoolbox}
\usepackage{hyperref}
\usepackage{cleveref}
\usepackage[a4paper, total={6in, 8in}]{geometry}
\nobibliography*

\newcommand{\myparagraph}[1]{\paragraph{#1}\mbox{}\\ \mbox{} \\}

\newtheorem{definition}{Definition}%alles neu
\newtheorem{proposition}{Proposition}
\newtheorem{theorem}{Theorem}
\newtheorem{lemma}{Lemma}
\newtheorem{remark}{Remark}
\newtheorem{assumption}{Assumption}
\newtheorem{example}{Example}
\newtheorem{corollary}{Corollary}
\newtheorem{problem}{Problem}

%%%%%%%%%commands
\newcommand{\f}[2]{#1\left(#2\right)}
\newcommand{\setdef}[2][]{
	\left\{
	\ifblank{#1}{}{#1 \hspace{.1cm} \middle| \hspace{.1cm}}
	#2
	\right\}
}
\newcommand{\var}[2]{{#1}_{\mathrm{#2}}}
\newcommand{\abs}[1]{\left|#1\right|}
\newcommand{\set}[1]{\mathcal{#1}}
\newcommand{\norm}[1]{\left\lVert#1\right\rVert}
\newcommand{\argmin}[1]{\underset{#1}{\hspace{.1cm}\mathrm{arg}\hspace{.05cm}\mathrm{min}} \hspace{.1cm}}
\newcommand{\argmax}[1]{\underset{#1}{\mathrm{arg}\hspace{.05cm}\mathrm{max}} \hspace{.1cm}}
\newcommand{\rank}[1]{\f{\text{rank}}{#1}}
\newcommand{\diag}[1]{\f{\text{diag}}{#1}}
\newcommand{\inv}[1]{#1^{-1}}
\newcommand{\rv}[1]{\mathrm{#1}}
\newcommand{\trace}[1]{\left\langle #1\right\rangle}
\newcommand{\vol}[1]{\f{\text{vol}}{#1}}

\newcommand{\ubar}[1]{\underaccent{\bar}{#1}}

\crefname{figure}{Fig.}{Figs.}
\crefname{equation}{}{}
\crefname{section}{Sec.}{Secs.}
\crefname{definition}{Def.}{Defs.}
\crefname{example}{Ex.}{Exs.}

\begin{document}
	\section{Primal to Dual}
	MOSEK solves problems of the form
	\begin{subequations}\label{eqs:SDP_primal}
		\begin{align}
			p^* = \min_{x,X}\quad	&c^Tx + \sum_{i=1}^{N} \trace{C^{(j)},X^{(i)}},\\
			\text{s.t.}\quad&\ubar{b}_j \leq a^{(j),T}x+\sum_{i=1}^{N}\trace{A^{(ij)},X^{(i)}} \leq \bar{b}_j, \ j\in\setdef{1,...,M},\label{eq:ineq_primal}\\
			&\ubar{d} \leq x \leq \bar{d},\\
			&x \in \set{K},\\
			&X_i \succeq 0, \ i\in\setdef{1,...,N},
		\end{align}
	\end{subequations}
	where $x\in\mathbb{R}^n$ contains $n$ scalar optimization variables, $X^{(i)}\in\mathbb{R}^{n\times n}$, $X^{(i)} = X^{(i,T)}\succeq 0$, are $N$ positive semi-definite optimization matrices (each matrix has $\frac{n (n-1)}{2}$ free optimization variables), $c\in\mathbb{R}^n$, $C^{(i)}\in\mathbb{R}^{n\times n}$, $\ubar{b}\in\mathbb{R}^{M}$, $\bar{b}\in\mathbb{R}^{M}$, $A^T = \begin{bmatrix}a^{(1)},&...,&a^{(M)}\end{bmatrix}$, $a^{(i)}\in\mathbb{R}^{n}$, $A^{(ij)}\in\mathbb{R}^{n\times n}$, $\ubar{d}\in\mathbb{R}^{n}$, and $\bar{d}\in\mathbb{R}^n$, and $\set{K} = \var{\set{K}}{quad}\times ...$, where $\var{\set{K}}{\cdot}$ are convex cones supported by MOSEK.
	
	We now derive the corresponding dual problem, since most (if not all) problems in CORA can be more efficiently modeled using the dual formulation of \cref{eqs:SDP_primal}. The lagrange dual function is given by
	\begin{equation}\label{eq:dual_not_simplifed}
		\begin{aligned}
			&\f{\Theta}{\ubar{\mu},\bar{\mu},\ubar{\eta},\bar{\eta},s,S^{(i)}} = \\
			&\inf_{x,X} \Big\{ c^Tx + \sum_{i=1}^{N} \trace{C^{(i)},X^{(i)}} + \Big.\\
			&\Big.\sum_{j=1}^{M} \ubar{\mu}_j\left(\ubar{b}_j - a^{(j),T}x-\sum_{i=1}^{N}\trace{A^{(ij)},X^{(i)}}\right)+\sum_{j=1}^{M} \bar{\mu}_j\left(-\bar{b}_j + a^{(j),T}x+\sum_{i=1}^{N}\trace{A^{(ij)},X^{(i)}}\right)+\Big.\\
			&\Big. \ubar{\eta}^T\left(\ubar{d}-x\right) + \bar{\eta}^T\left(x-\bar{d}\right)+\Big.\\ &\Big. -s^Tx+\Big.\\
			&\Big. -\sum_{i=1}^{N}\trace{S^{(i)},X^{(i)}} \Big\},
		\end{aligned}
	\end{equation}
	where $\ubar{\mu}\in\mathbb{R}^M$, $\bar{\mu}\in\mathbb{R}^M$, $\ubar{\eta}\in\mathbb{R}^n$, $\bar{\eta}\in\mathbb{R}^n$, $s\in\mathbb{R}^n$, and $S^{(i)}\in\mathbb{R}^{n\times n}$, $i\in\setdef{1,...,N}$.
	The first 3 equations follow from standard duality theory. Worth mentioning here are probably $s$ and $S^{(i)}$: Since we are transforming into the dual space for a minimum, we want $\f{\Theta}{\ubar{\mu},\bar{\mu},\ubar{\eta},\bar{\eta},s,S^{(i)}}\leq p^*$ (weak duality). The dual cone is given by $\f{\text{dual}}{\set{K}} = \setdef[z\in\mathbb{R}^n]{z^Tx\geq 0, \ x\in\set{K}}$. Thus, we subtract $s^Tx$, where $s\in\f{\text{dual}}{\set{K}}$. The dual cone of the positive semi-definite cone $\var{\set{K}}{psd}$ is given by $\f{\text{dual}}{\var{\set{K}}{psd}}=\setdef[Z\in\mathbb{S}_+^{n\times n}]{\trace{Z,X}\geq 0,\ X\in\var{\set{K}}{psd}}$, where $Z\in\mathbb{S}_+^{n\times n}$ since the cone of SDP matrices is self-dual. Thus we subtract for the same reason mentioned above.
	
	As a next step, we simplify \cref{eq:dual_not_simplifed} to
	\begin{equation}
		\begin{aligned}
			&\f{\Theta}{\ubar{\mu},\bar{\mu},\ubar{\eta},\bar{\eta},s,S^{(i)}} =\\
			&\inf_{x,X} \Big\{\left(c - A^T\ubar{\mu} + A^T\bar{\mu} -\ubar{\eta}+\bar{\eta} - s\right)x+ \Big.\\
			&\Big. \sum_{i=1}^{N}\trace{C^{(i)}-\sum_{j=1}^{M}A^{(ij)}\left(\ubar{\mu}_i-\bar{\mu}_i\right) -S^{(i)},X^{(i)}}\Big\}\\
			&+\ubar{b}^T\ubar{\mu} - \bar{b}^T\bar{\mu} +\ubar{\eta}^T\ubar{d}-\bar{\eta}\bar{d}.
		\end{aligned}
	\end{equation}
	Thus
	\begin{equation}
		\f{\Theta}{\ubar{\mu},\bar{\mu},\ubar{\eta},\bar{\eta},s,S^{(i)}} =
		\begin{cases}
			\ubar{b}^T\ubar{\mu} - \bar{b}^T\bar{\mu} +\ubar{\eta}^T\ubar{d}-\bar{\eta}\bar{d}, &\begin{bmatrix}
				c - A^T\ubar{\mu} + A^T\bar{\mu} -\ubar{\eta}+\bar{\eta} - s=0\\
				C^{(i)}-\sum_{j=1}^{M}A^{(ij)}\left(\ubar{\mu}_i-\bar{\mu}_i\right) \succeq 0, \ i\in\setdef{1,...,N},
			\end{bmatrix},\\
			-\infty,& \text{otherwise}
		\end{cases}.
	\end{equation}
	Hence, the dual problem is given by $\max_{\ubar{\mu}\geq 0,\bar{\mu}\geq 0,\ubar{\eta}\geq 0,\bar{\eta}\geq 0,s} \f{\Theta}{\ubar{\mu},\bar{\mu},\ubar{\eta},\bar{\eta},s}$, i.e.
	\begin{subequations}\label{eqs:SDP_dual}
		\begin{align}
			\max_{\ubar{\mu},\bar{\mu},\ubar{\eta},\bar{\eta},s} &\ubar{b}^T\ubar{\mu} - \bar{b}^T\bar{\mu} +\ubar{d}^T\ubar{\eta}-\bar{d}^T\bar{\eta},\\
			\text{s.t.} \quad 	&c - A^T\left(\ubar{\mu}-\bar{\mu}\right) -\ubar{\eta}+\bar{\eta} - s=0,\\
								&C^{(i)}-\sum_{j=1}^{M}A^{(ij)}\left(\ubar{\mu}_i-\bar{\mu}_i\right) \succeq 0, \ i\in\setdef{1,...,N},\\
								&s\in\f{\text{dual}}{\set{K}},\\
								&\begin{bmatrix}\ubar{\mu}^T,&\bar{\mu}^T,&\ubar{\eta}^T,&\bar{\eta}^T\end{bmatrix}\geq 0.
		\end{align}
	\end{subequations}

	\section{Special Dual Cases}
	For certain functions in CORA, special cases of \cref{eqs:SDP_dual} are used, which we will quickly discuss here.
		
		\subsection{Equality Constraint}
		If \cref{eq:ineq_primal} is actually an equality constraint, we have $\ubar{b} = \bar{b} = b$, and \cref{eqs:SDP_dual} simplifies to 
		\begin{subequations}\label{eqs:SDP_dual_equality}
			\begin{align}
				\max_{\mu,\ubar{\eta},\bar{\eta},s} &\bar{b}^T\mu +\ubar{d}^T\ubar{\eta}-\bar{d}^T\bar{\eta},\\
				\text{s.t.} \quad 	&c - A^T\mu -\ubar{\eta}+\bar{\eta} - s=0, \label{eq:SDP_dual_eq_equality}\\
				&C^{(i)}-\sum_{j=1}^{M}A^{(ij)}\mu_i \succeq 0, \ i\in\setdef{1,...,N},\\
				&s\in\f{\text{dual}}{\set{K}},\\
				&\begin{bmatrix}\ubar{\eta}^T,&\bar{\eta}^T\end{bmatrix}\geq 0.
			\end{align}
		\end{subequations}
	
		\subsection{No State Bounds}
		If \cref{eqs:SDP_primal} does not have any state bounds, $\bar{d} = -\ubar{d} = \infty$, and \cref{eqs:SDP_dual} simplifies to
		\begin{subequations}
			\begin{align}
				\max_{\ubar{\mu},\bar{\mu},s},\ &\ubar{b}^T\ubar{\mu} - \bar{b}^T\bar{\mu},\\
				\text{s.t.} \quad 	&c - A^T\left(\ubar{\mu}-\bar{\mu}\right) - s=0,\\
				&C^{(i)}-\sum_{j=1}^{M}A^{(ij)}\left(\ubar{\mu}_i-\bar{\mu}_i\right) \succeq 0, \ i\in\setdef{1,...,N},\\
				&s\in\f{\text{dual}}{\set{K}},\\
				&\begin{bmatrix}\ubar{\mu}^T,&\bar{\mu}^T\end{bmatrix}\geq 0.
			\end{align}
		\end{subequations}
		
		\subsection{Greater Zero Constraint}
		Assume that we identified $A$, $A^{(ij)}$, $c$, $C^{(ij)}$ from \cref{eqs:SDP_dual_equality} ($\ubar{b}=\bar{b}=b$), but additionally require the constraint $\mu_k = \ubar{\mu}_k-\bar{\mu}_k \geq 0$, where $k\leq K\leq M$. Substituting $c$, $A^T$, $\ubar{d}$, $\bar{d}$, $\ubar{\eta}$, $\bar{\eta}$, $s$ in \cref{eq:SDP_dual_eq_equality} by $\begin{bmatrix}c\\0_K\end{bmatrix}$, $\begin{bmatrix}A^T\\-I_{K\times K},&0\end{bmatrix}$, $\begin{bmatrix}\ubar{d}\\0_K\end{bmatrix}$, $ \begin{bmatrix}\bar{d}\\\infty_K\end{bmatrix}$, $\begin{bmatrix}\ubar{\eta}\\\ubar{\tilde{\eta}}\end{bmatrix}$, $\begin{bmatrix}\bar{\eta}\\\bar{\tilde{\eta}}\end{bmatrix}$ transforms \cref{eqs:SDP_dual_equality} into
		\begin{subequations}
			\begin{align}
				\max_{\mu,\ubar{\eta},\bar{\eta},s} &\bar{b}^T\mu +\ubar{d}^T\ubar{\eta}-\bar{d}^T\bar{\eta},\\
				\text{s.t.} \quad 	&c - A^T\mu -\ubar{\eta}+\bar{\eta} - s=0,\\
									&-\mu_k+\tilde{\bar{\eta}}_k -\tilde{s}_k = 0, \ k\in\setdef{1,...,K},\label{eq:geq_0_dual}\\
				&C^{(i)}-\sum_{j=1}^{M}A^{(ij)}\mu_i \succeq 0, \ i\in\setdef{1,...,N},\\
				&s\in\f{\text{dual}}{\set{K}},\\
				&\begin{bmatrix}\ubar{\eta}^T,&\bar{\eta}^T\end{bmatrix}\geq 0,
			\end{align}
		\end{subequations} 
		because $\bar{\tilde{\eta}}=0$ (since it is multiplied by $\infty$ in the objective function). Further, if we do not constrain the primal scalar variables (that we introduced by this extension of $c$, $A^T$ etc.), i.e., constrain them to be in $\var{\set{K}}{added}=\setdef[\tilde{x}]{\tilde{x}\in\mathbb{R}^K}$, the corresponding dual cone $\f{\text{dual}}{\var{\set{K}}{add}} = \setdef[\tilde{s}]{\tilde{s}^T\tilde{x}\geq 0, \ \tilde{x}\in\var{\set{K}}{add}} = \setdef{0}$ constrains $\tilde{s}$ to $0$. Thus \cref{eq:geq_0_dual} becomes $\mu_k=\tilde{\bar{\eta}}_k \geq 0$.
		
		Helpful links\footnote{\url{https://math.stackexchange.com/questions/4104095/largest-eigenvalue-as-semidefinite-program-in-standard-form}}
		
	\section{Geomean in SDP Dual Standard Form}
	Often, SDP problems for ellipsoids include a volume-optimizing objective function such as $\log\det \inv{Q}$, where $Q\in\mathbb{S}_+^{n\times n}$ is, e.g., a positive-definite matrix defining an ellipsoid. This can be transformed into standard form as explained in \footnote{Sec. 4.2.2, \url{https://docs.mosek.com/MOSEKModelingCookbook-v2.pdf}}.
\end{document}