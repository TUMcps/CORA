\documentclass[crop=true]{standalone}

% tikz
\usepackage{tikzscale}
\usepackage{pgfplots}
\usepackage{mathtools}
\pgfplotsset{compat=1.18}
% \usepgfplotslibrary{external}
% \tikzexternalize[prefix=./]
% \newcommand{\includetikz}[1]{%
%     \tikzsetnextfilename{#1}%
%     \input{#1.tikz}%
% }
% explicitly set line width to avoid different widths in different pdf viewers
\tikzset{every picture/.style={line width=0.5pt}}
% groupplots
\usepgfplotslibrary{groupplots}
\usetikzlibrary{pgfplots.groupplots}
\usetikzlibrary{matrix}

\begin{document}
%
% HOWTO: Convert a Matlab figure to svg via tikz
%
% 1. Create and save Matlab figure      								savefig('<figure-path>/<figure-name>.fig')
% 2. Convert fig to tikz:										convertToTikz('<figure-path>/<figure-name>.fig')
% 3. Update <figure-path>/<figure-name> in the \input command below
% 4. Create dvi of figure:										pdflatex -output-format=dvi main
% 5. Convert dvi to svg:										dvisvgm main.dvi --font-format=woff --optimize -o <figure-path>/<figure-name>.svg
% 6. Add to website         										<img src="<figure-path>/<figure-name>.svg"/>
%
% \input{./<figure-path>/<figure-name>.tikz}
% This file was created by matlab2tikz.
%
\definecolor{mycolor1}{rgb}{0.27060,0.58820,1.00000}%
%
\begin{tikzpicture}
\footnotesize

\begin{axis}[%
width=8cm,
height=4.5cm,
at={(0in,0in)},
scale only axis,
xmin=0,
xmax=1.7,
ymin=0,
ymax=1.2,
axis background/.style={fill=white},
title style={font=\bfseries},
title={Bouncing Ball},
legend style={legend cell align=left, align=left, draw=white!15!black}
]

\addplot[area legend, draw=mycolor1, fill=mycolor1]
table[row sep=crcr] {%
x	y\\
0	0.947\\
0.02	0.947\\
0.02	1.0515\\
0	1.0515\\
0	0.947\\
}--cycle;
\addlegendentry{Reachable set $\mathcal{R}$}


\addplot[area legend, draw=mycolor1, fill=mycolor1, forget plot]
table[row sep=crcr] {%
x	y\\
0.02	0.9402\\
0.04	0.9402\\
0.04	1.0505\\
0.02	1.0505\\
0.02	0.9402\\
}--cycle;

\addplot[area legend, draw=mycolor1, fill=mycolor1, forget plot]
table[row sep=crcr] {%
x	y\\
0.04	0.9293\\
0.06	0.9293\\
0.06	1.0456\\
0.04	1.0456\\
0.04	0.9293\\
}--cycle;

\addplot[area legend, draw=mycolor1, fill=mycolor1, forget plot]
table[row sep=crcr] {%
x	y\\
0.06	0.9146\\
0.08	0.9146\\
0.08	1.0368\\
0.06	1.0368\\
0.06	0.9146\\
}--cycle;

\addplot[area legend, draw=mycolor1, fill=mycolor1, forget plot]
table[row sep=crcr] {%
x	y\\
0.08	0.896\\
0.1	0.896\\
0.1	1.0241\\
0.08	1.0241\\
0.08	0.896\\
}--cycle;

\addplot[area legend, draw=mycolor1, fill=mycolor1, forget plot]
table[row sep=crcr] {%
x	y\\
0.1	0.8734\\
0.12	0.8734\\
0.12	1.0074\\
0.1	1.0074\\
0.1	0.8734\\
}--cycle;

\addplot[area legend, draw=mycolor1, fill=mycolor1, forget plot]
table[row sep=crcr] {%
x	y\\
0.12	0.8469\\
0.14	0.8469\\
0.14	0.9869\\
0.12	0.9869\\
0.12	0.8469\\
}--cycle;

\addplot[area legend, draw=mycolor1, fill=mycolor1, forget plot]
table[row sep=crcr] {%
x	y\\
0.14	0.8164\\
0.16	0.8164\\
0.16	0.9624\\
0.14	0.9624\\
0.14	0.8164\\
}--cycle;

\addplot[area legend, draw=mycolor1, fill=mycolor1, forget plot]
table[row sep=crcr] {%
x	y\\
0.16	0.7821\\
0.18	0.7821\\
0.18	0.9339\\
0.16	0.9339\\
0.16	0.7821\\
}--cycle;

\addplot[area legend, draw=mycolor1, fill=mycolor1, forget plot]
table[row sep=crcr] {%
x	y\\
0.18	0.7438\\
0.2	0.7438\\
0.2	0.9016\\
0.18	0.9016\\
0.18	0.7438\\
}--cycle;

\addplot[area legend, draw=mycolor1, fill=mycolor1, forget plot]
table[row sep=crcr] {%
x	y\\
0.2	0.7016\\
0.22	0.7016\\
0.22	0.8653\\
0.2	0.8653\\
0.2	0.7016\\
}--cycle;

\addplot[area legend, draw=mycolor1, fill=mycolor1, forget plot]
table[row sep=crcr] {%
x	y\\
0.22	0.6555\\
0.24	0.6555\\
0.24	0.8251\\
0.22	0.8251\\
0.22	0.6555\\
}--cycle;

\addplot[area legend, draw=mycolor1, fill=mycolor1, forget plot]
table[row sep=crcr] {%
x	y\\
0.24	0.6054\\
0.26	0.6054\\
0.26	0.781\\
0.24	0.781\\
0.24	0.6054\\
}--cycle;

\addplot[area legend, draw=mycolor1, fill=mycolor1, forget plot]
table[row sep=crcr] {%
x	y\\
0.26	0.5514\\
0.28	0.5514\\
0.28	0.7329\\
0.26	0.7329\\
0.26	0.5514\\
}--cycle;

\addplot[area legend, draw=mycolor1, fill=mycolor1, forget plot]
table[row sep=crcr] {%
x	y\\
0.28	0.4935\\
0.3	0.4935\\
0.3	0.6809\\
0.28	0.6809\\
0.28	0.4935\\
}--cycle;

\addplot[area legend, draw=mycolor1, fill=mycolor1, forget plot]
table[row sep=crcr] {%
x	y\\
0.3	0.4317\\
0.32	0.4317\\
0.32	0.625\\
0.3	0.625\\
0.3	0.4317\\
}--cycle;

\addplot[area legend, draw=mycolor1, fill=mycolor1, forget plot]
table[row sep=crcr] {%
x	y\\
0.32	0.366\\
0.34	0.366\\
0.34	0.5652\\
0.32	0.5652\\
0.32	0.366\\
}--cycle;

\addplot[area legend, draw=mycolor1, fill=mycolor1, forget plot]
table[row sep=crcr] {%
x	y\\
0.34	0.2963\\
0.36	0.2963\\
0.36	0.5015\\
0.34	0.5015\\
0.34	0.2963\\
}--cycle;

\addplot[area legend, draw=mycolor1, fill=mycolor1, forget plot]
table[row sep=crcr] {%
x	y\\
0.36	0.2227\\
0.38	0.2227\\
0.38	0.4338\\
0.36	0.4338\\
0.36	0.2227\\
}--cycle;

\addplot[area legend, draw=mycolor1, fill=mycolor1, forget plot]
table[row sep=crcr] {%
x	y\\
0.38	0.1452\\
0.4	0.1452\\
0.4	0.3622\\
0.38	0.3622\\
0.38	0.1452\\
}--cycle;

\addplot[area legend, draw=mycolor1, fill=mycolor1, forget plot]
table[row sep=crcr] {%
x	y\\
0.4	0.0638\\
0.42	0.0638\\
0.42	0.2867\\
0.4	0.2867\\
0.4	0.0638\\
}--cycle;

\addplot[area legend, draw=mycolor1, fill=mycolor1, forget plot]
table[row sep=crcr] {%
x	y\\
0.42	-0.0216\\
0.44	-0.0216\\
0.44	0.2072\\
0.42	0.2072\\
0.42	-0.0216\\
}--cycle;

\addplot[area legend, draw=mycolor1, fill=mycolor1, forget plot]
table[row sep=crcr] {%
x	y\\
0.44	-0.1109\\
0.46	-0.1109\\
0.46	0.1239\\
0.44	0.1239\\
0.44	-0.1109\\
}--cycle;

\addplot[area legend, draw=mycolor1, fill=mycolor1, forget plot]
table[row sep=crcr] {%
x	y\\
0.46	-0.2041\\
0.48	-0.2041\\
0.48	0.0366\\
0.46	0.0366\\
0.46	-0.2041\\
}--cycle;

\addplot[area legend, draw=mycolor1, fill=mycolor1, forget plot]
table[row sep=crcr] {%
x	y\\
0.48	-0.3013\\
0.5	-0.3013\\
0.5	-0.0546\\
0.48	-0.0546\\
0.48	-0.3013\\
}--cycle;

\addplot[area legend, draw=mycolor1, fill=mycolor1, forget plot]
table[row sep=crcr] {%
x	y\\
0.42	-0.0023\\
0.5	-0.0023\\
0.5	0.0672\\
0.42	0.0672\\
0.42	-0.0023\\
}--cycle;

\addplot[area legend, draw=mycolor1, fill=mycolor1, forget plot]
table[row sep=crcr] {%
x	y\\
0.44	0.0599\\
0.52	0.0599\\
0.52	0.1299\\
0.44	0.1299\\
0.44	0.0599\\
}--cycle;

\addplot[area legend, draw=mycolor1, fill=mycolor1, forget plot]
table[row sep=crcr] {%
x	y\\
0.46	0.1181\\
0.54	0.1181\\
0.54	0.1888\\
0.46	0.1888\\
0.46	0.1181\\
}--cycle;

\addplot[area legend, draw=mycolor1, fill=mycolor1, forget plot]
table[row sep=crcr] {%
x	y\\
0.48	0.1724\\
0.56	0.1724\\
0.56	0.2437\\
0.48	0.2437\\
0.48	0.1724\\
}--cycle;

\addplot[area legend, draw=mycolor1, fill=mycolor1, forget plot]
table[row sep=crcr] {%
x	y\\
0.5	0.2228\\
0.58	0.2228\\
0.58	0.2946\\
0.5	0.2946\\
0.5	0.2228\\
}--cycle;

\addplot[area legend, draw=mycolor1, fill=mycolor1, forget plot]
table[row sep=crcr] {%
x	y\\
0.52	0.2693\\
0.6	0.2693\\
0.6	0.3417\\
0.52	0.3417\\
0.52	0.2693\\
}--cycle;

\addplot[area legend, draw=mycolor1, fill=mycolor1, forget plot]
table[row sep=crcr] {%
x	y\\
0.54	0.3118\\
0.62	0.3118\\
0.62	0.3848\\
0.54	0.3848\\
0.54	0.3118\\
}--cycle;

\addplot[area legend, draw=mycolor1, fill=mycolor1, forget plot]
table[row sep=crcr] {%
x	y\\
0.56	0.3504\\
0.64	0.3504\\
0.64	0.4241\\
0.56	0.4241\\
0.56	0.3504\\
}--cycle;

\addplot[area legend, draw=mycolor1, fill=mycolor1, forget plot]
table[row sep=crcr] {%
x	y\\
0.58	0.3851\\
0.66	0.3851\\
0.66	0.4593\\
0.58	0.4593\\
0.58	0.3851\\
}--cycle;

\addplot[area legend, draw=mycolor1, fill=mycolor1, forget plot]
table[row sep=crcr] {%
x	y\\
0.6	0.4159\\
0.68	0.4159\\
0.68	0.4907\\
0.6	0.4907\\
0.6	0.4159\\
}--cycle;

\addplot[area legend, draw=mycolor1, fill=mycolor1, forget plot]
table[row sep=crcr] {%
x	y\\
0.62	0.4427\\
0.7	0.4427\\
0.7	0.5181\\
0.62	0.5181\\
0.62	0.4427\\
}--cycle;

\addplot[area legend, draw=mycolor1, fill=mycolor1, forget plot]
table[row sep=crcr] {%
x	y\\
0.64	0.4657\\
0.72	0.4657\\
0.72	0.5417\\
0.64	0.5417\\
0.64	0.4657\\
}--cycle;

\addplot[area legend, draw=mycolor1, fill=mycolor1, forget plot]
table[row sep=crcr] {%
x	y\\
0.66	0.4846\\
0.74	0.4846\\
0.74	0.5612\\
0.66	0.5612\\
0.66	0.4846\\
}--cycle;

\addplot[area legend, draw=mycolor1, fill=mycolor1, forget plot]
table[row sep=crcr] {%
x	y\\
0.68	0.4997\\
0.76	0.4997\\
0.76	0.5769\\
0.68	0.5769\\
0.68	0.4997\\
}--cycle;

\addplot[area legend, draw=mycolor1, fill=mycolor1, forget plot]
table[row sep=crcr] {%
x	y\\
0.7	0.5109\\
0.78	0.5109\\
0.78	0.5887\\
0.7	0.5887\\
0.7	0.5109\\
}--cycle;

\addplot[area legend, draw=mycolor1, fill=mycolor1, forget plot]
table[row sep=crcr] {%
x	y\\
0.72	0.5181\\
0.8	0.5181\\
0.8	0.5965\\
0.72	0.5965\\
0.72	0.5181\\
}--cycle;

\addplot[area legend, draw=mycolor1, fill=mycolor1, forget plot]
table[row sep=crcr] {%
x	y\\
0.74	0.5214\\
0.82	0.5214\\
0.82	0.6004\\
0.74	0.6004\\
0.74	0.5214\\
}--cycle;

\addplot[area legend, draw=mycolor1, fill=mycolor1, forget plot]
table[row sep=crcr] {%
x	y\\
0.76	0.5185\\
0.84	0.5185\\
0.84	0.6026\\
0.76	0.6026\\
0.76	0.5185\\
}--cycle;

\addplot[area legend, draw=mycolor1, fill=mycolor1, forget plot]
table[row sep=crcr] {%
x	y\\
0.78	0.51\\
0.86	0.51\\
0.86	0.6026\\
0.78	0.6026\\
0.78	0.51\\
}--cycle;

\addplot[area legend, draw=mycolor1, fill=mycolor1, forget plot]
table[row sep=crcr] {%
x	y\\
0.8	0.4976\\
0.88	0.4976\\
0.88	0.5987\\
0.8	0.5987\\
0.8	0.4976\\
}--cycle;

\addplot[area legend, draw=mycolor1, fill=mycolor1, forget plot]
table[row sep=crcr] {%
x	y\\
0.82	0.4813\\
0.9	0.4813\\
0.9	0.5908\\
0.82	0.5908\\
0.82	0.4813\\
}--cycle;

\addplot[area legend, draw=mycolor1, fill=mycolor1, forget plot]
table[row sep=crcr] {%
x	y\\
0.84	0.461\\
0.92	0.461\\
0.92	0.579\\
0.84	0.579\\
0.84	0.461\\
}--cycle;

\addplot[area legend, draw=mycolor1, fill=mycolor1, forget plot]
table[row sep=crcr] {%
x	y\\
0.86	0.4369\\
0.94	0.4369\\
0.94	0.5633\\
0.86	0.5633\\
0.86	0.4369\\
}--cycle;

\addplot[area legend, draw=mycolor1, fill=mycolor1, forget plot]
table[row sep=crcr] {%
x	y\\
0.88	0.4088\\
0.96	0.4088\\
0.96	0.5436\\
0.88	0.5436\\
0.88	0.4088\\
}--cycle;

\addplot[area legend, draw=mycolor1, fill=mycolor1, forget plot]
table[row sep=crcr] {%
x	y\\
0.9	0.3767\\
0.98	0.3767\\
0.98	0.52\\
0.9	0.52\\
0.9	0.3767\\
}--cycle;

\addplot[area legend, draw=mycolor1, fill=mycolor1, forget plot]
table[row sep=crcr] {%
x	y\\
0.92	0.3408\\
1	0.3408\\
1	0.4925\\
0.92	0.4925\\
0.92	0.3408\\
}--cycle;

\addplot[area legend, draw=mycolor1, fill=mycolor1, forget plot]
table[row sep=crcr] {%
x	y\\
0.94	0.3009\\
1.02	0.3009\\
1.02	0.4611\\
0.94	0.4611\\
0.94	0.3009\\
}--cycle;

\addplot[area legend, draw=mycolor1, fill=mycolor1, forget plot]
table[row sep=crcr] {%
x	y\\
0.96	0.2571\\
1.04	0.2571\\
1.04	0.4258\\
0.96	0.4258\\
0.96	0.2571\\
}--cycle;

\addplot[area legend, draw=mycolor1, fill=mycolor1, forget plot]
table[row sep=crcr] {%
x	y\\
0.98	0.2094\\
1.06	0.2094\\
1.06	0.3865\\
0.98	0.3865\\
0.98	0.2094\\
}--cycle;

\addplot[area legend, draw=mycolor1, fill=mycolor1, forget plot]
table[row sep=crcr] {%
x	y\\
1	0.1578\\
1.08	0.1578\\
1.08	0.3433\\
1	0.3433\\
1	0.1578\\
}--cycle;

\addplot[area legend, draw=mycolor1, fill=mycolor1, forget plot]
table[row sep=crcr] {%
x	y\\
1.02	0.1022\\
1.1	0.1022\\
1.1	0.2962\\
1.02	0.2962\\
1.02	0.1022\\
}--cycle;

\addplot[area legend, draw=mycolor1, fill=mycolor1, forget plot]
table[row sep=crcr] {%
x	y\\
1.04	0.0427\\
1.12	0.0427\\
1.12	0.2451\\
1.04	0.2451\\
1.04	0.0427\\
}--cycle;

\addplot[area legend, draw=mycolor1, fill=mycolor1, forget plot]
table[row sep=crcr] {%
x	y\\
1.06	-0.0207\\
1.14	-0.0207\\
1.14	0.1902\\
1.06	0.1902\\
1.06	-0.0207\\
}--cycle;

\addplot[area legend, draw=mycolor1, fill=mycolor1, forget plot]
table[row sep=crcr] {%
x	y\\
1.08	-0.088\\
1.16	-0.088\\
1.16	0.1313\\
1.08	0.1313\\
1.08	-0.088\\
}--cycle;

\addplot[area legend, draw=mycolor1, fill=mycolor1, forget plot]
table[row sep=crcr] {%
x	y\\
1.1	-0.1593\\
1.18	-0.1593\\
1.18	0.0685\\
1.1	0.0685\\
1.1	-0.1593\\
}--cycle;

\addplot[area legend, draw=mycolor1, fill=mycolor1, forget plot]
table[row sep=crcr] {%
x	y\\
1.12	-0.2344\\
1.2	-0.2344\\
1.2	0.0017\\
1.12	0.0017\\
1.12	-0.2344\\
}--cycle;

\addplot[area legend, draw=mycolor1, fill=mycolor1, forget plot]
table[row sep=crcr] {%
x	y\\
1.14	-0.3135\\
1.22	-0.3135\\
1.22	-0.0689\\
1.14	-0.0689\\
1.14	-0.3135\\
}--cycle;

\addplot[area legend, draw=mycolor1, fill=mycolor1, forget plot]
table[row sep=crcr] {%
x	y\\
1.06	-0.0027\\
1.22	-0.0027\\
1.22	0.0511\\
1.06	0.0511\\
1.06	-0.0027\\
}--cycle;

\addplot[area legend, draw=mycolor1, fill=mycolor1, forget plot]
table[row sep=crcr] {%
x	y\\
1.08	0.0426\\
1.24	0.0426\\
1.24	0.0979\\
1.08	0.0979\\
1.08	0.0426\\
}--cycle;

\addplot[area legend, draw=mycolor1, fill=mycolor1, forget plot]
table[row sep=crcr] {%
x	y\\
1.1	0.084\\
1.26	0.084\\
1.26	0.1407\\
1.1	0.1407\\
1.1	0.084\\
}--cycle;

\addplot[area legend, draw=mycolor1, fill=mycolor1, forget plot]
table[row sep=crcr] {%
x	y\\
1.12	0.1215\\
1.28	0.1215\\
1.28	0.1796\\
1.12	0.1796\\
1.12	0.1215\\
}--cycle;

\addplot[area legend, draw=mycolor1, fill=mycolor1, forget plot]
table[row sep=crcr] {%
x	y\\
1.14	0.155\\
1.3	0.155\\
1.3	0.2145\\
1.14	0.2145\\
1.14	0.155\\
}--cycle;

\addplot[area legend, draw=mycolor1, fill=mycolor1, forget plot]
table[row sep=crcr] {%
x	y\\
1.16	0.1846\\
1.32	0.1846\\
1.32	0.2456\\
1.16	0.2456\\
1.16	0.1846\\
}--cycle;

\addplot[area legend, draw=mycolor1, fill=mycolor1, forget plot]
table[row sep=crcr] {%
x	y\\
1.18	0.2103\\
1.34	0.2103\\
1.34	0.2727\\
1.18	0.2727\\
1.18	0.2103\\
}--cycle;

\addplot[area legend, draw=mycolor1, fill=mycolor1, forget plot]
table[row sep=crcr] {%
x	y\\
1.2	0.232\\
1.36	0.232\\
1.36	0.2959\\
1.2	0.2959\\
1.2	0.232\\
}--cycle;

\addplot[area legend, draw=mycolor1, fill=mycolor1, forget plot]
table[row sep=crcr] {%
x	y\\
1.22	0.2499\\
1.38	0.2499\\
1.38	0.3151\\
1.22	0.3151\\
1.22	0.2499\\
}--cycle;

\addplot[area legend, draw=mycolor1, fill=mycolor1, forget plot]
table[row sep=crcr] {%
x	y\\
1.24	0.2638\\
1.4	0.2638\\
1.4	0.3305\\
1.24	0.3305\\
1.24	0.2638\\
}--cycle;

\addplot[area legend, draw=mycolor1, fill=mycolor1, forget plot]
table[row sep=crcr] {%
x	y\\
1.26	0.2737\\
1.42	0.2737\\
1.42	0.3419\\
1.26	0.3419\\
1.26	0.2737\\
}--cycle;

\addplot[area legend, draw=mycolor1, fill=mycolor1, forget plot]
table[row sep=crcr] {%
x	y\\
1.28	0.2798\\
1.44	0.2798\\
1.44	0.3494\\
1.28	0.3494\\
1.28	0.2798\\
}--cycle;

\addplot[area legend, draw=mycolor1, fill=mycolor1, forget plot]
table[row sep=crcr] {%
x	y\\
1.3	0.2819\\
1.46	0.2819\\
1.46	0.3529\\
1.3	0.3529\\
1.3	0.2819\\
}--cycle;

\addplot[area legend, draw=mycolor1, fill=mycolor1, forget plot]
table[row sep=crcr] {%
x	y\\
1.32	0.2771\\
1.48	0.2771\\
1.48	0.3556\\
1.32	0.3556\\
1.32	0.2771\\
}--cycle;

\addplot[area legend, draw=mycolor1, fill=mycolor1, forget plot]
table[row sep=crcr] {%
x	y\\
1.34	0.2675\\
1.5	0.2675\\
1.5	0.3552\\
1.34	0.3552\\
1.34	0.2675\\
}--cycle;

\addplot[area legend, draw=mycolor1, fill=mycolor1, forget plot]
table[row sep=crcr] {%
x	y\\
1.36	0.2539\\
1.52	0.2539\\
1.52	0.351\\
1.36	0.351\\
1.36	0.2539\\
}--cycle;

\addplot[area legend, draw=mycolor1, fill=mycolor1, forget plot]
table[row sep=crcr] {%
x	y\\
1.38	0.2364\\
1.54	0.2364\\
1.54	0.3428\\
1.38	0.3428\\
1.38	0.2364\\
}--cycle;

\addplot[area legend, draw=mycolor1, fill=mycolor1, forget plot]
table[row sep=crcr] {%
x	y\\
1.4	0.215\\
1.56	0.215\\
1.56	0.3306\\
1.4	0.3306\\
1.4	0.215\\
}--cycle;

\addplot[area legend, draw=mycolor1, fill=mycolor1, forget plot]
table[row sep=crcr] {%
x	y\\
1.42	0.1897\\
1.58	0.1897\\
1.58	0.3146\\
1.42	0.3146\\
1.42	0.1897\\
}--cycle;

\addplot[area legend, draw=mycolor1, fill=mycolor1, forget plot]
table[row sep=crcr] {%
x	y\\
1.44	0.1605\\
1.6	0.1605\\
1.6	0.2946\\
1.44	0.2946\\
1.44	0.1605\\
}--cycle;

\addplot[area legend, draw=mycolor1, fill=mycolor1, forget plot]
table[row sep=crcr] {%
x	y\\
1.46	0.1273\\
1.62	0.1273\\
1.62	0.2707\\
1.46	0.2707\\
1.46	0.1273\\
}--cycle;

\addplot[area legend, draw=mycolor1, fill=mycolor1, forget plot]
table[row sep=crcr] {%
x	y\\
1.48	0.0902\\
1.64	0.0902\\
1.64	0.2429\\
1.48	0.2429\\
1.48	0.0902\\
}--cycle;

\addplot[area legend, draw=mycolor1, fill=mycolor1, forget plot]
table[row sep=crcr] {%
x	y\\
1.5	0.0491\\
1.66	0.0491\\
1.66	0.2111\\
1.5	0.2111\\
1.5	0.0491\\
}--cycle;

\addplot[area legend, draw=mycolor1, fill=mycolor1, forget plot]
table[row sep=crcr] {%
x	y\\
1.52	0.0042\\
1.68	0.0042\\
1.68	0.1754\\
1.52	0.1754\\
1.52	0.0042\\
}--cycle;

\addplot[area legend, draw=mycolor1, fill=mycolor1, forget plot]
table[row sep=crcr] {%
x	y\\
1.54	-0.0447\\
1.7	-0.0447\\
1.7	0.1358\\
1.54	0.1358\\
1.54	-0.0447\\
}--cycle;

\addplot[area legend, draw=mycolor1, fill=mycolor1, forget plot]
table[row sep=crcr] {%
x	y\\
1.56	-0.0975\\
1.72	-0.0975\\
1.72	0.0923\\
1.56	0.0923\\
1.56	-0.0975\\
}--cycle;

\addplot[area legend, draw=mycolor1, fill=mycolor1, forget plot]
table[row sep=crcr] {%
x	y\\
1.58	-0.1542\\
1.74	-0.1542\\
1.74	0.0449\\
1.58	0.0449\\
1.58	-0.1542\\
}--cycle;

\addplot[area legend, draw=mycolor1, fill=mycolor1, forget plot]
table[row sep=crcr] {%
x	y\\
1.6	-0.2149\\
1.76	-0.2149\\
1.76	-0.0065\\
1.6	-0.0065\\
1.6	-0.2149\\
}--cycle;

\addplot[area legend, draw=mycolor1, fill=mycolor1, forget plot]
table[row sep=crcr] {%
x	y\\
1.54	-0.0022\\
1.76	-0.0022\\
1.76	0.0382\\
1.54	0.0382\\
1.54	-0.0022\\
}--cycle;

\addplot[area legend, draw=mycolor1, fill=mycolor1, forget plot]
table[row sep=crcr] {%
x	y\\
1.56	0.0311\\
1.78	0.0311\\
1.78	0.072\\
1.56	0.072\\
1.56	0.0311\\
}--cycle;

\addplot[area legend, draw=mycolor1, fill=mycolor1, forget plot]
table[row sep=crcr] {%
x	y\\
1.58	0.0604\\
1.8	0.0604\\
1.8	0.1019\\
1.58	0.1019\\
1.58	0.0604\\
}--cycle;

\addplot[area legend, draw=mycolor1, fill=mycolor1, forget plot]
table[row sep=crcr] {%
x	y\\
1.6	0.0858\\
1.82	0.0858\\
1.82	0.1279\\
1.6	0.1279\\
1.6	0.0858\\
}--cycle;

\addplot[area legend, draw=mycolor1, fill=mycolor1, forget plot]
table[row sep=crcr] {%
x	y\\
1.62	0.1073\\
1.84	0.1073\\
1.84	0.1499\\
1.62	0.1499\\
1.62	0.1073\\
}--cycle;

\addplot[area legend, draw=mycolor1, fill=mycolor1, forget plot]
table[row sep=crcr] {%
x	y\\
1.64	0.1249\\
1.86	0.1249\\
1.86	0.168\\
1.64	0.168\\
1.64	0.1249\\
}--cycle;

\addplot[area legend, draw=mycolor1, fill=mycolor1, forget plot]
table[row sep=crcr] {%
x	y\\
1.66	0.1386\\
1.88	0.1386\\
1.88	0.1822\\
1.66	0.1822\\
1.66	0.1386\\
}--cycle;

\addplot[area legend, draw=mycolor1, fill=mycolor1, forget plot]
table[row sep=crcr] {%
x	y\\
1.68	0.1483\\
1.9	0.1483\\
1.9	0.1924\\
1.68	0.1924\\
1.68	0.1483\\
}--cycle;

\addplot[area legend, draw=black, fill=white, forget plot]
table[row sep=crcr] {%
x	y\\
0	0.95\\
0	0.95\\
}--cycle;
\addplot [color=black]
  table[row sep=crcr]{%
0	0.95\\
0.0008	0.95\\
0.001	0.9499\\
0.0023	0.9499\\
0.0036	0.9498\\
0.0049	0.9496\\
0.0061	0.9495\\
0.0109	0.9489\\
0.0156	0.948\\
0.0203	0.947\\
0.025	0.9457\\
0.0298	0.9442\\
0.0345	0.9424\\
0.0392	0.9405\\
0.0439	0.9383\\
0.0486	0.936\\
0.0534	0.9334\\
0.0581	0.9305\\
0.0628	0.9275\\
0.0675	0.9243\\
0.0723	0.9208\\
0.077	0.9171\\
0.0817	0.9132\\
0.0864	0.909\\
0.0911	0.9047\\
0.0959	0.9001\\
0.1006	0.8953\\
0.1053	0.8903\\
0.11	0.8851\\
0.1148	0.8797\\
0.1195	0.874\\
0.1242	0.8681\\
0.1289	0.862\\
0.1336	0.8557\\
0.1384	0.8492\\
0.1431	0.8424\\
0.1478	0.8354\\
0.1525	0.8283\\
0.1573	0.8208\\
0.162	0.8132\\
0.1667	0.8054\\
0.1714	0.7973\\
0.1761	0.789\\
0.1793	0.7833\\
0.1825	0.7775\\
0.1857	0.7716\\
0.1889	0.7656\\
};
\addlegendentry{Simulations}

\addplot [color=black, forget plot]
  table[row sep=crcr]{%
0.1889	0.7656\\
0.1936	0.7565\\
0.1983	0.7471\\
0.2031	0.7376\\
0.2078	0.7279\\
0.2125	0.7179\\
0.2172	0.7077\\
0.2219	0.6973\\
0.2267	0.6867\\
0.2314	0.6758\\
0.2361	0.6647\\
0.2408	0.6535\\
0.2456	0.642\\
0.2503	0.6302\\
0.255	0.6183\\
0.2597	0.6061\\
0.2644	0.5938\\
0.2692	0.5812\\
0.2739	0.5684\\
0.2786	0.5553\\
0.2833	0.5421\\
0.2881	0.5286\\
0.2928	0.5149\\
0.2975	0.501\\
0.3022	0.4869\\
0.3069	0.4725\\
0.3117	0.458\\
0.3164	0.4432\\
0.3211	0.4282\\
0.3258	0.413\\
0.3306	0.3975\\
0.3353	0.3819\\
0.34	0.366\\
0.3447	0.3499\\
0.3494	0.3336\\
0.3542	0.317\\
0.3589	0.3003\\
0.3636	0.2833\\
0.3683	0.2661\\
0.3731	0.2487\\
0.3778	0.2311\\
};
\addplot [color=black, forget plot]
  table[row sep=crcr]{%
0.3778	0.2311\\
0.384	0.2077\\
0.3871	0.1958\\
0.3901	0.1839\\
0.3949	0.1655\\
0.3996	0.1468\\
0.4043	0.128\\
0.409	0.1089\\
0.4138	0.0896\\
0.4185	0.0701\\
0.4232	0.0504\\
0.4279	0.0304\\
0.4297	0.0229\\
0.4315	0.0153\\
0.4332	0.0077\\
0.435	-0\\
};
\addplot [color=black, forget plot]
  table[row sep=crcr]{%
0.435	0\\
0.435	0.0001\\
0.4351	0.0001\\
0.4351	0.0002\\
0.4352	0.0005\\
0.4352	0.0007\\
0.4353	0.001\\
0.4354	0.0012\\
0.4358	0.0025\\
0.4362	0.0037\\
0.4366	0.005\\
0.4369	0.0062\\
0.4389	0.0124\\
0.4408	0.0186\\
0.4428	0.0248\\
0.4447	0.0309\\
0.448	0.0412\\
0.4513	0.0514\\
0.4579	0.0714\\
0.4612	0.0813\\
0.4645	0.091\\
0.4677	0.1007\\
0.4743	0.1197\\
0.4809	0.1383\\
0.4842	0.1474\\
0.4875	0.1564\\
0.4908	0.1653\\
0.4941	0.1741\\
0.4974	0.1828\\
0.5007	0.1914\\
0.5039	0.1999\\
0.5072	0.2083\\
0.5138	0.2247\\
0.5171	0.2328\\
0.5237	0.2486\\
0.527	0.2563\\
0.5336	0.2715\\
0.5369	0.2789\\
0.5401	0.2862\\
0.5434	0.2934\\
0.5467	0.3005\\
0.55	0.3075\\
0.5533	0.3144\\
0.5566	0.3212\\
0.5599	0.3279\\
0.5632	0.3344\\
0.5641	0.3362\\
0.5649	0.3379\\
0.5667	0.3413\\
};
\addplot [color=black, forget plot]
  table[row sep=crcr]{%
0.5667	0.3413\\
0.5714	0.3504\\
0.5761	0.3592\\
0.5808	0.3679\\
0.5856	0.3763\\
0.5903	0.3845\\
0.595	0.3925\\
0.5997	0.4003\\
0.6044	0.4078\\
0.6092	0.4152\\
0.6139	0.4223\\
0.6186	0.4292\\
0.6233	0.4358\\
0.6281	0.4423\\
0.6328	0.4485\\
0.6375	0.4546\\
0.6422	0.4604\\
0.6469	0.466\\
0.6517	0.4713\\
0.6564	0.4765\\
0.6611	0.4814\\
0.6658	0.4861\\
0.6706	0.4906\\
0.6753	0.4949\\
0.68	0.4989\\
0.6847	0.5027\\
0.6894	0.5064\\
0.6942	0.5098\\
0.6989	0.5129\\
0.7036	0.5159\\
0.7083	0.5186\\
0.7131	0.5212\\
0.7178	0.5235\\
0.7225	0.5255\\
0.7272	0.5274\\
0.7319	0.5291\\
0.7367	0.5305\\
0.7414	0.5317\\
0.7461	0.5327\\
0.7508	0.5334\\
0.7556	0.534\\
};
\addplot [color=black, forget plot]
  table[row sep=crcr]{%
0.7556	0.534\\
0.756	0.534\\
0.7565	0.5341\\
0.757	0.5341\\
0.7575	0.5342\\
0.7613	0.5344\\
0.7651	0.5344\\
};
\addplot [color=black, forget plot]
  table[row sep=crcr]{%
0.7651	0.5344\\
0.7683	0.5344\\
0.7747	0.534\\
0.7811	0.5332\\
0.7856	0.5324\\
0.7901	0.5314\\
0.7946	0.5302\\
0.799	0.5288\\
0.8035	0.5272\\
0.808	0.5254\\
0.8125	0.5234\\
0.817	0.5213\\
0.8215	0.5189\\
0.8259	0.5163\\
0.8304	0.5135\\
0.8349	0.5106\\
0.8394	0.5074\\
0.8439	0.504\\
0.8484	0.5005\\
0.8528	0.4967\\
0.8573	0.4927\\
0.8618	0.4886\\
0.8663	0.4842\\
0.8708	0.4797\\
0.8753	0.4749\\
0.8797	0.47\\
0.8842	0.4649\\
0.8887	0.4595\\
0.8932	0.454\\
0.8977	0.4483\\
0.9022	0.4423\\
0.9066	0.4362\\
0.9111	0.4299\\
0.9156	0.4234\\
0.9201	0.4166\\
0.9246	0.4097\\
0.9291	0.4026\\
0.9335	0.3953\\
0.938	0.3878\\
0.9425	0.3801\\
0.943	0.3792\\
0.944	0.3776\\
0.9444	0.3767\\
};
\addplot [color=black, forget plot]
  table[row sep=crcr]{%
0.9444	0.3767\\
0.9492	0.3683\\
0.9539	0.3597\\
0.9586	0.3508\\
0.9633	0.3417\\
0.9681	0.3324\\
0.9728	0.3229\\
0.9775	0.3132\\
0.9822	0.3032\\
0.9869	0.2931\\
0.9917	0.2827\\
0.9964	0.2721\\
1.0011	0.2613\\
1.0058	0.2502\\
1.0106	0.239\\
1.0153	0.2275\\
1.02	0.2158\\
1.0247	0.2039\\
1.0294	0.1917\\
1.0342	0.1794\\
1.0389	0.1668\\
1.0436	0.154\\
1.0483	0.141\\
1.0531	0.1278\\
1.0578	0.1143\\
1.0625	0.1007\\
1.0672	0.0868\\
1.0767	0.0583\\
1.0813	0.0441\\
1.0859	0.0296\\
1.0906	0.0149\\
1.0952	-0\\
};
\addplot [color=black, forget plot]
  table[row sep=crcr]{%
1.0952	0\\
1.0952	0.0001\\
1.0953	0.0002\\
1.0954	0.0005\\
1.0955	0.0007\\
1.0956	0.001\\
1.0957	0.0012\\
1.0962	0.0025\\
1.0967	0.0037\\
1.0973	0.005\\
1.0978	0.0062\\
1.0987	0.0085\\
1.0997	0.0108\\
1.1006	0.013\\
1.1016	0.0153\\
1.1025	0.0175\\
1.1035	0.0198\\
1.1044	0.022\\
1.1054	0.0242\\
1.1063	0.0265\\
1.1083	0.0309\\
1.1092	0.033\\
1.1102	0.0352\\
1.1111	0.0374\\
1.1121	0.0396\\
1.113	0.0417\\
1.114	0.0439\\
1.1149	0.046\\
1.1159	0.0481\\
1.1168	0.0502\\
1.1178	0.0523\\
1.1187	0.0544\\
1.1197	0.0565\\
1.1206	0.0586\\
1.1226	0.0628\\
1.1235	0.0648\\
1.1245	0.0669\\
1.1254	0.0689\\
1.1264	0.0709\\
1.1273	0.0729\\
1.1283	0.0749\\
1.1292	0.0769\\
1.1302	0.0789\\
1.1311	0.0809\\
1.1324	0.0835\\
1.1327	0.0842\\
1.133	0.0848\\
1.1333	0.0855\\
};
\addplot [color=black, forget plot]
  table[row sep=crcr]{%
1.1333	0.0855\\
1.1354	0.0897\\
1.1375	0.094\\
1.1396	0.0982\\
1.1417	0.1023\\
1.1464	0.1115\\
1.1511	0.1205\\
1.1559	0.1293\\
1.1606	0.1378\\
1.1653	0.1461\\
1.17	0.1543\\
1.1747	0.1621\\
1.1795	0.1698\\
1.1842	0.1773\\
1.1889	0.1845\\
1.1936	0.1915\\
1.1984	0.1983\\
1.2031	0.2049\\
1.2078	0.2113\\
1.2125	0.2174\\
1.2172	0.2233\\
1.222	0.229\\
1.2267	0.2345\\
1.2314	0.2398\\
1.2361	0.2449\\
1.2409	0.2497\\
1.2456	0.2543\\
1.2503	0.2587\\
1.255	0.2629\\
1.2597	0.2668\\
1.2645	0.2706\\
1.2692	0.2741\\
1.2739	0.2774\\
1.2786	0.2805\\
1.2834	0.2833\\
1.2881	0.286\\
1.2928	0.2884\\
1.2975	0.2906\\
1.3022	0.2926\\
1.307	0.2943\\
1.3117	0.2959\\
1.3143	0.2967\\
1.317	0.2974\\
1.3222	0.2986\\
};
\addplot [color=black, forget plot]
  table[row sep=crcr]{%
1.3222	0.2986\\
1.3233	0.2988\\
1.3243	0.299\\
1.3253	0.2991\\
1.3264	0.2993\\
1.3305	0.2999\\
1.3346	0.3003\\
1.3387	0.3005\\
1.3428	0.3006\\
};
\addplot [color=black, forget plot]
  table[row sep=crcr]{%
1.3428	0.3006\\
1.346	0.3006\\
1.3524	0.3002\\
1.3588	0.2994\\
1.363	0.2986\\
1.3672	0.2977\\
1.3714	0.2966\\
1.3756	0.2953\\
1.3798	0.2939\\
1.384	0.2923\\
1.3882	0.2905\\
1.3924	0.2885\\
1.3966	0.2864\\
1.4009	0.2841\\
1.4051	0.2816\\
1.4093	0.2789\\
1.4135	0.2761\\
1.4177	0.2731\\
1.4219	0.2699\\
1.4261	0.2666\\
1.4303	0.263\\
1.4345	0.2593\\
1.4387	0.2555\\
1.4429	0.2514\\
1.4471	0.2472\\
1.4514	0.2428\\
1.4556	0.2382\\
1.4598	0.2335\\
1.464	0.2286\\
1.4682	0.2235\\
1.4724	0.2182\\
1.4766	0.2128\\
1.4808	0.2072\\
1.485	0.2014\\
1.4892	0.1954\\
1.4934	0.1893\\
1.4979	0.1827\\
1.5023	0.1758\\
1.5067	0.1688\\
1.5111	0.1616\\
};
\addplot [color=black, forget plot]
  table[row sep=crcr]{%
1.5111	0.1616\\
1.5158	0.1537\\
1.5206	0.1456\\
1.5253	0.1372\\
1.53	0.1287\\
1.5347	0.1199\\
1.5394	0.1109\\
1.5442	0.1017\\
1.5489	0.0922\\
1.5536	0.0826\\
1.5583	0.0727\\
1.5631	0.0626\\
1.5678	0.0523\\
1.5725	0.0418\\
1.5772	0.031\\
1.5819	0.02\\
1.5867	0.0089\\
1.5876	0.0067\\
1.5885	0.0044\\
1.5903	-0\\
};
\addplot [color=black, forget plot]
  table[row sep=crcr]{%
1.5903	0\\
1.5904	0.0001\\
1.5904	0.0002\\
1.5906	0.0005\\
1.5907	0.0007\\
1.5909	0.001\\
1.591	0.0012\\
1.5917	0.0025\\
1.5931	0.0049\\
1.5938	0.0062\\
1.5992	0.0158\\
1.602	0.0205\\
1.6047	0.0252\\
1.6075	0.0298\\
1.6102	0.0343\\
1.6129	0.0387\\
1.6157	0.043\\
1.6184	0.0473\\
1.6212	0.0515\\
1.6239	0.0556\\
1.6267	0.0597\\
1.6294	0.0637\\
1.6321	0.0676\\
1.6349	0.0714\\
1.6376	0.0752\\
1.6404	0.0788\\
1.6431	0.0825\\
1.6458	0.086\\
1.6486	0.0895\\
1.6513	0.0928\\
1.6541	0.0962\\
1.6568	0.0994\\
1.6596	0.1026\\
1.6623	0.1057\\
1.665	0.1087\\
1.6678	0.1116\\
1.6705	0.1145\\
1.6787	0.1227\\
1.6815	0.1253\\
1.6842	0.1278\\
1.687	0.1302\\
1.6897	0.1326\\
1.6925	0.1349\\
1.6943	0.1364\\
1.6962	0.1379\\
1.6981	0.1393\\
1.7	0.1408\\
};
\addplot [color=black, forget plot]
  table[row sep=crcr]{%
0	0.95\\
0.0008	0.95\\
0.001	0.9499\\
0.0023	0.9499\\
0.0036	0.9498\\
0.0049	0.9496\\
0.0061	0.9495\\
0.0109	0.9489\\
0.0156	0.948\\
0.0203	0.947\\
0.025	0.9457\\
0.0298	0.9442\\
0.0345	0.9424\\
0.0392	0.9405\\
0.0439	0.9383\\
0.0486	0.936\\
0.0534	0.9334\\
0.0581	0.9305\\
0.0628	0.9275\\
0.0675	0.9243\\
0.0723	0.9208\\
0.077	0.9171\\
0.0817	0.9132\\
0.0864	0.909\\
0.0911	0.9047\\
0.0959	0.9001\\
0.1006	0.8953\\
0.1053	0.8903\\
0.11	0.8851\\
0.1148	0.8797\\
0.1195	0.874\\
0.1242	0.8681\\
0.1289	0.862\\
0.1336	0.8557\\
0.1384	0.8492\\
0.1431	0.8424\\
0.1478	0.8354\\
0.1525	0.8283\\
0.1573	0.8208\\
0.162	0.8132\\
0.1667	0.8054\\
0.1714	0.7973\\
0.1761	0.789\\
0.1793	0.7833\\
0.1825	0.7775\\
0.1857	0.7716\\
0.1889	0.7656\\
};
\addplot [color=black, forget plot]
  table[row sep=crcr]{%
0.1889	0.7656\\
0.1936	0.7565\\
0.1983	0.7471\\
0.2031	0.7376\\
0.2078	0.7279\\
0.2125	0.7179\\
0.2172	0.7077\\
0.2219	0.6973\\
0.2267	0.6867\\
0.2314	0.6758\\
0.2361	0.6647\\
0.2408	0.6535\\
0.2456	0.642\\
0.2503	0.6302\\
0.255	0.6183\\
0.2597	0.6061\\
0.2644	0.5938\\
0.2692	0.5812\\
0.2739	0.5684\\
0.2786	0.5553\\
0.2833	0.5421\\
0.2881	0.5286\\
0.2928	0.5149\\
0.2975	0.501\\
0.3022	0.4869\\
0.3069	0.4725\\
0.3117	0.458\\
0.3164	0.4432\\
0.3211	0.4282\\
0.3258	0.413\\
0.3306	0.3975\\
0.3353	0.3819\\
0.34	0.366\\
0.3447	0.3499\\
0.3494	0.3336\\
0.3542	0.317\\
0.3589	0.3003\\
0.3636	0.2833\\
0.3683	0.2661\\
0.3731	0.2487\\
0.3778	0.2311\\
};
\addplot [color=black, forget plot]
  table[row sep=crcr]{%
0.3778	0.2311\\
0.384	0.2077\\
0.3871	0.1958\\
0.3901	0.1839\\
0.3949	0.1655\\
0.3996	0.1468\\
0.4043	0.128\\
0.409	0.1089\\
0.4138	0.0896\\
0.4185	0.0701\\
0.4232	0.0504\\
0.4279	0.0304\\
0.4297	0.0229\\
0.4315	0.0153\\
0.4332	0.0077\\
0.435	-0\\
};
\addplot [color=black, forget plot]
  table[row sep=crcr]{%
0.435	0\\
0.435	0.0001\\
0.4351	0.0001\\
0.4351	0.0002\\
0.4352	0.0005\\
0.4352	0.0007\\
0.4353	0.001\\
0.4354	0.0012\\
0.4358	0.0025\\
0.4362	0.0037\\
0.4366	0.005\\
0.4369	0.0062\\
0.4389	0.0124\\
0.4408	0.0186\\
0.4428	0.0248\\
0.4447	0.0309\\
0.448	0.0412\\
0.4513	0.0514\\
0.4579	0.0714\\
0.4612	0.0813\\
0.4645	0.091\\
0.4677	0.1007\\
0.4743	0.1197\\
0.4809	0.1383\\
0.4842	0.1474\\
0.4875	0.1564\\
0.4908	0.1653\\
0.4941	0.1741\\
0.4974	0.1828\\
0.5007	0.1914\\
0.5039	0.1999\\
0.5072	0.2083\\
0.5138	0.2247\\
0.5171	0.2328\\
0.5237	0.2486\\
0.527	0.2563\\
0.5336	0.2715\\
0.5369	0.2789\\
0.5401	0.2862\\
0.5434	0.2934\\
0.5467	0.3005\\
0.55	0.3075\\
0.5533	0.3144\\
0.5566	0.3212\\
0.5599	0.3279\\
0.5632	0.3344\\
0.5641	0.3362\\
0.5649	0.3379\\
0.5667	0.3413\\
};
\addplot [color=black, forget plot]
  table[row sep=crcr]{%
0.5667	0.3413\\
0.5714	0.3504\\
0.5761	0.3592\\
0.5808	0.3679\\
0.5856	0.3763\\
0.5903	0.3845\\
0.595	0.3925\\
0.5997	0.4003\\
0.6044	0.4078\\
0.6092	0.4152\\
0.6139	0.4223\\
0.6186	0.4292\\
0.6233	0.4358\\
0.6281	0.4423\\
0.6328	0.4485\\
0.6375	0.4546\\
0.6422	0.4604\\
0.6469	0.466\\
0.6517	0.4713\\
0.6564	0.4765\\
0.6611	0.4814\\
0.6658	0.4861\\
0.6706	0.4906\\
0.6753	0.4949\\
0.68	0.4989\\
0.6847	0.5027\\
0.6894	0.5064\\
0.6942	0.5098\\
0.6989	0.5129\\
0.7036	0.5159\\
0.7083	0.5186\\
0.7131	0.5212\\
0.7178	0.5235\\
0.7225	0.5255\\
0.7272	0.5274\\
0.7319	0.5291\\
0.7367	0.5305\\
0.7414	0.5317\\
0.7461	0.5327\\
0.7508	0.5334\\
0.7556	0.534\\
};
\addplot [color=black, forget plot]
  table[row sep=crcr]{%
0.7556	0.534\\
0.756	0.534\\
0.7565	0.5341\\
0.757	0.5341\\
0.7575	0.5342\\
0.7613	0.5344\\
0.7651	0.5344\\
};
\addplot [color=black, forget plot]
  table[row sep=crcr]{%
0.7651	0.5344\\
0.7683	0.5344\\
0.7747	0.534\\
0.7811	0.5332\\
0.7856	0.5324\\
0.7901	0.5314\\
0.7946	0.5302\\
0.799	0.5288\\
0.8035	0.5272\\
0.808	0.5254\\
0.8125	0.5234\\
0.817	0.5213\\
0.8215	0.5189\\
0.8259	0.5163\\
0.8304	0.5135\\
0.8349	0.5106\\
0.8394	0.5074\\
0.8439	0.504\\
0.8484	0.5005\\
0.8528	0.4967\\
0.8573	0.4927\\
0.8618	0.4886\\
0.8663	0.4842\\
0.8708	0.4797\\
0.8753	0.4749\\
0.8797	0.47\\
0.8842	0.4649\\
0.8887	0.4595\\
0.8932	0.454\\
0.8977	0.4483\\
0.9022	0.4423\\
0.9066	0.4362\\
0.9111	0.4299\\
0.9156	0.4234\\
0.9201	0.4166\\
0.9246	0.4097\\
0.9291	0.4026\\
0.9335	0.3953\\
0.938	0.3878\\
0.9425	0.3801\\
0.943	0.3792\\
0.944	0.3776\\
0.9444	0.3767\\
};
\addplot [color=black, forget plot]
  table[row sep=crcr]{%
0.9444	0.3767\\
0.9492	0.3683\\
0.9539	0.3597\\
0.9586	0.3508\\
0.9633	0.3417\\
0.9681	0.3324\\
0.9728	0.3229\\
0.9775	0.3132\\
0.9822	0.3032\\
0.9869	0.2931\\
0.9917	0.2827\\
0.9964	0.2721\\
1.0011	0.2613\\
1.0058	0.2502\\
1.0106	0.239\\
1.0153	0.2275\\
1.02	0.2158\\
1.0247	0.2039\\
1.0294	0.1917\\
1.0342	0.1794\\
1.0389	0.1668\\
1.0436	0.154\\
1.0483	0.141\\
1.0531	0.1278\\
1.0578	0.1143\\
1.0625	0.1007\\
1.0672	0.0868\\
1.0767	0.0583\\
1.0813	0.0441\\
1.0859	0.0296\\
1.0906	0.0149\\
1.0952	-0\\
};
\addplot [color=black, forget plot]
  table[row sep=crcr]{%
1.0952	0\\
1.0952	0.0001\\
1.0953	0.0002\\
1.0954	0.0005\\
1.0955	0.0007\\
1.0956	0.001\\
1.0957	0.0012\\
1.0962	0.0025\\
1.0967	0.0037\\
1.0973	0.005\\
1.0978	0.0062\\
1.0987	0.0085\\
1.0997	0.0108\\
1.1006	0.013\\
1.1016	0.0153\\
1.1025	0.0175\\
1.1035	0.0198\\
1.1044	0.022\\
1.1054	0.0242\\
1.1063	0.0265\\
1.1083	0.0309\\
1.1092	0.033\\
1.1102	0.0352\\
1.1111	0.0374\\
1.1121	0.0396\\
1.113	0.0417\\
1.114	0.0439\\
1.1149	0.046\\
1.1159	0.0481\\
1.1168	0.0502\\
1.1178	0.0523\\
1.1187	0.0544\\
1.1197	0.0565\\
1.1206	0.0586\\
1.1226	0.0628\\
1.1235	0.0648\\
1.1245	0.0669\\
1.1254	0.0689\\
1.1264	0.0709\\
1.1273	0.0729\\
1.1283	0.0749\\
1.1292	0.0769\\
1.1302	0.0789\\
1.1311	0.0809\\
1.1324	0.0835\\
1.1327	0.0842\\
1.133	0.0848\\
1.1333	0.0855\\
};
\addplot [color=black, forget plot]
  table[row sep=crcr]{%
1.1333	0.0855\\
1.1354	0.0897\\
1.1375	0.094\\
1.1396	0.0982\\
1.1417	0.1023\\
1.1464	0.1115\\
1.1511	0.1205\\
1.1559	0.1293\\
1.1606	0.1378\\
1.1653	0.1461\\
1.17	0.1543\\
1.1747	0.1621\\
1.1795	0.1698\\
1.1842	0.1773\\
1.1889	0.1845\\
1.1936	0.1915\\
1.1984	0.1983\\
1.2031	0.2049\\
1.2078	0.2113\\
1.2125	0.2174\\
1.2172	0.2233\\
1.222	0.229\\
1.2267	0.2345\\
1.2314	0.2398\\
1.2361	0.2449\\
1.2409	0.2497\\
1.2456	0.2543\\
1.2503	0.2587\\
1.255	0.2629\\
1.2597	0.2668\\
1.2645	0.2706\\
1.2692	0.2741\\
1.2739	0.2774\\
1.2786	0.2805\\
1.2834	0.2833\\
1.2881	0.286\\
1.2928	0.2884\\
1.2975	0.2906\\
1.3022	0.2926\\
1.307	0.2943\\
1.3117	0.2959\\
1.3143	0.2967\\
1.317	0.2974\\
1.3222	0.2986\\
};
\addplot [color=black, forget plot]
  table[row sep=crcr]{%
1.3222	0.2986\\
1.3233	0.2988\\
1.3243	0.299\\
1.3253	0.2991\\
1.3264	0.2993\\
1.3305	0.2999\\
1.3346	0.3003\\
1.3387	0.3005\\
1.3428	0.3006\\
};
\addplot [color=black, forget plot]
  table[row sep=crcr]{%
1.3428	0.3006\\
1.346	0.3006\\
1.3524	0.3002\\
1.3588	0.2994\\
1.363	0.2986\\
1.3672	0.2977\\
1.3714	0.2966\\
1.3756	0.2953\\
1.3798	0.2939\\
1.384	0.2923\\
1.3882	0.2905\\
1.3924	0.2885\\
1.3966	0.2864\\
1.4009	0.2841\\
1.4051	0.2816\\
1.4093	0.2789\\
1.4135	0.2761\\
1.4177	0.2731\\
1.4219	0.2699\\
1.4261	0.2666\\
1.4303	0.263\\
1.4345	0.2593\\
1.4387	0.2555\\
1.4429	0.2514\\
1.4471	0.2472\\
1.4514	0.2428\\
1.4556	0.2382\\
1.4598	0.2335\\
1.464	0.2286\\
1.4682	0.2235\\
1.4724	0.2182\\
1.4766	0.2128\\
1.4808	0.2072\\
1.485	0.2014\\
1.4892	0.1954\\
1.4934	0.1893\\
1.4979	0.1827\\
1.5023	0.1758\\
1.5067	0.1688\\
1.5111	0.1616\\
};
\addplot [color=black, forget plot]
  table[row sep=crcr]{%
1.5111	0.1616\\
1.5158	0.1537\\
1.5206	0.1456\\
1.5253	0.1372\\
1.53	0.1287\\
1.5347	0.1199\\
1.5394	0.1109\\
1.5442	0.1017\\
1.5489	0.0922\\
1.5536	0.0826\\
1.5583	0.0727\\
1.5631	0.0626\\
1.5678	0.0523\\
1.5725	0.0418\\
1.5772	0.031\\
1.5819	0.02\\
1.5867	0.0089\\
1.5876	0.0067\\
1.5885	0.0044\\
1.5903	-0\\
};
\addplot [color=black, forget plot]
  table[row sep=crcr]{%
1.5903	0\\
1.5904	0.0001\\
1.5904	0.0002\\
1.5906	0.0005\\
1.5907	0.0007\\
1.5909	0.001\\
1.591	0.0012\\
1.5917	0.0025\\
1.5931	0.0049\\
1.5938	0.0062\\
1.5992	0.0158\\
1.602	0.0205\\
1.6047	0.0252\\
1.6075	0.0298\\
1.6102	0.0343\\
1.6129	0.0387\\
1.6157	0.043\\
1.6184	0.0473\\
1.6212	0.0515\\
1.6239	0.0556\\
1.6267	0.0597\\
1.6294	0.0637\\
1.6321	0.0676\\
1.6349	0.0714\\
1.6376	0.0752\\
1.6404	0.0788\\
1.6431	0.0825\\
1.6458	0.086\\
1.6486	0.0895\\
1.6513	0.0928\\
1.6541	0.0962\\
1.6568	0.0994\\
1.6596	0.1026\\
1.6623	0.1057\\
1.665	0.1087\\
1.6678	0.1116\\
1.6705	0.1145\\
1.6787	0.1227\\
1.6815	0.1253\\
1.6842	0.1278\\
1.687	0.1302\\
1.6897	0.1326\\
1.6925	0.1349\\
1.6943	0.1364\\
1.6962	0.1379\\
1.6981	0.1393\\
1.7	0.1408\\
};
\addplot [color=black, forget plot]
  table[row sep=crcr]{%
0	1.05\\
0.0008	1.05\\
0.001	1.0499\\
0.0023	1.0499\\
0.0036	1.0498\\
0.0049	1.0496\\
0.0061	1.0495\\
0.0109	1.0489\\
0.0156	1.048\\
0.0203	1.047\\
0.025	1.0457\\
0.0298	1.0442\\
0.0345	1.0424\\
0.0392	1.0405\\
0.0439	1.0383\\
0.0486	1.036\\
0.0534	1.0334\\
0.0581	1.0305\\
0.0628	1.0275\\
0.0675	1.0243\\
0.0723	1.0208\\
0.077	1.0171\\
0.0817	1.0132\\
0.0864	1.009\\
0.0911	1.0047\\
0.0959	1.0001\\
0.1006	0.9953\\
0.1053	0.9903\\
0.11	0.9851\\
0.1148	0.9797\\
0.1195	0.974\\
0.1242	0.9681\\
0.1289	0.962\\
0.1336	0.9557\\
0.1384	0.9492\\
0.1431	0.9424\\
0.1478	0.9354\\
0.1525	0.9283\\
0.1573	0.9208\\
0.162	0.9132\\
0.1667	0.9054\\
0.1714	0.8973\\
0.1761	0.889\\
0.1793	0.8833\\
0.1825	0.8775\\
0.1857	0.8716\\
0.1889	0.8656\\
};
\addplot [color=black, forget plot]
  table[row sep=crcr]{%
0.1889	0.8656\\
0.1936	0.8565\\
0.1983	0.8471\\
0.2031	0.8376\\
0.2078	0.8279\\
0.2125	0.8179\\
0.2172	0.8077\\
0.2219	0.7973\\
0.2267	0.7867\\
0.2314	0.7758\\
0.2361	0.7647\\
0.2408	0.7535\\
0.2456	0.742\\
0.2503	0.7302\\
0.255	0.7183\\
0.2597	0.7061\\
0.2644	0.6938\\
0.2692	0.6812\\
0.2739	0.6684\\
0.2786	0.6553\\
0.2833	0.6421\\
0.2881	0.6286\\
0.2928	0.6149\\
0.2975	0.601\\
0.3022	0.5869\\
0.3069	0.5725\\
0.3117	0.558\\
0.3164	0.5432\\
0.3211	0.5282\\
0.3258	0.513\\
0.3306	0.4975\\
0.3353	0.4819\\
0.34	0.466\\
0.3447	0.4499\\
0.3494	0.4336\\
0.3542	0.417\\
0.3589	0.4003\\
0.3636	0.3833\\
0.3683	0.3661\\
0.3731	0.3487\\
0.3778	0.3311\\
};
\addplot [color=black, forget plot]
  table[row sep=crcr]{%
0.3778	0.3311\\
0.3822	0.3144\\
0.3866	0.2974\\
0.3911	0.2803\\
0.3955	0.263\\
0.4002	0.2444\\
0.4049	0.2255\\
0.4097	0.2064\\
0.4144	0.187\\
0.4191	0.1675\\
0.4238	0.1477\\
0.4285	0.1278\\
0.4333	0.1076\\
0.438	0.0871\\
0.4427	0.0665\\
0.4474	0.0457\\
0.4522	0.0246\\
0.4535	0.0185\\
0.4549	0.0123\\
0.4562	0.0062\\
0.4576	-0\\
};
\addplot [color=black, forget plot]
  table[row sep=crcr]{%
0.4576	0\\
0.4576	0.0002\\
0.4577	0.0002\\
0.4577	0.0005\\
0.4578	0.0007\\
0.4579	0.001\\
0.458	0.0012\\
0.4583	0.0025\\
0.4587	0.0037\\
0.4591	0.005\\
0.4594	0.0062\\
0.4613	0.0124\\
0.4631	0.0186\\
0.465	0.0248\\
0.4668	0.0309\\
0.4695	0.0399\\
0.4723	0.0489\\
0.4777	0.0665\\
0.4804	0.0752\\
0.4832	0.0838\\
0.4859	0.0924\\
0.4886	0.1009\\
0.4914	0.1093\\
0.4968	0.1259\\
0.4995	0.1341\\
0.5023	0.1422\\
0.505	0.1503\\
0.5077	0.1583\\
0.5104	0.1662\\
0.5132	0.174\\
0.5159	0.1818\\
0.5186	0.1894\\
0.5213	0.1971\\
0.5241	0.2046\\
0.5268	0.2121\\
0.5295	0.2195\\
0.5323	0.2268\\
0.5377	0.2412\\
0.5404	0.2483\\
0.5432	0.2553\\
0.5459	0.2623\\
0.5486	0.2692\\
0.5513	0.276\\
0.5541	0.2827\\
0.5568	0.2894\\
0.5595	0.296\\
0.5622	0.3025\\
0.565	0.309\\
0.5658	0.311\\
0.5662	0.3119\\
0.5667	0.3129\\
};
\addplot [color=black, forget plot]
  table[row sep=crcr]{%
0.5667	0.3129\\
0.5714	0.3239\\
0.5761	0.3345\\
0.5808	0.345\\
0.5856	0.3553\\
0.5903	0.3653\\
0.595	0.3751\\
0.5997	0.3847\\
0.6044	0.3941\\
0.6092	0.4033\\
0.6139	0.4122\\
0.6186	0.421\\
0.6233	0.4295\\
0.6281	0.4378\\
0.6328	0.4458\\
0.6375	0.4537\\
0.6422	0.4613\\
0.6469	0.4687\\
0.6517	0.4759\\
0.6564	0.4829\\
0.6611	0.4897\\
0.6658	0.4962\\
0.6706	0.5025\\
0.6753	0.5086\\
0.68	0.5145\\
0.6847	0.5202\\
0.6894	0.5256\\
0.6942	0.5308\\
0.6989	0.5359\\
0.7036	0.5406\\
0.7083	0.5452\\
0.7131	0.5496\\
0.7178	0.5537\\
0.7225	0.5576\\
0.7272	0.5613\\
0.7319	0.5648\\
0.7367	0.568\\
0.7414	0.5711\\
0.7461	0.5739\\
0.7508	0.5765\\
0.7556	0.5789\\
};
\addplot [color=black, forget plot]
  table[row sep=crcr]{%
0.7556	0.5789\\
0.758	0.58\\
0.763	0.5822\\
0.7654	0.5832\\
0.7701	0.5849\\
0.7749	0.5863\\
0.7796	0.5876\\
0.7843	0.5887\\
0.789	0.5895\\
0.7938	0.5901\\
0.7985	0.5905\\
0.8032	0.5907\\
0.8046	0.5907\\
};
\addplot [color=black, forget plot]
  table[row sep=crcr]{%
0.8046	0.5907\\
0.8072	0.5907\\
0.8078	0.5906\\
0.811	0.5905\\
0.8174	0.5899\\
0.8206	0.5894\\
0.8241	0.5888\\
0.8276	0.5881\\
0.8311	0.5873\\
0.8346	0.5863\\
0.8381	0.5852\\
0.8416	0.584\\
0.8451	0.5827\\
0.8521	0.5797\\
0.8556	0.578\\
0.8591	0.5762\\
0.8661	0.5722\\
0.8696	0.57\\
0.8731	0.5677\\
0.8766	0.5653\\
0.8801	0.5628\\
0.8835	0.5602\\
0.887	0.5574\\
0.8905	0.5545\\
0.894	0.5515\\
0.8975	0.5484\\
0.9045	0.5418\\
0.908	0.5383\\
0.9115	0.5347\\
0.9185	0.5271\\
0.9255	0.5191\\
0.9325	0.5105\\
0.9355	0.5067\\
0.9415	0.4989\\
0.9444	0.4948\\
};
\addplot [color=black, forget plot]
  table[row sep=crcr]{%
0.9444	0.4948\\
0.9492	0.4882\\
0.9539	0.4814\\
0.9586	0.4744\\
0.9633	0.4672\\
0.9681	0.4597\\
0.9728	0.452\\
0.9775	0.4441\\
0.9822	0.436\\
0.9869	0.4277\\
0.9917	0.4191\\
0.9964	0.4103\\
1.0011	0.4013\\
1.0058	0.3921\\
1.0106	0.3827\\
1.0153	0.3731\\
1.02	0.3632\\
1.0247	0.3531\\
1.0294	0.3428\\
1.0342	0.3323\\
1.0389	0.3215\\
1.0436	0.3106\\
1.0483	0.2994\\
1.0531	0.288\\
1.0578	0.2764\\
1.0625	0.2645\\
1.0672	0.2525\\
1.0719	0.2402\\
1.0767	0.2277\\
1.0814	0.215\\
1.0861	0.2021\\
1.0908	0.1889\\
1.0956	0.1756\\
1.1003	0.162\\
1.105	0.1482\\
1.1097	0.1341\\
1.1144	0.1199\\
1.1192	0.1054\\
1.1239	0.0908\\
1.1286	0.0759\\
1.1333	0.0607\\
};
\addplot [color=black, forget plot]
  table[row sep=crcr]{%
1.1333	0.0607\\
1.1343	0.0577\\
1.1352	0.0546\\
1.1362	0.0515\\
1.1371	0.0485\\
1.1408	0.0365\\
1.1444	0.0245\\
1.148	0.0123\\
1.1517	0\\
};
\addplot [color=black, forget plot]
  table[row sep=crcr]{%
1.1517	0\\
1.1517	0.0002\\
1.1518	0.0005\\
1.1519	0.0007\\
1.152	0.001\\
1.1521	0.0012\\
1.1526	0.0025\\
1.1531	0.0037\\
1.1536	0.005\\
1.1541	0.0062\\
1.1566	0.0124\\
1.159	0.0185\\
1.1615	0.0246\\
1.1639	0.0306\\
1.1682	0.0409\\
1.1725	0.051\\
1.1767	0.0609\\
1.181	0.0707\\
1.1853	0.0802\\
1.1895	0.0896\\
1.1938	0.0989\\
1.198	0.1079\\
1.2023	0.1167\\
1.2066	0.1254\\
1.2108	0.1339\\
1.2151	0.1423\\
1.2194	0.1504\\
1.2236	0.1584\\
1.2322	0.1738\\
1.2364	0.1812\\
1.2407	0.1884\\
1.245	0.1955\\
1.2492	0.2024\\
1.2535	0.2091\\
1.2577	0.2157\\
1.262	0.222\\
1.2663	0.2282\\
1.2705	0.2342\\
1.2748	0.24\\
1.2791	0.2457\\
1.2833	0.2512\\
1.2919	0.2616\\
1.2961	0.2665\\
1.3004	0.2712\\
1.3047	0.2758\\
1.3089	0.2802\\
1.3132	0.2844\\
1.3174	0.2885\\
1.3198	0.2907\\
1.321	0.2917\\
1.3222	0.2928\\
};
\addplot [color=black, forget plot]
  table[row sep=crcr]{%
1.3222	0.2928\\
1.3267	0.2967\\
1.3312	0.3003\\
1.3357	0.3038\\
1.3402	0.3071\\
1.345	0.3103\\
1.3497	0.3133\\
1.3544	0.316\\
1.3591	0.3186\\
1.3639	0.3209\\
1.3686	0.3231\\
1.3733	0.3249\\
1.378	0.3266\\
1.3827	0.3281\\
1.3875	0.3293\\
1.3922	0.3304\\
1.3969	0.3312\\
1.4007	0.3316\\
1.4044	0.332\\
1.4082	0.3322\\
1.4119	0.3323\\
};
\addplot [color=black, forget plot]
  table[row sep=crcr]{%
1.4119	0.3323\\
1.4132	0.3323\\
1.4138	0.3322\\
1.4151	0.3322\\
1.4176	0.3321\\
1.4226	0.3317\\
1.425	0.3314\\
1.4275	0.3311\\
1.43	0.3307\\
1.435	0.3297\\
1.4374	0.3291\\
1.4424	0.3277\\
1.4474	0.3261\\
1.4498	0.3252\\
1.4523	0.3243\\
1.4548	0.3233\\
1.4573	0.3222\\
1.4598	0.321\\
1.4622	0.3199\\
1.4672	0.3173\\
1.4722	0.3145\\
1.4746	0.313\\
1.4796	0.3098\\
1.4846	0.3064\\
1.487	0.3046\\
1.492	0.3008\\
1.4945	0.2988\\
1.4969	0.2968\\
1.5019	0.2926\\
1.5044	0.2903\\
1.5061	0.2888\\
1.5077	0.2872\\
1.5111	0.284\\
};
\addplot [color=black, forget plot]
  table[row sep=crcr]{%
1.5111	0.284\\
1.5158	0.2793\\
1.5206	0.2744\\
1.5253	0.2692\\
1.53	0.2639\\
1.5347	0.2583\\
1.5394	0.2525\\
1.5442	0.2465\\
1.5489	0.2403\\
1.5536	0.2338\\
1.5583	0.2271\\
1.5631	0.2202\\
1.5678	0.2131\\
1.5725	0.2058\\
1.5772	0.1983\\
1.5819	0.1905\\
1.5867	0.1825\\
1.5914	0.1743\\
1.5961	0.1659\\
1.6008	0.1572\\
1.6056	0.1484\\
1.6103	0.1393\\
1.615	0.13\\
1.6197	0.1205\\
1.6244	0.1107\\
1.6292	0.1008\\
1.6339	0.0906\\
1.6386	0.0802\\
1.6433	0.0696\\
1.6481	0.0588\\
1.6528	0.0477\\
1.6575	0.0365\\
1.6622	0.025\\
1.6672	0.0126\\
1.6722	-0\\
};
\addplot [color=black, forget plot]
  table[row sep=crcr]{%
1.6722	0\\
1.6722	0.0001\\
1.6723	0.0001\\
1.6723	0.0002\\
1.6724	0.0005\\
1.6726	0.0007\\
1.6727	0.001\\
1.6728	0.0012\\
1.6735	0.0025\\
1.6741	0.0037\\
1.6748	0.0049\\
1.6755	0.0062\\
1.6761	0.0075\\
1.6789	0.0127\\
1.6796	0.0139\\
1.6817	0.0178\\
1.6824	0.019\\
1.6831	0.0203\\
1.6838	0.0215\\
1.6845	0.0228\\
1.6852	0.024\\
1.6859	0.0253\\
1.6873	0.0277\\
1.688	0.029\\
1.6894	0.0314\\
1.69	0.0326\\
1.6949	0.041\\
1.6956	0.0421\\
1.697	0.0445\\
1.6977	0.0456\\
1.6983	0.0466\\
1.6988	0.0475\\
1.6994	0.0485\\
1.7	0.0494\\
};
\addplot [color=black, forget plot]
  table[row sep=crcr]{%
0	0.95\\
0.0008	0.95\\
0.001	0.9499\\
0.0023	0.9499\\
0.0036	0.9498\\
0.0049	0.9496\\
0.0061	0.9495\\
0.0109	0.9489\\
0.0156	0.948\\
0.0203	0.947\\
0.025	0.9457\\
0.0298	0.9442\\
0.0345	0.9424\\
0.0392	0.9405\\
0.0439	0.9383\\
0.0486	0.936\\
0.0534	0.9334\\
0.0581	0.9305\\
0.0628	0.9275\\
0.0675	0.9243\\
0.0723	0.9208\\
0.077	0.9171\\
0.0817	0.9132\\
0.0864	0.909\\
0.0911	0.9047\\
0.0959	0.9001\\
0.1006	0.8953\\
0.1053	0.8903\\
0.11	0.8851\\
0.1148	0.8797\\
0.1195	0.874\\
0.1242	0.8681\\
0.1289	0.862\\
0.1336	0.8557\\
0.1384	0.8492\\
0.1431	0.8424\\
0.1478	0.8354\\
0.1525	0.8283\\
0.1573	0.8208\\
0.162	0.8132\\
0.1667	0.8054\\
0.1714	0.7973\\
0.1761	0.789\\
0.1793	0.7833\\
0.1825	0.7775\\
0.1857	0.7716\\
0.1889	0.7656\\
};
\addplot [color=black, forget plot]
  table[row sep=crcr]{%
0.1889	0.7656\\
0.1936	0.7565\\
0.1983	0.7471\\
0.2031	0.7376\\
0.2078	0.7279\\
0.2125	0.7179\\
0.2172	0.7077\\
0.2219	0.6973\\
0.2267	0.6867\\
0.2314	0.6758\\
0.2361	0.6647\\
0.2408	0.6535\\
0.2456	0.642\\
0.2503	0.6302\\
0.255	0.6183\\
0.2597	0.6061\\
0.2644	0.5938\\
0.2692	0.5812\\
0.2739	0.5684\\
0.2786	0.5553\\
0.2833	0.5421\\
0.2881	0.5286\\
0.2928	0.5149\\
0.2975	0.501\\
0.3022	0.4869\\
0.3069	0.4725\\
0.3117	0.458\\
0.3164	0.4432\\
0.3211	0.4282\\
0.3258	0.413\\
0.3306	0.3975\\
0.3353	0.3819\\
0.34	0.366\\
0.3447	0.3499\\
0.3494	0.3336\\
0.3542	0.317\\
0.3589	0.3003\\
0.3636	0.2833\\
0.3683	0.2661\\
0.3731	0.2487\\
0.3778	0.2311\\
};
\addplot [color=black, forget plot]
  table[row sep=crcr]{%
0.3778	0.2311\\
0.384	0.2077\\
0.3871	0.1958\\
0.3901	0.1839\\
0.3949	0.1655\\
0.3996	0.1468\\
0.4043	0.128\\
0.409	0.1089\\
0.4138	0.0896\\
0.4185	0.0701\\
0.4232	0.0504\\
0.4279	0.0304\\
0.4297	0.0229\\
0.4315	0.0153\\
0.4332	0.0077\\
0.435	-0\\
};
\addplot [color=black, forget plot]
  table[row sep=crcr]{%
0.435	0\\
0.435	0.0001\\
0.4351	0.0001\\
0.4351	0.0002\\
0.4352	0.0005\\
0.4352	0.0007\\
0.4353	0.001\\
0.4354	0.0012\\
0.4358	0.0025\\
0.4362	0.0037\\
0.4366	0.005\\
0.4369	0.0062\\
0.4389	0.0124\\
0.4408	0.0186\\
0.4428	0.0248\\
0.4447	0.0309\\
0.448	0.0412\\
0.4513	0.0514\\
0.4579	0.0714\\
0.4612	0.0813\\
0.4645	0.091\\
0.4677	0.1007\\
0.4743	0.1197\\
0.4809	0.1383\\
0.4842	0.1474\\
0.4875	0.1564\\
0.4908	0.1653\\
0.4941	0.1741\\
0.4974	0.1828\\
0.5007	0.1914\\
0.5039	0.1999\\
0.5072	0.2083\\
0.5138	0.2247\\
0.5171	0.2328\\
0.5237	0.2486\\
0.527	0.2563\\
0.5336	0.2715\\
0.5369	0.2789\\
0.5401	0.2862\\
0.5434	0.2934\\
0.5467	0.3005\\
0.55	0.3075\\
0.5533	0.3144\\
0.5566	0.3212\\
0.5599	0.3279\\
0.5632	0.3344\\
0.5641	0.3362\\
0.5649	0.3379\\
0.5667	0.3413\\
};
\addplot [color=black, forget plot]
  table[row sep=crcr]{%
0.5667	0.3413\\
0.5714	0.3504\\
0.5761	0.3592\\
0.5808	0.3679\\
0.5856	0.3763\\
0.5903	0.3845\\
0.595	0.3925\\
0.5997	0.4003\\
0.6044	0.4078\\
0.6092	0.4152\\
0.6139	0.4223\\
0.6186	0.4292\\
0.6233	0.4358\\
0.6281	0.4423\\
0.6328	0.4485\\
0.6375	0.4546\\
0.6422	0.4604\\
0.6469	0.466\\
0.6517	0.4713\\
0.6564	0.4765\\
0.6611	0.4814\\
0.6658	0.4861\\
0.6706	0.4906\\
0.6753	0.4949\\
0.68	0.4989\\
0.6847	0.5027\\
0.6894	0.5064\\
0.6942	0.5098\\
0.6989	0.5129\\
0.7036	0.5159\\
0.7083	0.5186\\
0.7131	0.5212\\
0.7178	0.5235\\
0.7225	0.5255\\
0.7272	0.5274\\
0.7319	0.5291\\
0.7367	0.5305\\
0.7414	0.5317\\
0.7461	0.5327\\
0.7508	0.5334\\
0.7556	0.534\\
};
\addplot [color=black, forget plot]
  table[row sep=crcr]{%
0.7556	0.534\\
0.756	0.534\\
0.7565	0.5341\\
0.757	0.5341\\
0.7575	0.5342\\
0.7613	0.5344\\
0.7651	0.5344\\
};
\addplot [color=black, forget plot]
  table[row sep=crcr]{%
0.7651	0.5344\\
0.7683	0.5344\\
0.7747	0.534\\
0.7811	0.5332\\
0.7856	0.5324\\
0.7901	0.5314\\
0.7946	0.5302\\
0.799	0.5288\\
0.8035	0.5272\\
0.808	0.5254\\
0.8125	0.5234\\
0.817	0.5213\\
0.8215	0.5189\\
0.8259	0.5163\\
0.8304	0.5135\\
0.8349	0.5106\\
0.8394	0.5074\\
0.8439	0.504\\
0.8484	0.5005\\
0.8528	0.4967\\
0.8573	0.4927\\
0.8618	0.4886\\
0.8663	0.4842\\
0.8708	0.4797\\
0.8753	0.4749\\
0.8797	0.47\\
0.8842	0.4649\\
0.8887	0.4595\\
0.8932	0.454\\
0.8977	0.4483\\
0.9022	0.4423\\
0.9066	0.4362\\
0.9111	0.4299\\
0.9156	0.4234\\
0.9201	0.4166\\
0.9246	0.4097\\
0.9291	0.4026\\
0.9335	0.3953\\
0.938	0.3878\\
0.9425	0.3801\\
0.943	0.3792\\
0.944	0.3776\\
0.9444	0.3767\\
};
\addplot [color=black, forget plot]
  table[row sep=crcr]{%
0.9444	0.3767\\
0.9492	0.3683\\
0.9539	0.3597\\
0.9586	0.3508\\
0.9633	0.3417\\
0.9681	0.3324\\
0.9728	0.3229\\
0.9775	0.3132\\
0.9822	0.3032\\
0.9869	0.2931\\
0.9917	0.2827\\
0.9964	0.2721\\
1.0011	0.2613\\
1.0058	0.2502\\
1.0106	0.239\\
1.0153	0.2275\\
1.02	0.2158\\
1.0247	0.2039\\
1.0294	0.1917\\
1.0342	0.1794\\
1.0389	0.1668\\
1.0436	0.154\\
1.0483	0.141\\
1.0531	0.1278\\
1.0578	0.1143\\
1.0625	0.1007\\
1.0672	0.0868\\
1.0767	0.0583\\
1.0813	0.0441\\
1.0859	0.0296\\
1.0906	0.0149\\
1.0952	-0\\
};
\addplot [color=black, forget plot]
  table[row sep=crcr]{%
1.0952	0\\
1.0952	0.0001\\
1.0953	0.0002\\
1.0954	0.0005\\
1.0955	0.0007\\
1.0956	0.001\\
1.0957	0.0012\\
1.0962	0.0025\\
1.0967	0.0037\\
1.0973	0.005\\
1.0978	0.0062\\
1.0987	0.0085\\
1.0997	0.0108\\
1.1006	0.013\\
1.1016	0.0153\\
1.1025	0.0175\\
1.1035	0.0198\\
1.1044	0.022\\
1.1054	0.0242\\
1.1063	0.0265\\
1.1083	0.0309\\
1.1092	0.033\\
1.1102	0.0352\\
1.1111	0.0374\\
1.1121	0.0396\\
1.113	0.0417\\
1.114	0.0439\\
1.1149	0.046\\
1.1159	0.0481\\
1.1168	0.0502\\
1.1178	0.0523\\
1.1187	0.0544\\
1.1197	0.0565\\
1.1206	0.0586\\
1.1226	0.0628\\
1.1235	0.0648\\
1.1245	0.0669\\
1.1254	0.0689\\
1.1264	0.0709\\
1.1273	0.0729\\
1.1283	0.0749\\
1.1292	0.0769\\
1.1302	0.0789\\
1.1311	0.0809\\
1.1324	0.0835\\
1.1327	0.0842\\
1.133	0.0848\\
1.1333	0.0855\\
};
\addplot [color=black, forget plot]
  table[row sep=crcr]{%
1.1333	0.0855\\
1.1354	0.0897\\
1.1375	0.094\\
1.1396	0.0982\\
1.1417	0.1023\\
1.1464	0.1115\\
1.1511	0.1205\\
1.1559	0.1293\\
1.1606	0.1378\\
1.1653	0.1461\\
1.17	0.1543\\
1.1747	0.1621\\
1.1795	0.1698\\
1.1842	0.1773\\
1.1889	0.1845\\
1.1936	0.1915\\
1.1984	0.1983\\
1.2031	0.2049\\
1.2078	0.2113\\
1.2125	0.2174\\
1.2172	0.2233\\
1.222	0.229\\
1.2267	0.2345\\
1.2314	0.2398\\
1.2361	0.2449\\
1.2409	0.2497\\
1.2456	0.2543\\
1.2503	0.2587\\
1.255	0.2629\\
1.2597	0.2668\\
1.2645	0.2706\\
1.2692	0.2741\\
1.2739	0.2774\\
1.2786	0.2805\\
1.2834	0.2833\\
1.2881	0.286\\
1.2928	0.2884\\
1.2975	0.2906\\
1.3022	0.2926\\
1.307	0.2943\\
1.3117	0.2959\\
1.3143	0.2967\\
1.317	0.2974\\
1.3222	0.2986\\
};
\addplot [color=black, forget plot]
  table[row sep=crcr]{%
1.3222	0.2986\\
1.3233	0.2988\\
1.3243	0.299\\
1.3253	0.2991\\
1.3264	0.2993\\
1.3305	0.2999\\
1.3346	0.3003\\
1.3387	0.3005\\
1.3428	0.3006\\
};
\addplot [color=black, forget plot]
  table[row sep=crcr]{%
1.3428	0.3006\\
1.346	0.3006\\
1.3524	0.3002\\
1.3588	0.2994\\
1.363	0.2986\\
1.3672	0.2977\\
1.3714	0.2966\\
1.3756	0.2953\\
1.3798	0.2939\\
1.384	0.2923\\
1.3882	0.2905\\
1.3924	0.2885\\
1.3966	0.2864\\
1.4009	0.2841\\
1.4051	0.2816\\
1.4093	0.2789\\
1.4135	0.2761\\
1.4177	0.2731\\
1.4219	0.2699\\
1.4261	0.2666\\
1.4303	0.263\\
1.4345	0.2593\\
1.4387	0.2555\\
1.4429	0.2514\\
1.4471	0.2472\\
1.4514	0.2428\\
1.4556	0.2382\\
1.4598	0.2335\\
1.464	0.2286\\
1.4682	0.2235\\
1.4724	0.2182\\
1.4766	0.2128\\
1.4808	0.2072\\
1.485	0.2014\\
1.4892	0.1954\\
1.4934	0.1893\\
1.4979	0.1827\\
1.5023	0.1758\\
1.5067	0.1688\\
1.5111	0.1616\\
};
\addplot [color=black, forget plot]
  table[row sep=crcr]{%
1.5111	0.1616\\
1.5158	0.1537\\
1.5206	0.1456\\
1.5253	0.1372\\
1.53	0.1287\\
1.5347	0.1199\\
1.5394	0.1109\\
1.5442	0.1017\\
1.5489	0.0922\\
1.5536	0.0826\\
1.5583	0.0727\\
1.5631	0.0626\\
1.5678	0.0523\\
1.5725	0.0418\\
1.5772	0.031\\
1.5819	0.02\\
1.5867	0.0089\\
1.5876	0.0067\\
1.5885	0.0044\\
1.5903	-0\\
};
\addplot [color=black, forget plot]
  table[row sep=crcr]{%
1.5903	0\\
1.5904	0.0001\\
1.5904	0.0002\\
1.5906	0.0005\\
1.5907	0.0007\\
1.5909	0.001\\
1.591	0.0012\\
1.5917	0.0025\\
1.5931	0.0049\\
1.5938	0.0062\\
1.5992	0.0158\\
1.602	0.0205\\
1.6047	0.0252\\
1.6075	0.0298\\
1.6102	0.0343\\
1.6129	0.0387\\
1.6157	0.043\\
1.6184	0.0473\\
1.6212	0.0515\\
1.6239	0.0556\\
1.6267	0.0597\\
1.6294	0.0637\\
1.6321	0.0676\\
1.6349	0.0714\\
1.6376	0.0752\\
1.6404	0.0788\\
1.6431	0.0825\\
1.6458	0.086\\
1.6486	0.0895\\
1.6513	0.0928\\
1.6541	0.0962\\
1.6568	0.0994\\
1.6596	0.1026\\
1.6623	0.1057\\
1.665	0.1087\\
1.6678	0.1116\\
1.6705	0.1145\\
1.6787	0.1227\\
1.6815	0.1253\\
1.6842	0.1278\\
1.687	0.1302\\
1.6897	0.1326\\
1.6925	0.1349\\
1.6943	0.1364\\
1.6962	0.1379\\
1.6981	0.1393\\
1.7	0.1408\\
};
\addplot [color=black, forget plot]
  table[row sep=crcr]{%
0	1.05\\
0.0008	1.05\\
0.001	1.0499\\
0.0023	1.0499\\
0.0036	1.0498\\
0.0049	1.0496\\
0.0061	1.0495\\
0.0109	1.0489\\
0.0156	1.048\\
0.0203	1.047\\
0.025	1.0457\\
0.0298	1.0442\\
0.0345	1.0424\\
0.0392	1.0405\\
0.0439	1.0383\\
0.0486	1.036\\
0.0534	1.0334\\
0.0581	1.0305\\
0.0628	1.0275\\
0.0675	1.0243\\
0.0723	1.0208\\
0.077	1.0171\\
0.0817	1.0132\\
0.0864	1.009\\
0.0911	1.0047\\
0.0959	1.0001\\
0.1006	0.9953\\
0.1053	0.9903\\
0.11	0.9851\\
0.1148	0.9797\\
0.1195	0.974\\
0.1242	0.9681\\
0.1289	0.962\\
0.1336	0.9557\\
0.1384	0.9492\\
0.1431	0.9424\\
0.1478	0.9354\\
0.1525	0.9283\\
0.1573	0.9208\\
0.162	0.9132\\
0.1667	0.9054\\
0.1714	0.8973\\
0.1761	0.889\\
0.1793	0.8833\\
0.1825	0.8775\\
0.1857	0.8716\\
0.1889	0.8656\\
};
\addplot [color=black, forget plot]
  table[row sep=crcr]{%
0.1889	0.8656\\
0.1936	0.8565\\
0.1983	0.8471\\
0.2031	0.8376\\
0.2078	0.8279\\
0.2125	0.8179\\
0.2172	0.8077\\
0.2219	0.7973\\
0.2267	0.7867\\
0.2314	0.7758\\
0.2361	0.7647\\
0.2408	0.7535\\
0.2456	0.742\\
0.2503	0.7302\\
0.255	0.7183\\
0.2597	0.7061\\
0.2644	0.6938\\
0.2692	0.6812\\
0.2739	0.6684\\
0.2786	0.6553\\
0.2833	0.6421\\
0.2881	0.6286\\
0.2928	0.6149\\
0.2975	0.601\\
0.3022	0.5869\\
0.3069	0.5725\\
0.3117	0.558\\
0.3164	0.5432\\
0.3211	0.5282\\
0.3258	0.513\\
0.3306	0.4975\\
0.3353	0.4819\\
0.34	0.466\\
0.3447	0.4499\\
0.3494	0.4336\\
0.3542	0.417\\
0.3589	0.4003\\
0.3636	0.3833\\
0.3683	0.3661\\
0.3731	0.3487\\
0.3778	0.3311\\
};
\addplot [color=black, forget plot]
  table[row sep=crcr]{%
0.3778	0.3311\\
0.3822	0.3144\\
0.3866	0.2974\\
0.3911	0.2803\\
0.3955	0.263\\
0.4002	0.2444\\
0.4049	0.2255\\
0.4097	0.2064\\
0.4144	0.187\\
0.4191	0.1675\\
0.4238	0.1477\\
0.4285	0.1278\\
0.4333	0.1076\\
0.438	0.0871\\
0.4427	0.0665\\
0.4474	0.0457\\
0.4522	0.0246\\
0.4535	0.0185\\
0.4549	0.0123\\
0.4562	0.0062\\
0.4576	-0\\
};
\addplot [color=black, forget plot]
  table[row sep=crcr]{%
0.4576	0\\
0.4576	0.0002\\
0.4577	0.0002\\
0.4577	0.0005\\
0.4578	0.0007\\
0.4579	0.001\\
0.458	0.0012\\
0.4583	0.0025\\
0.4587	0.0037\\
0.4591	0.005\\
0.4594	0.0062\\
0.4613	0.0124\\
0.4631	0.0186\\
0.465	0.0248\\
0.4668	0.0309\\
0.4695	0.0399\\
0.4723	0.0489\\
0.4777	0.0665\\
0.4804	0.0752\\
0.4832	0.0838\\
0.4859	0.0924\\
0.4886	0.1009\\
0.4914	0.1093\\
0.4968	0.1259\\
0.4995	0.1341\\
0.5023	0.1422\\
0.505	0.1503\\
0.5077	0.1583\\
0.5104	0.1662\\
0.5132	0.174\\
0.5159	0.1818\\
0.5186	0.1894\\
0.5213	0.1971\\
0.5241	0.2046\\
0.5268	0.2121\\
0.5295	0.2195\\
0.5323	0.2268\\
0.5377	0.2412\\
0.5404	0.2483\\
0.5432	0.2553\\
0.5459	0.2623\\
0.5486	0.2692\\
0.5513	0.276\\
0.5541	0.2827\\
0.5568	0.2894\\
0.5595	0.296\\
0.5622	0.3025\\
0.565	0.309\\
0.5658	0.311\\
0.5662	0.3119\\
0.5667	0.3129\\
};
\addplot [color=black, forget plot]
  table[row sep=crcr]{%
0.5667	0.3129\\
0.5714	0.3239\\
0.5761	0.3345\\
0.5808	0.345\\
0.5856	0.3553\\
0.5903	0.3653\\
0.595	0.3751\\
0.5997	0.3847\\
0.6044	0.3941\\
0.6092	0.4033\\
0.6139	0.4122\\
0.6186	0.421\\
0.6233	0.4295\\
0.6281	0.4378\\
0.6328	0.4458\\
0.6375	0.4537\\
0.6422	0.4613\\
0.6469	0.4687\\
0.6517	0.4759\\
0.6564	0.4829\\
0.6611	0.4897\\
0.6658	0.4962\\
0.6706	0.5025\\
0.6753	0.5086\\
0.68	0.5145\\
0.6847	0.5202\\
0.6894	0.5256\\
0.6942	0.5308\\
0.6989	0.5359\\
0.7036	0.5406\\
0.7083	0.5452\\
0.7131	0.5496\\
0.7178	0.5537\\
0.7225	0.5576\\
0.7272	0.5613\\
0.7319	0.5648\\
0.7367	0.568\\
0.7414	0.5711\\
0.7461	0.5739\\
0.7508	0.5765\\
0.7556	0.5789\\
};
\addplot [color=black, forget plot]
  table[row sep=crcr]{%
0.7556	0.5789\\
0.758	0.58\\
0.763	0.5822\\
0.7654	0.5832\\
0.7701	0.5849\\
0.7749	0.5863\\
0.7796	0.5876\\
0.7843	0.5887\\
0.789	0.5895\\
0.7938	0.5901\\
0.7985	0.5905\\
0.8032	0.5907\\
0.8046	0.5907\\
};
\addplot [color=black, forget plot]
  table[row sep=crcr]{%
0.8046	0.5907\\
0.8072	0.5907\\
0.8078	0.5906\\
0.811	0.5905\\
0.8174	0.5899\\
0.8206	0.5894\\
0.8241	0.5888\\
0.8276	0.5881\\
0.8311	0.5873\\
0.8346	0.5863\\
0.8381	0.5852\\
0.8416	0.584\\
0.8451	0.5827\\
0.8521	0.5797\\
0.8556	0.578\\
0.8591	0.5762\\
0.8661	0.5722\\
0.8696	0.57\\
0.8731	0.5677\\
0.8766	0.5653\\
0.8801	0.5628\\
0.8835	0.5602\\
0.887	0.5574\\
0.8905	0.5545\\
0.894	0.5515\\
0.8975	0.5484\\
0.9045	0.5418\\
0.908	0.5383\\
0.9115	0.5347\\
0.9185	0.5271\\
0.9255	0.5191\\
0.9325	0.5105\\
0.9355	0.5067\\
0.9415	0.4989\\
0.9444	0.4948\\
};
\addplot [color=black, forget plot]
  table[row sep=crcr]{%
0.9444	0.4948\\
0.9492	0.4882\\
0.9539	0.4814\\
0.9586	0.4744\\
0.9633	0.4672\\
0.9681	0.4597\\
0.9728	0.452\\
0.9775	0.4441\\
0.9822	0.436\\
0.9869	0.4277\\
0.9917	0.4191\\
0.9964	0.4103\\
1.0011	0.4013\\
1.0058	0.3921\\
1.0106	0.3827\\
1.0153	0.3731\\
1.02	0.3632\\
1.0247	0.3531\\
1.0294	0.3428\\
1.0342	0.3323\\
1.0389	0.3215\\
1.0436	0.3106\\
1.0483	0.2994\\
1.0531	0.288\\
1.0578	0.2764\\
1.0625	0.2645\\
1.0672	0.2525\\
1.0719	0.2402\\
1.0767	0.2277\\
1.0814	0.215\\
1.0861	0.2021\\
1.0908	0.1889\\
1.0956	0.1756\\
1.1003	0.162\\
1.105	0.1482\\
1.1097	0.1341\\
1.1144	0.1199\\
1.1192	0.1054\\
1.1239	0.0908\\
1.1286	0.0759\\
1.1333	0.0607\\
};
\addplot [color=black, forget plot]
  table[row sep=crcr]{%
1.1333	0.0607\\
1.1343	0.0577\\
1.1352	0.0546\\
1.1362	0.0515\\
1.1371	0.0485\\
1.1408	0.0365\\
1.1444	0.0245\\
1.148	0.0123\\
1.1517	0\\
};
\addplot [color=black, forget plot]
  table[row sep=crcr]{%
1.1517	0\\
1.1517	0.0002\\
1.1518	0.0005\\
1.1519	0.0007\\
1.152	0.001\\
1.1521	0.0012\\
1.1526	0.0025\\
1.1531	0.0037\\
1.1536	0.005\\
1.1541	0.0062\\
1.1566	0.0124\\
1.159	0.0185\\
1.1615	0.0246\\
1.1639	0.0306\\
1.1682	0.0409\\
1.1725	0.051\\
1.1767	0.0609\\
1.181	0.0707\\
1.1853	0.0802\\
1.1895	0.0896\\
1.1938	0.0989\\
1.198	0.1079\\
1.2023	0.1167\\
1.2066	0.1254\\
1.2108	0.1339\\
1.2151	0.1423\\
1.2194	0.1504\\
1.2236	0.1584\\
1.2322	0.1738\\
1.2364	0.1812\\
1.2407	0.1884\\
1.245	0.1955\\
1.2492	0.2024\\
1.2535	0.2091\\
1.2577	0.2157\\
1.262	0.222\\
1.2663	0.2282\\
1.2705	0.2342\\
1.2748	0.24\\
1.2791	0.2457\\
1.2833	0.2512\\
1.2919	0.2616\\
1.2961	0.2665\\
1.3004	0.2712\\
1.3047	0.2758\\
1.3089	0.2802\\
1.3132	0.2844\\
1.3174	0.2885\\
1.3198	0.2907\\
1.321	0.2917\\
1.3222	0.2928\\
};
\addplot [color=black, forget plot]
  table[row sep=crcr]{%
1.3222	0.2928\\
1.3267	0.2967\\
1.3312	0.3003\\
1.3357	0.3038\\
1.3402	0.3071\\
1.345	0.3103\\
1.3497	0.3133\\
1.3544	0.316\\
1.3591	0.3186\\
1.3639	0.3209\\
1.3686	0.3231\\
1.3733	0.3249\\
1.378	0.3266\\
1.3827	0.3281\\
1.3875	0.3293\\
1.3922	0.3304\\
1.3969	0.3312\\
1.4007	0.3316\\
1.4044	0.332\\
1.4082	0.3322\\
1.4119	0.3323\\
};
\addplot [color=black, forget plot]
  table[row sep=crcr]{%
1.4119	0.3323\\
1.4132	0.3323\\
1.4138	0.3322\\
1.4151	0.3322\\
1.4176	0.3321\\
1.4226	0.3317\\
1.425	0.3314\\
1.4275	0.3311\\
1.43	0.3307\\
1.435	0.3297\\
1.4374	0.3291\\
1.4424	0.3277\\
1.4474	0.3261\\
1.4498	0.3252\\
1.4523	0.3243\\
1.4548	0.3233\\
1.4573	0.3222\\
1.4598	0.321\\
1.4622	0.3199\\
1.4672	0.3173\\
1.4722	0.3145\\
1.4746	0.313\\
1.4796	0.3098\\
1.4846	0.3064\\
1.487	0.3046\\
1.492	0.3008\\
1.4945	0.2988\\
1.4969	0.2968\\
1.5019	0.2926\\
1.5044	0.2903\\
1.5061	0.2888\\
1.5077	0.2872\\
1.5111	0.284\\
};
\addplot [color=black, forget plot]
  table[row sep=crcr]{%
1.5111	0.284\\
1.5158	0.2793\\
1.5206	0.2744\\
1.5253	0.2692\\
1.53	0.2639\\
1.5347	0.2583\\
1.5394	0.2525\\
1.5442	0.2465\\
1.5489	0.2403\\
1.5536	0.2338\\
1.5583	0.2271\\
1.5631	0.2202\\
1.5678	0.2131\\
1.5725	0.2058\\
1.5772	0.1983\\
1.5819	0.1905\\
1.5867	0.1825\\
1.5914	0.1743\\
1.5961	0.1659\\
1.6008	0.1572\\
1.6056	0.1484\\
1.6103	0.1393\\
1.615	0.13\\
1.6197	0.1205\\
1.6244	0.1107\\
1.6292	0.1008\\
1.6339	0.0906\\
1.6386	0.0802\\
1.6433	0.0696\\
1.6481	0.0588\\
1.6528	0.0477\\
1.6575	0.0365\\
1.6622	0.025\\
1.6672	0.0126\\
1.6722	-0\\
};
\addplot [color=black, forget plot]
  table[row sep=crcr]{%
1.6722	0\\
1.6722	0.0001\\
1.6723	0.0001\\
1.6723	0.0002\\
1.6724	0.0005\\
1.6726	0.0007\\
1.6727	0.001\\
1.6728	0.0012\\
1.6735	0.0025\\
1.6741	0.0037\\
1.6748	0.0049\\
1.6755	0.0062\\
1.6761	0.0075\\
1.6789	0.0127\\
1.6796	0.0139\\
1.6817	0.0178\\
1.6824	0.019\\
1.6831	0.0203\\
1.6838	0.0215\\
1.6845	0.0228\\
1.6852	0.024\\
1.6859	0.0253\\
1.6873	0.0277\\
1.688	0.029\\
1.6894	0.0314\\
1.69	0.0326\\
1.6949	0.041\\
1.6956	0.0421\\
1.697	0.0445\\
1.6977	0.0456\\
1.6983	0.0466\\
1.6988	0.0475\\
1.6994	0.0485\\
1.7	0.0494\\
};
\addplot [color=black, forget plot]
  table[row sep=crcr]{%
0	1.05\\
0.001	1.05\\
0.002	1.0501\\
0.0051	1.0501\\
};
\addplot [color=black, forget plot]
  table[row sep=crcr]{%
0.0051	1.0501\\
0.0083	1.0501\\
0.0147	1.0497\\
0.0211	1.0489\\
0.0257	1.048\\
0.0303	1.047\\
0.0349	1.0458\\
0.0395	1.0443\\
0.0441	1.0427\\
0.0487	1.0408\\
0.0533	1.0387\\
0.0579	1.0365\\
0.0624	1.034\\
0.067	1.0313\\
0.0716	1.0284\\
0.0762	1.0253\\
0.0808	1.022\\
0.0854	1.0185\\
0.09	1.0148\\
0.0946	1.0108\\
0.0992	1.0067\\
0.1038	1.0023\\
0.1084	0.9978\\
0.113	0.993\\
0.1176	0.9881\\
0.1222	0.9829\\
0.1268	0.9775\\
0.1314	0.9719\\
0.136	0.9661\\
0.1406	0.9601\\
0.1452	0.9539\\
0.1497	0.9475\\
0.1543	0.9409\\
0.1589	0.934\\
0.1635	0.927\\
0.1681	0.9198\\
0.1727	0.9123\\
0.1773	0.9046\\
0.1819	0.8968\\
0.1865	0.8887\\
0.1871	0.8876\\
0.1877	0.8866\\
0.1889	0.8844\\
};
\addplot [color=black, forget plot]
  table[row sep=crcr]{%
0.1889	0.8844\\
0.1936	0.8758\\
0.1983	0.867\\
0.2031	0.8579\\
0.2078	0.8486\\
0.2125	0.8391\\
0.2172	0.8294\\
0.2219	0.8195\\
0.2267	0.8093\\
0.2314	0.799\\
0.2361	0.7884\\
0.2408	0.7775\\
0.2456	0.7665\\
0.2503	0.7553\\
0.255	0.7438\\
0.2597	0.7321\\
0.2644	0.7202\\
0.2692	0.7081\\
0.2739	0.6957\\
0.2786	0.6832\\
0.2833	0.6704\\
0.2881	0.6574\\
0.2928	0.6442\\
0.2975	0.6308\\
0.3022	0.6171\\
0.3069	0.6032\\
0.3117	0.5891\\
0.3164	0.5748\\
0.3211	0.5603\\
0.3258	0.5455\\
0.3306	0.5306\\
0.3353	0.5154\\
0.34	0.5\\
0.3447	0.4844\\
0.3494	0.4685\\
0.3542	0.4525\\
0.3589	0.4362\\
0.3636	0.4197\\
0.3683	0.403\\
0.3731	0.386\\
0.3778	0.3689\\
};
\addplot [color=black, forget plot]
  table[row sep=crcr]{%
0.3778	0.3689\\
0.3825	0.3515\\
0.3872	0.3339\\
0.3919	0.3161\\
0.3967	0.2981\\
0.4014	0.2798\\
0.4061	0.2613\\
0.4108	0.2427\\
0.4156	0.2238\\
0.4203	0.2046\\
0.425	0.1853\\
0.4297	0.1657\\
0.4344	0.1459\\
0.4392	0.1259\\
0.4439	0.1057\\
0.4486	0.0853\\
0.4533	0.0646\\
0.4569	0.0487\\
0.4606	0.0326\\
0.4642	0.0164\\
0.4678	-0\\
};
\addplot [color=black, forget plot]
  table[row sep=crcr]{%
0.4678	0\\
0.4678	0.0002\\
0.4679	0.0002\\
0.4679	0.0005\\
0.468	0.0007\\
0.4681	0.001\\
0.4682	0.0012\\
0.4685	0.0025\\
0.4689	0.0037\\
0.4693	0.005\\
0.4696	0.0062\\
0.4715	0.0124\\
0.4733	0.0186\\
0.4752	0.0248\\
0.477	0.0309\\
0.4795	0.0391\\
0.482	0.0472\\
0.4844	0.0552\\
0.4869	0.0632\\
0.4894	0.0711\\
0.4918	0.079\\
0.4943	0.0868\\
0.4993	0.1022\\
0.5017	0.1098\\
0.5042	0.1174\\
0.5067	0.1249\\
0.5091	0.1324\\
0.5116	0.1397\\
0.5141	0.1471\\
0.5166	0.1543\\
0.519	0.1615\\
0.5215	0.1687\\
0.524	0.1757\\
0.5264	0.1828\\
0.5339	0.2035\\
0.5363	0.2103\\
0.5388	0.217\\
0.5413	0.2236\\
0.5437	0.2302\\
0.5462	0.2368\\
0.5487	0.2433\\
0.5512	0.2497\\
0.5536	0.2561\\
0.5561	0.2624\\
0.5587	0.269\\
0.5614	0.2756\\
0.564	0.2822\\
0.5667	0.2886\\
};
\addplot [color=black, forget plot]
  table[row sep=crcr]{%
0.5667	0.2886\\
0.5714	0.3\\
0.5761	0.3112\\
0.5808	0.3221\\
0.5856	0.3329\\
0.5903	0.3434\\
0.595	0.3537\\
0.5997	0.3637\\
0.6044	0.3736\\
0.6092	0.3832\\
0.6139	0.3927\\
0.6186	0.4019\\
0.6233	0.4108\\
0.6281	0.4196\\
0.6328	0.4281\\
0.6375	0.4365\\
0.6422	0.4446\\
0.6469	0.4525\\
0.6517	0.4601\\
0.6564	0.4676\\
0.6611	0.4748\\
0.6658	0.4818\\
0.6706	0.4886\\
0.6753	0.4952\\
0.68	0.5015\\
0.6847	0.5077\\
0.6894	0.5136\\
0.6942	0.5193\\
0.6989	0.5248\\
0.7036	0.53\\
0.7083	0.5351\\
0.7131	0.5399\\
0.7178	0.5445\\
0.7225	0.5489\\
0.7272	0.5531\\
0.7319	0.557\\
0.7367	0.5607\\
0.7414	0.5642\\
0.7461	0.5675\\
0.7508	0.5706\\
0.7556	0.5735\\
};
\addplot [color=black, forget plot]
  table[row sep=crcr]{%
0.7556	0.5735\\
0.7585	0.5752\\
0.7615	0.5768\\
0.7645	0.5783\\
0.7675	0.5797\\
0.7722	0.5818\\
0.7769	0.5836\\
0.7816	0.5853\\
0.7864	0.5867\\
0.7911	0.5879\\
0.7958	0.5889\\
0.8005	0.5897\\
0.8052	0.5902\\
0.81	0.5906\\
0.8124	0.5907\\
0.8148	0.5907\\
};
\addplot [color=black, forget plot]
  table[row sep=crcr]{%
0.8148	0.5907\\
0.8174	0.5907\\
0.818	0.5906\\
0.8212	0.5905\\
0.8276	0.5899\\
0.8308	0.5894\\
0.8341	0.5889\\
0.8405	0.5875\\
0.8438	0.5866\\
0.847	0.5856\\
0.8503	0.5845\\
0.8535	0.5834\\
0.8567	0.5821\\
0.86	0.5807\\
0.8632	0.5792\\
0.8665	0.5776\\
0.8697	0.5759\\
0.8729	0.5741\\
0.8762	0.5722\\
0.8794	0.5702\\
0.8827	0.5681\\
0.8859	0.5659\\
0.8892	0.5636\\
0.8924	0.5612\\
0.8956	0.5587\\
0.8989	0.556\\
0.9021	0.5533\\
0.9054	0.5505\\
0.9086	0.5476\\
0.9118	0.5445\\
0.9151	0.5414\\
0.9183	0.5382\\
0.9216	0.5348\\
0.9248	0.5314\\
0.928	0.5278\\
0.9313	0.5242\\
0.9345	0.5204\\
0.937	0.5175\\
0.9395	0.5145\\
0.942	0.5114\\
0.9444	0.5083\\
};
\addplot [color=black, forget plot]
  table[row sep=crcr]{%
0.9444	0.5083\\
0.9492	0.5022\\
0.9539	0.4958\\
0.9586	0.4893\\
0.9633	0.4825\\
0.9681	0.4755\\
0.9728	0.4683\\
0.9775	0.4609\\
0.9822	0.4533\\
0.9869	0.4454\\
0.9917	0.4373\\
0.9964	0.429\\
1.0011	0.4205\\
1.0058	0.4117\\
1.0106	0.4028\\
1.0153	0.3936\\
1.02	0.3842\\
1.0247	0.3746\\
1.0294	0.3648\\
1.0342	0.3547\\
1.0389	0.3444\\
1.0436	0.334\\
1.0483	0.3232\\
1.0531	0.3123\\
1.0578	0.3012\\
1.0625	0.2898\\
1.0672	0.2782\\
1.0719	0.2664\\
1.0767	0.2544\\
1.0814	0.2422\\
1.0861	0.2297\\
1.0908	0.217\\
1.0956	0.2041\\
1.1003	0.191\\
1.105	0.1777\\
1.1097	0.1641\\
1.1144	0.1504\\
1.1192	0.1364\\
1.1239	0.1222\\
1.1286	0.1077\\
1.1333	0.0931\\
};
\addplot [color=black, forget plot]
  table[row sep=crcr]{%
1.1333	0.0931\\
1.1378	0.079\\
1.1393	0.0742\\
1.144	0.0591\\
1.1488	0.0437\\
1.1535	0.0281\\
1.1582	0.0123\\
1.1591	0.0093\\
1.1609	0.0031\\
1.1619	-0\\
};
\addplot [color=black, forget plot]
  table[row sep=crcr]{%
1.1619	0\\
1.1619	0.0002\\
1.162	0.0005\\
1.1621	0.0007\\
1.1622	0.001\\
1.1623	0.0012\\
1.1628	0.0025\\
1.1633	0.0037\\
1.1638	0.005\\
1.1643	0.0062\\
1.1668	0.0124\\
1.1692	0.0185\\
1.1717	0.0246\\
1.1741	0.0306\\
1.1781	0.0403\\
1.1821	0.0498\\
1.1862	0.0592\\
1.1902	0.0684\\
1.1942	0.0774\\
1.1982	0.0863\\
1.2022	0.095\\
1.2062	0.1036\\
1.2102	0.112\\
1.2142	0.1203\\
1.2182	0.1284\\
1.2222	0.1363\\
1.2263	0.1441\\
1.2303	0.1517\\
1.2343	0.1592\\
1.2383	0.1665\\
1.2423	0.1736\\
1.2463	0.1806\\
1.2503	0.1875\\
1.2583	0.2007\\
1.2623	0.207\\
1.2663	0.2132\\
1.2704	0.2193\\
1.2744	0.2252\\
1.2784	0.2309\\
1.2824	0.2365\\
1.2864	0.2419\\
1.2904	0.2472\\
1.2944	0.2523\\
1.2984	0.2572\\
1.3024	0.262\\
1.3064	0.2666\\
1.3104	0.2711\\
1.3145	0.2754\\
1.3185	0.2796\\
1.3194	0.2805\\
1.3203	0.2815\\
1.3213	0.2824\\
1.3222	0.2833\\
};
\addplot [color=black, forget plot]
  table[row sep=crcr]{%
1.3222	0.2833\\
1.3269	0.2878\\
1.3317	0.2921\\
1.3364	0.2962\\
1.3411	0.3001\\
1.3458	0.3037\\
1.3506	0.3071\\
1.3553	0.3104\\
1.36	0.3133\\
1.3647	0.3161\\
1.3694	0.3187\\
1.3742	0.321\\
1.3789	0.3231\\
1.3836	0.325\\
1.3883	0.3267\\
1.3931	0.3281\\
1.3978	0.3294\\
1.4025	0.3304\\
1.4072	0.3312\\
1.4119	0.3318\\
1.4167	0.3321\\
1.418	0.3322\\
1.4194	0.3322\\
1.4208	0.3323\\
1.4221	0.3323\\
};
\addplot [color=black, forget plot]
  table[row sep=crcr]{%
1.4221	0.3323\\
1.4234	0.3323\\
1.424	0.3322\\
1.4253	0.3322\\
1.4275	0.3321\\
1.4298	0.332\\
1.432	0.3318\\
1.4342	0.3315\\
1.4364	0.3313\\
1.4387	0.3309\\
1.4431	0.3301\\
1.4453	0.3296\\
1.4476	0.3291\\
1.452	0.3279\\
1.4542	0.3272\\
1.4565	0.3265\\
1.4609	0.3249\\
1.4631	0.324\\
1.4654	0.3231\\
1.4698	0.3211\\
1.472	0.32\\
1.4743	0.3189\\
1.4765	0.3178\\
1.4787	0.3166\\
1.4809	0.3153\\
1.4832	0.314\\
1.4898	0.3098\\
1.4921	0.3083\\
1.4987	0.3035\\
1.501	0.3018\\
1.5054	0.2982\\
1.5068	0.2971\\
1.5083	0.2959\\
1.5097	0.2947\\
1.5111	0.2934\\
};
\addplot [color=black, forget plot]
  table[row sep=crcr]{%
1.5111	0.2934\\
1.5156	0.2894\\
1.5201	0.2852\\
1.5245	0.2808\\
1.529	0.2762\\
1.5337	0.2712\\
1.5384	0.2659\\
1.5432	0.2604\\
1.5479	0.2547\\
1.5526	0.2488\\
1.5573	0.2426\\
1.562	0.2362\\
1.5668	0.2296\\
1.5715	0.2228\\
1.5762	0.2158\\
1.5809	0.2086\\
1.5857	0.2011\\
1.5904	0.1934\\
1.5951	0.1855\\
1.5998	0.1774\\
1.6045	0.169\\
1.6093	0.1605\\
1.614	0.1517\\
1.6187	0.1427\\
1.6234	0.1335\\
1.6282	0.124\\
1.6329	0.1144\\
1.6376	0.1045\\
1.6423	0.0944\\
1.647	0.0841\\
1.6518	0.0736\\
1.6565	0.0628\\
1.6612	0.0519\\
1.6659	0.0407\\
1.6707	0.0293\\
1.6754	0.0177\\
1.6801	0.0058\\
1.6807	0.0044\\
1.6812	0.0029\\
1.6818	0.0015\\
1.6824	-0\\
};
\addplot [color=black, forget plot]
  table[row sep=crcr]{%
1.6824	0\\
1.6824	0.0001\\
1.6825	0.0002\\
1.6826	0.0005\\
1.6828	0.0007\\
1.6829	0.001\\
1.683	0.0012\\
1.6835	0.002\\
1.6839	0.0029\\
1.6843	0.0037\\
1.6848	0.0045\\
1.6852	0.0054\\
1.6857	0.0062\\
1.6861	0.007\\
1.6865	0.0079\\
1.687	0.0087\\
1.6874	0.0095\\
1.6879	0.0103\\
1.6883	0.0111\\
1.6887	0.012\\
1.6892	0.0128\\
1.6896	0.0136\\
1.6901	0.0144\\
1.6909	0.016\\
1.6914	0.0168\\
1.6918	0.0176\\
1.6923	0.0184\\
1.6931	0.02\\
1.6936	0.0208\\
1.694	0.0216\\
1.6945	0.0224\\
1.6953	0.024\\
1.6958	0.0248\\
1.6962	0.0256\\
1.6967	0.0263\\
1.6975	0.0279\\
1.698	0.0287\\
1.6984	0.0294\\
1.6989	0.0302\\
1.6992	0.0307\\
1.6994	0.0312\\
1.7	0.0322\\
};
\addplot [color=black, forget plot]
  table[row sep=crcr]{%
0	0.95\\
0.0008	0.95\\
0.001	0.9499\\
0.0023	0.9499\\
0.0036	0.9498\\
0.0049	0.9496\\
0.0061	0.9495\\
0.0109	0.9489\\
0.0156	0.948\\
0.0203	0.947\\
0.025	0.9457\\
0.0298	0.9442\\
0.0345	0.9424\\
0.0392	0.9405\\
0.0439	0.9383\\
0.0486	0.936\\
0.0534	0.9334\\
0.0581	0.9305\\
0.0628	0.9275\\
0.0675	0.9243\\
0.0723	0.9208\\
0.077	0.9171\\
0.0817	0.9132\\
0.0864	0.909\\
0.0911	0.9047\\
0.0959	0.9001\\
0.1006	0.8953\\
0.1053	0.8903\\
0.11	0.8851\\
0.1148	0.8797\\
0.1195	0.874\\
0.1242	0.8681\\
0.1289	0.862\\
0.1336	0.8557\\
0.1384	0.8492\\
0.1431	0.8424\\
0.1478	0.8354\\
0.1525	0.8283\\
0.1573	0.8208\\
0.162	0.8132\\
0.1667	0.8054\\
0.1714	0.7973\\
0.1761	0.789\\
0.1793	0.7833\\
0.1825	0.7775\\
0.1857	0.7716\\
0.1889	0.7656\\
};
\addplot [color=black, forget plot]
  table[row sep=crcr]{%
0.1889	0.7656\\
0.1936	0.7565\\
0.1983	0.7471\\
0.2031	0.7376\\
0.2078	0.7279\\
0.2125	0.7179\\
0.2172	0.7077\\
0.2219	0.6973\\
0.2267	0.6867\\
0.2314	0.6758\\
0.2361	0.6647\\
0.2408	0.6535\\
0.2456	0.642\\
0.2503	0.6302\\
0.255	0.6183\\
0.2597	0.6061\\
0.2644	0.5938\\
0.2692	0.5812\\
0.2739	0.5684\\
0.2786	0.5553\\
0.2833	0.5421\\
0.2881	0.5286\\
0.2928	0.5149\\
0.2975	0.501\\
0.3022	0.4869\\
0.3069	0.4725\\
0.3117	0.458\\
0.3164	0.4432\\
0.3211	0.4282\\
0.3258	0.413\\
0.3306	0.3975\\
0.3353	0.3819\\
0.34	0.366\\
0.3447	0.3499\\
0.3494	0.3336\\
0.3542	0.317\\
0.3589	0.3003\\
0.3636	0.2833\\
0.3683	0.2661\\
0.3731	0.2487\\
0.3778	0.2311\\
};
\addplot [color=black, forget plot]
  table[row sep=crcr]{%
0.3778	0.2311\\
0.384	0.2077\\
0.3871	0.1958\\
0.3901	0.1839\\
0.3949	0.1655\\
0.3996	0.1468\\
0.4043	0.128\\
0.409	0.1089\\
0.4138	0.0896\\
0.4185	0.0701\\
0.4232	0.0504\\
0.4279	0.0304\\
0.4297	0.0229\\
0.4315	0.0153\\
0.4332	0.0077\\
0.435	-0\\
};
\addplot [color=black, forget plot]
  table[row sep=crcr]{%
0.435	0\\
0.435	0.0001\\
0.4351	0.0001\\
0.4351	0.0002\\
0.4352	0.0005\\
0.4352	0.0007\\
0.4353	0.001\\
0.4354	0.0012\\
0.4358	0.0025\\
0.4362	0.0037\\
0.4366	0.005\\
0.4369	0.0062\\
0.4389	0.0124\\
0.4408	0.0186\\
0.4428	0.0248\\
0.4447	0.0309\\
0.448	0.0412\\
0.4513	0.0514\\
0.4579	0.0714\\
0.4612	0.0813\\
0.4645	0.091\\
0.4677	0.1007\\
0.4743	0.1197\\
0.4809	0.1383\\
0.4842	0.1474\\
0.4875	0.1564\\
0.4908	0.1653\\
0.4941	0.1741\\
0.4974	0.1828\\
0.5007	0.1914\\
0.5039	0.1999\\
0.5072	0.2083\\
0.5138	0.2247\\
0.5171	0.2328\\
0.5237	0.2486\\
0.527	0.2563\\
0.5336	0.2715\\
0.5369	0.2789\\
0.5401	0.2862\\
0.5434	0.2934\\
0.5467	0.3005\\
0.55	0.3075\\
0.5533	0.3144\\
0.5566	0.3212\\
0.5599	0.3279\\
0.5632	0.3344\\
0.5641	0.3362\\
0.5649	0.3379\\
0.5667	0.3413\\
};
\addplot [color=black, forget plot]
  table[row sep=crcr]{%
0.5667	0.3413\\
0.5714	0.3504\\
0.5761	0.3592\\
0.5808	0.3679\\
0.5856	0.3763\\
0.5903	0.3845\\
0.595	0.3925\\
0.5997	0.4003\\
0.6044	0.4078\\
0.6092	0.4152\\
0.6139	0.4223\\
0.6186	0.4292\\
0.6233	0.4358\\
0.6281	0.4423\\
0.6328	0.4485\\
0.6375	0.4546\\
0.6422	0.4604\\
0.6469	0.466\\
0.6517	0.4713\\
0.6564	0.4765\\
0.6611	0.4814\\
0.6658	0.4861\\
0.6706	0.4906\\
0.6753	0.4949\\
0.68	0.4989\\
0.6847	0.5027\\
0.6894	0.5064\\
0.6942	0.5098\\
0.6989	0.5129\\
0.7036	0.5159\\
0.7083	0.5186\\
0.7131	0.5212\\
0.7178	0.5235\\
0.7225	0.5255\\
0.7272	0.5274\\
0.7319	0.5291\\
0.7367	0.5305\\
0.7414	0.5317\\
0.7461	0.5327\\
0.7508	0.5334\\
0.7556	0.534\\
};
\addplot [color=black, forget plot]
  table[row sep=crcr]{%
0.7556	0.534\\
0.756	0.534\\
0.7565	0.5341\\
0.757	0.5341\\
0.7575	0.5342\\
0.7613	0.5344\\
0.7651	0.5344\\
};
\addplot [color=black, forget plot]
  table[row sep=crcr]{%
0.7651	0.5344\\
0.7683	0.5344\\
0.7747	0.534\\
0.7811	0.5332\\
0.7856	0.5324\\
0.7901	0.5314\\
0.7946	0.5302\\
0.799	0.5288\\
0.8035	0.5272\\
0.808	0.5254\\
0.8125	0.5234\\
0.817	0.5213\\
0.8215	0.5189\\
0.8259	0.5163\\
0.8304	0.5135\\
0.8349	0.5106\\
0.8394	0.5074\\
0.8439	0.504\\
0.8484	0.5005\\
0.8528	0.4967\\
0.8573	0.4927\\
0.8618	0.4886\\
0.8663	0.4842\\
0.8708	0.4797\\
0.8753	0.4749\\
0.8797	0.47\\
0.8842	0.4649\\
0.8887	0.4595\\
0.8932	0.454\\
0.8977	0.4483\\
0.9022	0.4423\\
0.9066	0.4362\\
0.9111	0.4299\\
0.9156	0.4234\\
0.9201	0.4166\\
0.9246	0.4097\\
0.9291	0.4026\\
0.9335	0.3953\\
0.938	0.3878\\
0.9425	0.3801\\
0.943	0.3792\\
0.944	0.3776\\
0.9444	0.3767\\
};
\addplot [color=black, forget plot]
  table[row sep=crcr]{%
0.9444	0.3767\\
0.9492	0.3683\\
0.9539	0.3597\\
0.9586	0.3508\\
0.9633	0.3417\\
0.9681	0.3324\\
0.9728	0.3229\\
0.9775	0.3132\\
0.9822	0.3032\\
0.9869	0.2931\\
0.9917	0.2827\\
0.9964	0.2721\\
1.0011	0.2613\\
1.0058	0.2502\\
1.0106	0.239\\
1.0153	0.2275\\
1.02	0.2158\\
1.0247	0.2039\\
1.0294	0.1917\\
1.0342	0.1794\\
1.0389	0.1668\\
1.0436	0.154\\
1.0483	0.141\\
1.0531	0.1278\\
1.0578	0.1143\\
1.0625	0.1007\\
1.0672	0.0868\\
1.0767	0.0583\\
1.0813	0.0441\\
1.0859	0.0296\\
1.0906	0.0149\\
1.0952	-0\\
};
\addplot [color=black, forget plot]
  table[row sep=crcr]{%
1.0952	0\\
1.0952	0.0001\\
1.0953	0.0002\\
1.0954	0.0005\\
1.0955	0.0007\\
1.0956	0.001\\
1.0957	0.0012\\
1.0962	0.0025\\
1.0967	0.0037\\
1.0973	0.005\\
1.0978	0.0062\\
1.0987	0.0085\\
1.0997	0.0108\\
1.1006	0.013\\
1.1016	0.0153\\
1.1025	0.0175\\
1.1035	0.0198\\
1.1044	0.022\\
1.1054	0.0242\\
1.1063	0.0265\\
1.1083	0.0309\\
1.1092	0.033\\
1.1102	0.0352\\
1.1111	0.0374\\
1.1121	0.0396\\
1.113	0.0417\\
1.114	0.0439\\
1.1149	0.046\\
1.1159	0.0481\\
1.1168	0.0502\\
1.1178	0.0523\\
1.1187	0.0544\\
1.1197	0.0565\\
1.1206	0.0586\\
1.1226	0.0628\\
1.1235	0.0648\\
1.1245	0.0669\\
1.1254	0.0689\\
1.1264	0.0709\\
1.1273	0.0729\\
1.1283	0.0749\\
1.1292	0.0769\\
1.1302	0.0789\\
1.1311	0.0809\\
1.1324	0.0835\\
1.1327	0.0842\\
1.133	0.0848\\
1.1333	0.0855\\
};
\addplot [color=black, forget plot]
  table[row sep=crcr]{%
1.1333	0.0855\\
1.1354	0.0897\\
1.1375	0.094\\
1.1396	0.0982\\
1.1417	0.1023\\
1.1464	0.1115\\
1.1511	0.1205\\
1.1559	0.1293\\
1.1606	0.1378\\
1.1653	0.1461\\
1.17	0.1543\\
1.1747	0.1621\\
1.1795	0.1698\\
1.1842	0.1773\\
1.1889	0.1845\\
1.1936	0.1915\\
1.1984	0.1983\\
1.2031	0.2049\\
1.2078	0.2113\\
1.2125	0.2174\\
1.2172	0.2233\\
1.222	0.229\\
1.2267	0.2345\\
1.2314	0.2398\\
1.2361	0.2449\\
1.2409	0.2497\\
1.2456	0.2543\\
1.2503	0.2587\\
1.255	0.2629\\
1.2597	0.2668\\
1.2645	0.2706\\
1.2692	0.2741\\
1.2739	0.2774\\
1.2786	0.2805\\
1.2834	0.2833\\
1.2881	0.286\\
1.2928	0.2884\\
1.2975	0.2906\\
1.3022	0.2926\\
1.307	0.2943\\
1.3117	0.2959\\
1.3143	0.2967\\
1.317	0.2974\\
1.3222	0.2986\\
};
\addplot [color=black, forget plot]
  table[row sep=crcr]{%
1.3222	0.2986\\
1.3233	0.2988\\
1.3243	0.299\\
1.3253	0.2991\\
1.3264	0.2993\\
1.3305	0.2999\\
1.3346	0.3003\\
1.3387	0.3005\\
1.3428	0.3006\\
};
\addplot [color=black, forget plot]
  table[row sep=crcr]{%
1.3428	0.3006\\
1.346	0.3006\\
1.3524	0.3002\\
1.3588	0.2994\\
1.363	0.2986\\
1.3672	0.2977\\
1.3714	0.2966\\
1.3756	0.2953\\
1.3798	0.2939\\
1.384	0.2923\\
1.3882	0.2905\\
1.3924	0.2885\\
1.3966	0.2864\\
1.4009	0.2841\\
1.4051	0.2816\\
1.4093	0.2789\\
1.4135	0.2761\\
1.4177	0.2731\\
1.4219	0.2699\\
1.4261	0.2666\\
1.4303	0.263\\
1.4345	0.2593\\
1.4387	0.2555\\
1.4429	0.2514\\
1.4471	0.2472\\
1.4514	0.2428\\
1.4556	0.2382\\
1.4598	0.2335\\
1.464	0.2286\\
1.4682	0.2235\\
1.4724	0.2182\\
1.4766	0.2128\\
1.4808	0.2072\\
1.485	0.2014\\
1.4892	0.1954\\
1.4934	0.1893\\
1.4979	0.1827\\
1.5023	0.1758\\
1.5067	0.1688\\
1.5111	0.1616\\
};
\addplot [color=black, forget plot]
  table[row sep=crcr]{%
1.5111	0.1616\\
1.5158	0.1537\\
1.5206	0.1456\\
1.5253	0.1372\\
1.53	0.1287\\
1.5347	0.1199\\
1.5394	0.1109\\
1.5442	0.1017\\
1.5489	0.0922\\
1.5536	0.0826\\
1.5583	0.0727\\
1.5631	0.0626\\
1.5678	0.0523\\
1.5725	0.0418\\
1.5772	0.031\\
1.5819	0.02\\
1.5867	0.0089\\
1.5876	0.0067\\
1.5885	0.0044\\
1.5903	-0\\
};
\addplot [color=black, forget plot]
  table[row sep=crcr]{%
1.5903	0\\
1.5904	0.0001\\
1.5904	0.0002\\
1.5906	0.0005\\
1.5907	0.0007\\
1.5909	0.001\\
1.591	0.0012\\
1.5917	0.0025\\
1.5931	0.0049\\
1.5938	0.0062\\
1.5992	0.0158\\
1.602	0.0205\\
1.6047	0.0252\\
1.6075	0.0298\\
1.6102	0.0343\\
1.6129	0.0387\\
1.6157	0.043\\
1.6184	0.0473\\
1.6212	0.0515\\
1.6239	0.0556\\
1.6267	0.0597\\
1.6294	0.0637\\
1.6321	0.0676\\
1.6349	0.0714\\
1.6376	0.0752\\
1.6404	0.0788\\
1.6431	0.0825\\
1.6458	0.086\\
1.6486	0.0895\\
1.6513	0.0928\\
1.6541	0.0962\\
1.6568	0.0994\\
1.6596	0.1026\\
1.6623	0.1057\\
1.665	0.1087\\
1.6678	0.1116\\
1.6705	0.1145\\
1.6787	0.1227\\
1.6815	0.1253\\
1.6842	0.1278\\
1.687	0.1302\\
1.6897	0.1326\\
1.6925	0.1349\\
1.6943	0.1364\\
1.6962	0.1379\\
1.6981	0.1393\\
1.7	0.1408\\
};
\addplot [color=black, forget plot]
  table[row sep=crcr]{%
0	1.05\\
0.0008	1.05\\
0.001	1.0499\\
0.0023	1.0499\\
0.0036	1.0498\\
0.0049	1.0496\\
0.0061	1.0495\\
0.0109	1.0489\\
0.0156	1.048\\
0.0203	1.047\\
0.025	1.0457\\
0.0298	1.0442\\
0.0345	1.0424\\
0.0392	1.0405\\
0.0439	1.0383\\
0.0486	1.036\\
0.0534	1.0334\\
0.0581	1.0305\\
0.0628	1.0275\\
0.0675	1.0243\\
0.0723	1.0208\\
0.077	1.0171\\
0.0817	1.0132\\
0.0864	1.009\\
0.0911	1.0047\\
0.0959	1.0001\\
0.1006	0.9953\\
0.1053	0.9903\\
0.11	0.9851\\
0.1148	0.9797\\
0.1195	0.974\\
0.1242	0.9681\\
0.1289	0.962\\
0.1336	0.9557\\
0.1384	0.9492\\
0.1431	0.9424\\
0.1478	0.9354\\
0.1525	0.9283\\
0.1573	0.9208\\
0.162	0.9132\\
0.1667	0.9054\\
0.1714	0.8973\\
0.1761	0.889\\
0.1793	0.8833\\
0.1825	0.8775\\
0.1857	0.8716\\
0.1889	0.8656\\
};
\addplot [color=black, forget plot]
  table[row sep=crcr]{%
0.1889	0.8656\\
0.1936	0.8565\\
0.1983	0.8471\\
0.2031	0.8376\\
0.2078	0.8279\\
0.2125	0.8179\\
0.2172	0.8077\\
0.2219	0.7973\\
0.2267	0.7867\\
0.2314	0.7758\\
0.2361	0.7647\\
0.2408	0.7535\\
0.2456	0.742\\
0.2503	0.7302\\
0.255	0.7183\\
0.2597	0.7061\\
0.2644	0.6938\\
0.2692	0.6812\\
0.2739	0.6684\\
0.2786	0.6553\\
0.2833	0.6421\\
0.2881	0.6286\\
0.2928	0.6149\\
0.2975	0.601\\
0.3022	0.5869\\
0.3069	0.5725\\
0.3117	0.558\\
0.3164	0.5432\\
0.3211	0.5282\\
0.3258	0.513\\
0.3306	0.4975\\
0.3353	0.4819\\
0.34	0.466\\
0.3447	0.4499\\
0.3494	0.4336\\
0.3542	0.417\\
0.3589	0.4003\\
0.3636	0.3833\\
0.3683	0.3661\\
0.3731	0.3487\\
0.3778	0.3311\\
};
\addplot [color=black, forget plot]
  table[row sep=crcr]{%
0.3778	0.3311\\
0.3822	0.3144\\
0.3866	0.2974\\
0.3911	0.2803\\
0.3955	0.263\\
0.4002	0.2444\\
0.4049	0.2255\\
0.4097	0.2064\\
0.4144	0.187\\
0.4191	0.1675\\
0.4238	0.1477\\
0.4285	0.1278\\
0.4333	0.1076\\
0.438	0.0871\\
0.4427	0.0665\\
0.4474	0.0457\\
0.4522	0.0246\\
0.4535	0.0185\\
0.4549	0.0123\\
0.4562	0.0062\\
0.4576	-0\\
};
\addplot [color=black, forget plot]
  table[row sep=crcr]{%
0.4576	0\\
0.4576	0.0002\\
0.4577	0.0002\\
0.4577	0.0005\\
0.4578	0.0007\\
0.4579	0.001\\
0.458	0.0012\\
0.4583	0.0025\\
0.4587	0.0037\\
0.4591	0.005\\
0.4594	0.0062\\
0.4613	0.0124\\
0.4631	0.0186\\
0.465	0.0248\\
0.4668	0.0309\\
0.4695	0.0399\\
0.4723	0.0489\\
0.4777	0.0665\\
0.4804	0.0752\\
0.4832	0.0838\\
0.4859	0.0924\\
0.4886	0.1009\\
0.4914	0.1093\\
0.4968	0.1259\\
0.4995	0.1341\\
0.5023	0.1422\\
0.505	0.1503\\
0.5077	0.1583\\
0.5104	0.1662\\
0.5132	0.174\\
0.5159	0.1818\\
0.5186	0.1894\\
0.5213	0.1971\\
0.5241	0.2046\\
0.5268	0.2121\\
0.5295	0.2195\\
0.5323	0.2268\\
0.5377	0.2412\\
0.5404	0.2483\\
0.5432	0.2553\\
0.5459	0.2623\\
0.5486	0.2692\\
0.5513	0.276\\
0.5541	0.2827\\
0.5568	0.2894\\
0.5595	0.296\\
0.5622	0.3025\\
0.565	0.309\\
0.5658	0.311\\
0.5662	0.3119\\
0.5667	0.3129\\
};
\addplot [color=black, forget plot]
  table[row sep=crcr]{%
0.5667	0.3129\\
0.5714	0.3239\\
0.5761	0.3345\\
0.5808	0.345\\
0.5856	0.3553\\
0.5903	0.3653\\
0.595	0.3751\\
0.5997	0.3847\\
0.6044	0.3941\\
0.6092	0.4033\\
0.6139	0.4122\\
0.6186	0.421\\
0.6233	0.4295\\
0.6281	0.4378\\
0.6328	0.4458\\
0.6375	0.4537\\
0.6422	0.4613\\
0.6469	0.4687\\
0.6517	0.4759\\
0.6564	0.4829\\
0.6611	0.4897\\
0.6658	0.4962\\
0.6706	0.5025\\
0.6753	0.5086\\
0.68	0.5145\\
0.6847	0.5202\\
0.6894	0.5256\\
0.6942	0.5308\\
0.6989	0.5359\\
0.7036	0.5406\\
0.7083	0.5452\\
0.7131	0.5496\\
0.7178	0.5537\\
0.7225	0.5576\\
0.7272	0.5613\\
0.7319	0.5648\\
0.7367	0.568\\
0.7414	0.5711\\
0.7461	0.5739\\
0.7508	0.5765\\
0.7556	0.5789\\
};
\addplot [color=black, forget plot]
  table[row sep=crcr]{%
0.7556	0.5789\\
0.758	0.58\\
0.763	0.5822\\
0.7654	0.5832\\
0.7701	0.5849\\
0.7749	0.5863\\
0.7796	0.5876\\
0.7843	0.5887\\
0.789	0.5895\\
0.7938	0.5901\\
0.7985	0.5905\\
0.8032	0.5907\\
0.8046	0.5907\\
};
\addplot [color=black, forget plot]
  table[row sep=crcr]{%
0.8046	0.5907\\
0.8072	0.5907\\
0.8078	0.5906\\
0.811	0.5905\\
0.8174	0.5899\\
0.8206	0.5894\\
0.8241	0.5888\\
0.8276	0.5881\\
0.8311	0.5873\\
0.8346	0.5863\\
0.8381	0.5852\\
0.8416	0.584\\
0.8451	0.5827\\
0.8521	0.5797\\
0.8556	0.578\\
0.8591	0.5762\\
0.8661	0.5722\\
0.8696	0.57\\
0.8731	0.5677\\
0.8766	0.5653\\
0.8801	0.5628\\
0.8835	0.5602\\
0.887	0.5574\\
0.8905	0.5545\\
0.894	0.5515\\
0.8975	0.5484\\
0.9045	0.5418\\
0.908	0.5383\\
0.9115	0.5347\\
0.9185	0.5271\\
0.9255	0.5191\\
0.9325	0.5105\\
0.9355	0.5067\\
0.9415	0.4989\\
0.9444	0.4948\\
};
\addplot [color=black, forget plot]
  table[row sep=crcr]{%
0.9444	0.4948\\
0.9492	0.4882\\
0.9539	0.4814\\
0.9586	0.4744\\
0.9633	0.4672\\
0.9681	0.4597\\
0.9728	0.452\\
0.9775	0.4441\\
0.9822	0.436\\
0.9869	0.4277\\
0.9917	0.4191\\
0.9964	0.4103\\
1.0011	0.4013\\
1.0058	0.3921\\
1.0106	0.3827\\
1.0153	0.3731\\
1.02	0.3632\\
1.0247	0.3531\\
1.0294	0.3428\\
1.0342	0.3323\\
1.0389	0.3215\\
1.0436	0.3106\\
1.0483	0.2994\\
1.0531	0.288\\
1.0578	0.2764\\
1.0625	0.2645\\
1.0672	0.2525\\
1.0719	0.2402\\
1.0767	0.2277\\
1.0814	0.215\\
1.0861	0.2021\\
1.0908	0.1889\\
1.0956	0.1756\\
1.1003	0.162\\
1.105	0.1482\\
1.1097	0.1341\\
1.1144	0.1199\\
1.1192	0.1054\\
1.1239	0.0908\\
1.1286	0.0759\\
1.1333	0.0607\\
};
\addplot [color=black, forget plot]
  table[row sep=crcr]{%
1.1333	0.0607\\
1.1343	0.0577\\
1.1352	0.0546\\
1.1362	0.0515\\
1.1371	0.0485\\
1.1408	0.0365\\
1.1444	0.0245\\
1.148	0.0123\\
1.1517	0\\
};
\addplot [color=black, forget plot]
  table[row sep=crcr]{%
1.1517	0\\
1.1517	0.0002\\
1.1518	0.0005\\
1.1519	0.0007\\
1.152	0.001\\
1.1521	0.0012\\
1.1526	0.0025\\
1.1531	0.0037\\
1.1536	0.005\\
1.1541	0.0062\\
1.1566	0.0124\\
1.159	0.0185\\
1.1615	0.0246\\
1.1639	0.0306\\
1.1682	0.0409\\
1.1725	0.051\\
1.1767	0.0609\\
1.181	0.0707\\
1.1853	0.0802\\
1.1895	0.0896\\
1.1938	0.0989\\
1.198	0.1079\\
1.2023	0.1167\\
1.2066	0.1254\\
1.2108	0.1339\\
1.2151	0.1423\\
1.2194	0.1504\\
1.2236	0.1584\\
1.2322	0.1738\\
1.2364	0.1812\\
1.2407	0.1884\\
1.245	0.1955\\
1.2492	0.2024\\
1.2535	0.2091\\
1.2577	0.2157\\
1.262	0.222\\
1.2663	0.2282\\
1.2705	0.2342\\
1.2748	0.24\\
1.2791	0.2457\\
1.2833	0.2512\\
1.2919	0.2616\\
1.2961	0.2665\\
1.3004	0.2712\\
1.3047	0.2758\\
1.3089	0.2802\\
1.3132	0.2844\\
1.3174	0.2885\\
1.3198	0.2907\\
1.321	0.2917\\
1.3222	0.2928\\
};
\addplot [color=black, forget plot]
  table[row sep=crcr]{%
1.3222	0.2928\\
1.3267	0.2967\\
1.3312	0.3003\\
1.3357	0.3038\\
1.3402	0.3071\\
1.345	0.3103\\
1.3497	0.3133\\
1.3544	0.316\\
1.3591	0.3186\\
1.3639	0.3209\\
1.3686	0.3231\\
1.3733	0.3249\\
1.378	0.3266\\
1.3827	0.3281\\
1.3875	0.3293\\
1.3922	0.3304\\
1.3969	0.3312\\
1.4007	0.3316\\
1.4044	0.332\\
1.4082	0.3322\\
1.4119	0.3323\\
};
\addplot [color=black, forget plot]
  table[row sep=crcr]{%
1.4119	0.3323\\
1.4132	0.3323\\
1.4138	0.3322\\
1.4151	0.3322\\
1.4176	0.3321\\
1.4226	0.3317\\
1.425	0.3314\\
1.4275	0.3311\\
1.43	0.3307\\
1.435	0.3297\\
1.4374	0.3291\\
1.4424	0.3277\\
1.4474	0.3261\\
1.4498	0.3252\\
1.4523	0.3243\\
1.4548	0.3233\\
1.4573	0.3222\\
1.4598	0.321\\
1.4622	0.3199\\
1.4672	0.3173\\
1.4722	0.3145\\
1.4746	0.313\\
1.4796	0.3098\\
1.4846	0.3064\\
1.487	0.3046\\
1.492	0.3008\\
1.4945	0.2988\\
1.4969	0.2968\\
1.5019	0.2926\\
1.5044	0.2903\\
1.5061	0.2888\\
1.5077	0.2872\\
1.5111	0.284\\
};
\addplot [color=black, forget plot]
  table[row sep=crcr]{%
1.5111	0.284\\
1.5158	0.2793\\
1.5206	0.2744\\
1.5253	0.2692\\
1.53	0.2639\\
1.5347	0.2583\\
1.5394	0.2525\\
1.5442	0.2465\\
1.5489	0.2403\\
1.5536	0.2338\\
1.5583	0.2271\\
1.5631	0.2202\\
1.5678	0.2131\\
1.5725	0.2058\\
1.5772	0.1983\\
1.5819	0.1905\\
1.5867	0.1825\\
1.5914	0.1743\\
1.5961	0.1659\\
1.6008	0.1572\\
1.6056	0.1484\\
1.6103	0.1393\\
1.615	0.13\\
1.6197	0.1205\\
1.6244	0.1107\\
1.6292	0.1008\\
1.6339	0.0906\\
1.6386	0.0802\\
1.6433	0.0696\\
1.6481	0.0588\\
1.6528	0.0477\\
1.6575	0.0365\\
1.6622	0.025\\
1.6672	0.0126\\
1.6722	-0\\
};
\addplot [color=black, forget plot]
  table[row sep=crcr]{%
1.6722	0\\
1.6722	0.0001\\
1.6723	0.0001\\
1.6723	0.0002\\
1.6724	0.0005\\
1.6726	0.0007\\
1.6727	0.001\\
1.6728	0.0012\\
1.6735	0.0025\\
1.6741	0.0037\\
1.6748	0.0049\\
1.6755	0.0062\\
1.6761	0.0075\\
1.6789	0.0127\\
1.6796	0.0139\\
1.6817	0.0178\\
1.6824	0.019\\
1.6831	0.0203\\
1.6838	0.0215\\
1.6845	0.0228\\
1.6852	0.024\\
1.6859	0.0253\\
1.6873	0.0277\\
1.688	0.029\\
1.6894	0.0314\\
1.69	0.0326\\
1.6949	0.041\\
1.6956	0.0421\\
1.697	0.0445\\
1.6977	0.0456\\
1.6983	0.0466\\
1.6988	0.0475\\
1.6994	0.0485\\
1.7	0.0494\\
};
\addplot [color=black, forget plot]
  table[row sep=crcr]{%
0	1.05\\
0.001	1.05\\
0.002	1.0501\\
0.0051	1.0501\\
};
\addplot [color=black, forget plot]
  table[row sep=crcr]{%
0.0051	1.0501\\
0.0083	1.0501\\
0.0147	1.0497\\
0.0211	1.0489\\
0.0257	1.048\\
0.0303	1.047\\
0.0349	1.0458\\
0.0395	1.0443\\
0.0441	1.0427\\
0.0487	1.0408\\
0.0533	1.0387\\
0.0579	1.0365\\
0.0624	1.034\\
0.067	1.0313\\
0.0716	1.0284\\
0.0762	1.0253\\
0.0808	1.022\\
0.0854	1.0185\\
0.09	1.0148\\
0.0946	1.0108\\
0.0992	1.0067\\
0.1038	1.0023\\
0.1084	0.9978\\
0.113	0.993\\
0.1176	0.9881\\
0.1222	0.9829\\
0.1268	0.9775\\
0.1314	0.9719\\
0.136	0.9661\\
0.1406	0.9601\\
0.1452	0.9539\\
0.1497	0.9475\\
0.1543	0.9409\\
0.1589	0.934\\
0.1635	0.927\\
0.1681	0.9198\\
0.1727	0.9123\\
0.1773	0.9046\\
0.1819	0.8968\\
0.1865	0.8887\\
0.1871	0.8876\\
0.1877	0.8866\\
0.1889	0.8844\\
};
\addplot [color=black, forget plot]
  table[row sep=crcr]{%
0.1889	0.8844\\
0.1936	0.8758\\
0.1983	0.867\\
0.2031	0.8579\\
0.2078	0.8486\\
0.2125	0.8391\\
0.2172	0.8294\\
0.2219	0.8195\\
0.2267	0.8093\\
0.2314	0.799\\
0.2361	0.7884\\
0.2408	0.7775\\
0.2456	0.7665\\
0.2503	0.7553\\
0.255	0.7438\\
0.2597	0.7321\\
0.2644	0.7202\\
0.2692	0.7081\\
0.2739	0.6957\\
0.2786	0.6832\\
0.2833	0.6704\\
0.2881	0.6574\\
0.2928	0.6442\\
0.2975	0.6308\\
0.3022	0.6171\\
0.3069	0.6032\\
0.3117	0.5891\\
0.3164	0.5748\\
0.3211	0.5603\\
0.3258	0.5455\\
0.3306	0.5306\\
0.3353	0.5154\\
0.34	0.5\\
0.3447	0.4844\\
0.3494	0.4685\\
0.3542	0.4525\\
0.3589	0.4362\\
0.3636	0.4197\\
0.3683	0.403\\
0.3731	0.386\\
0.3778	0.3689\\
};
\addplot [color=black, forget plot]
  table[row sep=crcr]{%
0.3778	0.3689\\
0.3825	0.3515\\
0.3872	0.3339\\
0.3919	0.3161\\
0.3967	0.2981\\
0.4014	0.2798\\
0.4061	0.2613\\
0.4108	0.2427\\
0.4156	0.2238\\
0.4203	0.2046\\
0.425	0.1853\\
0.4297	0.1657\\
0.4344	0.1459\\
0.4392	0.1259\\
0.4439	0.1057\\
0.4486	0.0853\\
0.4533	0.0646\\
0.4569	0.0487\\
0.4606	0.0326\\
0.4642	0.0164\\
0.4678	-0\\
};
\addplot [color=black, forget plot]
  table[row sep=crcr]{%
0.4678	0\\
0.4678	0.0002\\
0.4679	0.0002\\
0.4679	0.0005\\
0.468	0.0007\\
0.4681	0.001\\
0.4682	0.0012\\
0.4685	0.0025\\
0.4689	0.0037\\
0.4693	0.005\\
0.4696	0.0062\\
0.4715	0.0124\\
0.4733	0.0186\\
0.4752	0.0248\\
0.477	0.0309\\
0.4795	0.0391\\
0.482	0.0472\\
0.4844	0.0552\\
0.4869	0.0632\\
0.4894	0.0711\\
0.4918	0.079\\
0.4943	0.0868\\
0.4993	0.1022\\
0.5017	0.1098\\
0.5042	0.1174\\
0.5067	0.1249\\
0.5091	0.1324\\
0.5116	0.1397\\
0.5141	0.1471\\
0.5166	0.1543\\
0.519	0.1615\\
0.5215	0.1687\\
0.524	0.1757\\
0.5264	0.1828\\
0.5339	0.2035\\
0.5363	0.2103\\
0.5388	0.217\\
0.5413	0.2236\\
0.5437	0.2302\\
0.5462	0.2368\\
0.5487	0.2433\\
0.5512	0.2497\\
0.5536	0.2561\\
0.5561	0.2624\\
0.5587	0.269\\
0.5614	0.2756\\
0.564	0.2822\\
0.5667	0.2886\\
};
\addplot [color=black, forget plot]
  table[row sep=crcr]{%
0.5667	0.2886\\
0.5714	0.3\\
0.5761	0.3112\\
0.5808	0.3221\\
0.5856	0.3329\\
0.5903	0.3434\\
0.595	0.3537\\
0.5997	0.3637\\
0.6044	0.3736\\
0.6092	0.3832\\
0.6139	0.3927\\
0.6186	0.4019\\
0.6233	0.4108\\
0.6281	0.4196\\
0.6328	0.4281\\
0.6375	0.4365\\
0.6422	0.4446\\
0.6469	0.4525\\
0.6517	0.4601\\
0.6564	0.4676\\
0.6611	0.4748\\
0.6658	0.4818\\
0.6706	0.4886\\
0.6753	0.4952\\
0.68	0.5015\\
0.6847	0.5077\\
0.6894	0.5136\\
0.6942	0.5193\\
0.6989	0.5248\\
0.7036	0.53\\
0.7083	0.5351\\
0.7131	0.5399\\
0.7178	0.5445\\
0.7225	0.5489\\
0.7272	0.5531\\
0.7319	0.557\\
0.7367	0.5607\\
0.7414	0.5642\\
0.7461	0.5675\\
0.7508	0.5706\\
0.7556	0.5735\\
};
\addplot [color=black, forget plot]
  table[row sep=crcr]{%
0.7556	0.5735\\
0.7585	0.5752\\
0.7615	0.5768\\
0.7645	0.5783\\
0.7675	0.5797\\
0.7722	0.5818\\
0.7769	0.5836\\
0.7816	0.5853\\
0.7864	0.5867\\
0.7911	0.5879\\
0.7958	0.5889\\
0.8005	0.5897\\
0.8052	0.5902\\
0.81	0.5906\\
0.8124	0.5907\\
0.8148	0.5907\\
};
\addplot [color=black, forget plot]
  table[row sep=crcr]{%
0.8148	0.5907\\
0.8174	0.5907\\
0.818	0.5906\\
0.8212	0.5905\\
0.8276	0.5899\\
0.8308	0.5894\\
0.8341	0.5889\\
0.8405	0.5875\\
0.8438	0.5866\\
0.847	0.5856\\
0.8503	0.5845\\
0.8535	0.5834\\
0.8567	0.5821\\
0.86	0.5807\\
0.8632	0.5792\\
0.8665	0.5776\\
0.8697	0.5759\\
0.8729	0.5741\\
0.8762	0.5722\\
0.8794	0.5702\\
0.8827	0.5681\\
0.8859	0.5659\\
0.8892	0.5636\\
0.8924	0.5612\\
0.8956	0.5587\\
0.8989	0.556\\
0.9021	0.5533\\
0.9054	0.5505\\
0.9086	0.5476\\
0.9118	0.5445\\
0.9151	0.5414\\
0.9183	0.5382\\
0.9216	0.5348\\
0.9248	0.5314\\
0.928	0.5278\\
0.9313	0.5242\\
0.9345	0.5204\\
0.937	0.5175\\
0.9395	0.5145\\
0.942	0.5114\\
0.9444	0.5083\\
};
\addplot [color=black, forget plot]
  table[row sep=crcr]{%
0.9444	0.5083\\
0.9492	0.5022\\
0.9539	0.4958\\
0.9586	0.4893\\
0.9633	0.4825\\
0.9681	0.4755\\
0.9728	0.4683\\
0.9775	0.4609\\
0.9822	0.4533\\
0.9869	0.4454\\
0.9917	0.4373\\
0.9964	0.429\\
1.0011	0.4205\\
1.0058	0.4117\\
1.0106	0.4028\\
1.0153	0.3936\\
1.02	0.3842\\
1.0247	0.3746\\
1.0294	0.3648\\
1.0342	0.3547\\
1.0389	0.3444\\
1.0436	0.334\\
1.0483	0.3232\\
1.0531	0.3123\\
1.0578	0.3012\\
1.0625	0.2898\\
1.0672	0.2782\\
1.0719	0.2664\\
1.0767	0.2544\\
1.0814	0.2422\\
1.0861	0.2297\\
1.0908	0.217\\
1.0956	0.2041\\
1.1003	0.191\\
1.105	0.1777\\
1.1097	0.1641\\
1.1144	0.1504\\
1.1192	0.1364\\
1.1239	0.1222\\
1.1286	0.1077\\
1.1333	0.0931\\
};
\addplot [color=black, forget plot]
  table[row sep=crcr]{%
1.1333	0.0931\\
1.1378	0.079\\
1.1393	0.0742\\
1.144	0.0591\\
1.1488	0.0437\\
1.1535	0.0281\\
1.1582	0.0123\\
1.1591	0.0093\\
1.1609	0.0031\\
1.1619	-0\\
};
\addplot [color=black, forget plot]
  table[row sep=crcr]{%
1.1619	0\\
1.1619	0.0002\\
1.162	0.0005\\
1.1621	0.0007\\
1.1622	0.001\\
1.1623	0.0012\\
1.1628	0.0025\\
1.1633	0.0037\\
1.1638	0.005\\
1.1643	0.0062\\
1.1668	0.0124\\
1.1692	0.0185\\
1.1717	0.0246\\
1.1741	0.0306\\
1.1781	0.0403\\
1.1821	0.0498\\
1.1862	0.0592\\
1.1902	0.0684\\
1.1942	0.0774\\
1.1982	0.0863\\
1.2022	0.095\\
1.2062	0.1036\\
1.2102	0.112\\
1.2142	0.1203\\
1.2182	0.1284\\
1.2222	0.1363\\
1.2263	0.1441\\
1.2303	0.1517\\
1.2343	0.1592\\
1.2383	0.1665\\
1.2423	0.1736\\
1.2463	0.1806\\
1.2503	0.1875\\
1.2583	0.2007\\
1.2623	0.207\\
1.2663	0.2132\\
1.2704	0.2193\\
1.2744	0.2252\\
1.2784	0.2309\\
1.2824	0.2365\\
1.2864	0.2419\\
1.2904	0.2472\\
1.2944	0.2523\\
1.2984	0.2572\\
1.3024	0.262\\
1.3064	0.2666\\
1.3104	0.2711\\
1.3145	0.2754\\
1.3185	0.2796\\
1.3194	0.2805\\
1.3203	0.2815\\
1.3213	0.2824\\
1.3222	0.2833\\
};
\addplot [color=black, forget plot]
  table[row sep=crcr]{%
1.3222	0.2833\\
1.3269	0.2878\\
1.3317	0.2921\\
1.3364	0.2962\\
1.3411	0.3001\\
1.3458	0.3037\\
1.3506	0.3071\\
1.3553	0.3104\\
1.36	0.3133\\
1.3647	0.3161\\
1.3694	0.3187\\
1.3742	0.321\\
1.3789	0.3231\\
1.3836	0.325\\
1.3883	0.3267\\
1.3931	0.3281\\
1.3978	0.3294\\
1.4025	0.3304\\
1.4072	0.3312\\
1.4119	0.3318\\
1.4167	0.3321\\
1.418	0.3322\\
1.4194	0.3322\\
1.4208	0.3323\\
1.4221	0.3323\\
};
\addplot [color=black, forget plot]
  table[row sep=crcr]{%
1.4221	0.3323\\
1.4234	0.3323\\
1.424	0.3322\\
1.4253	0.3322\\
1.4275	0.3321\\
1.4298	0.332\\
1.432	0.3318\\
1.4342	0.3315\\
1.4364	0.3313\\
1.4387	0.3309\\
1.4431	0.3301\\
1.4453	0.3296\\
1.4476	0.3291\\
1.452	0.3279\\
1.4542	0.3272\\
1.4565	0.3265\\
1.4609	0.3249\\
1.4631	0.324\\
1.4654	0.3231\\
1.4698	0.3211\\
1.472	0.32\\
1.4743	0.3189\\
1.4765	0.3178\\
1.4787	0.3166\\
1.4809	0.3153\\
1.4832	0.314\\
1.4898	0.3098\\
1.4921	0.3083\\
1.4987	0.3035\\
1.501	0.3018\\
1.5054	0.2982\\
1.5068	0.2971\\
1.5083	0.2959\\
1.5097	0.2947\\
1.5111	0.2934\\
};
\addplot [color=black, forget plot]
  table[row sep=crcr]{%
1.5111	0.2934\\
1.5156	0.2894\\
1.5201	0.2852\\
1.5245	0.2808\\
1.529	0.2762\\
1.5337	0.2712\\
1.5384	0.2659\\
1.5432	0.2604\\
1.5479	0.2547\\
1.5526	0.2488\\
1.5573	0.2426\\
1.562	0.2362\\
1.5668	0.2296\\
1.5715	0.2228\\
1.5762	0.2158\\
1.5809	0.2086\\
1.5857	0.2011\\
1.5904	0.1934\\
1.5951	0.1855\\
1.5998	0.1774\\
1.6045	0.169\\
1.6093	0.1605\\
1.614	0.1517\\
1.6187	0.1427\\
1.6234	0.1335\\
1.6282	0.124\\
1.6329	0.1144\\
1.6376	0.1045\\
1.6423	0.0944\\
1.647	0.0841\\
1.6518	0.0736\\
1.6565	0.0628\\
1.6612	0.0519\\
1.6659	0.0407\\
1.6707	0.0293\\
1.6754	0.0177\\
1.6801	0.0058\\
1.6807	0.0044\\
1.6812	0.0029\\
1.6818	0.0015\\
1.6824	-0\\
};
\addplot [color=black, forget plot]
  table[row sep=crcr]{%
1.6824	0\\
1.6824	0.0001\\
1.6825	0.0002\\
1.6826	0.0005\\
1.6828	0.0007\\
1.6829	0.001\\
1.683	0.0012\\
1.6835	0.002\\
1.6839	0.0029\\
1.6843	0.0037\\
1.6848	0.0045\\
1.6852	0.0054\\
1.6857	0.0062\\
1.6861	0.007\\
1.6865	0.0079\\
1.687	0.0087\\
1.6874	0.0095\\
1.6879	0.0103\\
1.6883	0.0111\\
1.6887	0.012\\
1.6892	0.0128\\
1.6896	0.0136\\
1.6901	0.0144\\
1.6909	0.016\\
1.6914	0.0168\\
1.6918	0.0176\\
1.6923	0.0184\\
1.6931	0.02\\
1.6936	0.0208\\
1.694	0.0216\\
1.6945	0.0224\\
1.6953	0.024\\
1.6958	0.0248\\
1.6962	0.0256\\
1.6967	0.0263\\
1.6975	0.0279\\
1.698	0.0287\\
1.6984	0.0294\\
1.6989	0.0302\\
1.6992	0.0307\\
1.6994	0.0312\\
1.7	0.0322\\
};
\addplot [color=black, forget plot]
  table[row sep=crcr]{%
0	0.95\\
0.001	0.95\\
0.002	0.9501\\
0.0051	0.9501\\
};
\addplot [color=black, forget plot]
  table[row sep=crcr]{%
0.0051	0.9501\\
0.0083	0.9501\\
0.0147	0.9497\\
0.0211	0.9489\\
0.0257	0.948\\
0.0303	0.947\\
0.0349	0.9458\\
0.0395	0.9443\\
0.0441	0.9427\\
0.0487	0.9408\\
0.0533	0.9387\\
0.0579	0.9365\\
0.0624	0.934\\
0.067	0.9313\\
0.0716	0.9284\\
0.0762	0.9253\\
0.0808	0.922\\
0.0854	0.9185\\
0.09	0.9148\\
0.0946	0.9108\\
0.0992	0.9067\\
0.1038	0.9023\\
0.1084	0.8978\\
0.113	0.893\\
0.1176	0.8881\\
0.1222	0.8829\\
0.1268	0.8775\\
0.1314	0.8719\\
0.136	0.8661\\
0.1406	0.8601\\
0.1452	0.8539\\
0.1497	0.8475\\
0.1543	0.8409\\
0.1589	0.834\\
0.1635	0.827\\
0.1681	0.8198\\
0.1727	0.8123\\
0.1773	0.8046\\
0.1819	0.7968\\
0.1865	0.7887\\
0.1871	0.7876\\
0.1877	0.7866\\
0.1889	0.7844\\
};
\addplot [color=black, forget plot]
  table[row sep=crcr]{%
0.1889	0.7844\\
0.1936	0.7758\\
0.1983	0.767\\
0.2031	0.7579\\
0.2078	0.7486\\
0.2125	0.7391\\
0.2172	0.7294\\
0.2219	0.7195\\
0.2267	0.7093\\
0.2314	0.699\\
0.2361	0.6884\\
0.2408	0.6775\\
0.2456	0.6665\\
0.2503	0.6553\\
0.255	0.6438\\
0.2597	0.6321\\
0.2644	0.6202\\
0.2692	0.6081\\
0.2739	0.5957\\
0.2786	0.5832\\
0.2833	0.5704\\
0.2881	0.5574\\
0.2928	0.5442\\
0.2975	0.5308\\
0.3022	0.5171\\
0.3069	0.5032\\
0.3117	0.4891\\
0.3164	0.4748\\
0.3211	0.4603\\
0.3258	0.4455\\
0.3306	0.4306\\
0.3353	0.4154\\
0.34	0.4\\
0.3447	0.3844\\
0.3494	0.3685\\
0.3542	0.3525\\
0.3589	0.3362\\
0.3636	0.3197\\
0.3683	0.303\\
0.3731	0.286\\
0.3778	0.2689\\
};
\addplot [color=black, forget plot]
  table[row sep=crcr]{%
0.3778	0.2689\\
0.3815	0.2553\\
0.3852	0.2416\\
0.3926	0.2138\\
0.3973	0.1957\\
0.402	0.1774\\
0.4067	0.1589\\
0.4114	0.1402\\
0.4162	0.1213\\
0.4209	0.1021\\
0.4256	0.0828\\
0.4303	0.0632\\
0.4341	0.0476\\
0.4378	0.0319\\
0.4415	0.016\\
0.4452	0\\
};
\addplot [color=black, forget plot]
  table[row sep=crcr]{%
0.4452	0\\
0.4452	0.0001\\
0.4453	0.0002\\
0.4454	0.0005\\
0.4454	0.0007\\
0.4455	0.001\\
0.4456	0.0012\\
0.446	0.0025\\
0.4464	0.0037\\
0.4468	0.005\\
0.4471	0.0062\\
0.4491	0.0124\\
0.451	0.0186\\
0.453	0.0248\\
0.4549	0.0309\\
0.4579	0.0404\\
0.461	0.0498\\
0.464	0.0591\\
0.467	0.0683\\
0.4701	0.0775\\
0.4731	0.0865\\
0.4762	0.0955\\
0.4822	0.1131\\
0.4853	0.1218\\
0.4883	0.1304\\
0.4913	0.1389\\
0.4944	0.1473\\
0.5004	0.1639\\
0.5035	0.172\\
0.5065	0.1801\\
0.5096	0.188\\
0.5126	0.1959\\
0.5156	0.2037\\
0.5187	0.2114\\
0.5217	0.219\\
0.5247	0.2265\\
0.5278	0.2339\\
0.5338	0.2485\\
0.5369	0.2556\\
0.5399	0.2627\\
0.5429	0.2696\\
0.546	0.2765\\
0.549	0.2833\\
0.5521	0.29\\
0.5551	0.2966\\
0.5581	0.3031\\
0.5612	0.3095\\
0.5642	0.3159\\
0.5648	0.3171\\
0.5654	0.3184\\
0.5661	0.3197\\
0.5667	0.3209\\
};
\addplot [color=black, forget plot]
  table[row sep=crcr]{%
0.5667	0.3209\\
0.5714	0.3305\\
0.5761	0.3398\\
0.5808	0.3489\\
0.5856	0.3578\\
0.5903	0.3665\\
0.595	0.375\\
0.5997	0.3832\\
0.6044	0.3912\\
0.6092	0.3991\\
0.6139	0.4066\\
0.6186	0.414\\
0.6233	0.4212\\
0.6281	0.4281\\
0.6328	0.4348\\
0.6375	0.4413\\
0.6422	0.4476\\
0.6469	0.4536\\
0.6517	0.4595\\
0.6564	0.4651\\
0.6611	0.4705\\
0.6658	0.4757\\
0.6706	0.4806\\
0.6753	0.4854\\
0.68	0.4899\\
0.6847	0.4942\\
0.6894	0.4983\\
0.6942	0.5022\\
0.6989	0.5058\\
0.7036	0.5092\\
0.7083	0.5124\\
0.7131	0.5154\\
0.7178	0.5182\\
0.7225	0.5208\\
0.7272	0.5231\\
0.7319	0.5252\\
0.7367	0.5271\\
0.7414	0.5288\\
0.7461	0.5303\\
0.7508	0.5315\\
0.7556	0.5325\\
};
\addplot [color=black, forget plot]
  table[row sep=crcr]{%
0.7556	0.5325\\
0.7565	0.5327\\
0.7585	0.5331\\
0.7595	0.5332\\
0.7635	0.5338\\
0.7674	0.5341\\
0.7714	0.5344\\
0.7753	0.5344\\
};
\addplot [color=black, forget plot]
  table[row sep=crcr]{%
0.7753	0.5344\\
0.7785	0.5344\\
0.7849	0.534\\
0.7913	0.5332\\
0.7955	0.5324\\
0.7998	0.5315\\
0.804	0.5304\\
0.8082	0.5291\\
0.8124	0.5277\\
0.8167	0.5261\\
0.8251	0.5223\\
0.8294	0.5201\\
0.8336	0.5178\\
0.8378	0.5153\\
0.842	0.5126\\
0.8463	0.5097\\
0.8505	0.5067\\
0.8547	0.5035\\
0.859	0.5001\\
0.8632	0.4966\\
0.8674	0.4928\\
0.8716	0.4889\\
0.8759	0.4848\\
0.8801	0.4806\\
0.8843	0.4761\\
0.8886	0.4715\\
0.8928	0.4667\\
0.897	0.4618\\
0.9012	0.4567\\
0.9055	0.4513\\
0.9097	0.4459\\
0.9139	0.4402\\
0.9182	0.4344\\
0.9266	0.4222\\
0.9311	0.4154\\
0.9355	0.4085\\
0.94	0.4014\\
0.9444	0.3941\\
};
\addplot [color=black, forget plot]
  table[row sep=crcr]{%
0.9444	0.3941\\
0.9492	0.3862\\
0.9539	0.378\\
0.9586	0.3696\\
0.9633	0.361\\
0.9681	0.3522\\
0.9728	0.3432\\
0.9775	0.3339\\
0.9822	0.3244\\
0.9869	0.3148\\
0.9917	0.3048\\
0.9964	0.2947\\
1.0011	0.2844\\
1.0058	0.2738\\
1.0106	0.263\\
1.0153	0.252\\
1.02	0.2408\\
1.0247	0.2293\\
1.0294	0.2177\\
1.0342	0.2058\\
1.0389	0.1937\\
1.0436	0.1814\\
1.0483	0.1688\\
1.0531	0.1561\\
1.0578	0.1431\\
1.0625	0.1299\\
1.0672	0.1165\\
1.0719	0.1028\\
1.0767	0.089\\
1.0814	0.0749\\
1.0861	0.0606\\
1.0908	0.0461\\
1.0956	0.0314\\
1.098	0.0236\\
1.1005	0.0158\\
1.1029	0.0079\\
1.1054	-0\\
};
\addplot [color=black, forget plot]
  table[row sep=crcr]{%
1.1054	0\\
1.1054	0.0001\\
1.1055	0.0002\\
1.1056	0.0005\\
1.1057	0.0007\\
1.1058	0.001\\
1.1059	0.0012\\
1.1064	0.0025\\
1.1069	0.0037\\
1.1074	0.005\\
1.108	0.0062\\
1.1087	0.0079\\
1.1094	0.0095\\
1.1108	0.0129\\
1.1115	0.0145\\
1.1122	0.0162\\
1.1129	0.0178\\
1.1135	0.0195\\
1.1149	0.0227\\
1.1156	0.0244\\
1.1233	0.042\\
1.124	0.0435\\
1.1254	0.0467\\
1.1261	0.0482\\
1.1268	0.0498\\
1.1275	0.0513\\
1.1282	0.0529\\
1.1296	0.0559\\
1.1303	0.0575\\
1.1311	0.0591\\
1.1318	0.0608\\
1.1326	0.0624\\
1.1333	0.064\\
};
\addplot [color=black, forget plot]
  table[row sep=crcr]{%
1.1333	0.064\\
1.1378	0.0736\\
1.1393	0.0767\\
1.144	0.0865\\
1.1487	0.0961\\
1.1535	0.1054\\
1.1582	0.1145\\
1.1629	0.1235\\
1.1676	0.1322\\
1.1724	0.1406\\
1.1771	0.1489\\
1.1818	0.1569\\
1.1865	0.1647\\
1.1912	0.1723\\
1.196	0.1797\\
1.2007	0.1869\\
1.2054	0.1938\\
1.2101	0.2006\\
1.2149	0.2071\\
1.2196	0.2134\\
1.2243	0.2194\\
1.229	0.2253\\
1.2337	0.2309\\
1.2385	0.2363\\
1.2432	0.2415\\
1.2479	0.2465\\
1.2526	0.2513\\
1.2574	0.2558\\
1.2621	0.2601\\
1.2668	0.2642\\
1.2715	0.2681\\
1.2762	0.2718\\
1.281	0.2752\\
1.2857	0.2784\\
1.2904	0.2814\\
1.2951	0.2842\\
1.2999	0.2868\\
1.3046	0.2891\\
1.3093	0.2913\\
1.3125	0.2926\\
1.3158	0.2938\\
1.319	0.295\\
1.3222	0.296\\
};
\addplot [color=black, forget plot]
  table[row sep=crcr]{%
1.3222	0.296\\
1.3238	0.2964\\
1.3253	0.2969\\
1.3269	0.2973\\
1.3284	0.2977\\
1.3331	0.2987\\
1.3378	0.2995\\
1.3426	0.3001\\
1.3473	0.3005\\
1.3487	0.3005\\
1.3501	0.3006\\
1.353	0.3006\\
};
\addplot [color=black, forget plot]
  table[row sep=crcr]{%
1.353	0.3006\\
1.3562	0.3006\\
1.3626	0.3002\\
1.369	0.2994\\
1.3729	0.2987\\
1.3769	0.2978\\
1.3808	0.2968\\
1.3848	0.2957\\
1.3887	0.2944\\
1.3927	0.2929\\
1.3966	0.2913\\
1.4006	0.2895\\
1.4045	0.2876\\
1.4085	0.2855\\
1.4125	0.2833\\
1.4164	0.2809\\
1.4204	0.2783\\
1.4243	0.2757\\
1.4283	0.2728\\
1.4322	0.2698\\
1.4362	0.2667\\
1.4401	0.2634\\
1.4441	0.2599\\
1.448	0.2563\\
1.452	0.2525\\
1.4559	0.2486\\
1.4599	0.2445\\
1.4639	0.2403\\
1.4678	0.2359\\
1.4718	0.2314\\
1.4757	0.2267\\
1.4797	0.2219\\
1.4836	0.2169\\
1.4876	0.2118\\
1.4915	0.2065\\
1.4955	0.201\\
1.4994	0.1955\\
1.5033	0.1898\\
1.5111	0.178\\
};
\addplot [color=black, forget plot]
  table[row sep=crcr]{%
1.5111	0.178\\
1.5158	0.1705\\
1.5206	0.1629\\
1.5253	0.155\\
1.53	0.1469\\
1.5347	0.1386\\
1.5394	0.1301\\
1.5442	0.1213\\
1.5489	0.1123\\
1.5536	0.1032\\
1.5583	0.0938\\
1.5631	0.0841\\
1.5678	0.0743\\
1.5725	0.0642\\
1.5772	0.0539\\
1.5819	0.0435\\
1.5867	0.0327\\
1.5901	0.0247\\
1.5936	0.0166\\
1.5971	0.0084\\
1.6005	-0\\
};
\addplot [color=black, forget plot]
  table[row sep=crcr]{%
1.6005	0\\
1.6006	0.0001\\
1.6006	0.0002\\
1.6008	0.0005\\
1.6009	0.0007\\
1.6011	0.001\\
1.6012	0.0012\\
1.6019	0.0025\\
1.6033	0.0049\\
1.604	0.0062\\
1.6064	0.0106\\
1.6139	0.0235\\
1.6164	0.0276\\
1.6189	0.0318\\
1.6214	0.0358\\
1.6238	0.0398\\
1.6288	0.0476\\
1.6338	0.0552\\
1.6363	0.0588\\
1.6388	0.0625\\
1.6413	0.066\\
1.6437	0.0695\\
1.6462	0.073\\
1.6487	0.0764\\
1.6537	0.083\\
1.6562	0.0862\\
1.6587	0.0893\\
1.6611	0.0924\\
1.6661	0.0984\\
1.6686	0.1013\\
1.6736	0.1069\\
1.6786	0.1123\\
1.681	0.1149\\
1.686	0.1199\\
1.6885	0.1223\\
1.6935	0.1269\\
1.6951	0.1284\\
1.6967	0.1298\\
1.6984	0.1313\\
1.7	0.1326\\
};
\addplot [color=black, forget plot]
  table[row sep=crcr]{%
0	0.95\\
0.0008	0.95\\
0.001	0.9499\\
0.0023	0.9499\\
0.0036	0.9498\\
0.0049	0.9496\\
0.0061	0.9495\\
0.0109	0.9489\\
0.0156	0.948\\
0.0203	0.947\\
0.025	0.9457\\
0.0298	0.9442\\
0.0345	0.9424\\
0.0392	0.9405\\
0.0439	0.9383\\
0.0486	0.936\\
0.0534	0.9334\\
0.0581	0.9305\\
0.0628	0.9275\\
0.0675	0.9243\\
0.0723	0.9208\\
0.077	0.9171\\
0.0817	0.9132\\
0.0864	0.909\\
0.0911	0.9047\\
0.0959	0.9001\\
0.1006	0.8953\\
0.1053	0.8903\\
0.11	0.8851\\
0.1148	0.8797\\
0.1195	0.874\\
0.1242	0.8681\\
0.1289	0.862\\
0.1336	0.8557\\
0.1384	0.8492\\
0.1431	0.8424\\
0.1478	0.8354\\
0.1525	0.8283\\
0.1573	0.8208\\
0.162	0.8132\\
0.1667	0.8054\\
0.1714	0.7973\\
0.1761	0.789\\
0.1793	0.7833\\
0.1825	0.7775\\
0.1857	0.7716\\
0.1889	0.7656\\
};
\addplot [color=black, forget plot]
  table[row sep=crcr]{%
0.1889	0.7656\\
0.1936	0.7565\\
0.1983	0.7471\\
0.2031	0.7376\\
0.2078	0.7279\\
0.2125	0.7179\\
0.2172	0.7077\\
0.2219	0.6973\\
0.2267	0.6867\\
0.2314	0.6758\\
0.2361	0.6647\\
0.2408	0.6535\\
0.2456	0.642\\
0.2503	0.6302\\
0.255	0.6183\\
0.2597	0.6061\\
0.2644	0.5938\\
0.2692	0.5812\\
0.2739	0.5684\\
0.2786	0.5553\\
0.2833	0.5421\\
0.2881	0.5286\\
0.2928	0.5149\\
0.2975	0.501\\
0.3022	0.4869\\
0.3069	0.4725\\
0.3117	0.458\\
0.3164	0.4432\\
0.3211	0.4282\\
0.3258	0.413\\
0.3306	0.3975\\
0.3353	0.3819\\
0.34	0.366\\
0.3447	0.3499\\
0.3494	0.3336\\
0.3542	0.317\\
0.3589	0.3003\\
0.3636	0.2833\\
0.3683	0.2661\\
0.3731	0.2487\\
0.3778	0.2311\\
};
\addplot [color=black, forget plot]
  table[row sep=crcr]{%
0.3778	0.2311\\
0.384	0.2077\\
0.3871	0.1958\\
0.3901	0.1839\\
0.3949	0.1655\\
0.3996	0.1468\\
0.4043	0.128\\
0.409	0.1089\\
0.4138	0.0896\\
0.4185	0.0701\\
0.4232	0.0504\\
0.4279	0.0304\\
0.4297	0.0229\\
0.4315	0.0153\\
0.4332	0.0077\\
0.435	-0\\
};
\addplot [color=black, forget plot]
  table[row sep=crcr]{%
0.435	0\\
0.435	0.0001\\
0.4351	0.0001\\
0.4351	0.0002\\
0.4352	0.0005\\
0.4352	0.0007\\
0.4353	0.001\\
0.4354	0.0012\\
0.4358	0.0025\\
0.4362	0.0037\\
0.4366	0.005\\
0.4369	0.0062\\
0.4389	0.0124\\
0.4408	0.0186\\
0.4428	0.0248\\
0.4447	0.0309\\
0.448	0.0412\\
0.4513	0.0514\\
0.4579	0.0714\\
0.4612	0.0813\\
0.4645	0.091\\
0.4677	0.1007\\
0.4743	0.1197\\
0.4809	0.1383\\
0.4842	0.1474\\
0.4875	0.1564\\
0.4908	0.1653\\
0.4941	0.1741\\
0.4974	0.1828\\
0.5007	0.1914\\
0.5039	0.1999\\
0.5072	0.2083\\
0.5138	0.2247\\
0.5171	0.2328\\
0.5237	0.2486\\
0.527	0.2563\\
0.5336	0.2715\\
0.5369	0.2789\\
0.5401	0.2862\\
0.5434	0.2934\\
0.5467	0.3005\\
0.55	0.3075\\
0.5533	0.3144\\
0.5566	0.3212\\
0.5599	0.3279\\
0.5632	0.3344\\
0.5641	0.3362\\
0.5649	0.3379\\
0.5667	0.3413\\
};
\addplot [color=black, forget plot]
  table[row sep=crcr]{%
0.5667	0.3413\\
0.5714	0.3504\\
0.5761	0.3592\\
0.5808	0.3679\\
0.5856	0.3763\\
0.5903	0.3845\\
0.595	0.3925\\
0.5997	0.4003\\
0.6044	0.4078\\
0.6092	0.4152\\
0.6139	0.4223\\
0.6186	0.4292\\
0.6233	0.4358\\
0.6281	0.4423\\
0.6328	0.4485\\
0.6375	0.4546\\
0.6422	0.4604\\
0.6469	0.466\\
0.6517	0.4713\\
0.6564	0.4765\\
0.6611	0.4814\\
0.6658	0.4861\\
0.6706	0.4906\\
0.6753	0.4949\\
0.68	0.4989\\
0.6847	0.5027\\
0.6894	0.5064\\
0.6942	0.5098\\
0.6989	0.5129\\
0.7036	0.5159\\
0.7083	0.5186\\
0.7131	0.5212\\
0.7178	0.5235\\
0.7225	0.5255\\
0.7272	0.5274\\
0.7319	0.5291\\
0.7367	0.5305\\
0.7414	0.5317\\
0.7461	0.5327\\
0.7508	0.5334\\
0.7556	0.534\\
};
\addplot [color=black, forget plot]
  table[row sep=crcr]{%
0.7556	0.534\\
0.756	0.534\\
0.7565	0.5341\\
0.757	0.5341\\
0.7575	0.5342\\
0.7613	0.5344\\
0.7651	0.5344\\
};
\addplot [color=black, forget plot]
  table[row sep=crcr]{%
0.7651	0.5344\\
0.7683	0.5344\\
0.7747	0.534\\
0.7811	0.5332\\
0.7856	0.5324\\
0.7901	0.5314\\
0.7946	0.5302\\
0.799	0.5288\\
0.8035	0.5272\\
0.808	0.5254\\
0.8125	0.5234\\
0.817	0.5213\\
0.8215	0.5189\\
0.8259	0.5163\\
0.8304	0.5135\\
0.8349	0.5106\\
0.8394	0.5074\\
0.8439	0.504\\
0.8484	0.5005\\
0.8528	0.4967\\
0.8573	0.4927\\
0.8618	0.4886\\
0.8663	0.4842\\
0.8708	0.4797\\
0.8753	0.4749\\
0.8797	0.47\\
0.8842	0.4649\\
0.8887	0.4595\\
0.8932	0.454\\
0.8977	0.4483\\
0.9022	0.4423\\
0.9066	0.4362\\
0.9111	0.4299\\
0.9156	0.4234\\
0.9201	0.4166\\
0.9246	0.4097\\
0.9291	0.4026\\
0.9335	0.3953\\
0.938	0.3878\\
0.9425	0.3801\\
0.943	0.3792\\
0.944	0.3776\\
0.9444	0.3767\\
};
\addplot [color=black, forget plot]
  table[row sep=crcr]{%
0.9444	0.3767\\
0.9492	0.3683\\
0.9539	0.3597\\
0.9586	0.3508\\
0.9633	0.3417\\
0.9681	0.3324\\
0.9728	0.3229\\
0.9775	0.3132\\
0.9822	0.3032\\
0.9869	0.2931\\
0.9917	0.2827\\
0.9964	0.2721\\
1.0011	0.2613\\
1.0058	0.2502\\
1.0106	0.239\\
1.0153	0.2275\\
1.02	0.2158\\
1.0247	0.2039\\
1.0294	0.1917\\
1.0342	0.1794\\
1.0389	0.1668\\
1.0436	0.154\\
1.0483	0.141\\
1.0531	0.1278\\
1.0578	0.1143\\
1.0625	0.1007\\
1.0672	0.0868\\
1.0767	0.0583\\
1.0813	0.0441\\
1.0859	0.0296\\
1.0906	0.0149\\
1.0952	-0\\
};
\addplot [color=black, forget plot]
  table[row sep=crcr]{%
1.0952	0\\
1.0952	0.0001\\
1.0953	0.0002\\
1.0954	0.0005\\
1.0955	0.0007\\
1.0956	0.001\\
1.0957	0.0012\\
1.0962	0.0025\\
1.0967	0.0037\\
1.0973	0.005\\
1.0978	0.0062\\
1.0987	0.0085\\
1.0997	0.0108\\
1.1006	0.013\\
1.1016	0.0153\\
1.1025	0.0175\\
1.1035	0.0198\\
1.1044	0.022\\
1.1054	0.0242\\
1.1063	0.0265\\
1.1083	0.0309\\
1.1092	0.033\\
1.1102	0.0352\\
1.1111	0.0374\\
1.1121	0.0396\\
1.113	0.0417\\
1.114	0.0439\\
1.1149	0.046\\
1.1159	0.0481\\
1.1168	0.0502\\
1.1178	0.0523\\
1.1187	0.0544\\
1.1197	0.0565\\
1.1206	0.0586\\
1.1226	0.0628\\
1.1235	0.0648\\
1.1245	0.0669\\
1.1254	0.0689\\
1.1264	0.0709\\
1.1273	0.0729\\
1.1283	0.0749\\
1.1292	0.0769\\
1.1302	0.0789\\
1.1311	0.0809\\
1.1324	0.0835\\
1.1327	0.0842\\
1.133	0.0848\\
1.1333	0.0855\\
};
\addplot [color=black, forget plot]
  table[row sep=crcr]{%
1.1333	0.0855\\
1.1354	0.0897\\
1.1375	0.094\\
1.1396	0.0982\\
1.1417	0.1023\\
1.1464	0.1115\\
1.1511	0.1205\\
1.1559	0.1293\\
1.1606	0.1378\\
1.1653	0.1461\\
1.17	0.1543\\
1.1747	0.1621\\
1.1795	0.1698\\
1.1842	0.1773\\
1.1889	0.1845\\
1.1936	0.1915\\
1.1984	0.1983\\
1.2031	0.2049\\
1.2078	0.2113\\
1.2125	0.2174\\
1.2172	0.2233\\
1.222	0.229\\
1.2267	0.2345\\
1.2314	0.2398\\
1.2361	0.2449\\
1.2409	0.2497\\
1.2456	0.2543\\
1.2503	0.2587\\
1.255	0.2629\\
1.2597	0.2668\\
1.2645	0.2706\\
1.2692	0.2741\\
1.2739	0.2774\\
1.2786	0.2805\\
1.2834	0.2833\\
1.2881	0.286\\
1.2928	0.2884\\
1.2975	0.2906\\
1.3022	0.2926\\
1.307	0.2943\\
1.3117	0.2959\\
1.3143	0.2967\\
1.317	0.2974\\
1.3222	0.2986\\
};
\addplot [color=black, forget plot]
  table[row sep=crcr]{%
1.3222	0.2986\\
1.3233	0.2988\\
1.3243	0.299\\
1.3253	0.2991\\
1.3264	0.2993\\
1.3305	0.2999\\
1.3346	0.3003\\
1.3387	0.3005\\
1.3428	0.3006\\
};
\addplot [color=black, forget plot]
  table[row sep=crcr]{%
1.3428	0.3006\\
1.346	0.3006\\
1.3524	0.3002\\
1.3588	0.2994\\
1.363	0.2986\\
1.3672	0.2977\\
1.3714	0.2966\\
1.3756	0.2953\\
1.3798	0.2939\\
1.384	0.2923\\
1.3882	0.2905\\
1.3924	0.2885\\
1.3966	0.2864\\
1.4009	0.2841\\
1.4051	0.2816\\
1.4093	0.2789\\
1.4135	0.2761\\
1.4177	0.2731\\
1.4219	0.2699\\
1.4261	0.2666\\
1.4303	0.263\\
1.4345	0.2593\\
1.4387	0.2555\\
1.4429	0.2514\\
1.4471	0.2472\\
1.4514	0.2428\\
1.4556	0.2382\\
1.4598	0.2335\\
1.464	0.2286\\
1.4682	0.2235\\
1.4724	0.2182\\
1.4766	0.2128\\
1.4808	0.2072\\
1.485	0.2014\\
1.4892	0.1954\\
1.4934	0.1893\\
1.4979	0.1827\\
1.5023	0.1758\\
1.5067	0.1688\\
1.5111	0.1616\\
};
\addplot [color=black, forget plot]
  table[row sep=crcr]{%
1.5111	0.1616\\
1.5158	0.1537\\
1.5206	0.1456\\
1.5253	0.1372\\
1.53	0.1287\\
1.5347	0.1199\\
1.5394	0.1109\\
1.5442	0.1017\\
1.5489	0.0922\\
1.5536	0.0826\\
1.5583	0.0727\\
1.5631	0.0626\\
1.5678	0.0523\\
1.5725	0.0418\\
1.5772	0.031\\
1.5819	0.02\\
1.5867	0.0089\\
1.5876	0.0067\\
1.5885	0.0044\\
1.5903	-0\\
};
\addplot [color=black, forget plot]
  table[row sep=crcr]{%
1.5903	0\\
1.5904	0.0001\\
1.5904	0.0002\\
1.5906	0.0005\\
1.5907	0.0007\\
1.5909	0.001\\
1.591	0.0012\\
1.5917	0.0025\\
1.5931	0.0049\\
1.5938	0.0062\\
1.5992	0.0158\\
1.602	0.0205\\
1.6047	0.0252\\
1.6075	0.0298\\
1.6102	0.0343\\
1.6129	0.0387\\
1.6157	0.043\\
1.6184	0.0473\\
1.6212	0.0515\\
1.6239	0.0556\\
1.6267	0.0597\\
1.6294	0.0637\\
1.6321	0.0676\\
1.6349	0.0714\\
1.6376	0.0752\\
1.6404	0.0788\\
1.6431	0.0825\\
1.6458	0.086\\
1.6486	0.0895\\
1.6513	0.0928\\
1.6541	0.0962\\
1.6568	0.0994\\
1.6596	0.1026\\
1.6623	0.1057\\
1.665	0.1087\\
1.6678	0.1116\\
1.6705	0.1145\\
1.6787	0.1227\\
1.6815	0.1253\\
1.6842	0.1278\\
1.687	0.1302\\
1.6897	0.1326\\
1.6925	0.1349\\
1.6943	0.1364\\
1.6962	0.1379\\
1.6981	0.1393\\
1.7	0.1408\\
};
\addplot [color=black, forget plot]
  table[row sep=crcr]{%
0	1.05\\
0.0008	1.05\\
0.001	1.0499\\
0.0023	1.0499\\
0.0036	1.0498\\
0.0049	1.0496\\
0.0061	1.0495\\
0.0109	1.0489\\
0.0156	1.048\\
0.0203	1.047\\
0.025	1.0457\\
0.0298	1.0442\\
0.0345	1.0424\\
0.0392	1.0405\\
0.0439	1.0383\\
0.0486	1.036\\
0.0534	1.0334\\
0.0581	1.0305\\
0.0628	1.0275\\
0.0675	1.0243\\
0.0723	1.0208\\
0.077	1.0171\\
0.0817	1.0132\\
0.0864	1.009\\
0.0911	1.0047\\
0.0959	1.0001\\
0.1006	0.9953\\
0.1053	0.9903\\
0.11	0.9851\\
0.1148	0.9797\\
0.1195	0.974\\
0.1242	0.9681\\
0.1289	0.962\\
0.1336	0.9557\\
0.1384	0.9492\\
0.1431	0.9424\\
0.1478	0.9354\\
0.1525	0.9283\\
0.1573	0.9208\\
0.162	0.9132\\
0.1667	0.9054\\
0.1714	0.8973\\
0.1761	0.889\\
0.1793	0.8833\\
0.1825	0.8775\\
0.1857	0.8716\\
0.1889	0.8656\\
};
\addplot [color=black, forget plot]
  table[row sep=crcr]{%
0.1889	0.8656\\
0.1936	0.8565\\
0.1983	0.8471\\
0.2031	0.8376\\
0.2078	0.8279\\
0.2125	0.8179\\
0.2172	0.8077\\
0.2219	0.7973\\
0.2267	0.7867\\
0.2314	0.7758\\
0.2361	0.7647\\
0.2408	0.7535\\
0.2456	0.742\\
0.2503	0.7302\\
0.255	0.7183\\
0.2597	0.7061\\
0.2644	0.6938\\
0.2692	0.6812\\
0.2739	0.6684\\
0.2786	0.6553\\
0.2833	0.6421\\
0.2881	0.6286\\
0.2928	0.6149\\
0.2975	0.601\\
0.3022	0.5869\\
0.3069	0.5725\\
0.3117	0.558\\
0.3164	0.5432\\
0.3211	0.5282\\
0.3258	0.513\\
0.3306	0.4975\\
0.3353	0.4819\\
0.34	0.466\\
0.3447	0.4499\\
0.3494	0.4336\\
0.3542	0.417\\
0.3589	0.4003\\
0.3636	0.3833\\
0.3683	0.3661\\
0.3731	0.3487\\
0.3778	0.3311\\
};
\addplot [color=black, forget plot]
  table[row sep=crcr]{%
0.3778	0.3311\\
0.3822	0.3144\\
0.3866	0.2974\\
0.3911	0.2803\\
0.3955	0.263\\
0.4002	0.2444\\
0.4049	0.2255\\
0.4097	0.2064\\
0.4144	0.187\\
0.4191	0.1675\\
0.4238	0.1477\\
0.4285	0.1278\\
0.4333	0.1076\\
0.438	0.0871\\
0.4427	0.0665\\
0.4474	0.0457\\
0.4522	0.0246\\
0.4535	0.0185\\
0.4549	0.0123\\
0.4562	0.0062\\
0.4576	-0\\
};
\addplot [color=black, forget plot]
  table[row sep=crcr]{%
0.4576	0\\
0.4576	0.0002\\
0.4577	0.0002\\
0.4577	0.0005\\
0.4578	0.0007\\
0.4579	0.001\\
0.458	0.0012\\
0.4583	0.0025\\
0.4587	0.0037\\
0.4591	0.005\\
0.4594	0.0062\\
0.4613	0.0124\\
0.4631	0.0186\\
0.465	0.0248\\
0.4668	0.0309\\
0.4695	0.0399\\
0.4723	0.0489\\
0.4777	0.0665\\
0.4804	0.0752\\
0.4832	0.0838\\
0.4859	0.0924\\
0.4886	0.1009\\
0.4914	0.1093\\
0.4968	0.1259\\
0.4995	0.1341\\
0.5023	0.1422\\
0.505	0.1503\\
0.5077	0.1583\\
0.5104	0.1662\\
0.5132	0.174\\
0.5159	0.1818\\
0.5186	0.1894\\
0.5213	0.1971\\
0.5241	0.2046\\
0.5268	0.2121\\
0.5295	0.2195\\
0.5323	0.2268\\
0.5377	0.2412\\
0.5404	0.2483\\
0.5432	0.2553\\
0.5459	0.2623\\
0.5486	0.2692\\
0.5513	0.276\\
0.5541	0.2827\\
0.5568	0.2894\\
0.5595	0.296\\
0.5622	0.3025\\
0.565	0.309\\
0.5658	0.311\\
0.5662	0.3119\\
0.5667	0.3129\\
};
\addplot [color=black, forget plot]
  table[row sep=crcr]{%
0.5667	0.3129\\
0.5714	0.3239\\
0.5761	0.3345\\
0.5808	0.345\\
0.5856	0.3553\\
0.5903	0.3653\\
0.595	0.3751\\
0.5997	0.3847\\
0.6044	0.3941\\
0.6092	0.4033\\
0.6139	0.4122\\
0.6186	0.421\\
0.6233	0.4295\\
0.6281	0.4378\\
0.6328	0.4458\\
0.6375	0.4537\\
0.6422	0.4613\\
0.6469	0.4687\\
0.6517	0.4759\\
0.6564	0.4829\\
0.6611	0.4897\\
0.6658	0.4962\\
0.6706	0.5025\\
0.6753	0.5086\\
0.68	0.5145\\
0.6847	0.5202\\
0.6894	0.5256\\
0.6942	0.5308\\
0.6989	0.5359\\
0.7036	0.5406\\
0.7083	0.5452\\
0.7131	0.5496\\
0.7178	0.5537\\
0.7225	0.5576\\
0.7272	0.5613\\
0.7319	0.5648\\
0.7367	0.568\\
0.7414	0.5711\\
0.7461	0.5739\\
0.7508	0.5765\\
0.7556	0.5789\\
};
\addplot [color=black, forget plot]
  table[row sep=crcr]{%
0.7556	0.5789\\
0.758	0.58\\
0.763	0.5822\\
0.7654	0.5832\\
0.7701	0.5849\\
0.7749	0.5863\\
0.7796	0.5876\\
0.7843	0.5887\\
0.789	0.5895\\
0.7938	0.5901\\
0.7985	0.5905\\
0.8032	0.5907\\
0.8046	0.5907\\
};
\addplot [color=black, forget plot]
  table[row sep=crcr]{%
0.8046	0.5907\\
0.8072	0.5907\\
0.8078	0.5906\\
0.811	0.5905\\
0.8174	0.5899\\
0.8206	0.5894\\
0.8241	0.5888\\
0.8276	0.5881\\
0.8311	0.5873\\
0.8346	0.5863\\
0.8381	0.5852\\
0.8416	0.584\\
0.8451	0.5827\\
0.8521	0.5797\\
0.8556	0.578\\
0.8591	0.5762\\
0.8661	0.5722\\
0.8696	0.57\\
0.8731	0.5677\\
0.8766	0.5653\\
0.8801	0.5628\\
0.8835	0.5602\\
0.887	0.5574\\
0.8905	0.5545\\
0.894	0.5515\\
0.8975	0.5484\\
0.9045	0.5418\\
0.908	0.5383\\
0.9115	0.5347\\
0.9185	0.5271\\
0.9255	0.5191\\
0.9325	0.5105\\
0.9355	0.5067\\
0.9415	0.4989\\
0.9444	0.4948\\
};
\addplot [color=black, forget plot]
  table[row sep=crcr]{%
0.9444	0.4948\\
0.9492	0.4882\\
0.9539	0.4814\\
0.9586	0.4744\\
0.9633	0.4672\\
0.9681	0.4597\\
0.9728	0.452\\
0.9775	0.4441\\
0.9822	0.436\\
0.9869	0.4277\\
0.9917	0.4191\\
0.9964	0.4103\\
1.0011	0.4013\\
1.0058	0.3921\\
1.0106	0.3827\\
1.0153	0.3731\\
1.02	0.3632\\
1.0247	0.3531\\
1.0294	0.3428\\
1.0342	0.3323\\
1.0389	0.3215\\
1.0436	0.3106\\
1.0483	0.2994\\
1.0531	0.288\\
1.0578	0.2764\\
1.0625	0.2645\\
1.0672	0.2525\\
1.0719	0.2402\\
1.0767	0.2277\\
1.0814	0.215\\
1.0861	0.2021\\
1.0908	0.1889\\
1.0956	0.1756\\
1.1003	0.162\\
1.105	0.1482\\
1.1097	0.1341\\
1.1144	0.1199\\
1.1192	0.1054\\
1.1239	0.0908\\
1.1286	0.0759\\
1.1333	0.0607\\
};
\addplot [color=black, forget plot]
  table[row sep=crcr]{%
1.1333	0.0607\\
1.1343	0.0577\\
1.1352	0.0546\\
1.1362	0.0515\\
1.1371	0.0485\\
1.1408	0.0365\\
1.1444	0.0245\\
1.148	0.0123\\
1.1517	0\\
};
\addplot [color=black, forget plot]
  table[row sep=crcr]{%
1.1517	0\\
1.1517	0.0002\\
1.1518	0.0005\\
1.1519	0.0007\\
1.152	0.001\\
1.1521	0.0012\\
1.1526	0.0025\\
1.1531	0.0037\\
1.1536	0.005\\
1.1541	0.0062\\
1.1566	0.0124\\
1.159	0.0185\\
1.1615	0.0246\\
1.1639	0.0306\\
1.1682	0.0409\\
1.1725	0.051\\
1.1767	0.0609\\
1.181	0.0707\\
1.1853	0.0802\\
1.1895	0.0896\\
1.1938	0.0989\\
1.198	0.1079\\
1.2023	0.1167\\
1.2066	0.1254\\
1.2108	0.1339\\
1.2151	0.1423\\
1.2194	0.1504\\
1.2236	0.1584\\
1.2322	0.1738\\
1.2364	0.1812\\
1.2407	0.1884\\
1.245	0.1955\\
1.2492	0.2024\\
1.2535	0.2091\\
1.2577	0.2157\\
1.262	0.222\\
1.2663	0.2282\\
1.2705	0.2342\\
1.2748	0.24\\
1.2791	0.2457\\
1.2833	0.2512\\
1.2919	0.2616\\
1.2961	0.2665\\
1.3004	0.2712\\
1.3047	0.2758\\
1.3089	0.2802\\
1.3132	0.2844\\
1.3174	0.2885\\
1.3198	0.2907\\
1.321	0.2917\\
1.3222	0.2928\\
};
\addplot [color=black, forget plot]
  table[row sep=crcr]{%
1.3222	0.2928\\
1.3267	0.2967\\
1.3312	0.3003\\
1.3357	0.3038\\
1.3402	0.3071\\
1.345	0.3103\\
1.3497	0.3133\\
1.3544	0.316\\
1.3591	0.3186\\
1.3639	0.3209\\
1.3686	0.3231\\
1.3733	0.3249\\
1.378	0.3266\\
1.3827	0.3281\\
1.3875	0.3293\\
1.3922	0.3304\\
1.3969	0.3312\\
1.4007	0.3316\\
1.4044	0.332\\
1.4082	0.3322\\
1.4119	0.3323\\
};
\addplot [color=black, forget plot]
  table[row sep=crcr]{%
1.4119	0.3323\\
1.4132	0.3323\\
1.4138	0.3322\\
1.4151	0.3322\\
1.4176	0.3321\\
1.4226	0.3317\\
1.425	0.3314\\
1.4275	0.3311\\
1.43	0.3307\\
1.435	0.3297\\
1.4374	0.3291\\
1.4424	0.3277\\
1.4474	0.3261\\
1.4498	0.3252\\
1.4523	0.3243\\
1.4548	0.3233\\
1.4573	0.3222\\
1.4598	0.321\\
1.4622	0.3199\\
1.4672	0.3173\\
1.4722	0.3145\\
1.4746	0.313\\
1.4796	0.3098\\
1.4846	0.3064\\
1.487	0.3046\\
1.492	0.3008\\
1.4945	0.2988\\
1.4969	0.2968\\
1.5019	0.2926\\
1.5044	0.2903\\
1.5061	0.2888\\
1.5077	0.2872\\
1.5111	0.284\\
};
\addplot [color=black, forget plot]
  table[row sep=crcr]{%
1.5111	0.284\\
1.5158	0.2793\\
1.5206	0.2744\\
1.5253	0.2692\\
1.53	0.2639\\
1.5347	0.2583\\
1.5394	0.2525\\
1.5442	0.2465\\
1.5489	0.2403\\
1.5536	0.2338\\
1.5583	0.2271\\
1.5631	0.2202\\
1.5678	0.2131\\
1.5725	0.2058\\
1.5772	0.1983\\
1.5819	0.1905\\
1.5867	0.1825\\
1.5914	0.1743\\
1.5961	0.1659\\
1.6008	0.1572\\
1.6056	0.1484\\
1.6103	0.1393\\
1.615	0.13\\
1.6197	0.1205\\
1.6244	0.1107\\
1.6292	0.1008\\
1.6339	0.0906\\
1.6386	0.0802\\
1.6433	0.0696\\
1.6481	0.0588\\
1.6528	0.0477\\
1.6575	0.0365\\
1.6622	0.025\\
1.6672	0.0126\\
1.6722	-0\\
};
\addplot [color=black, forget plot]
  table[row sep=crcr]{%
1.6722	0\\
1.6722	0.0001\\
1.6723	0.0001\\
1.6723	0.0002\\
1.6724	0.0005\\
1.6726	0.0007\\
1.6727	0.001\\
1.6728	0.0012\\
1.6735	0.0025\\
1.6741	0.0037\\
1.6748	0.0049\\
1.6755	0.0062\\
1.6761	0.0075\\
1.6789	0.0127\\
1.6796	0.0139\\
1.6817	0.0178\\
1.6824	0.019\\
1.6831	0.0203\\
1.6838	0.0215\\
1.6845	0.0228\\
1.6852	0.024\\
1.6859	0.0253\\
1.6873	0.0277\\
1.688	0.029\\
1.6894	0.0314\\
1.69	0.0326\\
1.6949	0.041\\
1.6956	0.0421\\
1.697	0.0445\\
1.6977	0.0456\\
1.6983	0.0466\\
1.6988	0.0475\\
1.6994	0.0485\\
1.7	0.0494\\
};
\addplot [color=black, forget plot]
  table[row sep=crcr]{%
0	1.05\\
0.001	1.05\\
0.002	1.0501\\
0.0051	1.0501\\
};
\addplot [color=black, forget plot]
  table[row sep=crcr]{%
0.0051	1.0501\\
0.0083	1.0501\\
0.0147	1.0497\\
0.0211	1.0489\\
0.0257	1.048\\
0.0303	1.047\\
0.0349	1.0458\\
0.0395	1.0443\\
0.0441	1.0427\\
0.0487	1.0408\\
0.0533	1.0387\\
0.0579	1.0365\\
0.0624	1.034\\
0.067	1.0313\\
0.0716	1.0284\\
0.0762	1.0253\\
0.0808	1.022\\
0.0854	1.0185\\
0.09	1.0148\\
0.0946	1.0108\\
0.0992	1.0067\\
0.1038	1.0023\\
0.1084	0.9978\\
0.113	0.993\\
0.1176	0.9881\\
0.1222	0.9829\\
0.1268	0.9775\\
0.1314	0.9719\\
0.136	0.9661\\
0.1406	0.9601\\
0.1452	0.9539\\
0.1497	0.9475\\
0.1543	0.9409\\
0.1589	0.934\\
0.1635	0.927\\
0.1681	0.9198\\
0.1727	0.9123\\
0.1773	0.9046\\
0.1819	0.8968\\
0.1865	0.8887\\
0.1871	0.8876\\
0.1877	0.8866\\
0.1889	0.8844\\
};
\addplot [color=black, forget plot]
  table[row sep=crcr]{%
0.1889	0.8844\\
0.1936	0.8758\\
0.1983	0.867\\
0.2031	0.8579\\
0.2078	0.8486\\
0.2125	0.8391\\
0.2172	0.8294\\
0.2219	0.8195\\
0.2267	0.8093\\
0.2314	0.799\\
0.2361	0.7884\\
0.2408	0.7775\\
0.2456	0.7665\\
0.2503	0.7553\\
0.255	0.7438\\
0.2597	0.7321\\
0.2644	0.7202\\
0.2692	0.7081\\
0.2739	0.6957\\
0.2786	0.6832\\
0.2833	0.6704\\
0.2881	0.6574\\
0.2928	0.6442\\
0.2975	0.6308\\
0.3022	0.6171\\
0.3069	0.6032\\
0.3117	0.5891\\
0.3164	0.5748\\
0.3211	0.5603\\
0.3258	0.5455\\
0.3306	0.5306\\
0.3353	0.5154\\
0.34	0.5\\
0.3447	0.4844\\
0.3494	0.4685\\
0.3542	0.4525\\
0.3589	0.4362\\
0.3636	0.4197\\
0.3683	0.403\\
0.3731	0.386\\
0.3778	0.3689\\
};
\addplot [color=black, forget plot]
  table[row sep=crcr]{%
0.3778	0.3689\\
0.3825	0.3515\\
0.3872	0.3339\\
0.3919	0.3161\\
0.3967	0.2981\\
0.4014	0.2798\\
0.4061	0.2613\\
0.4108	0.2427\\
0.4156	0.2238\\
0.4203	0.2046\\
0.425	0.1853\\
0.4297	0.1657\\
0.4344	0.1459\\
0.4392	0.1259\\
0.4439	0.1057\\
0.4486	0.0853\\
0.4533	0.0646\\
0.4569	0.0487\\
0.4606	0.0326\\
0.4642	0.0164\\
0.4678	-0\\
};
\addplot [color=black, forget plot]
  table[row sep=crcr]{%
0.4678	0\\
0.4678	0.0002\\
0.4679	0.0002\\
0.4679	0.0005\\
0.468	0.0007\\
0.4681	0.001\\
0.4682	0.0012\\
0.4685	0.0025\\
0.4689	0.0037\\
0.4693	0.005\\
0.4696	0.0062\\
0.4715	0.0124\\
0.4733	0.0186\\
0.4752	0.0248\\
0.477	0.0309\\
0.4795	0.0391\\
0.482	0.0472\\
0.4844	0.0552\\
0.4869	0.0632\\
0.4894	0.0711\\
0.4918	0.079\\
0.4943	0.0868\\
0.4993	0.1022\\
0.5017	0.1098\\
0.5042	0.1174\\
0.5067	0.1249\\
0.5091	0.1324\\
0.5116	0.1397\\
0.5141	0.1471\\
0.5166	0.1543\\
0.519	0.1615\\
0.5215	0.1687\\
0.524	0.1757\\
0.5264	0.1828\\
0.5339	0.2035\\
0.5363	0.2103\\
0.5388	0.217\\
0.5413	0.2236\\
0.5437	0.2302\\
0.5462	0.2368\\
0.5487	0.2433\\
0.5512	0.2497\\
0.5536	0.2561\\
0.5561	0.2624\\
0.5587	0.269\\
0.5614	0.2756\\
0.564	0.2822\\
0.5667	0.2886\\
};
\addplot [color=black, forget plot]
  table[row sep=crcr]{%
0.5667	0.2886\\
0.5714	0.3\\
0.5761	0.3112\\
0.5808	0.3221\\
0.5856	0.3329\\
0.5903	0.3434\\
0.595	0.3537\\
0.5997	0.3637\\
0.6044	0.3736\\
0.6092	0.3832\\
0.6139	0.3927\\
0.6186	0.4019\\
0.6233	0.4108\\
0.6281	0.4196\\
0.6328	0.4281\\
0.6375	0.4365\\
0.6422	0.4446\\
0.6469	0.4525\\
0.6517	0.4601\\
0.6564	0.4676\\
0.6611	0.4748\\
0.6658	0.4818\\
0.6706	0.4886\\
0.6753	0.4952\\
0.68	0.5015\\
0.6847	0.5077\\
0.6894	0.5136\\
0.6942	0.5193\\
0.6989	0.5248\\
0.7036	0.53\\
0.7083	0.5351\\
0.7131	0.5399\\
0.7178	0.5445\\
0.7225	0.5489\\
0.7272	0.5531\\
0.7319	0.557\\
0.7367	0.5607\\
0.7414	0.5642\\
0.7461	0.5675\\
0.7508	0.5706\\
0.7556	0.5735\\
};
\addplot [color=black, forget plot]
  table[row sep=crcr]{%
0.7556	0.5735\\
0.7585	0.5752\\
0.7615	0.5768\\
0.7645	0.5783\\
0.7675	0.5797\\
0.7722	0.5818\\
0.7769	0.5836\\
0.7816	0.5853\\
0.7864	0.5867\\
0.7911	0.5879\\
0.7958	0.5889\\
0.8005	0.5897\\
0.8052	0.5902\\
0.81	0.5906\\
0.8124	0.5907\\
0.8148	0.5907\\
};
\addplot [color=black, forget plot]
  table[row sep=crcr]{%
0.8148	0.5907\\
0.8174	0.5907\\
0.818	0.5906\\
0.8212	0.5905\\
0.8276	0.5899\\
0.8308	0.5894\\
0.8341	0.5889\\
0.8405	0.5875\\
0.8438	0.5866\\
0.847	0.5856\\
0.8503	0.5845\\
0.8535	0.5834\\
0.8567	0.5821\\
0.86	0.5807\\
0.8632	0.5792\\
0.8665	0.5776\\
0.8697	0.5759\\
0.8729	0.5741\\
0.8762	0.5722\\
0.8794	0.5702\\
0.8827	0.5681\\
0.8859	0.5659\\
0.8892	0.5636\\
0.8924	0.5612\\
0.8956	0.5587\\
0.8989	0.556\\
0.9021	0.5533\\
0.9054	0.5505\\
0.9086	0.5476\\
0.9118	0.5445\\
0.9151	0.5414\\
0.9183	0.5382\\
0.9216	0.5348\\
0.9248	0.5314\\
0.928	0.5278\\
0.9313	0.5242\\
0.9345	0.5204\\
0.937	0.5175\\
0.9395	0.5145\\
0.942	0.5114\\
0.9444	0.5083\\
};
\addplot [color=black, forget plot]
  table[row sep=crcr]{%
0.9444	0.5083\\
0.9492	0.5022\\
0.9539	0.4958\\
0.9586	0.4893\\
0.9633	0.4825\\
0.9681	0.4755\\
0.9728	0.4683\\
0.9775	0.4609\\
0.9822	0.4533\\
0.9869	0.4454\\
0.9917	0.4373\\
0.9964	0.429\\
1.0011	0.4205\\
1.0058	0.4117\\
1.0106	0.4028\\
1.0153	0.3936\\
1.02	0.3842\\
1.0247	0.3746\\
1.0294	0.3648\\
1.0342	0.3547\\
1.0389	0.3444\\
1.0436	0.334\\
1.0483	0.3232\\
1.0531	0.3123\\
1.0578	0.3012\\
1.0625	0.2898\\
1.0672	0.2782\\
1.0719	0.2664\\
1.0767	0.2544\\
1.0814	0.2422\\
1.0861	0.2297\\
1.0908	0.217\\
1.0956	0.2041\\
1.1003	0.191\\
1.105	0.1777\\
1.1097	0.1641\\
1.1144	0.1504\\
1.1192	0.1364\\
1.1239	0.1222\\
1.1286	0.1077\\
1.1333	0.0931\\
};
\addplot [color=black, forget plot]
  table[row sep=crcr]{%
1.1333	0.0931\\
1.1378	0.079\\
1.1393	0.0742\\
1.144	0.0591\\
1.1488	0.0437\\
1.1535	0.0281\\
1.1582	0.0123\\
1.1591	0.0093\\
1.1609	0.0031\\
1.1619	-0\\
};
\addplot [color=black, forget plot]
  table[row sep=crcr]{%
1.1619	0\\
1.1619	0.0002\\
1.162	0.0005\\
1.1621	0.0007\\
1.1622	0.001\\
1.1623	0.0012\\
1.1628	0.0025\\
1.1633	0.0037\\
1.1638	0.005\\
1.1643	0.0062\\
1.1668	0.0124\\
1.1692	0.0185\\
1.1717	0.0246\\
1.1741	0.0306\\
1.1781	0.0403\\
1.1821	0.0498\\
1.1862	0.0592\\
1.1902	0.0684\\
1.1942	0.0774\\
1.1982	0.0863\\
1.2022	0.095\\
1.2062	0.1036\\
1.2102	0.112\\
1.2142	0.1203\\
1.2182	0.1284\\
1.2222	0.1363\\
1.2263	0.1441\\
1.2303	0.1517\\
1.2343	0.1592\\
1.2383	0.1665\\
1.2423	0.1736\\
1.2463	0.1806\\
1.2503	0.1875\\
1.2583	0.2007\\
1.2623	0.207\\
1.2663	0.2132\\
1.2704	0.2193\\
1.2744	0.2252\\
1.2784	0.2309\\
1.2824	0.2365\\
1.2864	0.2419\\
1.2904	0.2472\\
1.2944	0.2523\\
1.2984	0.2572\\
1.3024	0.262\\
1.3064	0.2666\\
1.3104	0.2711\\
1.3145	0.2754\\
1.3185	0.2796\\
1.3194	0.2805\\
1.3203	0.2815\\
1.3213	0.2824\\
1.3222	0.2833\\
};
\addplot [color=black, forget plot]
  table[row sep=crcr]{%
1.3222	0.2833\\
1.3269	0.2878\\
1.3317	0.2921\\
1.3364	0.2962\\
1.3411	0.3001\\
1.3458	0.3037\\
1.3506	0.3071\\
1.3553	0.3104\\
1.36	0.3133\\
1.3647	0.3161\\
1.3694	0.3187\\
1.3742	0.321\\
1.3789	0.3231\\
1.3836	0.325\\
1.3883	0.3267\\
1.3931	0.3281\\
1.3978	0.3294\\
1.4025	0.3304\\
1.4072	0.3312\\
1.4119	0.3318\\
1.4167	0.3321\\
1.418	0.3322\\
1.4194	0.3322\\
1.4208	0.3323\\
1.4221	0.3323\\
};
\addplot [color=black, forget plot]
  table[row sep=crcr]{%
1.4221	0.3323\\
1.4234	0.3323\\
1.424	0.3322\\
1.4253	0.3322\\
1.4275	0.3321\\
1.4298	0.332\\
1.432	0.3318\\
1.4342	0.3315\\
1.4364	0.3313\\
1.4387	0.3309\\
1.4431	0.3301\\
1.4453	0.3296\\
1.4476	0.3291\\
1.452	0.3279\\
1.4542	0.3272\\
1.4565	0.3265\\
1.4609	0.3249\\
1.4631	0.324\\
1.4654	0.3231\\
1.4698	0.3211\\
1.472	0.32\\
1.4743	0.3189\\
1.4765	0.3178\\
1.4787	0.3166\\
1.4809	0.3153\\
1.4832	0.314\\
1.4898	0.3098\\
1.4921	0.3083\\
1.4987	0.3035\\
1.501	0.3018\\
1.5054	0.2982\\
1.5068	0.2971\\
1.5083	0.2959\\
1.5097	0.2947\\
1.5111	0.2934\\
};
\addplot [color=black, forget plot]
  table[row sep=crcr]{%
1.5111	0.2934\\
1.5156	0.2894\\
1.5201	0.2852\\
1.5245	0.2808\\
1.529	0.2762\\
1.5337	0.2712\\
1.5384	0.2659\\
1.5432	0.2604\\
1.5479	0.2547\\
1.5526	0.2488\\
1.5573	0.2426\\
1.562	0.2362\\
1.5668	0.2296\\
1.5715	0.2228\\
1.5762	0.2158\\
1.5809	0.2086\\
1.5857	0.2011\\
1.5904	0.1934\\
1.5951	0.1855\\
1.5998	0.1774\\
1.6045	0.169\\
1.6093	0.1605\\
1.614	0.1517\\
1.6187	0.1427\\
1.6234	0.1335\\
1.6282	0.124\\
1.6329	0.1144\\
1.6376	0.1045\\
1.6423	0.0944\\
1.647	0.0841\\
1.6518	0.0736\\
1.6565	0.0628\\
1.6612	0.0519\\
1.6659	0.0407\\
1.6707	0.0293\\
1.6754	0.0177\\
1.6801	0.0058\\
1.6807	0.0044\\
1.6812	0.0029\\
1.6818	0.0015\\
1.6824	-0\\
};
\addplot [color=black, forget plot]
  table[row sep=crcr]{%
1.6824	0\\
1.6824	0.0001\\
1.6825	0.0002\\
1.6826	0.0005\\
1.6828	0.0007\\
1.6829	0.001\\
1.683	0.0012\\
1.6835	0.002\\
1.6839	0.0029\\
1.6843	0.0037\\
1.6848	0.0045\\
1.6852	0.0054\\
1.6857	0.0062\\
1.6861	0.007\\
1.6865	0.0079\\
1.687	0.0087\\
1.6874	0.0095\\
1.6879	0.0103\\
1.6883	0.0111\\
1.6887	0.012\\
1.6892	0.0128\\
1.6896	0.0136\\
1.6901	0.0144\\
1.6909	0.016\\
1.6914	0.0168\\
1.6918	0.0176\\
1.6923	0.0184\\
1.6931	0.02\\
1.6936	0.0208\\
1.694	0.0216\\
1.6945	0.0224\\
1.6953	0.024\\
1.6958	0.0248\\
1.6962	0.0256\\
1.6967	0.0263\\
1.6975	0.0279\\
1.698	0.0287\\
1.6984	0.0294\\
1.6989	0.0302\\
1.6992	0.0307\\
1.6994	0.0312\\
1.7	0.0322\\
};
\addplot [color=black, forget plot]
  table[row sep=crcr]{%
0	0.95\\
0.001	0.95\\
0.002	0.9501\\
0.0051	0.9501\\
};
\addplot [color=black, forget plot]
  table[row sep=crcr]{%
0.0051	0.9501\\
0.0083	0.9501\\
0.0147	0.9497\\
0.0211	0.9489\\
0.0257	0.948\\
0.0303	0.947\\
0.0349	0.9458\\
0.0395	0.9443\\
0.0441	0.9427\\
0.0487	0.9408\\
0.0533	0.9387\\
0.0579	0.9365\\
0.0624	0.934\\
0.067	0.9313\\
0.0716	0.9284\\
0.0762	0.9253\\
0.0808	0.922\\
0.0854	0.9185\\
0.09	0.9148\\
0.0946	0.9108\\
0.0992	0.9067\\
0.1038	0.9023\\
0.1084	0.8978\\
0.113	0.893\\
0.1176	0.8881\\
0.1222	0.8829\\
0.1268	0.8775\\
0.1314	0.8719\\
0.136	0.8661\\
0.1406	0.8601\\
0.1452	0.8539\\
0.1497	0.8475\\
0.1543	0.8409\\
0.1589	0.834\\
0.1635	0.827\\
0.1681	0.8198\\
0.1727	0.8123\\
0.1773	0.8046\\
0.1819	0.7968\\
0.1865	0.7887\\
0.1871	0.7876\\
0.1877	0.7866\\
0.1889	0.7844\\
};
\addplot [color=black, forget plot]
  table[row sep=crcr]{%
0.1889	0.7844\\
0.1936	0.7758\\
0.1983	0.767\\
0.2031	0.7579\\
0.2078	0.7486\\
0.2125	0.7391\\
0.2172	0.7294\\
0.2219	0.7195\\
0.2267	0.7093\\
0.2314	0.699\\
0.2361	0.6884\\
0.2408	0.6775\\
0.2456	0.6665\\
0.2503	0.6553\\
0.255	0.6438\\
0.2597	0.6321\\
0.2644	0.6202\\
0.2692	0.6081\\
0.2739	0.5957\\
0.2786	0.5832\\
0.2833	0.5704\\
0.2881	0.5574\\
0.2928	0.5442\\
0.2975	0.5308\\
0.3022	0.5171\\
0.3069	0.5032\\
0.3117	0.4891\\
0.3164	0.4748\\
0.3211	0.4603\\
0.3258	0.4455\\
0.3306	0.4306\\
0.3353	0.4154\\
0.34	0.4\\
0.3447	0.3844\\
0.3494	0.3685\\
0.3542	0.3525\\
0.3589	0.3362\\
0.3636	0.3197\\
0.3683	0.303\\
0.3731	0.286\\
0.3778	0.2689\\
};
\addplot [color=black, forget plot]
  table[row sep=crcr]{%
0.3778	0.2689\\
0.3815	0.2553\\
0.3852	0.2416\\
0.3926	0.2138\\
0.3973	0.1957\\
0.402	0.1774\\
0.4067	0.1589\\
0.4114	0.1402\\
0.4162	0.1213\\
0.4209	0.1021\\
0.4256	0.0828\\
0.4303	0.0632\\
0.4341	0.0476\\
0.4378	0.0319\\
0.4415	0.016\\
0.4452	0\\
};
\addplot [color=black, forget plot]
  table[row sep=crcr]{%
0.4452	0\\
0.4452	0.0001\\
0.4453	0.0002\\
0.4454	0.0005\\
0.4454	0.0007\\
0.4455	0.001\\
0.4456	0.0012\\
0.446	0.0025\\
0.4464	0.0037\\
0.4468	0.005\\
0.4471	0.0062\\
0.4491	0.0124\\
0.451	0.0186\\
0.453	0.0248\\
0.4549	0.0309\\
0.4579	0.0404\\
0.461	0.0498\\
0.464	0.0591\\
0.467	0.0683\\
0.4701	0.0775\\
0.4731	0.0865\\
0.4762	0.0955\\
0.4822	0.1131\\
0.4853	0.1218\\
0.4883	0.1304\\
0.4913	0.1389\\
0.4944	0.1473\\
0.5004	0.1639\\
0.5035	0.172\\
0.5065	0.1801\\
0.5096	0.188\\
0.5126	0.1959\\
0.5156	0.2037\\
0.5187	0.2114\\
0.5217	0.219\\
0.5247	0.2265\\
0.5278	0.2339\\
0.5338	0.2485\\
0.5369	0.2556\\
0.5399	0.2627\\
0.5429	0.2696\\
0.546	0.2765\\
0.549	0.2833\\
0.5521	0.29\\
0.5551	0.2966\\
0.5581	0.3031\\
0.5612	0.3095\\
0.5642	0.3159\\
0.5648	0.3171\\
0.5654	0.3184\\
0.5661	0.3197\\
0.5667	0.3209\\
};
\addplot [color=black, forget plot]
  table[row sep=crcr]{%
0.5667	0.3209\\
0.5714	0.3305\\
0.5761	0.3398\\
0.5808	0.3489\\
0.5856	0.3578\\
0.5903	0.3665\\
0.595	0.375\\
0.5997	0.3832\\
0.6044	0.3912\\
0.6092	0.3991\\
0.6139	0.4066\\
0.6186	0.414\\
0.6233	0.4212\\
0.6281	0.4281\\
0.6328	0.4348\\
0.6375	0.4413\\
0.6422	0.4476\\
0.6469	0.4536\\
0.6517	0.4595\\
0.6564	0.4651\\
0.6611	0.4705\\
0.6658	0.4757\\
0.6706	0.4806\\
0.6753	0.4854\\
0.68	0.4899\\
0.6847	0.4942\\
0.6894	0.4983\\
0.6942	0.5022\\
0.6989	0.5058\\
0.7036	0.5092\\
0.7083	0.5124\\
0.7131	0.5154\\
0.7178	0.5182\\
0.7225	0.5208\\
0.7272	0.5231\\
0.7319	0.5252\\
0.7367	0.5271\\
0.7414	0.5288\\
0.7461	0.5303\\
0.7508	0.5315\\
0.7556	0.5325\\
};
\addplot [color=black, forget plot]
  table[row sep=crcr]{%
0.7556	0.5325\\
0.7565	0.5327\\
0.7585	0.5331\\
0.7595	0.5332\\
0.7635	0.5338\\
0.7674	0.5341\\
0.7714	0.5344\\
0.7753	0.5344\\
};
\addplot [color=black, forget plot]
  table[row sep=crcr]{%
0.7753	0.5344\\
0.7785	0.5344\\
0.7849	0.534\\
0.7913	0.5332\\
0.7955	0.5324\\
0.7998	0.5315\\
0.804	0.5304\\
0.8082	0.5291\\
0.8124	0.5277\\
0.8167	0.5261\\
0.8251	0.5223\\
0.8294	0.5201\\
0.8336	0.5178\\
0.8378	0.5153\\
0.842	0.5126\\
0.8463	0.5097\\
0.8505	0.5067\\
0.8547	0.5035\\
0.859	0.5001\\
0.8632	0.4966\\
0.8674	0.4928\\
0.8716	0.4889\\
0.8759	0.4848\\
0.8801	0.4806\\
0.8843	0.4761\\
0.8886	0.4715\\
0.8928	0.4667\\
0.897	0.4618\\
0.9012	0.4567\\
0.9055	0.4513\\
0.9097	0.4459\\
0.9139	0.4402\\
0.9182	0.4344\\
0.9266	0.4222\\
0.9311	0.4154\\
0.9355	0.4085\\
0.94	0.4014\\
0.9444	0.3941\\
};
\addplot [color=black, forget plot]
  table[row sep=crcr]{%
0.9444	0.3941\\
0.9492	0.3862\\
0.9539	0.378\\
0.9586	0.3696\\
0.9633	0.361\\
0.9681	0.3522\\
0.9728	0.3432\\
0.9775	0.3339\\
0.9822	0.3244\\
0.9869	0.3148\\
0.9917	0.3048\\
0.9964	0.2947\\
1.0011	0.2844\\
1.0058	0.2738\\
1.0106	0.263\\
1.0153	0.252\\
1.02	0.2408\\
1.0247	0.2293\\
1.0294	0.2177\\
1.0342	0.2058\\
1.0389	0.1937\\
1.0436	0.1814\\
1.0483	0.1688\\
1.0531	0.1561\\
1.0578	0.1431\\
1.0625	0.1299\\
1.0672	0.1165\\
1.0719	0.1028\\
1.0767	0.089\\
1.0814	0.0749\\
1.0861	0.0606\\
1.0908	0.0461\\
1.0956	0.0314\\
1.098	0.0236\\
1.1005	0.0158\\
1.1029	0.0079\\
1.1054	-0\\
};
\addplot [color=black, forget plot]
  table[row sep=crcr]{%
1.1054	0\\
1.1054	0.0001\\
1.1055	0.0002\\
1.1056	0.0005\\
1.1057	0.0007\\
1.1058	0.001\\
1.1059	0.0012\\
1.1064	0.0025\\
1.1069	0.0037\\
1.1074	0.005\\
1.108	0.0062\\
1.1087	0.0079\\
1.1094	0.0095\\
1.1108	0.0129\\
1.1115	0.0145\\
1.1122	0.0162\\
1.1129	0.0178\\
1.1135	0.0195\\
1.1149	0.0227\\
1.1156	0.0244\\
1.1233	0.042\\
1.124	0.0435\\
1.1254	0.0467\\
1.1261	0.0482\\
1.1268	0.0498\\
1.1275	0.0513\\
1.1282	0.0529\\
1.1296	0.0559\\
1.1303	0.0575\\
1.1311	0.0591\\
1.1318	0.0608\\
1.1326	0.0624\\
1.1333	0.064\\
};
\addplot [color=black, forget plot]
  table[row sep=crcr]{%
1.1333	0.064\\
1.1378	0.0736\\
1.1393	0.0767\\
1.144	0.0865\\
1.1487	0.0961\\
1.1535	0.1054\\
1.1582	0.1145\\
1.1629	0.1235\\
1.1676	0.1322\\
1.1724	0.1406\\
1.1771	0.1489\\
1.1818	0.1569\\
1.1865	0.1647\\
1.1912	0.1723\\
1.196	0.1797\\
1.2007	0.1869\\
1.2054	0.1938\\
1.2101	0.2006\\
1.2149	0.2071\\
1.2196	0.2134\\
1.2243	0.2194\\
1.229	0.2253\\
1.2337	0.2309\\
1.2385	0.2363\\
1.2432	0.2415\\
1.2479	0.2465\\
1.2526	0.2513\\
1.2574	0.2558\\
1.2621	0.2601\\
1.2668	0.2642\\
1.2715	0.2681\\
1.2762	0.2718\\
1.281	0.2752\\
1.2857	0.2784\\
1.2904	0.2814\\
1.2951	0.2842\\
1.2999	0.2868\\
1.3046	0.2891\\
1.3093	0.2913\\
1.3125	0.2926\\
1.3158	0.2938\\
1.319	0.295\\
1.3222	0.296\\
};
\addplot [color=black, forget plot]
  table[row sep=crcr]{%
1.3222	0.296\\
1.3238	0.2964\\
1.3253	0.2969\\
1.3269	0.2973\\
1.3284	0.2977\\
1.3331	0.2987\\
1.3378	0.2995\\
1.3426	0.3001\\
1.3473	0.3005\\
1.3487	0.3005\\
1.3501	0.3006\\
1.353	0.3006\\
};
\addplot [color=black, forget plot]
  table[row sep=crcr]{%
1.353	0.3006\\
1.3562	0.3006\\
1.3626	0.3002\\
1.369	0.2994\\
1.3729	0.2987\\
1.3769	0.2978\\
1.3808	0.2968\\
1.3848	0.2957\\
1.3887	0.2944\\
1.3927	0.2929\\
1.3966	0.2913\\
1.4006	0.2895\\
1.4045	0.2876\\
1.4085	0.2855\\
1.4125	0.2833\\
1.4164	0.2809\\
1.4204	0.2783\\
1.4243	0.2757\\
1.4283	0.2728\\
1.4322	0.2698\\
1.4362	0.2667\\
1.4401	0.2634\\
1.4441	0.2599\\
1.448	0.2563\\
1.452	0.2525\\
1.4559	0.2486\\
1.4599	0.2445\\
1.4639	0.2403\\
1.4678	0.2359\\
1.4718	0.2314\\
1.4757	0.2267\\
1.4797	0.2219\\
1.4836	0.2169\\
1.4876	0.2118\\
1.4915	0.2065\\
1.4955	0.201\\
1.4994	0.1955\\
1.5033	0.1898\\
1.5111	0.178\\
};
\addplot [color=black, forget plot]
  table[row sep=crcr]{%
1.5111	0.178\\
1.5158	0.1705\\
1.5206	0.1629\\
1.5253	0.155\\
1.53	0.1469\\
1.5347	0.1386\\
1.5394	0.1301\\
1.5442	0.1213\\
1.5489	0.1123\\
1.5536	0.1032\\
1.5583	0.0938\\
1.5631	0.0841\\
1.5678	0.0743\\
1.5725	0.0642\\
1.5772	0.0539\\
1.5819	0.0435\\
1.5867	0.0327\\
1.5901	0.0247\\
1.5936	0.0166\\
1.5971	0.0084\\
1.6005	-0\\
};
\addplot [color=black, forget plot]
  table[row sep=crcr]{%
1.6005	0\\
1.6006	0.0001\\
1.6006	0.0002\\
1.6008	0.0005\\
1.6009	0.0007\\
1.6011	0.001\\
1.6012	0.0012\\
1.6019	0.0025\\
1.6033	0.0049\\
1.604	0.0062\\
1.6064	0.0106\\
1.6139	0.0235\\
1.6164	0.0276\\
1.6189	0.0318\\
1.6214	0.0358\\
1.6238	0.0398\\
1.6288	0.0476\\
1.6338	0.0552\\
1.6363	0.0588\\
1.6388	0.0625\\
1.6413	0.066\\
1.6437	0.0695\\
1.6462	0.073\\
1.6487	0.0764\\
1.6537	0.083\\
1.6562	0.0862\\
1.6587	0.0893\\
1.6611	0.0924\\
1.6661	0.0984\\
1.6686	0.1013\\
1.6736	0.1069\\
1.6786	0.1123\\
1.681	0.1149\\
1.686	0.1199\\
1.6885	0.1223\\
1.6935	0.1269\\
1.6951	0.1284\\
1.6967	0.1298\\
1.6984	0.1313\\
1.7	0.1326\\
};
\addplot [color=black, forget plot]
  table[row sep=crcr]{%
0	0.95\\
0.0032	0.95\\
};
\addplot [color=black, forget plot]
  table[row sep=crcr]{%
0.0032	0.95\\
0.0064	0.95\\
0.0128	0.9496\\
0.0192	0.9488\\
0.0238	0.948\\
0.0285	0.9469\\
0.0331	0.9457\\
0.0377	0.9442\\
0.0424	0.9425\\
0.047	0.9406\\
0.0517	0.9385\\
0.0563	0.9362\\
0.061	0.9337\\
0.0656	0.9309\\
0.0702	0.928\\
0.0749	0.9248\\
0.0795	0.9215\\
0.0842	0.9179\\
0.0888	0.9141\\
0.0935	0.9101\\
0.0981	0.9058\\
0.1027	0.9014\\
0.1074	0.8968\\
0.112	0.8919\\
0.1167	0.8869\\
0.1213	0.8816\\
0.126	0.8761\\
0.1306	0.8704\\
0.1352	0.8645\\
0.1399	0.8584\\
0.1445	0.852\\
0.1492	0.8455\\
0.1538	0.8387\\
0.1585	0.8318\\
0.1631	0.8246\\
0.1677	0.8172\\
0.1724	0.8096\\
0.177	0.8018\\
0.1817	0.7938\\
0.1863	0.7855\\
0.187	0.7844\\
0.1882	0.782\\
0.1889	0.7809\\
};
\addplot [color=black, forget plot]
  table[row sep=crcr]{%
0.1889	0.7809\\
0.1936	0.7722\\
0.1983	0.7632\\
0.2031	0.7541\\
0.2078	0.7447\\
0.2125	0.7351\\
0.2172	0.7253\\
0.2219	0.7153\\
0.2267	0.705\\
0.2314	0.6946\\
0.2361	0.6839\\
0.2408	0.673\\
0.2456	0.6619\\
0.2503	0.6505\\
0.255	0.639\\
0.2597	0.6272\\
0.2644	0.6152\\
0.2692	0.603\\
0.2739	0.5906\\
0.2786	0.5779\\
0.2833	0.5651\\
0.2881	0.552\\
0.2928	0.5387\\
0.2975	0.5251\\
0.3022	0.5114\\
0.3069	0.4974\\
0.3117	0.4832\\
0.3164	0.4688\\
0.3211	0.4542\\
0.3258	0.4394\\
0.3306	0.4243\\
0.3353	0.4091\\
0.34	0.3936\\
0.3447	0.3779\\
0.3494	0.3619\\
0.3542	0.3458\\
0.3589	0.3294\\
0.3636	0.3128\\
0.3683	0.296\\
0.3731	0.279\\
0.3778	0.2617\\
};
\addplot [color=black, forget plot]
  table[row sep=crcr]{%
0.3778	0.2617\\
0.3814	0.2485\\
0.3849	0.2352\\
0.3885	0.2217\\
0.3921	0.2081\\
0.3968	0.19\\
0.4015	0.1717\\
0.4063	0.1531\\
0.411	0.1343\\
0.4157	0.1153\\
0.4204	0.0961\\
0.4251	0.0767\\
0.4299	0.057\\
0.4332	0.0429\\
0.4366	0.0287\\
0.4399	0.0144\\
0.4433	0\\
};
\addplot [color=black, forget plot]
  table[row sep=crcr]{%
0.4433	0\\
0.4433	0.0002\\
0.4434	0.0005\\
0.4435	0.0007\\
0.4436	0.001\\
0.4436	0.0012\\
0.444	0.0025\\
0.4444	0.0037\\
0.4448	0.005\\
0.4452	0.0062\\
0.4471	0.0124\\
0.4491	0.0186\\
0.451	0.0248\\
0.453	0.0309\\
0.456	0.0405\\
0.4591	0.0501\\
0.4622	0.0596\\
0.4684	0.0782\\
0.4715	0.0874\\
0.4745	0.0965\\
0.4776	0.1055\\
0.4807	0.1144\\
0.4838	0.1232\\
0.4869	0.1319\\
0.49	0.1405\\
0.4931	0.149\\
0.4961	0.1575\\
0.5023	0.1741\\
0.5085	0.1903\\
0.5116	0.1983\\
0.5147	0.2061\\
0.5177	0.2139\\
0.5208	0.2216\\
0.5239	0.2292\\
0.527	0.2367\\
0.5301	0.2441\\
0.5332	0.2514\\
0.5362	0.2586\\
0.5393	0.2658\\
0.5455	0.2798\\
0.5517	0.2934\\
0.5548	0.3\\
0.5578	0.3066\\
0.5609	0.3131\\
0.564	0.3194\\
0.5647	0.3208\\
0.5653	0.3222\\
0.566	0.3235\\
0.5667	0.3249\\
};
\addplot [color=black, forget plot]
  table[row sep=crcr]{%
0.5667	0.3249\\
0.5714	0.3343\\
0.5761	0.3436\\
0.5808	0.3526\\
0.5856	0.3614\\
0.5903	0.37\\
0.595	0.3784\\
0.5997	0.3865\\
0.6044	0.3945\\
0.6092	0.4022\\
0.6139	0.4097\\
0.6186	0.417\\
0.6233	0.424\\
0.6281	0.4309\\
0.6328	0.4375\\
0.6375	0.4439\\
0.6422	0.4501\\
0.6469	0.456\\
0.6517	0.4618\\
0.6564	0.4673\\
0.6611	0.4726\\
0.6658	0.4777\\
0.6706	0.4826\\
0.6753	0.4872\\
0.68	0.4917\\
0.6847	0.4959\\
0.6894	0.4999\\
0.6942	0.5036\\
0.6989	0.5072\\
0.7036	0.5105\\
0.7083	0.5137\\
0.7131	0.5166\\
0.7178	0.5193\\
0.7225	0.5217\\
0.7272	0.524\\
0.7319	0.526\\
0.7367	0.5278\\
0.7414	0.5294\\
0.7461	0.5308\\
0.7508	0.5319\\
0.7556	0.5328\\
};
\addplot [color=black, forget plot]
  table[row sep=crcr]{%
0.7556	0.5328\\
0.7564	0.533\\
0.7573	0.5331\\
0.7582	0.5333\\
0.7627	0.5338\\
0.7662	0.5342\\
0.7734	0.5344\\
};
\addplot [color=black, forget plot]
  table[row sep=crcr]{%
0.7734	0.5344\\
0.7765	0.5344\\
0.7829	0.534\\
0.7861	0.5336\\
0.7894	0.5331\\
0.7936	0.5324\\
0.7979	0.5314\\
0.8022	0.5303\\
0.8065	0.529\\
0.8107	0.5275\\
0.815	0.5259\\
0.8193	0.5241\\
0.8236	0.522\\
0.8278	0.5198\\
0.8321	0.5175\\
0.8364	0.5149\\
0.8407	0.5122\\
0.845	0.5093\\
0.8492	0.5062\\
0.8535	0.5029\\
0.8578	0.4994\\
0.8621	0.4958\\
0.8663	0.492\\
0.8706	0.488\\
0.8749	0.4838\\
0.8792	0.4795\\
0.8835	0.4749\\
0.8877	0.4702\\
0.892	0.4653\\
0.8963	0.4603\\
0.9006	0.455\\
0.9048	0.4496\\
0.9091	0.444\\
0.9134	0.4382\\
0.9177	0.4322\\
0.9219	0.4261\\
0.9262	0.4198\\
0.9308	0.4128\\
0.9353	0.4057\\
0.9399	0.3984\\
0.9444	0.3908\\
};
\addplot [color=black, forget plot]
  table[row sep=crcr]{%
0.9444	0.3908\\
0.9492	0.3828\\
0.9539	0.3745\\
0.9586	0.3661\\
0.9633	0.3574\\
0.9681	0.3485\\
0.9728	0.3393\\
0.9775	0.33\\
0.9822	0.3204\\
0.9869	0.3106\\
0.9917	0.3006\\
0.9964	0.2904\\
1.0011	0.28\\
1.0058	0.2693\\
1.0106	0.2584\\
1.0153	0.2473\\
1.02	0.236\\
1.0247	0.2245\\
1.0294	0.2127\\
1.0342	0.2007\\
1.0389	0.1886\\
1.0436	0.1761\\
1.0483	0.1635\\
1.0531	0.1507\\
1.0578	0.1376\\
1.0625	0.1243\\
1.0672	0.1108\\
1.0719	0.0971\\
1.0767	0.0831\\
1.0814	0.069\\
1.0861	0.0546\\
1.0908	0.04\\
1.0956	0.0252\\
1.0975	0.0189\\
1.0995	0.0127\\
1.1015	0.0064\\
1.1034	-0\\
};
\addplot [color=black, forget plot]
  table[row sep=crcr]{%
1.1034	0\\
1.1034	0.0001\\
1.1035	0.0001\\
1.1035	0.0002\\
1.1036	0.0005\\
1.1037	0.0007\\
1.1038	0.001\\
1.1039	0.0012\\
1.1044	0.0025\\
1.105	0.0037\\
1.1055	0.005\\
1.106	0.0062\\
1.1067	0.008\\
1.1075	0.0098\\
1.1082	0.0116\\
1.109	0.0133\\
1.1097	0.0151\\
1.1105	0.0169\\
1.1112	0.0186\\
1.112	0.0204\\
1.1127	0.0221\\
1.1135	0.0239\\
1.1142	0.0256\\
1.115	0.0274\\
1.1157	0.0291\\
1.1165	0.0308\\
1.1172	0.0325\\
1.118	0.0342\\
1.1187	0.036\\
1.1195	0.0377\\
1.1202	0.0393\\
1.1209	0.041\\
1.1217	0.0427\\
1.1224	0.0444\\
1.1232	0.0461\\
1.1239	0.0477\\
1.1247	0.0494\\
1.1254	0.0511\\
1.1262	0.0527\\
1.1269	0.0544\\
1.1277	0.056\\
1.1284	0.0576\\
1.1292	0.0593\\
1.1299	0.0609\\
1.1307	0.0625\\
1.1314	0.0641\\
1.1322	0.0657\\
1.1329	0.0673\\
1.133	0.0676\\
1.1333	0.0682\\
};
\addplot [color=black, forget plot]
  table[row sep=crcr]{%
1.1333	0.0682\\
1.1349	0.0717\\
1.1365	0.075\\
1.1382	0.0784\\
1.1398	0.0817\\
1.1445	0.0914\\
1.1492	0.1009\\
1.1539	0.1101\\
1.1586	0.1191\\
1.1634	0.1279\\
1.1681	0.1365\\
1.1728	0.1449\\
1.1775	0.153\\
1.1823	0.161\\
1.187	0.1687\\
1.1917	0.1762\\
1.1964	0.1834\\
1.2011	0.1905\\
1.2059	0.1973\\
1.2106	0.2039\\
1.2153	0.2103\\
1.22	0.2165\\
1.2248	0.2224\\
1.2295	0.2282\\
1.2342	0.2337\\
1.2389	0.239\\
1.2436	0.2441\\
1.2531	0.2536\\
1.2578	0.258\\
1.2625	0.2622\\
1.2673	0.2662\\
1.272	0.27\\
1.2767	0.2735\\
1.2814	0.2769\\
1.2861	0.28\\
1.2909	0.2829\\
1.2956	0.2855\\
1.3003	0.288\\
1.305	0.2902\\
1.3098	0.2923\\
1.3129	0.2935\\
1.316	0.2946\\
1.3191	0.2956\\
1.3222	0.2965\\
};
\addplot [color=black, forget plot]
  table[row sep=crcr]{%
1.3222	0.2965\\
1.3237	0.2969\\
1.3251	0.2973\\
1.3266	0.2977\\
1.328	0.298\\
1.3327	0.299\\
1.3374	0.2997\\
1.3422	0.3002\\
1.3469	0.3005\\
1.3479	0.3006\\
1.351	0.3006\\
};
\addplot [color=black, forget plot]
  table[row sep=crcr]{%
1.351	0.3006\\
1.3542	0.3006\\
1.3574	0.3004\\
1.3638	0.2998\\
1.367	0.2993\\
1.371	0.2986\\
1.375	0.2978\\
1.379	0.2968\\
1.383	0.2956\\
1.387	0.2942\\
1.391	0.2927\\
1.395	0.2911\\
1.399	0.2893\\
1.403	0.2873\\
1.407	0.2852\\
1.411	0.2829\\
1.415	0.2805\\
1.419	0.2779\\
1.423	0.2751\\
1.427	0.2722\\
1.431	0.2692\\
1.439	0.2626\\
1.443	0.259\\
1.447	0.2553\\
1.451	0.2515\\
1.4551	0.2475\\
1.4591	0.2433\\
1.4631	0.239\\
1.4671	0.2345\\
1.4711	0.2299\\
1.4751	0.2251\\
1.4791	0.2201\\
1.4831	0.215\\
1.4871	0.2098\\
1.4951	0.1988\\
1.4991	0.193\\
1.5031	0.1871\\
1.5071	0.1811\\
1.5111	0.1748\\
};
\addplot [color=black, forget plot]
  table[row sep=crcr]{%
1.5111	0.1748\\
1.5158	0.1673\\
1.5206	0.1596\\
1.5253	0.1516\\
1.53	0.1434\\
1.5347	0.135\\
1.5394	0.1264\\
1.5442	0.1176\\
1.5489	0.1085\\
1.5536	0.0992\\
1.5583	0.0897\\
1.5631	0.08\\
1.5678	0.0701\\
1.5725	0.0599\\
1.5772	0.0495\\
1.5819	0.039\\
1.5867	0.0282\\
1.5896	0.0212\\
1.5956	0.0072\\
1.5985	-0\\
};
\addplot [color=black, forget plot]
  table[row sep=crcr]{%
1.5985	0\\
1.5986	0.0001\\
1.5986	0.0002\\
1.5987	0.0002\\
1.5988	0.0005\\
1.5989	0.0007\\
1.5991	0.001\\
1.5992	0.0012\\
1.5999	0.0025\\
1.6013	0.0049\\
1.602	0.0062\\
1.6045	0.0107\\
1.607	0.0151\\
1.6096	0.0195\\
1.6146	0.0281\\
1.6172	0.0322\\
1.6197	0.0364\\
1.6223	0.0404\\
1.6273	0.0484\\
1.6299	0.0522\\
1.6349	0.0598\\
1.6375	0.0635\\
1.64	0.0671\\
1.6425	0.0706\\
1.6451	0.0741\\
1.6476	0.0776\\
1.6502	0.0809\\
1.6552	0.0875\\
1.6578	0.0907\\
1.6603	0.0938\\
1.6628	0.0968\\
1.6654	0.0998\\
1.6704	0.1056\\
1.673	0.1084\\
1.6755	0.1111\\
1.6781	0.1138\\
1.6831	0.119\\
1.6857	0.1215\\
1.6882	0.1239\\
1.6907	0.1262\\
1.6933	0.1285\\
1.695	0.13\\
1.6966	0.1315\\
1.7	0.1343\\
};
\addplot [color=black, forget plot]
  table[row sep=crcr]{%
0	0.95\\
0.0008	0.95\\
0.001	0.9499\\
0.0023	0.9499\\
0.0036	0.9498\\
0.0049	0.9496\\
0.0061	0.9495\\
0.0109	0.9489\\
0.0156	0.948\\
0.0203	0.947\\
0.025	0.9457\\
0.0298	0.9442\\
0.0345	0.9424\\
0.0392	0.9405\\
0.0439	0.9383\\
0.0486	0.936\\
0.0534	0.9334\\
0.0581	0.9305\\
0.0628	0.9275\\
0.0675	0.9243\\
0.0723	0.9208\\
0.077	0.9171\\
0.0817	0.9132\\
0.0864	0.909\\
0.0911	0.9047\\
0.0959	0.9001\\
0.1006	0.8953\\
0.1053	0.8903\\
0.11	0.8851\\
0.1148	0.8797\\
0.1195	0.874\\
0.1242	0.8681\\
0.1289	0.862\\
0.1336	0.8557\\
0.1384	0.8492\\
0.1431	0.8424\\
0.1478	0.8354\\
0.1525	0.8283\\
0.1573	0.8208\\
0.162	0.8132\\
0.1667	0.8054\\
0.1714	0.7973\\
0.1761	0.789\\
0.1793	0.7833\\
0.1825	0.7775\\
0.1857	0.7716\\
0.1889	0.7656\\
};
\addplot [color=black, forget plot]
  table[row sep=crcr]{%
0.1889	0.7656\\
0.1936	0.7565\\
0.1983	0.7471\\
0.2031	0.7376\\
0.2078	0.7279\\
0.2125	0.7179\\
0.2172	0.7077\\
0.2219	0.6973\\
0.2267	0.6867\\
0.2314	0.6758\\
0.2361	0.6647\\
0.2408	0.6535\\
0.2456	0.642\\
0.2503	0.6302\\
0.255	0.6183\\
0.2597	0.6061\\
0.2644	0.5938\\
0.2692	0.5812\\
0.2739	0.5684\\
0.2786	0.5553\\
0.2833	0.5421\\
0.2881	0.5286\\
0.2928	0.5149\\
0.2975	0.501\\
0.3022	0.4869\\
0.3069	0.4725\\
0.3117	0.458\\
0.3164	0.4432\\
0.3211	0.4282\\
0.3258	0.413\\
0.3306	0.3975\\
0.3353	0.3819\\
0.34	0.366\\
0.3447	0.3499\\
0.3494	0.3336\\
0.3542	0.317\\
0.3589	0.3003\\
0.3636	0.2833\\
0.3683	0.2661\\
0.3731	0.2487\\
0.3778	0.2311\\
};
\addplot [color=black, forget plot]
  table[row sep=crcr]{%
0.3778	0.2311\\
0.384	0.2077\\
0.3871	0.1958\\
0.3901	0.1839\\
0.3949	0.1655\\
0.3996	0.1468\\
0.4043	0.128\\
0.409	0.1089\\
0.4138	0.0896\\
0.4185	0.0701\\
0.4232	0.0504\\
0.4279	0.0304\\
0.4297	0.0229\\
0.4315	0.0153\\
0.4332	0.0077\\
0.435	-0\\
};
\addplot [color=black, forget plot]
  table[row sep=crcr]{%
0.435	0\\
0.435	0.0001\\
0.4351	0.0001\\
0.4351	0.0002\\
0.4352	0.0005\\
0.4352	0.0007\\
0.4353	0.001\\
0.4354	0.0012\\
0.4358	0.0025\\
0.4362	0.0037\\
0.4366	0.005\\
0.4369	0.0062\\
0.4389	0.0124\\
0.4408	0.0186\\
0.4428	0.0248\\
0.4447	0.0309\\
0.448	0.0412\\
0.4513	0.0514\\
0.4579	0.0714\\
0.4612	0.0813\\
0.4645	0.091\\
0.4677	0.1007\\
0.4743	0.1197\\
0.4809	0.1383\\
0.4842	0.1474\\
0.4875	0.1564\\
0.4908	0.1653\\
0.4941	0.1741\\
0.4974	0.1828\\
0.5007	0.1914\\
0.5039	0.1999\\
0.5072	0.2083\\
0.5138	0.2247\\
0.5171	0.2328\\
0.5237	0.2486\\
0.527	0.2563\\
0.5336	0.2715\\
0.5369	0.2789\\
0.5401	0.2862\\
0.5434	0.2934\\
0.5467	0.3005\\
0.55	0.3075\\
0.5533	0.3144\\
0.5566	0.3212\\
0.5599	0.3279\\
0.5632	0.3344\\
0.5641	0.3362\\
0.5649	0.3379\\
0.5667	0.3413\\
};
\addplot [color=black, forget plot]
  table[row sep=crcr]{%
0.5667	0.3413\\
0.5714	0.3504\\
0.5761	0.3592\\
0.5808	0.3679\\
0.5856	0.3763\\
0.5903	0.3845\\
0.595	0.3925\\
0.5997	0.4003\\
0.6044	0.4078\\
0.6092	0.4152\\
0.6139	0.4223\\
0.6186	0.4292\\
0.6233	0.4358\\
0.6281	0.4423\\
0.6328	0.4485\\
0.6375	0.4546\\
0.6422	0.4604\\
0.6469	0.466\\
0.6517	0.4713\\
0.6564	0.4765\\
0.6611	0.4814\\
0.6658	0.4861\\
0.6706	0.4906\\
0.6753	0.4949\\
0.68	0.4989\\
0.6847	0.5027\\
0.6894	0.5064\\
0.6942	0.5098\\
0.6989	0.5129\\
0.7036	0.5159\\
0.7083	0.5186\\
0.7131	0.5212\\
0.7178	0.5235\\
0.7225	0.5255\\
0.7272	0.5274\\
0.7319	0.5291\\
0.7367	0.5305\\
0.7414	0.5317\\
0.7461	0.5327\\
0.7508	0.5334\\
0.7556	0.534\\
};
\addplot [color=black, forget plot]
  table[row sep=crcr]{%
0.7556	0.534\\
0.756	0.534\\
0.7565	0.5341\\
0.757	0.5341\\
0.7575	0.5342\\
0.7613	0.5344\\
0.7651	0.5344\\
};
\addplot [color=black, forget plot]
  table[row sep=crcr]{%
0.7651	0.5344\\
0.7683	0.5344\\
0.7747	0.534\\
0.7811	0.5332\\
0.7856	0.5324\\
0.7901	0.5314\\
0.7946	0.5302\\
0.799	0.5288\\
0.8035	0.5272\\
0.808	0.5254\\
0.8125	0.5234\\
0.817	0.5213\\
0.8215	0.5189\\
0.8259	0.5163\\
0.8304	0.5135\\
0.8349	0.5106\\
0.8394	0.5074\\
0.8439	0.504\\
0.8484	0.5005\\
0.8528	0.4967\\
0.8573	0.4927\\
0.8618	0.4886\\
0.8663	0.4842\\
0.8708	0.4797\\
0.8753	0.4749\\
0.8797	0.47\\
0.8842	0.4649\\
0.8887	0.4595\\
0.8932	0.454\\
0.8977	0.4483\\
0.9022	0.4423\\
0.9066	0.4362\\
0.9111	0.4299\\
0.9156	0.4234\\
0.9201	0.4166\\
0.9246	0.4097\\
0.9291	0.4026\\
0.9335	0.3953\\
0.938	0.3878\\
0.9425	0.3801\\
0.943	0.3792\\
0.944	0.3776\\
0.9444	0.3767\\
};
\addplot [color=black, forget plot]
  table[row sep=crcr]{%
0.9444	0.3767\\
0.9492	0.3683\\
0.9539	0.3597\\
0.9586	0.3508\\
0.9633	0.3417\\
0.9681	0.3324\\
0.9728	0.3229\\
0.9775	0.3132\\
0.9822	0.3032\\
0.9869	0.2931\\
0.9917	0.2827\\
0.9964	0.2721\\
1.0011	0.2613\\
1.0058	0.2502\\
1.0106	0.239\\
1.0153	0.2275\\
1.02	0.2158\\
1.0247	0.2039\\
1.0294	0.1917\\
1.0342	0.1794\\
1.0389	0.1668\\
1.0436	0.154\\
1.0483	0.141\\
1.0531	0.1278\\
1.0578	0.1143\\
1.0625	0.1007\\
1.0672	0.0868\\
1.0767	0.0583\\
1.0813	0.0441\\
1.0859	0.0296\\
1.0906	0.0149\\
1.0952	-0\\
};
\addplot [color=black, forget plot]
  table[row sep=crcr]{%
1.0952	0\\
1.0952	0.0001\\
1.0953	0.0002\\
1.0954	0.0005\\
1.0955	0.0007\\
1.0956	0.001\\
1.0957	0.0012\\
1.0962	0.0025\\
1.0967	0.0037\\
1.0973	0.005\\
1.0978	0.0062\\
1.0987	0.0085\\
1.0997	0.0108\\
1.1006	0.013\\
1.1016	0.0153\\
1.1025	0.0175\\
1.1035	0.0198\\
1.1044	0.022\\
1.1054	0.0242\\
1.1063	0.0265\\
1.1083	0.0309\\
1.1092	0.033\\
1.1102	0.0352\\
1.1111	0.0374\\
1.1121	0.0396\\
1.113	0.0417\\
1.114	0.0439\\
1.1149	0.046\\
1.1159	0.0481\\
1.1168	0.0502\\
1.1178	0.0523\\
1.1187	0.0544\\
1.1197	0.0565\\
1.1206	0.0586\\
1.1226	0.0628\\
1.1235	0.0648\\
1.1245	0.0669\\
1.1254	0.0689\\
1.1264	0.0709\\
1.1273	0.0729\\
1.1283	0.0749\\
1.1292	0.0769\\
1.1302	0.0789\\
1.1311	0.0809\\
1.1324	0.0835\\
1.1327	0.0842\\
1.133	0.0848\\
1.1333	0.0855\\
};
\addplot [color=black, forget plot]
  table[row sep=crcr]{%
1.1333	0.0855\\
1.1354	0.0897\\
1.1375	0.094\\
1.1396	0.0982\\
1.1417	0.1023\\
1.1464	0.1115\\
1.1511	0.1205\\
1.1559	0.1293\\
1.1606	0.1378\\
1.1653	0.1461\\
1.17	0.1543\\
1.1747	0.1621\\
1.1795	0.1698\\
1.1842	0.1773\\
1.1889	0.1845\\
1.1936	0.1915\\
1.1984	0.1983\\
1.2031	0.2049\\
1.2078	0.2113\\
1.2125	0.2174\\
1.2172	0.2233\\
1.222	0.229\\
1.2267	0.2345\\
1.2314	0.2398\\
1.2361	0.2449\\
1.2409	0.2497\\
1.2456	0.2543\\
1.2503	0.2587\\
1.255	0.2629\\
1.2597	0.2668\\
1.2645	0.2706\\
1.2692	0.2741\\
1.2739	0.2774\\
1.2786	0.2805\\
1.2834	0.2833\\
1.2881	0.286\\
1.2928	0.2884\\
1.2975	0.2906\\
1.3022	0.2926\\
1.307	0.2943\\
1.3117	0.2959\\
1.3143	0.2967\\
1.317	0.2974\\
1.3222	0.2986\\
};
\addplot [color=black, forget plot]
  table[row sep=crcr]{%
1.3222	0.2986\\
1.3233	0.2988\\
1.3243	0.299\\
1.3253	0.2991\\
1.3264	0.2993\\
1.3305	0.2999\\
1.3346	0.3003\\
1.3387	0.3005\\
1.3428	0.3006\\
};
\addplot [color=black, forget plot]
  table[row sep=crcr]{%
1.3428	0.3006\\
1.346	0.3006\\
1.3524	0.3002\\
1.3588	0.2994\\
1.363	0.2986\\
1.3672	0.2977\\
1.3714	0.2966\\
1.3756	0.2953\\
1.3798	0.2939\\
1.384	0.2923\\
1.3882	0.2905\\
1.3924	0.2885\\
1.3966	0.2864\\
1.4009	0.2841\\
1.4051	0.2816\\
1.4093	0.2789\\
1.4135	0.2761\\
1.4177	0.2731\\
1.4219	0.2699\\
1.4261	0.2666\\
1.4303	0.263\\
1.4345	0.2593\\
1.4387	0.2555\\
1.4429	0.2514\\
1.4471	0.2472\\
1.4514	0.2428\\
1.4556	0.2382\\
1.4598	0.2335\\
1.464	0.2286\\
1.4682	0.2235\\
1.4724	0.2182\\
1.4766	0.2128\\
1.4808	0.2072\\
1.485	0.2014\\
1.4892	0.1954\\
1.4934	0.1893\\
1.4979	0.1827\\
1.5023	0.1758\\
1.5067	0.1688\\
1.5111	0.1616\\
};
\addplot [color=black, forget plot]
  table[row sep=crcr]{%
1.5111	0.1616\\
1.5158	0.1537\\
1.5206	0.1456\\
1.5253	0.1372\\
1.53	0.1287\\
1.5347	0.1199\\
1.5394	0.1109\\
1.5442	0.1017\\
1.5489	0.0922\\
1.5536	0.0826\\
1.5583	0.0727\\
1.5631	0.0626\\
1.5678	0.0523\\
1.5725	0.0418\\
1.5772	0.031\\
1.5819	0.02\\
1.5867	0.0089\\
1.5876	0.0067\\
1.5885	0.0044\\
1.5903	-0\\
};
\addplot [color=black, forget plot]
  table[row sep=crcr]{%
1.5903	0\\
1.5904	0.0001\\
1.5904	0.0002\\
1.5906	0.0005\\
1.5907	0.0007\\
1.5909	0.001\\
1.591	0.0012\\
1.5917	0.0025\\
1.5931	0.0049\\
1.5938	0.0062\\
1.5992	0.0158\\
1.602	0.0205\\
1.6047	0.0252\\
1.6075	0.0298\\
1.6102	0.0343\\
1.6129	0.0387\\
1.6157	0.043\\
1.6184	0.0473\\
1.6212	0.0515\\
1.6239	0.0556\\
1.6267	0.0597\\
1.6294	0.0637\\
1.6321	0.0676\\
1.6349	0.0714\\
1.6376	0.0752\\
1.6404	0.0788\\
1.6431	0.0825\\
1.6458	0.086\\
1.6486	0.0895\\
1.6513	0.0928\\
1.6541	0.0962\\
1.6568	0.0994\\
1.6596	0.1026\\
1.6623	0.1057\\
1.665	0.1087\\
1.6678	0.1116\\
1.6705	0.1145\\
1.6787	0.1227\\
1.6815	0.1253\\
1.6842	0.1278\\
1.687	0.1302\\
1.6897	0.1326\\
1.6925	0.1349\\
1.6943	0.1364\\
1.6962	0.1379\\
1.6981	0.1393\\
1.7	0.1408\\
};
\addplot [color=black, forget plot]
  table[row sep=crcr]{%
0	1.05\\
0.0008	1.05\\
0.001	1.0499\\
0.0023	1.0499\\
0.0036	1.0498\\
0.0049	1.0496\\
0.0061	1.0495\\
0.0109	1.0489\\
0.0156	1.048\\
0.0203	1.047\\
0.025	1.0457\\
0.0298	1.0442\\
0.0345	1.0424\\
0.0392	1.0405\\
0.0439	1.0383\\
0.0486	1.036\\
0.0534	1.0334\\
0.0581	1.0305\\
0.0628	1.0275\\
0.0675	1.0243\\
0.0723	1.0208\\
0.077	1.0171\\
0.0817	1.0132\\
0.0864	1.009\\
0.0911	1.0047\\
0.0959	1.0001\\
0.1006	0.9953\\
0.1053	0.9903\\
0.11	0.9851\\
0.1148	0.9797\\
0.1195	0.974\\
0.1242	0.9681\\
0.1289	0.962\\
0.1336	0.9557\\
0.1384	0.9492\\
0.1431	0.9424\\
0.1478	0.9354\\
0.1525	0.9283\\
0.1573	0.9208\\
0.162	0.9132\\
0.1667	0.9054\\
0.1714	0.8973\\
0.1761	0.889\\
0.1793	0.8833\\
0.1825	0.8775\\
0.1857	0.8716\\
0.1889	0.8656\\
};
\addplot [color=black, forget plot]
  table[row sep=crcr]{%
0.1889	0.8656\\
0.1936	0.8565\\
0.1983	0.8471\\
0.2031	0.8376\\
0.2078	0.8279\\
0.2125	0.8179\\
0.2172	0.8077\\
0.2219	0.7973\\
0.2267	0.7867\\
0.2314	0.7758\\
0.2361	0.7647\\
0.2408	0.7535\\
0.2456	0.742\\
0.2503	0.7302\\
0.255	0.7183\\
0.2597	0.7061\\
0.2644	0.6938\\
0.2692	0.6812\\
0.2739	0.6684\\
0.2786	0.6553\\
0.2833	0.6421\\
0.2881	0.6286\\
0.2928	0.6149\\
0.2975	0.601\\
0.3022	0.5869\\
0.3069	0.5725\\
0.3117	0.558\\
0.3164	0.5432\\
0.3211	0.5282\\
0.3258	0.513\\
0.3306	0.4975\\
0.3353	0.4819\\
0.34	0.466\\
0.3447	0.4499\\
0.3494	0.4336\\
0.3542	0.417\\
0.3589	0.4003\\
0.3636	0.3833\\
0.3683	0.3661\\
0.3731	0.3487\\
0.3778	0.3311\\
};
\addplot [color=black, forget plot]
  table[row sep=crcr]{%
0.3778	0.3311\\
0.3822	0.3144\\
0.3866	0.2974\\
0.3911	0.2803\\
0.3955	0.263\\
0.4002	0.2444\\
0.4049	0.2255\\
0.4097	0.2064\\
0.4144	0.187\\
0.4191	0.1675\\
0.4238	0.1477\\
0.4285	0.1278\\
0.4333	0.1076\\
0.438	0.0871\\
0.4427	0.0665\\
0.4474	0.0457\\
0.4522	0.0246\\
0.4535	0.0185\\
0.4549	0.0123\\
0.4562	0.0062\\
0.4576	-0\\
};
\addplot [color=black, forget plot]
  table[row sep=crcr]{%
0.4576	0\\
0.4576	0.0002\\
0.4577	0.0002\\
0.4577	0.0005\\
0.4578	0.0007\\
0.4579	0.001\\
0.458	0.0012\\
0.4583	0.0025\\
0.4587	0.0037\\
0.4591	0.005\\
0.4594	0.0062\\
0.4613	0.0124\\
0.4631	0.0186\\
0.465	0.0248\\
0.4668	0.0309\\
0.4695	0.0399\\
0.4723	0.0489\\
0.4777	0.0665\\
0.4804	0.0752\\
0.4832	0.0838\\
0.4859	0.0924\\
0.4886	0.1009\\
0.4914	0.1093\\
0.4968	0.1259\\
0.4995	0.1341\\
0.5023	0.1422\\
0.505	0.1503\\
0.5077	0.1583\\
0.5104	0.1662\\
0.5132	0.174\\
0.5159	0.1818\\
0.5186	0.1894\\
0.5213	0.1971\\
0.5241	0.2046\\
0.5268	0.2121\\
0.5295	0.2195\\
0.5323	0.2268\\
0.5377	0.2412\\
0.5404	0.2483\\
0.5432	0.2553\\
0.5459	0.2623\\
0.5486	0.2692\\
0.5513	0.276\\
0.5541	0.2827\\
0.5568	0.2894\\
0.5595	0.296\\
0.5622	0.3025\\
0.565	0.309\\
0.5658	0.311\\
0.5662	0.3119\\
0.5667	0.3129\\
};
\addplot [color=black, forget plot]
  table[row sep=crcr]{%
0.5667	0.3129\\
0.5714	0.3239\\
0.5761	0.3345\\
0.5808	0.345\\
0.5856	0.3553\\
0.5903	0.3653\\
0.595	0.3751\\
0.5997	0.3847\\
0.6044	0.3941\\
0.6092	0.4033\\
0.6139	0.4122\\
0.6186	0.421\\
0.6233	0.4295\\
0.6281	0.4378\\
0.6328	0.4458\\
0.6375	0.4537\\
0.6422	0.4613\\
0.6469	0.4687\\
0.6517	0.4759\\
0.6564	0.4829\\
0.6611	0.4897\\
0.6658	0.4962\\
0.6706	0.5025\\
0.6753	0.5086\\
0.68	0.5145\\
0.6847	0.5202\\
0.6894	0.5256\\
0.6942	0.5308\\
0.6989	0.5359\\
0.7036	0.5406\\
0.7083	0.5452\\
0.7131	0.5496\\
0.7178	0.5537\\
0.7225	0.5576\\
0.7272	0.5613\\
0.7319	0.5648\\
0.7367	0.568\\
0.7414	0.5711\\
0.7461	0.5739\\
0.7508	0.5765\\
0.7556	0.5789\\
};
\addplot [color=black, forget plot]
  table[row sep=crcr]{%
0.7556	0.5789\\
0.758	0.58\\
0.763	0.5822\\
0.7654	0.5832\\
0.7701	0.5849\\
0.7749	0.5863\\
0.7796	0.5876\\
0.7843	0.5887\\
0.789	0.5895\\
0.7938	0.5901\\
0.7985	0.5905\\
0.8032	0.5907\\
0.8046	0.5907\\
};
\addplot [color=black, forget plot]
  table[row sep=crcr]{%
0.8046	0.5907\\
0.8072	0.5907\\
0.8078	0.5906\\
0.811	0.5905\\
0.8174	0.5899\\
0.8206	0.5894\\
0.8241	0.5888\\
0.8276	0.5881\\
0.8311	0.5873\\
0.8346	0.5863\\
0.8381	0.5852\\
0.8416	0.584\\
0.8451	0.5827\\
0.8521	0.5797\\
0.8556	0.578\\
0.8591	0.5762\\
0.8661	0.5722\\
0.8696	0.57\\
0.8731	0.5677\\
0.8766	0.5653\\
0.8801	0.5628\\
0.8835	0.5602\\
0.887	0.5574\\
0.8905	0.5545\\
0.894	0.5515\\
0.8975	0.5484\\
0.9045	0.5418\\
0.908	0.5383\\
0.9115	0.5347\\
0.9185	0.5271\\
0.9255	0.5191\\
0.9325	0.5105\\
0.9355	0.5067\\
0.9415	0.4989\\
0.9444	0.4948\\
};
\addplot [color=black, forget plot]
  table[row sep=crcr]{%
0.9444	0.4948\\
0.9492	0.4882\\
0.9539	0.4814\\
0.9586	0.4744\\
0.9633	0.4672\\
0.9681	0.4597\\
0.9728	0.452\\
0.9775	0.4441\\
0.9822	0.436\\
0.9869	0.4277\\
0.9917	0.4191\\
0.9964	0.4103\\
1.0011	0.4013\\
1.0058	0.3921\\
1.0106	0.3827\\
1.0153	0.3731\\
1.02	0.3632\\
1.0247	0.3531\\
1.0294	0.3428\\
1.0342	0.3323\\
1.0389	0.3215\\
1.0436	0.3106\\
1.0483	0.2994\\
1.0531	0.288\\
1.0578	0.2764\\
1.0625	0.2645\\
1.0672	0.2525\\
1.0719	0.2402\\
1.0767	0.2277\\
1.0814	0.215\\
1.0861	0.2021\\
1.0908	0.1889\\
1.0956	0.1756\\
1.1003	0.162\\
1.105	0.1482\\
1.1097	0.1341\\
1.1144	0.1199\\
1.1192	0.1054\\
1.1239	0.0908\\
1.1286	0.0759\\
1.1333	0.0607\\
};
\addplot [color=black, forget plot]
  table[row sep=crcr]{%
1.1333	0.0607\\
1.1343	0.0577\\
1.1352	0.0546\\
1.1362	0.0515\\
1.1371	0.0485\\
1.1408	0.0365\\
1.1444	0.0245\\
1.148	0.0123\\
1.1517	0\\
};
\addplot [color=black, forget plot]
  table[row sep=crcr]{%
1.1517	0\\
1.1517	0.0002\\
1.1518	0.0005\\
1.1519	0.0007\\
1.152	0.001\\
1.1521	0.0012\\
1.1526	0.0025\\
1.1531	0.0037\\
1.1536	0.005\\
1.1541	0.0062\\
1.1566	0.0124\\
1.159	0.0185\\
1.1615	0.0246\\
1.1639	0.0306\\
1.1682	0.0409\\
1.1725	0.051\\
1.1767	0.0609\\
1.181	0.0707\\
1.1853	0.0802\\
1.1895	0.0896\\
1.1938	0.0989\\
1.198	0.1079\\
1.2023	0.1167\\
1.2066	0.1254\\
1.2108	0.1339\\
1.2151	0.1423\\
1.2194	0.1504\\
1.2236	0.1584\\
1.2322	0.1738\\
1.2364	0.1812\\
1.2407	0.1884\\
1.245	0.1955\\
1.2492	0.2024\\
1.2535	0.2091\\
1.2577	0.2157\\
1.262	0.222\\
1.2663	0.2282\\
1.2705	0.2342\\
1.2748	0.24\\
1.2791	0.2457\\
1.2833	0.2512\\
1.2919	0.2616\\
1.2961	0.2665\\
1.3004	0.2712\\
1.3047	0.2758\\
1.3089	0.2802\\
1.3132	0.2844\\
1.3174	0.2885\\
1.3198	0.2907\\
1.321	0.2917\\
1.3222	0.2928\\
};
\addplot [color=black, forget plot]
  table[row sep=crcr]{%
1.3222	0.2928\\
1.3267	0.2967\\
1.3312	0.3003\\
1.3357	0.3038\\
1.3402	0.3071\\
1.345	0.3103\\
1.3497	0.3133\\
1.3544	0.316\\
1.3591	0.3186\\
1.3639	0.3209\\
1.3686	0.3231\\
1.3733	0.3249\\
1.378	0.3266\\
1.3827	0.3281\\
1.3875	0.3293\\
1.3922	0.3304\\
1.3969	0.3312\\
1.4007	0.3316\\
1.4044	0.332\\
1.4082	0.3322\\
1.4119	0.3323\\
};
\addplot [color=black, forget plot]
  table[row sep=crcr]{%
1.4119	0.3323\\
1.4132	0.3323\\
1.4138	0.3322\\
1.4151	0.3322\\
1.4176	0.3321\\
1.4226	0.3317\\
1.425	0.3314\\
1.4275	0.3311\\
1.43	0.3307\\
1.435	0.3297\\
1.4374	0.3291\\
1.4424	0.3277\\
1.4474	0.3261\\
1.4498	0.3252\\
1.4523	0.3243\\
1.4548	0.3233\\
1.4573	0.3222\\
1.4598	0.321\\
1.4622	0.3199\\
1.4672	0.3173\\
1.4722	0.3145\\
1.4746	0.313\\
1.4796	0.3098\\
1.4846	0.3064\\
1.487	0.3046\\
1.492	0.3008\\
1.4945	0.2988\\
1.4969	0.2968\\
1.5019	0.2926\\
1.5044	0.2903\\
1.5061	0.2888\\
1.5077	0.2872\\
1.5111	0.284\\
};
\addplot [color=black, forget plot]
  table[row sep=crcr]{%
1.5111	0.284\\
1.5158	0.2793\\
1.5206	0.2744\\
1.5253	0.2692\\
1.53	0.2639\\
1.5347	0.2583\\
1.5394	0.2525\\
1.5442	0.2465\\
1.5489	0.2403\\
1.5536	0.2338\\
1.5583	0.2271\\
1.5631	0.2202\\
1.5678	0.2131\\
1.5725	0.2058\\
1.5772	0.1983\\
1.5819	0.1905\\
1.5867	0.1825\\
1.5914	0.1743\\
1.5961	0.1659\\
1.6008	0.1572\\
1.6056	0.1484\\
1.6103	0.1393\\
1.615	0.13\\
1.6197	0.1205\\
1.6244	0.1107\\
1.6292	0.1008\\
1.6339	0.0906\\
1.6386	0.0802\\
1.6433	0.0696\\
1.6481	0.0588\\
1.6528	0.0477\\
1.6575	0.0365\\
1.6622	0.025\\
1.6672	0.0126\\
1.6722	-0\\
};
\addplot [color=black, forget plot]
  table[row sep=crcr]{%
1.6722	0\\
1.6722	0.0001\\
1.6723	0.0001\\
1.6723	0.0002\\
1.6724	0.0005\\
1.6726	0.0007\\
1.6727	0.001\\
1.6728	0.0012\\
1.6735	0.0025\\
1.6741	0.0037\\
1.6748	0.0049\\
1.6755	0.0062\\
1.6761	0.0075\\
1.6789	0.0127\\
1.6796	0.0139\\
1.6817	0.0178\\
1.6824	0.019\\
1.6831	0.0203\\
1.6838	0.0215\\
1.6845	0.0228\\
1.6852	0.024\\
1.6859	0.0253\\
1.6873	0.0277\\
1.688	0.029\\
1.6894	0.0314\\
1.69	0.0326\\
1.6949	0.041\\
1.6956	0.0421\\
1.697	0.0445\\
1.6977	0.0456\\
1.6983	0.0466\\
1.6988	0.0475\\
1.6994	0.0485\\
1.7	0.0494\\
};
\addplot [color=black, forget plot]
  table[row sep=crcr]{%
0	1.05\\
0.001	1.05\\
0.002	1.0501\\
0.0051	1.0501\\
};
\addplot [color=black, forget plot]
  table[row sep=crcr]{%
0.0051	1.0501\\
0.0083	1.0501\\
0.0147	1.0497\\
0.0211	1.0489\\
0.0257	1.048\\
0.0303	1.047\\
0.0349	1.0458\\
0.0395	1.0443\\
0.0441	1.0427\\
0.0487	1.0408\\
0.0533	1.0387\\
0.0579	1.0365\\
0.0624	1.034\\
0.067	1.0313\\
0.0716	1.0284\\
0.0762	1.0253\\
0.0808	1.022\\
0.0854	1.0185\\
0.09	1.0148\\
0.0946	1.0108\\
0.0992	1.0067\\
0.1038	1.0023\\
0.1084	0.9978\\
0.113	0.993\\
0.1176	0.9881\\
0.1222	0.9829\\
0.1268	0.9775\\
0.1314	0.9719\\
0.136	0.9661\\
0.1406	0.9601\\
0.1452	0.9539\\
0.1497	0.9475\\
0.1543	0.9409\\
0.1589	0.934\\
0.1635	0.927\\
0.1681	0.9198\\
0.1727	0.9123\\
0.1773	0.9046\\
0.1819	0.8968\\
0.1865	0.8887\\
0.1871	0.8876\\
0.1877	0.8866\\
0.1889	0.8844\\
};
\addplot [color=black, forget plot]
  table[row sep=crcr]{%
0.1889	0.8844\\
0.1936	0.8758\\
0.1983	0.867\\
0.2031	0.8579\\
0.2078	0.8486\\
0.2125	0.8391\\
0.2172	0.8294\\
0.2219	0.8195\\
0.2267	0.8093\\
0.2314	0.799\\
0.2361	0.7884\\
0.2408	0.7775\\
0.2456	0.7665\\
0.2503	0.7553\\
0.255	0.7438\\
0.2597	0.7321\\
0.2644	0.7202\\
0.2692	0.7081\\
0.2739	0.6957\\
0.2786	0.6832\\
0.2833	0.6704\\
0.2881	0.6574\\
0.2928	0.6442\\
0.2975	0.6308\\
0.3022	0.6171\\
0.3069	0.6032\\
0.3117	0.5891\\
0.3164	0.5748\\
0.3211	0.5603\\
0.3258	0.5455\\
0.3306	0.5306\\
0.3353	0.5154\\
0.34	0.5\\
0.3447	0.4844\\
0.3494	0.4685\\
0.3542	0.4525\\
0.3589	0.4362\\
0.3636	0.4197\\
0.3683	0.403\\
0.3731	0.386\\
0.3778	0.3689\\
};
\addplot [color=black, forget plot]
  table[row sep=crcr]{%
0.3778	0.3689\\
0.3825	0.3515\\
0.3872	0.3339\\
0.3919	0.3161\\
0.3967	0.2981\\
0.4014	0.2798\\
0.4061	0.2613\\
0.4108	0.2427\\
0.4156	0.2238\\
0.4203	0.2046\\
0.425	0.1853\\
0.4297	0.1657\\
0.4344	0.1459\\
0.4392	0.1259\\
0.4439	0.1057\\
0.4486	0.0853\\
0.4533	0.0646\\
0.4569	0.0487\\
0.4606	0.0326\\
0.4642	0.0164\\
0.4678	-0\\
};
\addplot [color=black, forget plot]
  table[row sep=crcr]{%
0.4678	0\\
0.4678	0.0002\\
0.4679	0.0002\\
0.4679	0.0005\\
0.468	0.0007\\
0.4681	0.001\\
0.4682	0.0012\\
0.4685	0.0025\\
0.4689	0.0037\\
0.4693	0.005\\
0.4696	0.0062\\
0.4715	0.0124\\
0.4733	0.0186\\
0.4752	0.0248\\
0.477	0.0309\\
0.4795	0.0391\\
0.482	0.0472\\
0.4844	0.0552\\
0.4869	0.0632\\
0.4894	0.0711\\
0.4918	0.079\\
0.4943	0.0868\\
0.4993	0.1022\\
0.5017	0.1098\\
0.5042	0.1174\\
0.5067	0.1249\\
0.5091	0.1324\\
0.5116	0.1397\\
0.5141	0.1471\\
0.5166	0.1543\\
0.519	0.1615\\
0.5215	0.1687\\
0.524	0.1757\\
0.5264	0.1828\\
0.5339	0.2035\\
0.5363	0.2103\\
0.5388	0.217\\
0.5413	0.2236\\
0.5437	0.2302\\
0.5462	0.2368\\
0.5487	0.2433\\
0.5512	0.2497\\
0.5536	0.2561\\
0.5561	0.2624\\
0.5587	0.269\\
0.5614	0.2756\\
0.564	0.2822\\
0.5667	0.2886\\
};
\addplot [color=black, forget plot]
  table[row sep=crcr]{%
0.5667	0.2886\\
0.5714	0.3\\
0.5761	0.3112\\
0.5808	0.3221\\
0.5856	0.3329\\
0.5903	0.3434\\
0.595	0.3537\\
0.5997	0.3637\\
0.6044	0.3736\\
0.6092	0.3832\\
0.6139	0.3927\\
0.6186	0.4019\\
0.6233	0.4108\\
0.6281	0.4196\\
0.6328	0.4281\\
0.6375	0.4365\\
0.6422	0.4446\\
0.6469	0.4525\\
0.6517	0.4601\\
0.6564	0.4676\\
0.6611	0.4748\\
0.6658	0.4818\\
0.6706	0.4886\\
0.6753	0.4952\\
0.68	0.5015\\
0.6847	0.5077\\
0.6894	0.5136\\
0.6942	0.5193\\
0.6989	0.5248\\
0.7036	0.53\\
0.7083	0.5351\\
0.7131	0.5399\\
0.7178	0.5445\\
0.7225	0.5489\\
0.7272	0.5531\\
0.7319	0.557\\
0.7367	0.5607\\
0.7414	0.5642\\
0.7461	0.5675\\
0.7508	0.5706\\
0.7556	0.5735\\
};
\addplot [color=black, forget plot]
  table[row sep=crcr]{%
0.7556	0.5735\\
0.7585	0.5752\\
0.7615	0.5768\\
0.7645	0.5783\\
0.7675	0.5797\\
0.7722	0.5818\\
0.7769	0.5836\\
0.7816	0.5853\\
0.7864	0.5867\\
0.7911	0.5879\\
0.7958	0.5889\\
0.8005	0.5897\\
0.8052	0.5902\\
0.81	0.5906\\
0.8124	0.5907\\
0.8148	0.5907\\
};
\addplot [color=black, forget plot]
  table[row sep=crcr]{%
0.8148	0.5907\\
0.8174	0.5907\\
0.818	0.5906\\
0.8212	0.5905\\
0.8276	0.5899\\
0.8308	0.5894\\
0.8341	0.5889\\
0.8405	0.5875\\
0.8438	0.5866\\
0.847	0.5856\\
0.8503	0.5845\\
0.8535	0.5834\\
0.8567	0.5821\\
0.86	0.5807\\
0.8632	0.5792\\
0.8665	0.5776\\
0.8697	0.5759\\
0.8729	0.5741\\
0.8762	0.5722\\
0.8794	0.5702\\
0.8827	0.5681\\
0.8859	0.5659\\
0.8892	0.5636\\
0.8924	0.5612\\
0.8956	0.5587\\
0.8989	0.556\\
0.9021	0.5533\\
0.9054	0.5505\\
0.9086	0.5476\\
0.9118	0.5445\\
0.9151	0.5414\\
0.9183	0.5382\\
0.9216	0.5348\\
0.9248	0.5314\\
0.928	0.5278\\
0.9313	0.5242\\
0.9345	0.5204\\
0.937	0.5175\\
0.9395	0.5145\\
0.942	0.5114\\
0.9444	0.5083\\
};
\addplot [color=black, forget plot]
  table[row sep=crcr]{%
0.9444	0.5083\\
0.9492	0.5022\\
0.9539	0.4958\\
0.9586	0.4893\\
0.9633	0.4825\\
0.9681	0.4755\\
0.9728	0.4683\\
0.9775	0.4609\\
0.9822	0.4533\\
0.9869	0.4454\\
0.9917	0.4373\\
0.9964	0.429\\
1.0011	0.4205\\
1.0058	0.4117\\
1.0106	0.4028\\
1.0153	0.3936\\
1.02	0.3842\\
1.0247	0.3746\\
1.0294	0.3648\\
1.0342	0.3547\\
1.0389	0.3444\\
1.0436	0.334\\
1.0483	0.3232\\
1.0531	0.3123\\
1.0578	0.3012\\
1.0625	0.2898\\
1.0672	0.2782\\
1.0719	0.2664\\
1.0767	0.2544\\
1.0814	0.2422\\
1.0861	0.2297\\
1.0908	0.217\\
1.0956	0.2041\\
1.1003	0.191\\
1.105	0.1777\\
1.1097	0.1641\\
1.1144	0.1504\\
1.1192	0.1364\\
1.1239	0.1222\\
1.1286	0.1077\\
1.1333	0.0931\\
};
\addplot [color=black, forget plot]
  table[row sep=crcr]{%
1.1333	0.0931\\
1.1378	0.079\\
1.1393	0.0742\\
1.144	0.0591\\
1.1488	0.0437\\
1.1535	0.0281\\
1.1582	0.0123\\
1.1591	0.0093\\
1.1609	0.0031\\
1.1619	-0\\
};
\addplot [color=black, forget plot]
  table[row sep=crcr]{%
1.1619	0\\
1.1619	0.0002\\
1.162	0.0005\\
1.1621	0.0007\\
1.1622	0.001\\
1.1623	0.0012\\
1.1628	0.0025\\
1.1633	0.0037\\
1.1638	0.005\\
1.1643	0.0062\\
1.1668	0.0124\\
1.1692	0.0185\\
1.1717	0.0246\\
1.1741	0.0306\\
1.1781	0.0403\\
1.1821	0.0498\\
1.1862	0.0592\\
1.1902	0.0684\\
1.1942	0.0774\\
1.1982	0.0863\\
1.2022	0.095\\
1.2062	0.1036\\
1.2102	0.112\\
1.2142	0.1203\\
1.2182	0.1284\\
1.2222	0.1363\\
1.2263	0.1441\\
1.2303	0.1517\\
1.2343	0.1592\\
1.2383	0.1665\\
1.2423	0.1736\\
1.2463	0.1806\\
1.2503	0.1875\\
1.2583	0.2007\\
1.2623	0.207\\
1.2663	0.2132\\
1.2704	0.2193\\
1.2744	0.2252\\
1.2784	0.2309\\
1.2824	0.2365\\
1.2864	0.2419\\
1.2904	0.2472\\
1.2944	0.2523\\
1.2984	0.2572\\
1.3024	0.262\\
1.3064	0.2666\\
1.3104	0.2711\\
1.3145	0.2754\\
1.3185	0.2796\\
1.3194	0.2805\\
1.3203	0.2815\\
1.3213	0.2824\\
1.3222	0.2833\\
};
\addplot [color=black, forget plot]
  table[row sep=crcr]{%
1.3222	0.2833\\
1.3269	0.2878\\
1.3317	0.2921\\
1.3364	0.2962\\
1.3411	0.3001\\
1.3458	0.3037\\
1.3506	0.3071\\
1.3553	0.3104\\
1.36	0.3133\\
1.3647	0.3161\\
1.3694	0.3187\\
1.3742	0.321\\
1.3789	0.3231\\
1.3836	0.325\\
1.3883	0.3267\\
1.3931	0.3281\\
1.3978	0.3294\\
1.4025	0.3304\\
1.4072	0.3312\\
1.4119	0.3318\\
1.4167	0.3321\\
1.418	0.3322\\
1.4194	0.3322\\
1.4208	0.3323\\
1.4221	0.3323\\
};
\addplot [color=black, forget plot]
  table[row sep=crcr]{%
1.4221	0.3323\\
1.4234	0.3323\\
1.424	0.3322\\
1.4253	0.3322\\
1.4275	0.3321\\
1.4298	0.332\\
1.432	0.3318\\
1.4342	0.3315\\
1.4364	0.3313\\
1.4387	0.3309\\
1.4431	0.3301\\
1.4453	0.3296\\
1.4476	0.3291\\
1.452	0.3279\\
1.4542	0.3272\\
1.4565	0.3265\\
1.4609	0.3249\\
1.4631	0.324\\
1.4654	0.3231\\
1.4698	0.3211\\
1.472	0.32\\
1.4743	0.3189\\
1.4765	0.3178\\
1.4787	0.3166\\
1.4809	0.3153\\
1.4832	0.314\\
1.4898	0.3098\\
1.4921	0.3083\\
1.4987	0.3035\\
1.501	0.3018\\
1.5054	0.2982\\
1.5068	0.2971\\
1.5083	0.2959\\
1.5097	0.2947\\
1.5111	0.2934\\
};
\addplot [color=black, forget plot]
  table[row sep=crcr]{%
1.5111	0.2934\\
1.5156	0.2894\\
1.5201	0.2852\\
1.5245	0.2808\\
1.529	0.2762\\
1.5337	0.2712\\
1.5384	0.2659\\
1.5432	0.2604\\
1.5479	0.2547\\
1.5526	0.2488\\
1.5573	0.2426\\
1.562	0.2362\\
1.5668	0.2296\\
1.5715	0.2228\\
1.5762	0.2158\\
1.5809	0.2086\\
1.5857	0.2011\\
1.5904	0.1934\\
1.5951	0.1855\\
1.5998	0.1774\\
1.6045	0.169\\
1.6093	0.1605\\
1.614	0.1517\\
1.6187	0.1427\\
1.6234	0.1335\\
1.6282	0.124\\
1.6329	0.1144\\
1.6376	0.1045\\
1.6423	0.0944\\
1.647	0.0841\\
1.6518	0.0736\\
1.6565	0.0628\\
1.6612	0.0519\\
1.6659	0.0407\\
1.6707	0.0293\\
1.6754	0.0177\\
1.6801	0.0058\\
1.6807	0.0044\\
1.6812	0.0029\\
1.6818	0.0015\\
1.6824	-0\\
};
\addplot [color=black, forget plot]
  table[row sep=crcr]{%
1.6824	0\\
1.6824	0.0001\\
1.6825	0.0002\\
1.6826	0.0005\\
1.6828	0.0007\\
1.6829	0.001\\
1.683	0.0012\\
1.6835	0.002\\
1.6839	0.0029\\
1.6843	0.0037\\
1.6848	0.0045\\
1.6852	0.0054\\
1.6857	0.0062\\
1.6861	0.007\\
1.6865	0.0079\\
1.687	0.0087\\
1.6874	0.0095\\
1.6879	0.0103\\
1.6883	0.0111\\
1.6887	0.012\\
1.6892	0.0128\\
1.6896	0.0136\\
1.6901	0.0144\\
1.6909	0.016\\
1.6914	0.0168\\
1.6918	0.0176\\
1.6923	0.0184\\
1.6931	0.02\\
1.6936	0.0208\\
1.694	0.0216\\
1.6945	0.0224\\
1.6953	0.024\\
1.6958	0.0248\\
1.6962	0.0256\\
1.6967	0.0263\\
1.6975	0.0279\\
1.698	0.0287\\
1.6984	0.0294\\
1.6989	0.0302\\
1.6992	0.0307\\
1.6994	0.0312\\
1.7	0.0322\\
};
\addplot [color=black, forget plot]
  table[row sep=crcr]{%
0	0.95\\
0.001	0.95\\
0.002	0.9501\\
0.0051	0.9501\\
};
\addplot [color=black, forget plot]
  table[row sep=crcr]{%
0.0051	0.9501\\
0.0083	0.9501\\
0.0147	0.9497\\
0.0211	0.9489\\
0.0257	0.948\\
0.0303	0.947\\
0.0349	0.9458\\
0.0395	0.9443\\
0.0441	0.9427\\
0.0487	0.9408\\
0.0533	0.9387\\
0.0579	0.9365\\
0.0624	0.934\\
0.067	0.9313\\
0.0716	0.9284\\
0.0762	0.9253\\
0.0808	0.922\\
0.0854	0.9185\\
0.09	0.9148\\
0.0946	0.9108\\
0.0992	0.9067\\
0.1038	0.9023\\
0.1084	0.8978\\
0.113	0.893\\
0.1176	0.8881\\
0.1222	0.8829\\
0.1268	0.8775\\
0.1314	0.8719\\
0.136	0.8661\\
0.1406	0.8601\\
0.1452	0.8539\\
0.1497	0.8475\\
0.1543	0.8409\\
0.1589	0.834\\
0.1635	0.827\\
0.1681	0.8198\\
0.1727	0.8123\\
0.1773	0.8046\\
0.1819	0.7968\\
0.1865	0.7887\\
0.1871	0.7876\\
0.1877	0.7866\\
0.1889	0.7844\\
};
\addplot [color=black, forget plot]
  table[row sep=crcr]{%
0.1889	0.7844\\
0.1936	0.7758\\
0.1983	0.767\\
0.2031	0.7579\\
0.2078	0.7486\\
0.2125	0.7391\\
0.2172	0.7294\\
0.2219	0.7195\\
0.2267	0.7093\\
0.2314	0.699\\
0.2361	0.6884\\
0.2408	0.6775\\
0.2456	0.6665\\
0.2503	0.6553\\
0.255	0.6438\\
0.2597	0.6321\\
0.2644	0.6202\\
0.2692	0.6081\\
0.2739	0.5957\\
0.2786	0.5832\\
0.2833	0.5704\\
0.2881	0.5574\\
0.2928	0.5442\\
0.2975	0.5308\\
0.3022	0.5171\\
0.3069	0.5032\\
0.3117	0.4891\\
0.3164	0.4748\\
0.3211	0.4603\\
0.3258	0.4455\\
0.3306	0.4306\\
0.3353	0.4154\\
0.34	0.4\\
0.3447	0.3844\\
0.3494	0.3685\\
0.3542	0.3525\\
0.3589	0.3362\\
0.3636	0.3197\\
0.3683	0.303\\
0.3731	0.286\\
0.3778	0.2689\\
};
\addplot [color=black, forget plot]
  table[row sep=crcr]{%
0.3778	0.2689\\
0.3815	0.2553\\
0.3852	0.2416\\
0.3926	0.2138\\
0.3973	0.1957\\
0.402	0.1774\\
0.4067	0.1589\\
0.4114	0.1402\\
0.4162	0.1213\\
0.4209	0.1021\\
0.4256	0.0828\\
0.4303	0.0632\\
0.4341	0.0476\\
0.4378	0.0319\\
0.4415	0.016\\
0.4452	0\\
};
\addplot [color=black, forget plot]
  table[row sep=crcr]{%
0.4452	0\\
0.4452	0.0001\\
0.4453	0.0002\\
0.4454	0.0005\\
0.4454	0.0007\\
0.4455	0.001\\
0.4456	0.0012\\
0.446	0.0025\\
0.4464	0.0037\\
0.4468	0.005\\
0.4471	0.0062\\
0.4491	0.0124\\
0.451	0.0186\\
0.453	0.0248\\
0.4549	0.0309\\
0.4579	0.0404\\
0.461	0.0498\\
0.464	0.0591\\
0.467	0.0683\\
0.4701	0.0775\\
0.4731	0.0865\\
0.4762	0.0955\\
0.4822	0.1131\\
0.4853	0.1218\\
0.4883	0.1304\\
0.4913	0.1389\\
0.4944	0.1473\\
0.5004	0.1639\\
0.5035	0.172\\
0.5065	0.1801\\
0.5096	0.188\\
0.5126	0.1959\\
0.5156	0.2037\\
0.5187	0.2114\\
0.5217	0.219\\
0.5247	0.2265\\
0.5278	0.2339\\
0.5338	0.2485\\
0.5369	0.2556\\
0.5399	0.2627\\
0.5429	0.2696\\
0.546	0.2765\\
0.549	0.2833\\
0.5521	0.29\\
0.5551	0.2966\\
0.5581	0.3031\\
0.5612	0.3095\\
0.5642	0.3159\\
0.5648	0.3171\\
0.5654	0.3184\\
0.5661	0.3197\\
0.5667	0.3209\\
};
\addplot [color=black, forget plot]
  table[row sep=crcr]{%
0.5667	0.3209\\
0.5714	0.3305\\
0.5761	0.3398\\
0.5808	0.3489\\
0.5856	0.3578\\
0.5903	0.3665\\
0.595	0.375\\
0.5997	0.3832\\
0.6044	0.3912\\
0.6092	0.3991\\
0.6139	0.4066\\
0.6186	0.414\\
0.6233	0.4212\\
0.6281	0.4281\\
0.6328	0.4348\\
0.6375	0.4413\\
0.6422	0.4476\\
0.6469	0.4536\\
0.6517	0.4595\\
0.6564	0.4651\\
0.6611	0.4705\\
0.6658	0.4757\\
0.6706	0.4806\\
0.6753	0.4854\\
0.68	0.4899\\
0.6847	0.4942\\
0.6894	0.4983\\
0.6942	0.5022\\
0.6989	0.5058\\
0.7036	0.5092\\
0.7083	0.5124\\
0.7131	0.5154\\
0.7178	0.5182\\
0.7225	0.5208\\
0.7272	0.5231\\
0.7319	0.5252\\
0.7367	0.5271\\
0.7414	0.5288\\
0.7461	0.5303\\
0.7508	0.5315\\
0.7556	0.5325\\
};
\addplot [color=black, forget plot]
  table[row sep=crcr]{%
0.7556	0.5325\\
0.7565	0.5327\\
0.7585	0.5331\\
0.7595	0.5332\\
0.7635	0.5338\\
0.7674	0.5341\\
0.7714	0.5344\\
0.7753	0.5344\\
};
\addplot [color=black, forget plot]
  table[row sep=crcr]{%
0.7753	0.5344\\
0.7785	0.5344\\
0.7849	0.534\\
0.7913	0.5332\\
0.7955	0.5324\\
0.7998	0.5315\\
0.804	0.5304\\
0.8082	0.5291\\
0.8124	0.5277\\
0.8167	0.5261\\
0.8251	0.5223\\
0.8294	0.5201\\
0.8336	0.5178\\
0.8378	0.5153\\
0.842	0.5126\\
0.8463	0.5097\\
0.8505	0.5067\\
0.8547	0.5035\\
0.859	0.5001\\
0.8632	0.4966\\
0.8674	0.4928\\
0.8716	0.4889\\
0.8759	0.4848\\
0.8801	0.4806\\
0.8843	0.4761\\
0.8886	0.4715\\
0.8928	0.4667\\
0.897	0.4618\\
0.9012	0.4567\\
0.9055	0.4513\\
0.9097	0.4459\\
0.9139	0.4402\\
0.9182	0.4344\\
0.9266	0.4222\\
0.9311	0.4154\\
0.9355	0.4085\\
0.94	0.4014\\
0.9444	0.3941\\
};
\addplot [color=black, forget plot]
  table[row sep=crcr]{%
0.9444	0.3941\\
0.9492	0.3862\\
0.9539	0.378\\
0.9586	0.3696\\
0.9633	0.361\\
0.9681	0.3522\\
0.9728	0.3432\\
0.9775	0.3339\\
0.9822	0.3244\\
0.9869	0.3148\\
0.9917	0.3048\\
0.9964	0.2947\\
1.0011	0.2844\\
1.0058	0.2738\\
1.0106	0.263\\
1.0153	0.252\\
1.02	0.2408\\
1.0247	0.2293\\
1.0294	0.2177\\
1.0342	0.2058\\
1.0389	0.1937\\
1.0436	0.1814\\
1.0483	0.1688\\
1.0531	0.1561\\
1.0578	0.1431\\
1.0625	0.1299\\
1.0672	0.1165\\
1.0719	0.1028\\
1.0767	0.089\\
1.0814	0.0749\\
1.0861	0.0606\\
1.0908	0.0461\\
1.0956	0.0314\\
1.098	0.0236\\
1.1005	0.0158\\
1.1029	0.0079\\
1.1054	-0\\
};
\addplot [color=black, forget plot]
  table[row sep=crcr]{%
1.1054	0\\
1.1054	0.0001\\
1.1055	0.0002\\
1.1056	0.0005\\
1.1057	0.0007\\
1.1058	0.001\\
1.1059	0.0012\\
1.1064	0.0025\\
1.1069	0.0037\\
1.1074	0.005\\
1.108	0.0062\\
1.1087	0.0079\\
1.1094	0.0095\\
1.1108	0.0129\\
1.1115	0.0145\\
1.1122	0.0162\\
1.1129	0.0178\\
1.1135	0.0195\\
1.1149	0.0227\\
1.1156	0.0244\\
1.1233	0.042\\
1.124	0.0435\\
1.1254	0.0467\\
1.1261	0.0482\\
1.1268	0.0498\\
1.1275	0.0513\\
1.1282	0.0529\\
1.1296	0.0559\\
1.1303	0.0575\\
1.1311	0.0591\\
1.1318	0.0608\\
1.1326	0.0624\\
1.1333	0.064\\
};
\addplot [color=black, forget plot]
  table[row sep=crcr]{%
1.1333	0.064\\
1.1378	0.0736\\
1.1393	0.0767\\
1.144	0.0865\\
1.1487	0.0961\\
1.1535	0.1054\\
1.1582	0.1145\\
1.1629	0.1235\\
1.1676	0.1322\\
1.1724	0.1406\\
1.1771	0.1489\\
1.1818	0.1569\\
1.1865	0.1647\\
1.1912	0.1723\\
1.196	0.1797\\
1.2007	0.1869\\
1.2054	0.1938\\
1.2101	0.2006\\
1.2149	0.2071\\
1.2196	0.2134\\
1.2243	0.2194\\
1.229	0.2253\\
1.2337	0.2309\\
1.2385	0.2363\\
1.2432	0.2415\\
1.2479	0.2465\\
1.2526	0.2513\\
1.2574	0.2558\\
1.2621	0.2601\\
1.2668	0.2642\\
1.2715	0.2681\\
1.2762	0.2718\\
1.281	0.2752\\
1.2857	0.2784\\
1.2904	0.2814\\
1.2951	0.2842\\
1.2999	0.2868\\
1.3046	0.2891\\
1.3093	0.2913\\
1.3125	0.2926\\
1.3158	0.2938\\
1.319	0.295\\
1.3222	0.296\\
};
\addplot [color=black, forget plot]
  table[row sep=crcr]{%
1.3222	0.296\\
1.3238	0.2964\\
1.3253	0.2969\\
1.3269	0.2973\\
1.3284	0.2977\\
1.3331	0.2987\\
1.3378	0.2995\\
1.3426	0.3001\\
1.3473	0.3005\\
1.3487	0.3005\\
1.3501	0.3006\\
1.353	0.3006\\
};
\addplot [color=black, forget plot]
  table[row sep=crcr]{%
1.353	0.3006\\
1.3562	0.3006\\
1.3626	0.3002\\
1.369	0.2994\\
1.3729	0.2987\\
1.3769	0.2978\\
1.3808	0.2968\\
1.3848	0.2957\\
1.3887	0.2944\\
1.3927	0.2929\\
1.3966	0.2913\\
1.4006	0.2895\\
1.4045	0.2876\\
1.4085	0.2855\\
1.4125	0.2833\\
1.4164	0.2809\\
1.4204	0.2783\\
1.4243	0.2757\\
1.4283	0.2728\\
1.4322	0.2698\\
1.4362	0.2667\\
1.4401	0.2634\\
1.4441	0.2599\\
1.448	0.2563\\
1.452	0.2525\\
1.4559	0.2486\\
1.4599	0.2445\\
1.4639	0.2403\\
1.4678	0.2359\\
1.4718	0.2314\\
1.4757	0.2267\\
1.4797	0.2219\\
1.4836	0.2169\\
1.4876	0.2118\\
1.4915	0.2065\\
1.4955	0.201\\
1.4994	0.1955\\
1.5033	0.1898\\
1.5111	0.178\\
};
\addplot [color=black, forget plot]
  table[row sep=crcr]{%
1.5111	0.178\\
1.5158	0.1705\\
1.5206	0.1629\\
1.5253	0.155\\
1.53	0.1469\\
1.5347	0.1386\\
1.5394	0.1301\\
1.5442	0.1213\\
1.5489	0.1123\\
1.5536	0.1032\\
1.5583	0.0938\\
1.5631	0.0841\\
1.5678	0.0743\\
1.5725	0.0642\\
1.5772	0.0539\\
1.5819	0.0435\\
1.5867	0.0327\\
1.5901	0.0247\\
1.5936	0.0166\\
1.5971	0.0084\\
1.6005	-0\\
};
\addplot [color=black, forget plot]
  table[row sep=crcr]{%
1.6005	0\\
1.6006	0.0001\\
1.6006	0.0002\\
1.6008	0.0005\\
1.6009	0.0007\\
1.6011	0.001\\
1.6012	0.0012\\
1.6019	0.0025\\
1.6033	0.0049\\
1.604	0.0062\\
1.6064	0.0106\\
1.6139	0.0235\\
1.6164	0.0276\\
1.6189	0.0318\\
1.6214	0.0358\\
1.6238	0.0398\\
1.6288	0.0476\\
1.6338	0.0552\\
1.6363	0.0588\\
1.6388	0.0625\\
1.6413	0.066\\
1.6437	0.0695\\
1.6462	0.073\\
1.6487	0.0764\\
1.6537	0.083\\
1.6562	0.0862\\
1.6587	0.0893\\
1.6611	0.0924\\
1.6661	0.0984\\
1.6686	0.1013\\
1.6736	0.1069\\
1.6786	0.1123\\
1.681	0.1149\\
1.686	0.1199\\
1.6885	0.1223\\
1.6935	0.1269\\
1.6951	0.1284\\
1.6967	0.1298\\
1.6984	0.1313\\
1.7	0.1326\\
};
\addplot [color=black, forget plot]
  table[row sep=crcr]{%
0	0.95\\
0.0032	0.95\\
};
\addplot [color=black, forget plot]
  table[row sep=crcr]{%
0.0032	0.95\\
0.0064	0.95\\
0.0128	0.9496\\
0.0192	0.9488\\
0.0238	0.948\\
0.0285	0.9469\\
0.0331	0.9457\\
0.0377	0.9442\\
0.0424	0.9425\\
0.047	0.9406\\
0.0517	0.9385\\
0.0563	0.9362\\
0.061	0.9337\\
0.0656	0.9309\\
0.0702	0.928\\
0.0749	0.9248\\
0.0795	0.9215\\
0.0842	0.9179\\
0.0888	0.9141\\
0.0935	0.9101\\
0.0981	0.9058\\
0.1027	0.9014\\
0.1074	0.8968\\
0.112	0.8919\\
0.1167	0.8869\\
0.1213	0.8816\\
0.126	0.8761\\
0.1306	0.8704\\
0.1352	0.8645\\
0.1399	0.8584\\
0.1445	0.852\\
0.1492	0.8455\\
0.1538	0.8387\\
0.1585	0.8318\\
0.1631	0.8246\\
0.1677	0.8172\\
0.1724	0.8096\\
0.177	0.8018\\
0.1817	0.7938\\
0.1863	0.7855\\
0.187	0.7844\\
0.1882	0.782\\
0.1889	0.7809\\
};
\addplot [color=black, forget plot]
  table[row sep=crcr]{%
0.1889	0.7809\\
0.1936	0.7722\\
0.1983	0.7632\\
0.2031	0.7541\\
0.2078	0.7447\\
0.2125	0.7351\\
0.2172	0.7253\\
0.2219	0.7153\\
0.2267	0.705\\
0.2314	0.6946\\
0.2361	0.6839\\
0.2408	0.673\\
0.2456	0.6619\\
0.2503	0.6505\\
0.255	0.639\\
0.2597	0.6272\\
0.2644	0.6152\\
0.2692	0.603\\
0.2739	0.5906\\
0.2786	0.5779\\
0.2833	0.5651\\
0.2881	0.552\\
0.2928	0.5387\\
0.2975	0.5251\\
0.3022	0.5114\\
0.3069	0.4974\\
0.3117	0.4832\\
0.3164	0.4688\\
0.3211	0.4542\\
0.3258	0.4394\\
0.3306	0.4243\\
0.3353	0.4091\\
0.34	0.3936\\
0.3447	0.3779\\
0.3494	0.3619\\
0.3542	0.3458\\
0.3589	0.3294\\
0.3636	0.3128\\
0.3683	0.296\\
0.3731	0.279\\
0.3778	0.2617\\
};
\addplot [color=black, forget plot]
  table[row sep=crcr]{%
0.3778	0.2617\\
0.3814	0.2485\\
0.3849	0.2352\\
0.3885	0.2217\\
0.3921	0.2081\\
0.3968	0.19\\
0.4015	0.1717\\
0.4063	0.1531\\
0.411	0.1343\\
0.4157	0.1153\\
0.4204	0.0961\\
0.4251	0.0767\\
0.4299	0.057\\
0.4332	0.0429\\
0.4366	0.0287\\
0.4399	0.0144\\
0.4433	0\\
};
\addplot [color=black, forget plot]
  table[row sep=crcr]{%
0.4433	0\\
0.4433	0.0002\\
0.4434	0.0005\\
0.4435	0.0007\\
0.4436	0.001\\
0.4436	0.0012\\
0.444	0.0025\\
0.4444	0.0037\\
0.4448	0.005\\
0.4452	0.0062\\
0.4471	0.0124\\
0.4491	0.0186\\
0.451	0.0248\\
0.453	0.0309\\
0.456	0.0405\\
0.4591	0.0501\\
0.4622	0.0596\\
0.4684	0.0782\\
0.4715	0.0874\\
0.4745	0.0965\\
0.4776	0.1055\\
0.4807	0.1144\\
0.4838	0.1232\\
0.4869	0.1319\\
0.49	0.1405\\
0.4931	0.149\\
0.4961	0.1575\\
0.5023	0.1741\\
0.5085	0.1903\\
0.5116	0.1983\\
0.5147	0.2061\\
0.5177	0.2139\\
0.5208	0.2216\\
0.5239	0.2292\\
0.527	0.2367\\
0.5301	0.2441\\
0.5332	0.2514\\
0.5362	0.2586\\
0.5393	0.2658\\
0.5455	0.2798\\
0.5517	0.2934\\
0.5548	0.3\\
0.5578	0.3066\\
0.5609	0.3131\\
0.564	0.3194\\
0.5647	0.3208\\
0.5653	0.3222\\
0.566	0.3235\\
0.5667	0.3249\\
};
\addplot [color=black, forget plot]
  table[row sep=crcr]{%
0.5667	0.3249\\
0.5714	0.3343\\
0.5761	0.3436\\
0.5808	0.3526\\
0.5856	0.3614\\
0.5903	0.37\\
0.595	0.3784\\
0.5997	0.3865\\
0.6044	0.3945\\
0.6092	0.4022\\
0.6139	0.4097\\
0.6186	0.417\\
0.6233	0.424\\
0.6281	0.4309\\
0.6328	0.4375\\
0.6375	0.4439\\
0.6422	0.4501\\
0.6469	0.456\\
0.6517	0.4618\\
0.6564	0.4673\\
0.6611	0.4726\\
0.6658	0.4777\\
0.6706	0.4826\\
0.6753	0.4872\\
0.68	0.4917\\
0.6847	0.4959\\
0.6894	0.4999\\
0.6942	0.5036\\
0.6989	0.5072\\
0.7036	0.5105\\
0.7083	0.5137\\
0.7131	0.5166\\
0.7178	0.5193\\
0.7225	0.5217\\
0.7272	0.524\\
0.7319	0.526\\
0.7367	0.5278\\
0.7414	0.5294\\
0.7461	0.5308\\
0.7508	0.5319\\
0.7556	0.5328\\
};
\addplot [color=black, forget plot]
  table[row sep=crcr]{%
0.7556	0.5328\\
0.7564	0.533\\
0.7573	0.5331\\
0.7582	0.5333\\
0.7627	0.5338\\
0.7662	0.5342\\
0.7734	0.5344\\
};
\addplot [color=black, forget plot]
  table[row sep=crcr]{%
0.7734	0.5344\\
0.7765	0.5344\\
0.7829	0.534\\
0.7861	0.5336\\
0.7894	0.5331\\
0.7936	0.5324\\
0.7979	0.5314\\
0.8022	0.5303\\
0.8065	0.529\\
0.8107	0.5275\\
0.815	0.5259\\
0.8193	0.5241\\
0.8236	0.522\\
0.8278	0.5198\\
0.8321	0.5175\\
0.8364	0.5149\\
0.8407	0.5122\\
0.845	0.5093\\
0.8492	0.5062\\
0.8535	0.5029\\
0.8578	0.4994\\
0.8621	0.4958\\
0.8663	0.492\\
0.8706	0.488\\
0.8749	0.4838\\
0.8792	0.4795\\
0.8835	0.4749\\
0.8877	0.4702\\
0.892	0.4653\\
0.8963	0.4603\\
0.9006	0.455\\
0.9048	0.4496\\
0.9091	0.444\\
0.9134	0.4382\\
0.9177	0.4322\\
0.9219	0.4261\\
0.9262	0.4198\\
0.9308	0.4128\\
0.9353	0.4057\\
0.9399	0.3984\\
0.9444	0.3908\\
};
\addplot [color=black, forget plot]
  table[row sep=crcr]{%
0.9444	0.3908\\
0.9492	0.3828\\
0.9539	0.3745\\
0.9586	0.3661\\
0.9633	0.3574\\
0.9681	0.3485\\
0.9728	0.3393\\
0.9775	0.33\\
0.9822	0.3204\\
0.9869	0.3106\\
0.9917	0.3006\\
0.9964	0.2904\\
1.0011	0.28\\
1.0058	0.2693\\
1.0106	0.2584\\
1.0153	0.2473\\
1.02	0.236\\
1.0247	0.2245\\
1.0294	0.2127\\
1.0342	0.2007\\
1.0389	0.1886\\
1.0436	0.1761\\
1.0483	0.1635\\
1.0531	0.1507\\
1.0578	0.1376\\
1.0625	0.1243\\
1.0672	0.1108\\
1.0719	0.0971\\
1.0767	0.0831\\
1.0814	0.069\\
1.0861	0.0546\\
1.0908	0.04\\
1.0956	0.0252\\
1.0975	0.0189\\
1.0995	0.0127\\
1.1015	0.0064\\
1.1034	-0\\
};
\addplot [color=black, forget plot]
  table[row sep=crcr]{%
1.1034	0\\
1.1034	0.0001\\
1.1035	0.0001\\
1.1035	0.0002\\
1.1036	0.0005\\
1.1037	0.0007\\
1.1038	0.001\\
1.1039	0.0012\\
1.1044	0.0025\\
1.105	0.0037\\
1.1055	0.005\\
1.106	0.0062\\
1.1067	0.008\\
1.1075	0.0098\\
1.1082	0.0116\\
1.109	0.0133\\
1.1097	0.0151\\
1.1105	0.0169\\
1.1112	0.0186\\
1.112	0.0204\\
1.1127	0.0221\\
1.1135	0.0239\\
1.1142	0.0256\\
1.115	0.0274\\
1.1157	0.0291\\
1.1165	0.0308\\
1.1172	0.0325\\
1.118	0.0342\\
1.1187	0.036\\
1.1195	0.0377\\
1.1202	0.0393\\
1.1209	0.041\\
1.1217	0.0427\\
1.1224	0.0444\\
1.1232	0.0461\\
1.1239	0.0477\\
1.1247	0.0494\\
1.1254	0.0511\\
1.1262	0.0527\\
1.1269	0.0544\\
1.1277	0.056\\
1.1284	0.0576\\
1.1292	0.0593\\
1.1299	0.0609\\
1.1307	0.0625\\
1.1314	0.0641\\
1.1322	0.0657\\
1.1329	0.0673\\
1.133	0.0676\\
1.1333	0.0682\\
};
\addplot [color=black, forget plot]
  table[row sep=crcr]{%
1.1333	0.0682\\
1.1349	0.0717\\
1.1365	0.075\\
1.1382	0.0784\\
1.1398	0.0817\\
1.1445	0.0914\\
1.1492	0.1009\\
1.1539	0.1101\\
1.1586	0.1191\\
1.1634	0.1279\\
1.1681	0.1365\\
1.1728	0.1449\\
1.1775	0.153\\
1.1823	0.161\\
1.187	0.1687\\
1.1917	0.1762\\
1.1964	0.1834\\
1.2011	0.1905\\
1.2059	0.1973\\
1.2106	0.2039\\
1.2153	0.2103\\
1.22	0.2165\\
1.2248	0.2224\\
1.2295	0.2282\\
1.2342	0.2337\\
1.2389	0.239\\
1.2436	0.2441\\
1.2531	0.2536\\
1.2578	0.258\\
1.2625	0.2622\\
1.2673	0.2662\\
1.272	0.27\\
1.2767	0.2735\\
1.2814	0.2769\\
1.2861	0.28\\
1.2909	0.2829\\
1.2956	0.2855\\
1.3003	0.288\\
1.305	0.2902\\
1.3098	0.2923\\
1.3129	0.2935\\
1.316	0.2946\\
1.3191	0.2956\\
1.3222	0.2965\\
};
\addplot [color=black, forget plot]
  table[row sep=crcr]{%
1.3222	0.2965\\
1.3237	0.2969\\
1.3251	0.2973\\
1.3266	0.2977\\
1.328	0.298\\
1.3327	0.299\\
1.3374	0.2997\\
1.3422	0.3002\\
1.3469	0.3005\\
1.3479	0.3006\\
1.351	0.3006\\
};
\addplot [color=black, forget plot]
  table[row sep=crcr]{%
1.351	0.3006\\
1.3542	0.3006\\
1.3574	0.3004\\
1.3638	0.2998\\
1.367	0.2993\\
1.371	0.2986\\
1.375	0.2978\\
1.379	0.2968\\
1.383	0.2956\\
1.387	0.2942\\
1.391	0.2927\\
1.395	0.2911\\
1.399	0.2893\\
1.403	0.2873\\
1.407	0.2852\\
1.411	0.2829\\
1.415	0.2805\\
1.419	0.2779\\
1.423	0.2751\\
1.427	0.2722\\
1.431	0.2692\\
1.439	0.2626\\
1.443	0.259\\
1.447	0.2553\\
1.451	0.2515\\
1.4551	0.2475\\
1.4591	0.2433\\
1.4631	0.239\\
1.4671	0.2345\\
1.4711	0.2299\\
1.4751	0.2251\\
1.4791	0.2201\\
1.4831	0.215\\
1.4871	0.2098\\
1.4951	0.1988\\
1.4991	0.193\\
1.5031	0.1871\\
1.5071	0.1811\\
1.5111	0.1748\\
};
\addplot [color=black, forget plot]
  table[row sep=crcr]{%
1.5111	0.1748\\
1.5158	0.1673\\
1.5206	0.1596\\
1.5253	0.1516\\
1.53	0.1434\\
1.5347	0.135\\
1.5394	0.1264\\
1.5442	0.1176\\
1.5489	0.1085\\
1.5536	0.0992\\
1.5583	0.0897\\
1.5631	0.08\\
1.5678	0.0701\\
1.5725	0.0599\\
1.5772	0.0495\\
1.5819	0.039\\
1.5867	0.0282\\
1.5896	0.0212\\
1.5956	0.0072\\
1.5985	-0\\
};
\addplot [color=black, forget plot]
  table[row sep=crcr]{%
1.5985	0\\
1.5986	0.0001\\
1.5986	0.0002\\
1.5987	0.0002\\
1.5988	0.0005\\
1.5989	0.0007\\
1.5991	0.001\\
1.5992	0.0012\\
1.5999	0.0025\\
1.6013	0.0049\\
1.602	0.0062\\
1.6045	0.0107\\
1.607	0.0151\\
1.6096	0.0195\\
1.6146	0.0281\\
1.6172	0.0322\\
1.6197	0.0364\\
1.6223	0.0404\\
1.6273	0.0484\\
1.6299	0.0522\\
1.6349	0.0598\\
1.6375	0.0635\\
1.64	0.0671\\
1.6425	0.0706\\
1.6451	0.0741\\
1.6476	0.0776\\
1.6502	0.0809\\
1.6552	0.0875\\
1.6578	0.0907\\
1.6603	0.0938\\
1.6628	0.0968\\
1.6654	0.0998\\
1.6704	0.1056\\
1.673	0.1084\\
1.6755	0.1111\\
1.6781	0.1138\\
1.6831	0.119\\
1.6857	0.1215\\
1.6882	0.1239\\
1.6907	0.1262\\
1.6933	0.1285\\
1.695	0.13\\
1.6966	0.1315\\
1.7	0.1343\\
};
\addplot [color=black, forget plot]
  table[row sep=crcr]{%
0	0.9932\\
0.0008	0.9932\\
0.001	0.9933\\
0.005	0.9933\\
};
\addplot [color=black, forget plot]
  table[row sep=crcr]{%
0.005	0.9933\\
0.0082	0.9933\\
0.0146	0.9929\\
0.021	0.9921\\
0.0256	0.9913\\
0.0302	0.9902\\
0.0348	0.989\\
0.0394	0.9875\\
0.044	0.9859\\
0.0486	0.984\\
0.0532	0.982\\
0.0578	0.9797\\
0.0624	0.9772\\
0.067	0.9745\\
0.0716	0.9716\\
0.0762	0.9685\\
0.0808	0.9652\\
0.0854	0.9617\\
0.09	0.9579\\
0.0946	0.954\\
0.0992	0.9499\\
0.1038	0.9455\\
0.1084	0.941\\
0.113	0.9362\\
0.1176	0.9312\\
0.1222	0.9261\\
0.1267	0.9207\\
0.1313	0.9151\\
0.1359	0.9093\\
0.1405	0.9033\\
0.1451	0.8971\\
0.1497	0.8906\\
0.1543	0.884\\
0.1589	0.8772\\
0.1635	0.8701\\
0.1681	0.8629\\
0.1727	0.8554\\
0.1773	0.8478\\
0.1819	0.8399\\
0.1865	0.8318\\
0.1871	0.8307\\
0.1877	0.8297\\
0.1889	0.8275\\
};
\addplot [color=black, forget plot]
  table[row sep=crcr]{%
0.1889	0.8275\\
0.1936	0.8189\\
0.1983	0.8101\\
0.2031	0.801\\
0.2078	0.7917\\
0.2125	0.7822\\
0.2172	0.7725\\
0.2219	0.7625\\
0.2267	0.7524\\
0.2314	0.742\\
0.2361	0.7314\\
0.2408	0.7206\\
0.2456	0.7096\\
0.2503	0.6983\\
0.255	0.6868\\
0.2597	0.6752\\
0.2644	0.6632\\
0.2692	0.6511\\
0.2739	0.6388\\
0.2786	0.6262\\
0.2833	0.6134\\
0.2881	0.6004\\
0.2928	0.5872\\
0.2975	0.5738\\
0.3022	0.5601\\
0.3069	0.5462\\
0.3117	0.5321\\
0.3164	0.5178\\
0.3211	0.5033\\
0.3258	0.4885\\
0.3306	0.4736\\
0.3353	0.4584\\
0.34	0.443\\
0.3447	0.4273\\
0.3494	0.4115\\
0.3542	0.3954\\
0.3589	0.3791\\
0.3636	0.3626\\
0.3683	0.3459\\
0.3731	0.329\\
0.3778	0.3118\\
};
\addplot [color=black, forget plot]
  table[row sep=crcr]{%
0.3778	0.3118\\
0.3821	0.2961\\
0.3863	0.2801\\
0.3906	0.264\\
0.3949	0.2477\\
0.3996	0.2296\\
0.4044	0.2112\\
0.4091	0.1926\\
0.4138	0.1737\\
0.4185	0.1547\\
0.4232	0.1354\\
0.428	0.1159\\
0.4327	0.0962\\
0.4374	0.0763\\
0.4421	0.0562\\
0.4469	0.0358\\
0.4516	0.0152\\
0.4524	0.0114\\
0.4542	0.0038\\
0.455	-0\\
};
\addplot [color=black, forget plot]
  table[row sep=crcr]{%
0.455	0\\
0.4551	0.0001\\
0.4551	0.0002\\
0.4552	0.0005\\
0.4553	0.0007\\
0.4553	0.001\\
0.4554	0.0012\\
0.4558	0.0025\\
0.4562	0.0037\\
0.4565	0.005\\
0.4569	0.0062\\
0.4626	0.0248\\
0.4645	0.0309\\
0.4673	0.0399\\
0.4729	0.0575\\
0.4785	0.0749\\
0.4813	0.0834\\
0.484	0.0919\\
0.4868	0.1003\\
0.4924	0.1169\\
0.4952	0.1251\\
0.498	0.1332\\
0.5008	0.1412\\
0.5064	0.157\\
0.5092	0.1648\\
0.512	0.1725\\
0.5147	0.1802\\
0.5203	0.1952\\
0.5259	0.21\\
0.5315	0.2244\\
0.5371	0.2386\\
0.5399	0.2455\\
0.5426	0.2524\\
0.5454	0.2592\\
0.551	0.2726\\
0.5538	0.2792\\
0.5566	0.2857\\
0.5594	0.2921\\
0.5654	0.3056\\
0.5658	0.3066\\
0.5662	0.3075\\
0.5667	0.3085\\
};
\addplot [color=black, forget plot]
  table[row sep=crcr]{%
0.5667	0.3085\\
0.5714	0.3188\\
0.5761	0.329\\
0.5808	0.3389\\
0.5856	0.3486\\
0.5903	0.3581\\
0.595	0.3673\\
0.5997	0.3764\\
0.6044	0.3852\\
0.6092	0.3938\\
0.6139	0.4022\\
0.6186	0.4103\\
0.6233	0.4183\\
0.6281	0.426\\
0.6328	0.4335\\
0.6375	0.4408\\
0.6422	0.4479\\
0.6469	0.4547\\
0.6517	0.4614\\
0.6564	0.4678\\
0.6611	0.474\\
0.6658	0.48\\
0.6706	0.4857\\
0.6753	0.4913\\
0.68	0.4966\\
0.6847	0.5017\\
0.6894	0.5066\\
0.6942	0.5113\\
0.6989	0.5157\\
0.7036	0.5199\\
0.7083	0.524\\
0.7131	0.5277\\
0.7178	0.5313\\
0.7225	0.5347\\
0.7272	0.5378\\
0.7319	0.5407\\
0.7367	0.5434\\
0.7414	0.5459\\
0.7461	0.5482\\
0.7508	0.5502\\
0.7556	0.552\\
};
\addplot [color=black, forget plot]
  table[row sep=crcr]{%
0.7556	0.552\\
0.7574	0.5527\\
0.7593	0.5533\\
0.7611	0.5539\\
0.763	0.5545\\
0.7677	0.5557\\
0.7724	0.5568\\
0.7772	0.5576\\
0.7819	0.5582\\
0.7845	0.5584\\
0.7872	0.5586\\
0.7926	0.5588\\
};
\addplot [color=black, forget plot]
  table[row sep=crcr]{%
0.7926	0.5588\\
0.7929	0.5588\\
0.7931	0.5587\\
0.7958	0.5587\\
0.799	0.5586\\
0.8022	0.5583\\
0.8086	0.5575\\
0.8124	0.5568\\
0.8162	0.556\\
0.8199	0.5551\\
0.8237	0.554\\
0.8313	0.5514\\
0.8351	0.5499\\
0.8389	0.5482\\
0.8427	0.5464\\
0.8465	0.5445\\
0.8503	0.5424\\
0.8541	0.5402\\
0.8579	0.5378\\
0.8617	0.5353\\
0.8693	0.5299\\
0.8769	0.5239\\
0.8845	0.5173\\
0.8883	0.5138\\
0.8921	0.5102\\
0.8959	0.5064\\
0.8997	0.5025\\
0.9035	0.4984\\
0.9073	0.4942\\
0.9149	0.4854\\
0.9225	0.476\\
0.9263	0.4711\\
0.9301	0.466\\
0.9337	0.4611\\
0.9373	0.4561\\
0.9408	0.4509\\
0.9444	0.4456\\
};
\addplot [color=black, forget plot]
  table[row sep=crcr]{%
0.9444	0.4456\\
0.9492	0.4384\\
0.9539	0.4311\\
0.9586	0.4235\\
0.9633	0.4157\\
0.9681	0.4077\\
0.9728	0.3994\\
0.9775	0.391\\
0.9822	0.3823\\
0.9869	0.3734\\
0.9917	0.3643\\
0.9964	0.355\\
1.0011	0.3454\\
1.0058	0.3356\\
1.0106	0.3257\\
1.0153	0.3154\\
1.02	0.305\\
1.0247	0.2944\\
1.0294	0.2835\\
1.0342	0.2724\\
1.0389	0.2611\\
1.0436	0.2496\\
1.0483	0.2379\\
1.0531	0.2259\\
1.0578	0.2137\\
1.0625	0.2013\\
1.0672	0.1887\\
1.0719	0.1759\\
1.0767	0.1628\\
1.0814	0.1496\\
1.0861	0.1361\\
1.0908	0.1224\\
1.0956	0.1084\\
1.1003	0.0943\\
1.105	0.0799\\
1.1097	0.0653\\
1.1144	0.0505\\
1.1184	0.0381\\
1.1223	0.0256\\
1.1262	0.0129\\
1.1301	0\\
};
\addplot [color=black, forget plot]
  table[row sep=crcr]{%
1.1301	0\\
1.1301	0.0002\\
1.1302	0.0002\\
1.1302	0.0004\\
1.1306	0.0012\\
1.1306	0.0014\\
1.131	0.0022\\
1.131	0.0024\\
1.1315	0.0034\\
1.1315	0.0036\\
1.1319	0.0044\\
1.1319	0.0046\\
1.1324	0.0056\\
1.1324	0.0058\\
1.1328	0.0066\\
1.1328	0.0068\\
1.1331	0.0074\\
1.1331	0.0076\\
1.1332	0.0077\\
1.1333	0.0079\\
1.1333	0.0081\\
};
\addplot [color=black, forget plot]
  table[row sep=crcr]{%
1.1333	0.0081\\
1.1337	0.0089\\
1.1338	0.0093\\
1.134	0.0097\\
1.1356	0.0137\\
1.1365	0.0157\\
1.1373	0.0177\\
1.1414	0.0275\\
1.1455	0.0372\\
1.1497	0.0468\\
1.1538	0.0561\\
1.1585	0.0667\\
1.1632	0.077\\
1.168	0.087\\
1.1727	0.0969\\
1.1774	0.1065\\
1.1821	0.116\\
1.1868	0.1252\\
1.1916	0.1342\\
1.1963	0.1429\\
1.201	0.1515\\
1.2057	0.1598\\
1.2105	0.1679\\
1.2152	0.1758\\
1.2199	0.1835\\
1.2246	0.1909\\
1.2293	0.1982\\
1.2341	0.2052\\
1.2388	0.212\\
1.2435	0.2186\\
1.2482	0.2249\\
1.253	0.2311\\
1.2577	0.237\\
1.2624	0.2427\\
1.2671	0.2482\\
1.2718	0.2535\\
1.2766	0.2585\\
1.2813	0.2634\\
1.286	0.268\\
1.2907	0.2724\\
1.2955	0.2765\\
1.3002	0.2805\\
1.3049	0.2842\\
1.3092	0.2875\\
1.3136	0.2905\\
1.3179	0.2934\\
1.3222	0.2961\\
};
\addplot [color=black, forget plot]
  table[row sep=crcr]{%
1.3222	0.2961\\
1.3253	0.2978\\
1.3283	0.2995\\
1.3345	0.3027\\
1.3392	0.3048\\
1.3439	0.3067\\
1.3486	0.3084\\
1.3534	0.3099\\
1.3581	0.3112\\
1.3628	0.3123\\
1.3675	0.3131\\
1.3723	0.3137\\
1.375	0.314\\
1.3777	0.3142\\
1.3805	0.3143\\
1.3832	0.3143\\
};
\addplot [color=black, forget plot]
  table[row sep=crcr]{%
1.3832	0.3143\\
1.3858	0.3143\\
1.3864	0.3142\\
1.3896	0.3141\\
1.396	0.3135\\
1.4024	0.3125\\
1.4088	0.3111\\
1.4152	0.3093\\
1.4216	0.3071\\
1.428	0.3045\\
1.4344	0.3015\\
1.4376	0.2998\\
1.444	0.2962\\
1.4504	0.2922\\
1.4536	0.29\\
1.4567	0.2878\\
1.4631	0.283\\
1.4663	0.2804\\
1.4727	0.275\\
1.4791	0.2692\\
1.4823	0.2661\\
1.4887	0.2597\\
1.4951	0.2529\\
1.4983	0.2493\\
1.5015	0.2456\\
1.5063	0.24\\
1.5087	0.237\\
1.5111	0.2341\\
};
\addplot [color=black, forget plot]
  table[row sep=crcr]{%
1.5111	0.2341\\
1.5158	0.228\\
1.5206	0.2218\\
1.5253	0.2153\\
1.53	0.2086\\
1.5347	0.2017\\
1.5394	0.1946\\
1.5442	0.1872\\
1.5489	0.1796\\
1.5536	0.1719\\
1.5583	0.1639\\
1.5631	0.1556\\
1.5678	0.1472\\
1.5725	0.1385\\
1.5772	0.1297\\
1.5819	0.1206\\
1.5867	0.1112\\
1.5914	0.1017\\
1.5961	0.092\\
1.6008	0.082\\
1.6056	0.0718\\
1.6103	0.0614\\
1.615	0.0508\\
1.6197	0.0399\\
1.6244	0.0288\\
1.6274	0.0218\\
1.6304	0.0146\\
1.6334	0.0073\\
1.6363	-0\\
};
\addplot [color=black, forget plot]
  table[row sep=crcr]{%
1.6363	0\\
1.6364	0.0001\\
1.6364	0.0002\\
1.6366	0.0005\\
1.6367	0.0007\\
1.6369	0.001\\
1.637	0.0012\\
1.6377	0.0025\\
1.6383	0.0037\\
1.639	0.0049\\
1.6397	0.0062\\
1.6413	0.0091\\
1.6429	0.0119\\
1.6445	0.0148\\
1.646	0.0176\\
1.6492	0.0232\\
1.6508	0.0259\\
1.6524	0.0287\\
1.6556	0.0341\\
1.6636	0.0471\\
1.6651	0.0496\\
1.6683	0.0546\\
1.6763	0.0666\\
1.6811	0.0735\\
1.6827	0.0757\\
1.6842	0.078\\
1.6858	0.0802\\
1.6874	0.0823\\
1.689	0.0845\\
1.6954	0.0929\\
1.697	0.0949\\
1.6977	0.0959\\
1.6985	0.0968\\
1.6992	0.0977\\
1.7	0.0987\\
};
\addplot [color=black, forget plot]
  table[row sep=crcr]{%
0	0.95\\
0.0008	0.95\\
0.001	0.9499\\
0.0023	0.9499\\
0.0036	0.9498\\
0.0049	0.9496\\
0.0061	0.9495\\
0.0109	0.9489\\
0.0156	0.948\\
0.0203	0.947\\
0.025	0.9457\\
0.0298	0.9442\\
0.0345	0.9424\\
0.0392	0.9405\\
0.0439	0.9383\\
0.0486	0.936\\
0.0534	0.9334\\
0.0581	0.9305\\
0.0628	0.9275\\
0.0675	0.9243\\
0.0723	0.9208\\
0.077	0.9171\\
0.0817	0.9132\\
0.0864	0.909\\
0.0911	0.9047\\
0.0959	0.9001\\
0.1006	0.8953\\
0.1053	0.8903\\
0.11	0.8851\\
0.1148	0.8797\\
0.1195	0.874\\
0.1242	0.8681\\
0.1289	0.862\\
0.1336	0.8557\\
0.1384	0.8492\\
0.1431	0.8424\\
0.1478	0.8354\\
0.1525	0.8283\\
0.1573	0.8208\\
0.162	0.8132\\
0.1667	0.8054\\
0.1714	0.7973\\
0.1761	0.789\\
0.1793	0.7833\\
0.1825	0.7775\\
0.1857	0.7716\\
0.1889	0.7656\\
};
\addplot [color=black, forget plot]
  table[row sep=crcr]{%
0.1889	0.7656\\
0.1936	0.7565\\
0.1983	0.7471\\
0.2031	0.7376\\
0.2078	0.7279\\
0.2125	0.7179\\
0.2172	0.7077\\
0.2219	0.6973\\
0.2267	0.6867\\
0.2314	0.6758\\
0.2361	0.6647\\
0.2408	0.6535\\
0.2456	0.642\\
0.2503	0.6302\\
0.255	0.6183\\
0.2597	0.6061\\
0.2644	0.5938\\
0.2692	0.5812\\
0.2739	0.5684\\
0.2786	0.5553\\
0.2833	0.5421\\
0.2881	0.5286\\
0.2928	0.5149\\
0.2975	0.501\\
0.3022	0.4869\\
0.3069	0.4725\\
0.3117	0.458\\
0.3164	0.4432\\
0.3211	0.4282\\
0.3258	0.413\\
0.3306	0.3975\\
0.3353	0.3819\\
0.34	0.366\\
0.3447	0.3499\\
0.3494	0.3336\\
0.3542	0.317\\
0.3589	0.3003\\
0.3636	0.2833\\
0.3683	0.2661\\
0.3731	0.2487\\
0.3778	0.2311\\
};
\addplot [color=black, forget plot]
  table[row sep=crcr]{%
0.3778	0.2311\\
0.384	0.2077\\
0.3871	0.1958\\
0.3901	0.1839\\
0.3949	0.1655\\
0.3996	0.1468\\
0.4043	0.128\\
0.409	0.1089\\
0.4138	0.0896\\
0.4185	0.0701\\
0.4232	0.0504\\
0.4279	0.0304\\
0.4297	0.0229\\
0.4315	0.0153\\
0.4332	0.0077\\
0.435	-0\\
};
\addplot [color=black, forget plot]
  table[row sep=crcr]{%
0.435	0\\
0.435	0.0001\\
0.4351	0.0001\\
0.4351	0.0002\\
0.4352	0.0005\\
0.4352	0.0007\\
0.4353	0.001\\
0.4354	0.0012\\
0.4358	0.0025\\
0.4362	0.0037\\
0.4366	0.005\\
0.4369	0.0062\\
0.4389	0.0124\\
0.4408	0.0186\\
0.4428	0.0248\\
0.4447	0.0309\\
0.448	0.0412\\
0.4513	0.0514\\
0.4579	0.0714\\
0.4612	0.0813\\
0.4645	0.091\\
0.4677	0.1007\\
0.4743	0.1197\\
0.4809	0.1383\\
0.4842	0.1474\\
0.4875	0.1564\\
0.4908	0.1653\\
0.4941	0.1741\\
0.4974	0.1828\\
0.5007	0.1914\\
0.5039	0.1999\\
0.5072	0.2083\\
0.5138	0.2247\\
0.5171	0.2328\\
0.5237	0.2486\\
0.527	0.2563\\
0.5336	0.2715\\
0.5369	0.2789\\
0.5401	0.2862\\
0.5434	0.2934\\
0.5467	0.3005\\
0.55	0.3075\\
0.5533	0.3144\\
0.5566	0.3212\\
0.5599	0.3279\\
0.5632	0.3344\\
0.5641	0.3362\\
0.5649	0.3379\\
0.5667	0.3413\\
};
\addplot [color=black, forget plot]
  table[row sep=crcr]{%
0.5667	0.3413\\
0.5714	0.3504\\
0.5761	0.3592\\
0.5808	0.3679\\
0.5856	0.3763\\
0.5903	0.3845\\
0.595	0.3925\\
0.5997	0.4003\\
0.6044	0.4078\\
0.6092	0.4152\\
0.6139	0.4223\\
0.6186	0.4292\\
0.6233	0.4358\\
0.6281	0.4423\\
0.6328	0.4485\\
0.6375	0.4546\\
0.6422	0.4604\\
0.6469	0.466\\
0.6517	0.4713\\
0.6564	0.4765\\
0.6611	0.4814\\
0.6658	0.4861\\
0.6706	0.4906\\
0.6753	0.4949\\
0.68	0.4989\\
0.6847	0.5027\\
0.6894	0.5064\\
0.6942	0.5098\\
0.6989	0.5129\\
0.7036	0.5159\\
0.7083	0.5186\\
0.7131	0.5212\\
0.7178	0.5235\\
0.7225	0.5255\\
0.7272	0.5274\\
0.7319	0.5291\\
0.7367	0.5305\\
0.7414	0.5317\\
0.7461	0.5327\\
0.7508	0.5334\\
0.7556	0.534\\
};
\addplot [color=black, forget plot]
  table[row sep=crcr]{%
0.7556	0.534\\
0.756	0.534\\
0.7565	0.5341\\
0.757	0.5341\\
0.7575	0.5342\\
0.7613	0.5344\\
0.7651	0.5344\\
};
\addplot [color=black, forget plot]
  table[row sep=crcr]{%
0.7651	0.5344\\
0.7683	0.5344\\
0.7747	0.534\\
0.7811	0.5332\\
0.7856	0.5324\\
0.7901	0.5314\\
0.7946	0.5302\\
0.799	0.5288\\
0.8035	0.5272\\
0.808	0.5254\\
0.8125	0.5234\\
0.817	0.5213\\
0.8215	0.5189\\
0.8259	0.5163\\
0.8304	0.5135\\
0.8349	0.5106\\
0.8394	0.5074\\
0.8439	0.504\\
0.8484	0.5005\\
0.8528	0.4967\\
0.8573	0.4927\\
0.8618	0.4886\\
0.8663	0.4842\\
0.8708	0.4797\\
0.8753	0.4749\\
0.8797	0.47\\
0.8842	0.4649\\
0.8887	0.4595\\
0.8932	0.454\\
0.8977	0.4483\\
0.9022	0.4423\\
0.9066	0.4362\\
0.9111	0.4299\\
0.9156	0.4234\\
0.9201	0.4166\\
0.9246	0.4097\\
0.9291	0.4026\\
0.9335	0.3953\\
0.938	0.3878\\
0.9425	0.3801\\
0.943	0.3792\\
0.944	0.3776\\
0.9444	0.3767\\
};
\addplot [color=black, forget plot]
  table[row sep=crcr]{%
0.9444	0.3767\\
0.9492	0.3683\\
0.9539	0.3597\\
0.9586	0.3508\\
0.9633	0.3417\\
0.9681	0.3324\\
0.9728	0.3229\\
0.9775	0.3132\\
0.9822	0.3032\\
0.9869	0.2931\\
0.9917	0.2827\\
0.9964	0.2721\\
1.0011	0.2613\\
1.0058	0.2502\\
1.0106	0.239\\
1.0153	0.2275\\
1.02	0.2158\\
1.0247	0.2039\\
1.0294	0.1917\\
1.0342	0.1794\\
1.0389	0.1668\\
1.0436	0.154\\
1.0483	0.141\\
1.0531	0.1278\\
1.0578	0.1143\\
1.0625	0.1007\\
1.0672	0.0868\\
1.0767	0.0583\\
1.0813	0.0441\\
1.0859	0.0296\\
1.0906	0.0149\\
1.0952	-0\\
};
\addplot [color=black, forget plot]
  table[row sep=crcr]{%
1.0952	0\\
1.0952	0.0001\\
1.0953	0.0002\\
1.0954	0.0005\\
1.0955	0.0007\\
1.0956	0.001\\
1.0957	0.0012\\
1.0962	0.0025\\
1.0967	0.0037\\
1.0973	0.005\\
1.0978	0.0062\\
1.0987	0.0085\\
1.0997	0.0108\\
1.1006	0.013\\
1.1016	0.0153\\
1.1025	0.0175\\
1.1035	0.0198\\
1.1044	0.022\\
1.1054	0.0242\\
1.1063	0.0265\\
1.1083	0.0309\\
1.1092	0.033\\
1.1102	0.0352\\
1.1111	0.0374\\
1.1121	0.0396\\
1.113	0.0417\\
1.114	0.0439\\
1.1149	0.046\\
1.1159	0.0481\\
1.1168	0.0502\\
1.1178	0.0523\\
1.1187	0.0544\\
1.1197	0.0565\\
1.1206	0.0586\\
1.1226	0.0628\\
1.1235	0.0648\\
1.1245	0.0669\\
1.1254	0.0689\\
1.1264	0.0709\\
1.1273	0.0729\\
1.1283	0.0749\\
1.1292	0.0769\\
1.1302	0.0789\\
1.1311	0.0809\\
1.1324	0.0835\\
1.1327	0.0842\\
1.133	0.0848\\
1.1333	0.0855\\
};
\addplot [color=black, forget plot]
  table[row sep=crcr]{%
1.1333	0.0855\\
1.1354	0.0897\\
1.1375	0.094\\
1.1396	0.0982\\
1.1417	0.1023\\
1.1464	0.1115\\
1.1511	0.1205\\
1.1559	0.1293\\
1.1606	0.1378\\
1.1653	0.1461\\
1.17	0.1543\\
1.1747	0.1621\\
1.1795	0.1698\\
1.1842	0.1773\\
1.1889	0.1845\\
1.1936	0.1915\\
1.1984	0.1983\\
1.2031	0.2049\\
1.2078	0.2113\\
1.2125	0.2174\\
1.2172	0.2233\\
1.222	0.229\\
1.2267	0.2345\\
1.2314	0.2398\\
1.2361	0.2449\\
1.2409	0.2497\\
1.2456	0.2543\\
1.2503	0.2587\\
1.255	0.2629\\
1.2597	0.2668\\
1.2645	0.2706\\
1.2692	0.2741\\
1.2739	0.2774\\
1.2786	0.2805\\
1.2834	0.2833\\
1.2881	0.286\\
1.2928	0.2884\\
1.2975	0.2906\\
1.3022	0.2926\\
1.307	0.2943\\
1.3117	0.2959\\
1.3143	0.2967\\
1.317	0.2974\\
1.3222	0.2986\\
};
\addplot [color=black, forget plot]
  table[row sep=crcr]{%
1.3222	0.2986\\
1.3233	0.2988\\
1.3243	0.299\\
1.3253	0.2991\\
1.3264	0.2993\\
1.3305	0.2999\\
1.3346	0.3003\\
1.3387	0.3005\\
1.3428	0.3006\\
};
\addplot [color=black, forget plot]
  table[row sep=crcr]{%
1.3428	0.3006\\
1.346	0.3006\\
1.3524	0.3002\\
1.3588	0.2994\\
1.363	0.2986\\
1.3672	0.2977\\
1.3714	0.2966\\
1.3756	0.2953\\
1.3798	0.2939\\
1.384	0.2923\\
1.3882	0.2905\\
1.3924	0.2885\\
1.3966	0.2864\\
1.4009	0.2841\\
1.4051	0.2816\\
1.4093	0.2789\\
1.4135	0.2761\\
1.4177	0.2731\\
1.4219	0.2699\\
1.4261	0.2666\\
1.4303	0.263\\
1.4345	0.2593\\
1.4387	0.2555\\
1.4429	0.2514\\
1.4471	0.2472\\
1.4514	0.2428\\
1.4556	0.2382\\
1.4598	0.2335\\
1.464	0.2286\\
1.4682	0.2235\\
1.4724	0.2182\\
1.4766	0.2128\\
1.4808	0.2072\\
1.485	0.2014\\
1.4892	0.1954\\
1.4934	0.1893\\
1.4979	0.1827\\
1.5023	0.1758\\
1.5067	0.1688\\
1.5111	0.1616\\
};
\addplot [color=black, forget plot]
  table[row sep=crcr]{%
1.5111	0.1616\\
1.5158	0.1537\\
1.5206	0.1456\\
1.5253	0.1372\\
1.53	0.1287\\
1.5347	0.1199\\
1.5394	0.1109\\
1.5442	0.1017\\
1.5489	0.0922\\
1.5536	0.0826\\
1.5583	0.0727\\
1.5631	0.0626\\
1.5678	0.0523\\
1.5725	0.0418\\
1.5772	0.031\\
1.5819	0.02\\
1.5867	0.0089\\
1.5876	0.0067\\
1.5885	0.0044\\
1.5903	-0\\
};
\addplot [color=black, forget plot]
  table[row sep=crcr]{%
1.5903	0\\
1.5904	0.0001\\
1.5904	0.0002\\
1.5906	0.0005\\
1.5907	0.0007\\
1.5909	0.001\\
1.591	0.0012\\
1.5917	0.0025\\
1.5931	0.0049\\
1.5938	0.0062\\
1.5992	0.0158\\
1.602	0.0205\\
1.6047	0.0252\\
1.6075	0.0298\\
1.6102	0.0343\\
1.6129	0.0387\\
1.6157	0.043\\
1.6184	0.0473\\
1.6212	0.0515\\
1.6239	0.0556\\
1.6267	0.0597\\
1.6294	0.0637\\
1.6321	0.0676\\
1.6349	0.0714\\
1.6376	0.0752\\
1.6404	0.0788\\
1.6431	0.0825\\
1.6458	0.086\\
1.6486	0.0895\\
1.6513	0.0928\\
1.6541	0.0962\\
1.6568	0.0994\\
1.6596	0.1026\\
1.6623	0.1057\\
1.665	0.1087\\
1.6678	0.1116\\
1.6705	0.1145\\
1.6787	0.1227\\
1.6815	0.1253\\
1.6842	0.1278\\
1.687	0.1302\\
1.6897	0.1326\\
1.6925	0.1349\\
1.6943	0.1364\\
1.6962	0.1379\\
1.6981	0.1393\\
1.7	0.1408\\
};
\addplot [color=black, forget plot]
  table[row sep=crcr]{%
0	1.05\\
0.0008	1.05\\
0.001	1.0499\\
0.0023	1.0499\\
0.0036	1.0498\\
0.0049	1.0496\\
0.0061	1.0495\\
0.0109	1.0489\\
0.0156	1.048\\
0.0203	1.047\\
0.025	1.0457\\
0.0298	1.0442\\
0.0345	1.0424\\
0.0392	1.0405\\
0.0439	1.0383\\
0.0486	1.036\\
0.0534	1.0334\\
0.0581	1.0305\\
0.0628	1.0275\\
0.0675	1.0243\\
0.0723	1.0208\\
0.077	1.0171\\
0.0817	1.0132\\
0.0864	1.009\\
0.0911	1.0047\\
0.0959	1.0001\\
0.1006	0.9953\\
0.1053	0.9903\\
0.11	0.9851\\
0.1148	0.9797\\
0.1195	0.974\\
0.1242	0.9681\\
0.1289	0.962\\
0.1336	0.9557\\
0.1384	0.9492\\
0.1431	0.9424\\
0.1478	0.9354\\
0.1525	0.9283\\
0.1573	0.9208\\
0.162	0.9132\\
0.1667	0.9054\\
0.1714	0.8973\\
0.1761	0.889\\
0.1793	0.8833\\
0.1825	0.8775\\
0.1857	0.8716\\
0.1889	0.8656\\
};
\addplot [color=black, forget plot]
  table[row sep=crcr]{%
0.1889	0.8656\\
0.1936	0.8565\\
0.1983	0.8471\\
0.2031	0.8376\\
0.2078	0.8279\\
0.2125	0.8179\\
0.2172	0.8077\\
0.2219	0.7973\\
0.2267	0.7867\\
0.2314	0.7758\\
0.2361	0.7647\\
0.2408	0.7535\\
0.2456	0.742\\
0.2503	0.7302\\
0.255	0.7183\\
0.2597	0.7061\\
0.2644	0.6938\\
0.2692	0.6812\\
0.2739	0.6684\\
0.2786	0.6553\\
0.2833	0.6421\\
0.2881	0.6286\\
0.2928	0.6149\\
0.2975	0.601\\
0.3022	0.5869\\
0.3069	0.5725\\
0.3117	0.558\\
0.3164	0.5432\\
0.3211	0.5282\\
0.3258	0.513\\
0.3306	0.4975\\
0.3353	0.4819\\
0.34	0.466\\
0.3447	0.4499\\
0.3494	0.4336\\
0.3542	0.417\\
0.3589	0.4003\\
0.3636	0.3833\\
0.3683	0.3661\\
0.3731	0.3487\\
0.3778	0.3311\\
};
\addplot [color=black, forget plot]
  table[row sep=crcr]{%
0.3778	0.3311\\
0.3822	0.3144\\
0.3866	0.2974\\
0.3911	0.2803\\
0.3955	0.263\\
0.4002	0.2444\\
0.4049	0.2255\\
0.4097	0.2064\\
0.4144	0.187\\
0.4191	0.1675\\
0.4238	0.1477\\
0.4285	0.1278\\
0.4333	0.1076\\
0.438	0.0871\\
0.4427	0.0665\\
0.4474	0.0457\\
0.4522	0.0246\\
0.4535	0.0185\\
0.4549	0.0123\\
0.4562	0.0062\\
0.4576	-0\\
};
\addplot [color=black, forget plot]
  table[row sep=crcr]{%
0.4576	0\\
0.4576	0.0002\\
0.4577	0.0002\\
0.4577	0.0005\\
0.4578	0.0007\\
0.4579	0.001\\
0.458	0.0012\\
0.4583	0.0025\\
0.4587	0.0037\\
0.4591	0.005\\
0.4594	0.0062\\
0.4613	0.0124\\
0.4631	0.0186\\
0.465	0.0248\\
0.4668	0.0309\\
0.4695	0.0399\\
0.4723	0.0489\\
0.4777	0.0665\\
0.4804	0.0752\\
0.4832	0.0838\\
0.4859	0.0924\\
0.4886	0.1009\\
0.4914	0.1093\\
0.4968	0.1259\\
0.4995	0.1341\\
0.5023	0.1422\\
0.505	0.1503\\
0.5077	0.1583\\
0.5104	0.1662\\
0.5132	0.174\\
0.5159	0.1818\\
0.5186	0.1894\\
0.5213	0.1971\\
0.5241	0.2046\\
0.5268	0.2121\\
0.5295	0.2195\\
0.5323	0.2268\\
0.5377	0.2412\\
0.5404	0.2483\\
0.5432	0.2553\\
0.5459	0.2623\\
0.5486	0.2692\\
0.5513	0.276\\
0.5541	0.2827\\
0.5568	0.2894\\
0.5595	0.296\\
0.5622	0.3025\\
0.565	0.309\\
0.5658	0.311\\
0.5662	0.3119\\
0.5667	0.3129\\
};
\addplot [color=black, forget plot]
  table[row sep=crcr]{%
0.5667	0.3129\\
0.5714	0.3239\\
0.5761	0.3345\\
0.5808	0.345\\
0.5856	0.3553\\
0.5903	0.3653\\
0.595	0.3751\\
0.5997	0.3847\\
0.6044	0.3941\\
0.6092	0.4033\\
0.6139	0.4122\\
0.6186	0.421\\
0.6233	0.4295\\
0.6281	0.4378\\
0.6328	0.4458\\
0.6375	0.4537\\
0.6422	0.4613\\
0.6469	0.4687\\
0.6517	0.4759\\
0.6564	0.4829\\
0.6611	0.4897\\
0.6658	0.4962\\
0.6706	0.5025\\
0.6753	0.5086\\
0.68	0.5145\\
0.6847	0.5202\\
0.6894	0.5256\\
0.6942	0.5308\\
0.6989	0.5359\\
0.7036	0.5406\\
0.7083	0.5452\\
0.7131	0.5496\\
0.7178	0.5537\\
0.7225	0.5576\\
0.7272	0.5613\\
0.7319	0.5648\\
0.7367	0.568\\
0.7414	0.5711\\
0.7461	0.5739\\
0.7508	0.5765\\
0.7556	0.5789\\
};
\addplot [color=black, forget plot]
  table[row sep=crcr]{%
0.7556	0.5789\\
0.758	0.58\\
0.763	0.5822\\
0.7654	0.5832\\
0.7701	0.5849\\
0.7749	0.5863\\
0.7796	0.5876\\
0.7843	0.5887\\
0.789	0.5895\\
0.7938	0.5901\\
0.7985	0.5905\\
0.8032	0.5907\\
0.8046	0.5907\\
};
\addplot [color=black, forget plot]
  table[row sep=crcr]{%
0.8046	0.5907\\
0.8072	0.5907\\
0.8078	0.5906\\
0.811	0.5905\\
0.8174	0.5899\\
0.8206	0.5894\\
0.8241	0.5888\\
0.8276	0.5881\\
0.8311	0.5873\\
0.8346	0.5863\\
0.8381	0.5852\\
0.8416	0.584\\
0.8451	0.5827\\
0.8521	0.5797\\
0.8556	0.578\\
0.8591	0.5762\\
0.8661	0.5722\\
0.8696	0.57\\
0.8731	0.5677\\
0.8766	0.5653\\
0.8801	0.5628\\
0.8835	0.5602\\
0.887	0.5574\\
0.8905	0.5545\\
0.894	0.5515\\
0.8975	0.5484\\
0.9045	0.5418\\
0.908	0.5383\\
0.9115	0.5347\\
0.9185	0.5271\\
0.9255	0.5191\\
0.9325	0.5105\\
0.9355	0.5067\\
0.9415	0.4989\\
0.9444	0.4948\\
};
\addplot [color=black, forget plot]
  table[row sep=crcr]{%
0.9444	0.4948\\
0.9492	0.4882\\
0.9539	0.4814\\
0.9586	0.4744\\
0.9633	0.4672\\
0.9681	0.4597\\
0.9728	0.452\\
0.9775	0.4441\\
0.9822	0.436\\
0.9869	0.4277\\
0.9917	0.4191\\
0.9964	0.4103\\
1.0011	0.4013\\
1.0058	0.3921\\
1.0106	0.3827\\
1.0153	0.3731\\
1.02	0.3632\\
1.0247	0.3531\\
1.0294	0.3428\\
1.0342	0.3323\\
1.0389	0.3215\\
1.0436	0.3106\\
1.0483	0.2994\\
1.0531	0.288\\
1.0578	0.2764\\
1.0625	0.2645\\
1.0672	0.2525\\
1.0719	0.2402\\
1.0767	0.2277\\
1.0814	0.215\\
1.0861	0.2021\\
1.0908	0.1889\\
1.0956	0.1756\\
1.1003	0.162\\
1.105	0.1482\\
1.1097	0.1341\\
1.1144	0.1199\\
1.1192	0.1054\\
1.1239	0.0908\\
1.1286	0.0759\\
1.1333	0.0607\\
};
\addplot [color=black, forget plot]
  table[row sep=crcr]{%
1.1333	0.0607\\
1.1343	0.0577\\
1.1352	0.0546\\
1.1362	0.0515\\
1.1371	0.0485\\
1.1408	0.0365\\
1.1444	0.0245\\
1.148	0.0123\\
1.1517	0\\
};
\addplot [color=black, forget plot]
  table[row sep=crcr]{%
1.1517	0\\
1.1517	0.0002\\
1.1518	0.0005\\
1.1519	0.0007\\
1.152	0.001\\
1.1521	0.0012\\
1.1526	0.0025\\
1.1531	0.0037\\
1.1536	0.005\\
1.1541	0.0062\\
1.1566	0.0124\\
1.159	0.0185\\
1.1615	0.0246\\
1.1639	0.0306\\
1.1682	0.0409\\
1.1725	0.051\\
1.1767	0.0609\\
1.181	0.0707\\
1.1853	0.0802\\
1.1895	0.0896\\
1.1938	0.0989\\
1.198	0.1079\\
1.2023	0.1167\\
1.2066	0.1254\\
1.2108	0.1339\\
1.2151	0.1423\\
1.2194	0.1504\\
1.2236	0.1584\\
1.2322	0.1738\\
1.2364	0.1812\\
1.2407	0.1884\\
1.245	0.1955\\
1.2492	0.2024\\
1.2535	0.2091\\
1.2577	0.2157\\
1.262	0.222\\
1.2663	0.2282\\
1.2705	0.2342\\
1.2748	0.24\\
1.2791	0.2457\\
1.2833	0.2512\\
1.2919	0.2616\\
1.2961	0.2665\\
1.3004	0.2712\\
1.3047	0.2758\\
1.3089	0.2802\\
1.3132	0.2844\\
1.3174	0.2885\\
1.3198	0.2907\\
1.321	0.2917\\
1.3222	0.2928\\
};
\addplot [color=black, forget plot]
  table[row sep=crcr]{%
1.3222	0.2928\\
1.3267	0.2967\\
1.3312	0.3003\\
1.3357	0.3038\\
1.3402	0.3071\\
1.345	0.3103\\
1.3497	0.3133\\
1.3544	0.316\\
1.3591	0.3186\\
1.3639	0.3209\\
1.3686	0.3231\\
1.3733	0.3249\\
1.378	0.3266\\
1.3827	0.3281\\
1.3875	0.3293\\
1.3922	0.3304\\
1.3969	0.3312\\
1.4007	0.3316\\
1.4044	0.332\\
1.4082	0.3322\\
1.4119	0.3323\\
};
\addplot [color=black, forget plot]
  table[row sep=crcr]{%
1.4119	0.3323\\
1.4132	0.3323\\
1.4138	0.3322\\
1.4151	0.3322\\
1.4176	0.3321\\
1.4226	0.3317\\
1.425	0.3314\\
1.4275	0.3311\\
1.43	0.3307\\
1.435	0.3297\\
1.4374	0.3291\\
1.4424	0.3277\\
1.4474	0.3261\\
1.4498	0.3252\\
1.4523	0.3243\\
1.4548	0.3233\\
1.4573	0.3222\\
1.4598	0.321\\
1.4622	0.3199\\
1.4672	0.3173\\
1.4722	0.3145\\
1.4746	0.313\\
1.4796	0.3098\\
1.4846	0.3064\\
1.487	0.3046\\
1.492	0.3008\\
1.4945	0.2988\\
1.4969	0.2968\\
1.5019	0.2926\\
1.5044	0.2903\\
1.5061	0.2888\\
1.5077	0.2872\\
1.5111	0.284\\
};
\addplot [color=black, forget plot]
  table[row sep=crcr]{%
1.5111	0.284\\
1.5158	0.2793\\
1.5206	0.2744\\
1.5253	0.2692\\
1.53	0.2639\\
1.5347	0.2583\\
1.5394	0.2525\\
1.5442	0.2465\\
1.5489	0.2403\\
1.5536	0.2338\\
1.5583	0.2271\\
1.5631	0.2202\\
1.5678	0.2131\\
1.5725	0.2058\\
1.5772	0.1983\\
1.5819	0.1905\\
1.5867	0.1825\\
1.5914	0.1743\\
1.5961	0.1659\\
1.6008	0.1572\\
1.6056	0.1484\\
1.6103	0.1393\\
1.615	0.13\\
1.6197	0.1205\\
1.6244	0.1107\\
1.6292	0.1008\\
1.6339	0.0906\\
1.6386	0.0802\\
1.6433	0.0696\\
1.6481	0.0588\\
1.6528	0.0477\\
1.6575	0.0365\\
1.6622	0.025\\
1.6672	0.0126\\
1.6722	-0\\
};
\addplot [color=black, forget plot]
  table[row sep=crcr]{%
1.6722	0\\
1.6722	0.0001\\
1.6723	0.0001\\
1.6723	0.0002\\
1.6724	0.0005\\
1.6726	0.0007\\
1.6727	0.001\\
1.6728	0.0012\\
1.6735	0.0025\\
1.6741	0.0037\\
1.6748	0.0049\\
1.6755	0.0062\\
1.6761	0.0075\\
1.6789	0.0127\\
1.6796	0.0139\\
1.6817	0.0178\\
1.6824	0.019\\
1.6831	0.0203\\
1.6838	0.0215\\
1.6845	0.0228\\
1.6852	0.024\\
1.6859	0.0253\\
1.6873	0.0277\\
1.688	0.029\\
1.6894	0.0314\\
1.69	0.0326\\
1.6949	0.041\\
1.6956	0.0421\\
1.697	0.0445\\
1.6977	0.0456\\
1.6983	0.0466\\
1.6988	0.0475\\
1.6994	0.0485\\
1.7	0.0494\\
};
\addplot [color=black, forget plot]
  table[row sep=crcr]{%
0	1.05\\
0.001	1.05\\
0.002	1.0501\\
0.0051	1.0501\\
};
\addplot [color=black, forget plot]
  table[row sep=crcr]{%
0.0051	1.0501\\
0.0083	1.0501\\
0.0147	1.0497\\
0.0211	1.0489\\
0.0257	1.048\\
0.0303	1.047\\
0.0349	1.0458\\
0.0395	1.0443\\
0.0441	1.0427\\
0.0487	1.0408\\
0.0533	1.0387\\
0.0579	1.0365\\
0.0624	1.034\\
0.067	1.0313\\
0.0716	1.0284\\
0.0762	1.0253\\
0.0808	1.022\\
0.0854	1.0185\\
0.09	1.0148\\
0.0946	1.0108\\
0.0992	1.0067\\
0.1038	1.0023\\
0.1084	0.9978\\
0.113	0.993\\
0.1176	0.9881\\
0.1222	0.9829\\
0.1268	0.9775\\
0.1314	0.9719\\
0.136	0.9661\\
0.1406	0.9601\\
0.1452	0.9539\\
0.1497	0.9475\\
0.1543	0.9409\\
0.1589	0.934\\
0.1635	0.927\\
0.1681	0.9198\\
0.1727	0.9123\\
0.1773	0.9046\\
0.1819	0.8968\\
0.1865	0.8887\\
0.1871	0.8876\\
0.1877	0.8866\\
0.1889	0.8844\\
};
\addplot [color=black, forget plot]
  table[row sep=crcr]{%
0.1889	0.8844\\
0.1936	0.8758\\
0.1983	0.867\\
0.2031	0.8579\\
0.2078	0.8486\\
0.2125	0.8391\\
0.2172	0.8294\\
0.2219	0.8195\\
0.2267	0.8093\\
0.2314	0.799\\
0.2361	0.7884\\
0.2408	0.7775\\
0.2456	0.7665\\
0.2503	0.7553\\
0.255	0.7438\\
0.2597	0.7321\\
0.2644	0.7202\\
0.2692	0.7081\\
0.2739	0.6957\\
0.2786	0.6832\\
0.2833	0.6704\\
0.2881	0.6574\\
0.2928	0.6442\\
0.2975	0.6308\\
0.3022	0.6171\\
0.3069	0.6032\\
0.3117	0.5891\\
0.3164	0.5748\\
0.3211	0.5603\\
0.3258	0.5455\\
0.3306	0.5306\\
0.3353	0.5154\\
0.34	0.5\\
0.3447	0.4844\\
0.3494	0.4685\\
0.3542	0.4525\\
0.3589	0.4362\\
0.3636	0.4197\\
0.3683	0.403\\
0.3731	0.386\\
0.3778	0.3689\\
};
\addplot [color=black, forget plot]
  table[row sep=crcr]{%
0.3778	0.3689\\
0.3825	0.3515\\
0.3872	0.3339\\
0.3919	0.3161\\
0.3967	0.2981\\
0.4014	0.2798\\
0.4061	0.2613\\
0.4108	0.2427\\
0.4156	0.2238\\
0.4203	0.2046\\
0.425	0.1853\\
0.4297	0.1657\\
0.4344	0.1459\\
0.4392	0.1259\\
0.4439	0.1057\\
0.4486	0.0853\\
0.4533	0.0646\\
0.4569	0.0487\\
0.4606	0.0326\\
0.4642	0.0164\\
0.4678	-0\\
};
\addplot [color=black, forget plot]
  table[row sep=crcr]{%
0.4678	0\\
0.4678	0.0002\\
0.4679	0.0002\\
0.4679	0.0005\\
0.468	0.0007\\
0.4681	0.001\\
0.4682	0.0012\\
0.4685	0.0025\\
0.4689	0.0037\\
0.4693	0.005\\
0.4696	0.0062\\
0.4715	0.0124\\
0.4733	0.0186\\
0.4752	0.0248\\
0.477	0.0309\\
0.4795	0.0391\\
0.482	0.0472\\
0.4844	0.0552\\
0.4869	0.0632\\
0.4894	0.0711\\
0.4918	0.079\\
0.4943	0.0868\\
0.4993	0.1022\\
0.5017	0.1098\\
0.5042	0.1174\\
0.5067	0.1249\\
0.5091	0.1324\\
0.5116	0.1397\\
0.5141	0.1471\\
0.5166	0.1543\\
0.519	0.1615\\
0.5215	0.1687\\
0.524	0.1757\\
0.5264	0.1828\\
0.5339	0.2035\\
0.5363	0.2103\\
0.5388	0.217\\
0.5413	0.2236\\
0.5437	0.2302\\
0.5462	0.2368\\
0.5487	0.2433\\
0.5512	0.2497\\
0.5536	0.2561\\
0.5561	0.2624\\
0.5587	0.269\\
0.5614	0.2756\\
0.564	0.2822\\
0.5667	0.2886\\
};
\addplot [color=black, forget plot]
  table[row sep=crcr]{%
0.5667	0.2886\\
0.5714	0.3\\
0.5761	0.3112\\
0.5808	0.3221\\
0.5856	0.3329\\
0.5903	0.3434\\
0.595	0.3537\\
0.5997	0.3637\\
0.6044	0.3736\\
0.6092	0.3832\\
0.6139	0.3927\\
0.6186	0.4019\\
0.6233	0.4108\\
0.6281	0.4196\\
0.6328	0.4281\\
0.6375	0.4365\\
0.6422	0.4446\\
0.6469	0.4525\\
0.6517	0.4601\\
0.6564	0.4676\\
0.6611	0.4748\\
0.6658	0.4818\\
0.6706	0.4886\\
0.6753	0.4952\\
0.68	0.5015\\
0.6847	0.5077\\
0.6894	0.5136\\
0.6942	0.5193\\
0.6989	0.5248\\
0.7036	0.53\\
0.7083	0.5351\\
0.7131	0.5399\\
0.7178	0.5445\\
0.7225	0.5489\\
0.7272	0.5531\\
0.7319	0.557\\
0.7367	0.5607\\
0.7414	0.5642\\
0.7461	0.5675\\
0.7508	0.5706\\
0.7556	0.5735\\
};
\addplot [color=black, forget plot]
  table[row sep=crcr]{%
0.7556	0.5735\\
0.7585	0.5752\\
0.7615	0.5768\\
0.7645	0.5783\\
0.7675	0.5797\\
0.7722	0.5818\\
0.7769	0.5836\\
0.7816	0.5853\\
0.7864	0.5867\\
0.7911	0.5879\\
0.7958	0.5889\\
0.8005	0.5897\\
0.8052	0.5902\\
0.81	0.5906\\
0.8124	0.5907\\
0.8148	0.5907\\
};
\addplot [color=black, forget plot]
  table[row sep=crcr]{%
0.8148	0.5907\\
0.8174	0.5907\\
0.818	0.5906\\
0.8212	0.5905\\
0.8276	0.5899\\
0.8308	0.5894\\
0.8341	0.5889\\
0.8405	0.5875\\
0.8438	0.5866\\
0.847	0.5856\\
0.8503	0.5845\\
0.8535	0.5834\\
0.8567	0.5821\\
0.86	0.5807\\
0.8632	0.5792\\
0.8665	0.5776\\
0.8697	0.5759\\
0.8729	0.5741\\
0.8762	0.5722\\
0.8794	0.5702\\
0.8827	0.5681\\
0.8859	0.5659\\
0.8892	0.5636\\
0.8924	0.5612\\
0.8956	0.5587\\
0.8989	0.556\\
0.9021	0.5533\\
0.9054	0.5505\\
0.9086	0.5476\\
0.9118	0.5445\\
0.9151	0.5414\\
0.9183	0.5382\\
0.9216	0.5348\\
0.9248	0.5314\\
0.928	0.5278\\
0.9313	0.5242\\
0.9345	0.5204\\
0.937	0.5175\\
0.9395	0.5145\\
0.942	0.5114\\
0.9444	0.5083\\
};
\addplot [color=black, forget plot]
  table[row sep=crcr]{%
0.9444	0.5083\\
0.9492	0.5022\\
0.9539	0.4958\\
0.9586	0.4893\\
0.9633	0.4825\\
0.9681	0.4755\\
0.9728	0.4683\\
0.9775	0.4609\\
0.9822	0.4533\\
0.9869	0.4454\\
0.9917	0.4373\\
0.9964	0.429\\
1.0011	0.4205\\
1.0058	0.4117\\
1.0106	0.4028\\
1.0153	0.3936\\
1.02	0.3842\\
1.0247	0.3746\\
1.0294	0.3648\\
1.0342	0.3547\\
1.0389	0.3444\\
1.0436	0.334\\
1.0483	0.3232\\
1.0531	0.3123\\
1.0578	0.3012\\
1.0625	0.2898\\
1.0672	0.2782\\
1.0719	0.2664\\
1.0767	0.2544\\
1.0814	0.2422\\
1.0861	0.2297\\
1.0908	0.217\\
1.0956	0.2041\\
1.1003	0.191\\
1.105	0.1777\\
1.1097	0.1641\\
1.1144	0.1504\\
1.1192	0.1364\\
1.1239	0.1222\\
1.1286	0.1077\\
1.1333	0.0931\\
};
\addplot [color=black, forget plot]
  table[row sep=crcr]{%
1.1333	0.0931\\
1.1378	0.079\\
1.1393	0.0742\\
1.144	0.0591\\
1.1488	0.0437\\
1.1535	0.0281\\
1.1582	0.0123\\
1.1591	0.0093\\
1.1609	0.0031\\
1.1619	-0\\
};
\addplot [color=black, forget plot]
  table[row sep=crcr]{%
1.1619	0\\
1.1619	0.0002\\
1.162	0.0005\\
1.1621	0.0007\\
1.1622	0.001\\
1.1623	0.0012\\
1.1628	0.0025\\
1.1633	0.0037\\
1.1638	0.005\\
1.1643	0.0062\\
1.1668	0.0124\\
1.1692	0.0185\\
1.1717	0.0246\\
1.1741	0.0306\\
1.1781	0.0403\\
1.1821	0.0498\\
1.1862	0.0592\\
1.1902	0.0684\\
1.1942	0.0774\\
1.1982	0.0863\\
1.2022	0.095\\
1.2062	0.1036\\
1.2102	0.112\\
1.2142	0.1203\\
1.2182	0.1284\\
1.2222	0.1363\\
1.2263	0.1441\\
1.2303	0.1517\\
1.2343	0.1592\\
1.2383	0.1665\\
1.2423	0.1736\\
1.2463	0.1806\\
1.2503	0.1875\\
1.2583	0.2007\\
1.2623	0.207\\
1.2663	0.2132\\
1.2704	0.2193\\
1.2744	0.2252\\
1.2784	0.2309\\
1.2824	0.2365\\
1.2864	0.2419\\
1.2904	0.2472\\
1.2944	0.2523\\
1.2984	0.2572\\
1.3024	0.262\\
1.3064	0.2666\\
1.3104	0.2711\\
1.3145	0.2754\\
1.3185	0.2796\\
1.3194	0.2805\\
1.3203	0.2815\\
1.3213	0.2824\\
1.3222	0.2833\\
};
\addplot [color=black, forget plot]
  table[row sep=crcr]{%
1.3222	0.2833\\
1.3269	0.2878\\
1.3317	0.2921\\
1.3364	0.2962\\
1.3411	0.3001\\
1.3458	0.3037\\
1.3506	0.3071\\
1.3553	0.3104\\
1.36	0.3133\\
1.3647	0.3161\\
1.3694	0.3187\\
1.3742	0.321\\
1.3789	0.3231\\
1.3836	0.325\\
1.3883	0.3267\\
1.3931	0.3281\\
1.3978	0.3294\\
1.4025	0.3304\\
1.4072	0.3312\\
1.4119	0.3318\\
1.4167	0.3321\\
1.418	0.3322\\
1.4194	0.3322\\
1.4208	0.3323\\
1.4221	0.3323\\
};
\addplot [color=black, forget plot]
  table[row sep=crcr]{%
1.4221	0.3323\\
1.4234	0.3323\\
1.424	0.3322\\
1.4253	0.3322\\
1.4275	0.3321\\
1.4298	0.332\\
1.432	0.3318\\
1.4342	0.3315\\
1.4364	0.3313\\
1.4387	0.3309\\
1.4431	0.3301\\
1.4453	0.3296\\
1.4476	0.3291\\
1.452	0.3279\\
1.4542	0.3272\\
1.4565	0.3265\\
1.4609	0.3249\\
1.4631	0.324\\
1.4654	0.3231\\
1.4698	0.3211\\
1.472	0.32\\
1.4743	0.3189\\
1.4765	0.3178\\
1.4787	0.3166\\
1.4809	0.3153\\
1.4832	0.314\\
1.4898	0.3098\\
1.4921	0.3083\\
1.4987	0.3035\\
1.501	0.3018\\
1.5054	0.2982\\
1.5068	0.2971\\
1.5083	0.2959\\
1.5097	0.2947\\
1.5111	0.2934\\
};
\addplot [color=black, forget plot]
  table[row sep=crcr]{%
1.5111	0.2934\\
1.5156	0.2894\\
1.5201	0.2852\\
1.5245	0.2808\\
1.529	0.2762\\
1.5337	0.2712\\
1.5384	0.2659\\
1.5432	0.2604\\
1.5479	0.2547\\
1.5526	0.2488\\
1.5573	0.2426\\
1.562	0.2362\\
1.5668	0.2296\\
1.5715	0.2228\\
1.5762	0.2158\\
1.5809	0.2086\\
1.5857	0.2011\\
1.5904	0.1934\\
1.5951	0.1855\\
1.5998	0.1774\\
1.6045	0.169\\
1.6093	0.1605\\
1.614	0.1517\\
1.6187	0.1427\\
1.6234	0.1335\\
1.6282	0.124\\
1.6329	0.1144\\
1.6376	0.1045\\
1.6423	0.0944\\
1.647	0.0841\\
1.6518	0.0736\\
1.6565	0.0628\\
1.6612	0.0519\\
1.6659	0.0407\\
1.6707	0.0293\\
1.6754	0.0177\\
1.6801	0.0058\\
1.6807	0.0044\\
1.6812	0.0029\\
1.6818	0.0015\\
1.6824	-0\\
};
\addplot [color=black, forget plot]
  table[row sep=crcr]{%
1.6824	0\\
1.6824	0.0001\\
1.6825	0.0002\\
1.6826	0.0005\\
1.6828	0.0007\\
1.6829	0.001\\
1.683	0.0012\\
1.6835	0.002\\
1.6839	0.0029\\
1.6843	0.0037\\
1.6848	0.0045\\
1.6852	0.0054\\
1.6857	0.0062\\
1.6861	0.007\\
1.6865	0.0079\\
1.687	0.0087\\
1.6874	0.0095\\
1.6879	0.0103\\
1.6883	0.0111\\
1.6887	0.012\\
1.6892	0.0128\\
1.6896	0.0136\\
1.6901	0.0144\\
1.6909	0.016\\
1.6914	0.0168\\
1.6918	0.0176\\
1.6923	0.0184\\
1.6931	0.02\\
1.6936	0.0208\\
1.694	0.0216\\
1.6945	0.0224\\
1.6953	0.024\\
1.6958	0.0248\\
1.6962	0.0256\\
1.6967	0.0263\\
1.6975	0.0279\\
1.698	0.0287\\
1.6984	0.0294\\
1.6989	0.0302\\
1.6992	0.0307\\
1.6994	0.0312\\
1.7	0.0322\\
};
\addplot [color=black, forget plot]
  table[row sep=crcr]{%
0	0.95\\
0.001	0.95\\
0.002	0.9501\\
0.0051	0.9501\\
};
\addplot [color=black, forget plot]
  table[row sep=crcr]{%
0.0051	0.9501\\
0.0083	0.9501\\
0.0147	0.9497\\
0.0211	0.9489\\
0.0257	0.948\\
0.0303	0.947\\
0.0349	0.9458\\
0.0395	0.9443\\
0.0441	0.9427\\
0.0487	0.9408\\
0.0533	0.9387\\
0.0579	0.9365\\
0.0624	0.934\\
0.067	0.9313\\
0.0716	0.9284\\
0.0762	0.9253\\
0.0808	0.922\\
0.0854	0.9185\\
0.09	0.9148\\
0.0946	0.9108\\
0.0992	0.9067\\
0.1038	0.9023\\
0.1084	0.8978\\
0.113	0.893\\
0.1176	0.8881\\
0.1222	0.8829\\
0.1268	0.8775\\
0.1314	0.8719\\
0.136	0.8661\\
0.1406	0.8601\\
0.1452	0.8539\\
0.1497	0.8475\\
0.1543	0.8409\\
0.1589	0.834\\
0.1635	0.827\\
0.1681	0.8198\\
0.1727	0.8123\\
0.1773	0.8046\\
0.1819	0.7968\\
0.1865	0.7887\\
0.1871	0.7876\\
0.1877	0.7866\\
0.1889	0.7844\\
};
\addplot [color=black, forget plot]
  table[row sep=crcr]{%
0.1889	0.7844\\
0.1936	0.7758\\
0.1983	0.767\\
0.2031	0.7579\\
0.2078	0.7486\\
0.2125	0.7391\\
0.2172	0.7294\\
0.2219	0.7195\\
0.2267	0.7093\\
0.2314	0.699\\
0.2361	0.6884\\
0.2408	0.6775\\
0.2456	0.6665\\
0.2503	0.6553\\
0.255	0.6438\\
0.2597	0.6321\\
0.2644	0.6202\\
0.2692	0.6081\\
0.2739	0.5957\\
0.2786	0.5832\\
0.2833	0.5704\\
0.2881	0.5574\\
0.2928	0.5442\\
0.2975	0.5308\\
0.3022	0.5171\\
0.3069	0.5032\\
0.3117	0.4891\\
0.3164	0.4748\\
0.3211	0.4603\\
0.3258	0.4455\\
0.3306	0.4306\\
0.3353	0.4154\\
0.34	0.4\\
0.3447	0.3844\\
0.3494	0.3685\\
0.3542	0.3525\\
0.3589	0.3362\\
0.3636	0.3197\\
0.3683	0.303\\
0.3731	0.286\\
0.3778	0.2689\\
};
\addplot [color=black, forget plot]
  table[row sep=crcr]{%
0.3778	0.2689\\
0.3815	0.2553\\
0.3852	0.2416\\
0.3926	0.2138\\
0.3973	0.1957\\
0.402	0.1774\\
0.4067	0.1589\\
0.4114	0.1402\\
0.4162	0.1213\\
0.4209	0.1021\\
0.4256	0.0828\\
0.4303	0.0632\\
0.4341	0.0476\\
0.4378	0.0319\\
0.4415	0.016\\
0.4452	0\\
};
\addplot [color=black, forget plot]
  table[row sep=crcr]{%
0.4452	0\\
0.4452	0.0001\\
0.4453	0.0002\\
0.4454	0.0005\\
0.4454	0.0007\\
0.4455	0.001\\
0.4456	0.0012\\
0.446	0.0025\\
0.4464	0.0037\\
0.4468	0.005\\
0.4471	0.0062\\
0.4491	0.0124\\
0.451	0.0186\\
0.453	0.0248\\
0.4549	0.0309\\
0.4579	0.0404\\
0.461	0.0498\\
0.464	0.0591\\
0.467	0.0683\\
0.4701	0.0775\\
0.4731	0.0865\\
0.4762	0.0955\\
0.4822	0.1131\\
0.4853	0.1218\\
0.4883	0.1304\\
0.4913	0.1389\\
0.4944	0.1473\\
0.5004	0.1639\\
0.5035	0.172\\
0.5065	0.1801\\
0.5096	0.188\\
0.5126	0.1959\\
0.5156	0.2037\\
0.5187	0.2114\\
0.5217	0.219\\
0.5247	0.2265\\
0.5278	0.2339\\
0.5338	0.2485\\
0.5369	0.2556\\
0.5399	0.2627\\
0.5429	0.2696\\
0.546	0.2765\\
0.549	0.2833\\
0.5521	0.29\\
0.5551	0.2966\\
0.5581	0.3031\\
0.5612	0.3095\\
0.5642	0.3159\\
0.5648	0.3171\\
0.5654	0.3184\\
0.5661	0.3197\\
0.5667	0.3209\\
};
\addplot [color=black, forget plot]
  table[row sep=crcr]{%
0.5667	0.3209\\
0.5714	0.3305\\
0.5761	0.3398\\
0.5808	0.3489\\
0.5856	0.3578\\
0.5903	0.3665\\
0.595	0.375\\
0.5997	0.3832\\
0.6044	0.3912\\
0.6092	0.3991\\
0.6139	0.4066\\
0.6186	0.414\\
0.6233	0.4212\\
0.6281	0.4281\\
0.6328	0.4348\\
0.6375	0.4413\\
0.6422	0.4476\\
0.6469	0.4536\\
0.6517	0.4595\\
0.6564	0.4651\\
0.6611	0.4705\\
0.6658	0.4757\\
0.6706	0.4806\\
0.6753	0.4854\\
0.68	0.4899\\
0.6847	0.4942\\
0.6894	0.4983\\
0.6942	0.5022\\
0.6989	0.5058\\
0.7036	0.5092\\
0.7083	0.5124\\
0.7131	0.5154\\
0.7178	0.5182\\
0.7225	0.5208\\
0.7272	0.5231\\
0.7319	0.5252\\
0.7367	0.5271\\
0.7414	0.5288\\
0.7461	0.5303\\
0.7508	0.5315\\
0.7556	0.5325\\
};
\addplot [color=black, forget plot]
  table[row sep=crcr]{%
0.7556	0.5325\\
0.7565	0.5327\\
0.7585	0.5331\\
0.7595	0.5332\\
0.7635	0.5338\\
0.7674	0.5341\\
0.7714	0.5344\\
0.7753	0.5344\\
};
\addplot [color=black, forget plot]
  table[row sep=crcr]{%
0.7753	0.5344\\
0.7785	0.5344\\
0.7849	0.534\\
0.7913	0.5332\\
0.7955	0.5324\\
0.7998	0.5315\\
0.804	0.5304\\
0.8082	0.5291\\
0.8124	0.5277\\
0.8167	0.5261\\
0.8251	0.5223\\
0.8294	0.5201\\
0.8336	0.5178\\
0.8378	0.5153\\
0.842	0.5126\\
0.8463	0.5097\\
0.8505	0.5067\\
0.8547	0.5035\\
0.859	0.5001\\
0.8632	0.4966\\
0.8674	0.4928\\
0.8716	0.4889\\
0.8759	0.4848\\
0.8801	0.4806\\
0.8843	0.4761\\
0.8886	0.4715\\
0.8928	0.4667\\
0.897	0.4618\\
0.9012	0.4567\\
0.9055	0.4513\\
0.9097	0.4459\\
0.9139	0.4402\\
0.9182	0.4344\\
0.9266	0.4222\\
0.9311	0.4154\\
0.9355	0.4085\\
0.94	0.4014\\
0.9444	0.3941\\
};
\addplot [color=black, forget plot]
  table[row sep=crcr]{%
0.9444	0.3941\\
0.9492	0.3862\\
0.9539	0.378\\
0.9586	0.3696\\
0.9633	0.361\\
0.9681	0.3522\\
0.9728	0.3432\\
0.9775	0.3339\\
0.9822	0.3244\\
0.9869	0.3148\\
0.9917	0.3048\\
0.9964	0.2947\\
1.0011	0.2844\\
1.0058	0.2738\\
1.0106	0.263\\
1.0153	0.252\\
1.02	0.2408\\
1.0247	0.2293\\
1.0294	0.2177\\
1.0342	0.2058\\
1.0389	0.1937\\
1.0436	0.1814\\
1.0483	0.1688\\
1.0531	0.1561\\
1.0578	0.1431\\
1.0625	0.1299\\
1.0672	0.1165\\
1.0719	0.1028\\
1.0767	0.089\\
1.0814	0.0749\\
1.0861	0.0606\\
1.0908	0.0461\\
1.0956	0.0314\\
1.098	0.0236\\
1.1005	0.0158\\
1.1029	0.0079\\
1.1054	-0\\
};
\addplot [color=black, forget plot]
  table[row sep=crcr]{%
1.1054	0\\
1.1054	0.0001\\
1.1055	0.0002\\
1.1056	0.0005\\
1.1057	0.0007\\
1.1058	0.001\\
1.1059	0.0012\\
1.1064	0.0025\\
1.1069	0.0037\\
1.1074	0.005\\
1.108	0.0062\\
1.1087	0.0079\\
1.1094	0.0095\\
1.1108	0.0129\\
1.1115	0.0145\\
1.1122	0.0162\\
1.1129	0.0178\\
1.1135	0.0195\\
1.1149	0.0227\\
1.1156	0.0244\\
1.1233	0.042\\
1.124	0.0435\\
1.1254	0.0467\\
1.1261	0.0482\\
1.1268	0.0498\\
1.1275	0.0513\\
1.1282	0.0529\\
1.1296	0.0559\\
1.1303	0.0575\\
1.1311	0.0591\\
1.1318	0.0608\\
1.1326	0.0624\\
1.1333	0.064\\
};
\addplot [color=black, forget plot]
  table[row sep=crcr]{%
1.1333	0.064\\
1.1378	0.0736\\
1.1393	0.0767\\
1.144	0.0865\\
1.1487	0.0961\\
1.1535	0.1054\\
1.1582	0.1145\\
1.1629	0.1235\\
1.1676	0.1322\\
1.1724	0.1406\\
1.1771	0.1489\\
1.1818	0.1569\\
1.1865	0.1647\\
1.1912	0.1723\\
1.196	0.1797\\
1.2007	0.1869\\
1.2054	0.1938\\
1.2101	0.2006\\
1.2149	0.2071\\
1.2196	0.2134\\
1.2243	0.2194\\
1.229	0.2253\\
1.2337	0.2309\\
1.2385	0.2363\\
1.2432	0.2415\\
1.2479	0.2465\\
1.2526	0.2513\\
1.2574	0.2558\\
1.2621	0.2601\\
1.2668	0.2642\\
1.2715	0.2681\\
1.2762	0.2718\\
1.281	0.2752\\
1.2857	0.2784\\
1.2904	0.2814\\
1.2951	0.2842\\
1.2999	0.2868\\
1.3046	0.2891\\
1.3093	0.2913\\
1.3125	0.2926\\
1.3158	0.2938\\
1.319	0.295\\
1.3222	0.296\\
};
\addplot [color=black, forget plot]
  table[row sep=crcr]{%
1.3222	0.296\\
1.3238	0.2964\\
1.3253	0.2969\\
1.3269	0.2973\\
1.3284	0.2977\\
1.3331	0.2987\\
1.3378	0.2995\\
1.3426	0.3001\\
1.3473	0.3005\\
1.3487	0.3005\\
1.3501	0.3006\\
1.353	0.3006\\
};
\addplot [color=black, forget plot]
  table[row sep=crcr]{%
1.353	0.3006\\
1.3562	0.3006\\
1.3626	0.3002\\
1.369	0.2994\\
1.3729	0.2987\\
1.3769	0.2978\\
1.3808	0.2968\\
1.3848	0.2957\\
1.3887	0.2944\\
1.3927	0.2929\\
1.3966	0.2913\\
1.4006	0.2895\\
1.4045	0.2876\\
1.4085	0.2855\\
1.4125	0.2833\\
1.4164	0.2809\\
1.4204	0.2783\\
1.4243	0.2757\\
1.4283	0.2728\\
1.4322	0.2698\\
1.4362	0.2667\\
1.4401	0.2634\\
1.4441	0.2599\\
1.448	0.2563\\
1.452	0.2525\\
1.4559	0.2486\\
1.4599	0.2445\\
1.4639	0.2403\\
1.4678	0.2359\\
1.4718	0.2314\\
1.4757	0.2267\\
1.4797	0.2219\\
1.4836	0.2169\\
1.4876	0.2118\\
1.4915	0.2065\\
1.4955	0.201\\
1.4994	0.1955\\
1.5033	0.1898\\
1.5111	0.178\\
};
\addplot [color=black, forget plot]
  table[row sep=crcr]{%
1.5111	0.178\\
1.5158	0.1705\\
1.5206	0.1629\\
1.5253	0.155\\
1.53	0.1469\\
1.5347	0.1386\\
1.5394	0.1301\\
1.5442	0.1213\\
1.5489	0.1123\\
1.5536	0.1032\\
1.5583	0.0938\\
1.5631	0.0841\\
1.5678	0.0743\\
1.5725	0.0642\\
1.5772	0.0539\\
1.5819	0.0435\\
1.5867	0.0327\\
1.5901	0.0247\\
1.5936	0.0166\\
1.5971	0.0084\\
1.6005	-0\\
};
\addplot [color=black, forget plot]
  table[row sep=crcr]{%
1.6005	0\\
1.6006	0.0001\\
1.6006	0.0002\\
1.6008	0.0005\\
1.6009	0.0007\\
1.6011	0.001\\
1.6012	0.0012\\
1.6019	0.0025\\
1.6033	0.0049\\
1.604	0.0062\\
1.6064	0.0106\\
1.6139	0.0235\\
1.6164	0.0276\\
1.6189	0.0318\\
1.6214	0.0358\\
1.6238	0.0398\\
1.6288	0.0476\\
1.6338	0.0552\\
1.6363	0.0588\\
1.6388	0.0625\\
1.6413	0.066\\
1.6437	0.0695\\
1.6462	0.073\\
1.6487	0.0764\\
1.6537	0.083\\
1.6562	0.0862\\
1.6587	0.0893\\
1.6611	0.0924\\
1.6661	0.0984\\
1.6686	0.1013\\
1.6736	0.1069\\
1.6786	0.1123\\
1.681	0.1149\\
1.686	0.1199\\
1.6885	0.1223\\
1.6935	0.1269\\
1.6951	0.1284\\
1.6967	0.1298\\
1.6984	0.1313\\
1.7	0.1326\\
};
\addplot [color=black, forget plot]
  table[row sep=crcr]{%
0	0.95\\
0.0032	0.95\\
};
\addplot [color=black, forget plot]
  table[row sep=crcr]{%
0.0032	0.95\\
0.0064	0.95\\
0.0128	0.9496\\
0.0192	0.9488\\
0.0238	0.948\\
0.0285	0.9469\\
0.0331	0.9457\\
0.0377	0.9442\\
0.0424	0.9425\\
0.047	0.9406\\
0.0517	0.9385\\
0.0563	0.9362\\
0.061	0.9337\\
0.0656	0.9309\\
0.0702	0.928\\
0.0749	0.9248\\
0.0795	0.9215\\
0.0842	0.9179\\
0.0888	0.9141\\
0.0935	0.9101\\
0.0981	0.9058\\
0.1027	0.9014\\
0.1074	0.8968\\
0.112	0.8919\\
0.1167	0.8869\\
0.1213	0.8816\\
0.126	0.8761\\
0.1306	0.8704\\
0.1352	0.8645\\
0.1399	0.8584\\
0.1445	0.852\\
0.1492	0.8455\\
0.1538	0.8387\\
0.1585	0.8318\\
0.1631	0.8246\\
0.1677	0.8172\\
0.1724	0.8096\\
0.177	0.8018\\
0.1817	0.7938\\
0.1863	0.7855\\
0.187	0.7844\\
0.1882	0.782\\
0.1889	0.7809\\
};
\addplot [color=black, forget plot]
  table[row sep=crcr]{%
0.1889	0.7809\\
0.1936	0.7722\\
0.1983	0.7632\\
0.2031	0.7541\\
0.2078	0.7447\\
0.2125	0.7351\\
0.2172	0.7253\\
0.2219	0.7153\\
0.2267	0.705\\
0.2314	0.6946\\
0.2361	0.6839\\
0.2408	0.673\\
0.2456	0.6619\\
0.2503	0.6505\\
0.255	0.639\\
0.2597	0.6272\\
0.2644	0.6152\\
0.2692	0.603\\
0.2739	0.5906\\
0.2786	0.5779\\
0.2833	0.5651\\
0.2881	0.552\\
0.2928	0.5387\\
0.2975	0.5251\\
0.3022	0.5114\\
0.3069	0.4974\\
0.3117	0.4832\\
0.3164	0.4688\\
0.3211	0.4542\\
0.3258	0.4394\\
0.3306	0.4243\\
0.3353	0.4091\\
0.34	0.3936\\
0.3447	0.3779\\
0.3494	0.3619\\
0.3542	0.3458\\
0.3589	0.3294\\
0.3636	0.3128\\
0.3683	0.296\\
0.3731	0.279\\
0.3778	0.2617\\
};
\addplot [color=black, forget plot]
  table[row sep=crcr]{%
0.3778	0.2617\\
0.3814	0.2485\\
0.3849	0.2352\\
0.3885	0.2217\\
0.3921	0.2081\\
0.3968	0.19\\
0.4015	0.1717\\
0.4063	0.1531\\
0.411	0.1343\\
0.4157	0.1153\\
0.4204	0.0961\\
0.4251	0.0767\\
0.4299	0.057\\
0.4332	0.0429\\
0.4366	0.0287\\
0.4399	0.0144\\
0.4433	0\\
};
\addplot [color=black, forget plot]
  table[row sep=crcr]{%
0.4433	0\\
0.4433	0.0002\\
0.4434	0.0005\\
0.4435	0.0007\\
0.4436	0.001\\
0.4436	0.0012\\
0.444	0.0025\\
0.4444	0.0037\\
0.4448	0.005\\
0.4452	0.0062\\
0.4471	0.0124\\
0.4491	0.0186\\
0.451	0.0248\\
0.453	0.0309\\
0.456	0.0405\\
0.4591	0.0501\\
0.4622	0.0596\\
0.4684	0.0782\\
0.4715	0.0874\\
0.4745	0.0965\\
0.4776	0.1055\\
0.4807	0.1144\\
0.4838	0.1232\\
0.4869	0.1319\\
0.49	0.1405\\
0.4931	0.149\\
0.4961	0.1575\\
0.5023	0.1741\\
0.5085	0.1903\\
0.5116	0.1983\\
0.5147	0.2061\\
0.5177	0.2139\\
0.5208	0.2216\\
0.5239	0.2292\\
0.527	0.2367\\
0.5301	0.2441\\
0.5332	0.2514\\
0.5362	0.2586\\
0.5393	0.2658\\
0.5455	0.2798\\
0.5517	0.2934\\
0.5548	0.3\\
0.5578	0.3066\\
0.5609	0.3131\\
0.564	0.3194\\
0.5647	0.3208\\
0.5653	0.3222\\
0.566	0.3235\\
0.5667	0.3249\\
};
\addplot [color=black, forget plot]
  table[row sep=crcr]{%
0.5667	0.3249\\
0.5714	0.3343\\
0.5761	0.3436\\
0.5808	0.3526\\
0.5856	0.3614\\
0.5903	0.37\\
0.595	0.3784\\
0.5997	0.3865\\
0.6044	0.3945\\
0.6092	0.4022\\
0.6139	0.4097\\
0.6186	0.417\\
0.6233	0.424\\
0.6281	0.4309\\
0.6328	0.4375\\
0.6375	0.4439\\
0.6422	0.4501\\
0.6469	0.456\\
0.6517	0.4618\\
0.6564	0.4673\\
0.6611	0.4726\\
0.6658	0.4777\\
0.6706	0.4826\\
0.6753	0.4872\\
0.68	0.4917\\
0.6847	0.4959\\
0.6894	0.4999\\
0.6942	0.5036\\
0.6989	0.5072\\
0.7036	0.5105\\
0.7083	0.5137\\
0.7131	0.5166\\
0.7178	0.5193\\
0.7225	0.5217\\
0.7272	0.524\\
0.7319	0.526\\
0.7367	0.5278\\
0.7414	0.5294\\
0.7461	0.5308\\
0.7508	0.5319\\
0.7556	0.5328\\
};
\addplot [color=black, forget plot]
  table[row sep=crcr]{%
0.7556	0.5328\\
0.7564	0.533\\
0.7573	0.5331\\
0.7582	0.5333\\
0.7627	0.5338\\
0.7662	0.5342\\
0.7734	0.5344\\
};
\addplot [color=black, forget plot]
  table[row sep=crcr]{%
0.7734	0.5344\\
0.7765	0.5344\\
0.7829	0.534\\
0.7861	0.5336\\
0.7894	0.5331\\
0.7936	0.5324\\
0.7979	0.5314\\
0.8022	0.5303\\
0.8065	0.529\\
0.8107	0.5275\\
0.815	0.5259\\
0.8193	0.5241\\
0.8236	0.522\\
0.8278	0.5198\\
0.8321	0.5175\\
0.8364	0.5149\\
0.8407	0.5122\\
0.845	0.5093\\
0.8492	0.5062\\
0.8535	0.5029\\
0.8578	0.4994\\
0.8621	0.4958\\
0.8663	0.492\\
0.8706	0.488\\
0.8749	0.4838\\
0.8792	0.4795\\
0.8835	0.4749\\
0.8877	0.4702\\
0.892	0.4653\\
0.8963	0.4603\\
0.9006	0.455\\
0.9048	0.4496\\
0.9091	0.444\\
0.9134	0.4382\\
0.9177	0.4322\\
0.9219	0.4261\\
0.9262	0.4198\\
0.9308	0.4128\\
0.9353	0.4057\\
0.9399	0.3984\\
0.9444	0.3908\\
};
\addplot [color=black, forget plot]
  table[row sep=crcr]{%
0.9444	0.3908\\
0.9492	0.3828\\
0.9539	0.3745\\
0.9586	0.3661\\
0.9633	0.3574\\
0.9681	0.3485\\
0.9728	0.3393\\
0.9775	0.33\\
0.9822	0.3204\\
0.9869	0.3106\\
0.9917	0.3006\\
0.9964	0.2904\\
1.0011	0.28\\
1.0058	0.2693\\
1.0106	0.2584\\
1.0153	0.2473\\
1.02	0.236\\
1.0247	0.2245\\
1.0294	0.2127\\
1.0342	0.2007\\
1.0389	0.1886\\
1.0436	0.1761\\
1.0483	0.1635\\
1.0531	0.1507\\
1.0578	0.1376\\
1.0625	0.1243\\
1.0672	0.1108\\
1.0719	0.0971\\
1.0767	0.0831\\
1.0814	0.069\\
1.0861	0.0546\\
1.0908	0.04\\
1.0956	0.0252\\
1.0975	0.0189\\
1.0995	0.0127\\
1.1015	0.0064\\
1.1034	-0\\
};
\addplot [color=black, forget plot]
  table[row sep=crcr]{%
1.1034	0\\
1.1034	0.0001\\
1.1035	0.0001\\
1.1035	0.0002\\
1.1036	0.0005\\
1.1037	0.0007\\
1.1038	0.001\\
1.1039	0.0012\\
1.1044	0.0025\\
1.105	0.0037\\
1.1055	0.005\\
1.106	0.0062\\
1.1067	0.008\\
1.1075	0.0098\\
1.1082	0.0116\\
1.109	0.0133\\
1.1097	0.0151\\
1.1105	0.0169\\
1.1112	0.0186\\
1.112	0.0204\\
1.1127	0.0221\\
1.1135	0.0239\\
1.1142	0.0256\\
1.115	0.0274\\
1.1157	0.0291\\
1.1165	0.0308\\
1.1172	0.0325\\
1.118	0.0342\\
1.1187	0.036\\
1.1195	0.0377\\
1.1202	0.0393\\
1.1209	0.041\\
1.1217	0.0427\\
1.1224	0.0444\\
1.1232	0.0461\\
1.1239	0.0477\\
1.1247	0.0494\\
1.1254	0.0511\\
1.1262	0.0527\\
1.1269	0.0544\\
1.1277	0.056\\
1.1284	0.0576\\
1.1292	0.0593\\
1.1299	0.0609\\
1.1307	0.0625\\
1.1314	0.0641\\
1.1322	0.0657\\
1.1329	0.0673\\
1.133	0.0676\\
1.1333	0.0682\\
};
\addplot [color=black, forget plot]
  table[row sep=crcr]{%
1.1333	0.0682\\
1.1349	0.0717\\
1.1365	0.075\\
1.1382	0.0784\\
1.1398	0.0817\\
1.1445	0.0914\\
1.1492	0.1009\\
1.1539	0.1101\\
1.1586	0.1191\\
1.1634	0.1279\\
1.1681	0.1365\\
1.1728	0.1449\\
1.1775	0.153\\
1.1823	0.161\\
1.187	0.1687\\
1.1917	0.1762\\
1.1964	0.1834\\
1.2011	0.1905\\
1.2059	0.1973\\
1.2106	0.2039\\
1.2153	0.2103\\
1.22	0.2165\\
1.2248	0.2224\\
1.2295	0.2282\\
1.2342	0.2337\\
1.2389	0.239\\
1.2436	0.2441\\
1.2531	0.2536\\
1.2578	0.258\\
1.2625	0.2622\\
1.2673	0.2662\\
1.272	0.27\\
1.2767	0.2735\\
1.2814	0.2769\\
1.2861	0.28\\
1.2909	0.2829\\
1.2956	0.2855\\
1.3003	0.288\\
1.305	0.2902\\
1.3098	0.2923\\
1.3129	0.2935\\
1.316	0.2946\\
1.3191	0.2956\\
1.3222	0.2965\\
};
\addplot [color=black, forget plot]
  table[row sep=crcr]{%
1.3222	0.2965\\
1.3237	0.2969\\
1.3251	0.2973\\
1.3266	0.2977\\
1.328	0.298\\
1.3327	0.299\\
1.3374	0.2997\\
1.3422	0.3002\\
1.3469	0.3005\\
1.3479	0.3006\\
1.351	0.3006\\
};
\addplot [color=black, forget plot]
  table[row sep=crcr]{%
1.351	0.3006\\
1.3542	0.3006\\
1.3574	0.3004\\
1.3638	0.2998\\
1.367	0.2993\\
1.371	0.2986\\
1.375	0.2978\\
1.379	0.2968\\
1.383	0.2956\\
1.387	0.2942\\
1.391	0.2927\\
1.395	0.2911\\
1.399	0.2893\\
1.403	0.2873\\
1.407	0.2852\\
1.411	0.2829\\
1.415	0.2805\\
1.419	0.2779\\
1.423	0.2751\\
1.427	0.2722\\
1.431	0.2692\\
1.439	0.2626\\
1.443	0.259\\
1.447	0.2553\\
1.451	0.2515\\
1.4551	0.2475\\
1.4591	0.2433\\
1.4631	0.239\\
1.4671	0.2345\\
1.4711	0.2299\\
1.4751	0.2251\\
1.4791	0.2201\\
1.4831	0.215\\
1.4871	0.2098\\
1.4951	0.1988\\
1.4991	0.193\\
1.5031	0.1871\\
1.5071	0.1811\\
1.5111	0.1748\\
};
\addplot [color=black, forget plot]
  table[row sep=crcr]{%
1.5111	0.1748\\
1.5158	0.1673\\
1.5206	0.1596\\
1.5253	0.1516\\
1.53	0.1434\\
1.5347	0.135\\
1.5394	0.1264\\
1.5442	0.1176\\
1.5489	0.1085\\
1.5536	0.0992\\
1.5583	0.0897\\
1.5631	0.08\\
1.5678	0.0701\\
1.5725	0.0599\\
1.5772	0.0495\\
1.5819	0.039\\
1.5867	0.0282\\
1.5896	0.0212\\
1.5956	0.0072\\
1.5985	-0\\
};
\addplot [color=black, forget plot]
  table[row sep=crcr]{%
1.5985	0\\
1.5986	0.0001\\
1.5986	0.0002\\
1.5987	0.0002\\
1.5988	0.0005\\
1.5989	0.0007\\
1.5991	0.001\\
1.5992	0.0012\\
1.5999	0.0025\\
1.6013	0.0049\\
1.602	0.0062\\
1.6045	0.0107\\
1.607	0.0151\\
1.6096	0.0195\\
1.6146	0.0281\\
1.6172	0.0322\\
1.6197	0.0364\\
1.6223	0.0404\\
1.6273	0.0484\\
1.6299	0.0522\\
1.6349	0.0598\\
1.6375	0.0635\\
1.64	0.0671\\
1.6425	0.0706\\
1.6451	0.0741\\
1.6476	0.0776\\
1.6502	0.0809\\
1.6552	0.0875\\
1.6578	0.0907\\
1.6603	0.0938\\
1.6628	0.0968\\
1.6654	0.0998\\
1.6704	0.1056\\
1.673	0.1084\\
1.6755	0.1111\\
1.6781	0.1138\\
1.6831	0.119\\
1.6857	0.1215\\
1.6882	0.1239\\
1.6907	0.1262\\
1.6933	0.1285\\
1.695	0.13\\
1.6966	0.1315\\
1.7	0.1343\\
};
\addplot [color=black, forget plot]
  table[row sep=crcr]{%
0	0.9932\\
0.0008	0.9932\\
0.001	0.9933\\
0.005	0.9933\\
};
\addplot [color=black, forget plot]
  table[row sep=crcr]{%
0.005	0.9933\\
0.0082	0.9933\\
0.0146	0.9929\\
0.021	0.9921\\
0.0256	0.9913\\
0.0302	0.9902\\
0.0348	0.989\\
0.0394	0.9875\\
0.044	0.9859\\
0.0486	0.984\\
0.0532	0.982\\
0.0578	0.9797\\
0.0624	0.9772\\
0.067	0.9745\\
0.0716	0.9716\\
0.0762	0.9685\\
0.0808	0.9652\\
0.0854	0.9617\\
0.09	0.9579\\
0.0946	0.954\\
0.0992	0.9499\\
0.1038	0.9455\\
0.1084	0.941\\
0.113	0.9362\\
0.1176	0.9312\\
0.1222	0.9261\\
0.1267	0.9207\\
0.1313	0.9151\\
0.1359	0.9093\\
0.1405	0.9033\\
0.1451	0.8971\\
0.1497	0.8906\\
0.1543	0.884\\
0.1589	0.8772\\
0.1635	0.8701\\
0.1681	0.8629\\
0.1727	0.8554\\
0.1773	0.8478\\
0.1819	0.8399\\
0.1865	0.8318\\
0.1871	0.8307\\
0.1877	0.8297\\
0.1889	0.8275\\
};
\addplot [color=black, forget plot]
  table[row sep=crcr]{%
0.1889	0.8275\\
0.1936	0.8189\\
0.1983	0.8101\\
0.2031	0.801\\
0.2078	0.7917\\
0.2125	0.7822\\
0.2172	0.7725\\
0.2219	0.7625\\
0.2267	0.7524\\
0.2314	0.742\\
0.2361	0.7314\\
0.2408	0.7206\\
0.2456	0.7096\\
0.2503	0.6983\\
0.255	0.6868\\
0.2597	0.6752\\
0.2644	0.6632\\
0.2692	0.6511\\
0.2739	0.6388\\
0.2786	0.6262\\
0.2833	0.6134\\
0.2881	0.6004\\
0.2928	0.5872\\
0.2975	0.5738\\
0.3022	0.5601\\
0.3069	0.5462\\
0.3117	0.5321\\
0.3164	0.5178\\
0.3211	0.5033\\
0.3258	0.4885\\
0.3306	0.4736\\
0.3353	0.4584\\
0.34	0.443\\
0.3447	0.4273\\
0.3494	0.4115\\
0.3542	0.3954\\
0.3589	0.3791\\
0.3636	0.3626\\
0.3683	0.3459\\
0.3731	0.329\\
0.3778	0.3118\\
};
\addplot [color=black, forget plot]
  table[row sep=crcr]{%
0.3778	0.3118\\
0.3821	0.2961\\
0.3863	0.2801\\
0.3906	0.264\\
0.3949	0.2477\\
0.3996	0.2296\\
0.4044	0.2112\\
0.4091	0.1926\\
0.4138	0.1737\\
0.4185	0.1547\\
0.4232	0.1354\\
0.428	0.1159\\
0.4327	0.0962\\
0.4374	0.0763\\
0.4421	0.0562\\
0.4469	0.0358\\
0.4516	0.0152\\
0.4524	0.0114\\
0.4542	0.0038\\
0.455	-0\\
};
\addplot [color=black, forget plot]
  table[row sep=crcr]{%
0.455	0\\
0.4551	0.0001\\
0.4551	0.0002\\
0.4552	0.0005\\
0.4553	0.0007\\
0.4553	0.001\\
0.4554	0.0012\\
0.4558	0.0025\\
0.4562	0.0037\\
0.4565	0.005\\
0.4569	0.0062\\
0.4626	0.0248\\
0.4645	0.0309\\
0.4673	0.0399\\
0.4729	0.0575\\
0.4785	0.0749\\
0.4813	0.0834\\
0.484	0.0919\\
0.4868	0.1003\\
0.4924	0.1169\\
0.4952	0.1251\\
0.498	0.1332\\
0.5008	0.1412\\
0.5064	0.157\\
0.5092	0.1648\\
0.512	0.1725\\
0.5147	0.1802\\
0.5203	0.1952\\
0.5259	0.21\\
0.5315	0.2244\\
0.5371	0.2386\\
0.5399	0.2455\\
0.5426	0.2524\\
0.5454	0.2592\\
0.551	0.2726\\
0.5538	0.2792\\
0.5566	0.2857\\
0.5594	0.2921\\
0.5654	0.3056\\
0.5658	0.3066\\
0.5662	0.3075\\
0.5667	0.3085\\
};
\addplot [color=black, forget plot]
  table[row sep=crcr]{%
0.5667	0.3085\\
0.5714	0.3188\\
0.5761	0.329\\
0.5808	0.3389\\
0.5856	0.3486\\
0.5903	0.3581\\
0.595	0.3673\\
0.5997	0.3764\\
0.6044	0.3852\\
0.6092	0.3938\\
0.6139	0.4022\\
0.6186	0.4103\\
0.6233	0.4183\\
0.6281	0.426\\
0.6328	0.4335\\
0.6375	0.4408\\
0.6422	0.4479\\
0.6469	0.4547\\
0.6517	0.4614\\
0.6564	0.4678\\
0.6611	0.474\\
0.6658	0.48\\
0.6706	0.4857\\
0.6753	0.4913\\
0.68	0.4966\\
0.6847	0.5017\\
0.6894	0.5066\\
0.6942	0.5113\\
0.6989	0.5157\\
0.7036	0.5199\\
0.7083	0.524\\
0.7131	0.5277\\
0.7178	0.5313\\
0.7225	0.5347\\
0.7272	0.5378\\
0.7319	0.5407\\
0.7367	0.5434\\
0.7414	0.5459\\
0.7461	0.5482\\
0.7508	0.5502\\
0.7556	0.552\\
};
\addplot [color=black, forget plot]
  table[row sep=crcr]{%
0.7556	0.552\\
0.7574	0.5527\\
0.7593	0.5533\\
0.7611	0.5539\\
0.763	0.5545\\
0.7677	0.5557\\
0.7724	0.5568\\
0.7772	0.5576\\
0.7819	0.5582\\
0.7845	0.5584\\
0.7872	0.5586\\
0.7926	0.5588\\
};
\addplot [color=black, forget plot]
  table[row sep=crcr]{%
0.7926	0.5588\\
0.7929	0.5588\\
0.7931	0.5587\\
0.7958	0.5587\\
0.799	0.5586\\
0.8022	0.5583\\
0.8086	0.5575\\
0.8124	0.5568\\
0.8162	0.556\\
0.8199	0.5551\\
0.8237	0.554\\
0.8313	0.5514\\
0.8351	0.5499\\
0.8389	0.5482\\
0.8427	0.5464\\
0.8465	0.5445\\
0.8503	0.5424\\
0.8541	0.5402\\
0.8579	0.5378\\
0.8617	0.5353\\
0.8693	0.5299\\
0.8769	0.5239\\
0.8845	0.5173\\
0.8883	0.5138\\
0.8921	0.5102\\
0.8959	0.5064\\
0.8997	0.5025\\
0.9035	0.4984\\
0.9073	0.4942\\
0.9149	0.4854\\
0.9225	0.476\\
0.9263	0.4711\\
0.9301	0.466\\
0.9337	0.4611\\
0.9373	0.4561\\
0.9408	0.4509\\
0.9444	0.4456\\
};
\addplot [color=black, forget plot]
  table[row sep=crcr]{%
0.9444	0.4456\\
0.9492	0.4384\\
0.9539	0.4311\\
0.9586	0.4235\\
0.9633	0.4157\\
0.9681	0.4077\\
0.9728	0.3994\\
0.9775	0.391\\
0.9822	0.3823\\
0.9869	0.3734\\
0.9917	0.3643\\
0.9964	0.355\\
1.0011	0.3454\\
1.0058	0.3356\\
1.0106	0.3257\\
1.0153	0.3154\\
1.02	0.305\\
1.0247	0.2944\\
1.0294	0.2835\\
1.0342	0.2724\\
1.0389	0.2611\\
1.0436	0.2496\\
1.0483	0.2379\\
1.0531	0.2259\\
1.0578	0.2137\\
1.0625	0.2013\\
1.0672	0.1887\\
1.0719	0.1759\\
1.0767	0.1628\\
1.0814	0.1496\\
1.0861	0.1361\\
1.0908	0.1224\\
1.0956	0.1084\\
1.1003	0.0943\\
1.105	0.0799\\
1.1097	0.0653\\
1.1144	0.0505\\
1.1184	0.0381\\
1.1223	0.0256\\
1.1262	0.0129\\
1.1301	0\\
};
\addplot [color=black, forget plot]
  table[row sep=crcr]{%
1.1301	0\\
1.1301	0.0002\\
1.1302	0.0002\\
1.1302	0.0004\\
1.1306	0.0012\\
1.1306	0.0014\\
1.131	0.0022\\
1.131	0.0024\\
1.1315	0.0034\\
1.1315	0.0036\\
1.1319	0.0044\\
1.1319	0.0046\\
1.1324	0.0056\\
1.1324	0.0058\\
1.1328	0.0066\\
1.1328	0.0068\\
1.1331	0.0074\\
1.1331	0.0076\\
1.1332	0.0077\\
1.1333	0.0079\\
1.1333	0.0081\\
};
\addplot [color=black, forget plot]
  table[row sep=crcr]{%
1.1333	0.0081\\
1.1337	0.0089\\
1.1338	0.0093\\
1.134	0.0097\\
1.1356	0.0137\\
1.1365	0.0157\\
1.1373	0.0177\\
1.1414	0.0275\\
1.1455	0.0372\\
1.1497	0.0468\\
1.1538	0.0561\\
1.1585	0.0667\\
1.1632	0.077\\
1.168	0.087\\
1.1727	0.0969\\
1.1774	0.1065\\
1.1821	0.116\\
1.1868	0.1252\\
1.1916	0.1342\\
1.1963	0.1429\\
1.201	0.1515\\
1.2057	0.1598\\
1.2105	0.1679\\
1.2152	0.1758\\
1.2199	0.1835\\
1.2246	0.1909\\
1.2293	0.1982\\
1.2341	0.2052\\
1.2388	0.212\\
1.2435	0.2186\\
1.2482	0.2249\\
1.253	0.2311\\
1.2577	0.237\\
1.2624	0.2427\\
1.2671	0.2482\\
1.2718	0.2535\\
1.2766	0.2585\\
1.2813	0.2634\\
1.286	0.268\\
1.2907	0.2724\\
1.2955	0.2765\\
1.3002	0.2805\\
1.3049	0.2842\\
1.3092	0.2875\\
1.3136	0.2905\\
1.3179	0.2934\\
1.3222	0.2961\\
};
\addplot [color=black, forget plot]
  table[row sep=crcr]{%
1.3222	0.2961\\
1.3253	0.2978\\
1.3283	0.2995\\
1.3345	0.3027\\
1.3392	0.3048\\
1.3439	0.3067\\
1.3486	0.3084\\
1.3534	0.3099\\
1.3581	0.3112\\
1.3628	0.3123\\
1.3675	0.3131\\
1.3723	0.3137\\
1.375	0.314\\
1.3777	0.3142\\
1.3805	0.3143\\
1.3832	0.3143\\
};
\addplot [color=black, forget plot]
  table[row sep=crcr]{%
1.3832	0.3143\\
1.3858	0.3143\\
1.3864	0.3142\\
1.3896	0.3141\\
1.396	0.3135\\
1.4024	0.3125\\
1.4088	0.3111\\
1.4152	0.3093\\
1.4216	0.3071\\
1.428	0.3045\\
1.4344	0.3015\\
1.4376	0.2998\\
1.444	0.2962\\
1.4504	0.2922\\
1.4536	0.29\\
1.4567	0.2878\\
1.4631	0.283\\
1.4663	0.2804\\
1.4727	0.275\\
1.4791	0.2692\\
1.4823	0.2661\\
1.4887	0.2597\\
1.4951	0.2529\\
1.4983	0.2493\\
1.5015	0.2456\\
1.5063	0.24\\
1.5087	0.237\\
1.5111	0.2341\\
};
\addplot [color=black, forget plot]
  table[row sep=crcr]{%
1.5111	0.2341\\
1.5158	0.228\\
1.5206	0.2218\\
1.5253	0.2153\\
1.53	0.2086\\
1.5347	0.2017\\
1.5394	0.1946\\
1.5442	0.1872\\
1.5489	0.1796\\
1.5536	0.1719\\
1.5583	0.1639\\
1.5631	0.1556\\
1.5678	0.1472\\
1.5725	0.1385\\
1.5772	0.1297\\
1.5819	0.1206\\
1.5867	0.1112\\
1.5914	0.1017\\
1.5961	0.092\\
1.6008	0.082\\
1.6056	0.0718\\
1.6103	0.0614\\
1.615	0.0508\\
1.6197	0.0399\\
1.6244	0.0288\\
1.6274	0.0218\\
1.6304	0.0146\\
1.6334	0.0073\\
1.6363	-0\\
};
\addplot [color=black, forget plot]
  table[row sep=crcr]{%
1.6363	0\\
1.6364	0.0001\\
1.6364	0.0002\\
1.6366	0.0005\\
1.6367	0.0007\\
1.6369	0.001\\
1.637	0.0012\\
1.6377	0.0025\\
1.6383	0.0037\\
1.639	0.0049\\
1.6397	0.0062\\
1.6413	0.0091\\
1.6429	0.0119\\
1.6445	0.0148\\
1.646	0.0176\\
1.6492	0.0232\\
1.6508	0.0259\\
1.6524	0.0287\\
1.6556	0.0341\\
1.6636	0.0471\\
1.6651	0.0496\\
1.6683	0.0546\\
1.6763	0.0666\\
1.6811	0.0735\\
1.6827	0.0757\\
1.6842	0.078\\
1.6858	0.0802\\
1.6874	0.0823\\
1.689	0.0845\\
1.6954	0.0929\\
1.697	0.0949\\
1.6977	0.0959\\
1.6985	0.0968\\
1.6992	0.0977\\
1.7	0.0987\\
};
\addplot [color=black, forget plot]
  table[row sep=crcr]{%
0	0.9996\\
0.0018	0.9996\\
0.0054	0.9994\\
0.0073	0.9993\\
0.0091	0.9991\\
0.0139	0.9985\\
0.0186	0.9977\\
0.0233	0.9967\\
0.028	0.9954\\
0.0327	0.9939\\
0.0375	0.9922\\
0.0422	0.9903\\
0.0469	0.9882\\
0.0516	0.9858\\
0.0564	0.9833\\
0.0611	0.9805\\
0.0658	0.9775\\
0.0705	0.9742\\
0.0752	0.9708\\
0.08	0.9671\\
0.0847	0.9633\\
0.0894	0.9592\\
0.0941	0.9548\\
0.0989	0.9503\\
0.1036	0.9455\\
0.1083	0.9406\\
0.113	0.9354\\
0.1177	0.93\\
0.1225	0.9243\\
0.1272	0.9185\\
0.1319	0.9124\\
0.1366	0.9061\\
0.1414	0.8996\\
0.1461	0.8929\\
0.1508	0.8859\\
0.1555	0.8788\\
0.1602	0.8714\\
0.165	0.8638\\
0.1697	0.856\\
0.1744	0.8479\\
0.1791	0.8397\\
0.1816	0.8353\\
0.1864	0.8265\\
0.1889	0.8219\\
};
\addplot [color=black, forget plot]
  table[row sep=crcr]{%
0.1889	0.8219\\
0.1936	0.813\\
0.1983	0.8039\\
0.2031	0.7945\\
0.2078	0.7849\\
0.2125	0.7751\\
0.2172	0.7651\\
0.2219	0.7548\\
0.2267	0.7444\\
0.2314	0.7337\\
0.2361	0.7228\\
0.2408	0.7117\\
0.2456	0.7004\\
0.2503	0.6888\\
0.255	0.677\\
0.2597	0.665\\
0.2644	0.6528\\
0.2692	0.6404\\
0.2739	0.6278\\
0.2786	0.6149\\
0.2833	0.6018\\
0.2881	0.5885\\
0.2928	0.575\\
0.2975	0.5612\\
0.3022	0.5473\\
0.3069	0.5331\\
0.3117	0.5187\\
0.3164	0.5041\\
0.3211	0.4893\\
0.3258	0.4742\\
0.3306	0.4589\\
0.3353	0.4435\\
0.34	0.4277\\
0.3447	0.4118\\
0.3494	0.3957\\
0.3542	0.3793\\
0.3589	0.3627\\
0.3636	0.3459\\
0.3683	0.3289\\
0.3731	0.3117\\
0.3778	0.2942\\
};
\addplot [color=black, forget plot]
  table[row sep=crcr]{%
0.3778	0.2942\\
0.3818	0.2793\\
0.3857	0.2643\\
0.3897	0.2492\\
0.3937	0.2338\\
0.3984	0.2154\\
0.4031	0.1968\\
0.4078	0.1779\\
0.4126	0.1589\\
0.4173	0.1396\\
0.422	0.1201\\
0.4267	0.1003\\
0.4314	0.0804\\
0.4361	0.0606\\
0.4407	0.0406\\
0.4453	0.0204\\
0.45	-0\\
};
\addplot [color=black, forget plot]
  table[row sep=crcr]{%
0.45	0\\
0.45	0.0002\\
0.4501	0.0005\\
0.4502	0.0007\\
0.4503	0.001\\
0.4503	0.0012\\
0.4507	0.0025\\
0.4511	0.0037\\
0.4515	0.005\\
0.4519	0.0062\\
0.4537	0.0124\\
0.4575	0.0248\\
0.4594	0.0309\\
0.4623	0.0403\\
0.4653	0.0496\\
0.4682	0.0588\\
0.474	0.077\\
0.4798	0.0948\\
0.4828	0.1036\\
0.4857	0.1123\\
0.4915	0.1295\\
0.4973	0.1463\\
0.5003	0.1546\\
0.5032	0.1628\\
0.509	0.179\\
0.5148	0.1948\\
0.5178	0.2026\\
0.5207	0.2103\\
0.5265	0.2255\\
0.5323	0.2403\\
0.5353	0.2476\\
0.5382	0.2548\\
0.544	0.269\\
0.5469	0.2759\\
0.5499	0.2828\\
0.5528	0.2896\\
0.5557	0.2963\\
0.5615	0.3095\\
0.5644	0.3159\\
0.565	0.3171\\
0.5656	0.3184\\
0.5661	0.3196\\
0.5667	0.3208\\
};
\addplot [color=black, forget plot]
  table[row sep=crcr]{%
0.5667	0.3208\\
0.5714	0.331\\
0.5761	0.3409\\
0.5808	0.3506\\
0.5856	0.3602\\
0.5903	0.3695\\
0.595	0.3785\\
0.5997	0.3874\\
0.6044	0.396\\
0.6092	0.4044\\
0.6139	0.4126\\
0.6186	0.4206\\
0.6233	0.4284\\
0.6281	0.4359\\
0.6328	0.4433\\
0.6375	0.4504\\
0.6422	0.4573\\
0.6469	0.4639\\
0.6517	0.4704\\
0.6564	0.4766\\
0.6611	0.4826\\
0.6658	0.4884\\
0.6706	0.494\\
0.6753	0.4994\\
0.68	0.5045\\
0.6847	0.5094\\
0.6894	0.5141\\
0.6942	0.5186\\
0.6989	0.5229\\
0.7036	0.5269\\
0.7083	0.5307\\
0.7131	0.5343\\
0.7178	0.5377\\
0.7225	0.5409\\
0.7272	0.5438\\
0.7319	0.5466\\
0.7367	0.5491\\
0.7414	0.5514\\
0.7461	0.5535\\
0.7508	0.5553\\
0.7556	0.557\\
};
\addplot [color=black, forget plot]
  table[row sep=crcr]{%
0.7556	0.557\\
0.7572	0.5575\\
0.7589	0.558\\
0.7605	0.5585\\
0.7622	0.5589\\
0.7669	0.56\\
0.7716	0.5609\\
0.7764	0.5616\\
0.7811	0.562\\
0.7829	0.5622\\
0.7848	0.5622\\
0.7867	0.5623\\
0.7886	0.5623\\
};
\addplot [color=black, forget plot]
  table[row sep=crcr]{%
0.7886	0.5623\\
0.7918	0.5623\\
0.7982	0.5619\\
0.8046	0.5611\\
0.8085	0.5604\\
0.8124	0.5595\\
0.8163	0.5585\\
0.8202	0.5574\\
0.824	0.5561\\
0.8279	0.5547\\
0.8318	0.5531\\
0.8357	0.5514\\
0.8396	0.5495\\
0.8435	0.5475\\
0.8474	0.5453\\
0.8513	0.543\\
0.8552	0.5405\\
0.8591	0.5379\\
0.863	0.5351\\
0.8669	0.5322\\
0.8708	0.5291\\
0.8747	0.5259\\
0.8786	0.5225\\
0.8825	0.519\\
0.8864	0.5154\\
0.8942	0.5076\\
0.8981	0.5035\\
0.902	0.4992\\
0.9059	0.4948\\
0.9098	0.4902\\
0.9137	0.4855\\
0.9176	0.4807\\
0.9215	0.4757\\
0.9254	0.4705\\
0.9293	0.4652\\
0.9331	0.4599\\
0.9407	0.4489\\
0.9444	0.4431\\
};
\addplot [color=black, forget plot]
  table[row sep=crcr]{%
0.9444	0.4431\\
0.9492	0.4358\\
0.9539	0.4282\\
0.9586	0.4205\\
0.9633	0.4125\\
0.9681	0.4043\\
0.9728	0.3959\\
0.9775	0.3872\\
0.9822	0.3784\\
0.9869	0.3693\\
0.9917	0.36\\
0.9964	0.3505\\
1.0011	0.3407\\
1.0058	0.3308\\
1.0106	0.3206\\
1.0153	0.3102\\
1.02	0.2996\\
1.0247	0.2888\\
1.0294	0.2777\\
1.0342	0.2664\\
1.0389	0.255\\
1.0436	0.2432\\
1.0483	0.2313\\
1.0531	0.2192\\
1.0578	0.2068\\
1.0625	0.1942\\
1.0672	0.1814\\
1.0719	0.1684\\
1.0767	0.1552\\
1.0814	0.1417\\
1.0861	0.1281\\
1.0908	0.1142\\
1.0956	0.1\\
1.1003	0.0857\\
1.105	0.0712\\
1.1097	0.0564\\
1.1144	0.0414\\
1.1176	0.0312\\
1.1208	0.0209\\
1.124	0.0105\\
1.1272	-0\\
};
\addplot [color=black, forget plot]
  table[row sep=crcr]{%
1.1272	0\\
1.1272	0.0002\\
1.1273	0.0005\\
1.1274	0.0007\\
1.1275	0.001\\
1.1278	0.0016\\
1.1279	0.002\\
1.1281	0.0024\\
1.1283	0.0027\\
1.1284	0.0031\\
1.1286	0.0035\\
1.1287	0.0039\\
1.1289	0.0043\\
1.129	0.0047\\
1.1292	0.005\\
1.1293	0.0054\\
1.1295	0.0058\\
1.1296	0.0062\\
1.1298	0.0066\\
1.13	0.0069\\
1.1301	0.0073\\
1.1303	0.0077\\
1.1304	0.0081\\
1.1306	0.0085\\
1.1307	0.0088\\
1.1309	0.0092\\
1.131	0.0096\\
1.1312	0.01\\
1.1313	0.0104\\
1.1315	0.0107\\
1.1317	0.0111\\
1.1318	0.0115\\
1.132	0.0119\\
1.1321	0.0123\\
1.1323	0.0126\\
1.1324	0.013\\
1.1326	0.0134\\
1.1327	0.0138\\
1.1329	0.0141\\
1.133	0.0145\\
1.1332	0.0149\\
1.1332	0.015\\
1.1333	0.0151\\
1.1333	0.0152\\
};
\addplot [color=black, forget plot]
  table[row sep=crcr]{%
1.1333	0.0152\\
1.1336	0.016\\
1.134	0.0167\\
1.1346	0.0183\\
1.1362	0.0221\\
1.1377	0.0258\\
1.1393	0.0296\\
1.1409	0.0333\\
1.1456	0.0443\\
1.1503	0.0551\\
1.155	0.0657\\
1.1598	0.076\\
1.1645	0.0862\\
1.1692	0.0961\\
1.1739	0.1058\\
1.1787	0.1153\\
1.1834	0.1246\\
1.1881	0.1336\\
1.1928	0.1424\\
1.1975	0.1511\\
1.2023	0.1595\\
1.207	0.1676\\
1.2117	0.1756\\
1.2164	0.1833\\
1.2212	0.1908\\
1.2259	0.1981\\
1.2306	0.2052\\
1.2353	0.2121\\
1.24	0.2187\\
1.2448	0.2251\\
1.2495	0.2314\\
1.2542	0.2373\\
1.2589	0.2431\\
1.2637	0.2487\\
1.2684	0.254\\
1.2731	0.2591\\
1.2778	0.264\\
1.2825	0.2687\\
1.2873	0.2731\\
1.292	0.2774\\
1.2967	0.2814\\
1.3014	0.2852\\
1.3062	0.2888\\
1.3109	0.2921\\
1.3137	0.294\\
1.3166	0.2959\\
1.3222	0.2993\\
};
\addplot [color=black, forget plot]
  table[row sep=crcr]{%
1.3222	0.2993\\
1.3252	0.301\\
1.3281	0.3025\\
1.3311	0.304\\
1.3341	0.3054\\
1.3388	0.3075\\
1.3435	0.3094\\
1.3482	0.311\\
1.3529	0.3124\\
1.3577	0.3136\\
1.3624	0.3146\\
1.3671	0.3153\\
1.3718	0.3159\\
1.3741	0.3161\\
1.3765	0.3162\\
1.3788	0.3163\\
1.3811	0.3163\\
};
\addplot [color=black, forget plot]
  table[row sep=crcr]{%
1.3811	0.3163\\
1.3836	0.3163\\
1.3843	0.3162\\
1.3875	0.3161\\
1.3939	0.3155\\
1.4003	0.3145\\
1.4036	0.3138\\
1.4068	0.313\\
1.4101	0.3122\\
1.4133	0.3112\\
1.4166	0.3101\\
1.4198	0.3089\\
1.4231	0.3076\\
1.4263	0.3063\\
1.4296	0.3048\\
1.4328	0.3032\\
1.4361	0.3015\\
1.4393	0.2997\\
1.4426	0.2977\\
1.4458	0.2957\\
1.4491	0.2936\\
1.4523	0.2914\\
1.4556	0.2891\\
1.4588	0.2866\\
1.4621	0.2841\\
1.4653	0.2815\\
1.4686	0.2787\\
1.4718	0.2759\\
1.4784	0.2699\\
1.4816	0.2667\\
1.4849	0.2635\\
1.4881	0.2601\\
1.4914	0.2567\\
1.4946	0.2531\\
1.4979	0.2494\\
1.5011	0.2456\\
1.5036	0.2427\\
1.5111	0.2334\\
};
\addplot [color=black, forget plot]
  table[row sep=crcr]{%
1.5111	0.2334\\
1.5158	0.2272\\
1.5206	0.2209\\
1.5253	0.2143\\
1.53	0.2075\\
1.5347	0.2005\\
1.5394	0.1933\\
1.5442	0.1859\\
1.5489	0.1782\\
1.5536	0.1703\\
1.5583	0.1622\\
1.5631	0.1539\\
1.5678	0.1453\\
1.5725	0.1366\\
1.5772	0.1276\\
1.5819	0.1184\\
1.5867	0.109\\
1.5914	0.0994\\
1.5961	0.0895\\
1.6008	0.0794\\
1.6056	0.0692\\
1.6103	0.0587\\
1.615	0.0479\\
1.6197	0.037\\
1.6244	0.0258\\
1.6271	0.0195\\
1.6297	0.013\\
1.6324	0.0066\\
1.635	-0\\
};
\addplot [color=black, forget plot]
  table[row sep=crcr]{%
1.635	0\\
1.6351	0.0001\\
1.6351	0.0002\\
1.6353	0.0005\\
1.6354	0.0007\\
1.6355	0.001\\
1.6357	0.0012\\
1.6363	0.0025\\
1.6377	0.0049\\
1.6384	0.0062\\
1.64	0.0091\\
1.6416	0.0121\\
1.6432	0.015\\
1.6449	0.0179\\
1.6465	0.0208\\
1.6497	0.0264\\
1.6514	0.0292\\
1.653	0.032\\
1.6562	0.0374\\
1.6579	0.0401\\
1.6595	0.0427\\
1.6611	0.0454\\
1.6627	0.048\\
1.6644	0.0506\\
1.666	0.0531\\
1.6676	0.0557\\
1.6692	0.0582\\
1.6708	0.0606\\
1.6725	0.0631\\
1.6773	0.0703\\
1.679	0.0726\\
1.6838	0.0795\\
1.6855	0.0818\\
1.6887	0.0862\\
1.6903	0.0883\\
1.692	0.0905\\
1.6952	0.0947\\
1.7	0.1007\\
};
\addplot [color=black, forget plot]
  table[row sep=crcr]{%
0	0.95\\
0.0008	0.95\\
0.001	0.9499\\
0.0023	0.9499\\
0.0036	0.9498\\
0.0049	0.9496\\
0.0061	0.9495\\
0.0109	0.9489\\
0.0156	0.948\\
0.0203	0.947\\
0.025	0.9457\\
0.0298	0.9442\\
0.0345	0.9424\\
0.0392	0.9405\\
0.0439	0.9383\\
0.0486	0.936\\
0.0534	0.9334\\
0.0581	0.9305\\
0.0628	0.9275\\
0.0675	0.9243\\
0.0723	0.9208\\
0.077	0.9171\\
0.0817	0.9132\\
0.0864	0.909\\
0.0911	0.9047\\
0.0959	0.9001\\
0.1006	0.8953\\
0.1053	0.8903\\
0.11	0.8851\\
0.1148	0.8797\\
0.1195	0.874\\
0.1242	0.8681\\
0.1289	0.862\\
0.1336	0.8557\\
0.1384	0.8492\\
0.1431	0.8424\\
0.1478	0.8354\\
0.1525	0.8283\\
0.1573	0.8208\\
0.162	0.8132\\
0.1667	0.8054\\
0.1714	0.7973\\
0.1761	0.789\\
0.1793	0.7833\\
0.1825	0.7775\\
0.1857	0.7716\\
0.1889	0.7656\\
};
\addplot [color=black, forget plot]
  table[row sep=crcr]{%
0.1889	0.7656\\
0.1936	0.7565\\
0.1983	0.7471\\
0.2031	0.7376\\
0.2078	0.7279\\
0.2125	0.7179\\
0.2172	0.7077\\
0.2219	0.6973\\
0.2267	0.6867\\
0.2314	0.6758\\
0.2361	0.6647\\
0.2408	0.6535\\
0.2456	0.642\\
0.2503	0.6302\\
0.255	0.6183\\
0.2597	0.6061\\
0.2644	0.5938\\
0.2692	0.5812\\
0.2739	0.5684\\
0.2786	0.5553\\
0.2833	0.5421\\
0.2881	0.5286\\
0.2928	0.5149\\
0.2975	0.501\\
0.3022	0.4869\\
0.3069	0.4725\\
0.3117	0.458\\
0.3164	0.4432\\
0.3211	0.4282\\
0.3258	0.413\\
0.3306	0.3975\\
0.3353	0.3819\\
0.34	0.366\\
0.3447	0.3499\\
0.3494	0.3336\\
0.3542	0.317\\
0.3589	0.3003\\
0.3636	0.2833\\
0.3683	0.2661\\
0.3731	0.2487\\
0.3778	0.2311\\
};
\addplot [color=black, forget plot]
  table[row sep=crcr]{%
0.3778	0.2311\\
0.384	0.2077\\
0.3871	0.1958\\
0.3901	0.1839\\
0.3949	0.1655\\
0.3996	0.1468\\
0.4043	0.128\\
0.409	0.1089\\
0.4138	0.0896\\
0.4185	0.0701\\
0.4232	0.0504\\
0.4279	0.0304\\
0.4297	0.0229\\
0.4315	0.0153\\
0.4332	0.0077\\
0.435	-0\\
};
\addplot [color=black, forget plot]
  table[row sep=crcr]{%
0.435	0\\
0.435	0.0001\\
0.4351	0.0001\\
0.4351	0.0002\\
0.4352	0.0005\\
0.4352	0.0007\\
0.4353	0.001\\
0.4354	0.0012\\
0.4358	0.0025\\
0.4362	0.0037\\
0.4366	0.005\\
0.4369	0.0062\\
0.4389	0.0124\\
0.4408	0.0186\\
0.4428	0.0248\\
0.4447	0.0309\\
0.448	0.0412\\
0.4513	0.0514\\
0.4579	0.0714\\
0.4612	0.0813\\
0.4645	0.091\\
0.4677	0.1007\\
0.4743	0.1197\\
0.4809	0.1383\\
0.4842	0.1474\\
0.4875	0.1564\\
0.4908	0.1653\\
0.4941	0.1741\\
0.4974	0.1828\\
0.5007	0.1914\\
0.5039	0.1999\\
0.5072	0.2083\\
0.5138	0.2247\\
0.5171	0.2328\\
0.5237	0.2486\\
0.527	0.2563\\
0.5336	0.2715\\
0.5369	0.2789\\
0.5401	0.2862\\
0.5434	0.2934\\
0.5467	0.3005\\
0.55	0.3075\\
0.5533	0.3144\\
0.5566	0.3212\\
0.5599	0.3279\\
0.5632	0.3344\\
0.5641	0.3362\\
0.5649	0.3379\\
0.5667	0.3413\\
};
\addplot [color=black, forget plot]
  table[row sep=crcr]{%
0.5667	0.3413\\
0.5714	0.3504\\
0.5761	0.3592\\
0.5808	0.3679\\
0.5856	0.3763\\
0.5903	0.3845\\
0.595	0.3925\\
0.5997	0.4003\\
0.6044	0.4078\\
0.6092	0.4152\\
0.6139	0.4223\\
0.6186	0.4292\\
0.6233	0.4358\\
0.6281	0.4423\\
0.6328	0.4485\\
0.6375	0.4546\\
0.6422	0.4604\\
0.6469	0.466\\
0.6517	0.4713\\
0.6564	0.4765\\
0.6611	0.4814\\
0.6658	0.4861\\
0.6706	0.4906\\
0.6753	0.4949\\
0.68	0.4989\\
0.6847	0.5027\\
0.6894	0.5064\\
0.6942	0.5098\\
0.6989	0.5129\\
0.7036	0.5159\\
0.7083	0.5186\\
0.7131	0.5212\\
0.7178	0.5235\\
0.7225	0.5255\\
0.7272	0.5274\\
0.7319	0.5291\\
0.7367	0.5305\\
0.7414	0.5317\\
0.7461	0.5327\\
0.7508	0.5334\\
0.7556	0.534\\
};
\addplot [color=black, forget plot]
  table[row sep=crcr]{%
0.7556	0.534\\
0.756	0.534\\
0.7565	0.5341\\
0.757	0.5341\\
0.7575	0.5342\\
0.7613	0.5344\\
0.7651	0.5344\\
};
\addplot [color=black, forget plot]
  table[row sep=crcr]{%
0.7651	0.5344\\
0.7683	0.5344\\
0.7747	0.534\\
0.7811	0.5332\\
0.7856	0.5324\\
0.7901	0.5314\\
0.7946	0.5302\\
0.799	0.5288\\
0.8035	0.5272\\
0.808	0.5254\\
0.8125	0.5234\\
0.817	0.5213\\
0.8215	0.5189\\
0.8259	0.5163\\
0.8304	0.5135\\
0.8349	0.5106\\
0.8394	0.5074\\
0.8439	0.504\\
0.8484	0.5005\\
0.8528	0.4967\\
0.8573	0.4927\\
0.8618	0.4886\\
0.8663	0.4842\\
0.8708	0.4797\\
0.8753	0.4749\\
0.8797	0.47\\
0.8842	0.4649\\
0.8887	0.4595\\
0.8932	0.454\\
0.8977	0.4483\\
0.9022	0.4423\\
0.9066	0.4362\\
0.9111	0.4299\\
0.9156	0.4234\\
0.9201	0.4166\\
0.9246	0.4097\\
0.9291	0.4026\\
0.9335	0.3953\\
0.938	0.3878\\
0.9425	0.3801\\
0.943	0.3792\\
0.944	0.3776\\
0.9444	0.3767\\
};
\addplot [color=black, forget plot]
  table[row sep=crcr]{%
0.9444	0.3767\\
0.9492	0.3683\\
0.9539	0.3597\\
0.9586	0.3508\\
0.9633	0.3417\\
0.9681	0.3324\\
0.9728	0.3229\\
0.9775	0.3132\\
0.9822	0.3032\\
0.9869	0.2931\\
0.9917	0.2827\\
0.9964	0.2721\\
1.0011	0.2613\\
1.0058	0.2502\\
1.0106	0.239\\
1.0153	0.2275\\
1.02	0.2158\\
1.0247	0.2039\\
1.0294	0.1917\\
1.0342	0.1794\\
1.0389	0.1668\\
1.0436	0.154\\
1.0483	0.141\\
1.0531	0.1278\\
1.0578	0.1143\\
1.0625	0.1007\\
1.0672	0.0868\\
1.0767	0.0583\\
1.0813	0.0441\\
1.0859	0.0296\\
1.0906	0.0149\\
1.0952	-0\\
};
\addplot [color=black, forget plot]
  table[row sep=crcr]{%
1.0952	0\\
1.0952	0.0001\\
1.0953	0.0002\\
1.0954	0.0005\\
1.0955	0.0007\\
1.0956	0.001\\
1.0957	0.0012\\
1.0962	0.0025\\
1.0967	0.0037\\
1.0973	0.005\\
1.0978	0.0062\\
1.0987	0.0085\\
1.0997	0.0108\\
1.1006	0.013\\
1.1016	0.0153\\
1.1025	0.0175\\
1.1035	0.0198\\
1.1044	0.022\\
1.1054	0.0242\\
1.1063	0.0265\\
1.1083	0.0309\\
1.1092	0.033\\
1.1102	0.0352\\
1.1111	0.0374\\
1.1121	0.0396\\
1.113	0.0417\\
1.114	0.0439\\
1.1149	0.046\\
1.1159	0.0481\\
1.1168	0.0502\\
1.1178	0.0523\\
1.1187	0.0544\\
1.1197	0.0565\\
1.1206	0.0586\\
1.1226	0.0628\\
1.1235	0.0648\\
1.1245	0.0669\\
1.1254	0.0689\\
1.1264	0.0709\\
1.1273	0.0729\\
1.1283	0.0749\\
1.1292	0.0769\\
1.1302	0.0789\\
1.1311	0.0809\\
1.1324	0.0835\\
1.1327	0.0842\\
1.133	0.0848\\
1.1333	0.0855\\
};
\addplot [color=black, forget plot]
  table[row sep=crcr]{%
1.1333	0.0855\\
1.1354	0.0897\\
1.1375	0.094\\
1.1396	0.0982\\
1.1417	0.1023\\
1.1464	0.1115\\
1.1511	0.1205\\
1.1559	0.1293\\
1.1606	0.1378\\
1.1653	0.1461\\
1.17	0.1543\\
1.1747	0.1621\\
1.1795	0.1698\\
1.1842	0.1773\\
1.1889	0.1845\\
1.1936	0.1915\\
1.1984	0.1983\\
1.2031	0.2049\\
1.2078	0.2113\\
1.2125	0.2174\\
1.2172	0.2233\\
1.222	0.229\\
1.2267	0.2345\\
1.2314	0.2398\\
1.2361	0.2449\\
1.2409	0.2497\\
1.2456	0.2543\\
1.2503	0.2587\\
1.255	0.2629\\
1.2597	0.2668\\
1.2645	0.2706\\
1.2692	0.2741\\
1.2739	0.2774\\
1.2786	0.2805\\
1.2834	0.2833\\
1.2881	0.286\\
1.2928	0.2884\\
1.2975	0.2906\\
1.3022	0.2926\\
1.307	0.2943\\
1.3117	0.2959\\
1.3143	0.2967\\
1.317	0.2974\\
1.3222	0.2986\\
};
\addplot [color=black, forget plot]
  table[row sep=crcr]{%
1.3222	0.2986\\
1.3233	0.2988\\
1.3243	0.299\\
1.3253	0.2991\\
1.3264	0.2993\\
1.3305	0.2999\\
1.3346	0.3003\\
1.3387	0.3005\\
1.3428	0.3006\\
};
\addplot [color=black, forget plot]
  table[row sep=crcr]{%
1.3428	0.3006\\
1.346	0.3006\\
1.3524	0.3002\\
1.3588	0.2994\\
1.363	0.2986\\
1.3672	0.2977\\
1.3714	0.2966\\
1.3756	0.2953\\
1.3798	0.2939\\
1.384	0.2923\\
1.3882	0.2905\\
1.3924	0.2885\\
1.3966	0.2864\\
1.4009	0.2841\\
1.4051	0.2816\\
1.4093	0.2789\\
1.4135	0.2761\\
1.4177	0.2731\\
1.4219	0.2699\\
1.4261	0.2666\\
1.4303	0.263\\
1.4345	0.2593\\
1.4387	0.2555\\
1.4429	0.2514\\
1.4471	0.2472\\
1.4514	0.2428\\
1.4556	0.2382\\
1.4598	0.2335\\
1.464	0.2286\\
1.4682	0.2235\\
1.4724	0.2182\\
1.4766	0.2128\\
1.4808	0.2072\\
1.485	0.2014\\
1.4892	0.1954\\
1.4934	0.1893\\
1.4979	0.1827\\
1.5023	0.1758\\
1.5067	0.1688\\
1.5111	0.1616\\
};
\addplot [color=black, forget plot]
  table[row sep=crcr]{%
1.5111	0.1616\\
1.5158	0.1537\\
1.5206	0.1456\\
1.5253	0.1372\\
1.53	0.1287\\
1.5347	0.1199\\
1.5394	0.1109\\
1.5442	0.1017\\
1.5489	0.0922\\
1.5536	0.0826\\
1.5583	0.0727\\
1.5631	0.0626\\
1.5678	0.0523\\
1.5725	0.0418\\
1.5772	0.031\\
1.5819	0.02\\
1.5867	0.0089\\
1.5876	0.0067\\
1.5885	0.0044\\
1.5903	-0\\
};
\addplot [color=black, forget plot]
  table[row sep=crcr]{%
1.5903	0\\
1.5904	0.0001\\
1.5904	0.0002\\
1.5906	0.0005\\
1.5907	0.0007\\
1.5909	0.001\\
1.591	0.0012\\
1.5917	0.0025\\
1.5931	0.0049\\
1.5938	0.0062\\
1.5992	0.0158\\
1.602	0.0205\\
1.6047	0.0252\\
1.6075	0.0298\\
1.6102	0.0343\\
1.6129	0.0387\\
1.6157	0.043\\
1.6184	0.0473\\
1.6212	0.0515\\
1.6239	0.0556\\
1.6267	0.0597\\
1.6294	0.0637\\
1.6321	0.0676\\
1.6349	0.0714\\
1.6376	0.0752\\
1.6404	0.0788\\
1.6431	0.0825\\
1.6458	0.086\\
1.6486	0.0895\\
1.6513	0.0928\\
1.6541	0.0962\\
1.6568	0.0994\\
1.6596	0.1026\\
1.6623	0.1057\\
1.665	0.1087\\
1.6678	0.1116\\
1.6705	0.1145\\
1.6787	0.1227\\
1.6815	0.1253\\
1.6842	0.1278\\
1.687	0.1302\\
1.6897	0.1326\\
1.6925	0.1349\\
1.6943	0.1364\\
1.6962	0.1379\\
1.6981	0.1393\\
1.7	0.1408\\
};
\addplot [color=black, forget plot]
  table[row sep=crcr]{%
0	1.05\\
0.0008	1.05\\
0.001	1.0499\\
0.0023	1.0499\\
0.0036	1.0498\\
0.0049	1.0496\\
0.0061	1.0495\\
0.0109	1.0489\\
0.0156	1.048\\
0.0203	1.047\\
0.025	1.0457\\
0.0298	1.0442\\
0.0345	1.0424\\
0.0392	1.0405\\
0.0439	1.0383\\
0.0486	1.036\\
0.0534	1.0334\\
0.0581	1.0305\\
0.0628	1.0275\\
0.0675	1.0243\\
0.0723	1.0208\\
0.077	1.0171\\
0.0817	1.0132\\
0.0864	1.009\\
0.0911	1.0047\\
0.0959	1.0001\\
0.1006	0.9953\\
0.1053	0.9903\\
0.11	0.9851\\
0.1148	0.9797\\
0.1195	0.974\\
0.1242	0.9681\\
0.1289	0.962\\
0.1336	0.9557\\
0.1384	0.9492\\
0.1431	0.9424\\
0.1478	0.9354\\
0.1525	0.9283\\
0.1573	0.9208\\
0.162	0.9132\\
0.1667	0.9054\\
0.1714	0.8973\\
0.1761	0.889\\
0.1793	0.8833\\
0.1825	0.8775\\
0.1857	0.8716\\
0.1889	0.8656\\
};
\addplot [color=black, forget plot]
  table[row sep=crcr]{%
0.1889	0.8656\\
0.1936	0.8565\\
0.1983	0.8471\\
0.2031	0.8376\\
0.2078	0.8279\\
0.2125	0.8179\\
0.2172	0.8077\\
0.2219	0.7973\\
0.2267	0.7867\\
0.2314	0.7758\\
0.2361	0.7647\\
0.2408	0.7535\\
0.2456	0.742\\
0.2503	0.7302\\
0.255	0.7183\\
0.2597	0.7061\\
0.2644	0.6938\\
0.2692	0.6812\\
0.2739	0.6684\\
0.2786	0.6553\\
0.2833	0.6421\\
0.2881	0.6286\\
0.2928	0.6149\\
0.2975	0.601\\
0.3022	0.5869\\
0.3069	0.5725\\
0.3117	0.558\\
0.3164	0.5432\\
0.3211	0.5282\\
0.3258	0.513\\
0.3306	0.4975\\
0.3353	0.4819\\
0.34	0.466\\
0.3447	0.4499\\
0.3494	0.4336\\
0.3542	0.417\\
0.3589	0.4003\\
0.3636	0.3833\\
0.3683	0.3661\\
0.3731	0.3487\\
0.3778	0.3311\\
};
\addplot [color=black, forget plot]
  table[row sep=crcr]{%
0.3778	0.3311\\
0.3822	0.3144\\
0.3866	0.2974\\
0.3911	0.2803\\
0.3955	0.263\\
0.4002	0.2444\\
0.4049	0.2255\\
0.4097	0.2064\\
0.4144	0.187\\
0.4191	0.1675\\
0.4238	0.1477\\
0.4285	0.1278\\
0.4333	0.1076\\
0.438	0.0871\\
0.4427	0.0665\\
0.4474	0.0457\\
0.4522	0.0246\\
0.4535	0.0185\\
0.4549	0.0123\\
0.4562	0.0062\\
0.4576	-0\\
};
\addplot [color=black, forget plot]
  table[row sep=crcr]{%
0.4576	0\\
0.4576	0.0002\\
0.4577	0.0002\\
0.4577	0.0005\\
0.4578	0.0007\\
0.4579	0.001\\
0.458	0.0012\\
0.4583	0.0025\\
0.4587	0.0037\\
0.4591	0.005\\
0.4594	0.0062\\
0.4613	0.0124\\
0.4631	0.0186\\
0.465	0.0248\\
0.4668	0.0309\\
0.4695	0.0399\\
0.4723	0.0489\\
0.4777	0.0665\\
0.4804	0.0752\\
0.4832	0.0838\\
0.4859	0.0924\\
0.4886	0.1009\\
0.4914	0.1093\\
0.4968	0.1259\\
0.4995	0.1341\\
0.5023	0.1422\\
0.505	0.1503\\
0.5077	0.1583\\
0.5104	0.1662\\
0.5132	0.174\\
0.5159	0.1818\\
0.5186	0.1894\\
0.5213	0.1971\\
0.5241	0.2046\\
0.5268	0.2121\\
0.5295	0.2195\\
0.5323	0.2268\\
0.5377	0.2412\\
0.5404	0.2483\\
0.5432	0.2553\\
0.5459	0.2623\\
0.5486	0.2692\\
0.5513	0.276\\
0.5541	0.2827\\
0.5568	0.2894\\
0.5595	0.296\\
0.5622	0.3025\\
0.565	0.309\\
0.5658	0.311\\
0.5662	0.3119\\
0.5667	0.3129\\
};
\addplot [color=black, forget plot]
  table[row sep=crcr]{%
0.5667	0.3129\\
0.5714	0.3239\\
0.5761	0.3345\\
0.5808	0.345\\
0.5856	0.3553\\
0.5903	0.3653\\
0.595	0.3751\\
0.5997	0.3847\\
0.6044	0.3941\\
0.6092	0.4033\\
0.6139	0.4122\\
0.6186	0.421\\
0.6233	0.4295\\
0.6281	0.4378\\
0.6328	0.4458\\
0.6375	0.4537\\
0.6422	0.4613\\
0.6469	0.4687\\
0.6517	0.4759\\
0.6564	0.4829\\
0.6611	0.4897\\
0.6658	0.4962\\
0.6706	0.5025\\
0.6753	0.5086\\
0.68	0.5145\\
0.6847	0.5202\\
0.6894	0.5256\\
0.6942	0.5308\\
0.6989	0.5359\\
0.7036	0.5406\\
0.7083	0.5452\\
0.7131	0.5496\\
0.7178	0.5537\\
0.7225	0.5576\\
0.7272	0.5613\\
0.7319	0.5648\\
0.7367	0.568\\
0.7414	0.5711\\
0.7461	0.5739\\
0.7508	0.5765\\
0.7556	0.5789\\
};
\addplot [color=black, forget plot]
  table[row sep=crcr]{%
0.7556	0.5789\\
0.758	0.58\\
0.763	0.5822\\
0.7654	0.5832\\
0.7701	0.5849\\
0.7749	0.5863\\
0.7796	0.5876\\
0.7843	0.5887\\
0.789	0.5895\\
0.7938	0.5901\\
0.7985	0.5905\\
0.8032	0.5907\\
0.8046	0.5907\\
};
\addplot [color=black, forget plot]
  table[row sep=crcr]{%
0.8046	0.5907\\
0.8072	0.5907\\
0.8078	0.5906\\
0.811	0.5905\\
0.8174	0.5899\\
0.8206	0.5894\\
0.8241	0.5888\\
0.8276	0.5881\\
0.8311	0.5873\\
0.8346	0.5863\\
0.8381	0.5852\\
0.8416	0.584\\
0.8451	0.5827\\
0.8521	0.5797\\
0.8556	0.578\\
0.8591	0.5762\\
0.8661	0.5722\\
0.8696	0.57\\
0.8731	0.5677\\
0.8766	0.5653\\
0.8801	0.5628\\
0.8835	0.5602\\
0.887	0.5574\\
0.8905	0.5545\\
0.894	0.5515\\
0.8975	0.5484\\
0.9045	0.5418\\
0.908	0.5383\\
0.9115	0.5347\\
0.9185	0.5271\\
0.9255	0.5191\\
0.9325	0.5105\\
0.9355	0.5067\\
0.9415	0.4989\\
0.9444	0.4948\\
};
\addplot [color=black, forget plot]
  table[row sep=crcr]{%
0.9444	0.4948\\
0.9492	0.4882\\
0.9539	0.4814\\
0.9586	0.4744\\
0.9633	0.4672\\
0.9681	0.4597\\
0.9728	0.452\\
0.9775	0.4441\\
0.9822	0.436\\
0.9869	0.4277\\
0.9917	0.4191\\
0.9964	0.4103\\
1.0011	0.4013\\
1.0058	0.3921\\
1.0106	0.3827\\
1.0153	0.3731\\
1.02	0.3632\\
1.0247	0.3531\\
1.0294	0.3428\\
1.0342	0.3323\\
1.0389	0.3215\\
1.0436	0.3106\\
1.0483	0.2994\\
1.0531	0.288\\
1.0578	0.2764\\
1.0625	0.2645\\
1.0672	0.2525\\
1.0719	0.2402\\
1.0767	0.2277\\
1.0814	0.215\\
1.0861	0.2021\\
1.0908	0.1889\\
1.0956	0.1756\\
1.1003	0.162\\
1.105	0.1482\\
1.1097	0.1341\\
1.1144	0.1199\\
1.1192	0.1054\\
1.1239	0.0908\\
1.1286	0.0759\\
1.1333	0.0607\\
};
\addplot [color=black, forget plot]
  table[row sep=crcr]{%
1.1333	0.0607\\
1.1343	0.0577\\
1.1352	0.0546\\
1.1362	0.0515\\
1.1371	0.0485\\
1.1408	0.0365\\
1.1444	0.0245\\
1.148	0.0123\\
1.1517	0\\
};
\addplot [color=black, forget plot]
  table[row sep=crcr]{%
1.1517	0\\
1.1517	0.0002\\
1.1518	0.0005\\
1.1519	0.0007\\
1.152	0.001\\
1.1521	0.0012\\
1.1526	0.0025\\
1.1531	0.0037\\
1.1536	0.005\\
1.1541	0.0062\\
1.1566	0.0124\\
1.159	0.0185\\
1.1615	0.0246\\
1.1639	0.0306\\
1.1682	0.0409\\
1.1725	0.051\\
1.1767	0.0609\\
1.181	0.0707\\
1.1853	0.0802\\
1.1895	0.0896\\
1.1938	0.0989\\
1.198	0.1079\\
1.2023	0.1167\\
1.2066	0.1254\\
1.2108	0.1339\\
1.2151	0.1423\\
1.2194	0.1504\\
1.2236	0.1584\\
1.2322	0.1738\\
1.2364	0.1812\\
1.2407	0.1884\\
1.245	0.1955\\
1.2492	0.2024\\
1.2535	0.2091\\
1.2577	0.2157\\
1.262	0.222\\
1.2663	0.2282\\
1.2705	0.2342\\
1.2748	0.24\\
1.2791	0.2457\\
1.2833	0.2512\\
1.2919	0.2616\\
1.2961	0.2665\\
1.3004	0.2712\\
1.3047	0.2758\\
1.3089	0.2802\\
1.3132	0.2844\\
1.3174	0.2885\\
1.3198	0.2907\\
1.321	0.2917\\
1.3222	0.2928\\
};
\addplot [color=black, forget plot]
  table[row sep=crcr]{%
1.3222	0.2928\\
1.3267	0.2967\\
1.3312	0.3003\\
1.3357	0.3038\\
1.3402	0.3071\\
1.345	0.3103\\
1.3497	0.3133\\
1.3544	0.316\\
1.3591	0.3186\\
1.3639	0.3209\\
1.3686	0.3231\\
1.3733	0.3249\\
1.378	0.3266\\
1.3827	0.3281\\
1.3875	0.3293\\
1.3922	0.3304\\
1.3969	0.3312\\
1.4007	0.3316\\
1.4044	0.332\\
1.4082	0.3322\\
1.4119	0.3323\\
};
\addplot [color=black, forget plot]
  table[row sep=crcr]{%
1.4119	0.3323\\
1.4132	0.3323\\
1.4138	0.3322\\
1.4151	0.3322\\
1.4176	0.3321\\
1.4226	0.3317\\
1.425	0.3314\\
1.4275	0.3311\\
1.43	0.3307\\
1.435	0.3297\\
1.4374	0.3291\\
1.4424	0.3277\\
1.4474	0.3261\\
1.4498	0.3252\\
1.4523	0.3243\\
1.4548	0.3233\\
1.4573	0.3222\\
1.4598	0.321\\
1.4622	0.3199\\
1.4672	0.3173\\
1.4722	0.3145\\
1.4746	0.313\\
1.4796	0.3098\\
1.4846	0.3064\\
1.487	0.3046\\
1.492	0.3008\\
1.4945	0.2988\\
1.4969	0.2968\\
1.5019	0.2926\\
1.5044	0.2903\\
1.5061	0.2888\\
1.5077	0.2872\\
1.5111	0.284\\
};
\addplot [color=black, forget plot]
  table[row sep=crcr]{%
1.5111	0.284\\
1.5158	0.2793\\
1.5206	0.2744\\
1.5253	0.2692\\
1.53	0.2639\\
1.5347	0.2583\\
1.5394	0.2525\\
1.5442	0.2465\\
1.5489	0.2403\\
1.5536	0.2338\\
1.5583	0.2271\\
1.5631	0.2202\\
1.5678	0.2131\\
1.5725	0.2058\\
1.5772	0.1983\\
1.5819	0.1905\\
1.5867	0.1825\\
1.5914	0.1743\\
1.5961	0.1659\\
1.6008	0.1572\\
1.6056	0.1484\\
1.6103	0.1393\\
1.615	0.13\\
1.6197	0.1205\\
1.6244	0.1107\\
1.6292	0.1008\\
1.6339	0.0906\\
1.6386	0.0802\\
1.6433	0.0696\\
1.6481	0.0588\\
1.6528	0.0477\\
1.6575	0.0365\\
1.6622	0.025\\
1.6672	0.0126\\
1.6722	-0\\
};
\addplot [color=black, forget plot]
  table[row sep=crcr]{%
1.6722	0\\
1.6722	0.0001\\
1.6723	0.0001\\
1.6723	0.0002\\
1.6724	0.0005\\
1.6726	0.0007\\
1.6727	0.001\\
1.6728	0.0012\\
1.6735	0.0025\\
1.6741	0.0037\\
1.6748	0.0049\\
1.6755	0.0062\\
1.6761	0.0075\\
1.6789	0.0127\\
1.6796	0.0139\\
1.6817	0.0178\\
1.6824	0.019\\
1.6831	0.0203\\
1.6838	0.0215\\
1.6845	0.0228\\
1.6852	0.024\\
1.6859	0.0253\\
1.6873	0.0277\\
1.688	0.029\\
1.6894	0.0314\\
1.69	0.0326\\
1.6949	0.041\\
1.6956	0.0421\\
1.697	0.0445\\
1.6977	0.0456\\
1.6983	0.0466\\
1.6988	0.0475\\
1.6994	0.0485\\
1.7	0.0494\\
};
\addplot [color=black, forget plot]
  table[row sep=crcr]{%
0	1.05\\
0.001	1.05\\
0.002	1.0501\\
0.0051	1.0501\\
};
\addplot [color=black, forget plot]
  table[row sep=crcr]{%
0.0051	1.0501\\
0.0083	1.0501\\
0.0147	1.0497\\
0.0211	1.0489\\
0.0257	1.048\\
0.0303	1.047\\
0.0349	1.0458\\
0.0395	1.0443\\
0.0441	1.0427\\
0.0487	1.0408\\
0.0533	1.0387\\
0.0579	1.0365\\
0.0624	1.034\\
0.067	1.0313\\
0.0716	1.0284\\
0.0762	1.0253\\
0.0808	1.022\\
0.0854	1.0185\\
0.09	1.0148\\
0.0946	1.0108\\
0.0992	1.0067\\
0.1038	1.0023\\
0.1084	0.9978\\
0.113	0.993\\
0.1176	0.9881\\
0.1222	0.9829\\
0.1268	0.9775\\
0.1314	0.9719\\
0.136	0.9661\\
0.1406	0.9601\\
0.1452	0.9539\\
0.1497	0.9475\\
0.1543	0.9409\\
0.1589	0.934\\
0.1635	0.927\\
0.1681	0.9198\\
0.1727	0.9123\\
0.1773	0.9046\\
0.1819	0.8968\\
0.1865	0.8887\\
0.1871	0.8876\\
0.1877	0.8866\\
0.1889	0.8844\\
};
\addplot [color=black, forget plot]
  table[row sep=crcr]{%
0.1889	0.8844\\
0.1936	0.8758\\
0.1983	0.867\\
0.2031	0.8579\\
0.2078	0.8486\\
0.2125	0.8391\\
0.2172	0.8294\\
0.2219	0.8195\\
0.2267	0.8093\\
0.2314	0.799\\
0.2361	0.7884\\
0.2408	0.7775\\
0.2456	0.7665\\
0.2503	0.7553\\
0.255	0.7438\\
0.2597	0.7321\\
0.2644	0.7202\\
0.2692	0.7081\\
0.2739	0.6957\\
0.2786	0.6832\\
0.2833	0.6704\\
0.2881	0.6574\\
0.2928	0.6442\\
0.2975	0.6308\\
0.3022	0.6171\\
0.3069	0.6032\\
0.3117	0.5891\\
0.3164	0.5748\\
0.3211	0.5603\\
0.3258	0.5455\\
0.3306	0.5306\\
0.3353	0.5154\\
0.34	0.5\\
0.3447	0.4844\\
0.3494	0.4685\\
0.3542	0.4525\\
0.3589	0.4362\\
0.3636	0.4197\\
0.3683	0.403\\
0.3731	0.386\\
0.3778	0.3689\\
};
\addplot [color=black, forget plot]
  table[row sep=crcr]{%
0.3778	0.3689\\
0.3825	0.3515\\
0.3872	0.3339\\
0.3919	0.3161\\
0.3967	0.2981\\
0.4014	0.2798\\
0.4061	0.2613\\
0.4108	0.2427\\
0.4156	0.2238\\
0.4203	0.2046\\
0.425	0.1853\\
0.4297	0.1657\\
0.4344	0.1459\\
0.4392	0.1259\\
0.4439	0.1057\\
0.4486	0.0853\\
0.4533	0.0646\\
0.4569	0.0487\\
0.4606	0.0326\\
0.4642	0.0164\\
0.4678	-0\\
};
\addplot [color=black, forget plot]
  table[row sep=crcr]{%
0.4678	0\\
0.4678	0.0002\\
0.4679	0.0002\\
0.4679	0.0005\\
0.468	0.0007\\
0.4681	0.001\\
0.4682	0.0012\\
0.4685	0.0025\\
0.4689	0.0037\\
0.4693	0.005\\
0.4696	0.0062\\
0.4715	0.0124\\
0.4733	0.0186\\
0.4752	0.0248\\
0.477	0.0309\\
0.4795	0.0391\\
0.482	0.0472\\
0.4844	0.0552\\
0.4869	0.0632\\
0.4894	0.0711\\
0.4918	0.079\\
0.4943	0.0868\\
0.4993	0.1022\\
0.5017	0.1098\\
0.5042	0.1174\\
0.5067	0.1249\\
0.5091	0.1324\\
0.5116	0.1397\\
0.5141	0.1471\\
0.5166	0.1543\\
0.519	0.1615\\
0.5215	0.1687\\
0.524	0.1757\\
0.5264	0.1828\\
0.5339	0.2035\\
0.5363	0.2103\\
0.5388	0.217\\
0.5413	0.2236\\
0.5437	0.2302\\
0.5462	0.2368\\
0.5487	0.2433\\
0.5512	0.2497\\
0.5536	0.2561\\
0.5561	0.2624\\
0.5587	0.269\\
0.5614	0.2756\\
0.564	0.2822\\
0.5667	0.2886\\
};
\addplot [color=black, forget plot]
  table[row sep=crcr]{%
0.5667	0.2886\\
0.5714	0.3\\
0.5761	0.3112\\
0.5808	0.3221\\
0.5856	0.3329\\
0.5903	0.3434\\
0.595	0.3537\\
0.5997	0.3637\\
0.6044	0.3736\\
0.6092	0.3832\\
0.6139	0.3927\\
0.6186	0.4019\\
0.6233	0.4108\\
0.6281	0.4196\\
0.6328	0.4281\\
0.6375	0.4365\\
0.6422	0.4446\\
0.6469	0.4525\\
0.6517	0.4601\\
0.6564	0.4676\\
0.6611	0.4748\\
0.6658	0.4818\\
0.6706	0.4886\\
0.6753	0.4952\\
0.68	0.5015\\
0.6847	0.5077\\
0.6894	0.5136\\
0.6942	0.5193\\
0.6989	0.5248\\
0.7036	0.53\\
0.7083	0.5351\\
0.7131	0.5399\\
0.7178	0.5445\\
0.7225	0.5489\\
0.7272	0.5531\\
0.7319	0.557\\
0.7367	0.5607\\
0.7414	0.5642\\
0.7461	0.5675\\
0.7508	0.5706\\
0.7556	0.5735\\
};
\addplot [color=black, forget plot]
  table[row sep=crcr]{%
0.7556	0.5735\\
0.7585	0.5752\\
0.7615	0.5768\\
0.7645	0.5783\\
0.7675	0.5797\\
0.7722	0.5818\\
0.7769	0.5836\\
0.7816	0.5853\\
0.7864	0.5867\\
0.7911	0.5879\\
0.7958	0.5889\\
0.8005	0.5897\\
0.8052	0.5902\\
0.81	0.5906\\
0.8124	0.5907\\
0.8148	0.5907\\
};
\addplot [color=black, forget plot]
  table[row sep=crcr]{%
0.8148	0.5907\\
0.8174	0.5907\\
0.818	0.5906\\
0.8212	0.5905\\
0.8276	0.5899\\
0.8308	0.5894\\
0.8341	0.5889\\
0.8405	0.5875\\
0.8438	0.5866\\
0.847	0.5856\\
0.8503	0.5845\\
0.8535	0.5834\\
0.8567	0.5821\\
0.86	0.5807\\
0.8632	0.5792\\
0.8665	0.5776\\
0.8697	0.5759\\
0.8729	0.5741\\
0.8762	0.5722\\
0.8794	0.5702\\
0.8827	0.5681\\
0.8859	0.5659\\
0.8892	0.5636\\
0.8924	0.5612\\
0.8956	0.5587\\
0.8989	0.556\\
0.9021	0.5533\\
0.9054	0.5505\\
0.9086	0.5476\\
0.9118	0.5445\\
0.9151	0.5414\\
0.9183	0.5382\\
0.9216	0.5348\\
0.9248	0.5314\\
0.928	0.5278\\
0.9313	0.5242\\
0.9345	0.5204\\
0.937	0.5175\\
0.9395	0.5145\\
0.942	0.5114\\
0.9444	0.5083\\
};
\addplot [color=black, forget plot]
  table[row sep=crcr]{%
0.9444	0.5083\\
0.9492	0.5022\\
0.9539	0.4958\\
0.9586	0.4893\\
0.9633	0.4825\\
0.9681	0.4755\\
0.9728	0.4683\\
0.9775	0.4609\\
0.9822	0.4533\\
0.9869	0.4454\\
0.9917	0.4373\\
0.9964	0.429\\
1.0011	0.4205\\
1.0058	0.4117\\
1.0106	0.4028\\
1.0153	0.3936\\
1.02	0.3842\\
1.0247	0.3746\\
1.0294	0.3648\\
1.0342	0.3547\\
1.0389	0.3444\\
1.0436	0.334\\
1.0483	0.3232\\
1.0531	0.3123\\
1.0578	0.3012\\
1.0625	0.2898\\
1.0672	0.2782\\
1.0719	0.2664\\
1.0767	0.2544\\
1.0814	0.2422\\
1.0861	0.2297\\
1.0908	0.217\\
1.0956	0.2041\\
1.1003	0.191\\
1.105	0.1777\\
1.1097	0.1641\\
1.1144	0.1504\\
1.1192	0.1364\\
1.1239	0.1222\\
1.1286	0.1077\\
1.1333	0.0931\\
};
\addplot [color=black, forget plot]
  table[row sep=crcr]{%
1.1333	0.0931\\
1.1378	0.079\\
1.1393	0.0742\\
1.144	0.0591\\
1.1488	0.0437\\
1.1535	0.0281\\
1.1582	0.0123\\
1.1591	0.0093\\
1.1609	0.0031\\
1.1619	-0\\
};
\addplot [color=black, forget plot]
  table[row sep=crcr]{%
1.1619	0\\
1.1619	0.0002\\
1.162	0.0005\\
1.1621	0.0007\\
1.1622	0.001\\
1.1623	0.0012\\
1.1628	0.0025\\
1.1633	0.0037\\
1.1638	0.005\\
1.1643	0.0062\\
1.1668	0.0124\\
1.1692	0.0185\\
1.1717	0.0246\\
1.1741	0.0306\\
1.1781	0.0403\\
1.1821	0.0498\\
1.1862	0.0592\\
1.1902	0.0684\\
1.1942	0.0774\\
1.1982	0.0863\\
1.2022	0.095\\
1.2062	0.1036\\
1.2102	0.112\\
1.2142	0.1203\\
1.2182	0.1284\\
1.2222	0.1363\\
1.2263	0.1441\\
1.2303	0.1517\\
1.2343	0.1592\\
1.2383	0.1665\\
1.2423	0.1736\\
1.2463	0.1806\\
1.2503	0.1875\\
1.2583	0.2007\\
1.2623	0.207\\
1.2663	0.2132\\
1.2704	0.2193\\
1.2744	0.2252\\
1.2784	0.2309\\
1.2824	0.2365\\
1.2864	0.2419\\
1.2904	0.2472\\
1.2944	0.2523\\
1.2984	0.2572\\
1.3024	0.262\\
1.3064	0.2666\\
1.3104	0.2711\\
1.3145	0.2754\\
1.3185	0.2796\\
1.3194	0.2805\\
1.3203	0.2815\\
1.3213	0.2824\\
1.3222	0.2833\\
};
\addplot [color=black, forget plot]
  table[row sep=crcr]{%
1.3222	0.2833\\
1.3269	0.2878\\
1.3317	0.2921\\
1.3364	0.2962\\
1.3411	0.3001\\
1.3458	0.3037\\
1.3506	0.3071\\
1.3553	0.3104\\
1.36	0.3133\\
1.3647	0.3161\\
1.3694	0.3187\\
1.3742	0.321\\
1.3789	0.3231\\
1.3836	0.325\\
1.3883	0.3267\\
1.3931	0.3281\\
1.3978	0.3294\\
1.4025	0.3304\\
1.4072	0.3312\\
1.4119	0.3318\\
1.4167	0.3321\\
1.418	0.3322\\
1.4194	0.3322\\
1.4208	0.3323\\
1.4221	0.3323\\
};
\addplot [color=black, forget plot]
  table[row sep=crcr]{%
1.4221	0.3323\\
1.4234	0.3323\\
1.424	0.3322\\
1.4253	0.3322\\
1.4275	0.3321\\
1.4298	0.332\\
1.432	0.3318\\
1.4342	0.3315\\
1.4364	0.3313\\
1.4387	0.3309\\
1.4431	0.3301\\
1.4453	0.3296\\
1.4476	0.3291\\
1.452	0.3279\\
1.4542	0.3272\\
1.4565	0.3265\\
1.4609	0.3249\\
1.4631	0.324\\
1.4654	0.3231\\
1.4698	0.3211\\
1.472	0.32\\
1.4743	0.3189\\
1.4765	0.3178\\
1.4787	0.3166\\
1.4809	0.3153\\
1.4832	0.314\\
1.4898	0.3098\\
1.4921	0.3083\\
1.4987	0.3035\\
1.501	0.3018\\
1.5054	0.2982\\
1.5068	0.2971\\
1.5083	0.2959\\
1.5097	0.2947\\
1.5111	0.2934\\
};
\addplot [color=black, forget plot]
  table[row sep=crcr]{%
1.5111	0.2934\\
1.5156	0.2894\\
1.5201	0.2852\\
1.5245	0.2808\\
1.529	0.2762\\
1.5337	0.2712\\
1.5384	0.2659\\
1.5432	0.2604\\
1.5479	0.2547\\
1.5526	0.2488\\
1.5573	0.2426\\
1.562	0.2362\\
1.5668	0.2296\\
1.5715	0.2228\\
1.5762	0.2158\\
1.5809	0.2086\\
1.5857	0.2011\\
1.5904	0.1934\\
1.5951	0.1855\\
1.5998	0.1774\\
1.6045	0.169\\
1.6093	0.1605\\
1.614	0.1517\\
1.6187	0.1427\\
1.6234	0.1335\\
1.6282	0.124\\
1.6329	0.1144\\
1.6376	0.1045\\
1.6423	0.0944\\
1.647	0.0841\\
1.6518	0.0736\\
1.6565	0.0628\\
1.6612	0.0519\\
1.6659	0.0407\\
1.6707	0.0293\\
1.6754	0.0177\\
1.6801	0.0058\\
1.6807	0.0044\\
1.6812	0.0029\\
1.6818	0.0015\\
1.6824	-0\\
};
\addplot [color=black, forget plot]
  table[row sep=crcr]{%
1.6824	0\\
1.6824	0.0001\\
1.6825	0.0002\\
1.6826	0.0005\\
1.6828	0.0007\\
1.6829	0.001\\
1.683	0.0012\\
1.6835	0.002\\
1.6839	0.0029\\
1.6843	0.0037\\
1.6848	0.0045\\
1.6852	0.0054\\
1.6857	0.0062\\
1.6861	0.007\\
1.6865	0.0079\\
1.687	0.0087\\
1.6874	0.0095\\
1.6879	0.0103\\
1.6883	0.0111\\
1.6887	0.012\\
1.6892	0.0128\\
1.6896	0.0136\\
1.6901	0.0144\\
1.6909	0.016\\
1.6914	0.0168\\
1.6918	0.0176\\
1.6923	0.0184\\
1.6931	0.02\\
1.6936	0.0208\\
1.694	0.0216\\
1.6945	0.0224\\
1.6953	0.024\\
1.6958	0.0248\\
1.6962	0.0256\\
1.6967	0.0263\\
1.6975	0.0279\\
1.698	0.0287\\
1.6984	0.0294\\
1.6989	0.0302\\
1.6992	0.0307\\
1.6994	0.0312\\
1.7	0.0322\\
};
\addplot [color=black, forget plot]
  table[row sep=crcr]{%
0	0.95\\
0.001	0.95\\
0.002	0.9501\\
0.0051	0.9501\\
};
\addplot [color=black, forget plot]
  table[row sep=crcr]{%
0.0051	0.9501\\
0.0083	0.9501\\
0.0147	0.9497\\
0.0211	0.9489\\
0.0257	0.948\\
0.0303	0.947\\
0.0349	0.9458\\
0.0395	0.9443\\
0.0441	0.9427\\
0.0487	0.9408\\
0.0533	0.9387\\
0.0579	0.9365\\
0.0624	0.934\\
0.067	0.9313\\
0.0716	0.9284\\
0.0762	0.9253\\
0.0808	0.922\\
0.0854	0.9185\\
0.09	0.9148\\
0.0946	0.9108\\
0.0992	0.9067\\
0.1038	0.9023\\
0.1084	0.8978\\
0.113	0.893\\
0.1176	0.8881\\
0.1222	0.8829\\
0.1268	0.8775\\
0.1314	0.8719\\
0.136	0.8661\\
0.1406	0.8601\\
0.1452	0.8539\\
0.1497	0.8475\\
0.1543	0.8409\\
0.1589	0.834\\
0.1635	0.827\\
0.1681	0.8198\\
0.1727	0.8123\\
0.1773	0.8046\\
0.1819	0.7968\\
0.1865	0.7887\\
0.1871	0.7876\\
0.1877	0.7866\\
0.1889	0.7844\\
};
\addplot [color=black, forget plot]
  table[row sep=crcr]{%
0.1889	0.7844\\
0.1936	0.7758\\
0.1983	0.767\\
0.2031	0.7579\\
0.2078	0.7486\\
0.2125	0.7391\\
0.2172	0.7294\\
0.2219	0.7195\\
0.2267	0.7093\\
0.2314	0.699\\
0.2361	0.6884\\
0.2408	0.6775\\
0.2456	0.6665\\
0.2503	0.6553\\
0.255	0.6438\\
0.2597	0.6321\\
0.2644	0.6202\\
0.2692	0.6081\\
0.2739	0.5957\\
0.2786	0.5832\\
0.2833	0.5704\\
0.2881	0.5574\\
0.2928	0.5442\\
0.2975	0.5308\\
0.3022	0.5171\\
0.3069	0.5032\\
0.3117	0.4891\\
0.3164	0.4748\\
0.3211	0.4603\\
0.3258	0.4455\\
0.3306	0.4306\\
0.3353	0.4154\\
0.34	0.4\\
0.3447	0.3844\\
0.3494	0.3685\\
0.3542	0.3525\\
0.3589	0.3362\\
0.3636	0.3197\\
0.3683	0.303\\
0.3731	0.286\\
0.3778	0.2689\\
};
\addplot [color=black, forget plot]
  table[row sep=crcr]{%
0.3778	0.2689\\
0.3815	0.2553\\
0.3852	0.2416\\
0.3926	0.2138\\
0.3973	0.1957\\
0.402	0.1774\\
0.4067	0.1589\\
0.4114	0.1402\\
0.4162	0.1213\\
0.4209	0.1021\\
0.4256	0.0828\\
0.4303	0.0632\\
0.4341	0.0476\\
0.4378	0.0319\\
0.4415	0.016\\
0.4452	0\\
};
\addplot [color=black, forget plot]
  table[row sep=crcr]{%
0.4452	0\\
0.4452	0.0001\\
0.4453	0.0002\\
0.4454	0.0005\\
0.4454	0.0007\\
0.4455	0.001\\
0.4456	0.0012\\
0.446	0.0025\\
0.4464	0.0037\\
0.4468	0.005\\
0.4471	0.0062\\
0.4491	0.0124\\
0.451	0.0186\\
0.453	0.0248\\
0.4549	0.0309\\
0.4579	0.0404\\
0.461	0.0498\\
0.464	0.0591\\
0.467	0.0683\\
0.4701	0.0775\\
0.4731	0.0865\\
0.4762	0.0955\\
0.4822	0.1131\\
0.4853	0.1218\\
0.4883	0.1304\\
0.4913	0.1389\\
0.4944	0.1473\\
0.5004	0.1639\\
0.5035	0.172\\
0.5065	0.1801\\
0.5096	0.188\\
0.5126	0.1959\\
0.5156	0.2037\\
0.5187	0.2114\\
0.5217	0.219\\
0.5247	0.2265\\
0.5278	0.2339\\
0.5338	0.2485\\
0.5369	0.2556\\
0.5399	0.2627\\
0.5429	0.2696\\
0.546	0.2765\\
0.549	0.2833\\
0.5521	0.29\\
0.5551	0.2966\\
0.5581	0.3031\\
0.5612	0.3095\\
0.5642	0.3159\\
0.5648	0.3171\\
0.5654	0.3184\\
0.5661	0.3197\\
0.5667	0.3209\\
};
\addplot [color=black, forget plot]
  table[row sep=crcr]{%
0.5667	0.3209\\
0.5714	0.3305\\
0.5761	0.3398\\
0.5808	0.3489\\
0.5856	0.3578\\
0.5903	0.3665\\
0.595	0.375\\
0.5997	0.3832\\
0.6044	0.3912\\
0.6092	0.3991\\
0.6139	0.4066\\
0.6186	0.414\\
0.6233	0.4212\\
0.6281	0.4281\\
0.6328	0.4348\\
0.6375	0.4413\\
0.6422	0.4476\\
0.6469	0.4536\\
0.6517	0.4595\\
0.6564	0.4651\\
0.6611	0.4705\\
0.6658	0.4757\\
0.6706	0.4806\\
0.6753	0.4854\\
0.68	0.4899\\
0.6847	0.4942\\
0.6894	0.4983\\
0.6942	0.5022\\
0.6989	0.5058\\
0.7036	0.5092\\
0.7083	0.5124\\
0.7131	0.5154\\
0.7178	0.5182\\
0.7225	0.5208\\
0.7272	0.5231\\
0.7319	0.5252\\
0.7367	0.5271\\
0.7414	0.5288\\
0.7461	0.5303\\
0.7508	0.5315\\
0.7556	0.5325\\
};
\addplot [color=black, forget plot]
  table[row sep=crcr]{%
0.7556	0.5325\\
0.7565	0.5327\\
0.7585	0.5331\\
0.7595	0.5332\\
0.7635	0.5338\\
0.7674	0.5341\\
0.7714	0.5344\\
0.7753	0.5344\\
};
\addplot [color=black, forget plot]
  table[row sep=crcr]{%
0.7753	0.5344\\
0.7785	0.5344\\
0.7849	0.534\\
0.7913	0.5332\\
0.7955	0.5324\\
0.7998	0.5315\\
0.804	0.5304\\
0.8082	0.5291\\
0.8124	0.5277\\
0.8167	0.5261\\
0.8251	0.5223\\
0.8294	0.5201\\
0.8336	0.5178\\
0.8378	0.5153\\
0.842	0.5126\\
0.8463	0.5097\\
0.8505	0.5067\\
0.8547	0.5035\\
0.859	0.5001\\
0.8632	0.4966\\
0.8674	0.4928\\
0.8716	0.4889\\
0.8759	0.4848\\
0.8801	0.4806\\
0.8843	0.4761\\
0.8886	0.4715\\
0.8928	0.4667\\
0.897	0.4618\\
0.9012	0.4567\\
0.9055	0.4513\\
0.9097	0.4459\\
0.9139	0.4402\\
0.9182	0.4344\\
0.9266	0.4222\\
0.9311	0.4154\\
0.9355	0.4085\\
0.94	0.4014\\
0.9444	0.3941\\
};
\addplot [color=black, forget plot]
  table[row sep=crcr]{%
0.9444	0.3941\\
0.9492	0.3862\\
0.9539	0.378\\
0.9586	0.3696\\
0.9633	0.361\\
0.9681	0.3522\\
0.9728	0.3432\\
0.9775	0.3339\\
0.9822	0.3244\\
0.9869	0.3148\\
0.9917	0.3048\\
0.9964	0.2947\\
1.0011	0.2844\\
1.0058	0.2738\\
1.0106	0.263\\
1.0153	0.252\\
1.02	0.2408\\
1.0247	0.2293\\
1.0294	0.2177\\
1.0342	0.2058\\
1.0389	0.1937\\
1.0436	0.1814\\
1.0483	0.1688\\
1.0531	0.1561\\
1.0578	0.1431\\
1.0625	0.1299\\
1.0672	0.1165\\
1.0719	0.1028\\
1.0767	0.089\\
1.0814	0.0749\\
1.0861	0.0606\\
1.0908	0.0461\\
1.0956	0.0314\\
1.098	0.0236\\
1.1005	0.0158\\
1.1029	0.0079\\
1.1054	-0\\
};
\addplot [color=black, forget plot]
  table[row sep=crcr]{%
1.1054	0\\
1.1054	0.0001\\
1.1055	0.0002\\
1.1056	0.0005\\
1.1057	0.0007\\
1.1058	0.001\\
1.1059	0.0012\\
1.1064	0.0025\\
1.1069	0.0037\\
1.1074	0.005\\
1.108	0.0062\\
1.1087	0.0079\\
1.1094	0.0095\\
1.1108	0.0129\\
1.1115	0.0145\\
1.1122	0.0162\\
1.1129	0.0178\\
1.1135	0.0195\\
1.1149	0.0227\\
1.1156	0.0244\\
1.1233	0.042\\
1.124	0.0435\\
1.1254	0.0467\\
1.1261	0.0482\\
1.1268	0.0498\\
1.1275	0.0513\\
1.1282	0.0529\\
1.1296	0.0559\\
1.1303	0.0575\\
1.1311	0.0591\\
1.1318	0.0608\\
1.1326	0.0624\\
1.1333	0.064\\
};
\addplot [color=black, forget plot]
  table[row sep=crcr]{%
1.1333	0.064\\
1.1378	0.0736\\
1.1393	0.0767\\
1.144	0.0865\\
1.1487	0.0961\\
1.1535	0.1054\\
1.1582	0.1145\\
1.1629	0.1235\\
1.1676	0.1322\\
1.1724	0.1406\\
1.1771	0.1489\\
1.1818	0.1569\\
1.1865	0.1647\\
1.1912	0.1723\\
1.196	0.1797\\
1.2007	0.1869\\
1.2054	0.1938\\
1.2101	0.2006\\
1.2149	0.2071\\
1.2196	0.2134\\
1.2243	0.2194\\
1.229	0.2253\\
1.2337	0.2309\\
1.2385	0.2363\\
1.2432	0.2415\\
1.2479	0.2465\\
1.2526	0.2513\\
1.2574	0.2558\\
1.2621	0.2601\\
1.2668	0.2642\\
1.2715	0.2681\\
1.2762	0.2718\\
1.281	0.2752\\
1.2857	0.2784\\
1.2904	0.2814\\
1.2951	0.2842\\
1.2999	0.2868\\
1.3046	0.2891\\
1.3093	0.2913\\
1.3125	0.2926\\
1.3158	0.2938\\
1.319	0.295\\
1.3222	0.296\\
};
\addplot [color=black, forget plot]
  table[row sep=crcr]{%
1.3222	0.296\\
1.3238	0.2964\\
1.3253	0.2969\\
1.3269	0.2973\\
1.3284	0.2977\\
1.3331	0.2987\\
1.3378	0.2995\\
1.3426	0.3001\\
1.3473	0.3005\\
1.3487	0.3005\\
1.3501	0.3006\\
1.353	0.3006\\
};
\addplot [color=black, forget plot]
  table[row sep=crcr]{%
1.353	0.3006\\
1.3562	0.3006\\
1.3626	0.3002\\
1.369	0.2994\\
1.3729	0.2987\\
1.3769	0.2978\\
1.3808	0.2968\\
1.3848	0.2957\\
1.3887	0.2944\\
1.3927	0.2929\\
1.3966	0.2913\\
1.4006	0.2895\\
1.4045	0.2876\\
1.4085	0.2855\\
1.4125	0.2833\\
1.4164	0.2809\\
1.4204	0.2783\\
1.4243	0.2757\\
1.4283	0.2728\\
1.4322	0.2698\\
1.4362	0.2667\\
1.4401	0.2634\\
1.4441	0.2599\\
1.448	0.2563\\
1.452	0.2525\\
1.4559	0.2486\\
1.4599	0.2445\\
1.4639	0.2403\\
1.4678	0.2359\\
1.4718	0.2314\\
1.4757	0.2267\\
1.4797	0.2219\\
1.4836	0.2169\\
1.4876	0.2118\\
1.4915	0.2065\\
1.4955	0.201\\
1.4994	0.1955\\
1.5033	0.1898\\
1.5111	0.178\\
};
\addplot [color=black, forget plot]
  table[row sep=crcr]{%
1.5111	0.178\\
1.5158	0.1705\\
1.5206	0.1629\\
1.5253	0.155\\
1.53	0.1469\\
1.5347	0.1386\\
1.5394	0.1301\\
1.5442	0.1213\\
1.5489	0.1123\\
1.5536	0.1032\\
1.5583	0.0938\\
1.5631	0.0841\\
1.5678	0.0743\\
1.5725	0.0642\\
1.5772	0.0539\\
1.5819	0.0435\\
1.5867	0.0327\\
1.5901	0.0247\\
1.5936	0.0166\\
1.5971	0.0084\\
1.6005	-0\\
};
\addplot [color=black, forget plot]
  table[row sep=crcr]{%
1.6005	0\\
1.6006	0.0001\\
1.6006	0.0002\\
1.6008	0.0005\\
1.6009	0.0007\\
1.6011	0.001\\
1.6012	0.0012\\
1.6019	0.0025\\
1.6033	0.0049\\
1.604	0.0062\\
1.6064	0.0106\\
1.6139	0.0235\\
1.6164	0.0276\\
1.6189	0.0318\\
1.6214	0.0358\\
1.6238	0.0398\\
1.6288	0.0476\\
1.6338	0.0552\\
1.6363	0.0588\\
1.6388	0.0625\\
1.6413	0.066\\
1.6437	0.0695\\
1.6462	0.073\\
1.6487	0.0764\\
1.6537	0.083\\
1.6562	0.0862\\
1.6587	0.0893\\
1.6611	0.0924\\
1.6661	0.0984\\
1.6686	0.1013\\
1.6736	0.1069\\
1.6786	0.1123\\
1.681	0.1149\\
1.686	0.1199\\
1.6885	0.1223\\
1.6935	0.1269\\
1.6951	0.1284\\
1.6967	0.1298\\
1.6984	0.1313\\
1.7	0.1326\\
};
\addplot [color=black, forget plot]
  table[row sep=crcr]{%
0	0.95\\
0.0032	0.95\\
};
\addplot [color=black, forget plot]
  table[row sep=crcr]{%
0.0032	0.95\\
0.0064	0.95\\
0.0128	0.9496\\
0.0192	0.9488\\
0.0238	0.948\\
0.0285	0.9469\\
0.0331	0.9457\\
0.0377	0.9442\\
0.0424	0.9425\\
0.047	0.9406\\
0.0517	0.9385\\
0.0563	0.9362\\
0.061	0.9337\\
0.0656	0.9309\\
0.0702	0.928\\
0.0749	0.9248\\
0.0795	0.9215\\
0.0842	0.9179\\
0.0888	0.9141\\
0.0935	0.9101\\
0.0981	0.9058\\
0.1027	0.9014\\
0.1074	0.8968\\
0.112	0.8919\\
0.1167	0.8869\\
0.1213	0.8816\\
0.126	0.8761\\
0.1306	0.8704\\
0.1352	0.8645\\
0.1399	0.8584\\
0.1445	0.852\\
0.1492	0.8455\\
0.1538	0.8387\\
0.1585	0.8318\\
0.1631	0.8246\\
0.1677	0.8172\\
0.1724	0.8096\\
0.177	0.8018\\
0.1817	0.7938\\
0.1863	0.7855\\
0.187	0.7844\\
0.1882	0.782\\
0.1889	0.7809\\
};
\addplot [color=black, forget plot]
  table[row sep=crcr]{%
0.1889	0.7809\\
0.1936	0.7722\\
0.1983	0.7632\\
0.2031	0.7541\\
0.2078	0.7447\\
0.2125	0.7351\\
0.2172	0.7253\\
0.2219	0.7153\\
0.2267	0.705\\
0.2314	0.6946\\
0.2361	0.6839\\
0.2408	0.673\\
0.2456	0.6619\\
0.2503	0.6505\\
0.255	0.639\\
0.2597	0.6272\\
0.2644	0.6152\\
0.2692	0.603\\
0.2739	0.5906\\
0.2786	0.5779\\
0.2833	0.5651\\
0.2881	0.552\\
0.2928	0.5387\\
0.2975	0.5251\\
0.3022	0.5114\\
0.3069	0.4974\\
0.3117	0.4832\\
0.3164	0.4688\\
0.3211	0.4542\\
0.3258	0.4394\\
0.3306	0.4243\\
0.3353	0.4091\\
0.34	0.3936\\
0.3447	0.3779\\
0.3494	0.3619\\
0.3542	0.3458\\
0.3589	0.3294\\
0.3636	0.3128\\
0.3683	0.296\\
0.3731	0.279\\
0.3778	0.2617\\
};
\addplot [color=black, forget plot]
  table[row sep=crcr]{%
0.3778	0.2617\\
0.3814	0.2485\\
0.3849	0.2352\\
0.3885	0.2217\\
0.3921	0.2081\\
0.3968	0.19\\
0.4015	0.1717\\
0.4063	0.1531\\
0.411	0.1343\\
0.4157	0.1153\\
0.4204	0.0961\\
0.4251	0.0767\\
0.4299	0.057\\
0.4332	0.0429\\
0.4366	0.0287\\
0.4399	0.0144\\
0.4433	0\\
};
\addplot [color=black, forget plot]
  table[row sep=crcr]{%
0.4433	0\\
0.4433	0.0002\\
0.4434	0.0005\\
0.4435	0.0007\\
0.4436	0.001\\
0.4436	0.0012\\
0.444	0.0025\\
0.4444	0.0037\\
0.4448	0.005\\
0.4452	0.0062\\
0.4471	0.0124\\
0.4491	0.0186\\
0.451	0.0248\\
0.453	0.0309\\
0.456	0.0405\\
0.4591	0.0501\\
0.4622	0.0596\\
0.4684	0.0782\\
0.4715	0.0874\\
0.4745	0.0965\\
0.4776	0.1055\\
0.4807	0.1144\\
0.4838	0.1232\\
0.4869	0.1319\\
0.49	0.1405\\
0.4931	0.149\\
0.4961	0.1575\\
0.5023	0.1741\\
0.5085	0.1903\\
0.5116	0.1983\\
0.5147	0.2061\\
0.5177	0.2139\\
0.5208	0.2216\\
0.5239	0.2292\\
0.527	0.2367\\
0.5301	0.2441\\
0.5332	0.2514\\
0.5362	0.2586\\
0.5393	0.2658\\
0.5455	0.2798\\
0.5517	0.2934\\
0.5548	0.3\\
0.5578	0.3066\\
0.5609	0.3131\\
0.564	0.3194\\
0.5647	0.3208\\
0.5653	0.3222\\
0.566	0.3235\\
0.5667	0.3249\\
};
\addplot [color=black, forget plot]
  table[row sep=crcr]{%
0.5667	0.3249\\
0.5714	0.3343\\
0.5761	0.3436\\
0.5808	0.3526\\
0.5856	0.3614\\
0.5903	0.37\\
0.595	0.3784\\
0.5997	0.3865\\
0.6044	0.3945\\
0.6092	0.4022\\
0.6139	0.4097\\
0.6186	0.417\\
0.6233	0.424\\
0.6281	0.4309\\
0.6328	0.4375\\
0.6375	0.4439\\
0.6422	0.4501\\
0.6469	0.456\\
0.6517	0.4618\\
0.6564	0.4673\\
0.6611	0.4726\\
0.6658	0.4777\\
0.6706	0.4826\\
0.6753	0.4872\\
0.68	0.4917\\
0.6847	0.4959\\
0.6894	0.4999\\
0.6942	0.5036\\
0.6989	0.5072\\
0.7036	0.5105\\
0.7083	0.5137\\
0.7131	0.5166\\
0.7178	0.5193\\
0.7225	0.5217\\
0.7272	0.524\\
0.7319	0.526\\
0.7367	0.5278\\
0.7414	0.5294\\
0.7461	0.5308\\
0.7508	0.5319\\
0.7556	0.5328\\
};
\addplot [color=black, forget plot]
  table[row sep=crcr]{%
0.7556	0.5328\\
0.7564	0.533\\
0.7573	0.5331\\
0.7582	0.5333\\
0.7627	0.5338\\
0.7662	0.5342\\
0.7734	0.5344\\
};
\addplot [color=black, forget plot]
  table[row sep=crcr]{%
0.7734	0.5344\\
0.7765	0.5344\\
0.7829	0.534\\
0.7861	0.5336\\
0.7894	0.5331\\
0.7936	0.5324\\
0.7979	0.5314\\
0.8022	0.5303\\
0.8065	0.529\\
0.8107	0.5275\\
0.815	0.5259\\
0.8193	0.5241\\
0.8236	0.522\\
0.8278	0.5198\\
0.8321	0.5175\\
0.8364	0.5149\\
0.8407	0.5122\\
0.845	0.5093\\
0.8492	0.5062\\
0.8535	0.5029\\
0.8578	0.4994\\
0.8621	0.4958\\
0.8663	0.492\\
0.8706	0.488\\
0.8749	0.4838\\
0.8792	0.4795\\
0.8835	0.4749\\
0.8877	0.4702\\
0.892	0.4653\\
0.8963	0.4603\\
0.9006	0.455\\
0.9048	0.4496\\
0.9091	0.444\\
0.9134	0.4382\\
0.9177	0.4322\\
0.9219	0.4261\\
0.9262	0.4198\\
0.9308	0.4128\\
0.9353	0.4057\\
0.9399	0.3984\\
0.9444	0.3908\\
};
\addplot [color=black, forget plot]
  table[row sep=crcr]{%
0.9444	0.3908\\
0.9492	0.3828\\
0.9539	0.3745\\
0.9586	0.3661\\
0.9633	0.3574\\
0.9681	0.3485\\
0.9728	0.3393\\
0.9775	0.33\\
0.9822	0.3204\\
0.9869	0.3106\\
0.9917	0.3006\\
0.9964	0.2904\\
1.0011	0.28\\
1.0058	0.2693\\
1.0106	0.2584\\
1.0153	0.2473\\
1.02	0.236\\
1.0247	0.2245\\
1.0294	0.2127\\
1.0342	0.2007\\
1.0389	0.1886\\
1.0436	0.1761\\
1.0483	0.1635\\
1.0531	0.1507\\
1.0578	0.1376\\
1.0625	0.1243\\
1.0672	0.1108\\
1.0719	0.0971\\
1.0767	0.0831\\
1.0814	0.069\\
1.0861	0.0546\\
1.0908	0.04\\
1.0956	0.0252\\
1.0975	0.0189\\
1.0995	0.0127\\
1.1015	0.0064\\
1.1034	-0\\
};
\addplot [color=black, forget plot]
  table[row sep=crcr]{%
1.1034	0\\
1.1034	0.0001\\
1.1035	0.0001\\
1.1035	0.0002\\
1.1036	0.0005\\
1.1037	0.0007\\
1.1038	0.001\\
1.1039	0.0012\\
1.1044	0.0025\\
1.105	0.0037\\
1.1055	0.005\\
1.106	0.0062\\
1.1067	0.008\\
1.1075	0.0098\\
1.1082	0.0116\\
1.109	0.0133\\
1.1097	0.0151\\
1.1105	0.0169\\
1.1112	0.0186\\
1.112	0.0204\\
1.1127	0.0221\\
1.1135	0.0239\\
1.1142	0.0256\\
1.115	0.0274\\
1.1157	0.0291\\
1.1165	0.0308\\
1.1172	0.0325\\
1.118	0.0342\\
1.1187	0.036\\
1.1195	0.0377\\
1.1202	0.0393\\
1.1209	0.041\\
1.1217	0.0427\\
1.1224	0.0444\\
1.1232	0.0461\\
1.1239	0.0477\\
1.1247	0.0494\\
1.1254	0.0511\\
1.1262	0.0527\\
1.1269	0.0544\\
1.1277	0.056\\
1.1284	0.0576\\
1.1292	0.0593\\
1.1299	0.0609\\
1.1307	0.0625\\
1.1314	0.0641\\
1.1322	0.0657\\
1.1329	0.0673\\
1.133	0.0676\\
1.1333	0.0682\\
};
\addplot [color=black, forget plot]
  table[row sep=crcr]{%
1.1333	0.0682\\
1.1349	0.0717\\
1.1365	0.075\\
1.1382	0.0784\\
1.1398	0.0817\\
1.1445	0.0914\\
1.1492	0.1009\\
1.1539	0.1101\\
1.1586	0.1191\\
1.1634	0.1279\\
1.1681	0.1365\\
1.1728	0.1449\\
1.1775	0.153\\
1.1823	0.161\\
1.187	0.1687\\
1.1917	0.1762\\
1.1964	0.1834\\
1.2011	0.1905\\
1.2059	0.1973\\
1.2106	0.2039\\
1.2153	0.2103\\
1.22	0.2165\\
1.2248	0.2224\\
1.2295	0.2282\\
1.2342	0.2337\\
1.2389	0.239\\
1.2436	0.2441\\
1.2531	0.2536\\
1.2578	0.258\\
1.2625	0.2622\\
1.2673	0.2662\\
1.272	0.27\\
1.2767	0.2735\\
1.2814	0.2769\\
1.2861	0.28\\
1.2909	0.2829\\
1.2956	0.2855\\
1.3003	0.288\\
1.305	0.2902\\
1.3098	0.2923\\
1.3129	0.2935\\
1.316	0.2946\\
1.3191	0.2956\\
1.3222	0.2965\\
};
\addplot [color=black, forget plot]
  table[row sep=crcr]{%
1.3222	0.2965\\
1.3237	0.2969\\
1.3251	0.2973\\
1.3266	0.2977\\
1.328	0.298\\
1.3327	0.299\\
1.3374	0.2997\\
1.3422	0.3002\\
1.3469	0.3005\\
1.3479	0.3006\\
1.351	0.3006\\
};
\addplot [color=black, forget plot]
  table[row sep=crcr]{%
1.351	0.3006\\
1.3542	0.3006\\
1.3574	0.3004\\
1.3638	0.2998\\
1.367	0.2993\\
1.371	0.2986\\
1.375	0.2978\\
1.379	0.2968\\
1.383	0.2956\\
1.387	0.2942\\
1.391	0.2927\\
1.395	0.2911\\
1.399	0.2893\\
1.403	0.2873\\
1.407	0.2852\\
1.411	0.2829\\
1.415	0.2805\\
1.419	0.2779\\
1.423	0.2751\\
1.427	0.2722\\
1.431	0.2692\\
1.439	0.2626\\
1.443	0.259\\
1.447	0.2553\\
1.451	0.2515\\
1.4551	0.2475\\
1.4591	0.2433\\
1.4631	0.239\\
1.4671	0.2345\\
1.4711	0.2299\\
1.4751	0.2251\\
1.4791	0.2201\\
1.4831	0.215\\
1.4871	0.2098\\
1.4951	0.1988\\
1.4991	0.193\\
1.5031	0.1871\\
1.5071	0.1811\\
1.5111	0.1748\\
};
\addplot [color=black, forget plot]
  table[row sep=crcr]{%
1.5111	0.1748\\
1.5158	0.1673\\
1.5206	0.1596\\
1.5253	0.1516\\
1.53	0.1434\\
1.5347	0.135\\
1.5394	0.1264\\
1.5442	0.1176\\
1.5489	0.1085\\
1.5536	0.0992\\
1.5583	0.0897\\
1.5631	0.08\\
1.5678	0.0701\\
1.5725	0.0599\\
1.5772	0.0495\\
1.5819	0.039\\
1.5867	0.0282\\
1.5896	0.0212\\
1.5956	0.0072\\
1.5985	-0\\
};
\addplot [color=black, forget plot]
  table[row sep=crcr]{%
1.5985	0\\
1.5986	0.0001\\
1.5986	0.0002\\
1.5987	0.0002\\
1.5988	0.0005\\
1.5989	0.0007\\
1.5991	0.001\\
1.5992	0.0012\\
1.5999	0.0025\\
1.6013	0.0049\\
1.602	0.0062\\
1.6045	0.0107\\
1.607	0.0151\\
1.6096	0.0195\\
1.6146	0.0281\\
1.6172	0.0322\\
1.6197	0.0364\\
1.6223	0.0404\\
1.6273	0.0484\\
1.6299	0.0522\\
1.6349	0.0598\\
1.6375	0.0635\\
1.64	0.0671\\
1.6425	0.0706\\
1.6451	0.0741\\
1.6476	0.0776\\
1.6502	0.0809\\
1.6552	0.0875\\
1.6578	0.0907\\
1.6603	0.0938\\
1.6628	0.0968\\
1.6654	0.0998\\
1.6704	0.1056\\
1.673	0.1084\\
1.6755	0.1111\\
1.6781	0.1138\\
1.6831	0.119\\
1.6857	0.1215\\
1.6882	0.1239\\
1.6907	0.1262\\
1.6933	0.1285\\
1.695	0.13\\
1.6966	0.1315\\
1.7	0.1343\\
};
\addplot [color=black, forget plot]
  table[row sep=crcr]{%
0	0.9932\\
0.0008	0.9932\\
0.001	0.9933\\
0.005	0.9933\\
};
\addplot [color=black, forget plot]
  table[row sep=crcr]{%
0.005	0.9933\\
0.0082	0.9933\\
0.0146	0.9929\\
0.021	0.9921\\
0.0256	0.9913\\
0.0302	0.9902\\
0.0348	0.989\\
0.0394	0.9875\\
0.044	0.9859\\
0.0486	0.984\\
0.0532	0.982\\
0.0578	0.9797\\
0.0624	0.9772\\
0.067	0.9745\\
0.0716	0.9716\\
0.0762	0.9685\\
0.0808	0.9652\\
0.0854	0.9617\\
0.09	0.9579\\
0.0946	0.954\\
0.0992	0.9499\\
0.1038	0.9455\\
0.1084	0.941\\
0.113	0.9362\\
0.1176	0.9312\\
0.1222	0.9261\\
0.1267	0.9207\\
0.1313	0.9151\\
0.1359	0.9093\\
0.1405	0.9033\\
0.1451	0.8971\\
0.1497	0.8906\\
0.1543	0.884\\
0.1589	0.8772\\
0.1635	0.8701\\
0.1681	0.8629\\
0.1727	0.8554\\
0.1773	0.8478\\
0.1819	0.8399\\
0.1865	0.8318\\
0.1871	0.8307\\
0.1877	0.8297\\
0.1889	0.8275\\
};
\addplot [color=black, forget plot]
  table[row sep=crcr]{%
0.1889	0.8275\\
0.1936	0.8189\\
0.1983	0.8101\\
0.2031	0.801\\
0.2078	0.7917\\
0.2125	0.7822\\
0.2172	0.7725\\
0.2219	0.7625\\
0.2267	0.7524\\
0.2314	0.742\\
0.2361	0.7314\\
0.2408	0.7206\\
0.2456	0.7096\\
0.2503	0.6983\\
0.255	0.6868\\
0.2597	0.6752\\
0.2644	0.6632\\
0.2692	0.6511\\
0.2739	0.6388\\
0.2786	0.6262\\
0.2833	0.6134\\
0.2881	0.6004\\
0.2928	0.5872\\
0.2975	0.5738\\
0.3022	0.5601\\
0.3069	0.5462\\
0.3117	0.5321\\
0.3164	0.5178\\
0.3211	0.5033\\
0.3258	0.4885\\
0.3306	0.4736\\
0.3353	0.4584\\
0.34	0.443\\
0.3447	0.4273\\
0.3494	0.4115\\
0.3542	0.3954\\
0.3589	0.3791\\
0.3636	0.3626\\
0.3683	0.3459\\
0.3731	0.329\\
0.3778	0.3118\\
};
\addplot [color=black, forget plot]
  table[row sep=crcr]{%
0.3778	0.3118\\
0.3821	0.2961\\
0.3863	0.2801\\
0.3906	0.264\\
0.3949	0.2477\\
0.3996	0.2296\\
0.4044	0.2112\\
0.4091	0.1926\\
0.4138	0.1737\\
0.4185	0.1547\\
0.4232	0.1354\\
0.428	0.1159\\
0.4327	0.0962\\
0.4374	0.0763\\
0.4421	0.0562\\
0.4469	0.0358\\
0.4516	0.0152\\
0.4524	0.0114\\
0.4542	0.0038\\
0.455	-0\\
};
\addplot [color=black, forget plot]
  table[row sep=crcr]{%
0.455	0\\
0.4551	0.0001\\
0.4551	0.0002\\
0.4552	0.0005\\
0.4553	0.0007\\
0.4553	0.001\\
0.4554	0.0012\\
0.4558	0.0025\\
0.4562	0.0037\\
0.4565	0.005\\
0.4569	0.0062\\
0.4626	0.0248\\
0.4645	0.0309\\
0.4673	0.0399\\
0.4729	0.0575\\
0.4785	0.0749\\
0.4813	0.0834\\
0.484	0.0919\\
0.4868	0.1003\\
0.4924	0.1169\\
0.4952	0.1251\\
0.498	0.1332\\
0.5008	0.1412\\
0.5064	0.157\\
0.5092	0.1648\\
0.512	0.1725\\
0.5147	0.1802\\
0.5203	0.1952\\
0.5259	0.21\\
0.5315	0.2244\\
0.5371	0.2386\\
0.5399	0.2455\\
0.5426	0.2524\\
0.5454	0.2592\\
0.551	0.2726\\
0.5538	0.2792\\
0.5566	0.2857\\
0.5594	0.2921\\
0.5654	0.3056\\
0.5658	0.3066\\
0.5662	0.3075\\
0.5667	0.3085\\
};
\addplot [color=black, forget plot]
  table[row sep=crcr]{%
0.5667	0.3085\\
0.5714	0.3188\\
0.5761	0.329\\
0.5808	0.3389\\
0.5856	0.3486\\
0.5903	0.3581\\
0.595	0.3673\\
0.5997	0.3764\\
0.6044	0.3852\\
0.6092	0.3938\\
0.6139	0.4022\\
0.6186	0.4103\\
0.6233	0.4183\\
0.6281	0.426\\
0.6328	0.4335\\
0.6375	0.4408\\
0.6422	0.4479\\
0.6469	0.4547\\
0.6517	0.4614\\
0.6564	0.4678\\
0.6611	0.474\\
0.6658	0.48\\
0.6706	0.4857\\
0.6753	0.4913\\
0.68	0.4966\\
0.6847	0.5017\\
0.6894	0.5066\\
0.6942	0.5113\\
0.6989	0.5157\\
0.7036	0.5199\\
0.7083	0.524\\
0.7131	0.5277\\
0.7178	0.5313\\
0.7225	0.5347\\
0.7272	0.5378\\
0.7319	0.5407\\
0.7367	0.5434\\
0.7414	0.5459\\
0.7461	0.5482\\
0.7508	0.5502\\
0.7556	0.552\\
};
\addplot [color=black, forget plot]
  table[row sep=crcr]{%
0.7556	0.552\\
0.7574	0.5527\\
0.7593	0.5533\\
0.7611	0.5539\\
0.763	0.5545\\
0.7677	0.5557\\
0.7724	0.5568\\
0.7772	0.5576\\
0.7819	0.5582\\
0.7845	0.5584\\
0.7872	0.5586\\
0.7926	0.5588\\
};
\addplot [color=black, forget plot]
  table[row sep=crcr]{%
0.7926	0.5588\\
0.7929	0.5588\\
0.7931	0.5587\\
0.7958	0.5587\\
0.799	0.5586\\
0.8022	0.5583\\
0.8086	0.5575\\
0.8124	0.5568\\
0.8162	0.556\\
0.8199	0.5551\\
0.8237	0.554\\
0.8313	0.5514\\
0.8351	0.5499\\
0.8389	0.5482\\
0.8427	0.5464\\
0.8465	0.5445\\
0.8503	0.5424\\
0.8541	0.5402\\
0.8579	0.5378\\
0.8617	0.5353\\
0.8693	0.5299\\
0.8769	0.5239\\
0.8845	0.5173\\
0.8883	0.5138\\
0.8921	0.5102\\
0.8959	0.5064\\
0.8997	0.5025\\
0.9035	0.4984\\
0.9073	0.4942\\
0.9149	0.4854\\
0.9225	0.476\\
0.9263	0.4711\\
0.9301	0.466\\
0.9337	0.4611\\
0.9373	0.4561\\
0.9408	0.4509\\
0.9444	0.4456\\
};
\addplot [color=black, forget plot]
  table[row sep=crcr]{%
0.9444	0.4456\\
0.9492	0.4384\\
0.9539	0.4311\\
0.9586	0.4235\\
0.9633	0.4157\\
0.9681	0.4077\\
0.9728	0.3994\\
0.9775	0.391\\
0.9822	0.3823\\
0.9869	0.3734\\
0.9917	0.3643\\
0.9964	0.355\\
1.0011	0.3454\\
1.0058	0.3356\\
1.0106	0.3257\\
1.0153	0.3154\\
1.02	0.305\\
1.0247	0.2944\\
1.0294	0.2835\\
1.0342	0.2724\\
1.0389	0.2611\\
1.0436	0.2496\\
1.0483	0.2379\\
1.0531	0.2259\\
1.0578	0.2137\\
1.0625	0.2013\\
1.0672	0.1887\\
1.0719	0.1759\\
1.0767	0.1628\\
1.0814	0.1496\\
1.0861	0.1361\\
1.0908	0.1224\\
1.0956	0.1084\\
1.1003	0.0943\\
1.105	0.0799\\
1.1097	0.0653\\
1.1144	0.0505\\
1.1184	0.0381\\
1.1223	0.0256\\
1.1262	0.0129\\
1.1301	0\\
};
\addplot [color=black, forget plot]
  table[row sep=crcr]{%
1.1301	0\\
1.1301	0.0002\\
1.1302	0.0002\\
1.1302	0.0004\\
1.1306	0.0012\\
1.1306	0.0014\\
1.131	0.0022\\
1.131	0.0024\\
1.1315	0.0034\\
1.1315	0.0036\\
1.1319	0.0044\\
1.1319	0.0046\\
1.1324	0.0056\\
1.1324	0.0058\\
1.1328	0.0066\\
1.1328	0.0068\\
1.1331	0.0074\\
1.1331	0.0076\\
1.1332	0.0077\\
1.1333	0.0079\\
1.1333	0.0081\\
};
\addplot [color=black, forget plot]
  table[row sep=crcr]{%
1.1333	0.0081\\
1.1337	0.0089\\
1.1338	0.0093\\
1.134	0.0097\\
1.1356	0.0137\\
1.1365	0.0157\\
1.1373	0.0177\\
1.1414	0.0275\\
1.1455	0.0372\\
1.1497	0.0468\\
1.1538	0.0561\\
1.1585	0.0667\\
1.1632	0.077\\
1.168	0.087\\
1.1727	0.0969\\
1.1774	0.1065\\
1.1821	0.116\\
1.1868	0.1252\\
1.1916	0.1342\\
1.1963	0.1429\\
1.201	0.1515\\
1.2057	0.1598\\
1.2105	0.1679\\
1.2152	0.1758\\
1.2199	0.1835\\
1.2246	0.1909\\
1.2293	0.1982\\
1.2341	0.2052\\
1.2388	0.212\\
1.2435	0.2186\\
1.2482	0.2249\\
1.253	0.2311\\
1.2577	0.237\\
1.2624	0.2427\\
1.2671	0.2482\\
1.2718	0.2535\\
1.2766	0.2585\\
1.2813	0.2634\\
1.286	0.268\\
1.2907	0.2724\\
1.2955	0.2765\\
1.3002	0.2805\\
1.3049	0.2842\\
1.3092	0.2875\\
1.3136	0.2905\\
1.3179	0.2934\\
1.3222	0.2961\\
};
\addplot [color=black, forget plot]
  table[row sep=crcr]{%
1.3222	0.2961\\
1.3253	0.2978\\
1.3283	0.2995\\
1.3345	0.3027\\
1.3392	0.3048\\
1.3439	0.3067\\
1.3486	0.3084\\
1.3534	0.3099\\
1.3581	0.3112\\
1.3628	0.3123\\
1.3675	0.3131\\
1.3723	0.3137\\
1.375	0.314\\
1.3777	0.3142\\
1.3805	0.3143\\
1.3832	0.3143\\
};
\addplot [color=black, forget plot]
  table[row sep=crcr]{%
1.3832	0.3143\\
1.3858	0.3143\\
1.3864	0.3142\\
1.3896	0.3141\\
1.396	0.3135\\
1.4024	0.3125\\
1.4088	0.3111\\
1.4152	0.3093\\
1.4216	0.3071\\
1.428	0.3045\\
1.4344	0.3015\\
1.4376	0.2998\\
1.444	0.2962\\
1.4504	0.2922\\
1.4536	0.29\\
1.4567	0.2878\\
1.4631	0.283\\
1.4663	0.2804\\
1.4727	0.275\\
1.4791	0.2692\\
1.4823	0.2661\\
1.4887	0.2597\\
1.4951	0.2529\\
1.4983	0.2493\\
1.5015	0.2456\\
1.5063	0.24\\
1.5087	0.237\\
1.5111	0.2341\\
};
\addplot [color=black, forget plot]
  table[row sep=crcr]{%
1.5111	0.2341\\
1.5158	0.228\\
1.5206	0.2218\\
1.5253	0.2153\\
1.53	0.2086\\
1.5347	0.2017\\
1.5394	0.1946\\
1.5442	0.1872\\
1.5489	0.1796\\
1.5536	0.1719\\
1.5583	0.1639\\
1.5631	0.1556\\
1.5678	0.1472\\
1.5725	0.1385\\
1.5772	0.1297\\
1.5819	0.1206\\
1.5867	0.1112\\
1.5914	0.1017\\
1.5961	0.092\\
1.6008	0.082\\
1.6056	0.0718\\
1.6103	0.0614\\
1.615	0.0508\\
1.6197	0.0399\\
1.6244	0.0288\\
1.6274	0.0218\\
1.6304	0.0146\\
1.6334	0.0073\\
1.6363	-0\\
};
\addplot [color=black, forget plot]
  table[row sep=crcr]{%
1.6363	0\\
1.6364	0.0001\\
1.6364	0.0002\\
1.6366	0.0005\\
1.6367	0.0007\\
1.6369	0.001\\
1.637	0.0012\\
1.6377	0.0025\\
1.6383	0.0037\\
1.639	0.0049\\
1.6397	0.0062\\
1.6413	0.0091\\
1.6429	0.0119\\
1.6445	0.0148\\
1.646	0.0176\\
1.6492	0.0232\\
1.6508	0.0259\\
1.6524	0.0287\\
1.6556	0.0341\\
1.6636	0.0471\\
1.6651	0.0496\\
1.6683	0.0546\\
1.6763	0.0666\\
1.6811	0.0735\\
1.6827	0.0757\\
1.6842	0.078\\
1.6858	0.0802\\
1.6874	0.0823\\
1.689	0.0845\\
1.6954	0.0929\\
1.697	0.0949\\
1.6977	0.0959\\
1.6985	0.0968\\
1.6992	0.0977\\
1.7	0.0987\\
};
\addplot [color=black, forget plot]
  table[row sep=crcr]{%
0	0.9996\\
0.0018	0.9996\\
0.0054	0.9994\\
0.0073	0.9993\\
0.0091	0.9991\\
0.0139	0.9985\\
0.0186	0.9977\\
0.0233	0.9967\\
0.028	0.9954\\
0.0327	0.9939\\
0.0375	0.9922\\
0.0422	0.9903\\
0.0469	0.9882\\
0.0516	0.9858\\
0.0564	0.9833\\
0.0611	0.9805\\
0.0658	0.9775\\
0.0705	0.9742\\
0.0752	0.9708\\
0.08	0.9671\\
0.0847	0.9633\\
0.0894	0.9592\\
0.0941	0.9548\\
0.0989	0.9503\\
0.1036	0.9455\\
0.1083	0.9406\\
0.113	0.9354\\
0.1177	0.93\\
0.1225	0.9243\\
0.1272	0.9185\\
0.1319	0.9124\\
0.1366	0.9061\\
0.1414	0.8996\\
0.1461	0.8929\\
0.1508	0.8859\\
0.1555	0.8788\\
0.1602	0.8714\\
0.165	0.8638\\
0.1697	0.856\\
0.1744	0.8479\\
0.1791	0.8397\\
0.1816	0.8353\\
0.1864	0.8265\\
0.1889	0.8219\\
};
\addplot [color=black, forget plot]
  table[row sep=crcr]{%
0.1889	0.8219\\
0.1936	0.813\\
0.1983	0.8039\\
0.2031	0.7945\\
0.2078	0.7849\\
0.2125	0.7751\\
0.2172	0.7651\\
0.2219	0.7548\\
0.2267	0.7444\\
0.2314	0.7337\\
0.2361	0.7228\\
0.2408	0.7117\\
0.2456	0.7004\\
0.2503	0.6888\\
0.255	0.677\\
0.2597	0.665\\
0.2644	0.6528\\
0.2692	0.6404\\
0.2739	0.6278\\
0.2786	0.6149\\
0.2833	0.6018\\
0.2881	0.5885\\
0.2928	0.575\\
0.2975	0.5612\\
0.3022	0.5473\\
0.3069	0.5331\\
0.3117	0.5187\\
0.3164	0.5041\\
0.3211	0.4893\\
0.3258	0.4742\\
0.3306	0.4589\\
0.3353	0.4435\\
0.34	0.4277\\
0.3447	0.4118\\
0.3494	0.3957\\
0.3542	0.3793\\
0.3589	0.3627\\
0.3636	0.3459\\
0.3683	0.3289\\
0.3731	0.3117\\
0.3778	0.2942\\
};
\addplot [color=black, forget plot]
  table[row sep=crcr]{%
0.3778	0.2942\\
0.3818	0.2793\\
0.3857	0.2643\\
0.3897	0.2492\\
0.3937	0.2338\\
0.3984	0.2154\\
0.4031	0.1968\\
0.4078	0.1779\\
0.4126	0.1589\\
0.4173	0.1396\\
0.422	0.1201\\
0.4267	0.1003\\
0.4314	0.0804\\
0.4361	0.0606\\
0.4407	0.0406\\
0.4453	0.0204\\
0.45	-0\\
};
\addplot [color=black, forget plot]
  table[row sep=crcr]{%
0.45	0\\
0.45	0.0002\\
0.4501	0.0005\\
0.4502	0.0007\\
0.4503	0.001\\
0.4503	0.0012\\
0.4507	0.0025\\
0.4511	0.0037\\
0.4515	0.005\\
0.4519	0.0062\\
0.4537	0.0124\\
0.4575	0.0248\\
0.4594	0.0309\\
0.4623	0.0403\\
0.4653	0.0496\\
0.4682	0.0588\\
0.474	0.077\\
0.4798	0.0948\\
0.4828	0.1036\\
0.4857	0.1123\\
0.4915	0.1295\\
0.4973	0.1463\\
0.5003	0.1546\\
0.5032	0.1628\\
0.509	0.179\\
0.5148	0.1948\\
0.5178	0.2026\\
0.5207	0.2103\\
0.5265	0.2255\\
0.5323	0.2403\\
0.5353	0.2476\\
0.5382	0.2548\\
0.544	0.269\\
0.5469	0.2759\\
0.5499	0.2828\\
0.5528	0.2896\\
0.5557	0.2963\\
0.5615	0.3095\\
0.5644	0.3159\\
0.565	0.3171\\
0.5656	0.3184\\
0.5661	0.3196\\
0.5667	0.3208\\
};
\addplot [color=black, forget plot]
  table[row sep=crcr]{%
0.5667	0.3208\\
0.5714	0.331\\
0.5761	0.3409\\
0.5808	0.3506\\
0.5856	0.3602\\
0.5903	0.3695\\
0.595	0.3785\\
0.5997	0.3874\\
0.6044	0.396\\
0.6092	0.4044\\
0.6139	0.4126\\
0.6186	0.4206\\
0.6233	0.4284\\
0.6281	0.4359\\
0.6328	0.4433\\
0.6375	0.4504\\
0.6422	0.4573\\
0.6469	0.4639\\
0.6517	0.4704\\
0.6564	0.4766\\
0.6611	0.4826\\
0.6658	0.4884\\
0.6706	0.494\\
0.6753	0.4994\\
0.68	0.5045\\
0.6847	0.5094\\
0.6894	0.5141\\
0.6942	0.5186\\
0.6989	0.5229\\
0.7036	0.5269\\
0.7083	0.5307\\
0.7131	0.5343\\
0.7178	0.5377\\
0.7225	0.5409\\
0.7272	0.5438\\
0.7319	0.5466\\
0.7367	0.5491\\
0.7414	0.5514\\
0.7461	0.5535\\
0.7508	0.5553\\
0.7556	0.557\\
};
\addplot [color=black, forget plot]
  table[row sep=crcr]{%
0.7556	0.557\\
0.7572	0.5575\\
0.7589	0.558\\
0.7605	0.5585\\
0.7622	0.5589\\
0.7669	0.56\\
0.7716	0.5609\\
0.7764	0.5616\\
0.7811	0.562\\
0.7829	0.5622\\
0.7848	0.5622\\
0.7867	0.5623\\
0.7886	0.5623\\
};
\addplot [color=black, forget plot]
  table[row sep=crcr]{%
0.7886	0.5623\\
0.7918	0.5623\\
0.7982	0.5619\\
0.8046	0.5611\\
0.8085	0.5604\\
0.8124	0.5595\\
0.8163	0.5585\\
0.8202	0.5574\\
0.824	0.5561\\
0.8279	0.5547\\
0.8318	0.5531\\
0.8357	0.5514\\
0.8396	0.5495\\
0.8435	0.5475\\
0.8474	0.5453\\
0.8513	0.543\\
0.8552	0.5405\\
0.8591	0.5379\\
0.863	0.5351\\
0.8669	0.5322\\
0.8708	0.5291\\
0.8747	0.5259\\
0.8786	0.5225\\
0.8825	0.519\\
0.8864	0.5154\\
0.8942	0.5076\\
0.8981	0.5035\\
0.902	0.4992\\
0.9059	0.4948\\
0.9098	0.4902\\
0.9137	0.4855\\
0.9176	0.4807\\
0.9215	0.4757\\
0.9254	0.4705\\
0.9293	0.4652\\
0.9331	0.4599\\
0.9407	0.4489\\
0.9444	0.4431\\
};
\addplot [color=black, forget plot]
  table[row sep=crcr]{%
0.9444	0.4431\\
0.9492	0.4358\\
0.9539	0.4282\\
0.9586	0.4205\\
0.9633	0.4125\\
0.9681	0.4043\\
0.9728	0.3959\\
0.9775	0.3872\\
0.9822	0.3784\\
0.9869	0.3693\\
0.9917	0.36\\
0.9964	0.3505\\
1.0011	0.3407\\
1.0058	0.3308\\
1.0106	0.3206\\
1.0153	0.3102\\
1.02	0.2996\\
1.0247	0.2888\\
1.0294	0.2777\\
1.0342	0.2664\\
1.0389	0.255\\
1.0436	0.2432\\
1.0483	0.2313\\
1.0531	0.2192\\
1.0578	0.2068\\
1.0625	0.1942\\
1.0672	0.1814\\
1.0719	0.1684\\
1.0767	0.1552\\
1.0814	0.1417\\
1.0861	0.1281\\
1.0908	0.1142\\
1.0956	0.1\\
1.1003	0.0857\\
1.105	0.0712\\
1.1097	0.0564\\
1.1144	0.0414\\
1.1176	0.0312\\
1.1208	0.0209\\
1.124	0.0105\\
1.1272	-0\\
};
\addplot [color=black, forget plot]
  table[row sep=crcr]{%
1.1272	0\\
1.1272	0.0002\\
1.1273	0.0005\\
1.1274	0.0007\\
1.1275	0.001\\
1.1278	0.0016\\
1.1279	0.002\\
1.1281	0.0024\\
1.1283	0.0027\\
1.1284	0.0031\\
1.1286	0.0035\\
1.1287	0.0039\\
1.1289	0.0043\\
1.129	0.0047\\
1.1292	0.005\\
1.1293	0.0054\\
1.1295	0.0058\\
1.1296	0.0062\\
1.1298	0.0066\\
1.13	0.0069\\
1.1301	0.0073\\
1.1303	0.0077\\
1.1304	0.0081\\
1.1306	0.0085\\
1.1307	0.0088\\
1.1309	0.0092\\
1.131	0.0096\\
1.1312	0.01\\
1.1313	0.0104\\
1.1315	0.0107\\
1.1317	0.0111\\
1.1318	0.0115\\
1.132	0.0119\\
1.1321	0.0123\\
1.1323	0.0126\\
1.1324	0.013\\
1.1326	0.0134\\
1.1327	0.0138\\
1.1329	0.0141\\
1.133	0.0145\\
1.1332	0.0149\\
1.1332	0.015\\
1.1333	0.0151\\
1.1333	0.0152\\
};
\addplot [color=black, forget plot]
  table[row sep=crcr]{%
1.1333	0.0152\\
1.1336	0.016\\
1.134	0.0167\\
1.1346	0.0183\\
1.1362	0.0221\\
1.1377	0.0258\\
1.1393	0.0296\\
1.1409	0.0333\\
1.1456	0.0443\\
1.1503	0.0551\\
1.155	0.0657\\
1.1598	0.076\\
1.1645	0.0862\\
1.1692	0.0961\\
1.1739	0.1058\\
1.1787	0.1153\\
1.1834	0.1246\\
1.1881	0.1336\\
1.1928	0.1424\\
1.1975	0.1511\\
1.2023	0.1595\\
1.207	0.1676\\
1.2117	0.1756\\
1.2164	0.1833\\
1.2212	0.1908\\
1.2259	0.1981\\
1.2306	0.2052\\
1.2353	0.2121\\
1.24	0.2187\\
1.2448	0.2251\\
1.2495	0.2314\\
1.2542	0.2373\\
1.2589	0.2431\\
1.2637	0.2487\\
1.2684	0.254\\
1.2731	0.2591\\
1.2778	0.264\\
1.2825	0.2687\\
1.2873	0.2731\\
1.292	0.2774\\
1.2967	0.2814\\
1.3014	0.2852\\
1.3062	0.2888\\
1.3109	0.2921\\
1.3137	0.294\\
1.3166	0.2959\\
1.3222	0.2993\\
};
\addplot [color=black, forget plot]
  table[row sep=crcr]{%
1.3222	0.2993\\
1.3252	0.301\\
1.3281	0.3025\\
1.3311	0.304\\
1.3341	0.3054\\
1.3388	0.3075\\
1.3435	0.3094\\
1.3482	0.311\\
1.3529	0.3124\\
1.3577	0.3136\\
1.3624	0.3146\\
1.3671	0.3153\\
1.3718	0.3159\\
1.3741	0.3161\\
1.3765	0.3162\\
1.3788	0.3163\\
1.3811	0.3163\\
};
\addplot [color=black, forget plot]
  table[row sep=crcr]{%
1.3811	0.3163\\
1.3836	0.3163\\
1.3843	0.3162\\
1.3875	0.3161\\
1.3939	0.3155\\
1.4003	0.3145\\
1.4036	0.3138\\
1.4068	0.313\\
1.4101	0.3122\\
1.4133	0.3112\\
1.4166	0.3101\\
1.4198	0.3089\\
1.4231	0.3076\\
1.4263	0.3063\\
1.4296	0.3048\\
1.4328	0.3032\\
1.4361	0.3015\\
1.4393	0.2997\\
1.4426	0.2977\\
1.4458	0.2957\\
1.4491	0.2936\\
1.4523	0.2914\\
1.4556	0.2891\\
1.4588	0.2866\\
1.4621	0.2841\\
1.4653	0.2815\\
1.4686	0.2787\\
1.4718	0.2759\\
1.4784	0.2699\\
1.4816	0.2667\\
1.4849	0.2635\\
1.4881	0.2601\\
1.4914	0.2567\\
1.4946	0.2531\\
1.4979	0.2494\\
1.5011	0.2456\\
1.5036	0.2427\\
1.5111	0.2334\\
};
\addplot [color=black, forget plot]
  table[row sep=crcr]{%
1.5111	0.2334\\
1.5158	0.2272\\
1.5206	0.2209\\
1.5253	0.2143\\
1.53	0.2075\\
1.5347	0.2005\\
1.5394	0.1933\\
1.5442	0.1859\\
1.5489	0.1782\\
1.5536	0.1703\\
1.5583	0.1622\\
1.5631	0.1539\\
1.5678	0.1453\\
1.5725	0.1366\\
1.5772	0.1276\\
1.5819	0.1184\\
1.5867	0.109\\
1.5914	0.0994\\
1.5961	0.0895\\
1.6008	0.0794\\
1.6056	0.0692\\
1.6103	0.0587\\
1.615	0.0479\\
1.6197	0.037\\
1.6244	0.0258\\
1.6271	0.0195\\
1.6297	0.013\\
1.6324	0.0066\\
1.635	-0\\
};
\addplot [color=black, forget plot]
  table[row sep=crcr]{%
1.635	0\\
1.6351	0.0001\\
1.6351	0.0002\\
1.6353	0.0005\\
1.6354	0.0007\\
1.6355	0.001\\
1.6357	0.0012\\
1.6363	0.0025\\
1.6377	0.0049\\
1.6384	0.0062\\
1.64	0.0091\\
1.6416	0.0121\\
1.6432	0.015\\
1.6449	0.0179\\
1.6465	0.0208\\
1.6497	0.0264\\
1.6514	0.0292\\
1.653	0.032\\
1.6562	0.0374\\
1.6579	0.0401\\
1.6595	0.0427\\
1.6611	0.0454\\
1.6627	0.048\\
1.6644	0.0506\\
1.666	0.0531\\
1.6676	0.0557\\
1.6692	0.0582\\
1.6708	0.0606\\
1.6725	0.0631\\
1.6773	0.0703\\
1.679	0.0726\\
1.6838	0.0795\\
1.6855	0.0818\\
1.6887	0.0862\\
1.6903	0.0883\\
1.692	0.0905\\
1.6952	0.0947\\
1.7	0.1007\\
};
\addplot [color=black, forget plot]
  table[row sep=crcr]{%
0	0.981\\
0.0029	0.981\\
0.0043	0.9809\\
0.0073	0.9807\\
0.012	0.9802\\
0.0167	0.9795\\
0.0214	0.9785\\
0.0262	0.9774\\
0.0309	0.976\\
0.0356	0.9744\\
0.0403	0.9726\\
0.0451	0.9706\\
0.0498	0.9683\\
0.0545	0.9658\\
0.0592	0.9631\\
0.0639	0.9602\\
0.0687	0.9571\\
0.0734	0.9538\\
0.0781	0.9502\\
0.0828	0.9464\\
0.0876	0.9424\\
0.0923	0.9382\\
0.097	0.9338\\
0.1017	0.9291\\
0.1064	0.9242\\
0.1112	0.9191\\
0.1159	0.9138\\
0.1206	0.9083\\
0.1253	0.9025\\
0.1301	0.8966\\
0.1348	0.8904\\
0.1395	0.884\\
0.1442	0.8774\\
0.1489	0.8705\\
0.1537	0.8634\\
0.1584	0.8562\\
0.1631	0.8487\\
0.1678	0.8409\\
0.1726	0.833\\
0.1773	0.8248\\
0.1802	0.8197\\
0.1831	0.8145\\
0.1889	0.8039\\
};
\addplot [color=black, forget plot]
  table[row sep=crcr]{%
0.1889	0.8039\\
0.1936	0.7949\\
0.1983	0.7858\\
0.2031	0.7765\\
0.2078	0.7669\\
0.2125	0.7571\\
0.2172	0.7471\\
0.2219	0.7369\\
0.2267	0.7264\\
0.2314	0.7158\\
0.2361	0.7049\\
0.2408	0.6938\\
0.2456	0.6825\\
0.2503	0.6709\\
0.255	0.6592\\
0.2597	0.6472\\
0.2644	0.635\\
0.2692	0.6226\\
0.2739	0.6099\\
0.2786	0.5971\\
0.2833	0.584\\
0.2881	0.5707\\
0.2928	0.5572\\
0.2975	0.5435\\
0.3022	0.5296\\
0.3069	0.5154\\
0.3117	0.501\\
0.3164	0.4864\\
0.3211	0.4716\\
0.3258	0.4565\\
0.3306	0.4413\\
0.3353	0.4258\\
0.34	0.4101\\
0.3447	0.3942\\
0.3494	0.3781\\
0.3542	0.3617\\
0.3589	0.3451\\
0.3636	0.3284\\
0.3683	0.3113\\
0.3731	0.2941\\
0.3778	0.2767\\
};
\addplot [color=black, forget plot]
  table[row sep=crcr]{%
0.3778	0.2767\\
0.3815	0.2627\\
0.3853	0.2486\\
0.389	0.2344\\
0.3927	0.22\\
0.3975	0.2016\\
0.4022	0.183\\
0.4069	0.1643\\
0.4116	0.1452\\
0.4163	0.126\\
0.4211	0.1066\\
0.4258	0.0869\\
0.4305	0.067\\
0.4344	0.0505\\
0.4383	0.0338\\
0.4422	0.017\\
0.4461	0\\
};
\addplot [color=black, forget plot]
  table[row sep=crcr]{%
0.4461	0\\
0.4461	0.0002\\
0.4462	0.0005\\
0.4463	0.0007\\
0.4463	0.001\\
0.4464	0.0012\\
0.4468	0.0025\\
0.4472	0.0037\\
0.4476	0.005\\
0.4479	0.0062\\
0.4499	0.0124\\
0.4537	0.0248\\
0.4556	0.0309\\
0.4586	0.0405\\
0.4616	0.05\\
0.4646	0.0594\\
0.4676	0.0687\\
0.4707	0.078\\
0.4767	0.0962\\
0.4797	0.1052\\
0.4827	0.1141\\
0.4887	0.1315\\
0.4918	0.1402\\
0.4948	0.1487\\
0.5008	0.1655\\
0.5068	0.1819\\
0.5099	0.19\\
0.5129	0.198\\
0.5159	0.2059\\
0.5189	0.2137\\
0.5219	0.2214\\
0.5279	0.2366\\
0.531	0.244\\
0.534	0.2514\\
0.537	0.2587\\
0.54	0.2659\\
0.543	0.273\\
0.546	0.28\\
0.5491	0.2869\\
0.5551	0.3005\\
0.5611	0.3137\\
0.5641	0.3201\\
0.5648	0.3215\\
0.5654	0.3228\\
0.566	0.3242\\
0.5667	0.3255\\
};
\addplot [color=black, forget plot]
  table[row sep=crcr]{%
0.5667	0.3255\\
0.5714	0.3354\\
0.5761	0.345\\
0.5808	0.3544\\
0.5856	0.3636\\
0.5903	0.3725\\
0.595	0.3813\\
0.5997	0.3898\\
0.6044	0.3981\\
0.6092	0.4062\\
0.6139	0.4141\\
0.6186	0.4217\\
0.6233	0.4292\\
0.6281	0.4364\\
0.6328	0.4434\\
0.6375	0.4502\\
0.6422	0.4567\\
0.6469	0.4631\\
0.6517	0.4692\\
0.6564	0.4751\\
0.6611	0.4808\\
0.6658	0.4862\\
0.6706	0.4915\\
0.6753	0.4965\\
0.68	0.5013\\
0.6847	0.5059\\
0.6894	0.5103\\
0.6942	0.5144\\
0.6989	0.5184\\
0.7036	0.5221\\
0.7083	0.5256\\
0.7131	0.5289\\
0.7178	0.5319\\
0.7225	0.5348\\
0.7272	0.5374\\
0.7319	0.5398\\
0.7367	0.542\\
0.7414	0.544\\
0.7461	0.5457\\
0.7508	0.5472\\
0.7556	0.5485\\
};
\addplot [color=black, forget plot]
  table[row sep=crcr]{%
0.7556	0.5485\\
0.7569	0.5489\\
0.7595	0.5495\\
0.7608	0.5497\\
0.7655	0.5506\\
0.7702	0.5512\\
0.7749	0.5516\\
0.7797	0.5518\\
0.7815	0.5518\\
};
\addplot [color=black, forget plot]
  table[row sep=crcr]{%
0.7815	0.5518\\
0.7847	0.5518\\
0.7911	0.5514\\
0.7975	0.5506\\
0.8015	0.5499\\
0.8056	0.549\\
0.8097	0.5479\\
0.8138	0.5467\\
0.8178	0.5453\\
0.8219	0.5438\\
0.826	0.5421\\
0.8301	0.5403\\
0.8341	0.5382\\
0.8382	0.536\\
0.8423	0.5337\\
0.8464	0.5312\\
0.8504	0.5285\\
0.8545	0.5257\\
0.8586	0.5227\\
0.8627	0.5195\\
0.8667	0.5162\\
0.8708	0.5127\\
0.8749	0.509\\
0.879	0.5052\\
0.8871	0.4971\\
0.8912	0.4928\\
0.8953	0.4883\\
0.8993	0.4837\\
0.9034	0.4789\\
0.9075	0.474\\
0.9116	0.4688\\
0.9156	0.4636\\
0.9197	0.4581\\
0.9238	0.4525\\
0.9278	0.4467\\
0.932	0.4407\\
0.9361	0.4345\\
0.9403	0.4281\\
0.9444	0.4216\\
};
\addplot [color=black, forget plot]
  table[row sep=crcr]{%
0.9444	0.4216\\
0.9492	0.4139\\
0.9539	0.406\\
0.9586	0.3979\\
0.9633	0.3896\\
0.9681	0.3811\\
0.9728	0.3723\\
0.9775	0.3633\\
0.9822	0.3542\\
0.9869	0.3447\\
0.9917	0.3351\\
0.9964	0.3253\\
1.0011	0.3152\\
1.0058	0.3049\\
1.0106	0.2944\\
1.0153	0.2837\\
1.02	0.2728\\
1.0247	0.2616\\
1.0294	0.2502\\
1.0342	0.2386\\
1.0389	0.2268\\
1.0436	0.2148\\
1.0483	0.2025\\
1.0531	0.19\\
1.0578	0.1774\\
1.0625	0.1644\\
1.0672	0.1513\\
1.0719	0.138\\
1.0767	0.1244\\
1.0814	0.1106\\
1.0861	0.0966\\
1.0908	0.0824\\
1.0956	0.068\\
1.1003	0.0533\\
1.105	0.0384\\
1.1097	0.0233\\
1.1144	0.008\\
1.1151	0.006\\
1.1169	-0\\
};
\addplot [color=black, forget plot]
  table[row sep=crcr]{%
1.1169	0\\
1.1169	0.0002\\
1.117	0.0002\\
1.1171	0.0005\\
1.1172	0.0007\\
1.1173	0.001\\
1.1174	0.0012\\
1.1194	0.0062\\
1.1198	0.0073\\
1.1203	0.0083\\
1.1211	0.0103\\
1.1215	0.0112\\
1.1235	0.0162\\
1.124	0.0172\\
1.1248	0.0192\\
1.1252	0.0201\\
1.1268	0.0241\\
1.1272	0.025\\
1.1277	0.026\\
1.1281	0.027\\
1.1285	0.0279\\
1.1293	0.0299\\
1.1297	0.0308\\
1.1305	0.0328\\
1.1309	0.0337\\
1.1314	0.0347\\
1.1318	0.0356\\
1.1322	0.0366\\
1.1325	0.0373\\
1.1328	0.0379\\
1.133	0.0386\\
1.1333	0.0393\\
};
\addplot [color=black, forget plot]
  table[row sep=crcr]{%
1.1333	0.0393\\
1.1342	0.0412\\
1.135	0.0432\\
1.1359	0.0451\\
1.1368	0.0471\\
1.141	0.0567\\
1.1453	0.0662\\
1.1496	0.0754\\
1.1539	0.0845\\
1.1586	0.0944\\
1.1633	0.104\\
1.168	0.1134\\
1.1727	0.1225\\
1.1775	0.1315\\
1.1822	0.1402\\
1.1869	0.1488\\
1.1916	0.1571\\
1.1964	0.1651\\
1.2011	0.173\\
1.2058	0.1806\\
1.2105	0.1881\\
1.2152	0.1953\\
1.22	0.2023\\
1.2247	0.209\\
1.2294	0.2156\\
1.2341	0.2219\\
1.2389	0.228\\
1.2436	0.2339\\
1.2483	0.2396\\
1.253	0.2451\\
1.2577	0.2503\\
1.2625	0.2553\\
1.2672	0.2601\\
1.2719	0.2647\\
1.2766	0.2691\\
1.2814	0.2732\\
1.2861	0.2771\\
1.2908	0.2808\\
1.2955	0.2843\\
1.3002	0.2876\\
1.305	0.2906\\
1.3093	0.2932\\
1.3136	0.2956\\
1.3179	0.2979\\
1.3222	0.2999\\
};
\addplot [color=black, forget plot]
  table[row sep=crcr]{%
1.3222	0.2999\\
1.3245	0.301\\
1.3269	0.3019\\
1.3315	0.3037\\
1.3362	0.3053\\
1.341	0.3067\\
1.3457	0.3079\\
1.3504	0.3088\\
1.3549	0.3095\\
1.3594	0.31\\
1.3639	0.3103\\
1.3684	0.3104\\
};
\addplot [color=black, forget plot]
  table[row sep=crcr]{%
1.3684	0.3104\\
1.3716	0.3104\\
1.378	0.31\\
1.3844	0.3092\\
1.3916	0.3078\\
1.3951	0.3069\\
1.3987	0.3059\\
1.4023	0.3048\\
1.4058	0.3035\\
1.4094	0.3022\\
1.413	0.3007\\
1.4165	0.2991\\
1.4201	0.2973\\
1.4237	0.2954\\
1.4272	0.2934\\
1.4308	0.2913\\
1.4344	0.2891\\
1.4379	0.2867\\
1.4415	0.2842\\
1.4451	0.2816\\
1.4486	0.2789\\
1.4522	0.276\\
1.4558	0.273\\
1.4593	0.2699\\
1.4665	0.2633\\
1.4736	0.2562\\
1.4772	0.2524\\
1.4807	0.2486\\
1.4843	0.2446\\
1.4879	0.2404\\
1.4914	0.2362\\
1.495	0.2318\\
1.4986	0.2273\\
1.5017	0.2233\\
1.5048	0.2192\\
1.508	0.2149\\
1.5111	0.2106\\
};
\addplot [color=black, forget plot]
  table[row sep=crcr]{%
1.5111	0.2106\\
1.5158	0.2039\\
1.5206	0.1969\\
1.5253	0.1898\\
1.53	0.1824\\
1.5347	0.1748\\
1.5394	0.167\\
1.5442	0.159\\
1.5489	0.1507\\
1.5536	0.1422\\
1.5583	0.1336\\
1.5631	0.1246\\
1.5678	0.1155\\
1.5725	0.1062\\
1.5772	0.0966\\
1.5819	0.0868\\
1.5867	0.0768\\
1.5914	0.0666\\
1.5961	0.0562\\
1.6008	0.0455\\
1.6056	0.0346\\
1.6092	0.0262\\
1.6128	0.0176\\
1.6164	0.0089\\
1.62	-0\\
};
\addplot [color=black, forget plot]
  table[row sep=crcr]{%
1.62	0\\
1.62	0.0001\\
1.6201	0.0001\\
1.6201	0.0002\\
1.6203	0.0005\\
1.6204	0.0007\\
1.6205	0.001\\
1.6207	0.0012\\
1.6213	0.0025\\
1.6227	0.0049\\
1.6234	0.0062\\
1.6274	0.0134\\
1.6334	0.0239\\
1.6374	0.0307\\
1.6434	0.0406\\
1.6474	0.047\\
1.6534	0.0563\\
1.6594	0.0653\\
1.6614	0.0682\\
1.6634	0.071\\
1.6654	0.0739\\
1.6674	0.0767\\
1.6734	0.0848\\
1.6794	0.0926\\
1.6834	0.0976\\
1.6894	0.1048\\
1.6934	0.1094\\
1.6954	0.1116\\
1.6965	0.1129\\
1.6977	0.1142\\
1.6988	0.1154\\
1.7	0.1167\\
};
\addplot [color=black, forget plot]
  table[row sep=crcr]{%
0	0.95\\
0.0008	0.95\\
0.001	0.9499\\
0.0023	0.9499\\
0.0036	0.9498\\
0.0049	0.9496\\
0.0061	0.9495\\
0.0109	0.9489\\
0.0156	0.948\\
0.0203	0.947\\
0.025	0.9457\\
0.0298	0.9442\\
0.0345	0.9424\\
0.0392	0.9405\\
0.0439	0.9383\\
0.0486	0.936\\
0.0534	0.9334\\
0.0581	0.9305\\
0.0628	0.9275\\
0.0675	0.9243\\
0.0723	0.9208\\
0.077	0.9171\\
0.0817	0.9132\\
0.0864	0.909\\
0.0911	0.9047\\
0.0959	0.9001\\
0.1006	0.8953\\
0.1053	0.8903\\
0.11	0.8851\\
0.1148	0.8797\\
0.1195	0.874\\
0.1242	0.8681\\
0.1289	0.862\\
0.1336	0.8557\\
0.1384	0.8492\\
0.1431	0.8424\\
0.1478	0.8354\\
0.1525	0.8283\\
0.1573	0.8208\\
0.162	0.8132\\
0.1667	0.8054\\
0.1714	0.7973\\
0.1761	0.789\\
0.1793	0.7833\\
0.1825	0.7775\\
0.1857	0.7716\\
0.1889	0.7656\\
};
\addplot [color=black, forget plot]
  table[row sep=crcr]{%
0.1889	0.7656\\
0.1936	0.7565\\
0.1983	0.7471\\
0.2031	0.7376\\
0.2078	0.7279\\
0.2125	0.7179\\
0.2172	0.7077\\
0.2219	0.6973\\
0.2267	0.6867\\
0.2314	0.6758\\
0.2361	0.6647\\
0.2408	0.6535\\
0.2456	0.642\\
0.2503	0.6302\\
0.255	0.6183\\
0.2597	0.6061\\
0.2644	0.5938\\
0.2692	0.5812\\
0.2739	0.5684\\
0.2786	0.5553\\
0.2833	0.5421\\
0.2881	0.5286\\
0.2928	0.5149\\
0.2975	0.501\\
0.3022	0.4869\\
0.3069	0.4725\\
0.3117	0.458\\
0.3164	0.4432\\
0.3211	0.4282\\
0.3258	0.413\\
0.3306	0.3975\\
0.3353	0.3819\\
0.34	0.366\\
0.3447	0.3499\\
0.3494	0.3336\\
0.3542	0.317\\
0.3589	0.3003\\
0.3636	0.2833\\
0.3683	0.2661\\
0.3731	0.2487\\
0.3778	0.2311\\
};
\addplot [color=black, forget plot]
  table[row sep=crcr]{%
0.3778	0.2311\\
0.384	0.2077\\
0.3871	0.1958\\
0.3901	0.1839\\
0.3949	0.1655\\
0.3996	0.1468\\
0.4043	0.128\\
0.409	0.1089\\
0.4138	0.0896\\
0.4185	0.0701\\
0.4232	0.0504\\
0.4279	0.0304\\
0.4297	0.0229\\
0.4315	0.0153\\
0.4332	0.0077\\
0.435	-0\\
};
\addplot [color=black, forget plot]
  table[row sep=crcr]{%
0.435	0\\
0.435	0.0001\\
0.4351	0.0001\\
0.4351	0.0002\\
0.4352	0.0005\\
0.4352	0.0007\\
0.4353	0.001\\
0.4354	0.0012\\
0.4358	0.0025\\
0.4362	0.0037\\
0.4366	0.005\\
0.4369	0.0062\\
0.4389	0.0124\\
0.4408	0.0186\\
0.4428	0.0248\\
0.4447	0.0309\\
0.448	0.0412\\
0.4513	0.0514\\
0.4579	0.0714\\
0.4612	0.0813\\
0.4645	0.091\\
0.4677	0.1007\\
0.4743	0.1197\\
0.4809	0.1383\\
0.4842	0.1474\\
0.4875	0.1564\\
0.4908	0.1653\\
0.4941	0.1741\\
0.4974	0.1828\\
0.5007	0.1914\\
0.5039	0.1999\\
0.5072	0.2083\\
0.5138	0.2247\\
0.5171	0.2328\\
0.5237	0.2486\\
0.527	0.2563\\
0.5336	0.2715\\
0.5369	0.2789\\
0.5401	0.2862\\
0.5434	0.2934\\
0.5467	0.3005\\
0.55	0.3075\\
0.5533	0.3144\\
0.5566	0.3212\\
0.5599	0.3279\\
0.5632	0.3344\\
0.5641	0.3362\\
0.5649	0.3379\\
0.5667	0.3413\\
};
\addplot [color=black, forget plot]
  table[row sep=crcr]{%
0.5667	0.3413\\
0.5714	0.3504\\
0.5761	0.3592\\
0.5808	0.3679\\
0.5856	0.3763\\
0.5903	0.3845\\
0.595	0.3925\\
0.5997	0.4003\\
0.6044	0.4078\\
0.6092	0.4152\\
0.6139	0.4223\\
0.6186	0.4292\\
0.6233	0.4358\\
0.6281	0.4423\\
0.6328	0.4485\\
0.6375	0.4546\\
0.6422	0.4604\\
0.6469	0.466\\
0.6517	0.4713\\
0.6564	0.4765\\
0.6611	0.4814\\
0.6658	0.4861\\
0.6706	0.4906\\
0.6753	0.4949\\
0.68	0.4989\\
0.6847	0.5027\\
0.6894	0.5064\\
0.6942	0.5098\\
0.6989	0.5129\\
0.7036	0.5159\\
0.7083	0.5186\\
0.7131	0.5212\\
0.7178	0.5235\\
0.7225	0.5255\\
0.7272	0.5274\\
0.7319	0.5291\\
0.7367	0.5305\\
0.7414	0.5317\\
0.7461	0.5327\\
0.7508	0.5334\\
0.7556	0.534\\
};
\addplot [color=black, forget plot]
  table[row sep=crcr]{%
0.7556	0.534\\
0.756	0.534\\
0.7565	0.5341\\
0.757	0.5341\\
0.7575	0.5342\\
0.7613	0.5344\\
0.7651	0.5344\\
};
\addplot [color=black, forget plot]
  table[row sep=crcr]{%
0.7651	0.5344\\
0.7683	0.5344\\
0.7747	0.534\\
0.7811	0.5332\\
0.7856	0.5324\\
0.7901	0.5314\\
0.7946	0.5302\\
0.799	0.5288\\
0.8035	0.5272\\
0.808	0.5254\\
0.8125	0.5234\\
0.817	0.5213\\
0.8215	0.5189\\
0.8259	0.5163\\
0.8304	0.5135\\
0.8349	0.5106\\
0.8394	0.5074\\
0.8439	0.504\\
0.8484	0.5005\\
0.8528	0.4967\\
0.8573	0.4927\\
0.8618	0.4886\\
0.8663	0.4842\\
0.8708	0.4797\\
0.8753	0.4749\\
0.8797	0.47\\
0.8842	0.4649\\
0.8887	0.4595\\
0.8932	0.454\\
0.8977	0.4483\\
0.9022	0.4423\\
0.9066	0.4362\\
0.9111	0.4299\\
0.9156	0.4234\\
0.9201	0.4166\\
0.9246	0.4097\\
0.9291	0.4026\\
0.9335	0.3953\\
0.938	0.3878\\
0.9425	0.3801\\
0.943	0.3792\\
0.944	0.3776\\
0.9444	0.3767\\
};
\addplot [color=black, forget plot]
  table[row sep=crcr]{%
0.9444	0.3767\\
0.9492	0.3683\\
0.9539	0.3597\\
0.9586	0.3508\\
0.9633	0.3417\\
0.9681	0.3324\\
0.9728	0.3229\\
0.9775	0.3132\\
0.9822	0.3032\\
0.9869	0.2931\\
0.9917	0.2827\\
0.9964	0.2721\\
1.0011	0.2613\\
1.0058	0.2502\\
1.0106	0.239\\
1.0153	0.2275\\
1.02	0.2158\\
1.0247	0.2039\\
1.0294	0.1917\\
1.0342	0.1794\\
1.0389	0.1668\\
1.0436	0.154\\
1.0483	0.141\\
1.0531	0.1278\\
1.0578	0.1143\\
1.0625	0.1007\\
1.0672	0.0868\\
1.0767	0.0583\\
1.0813	0.0441\\
1.0859	0.0296\\
1.0906	0.0149\\
1.0952	-0\\
};
\addplot [color=black, forget plot]
  table[row sep=crcr]{%
1.0952	0\\
1.0952	0.0001\\
1.0953	0.0002\\
1.0954	0.0005\\
1.0955	0.0007\\
1.0956	0.001\\
1.0957	0.0012\\
1.0962	0.0025\\
1.0967	0.0037\\
1.0973	0.005\\
1.0978	0.0062\\
1.0987	0.0085\\
1.0997	0.0108\\
1.1006	0.013\\
1.1016	0.0153\\
1.1025	0.0175\\
1.1035	0.0198\\
1.1044	0.022\\
1.1054	0.0242\\
1.1063	0.0265\\
1.1083	0.0309\\
1.1092	0.033\\
1.1102	0.0352\\
1.1111	0.0374\\
1.1121	0.0396\\
1.113	0.0417\\
1.114	0.0439\\
1.1149	0.046\\
1.1159	0.0481\\
1.1168	0.0502\\
1.1178	0.0523\\
1.1187	0.0544\\
1.1197	0.0565\\
1.1206	0.0586\\
1.1226	0.0628\\
1.1235	0.0648\\
1.1245	0.0669\\
1.1254	0.0689\\
1.1264	0.0709\\
1.1273	0.0729\\
1.1283	0.0749\\
1.1292	0.0769\\
1.1302	0.0789\\
1.1311	0.0809\\
1.1324	0.0835\\
1.1327	0.0842\\
1.133	0.0848\\
1.1333	0.0855\\
};
\addplot [color=black, forget plot]
  table[row sep=crcr]{%
1.1333	0.0855\\
1.1354	0.0897\\
1.1375	0.094\\
1.1396	0.0982\\
1.1417	0.1023\\
1.1464	0.1115\\
1.1511	0.1205\\
1.1559	0.1293\\
1.1606	0.1378\\
1.1653	0.1461\\
1.17	0.1543\\
1.1747	0.1621\\
1.1795	0.1698\\
1.1842	0.1773\\
1.1889	0.1845\\
1.1936	0.1915\\
1.1984	0.1983\\
1.2031	0.2049\\
1.2078	0.2113\\
1.2125	0.2174\\
1.2172	0.2233\\
1.222	0.229\\
1.2267	0.2345\\
1.2314	0.2398\\
1.2361	0.2449\\
1.2409	0.2497\\
1.2456	0.2543\\
1.2503	0.2587\\
1.255	0.2629\\
1.2597	0.2668\\
1.2645	0.2706\\
1.2692	0.2741\\
1.2739	0.2774\\
1.2786	0.2805\\
1.2834	0.2833\\
1.2881	0.286\\
1.2928	0.2884\\
1.2975	0.2906\\
1.3022	0.2926\\
1.307	0.2943\\
1.3117	0.2959\\
1.3143	0.2967\\
1.317	0.2974\\
1.3222	0.2986\\
};
\addplot [color=black, forget plot]
  table[row sep=crcr]{%
1.3222	0.2986\\
1.3233	0.2988\\
1.3243	0.299\\
1.3253	0.2991\\
1.3264	0.2993\\
1.3305	0.2999\\
1.3346	0.3003\\
1.3387	0.3005\\
1.3428	0.3006\\
};
\addplot [color=black, forget plot]
  table[row sep=crcr]{%
1.3428	0.3006\\
1.346	0.3006\\
1.3524	0.3002\\
1.3588	0.2994\\
1.363	0.2986\\
1.3672	0.2977\\
1.3714	0.2966\\
1.3756	0.2953\\
1.3798	0.2939\\
1.384	0.2923\\
1.3882	0.2905\\
1.3924	0.2885\\
1.3966	0.2864\\
1.4009	0.2841\\
1.4051	0.2816\\
1.4093	0.2789\\
1.4135	0.2761\\
1.4177	0.2731\\
1.4219	0.2699\\
1.4261	0.2666\\
1.4303	0.263\\
1.4345	0.2593\\
1.4387	0.2555\\
1.4429	0.2514\\
1.4471	0.2472\\
1.4514	0.2428\\
1.4556	0.2382\\
1.4598	0.2335\\
1.464	0.2286\\
1.4682	0.2235\\
1.4724	0.2182\\
1.4766	0.2128\\
1.4808	0.2072\\
1.485	0.2014\\
1.4892	0.1954\\
1.4934	0.1893\\
1.4979	0.1827\\
1.5023	0.1758\\
1.5067	0.1688\\
1.5111	0.1616\\
};
\addplot [color=black, forget plot]
  table[row sep=crcr]{%
1.5111	0.1616\\
1.5158	0.1537\\
1.5206	0.1456\\
1.5253	0.1372\\
1.53	0.1287\\
1.5347	0.1199\\
1.5394	0.1109\\
1.5442	0.1017\\
1.5489	0.0922\\
1.5536	0.0826\\
1.5583	0.0727\\
1.5631	0.0626\\
1.5678	0.0523\\
1.5725	0.0418\\
1.5772	0.031\\
1.5819	0.02\\
1.5867	0.0089\\
1.5876	0.0067\\
1.5885	0.0044\\
1.5903	-0\\
};
\addplot [color=black, forget plot]
  table[row sep=crcr]{%
1.5903	0\\
1.5904	0.0001\\
1.5904	0.0002\\
1.5906	0.0005\\
1.5907	0.0007\\
1.5909	0.001\\
1.591	0.0012\\
1.5917	0.0025\\
1.5931	0.0049\\
1.5938	0.0062\\
1.5992	0.0158\\
1.602	0.0205\\
1.6047	0.0252\\
1.6075	0.0298\\
1.6102	0.0343\\
1.6129	0.0387\\
1.6157	0.043\\
1.6184	0.0473\\
1.6212	0.0515\\
1.6239	0.0556\\
1.6267	0.0597\\
1.6294	0.0637\\
1.6321	0.0676\\
1.6349	0.0714\\
1.6376	0.0752\\
1.6404	0.0788\\
1.6431	0.0825\\
1.6458	0.086\\
1.6486	0.0895\\
1.6513	0.0928\\
1.6541	0.0962\\
1.6568	0.0994\\
1.6596	0.1026\\
1.6623	0.1057\\
1.665	0.1087\\
1.6678	0.1116\\
1.6705	0.1145\\
1.6787	0.1227\\
1.6815	0.1253\\
1.6842	0.1278\\
1.687	0.1302\\
1.6897	0.1326\\
1.6925	0.1349\\
1.6943	0.1364\\
1.6962	0.1379\\
1.6981	0.1393\\
1.7	0.1408\\
};
\addplot [color=black, forget plot]
  table[row sep=crcr]{%
0	1.05\\
0.0008	1.05\\
0.001	1.0499\\
0.0023	1.0499\\
0.0036	1.0498\\
0.0049	1.0496\\
0.0061	1.0495\\
0.0109	1.0489\\
0.0156	1.048\\
0.0203	1.047\\
0.025	1.0457\\
0.0298	1.0442\\
0.0345	1.0424\\
0.0392	1.0405\\
0.0439	1.0383\\
0.0486	1.036\\
0.0534	1.0334\\
0.0581	1.0305\\
0.0628	1.0275\\
0.0675	1.0243\\
0.0723	1.0208\\
0.077	1.0171\\
0.0817	1.0132\\
0.0864	1.009\\
0.0911	1.0047\\
0.0959	1.0001\\
0.1006	0.9953\\
0.1053	0.9903\\
0.11	0.9851\\
0.1148	0.9797\\
0.1195	0.974\\
0.1242	0.9681\\
0.1289	0.962\\
0.1336	0.9557\\
0.1384	0.9492\\
0.1431	0.9424\\
0.1478	0.9354\\
0.1525	0.9283\\
0.1573	0.9208\\
0.162	0.9132\\
0.1667	0.9054\\
0.1714	0.8973\\
0.1761	0.889\\
0.1793	0.8833\\
0.1825	0.8775\\
0.1857	0.8716\\
0.1889	0.8656\\
};
\addplot [color=black, forget plot]
  table[row sep=crcr]{%
0.1889	0.8656\\
0.1936	0.8565\\
0.1983	0.8471\\
0.2031	0.8376\\
0.2078	0.8279\\
0.2125	0.8179\\
0.2172	0.8077\\
0.2219	0.7973\\
0.2267	0.7867\\
0.2314	0.7758\\
0.2361	0.7647\\
0.2408	0.7535\\
0.2456	0.742\\
0.2503	0.7302\\
0.255	0.7183\\
0.2597	0.7061\\
0.2644	0.6938\\
0.2692	0.6812\\
0.2739	0.6684\\
0.2786	0.6553\\
0.2833	0.6421\\
0.2881	0.6286\\
0.2928	0.6149\\
0.2975	0.601\\
0.3022	0.5869\\
0.3069	0.5725\\
0.3117	0.558\\
0.3164	0.5432\\
0.3211	0.5282\\
0.3258	0.513\\
0.3306	0.4975\\
0.3353	0.4819\\
0.34	0.466\\
0.3447	0.4499\\
0.3494	0.4336\\
0.3542	0.417\\
0.3589	0.4003\\
0.3636	0.3833\\
0.3683	0.3661\\
0.3731	0.3487\\
0.3778	0.3311\\
};
\addplot [color=black, forget plot]
  table[row sep=crcr]{%
0.3778	0.3311\\
0.3822	0.3144\\
0.3866	0.2974\\
0.3911	0.2803\\
0.3955	0.263\\
0.4002	0.2444\\
0.4049	0.2255\\
0.4097	0.2064\\
0.4144	0.187\\
0.4191	0.1675\\
0.4238	0.1477\\
0.4285	0.1278\\
0.4333	0.1076\\
0.438	0.0871\\
0.4427	0.0665\\
0.4474	0.0457\\
0.4522	0.0246\\
0.4535	0.0185\\
0.4549	0.0123\\
0.4562	0.0062\\
0.4576	-0\\
};
\addplot [color=black, forget plot]
  table[row sep=crcr]{%
0.4576	0\\
0.4576	0.0002\\
0.4577	0.0002\\
0.4577	0.0005\\
0.4578	0.0007\\
0.4579	0.001\\
0.458	0.0012\\
0.4583	0.0025\\
0.4587	0.0037\\
0.4591	0.005\\
0.4594	0.0062\\
0.4613	0.0124\\
0.4631	0.0186\\
0.465	0.0248\\
0.4668	0.0309\\
0.4695	0.0399\\
0.4723	0.0489\\
0.4777	0.0665\\
0.4804	0.0752\\
0.4832	0.0838\\
0.4859	0.0924\\
0.4886	0.1009\\
0.4914	0.1093\\
0.4968	0.1259\\
0.4995	0.1341\\
0.5023	0.1422\\
0.505	0.1503\\
0.5077	0.1583\\
0.5104	0.1662\\
0.5132	0.174\\
0.5159	0.1818\\
0.5186	0.1894\\
0.5213	0.1971\\
0.5241	0.2046\\
0.5268	0.2121\\
0.5295	0.2195\\
0.5323	0.2268\\
0.5377	0.2412\\
0.5404	0.2483\\
0.5432	0.2553\\
0.5459	0.2623\\
0.5486	0.2692\\
0.5513	0.276\\
0.5541	0.2827\\
0.5568	0.2894\\
0.5595	0.296\\
0.5622	0.3025\\
0.565	0.309\\
0.5658	0.311\\
0.5662	0.3119\\
0.5667	0.3129\\
};
\addplot [color=black, forget plot]
  table[row sep=crcr]{%
0.5667	0.3129\\
0.5714	0.3239\\
0.5761	0.3345\\
0.5808	0.345\\
0.5856	0.3553\\
0.5903	0.3653\\
0.595	0.3751\\
0.5997	0.3847\\
0.6044	0.3941\\
0.6092	0.4033\\
0.6139	0.4122\\
0.6186	0.421\\
0.6233	0.4295\\
0.6281	0.4378\\
0.6328	0.4458\\
0.6375	0.4537\\
0.6422	0.4613\\
0.6469	0.4687\\
0.6517	0.4759\\
0.6564	0.4829\\
0.6611	0.4897\\
0.6658	0.4962\\
0.6706	0.5025\\
0.6753	0.5086\\
0.68	0.5145\\
0.6847	0.5202\\
0.6894	0.5256\\
0.6942	0.5308\\
0.6989	0.5359\\
0.7036	0.5406\\
0.7083	0.5452\\
0.7131	0.5496\\
0.7178	0.5537\\
0.7225	0.5576\\
0.7272	0.5613\\
0.7319	0.5648\\
0.7367	0.568\\
0.7414	0.5711\\
0.7461	0.5739\\
0.7508	0.5765\\
0.7556	0.5789\\
};
\addplot [color=black, forget plot]
  table[row sep=crcr]{%
0.7556	0.5789\\
0.758	0.58\\
0.763	0.5822\\
0.7654	0.5832\\
0.7701	0.5849\\
0.7749	0.5863\\
0.7796	0.5876\\
0.7843	0.5887\\
0.789	0.5895\\
0.7938	0.5901\\
0.7985	0.5905\\
0.8032	0.5907\\
0.8046	0.5907\\
};
\addplot [color=black, forget plot]
  table[row sep=crcr]{%
0.8046	0.5907\\
0.8072	0.5907\\
0.8078	0.5906\\
0.811	0.5905\\
0.8174	0.5899\\
0.8206	0.5894\\
0.8241	0.5888\\
0.8276	0.5881\\
0.8311	0.5873\\
0.8346	0.5863\\
0.8381	0.5852\\
0.8416	0.584\\
0.8451	0.5827\\
0.8521	0.5797\\
0.8556	0.578\\
0.8591	0.5762\\
0.8661	0.5722\\
0.8696	0.57\\
0.8731	0.5677\\
0.8766	0.5653\\
0.8801	0.5628\\
0.8835	0.5602\\
0.887	0.5574\\
0.8905	0.5545\\
0.894	0.5515\\
0.8975	0.5484\\
0.9045	0.5418\\
0.908	0.5383\\
0.9115	0.5347\\
0.9185	0.5271\\
0.9255	0.5191\\
0.9325	0.5105\\
0.9355	0.5067\\
0.9415	0.4989\\
0.9444	0.4948\\
};
\addplot [color=black, forget plot]
  table[row sep=crcr]{%
0.9444	0.4948\\
0.9492	0.4882\\
0.9539	0.4814\\
0.9586	0.4744\\
0.9633	0.4672\\
0.9681	0.4597\\
0.9728	0.452\\
0.9775	0.4441\\
0.9822	0.436\\
0.9869	0.4277\\
0.9917	0.4191\\
0.9964	0.4103\\
1.0011	0.4013\\
1.0058	0.3921\\
1.0106	0.3827\\
1.0153	0.3731\\
1.02	0.3632\\
1.0247	0.3531\\
1.0294	0.3428\\
1.0342	0.3323\\
1.0389	0.3215\\
1.0436	0.3106\\
1.0483	0.2994\\
1.0531	0.288\\
1.0578	0.2764\\
1.0625	0.2645\\
1.0672	0.2525\\
1.0719	0.2402\\
1.0767	0.2277\\
1.0814	0.215\\
1.0861	0.2021\\
1.0908	0.1889\\
1.0956	0.1756\\
1.1003	0.162\\
1.105	0.1482\\
1.1097	0.1341\\
1.1144	0.1199\\
1.1192	0.1054\\
1.1239	0.0908\\
1.1286	0.0759\\
1.1333	0.0607\\
};
\addplot [color=black, forget plot]
  table[row sep=crcr]{%
1.1333	0.0607\\
1.1343	0.0577\\
1.1352	0.0546\\
1.1362	0.0515\\
1.1371	0.0485\\
1.1408	0.0365\\
1.1444	0.0245\\
1.148	0.0123\\
1.1517	0\\
};
\addplot [color=black, forget plot]
  table[row sep=crcr]{%
1.1517	0\\
1.1517	0.0002\\
1.1518	0.0005\\
1.1519	0.0007\\
1.152	0.001\\
1.1521	0.0012\\
1.1526	0.0025\\
1.1531	0.0037\\
1.1536	0.005\\
1.1541	0.0062\\
1.1566	0.0124\\
1.159	0.0185\\
1.1615	0.0246\\
1.1639	0.0306\\
1.1682	0.0409\\
1.1725	0.051\\
1.1767	0.0609\\
1.181	0.0707\\
1.1853	0.0802\\
1.1895	0.0896\\
1.1938	0.0989\\
1.198	0.1079\\
1.2023	0.1167\\
1.2066	0.1254\\
1.2108	0.1339\\
1.2151	0.1423\\
1.2194	0.1504\\
1.2236	0.1584\\
1.2322	0.1738\\
1.2364	0.1812\\
1.2407	0.1884\\
1.245	0.1955\\
1.2492	0.2024\\
1.2535	0.2091\\
1.2577	0.2157\\
1.262	0.222\\
1.2663	0.2282\\
1.2705	0.2342\\
1.2748	0.24\\
1.2791	0.2457\\
1.2833	0.2512\\
1.2919	0.2616\\
1.2961	0.2665\\
1.3004	0.2712\\
1.3047	0.2758\\
1.3089	0.2802\\
1.3132	0.2844\\
1.3174	0.2885\\
1.3198	0.2907\\
1.321	0.2917\\
1.3222	0.2928\\
};
\addplot [color=black, forget plot]
  table[row sep=crcr]{%
1.3222	0.2928\\
1.3267	0.2967\\
1.3312	0.3003\\
1.3357	0.3038\\
1.3402	0.3071\\
1.345	0.3103\\
1.3497	0.3133\\
1.3544	0.316\\
1.3591	0.3186\\
1.3639	0.3209\\
1.3686	0.3231\\
1.3733	0.3249\\
1.378	0.3266\\
1.3827	0.3281\\
1.3875	0.3293\\
1.3922	0.3304\\
1.3969	0.3312\\
1.4007	0.3316\\
1.4044	0.332\\
1.4082	0.3322\\
1.4119	0.3323\\
};
\addplot [color=black, forget plot]
  table[row sep=crcr]{%
1.4119	0.3323\\
1.4132	0.3323\\
1.4138	0.3322\\
1.4151	0.3322\\
1.4176	0.3321\\
1.4226	0.3317\\
1.425	0.3314\\
1.4275	0.3311\\
1.43	0.3307\\
1.435	0.3297\\
1.4374	0.3291\\
1.4424	0.3277\\
1.4474	0.3261\\
1.4498	0.3252\\
1.4523	0.3243\\
1.4548	0.3233\\
1.4573	0.3222\\
1.4598	0.321\\
1.4622	0.3199\\
1.4672	0.3173\\
1.4722	0.3145\\
1.4746	0.313\\
1.4796	0.3098\\
1.4846	0.3064\\
1.487	0.3046\\
1.492	0.3008\\
1.4945	0.2988\\
1.4969	0.2968\\
1.5019	0.2926\\
1.5044	0.2903\\
1.5061	0.2888\\
1.5077	0.2872\\
1.5111	0.284\\
};
\addplot [color=black, forget plot]
  table[row sep=crcr]{%
1.5111	0.284\\
1.5158	0.2793\\
1.5206	0.2744\\
1.5253	0.2692\\
1.53	0.2639\\
1.5347	0.2583\\
1.5394	0.2525\\
1.5442	0.2465\\
1.5489	0.2403\\
1.5536	0.2338\\
1.5583	0.2271\\
1.5631	0.2202\\
1.5678	0.2131\\
1.5725	0.2058\\
1.5772	0.1983\\
1.5819	0.1905\\
1.5867	0.1825\\
1.5914	0.1743\\
1.5961	0.1659\\
1.6008	0.1572\\
1.6056	0.1484\\
1.6103	0.1393\\
1.615	0.13\\
1.6197	0.1205\\
1.6244	0.1107\\
1.6292	0.1008\\
1.6339	0.0906\\
1.6386	0.0802\\
1.6433	0.0696\\
1.6481	0.0588\\
1.6528	0.0477\\
1.6575	0.0365\\
1.6622	0.025\\
1.6672	0.0126\\
1.6722	-0\\
};
\addplot [color=black, forget plot]
  table[row sep=crcr]{%
1.6722	0\\
1.6722	0.0001\\
1.6723	0.0001\\
1.6723	0.0002\\
1.6724	0.0005\\
1.6726	0.0007\\
1.6727	0.001\\
1.6728	0.0012\\
1.6735	0.0025\\
1.6741	0.0037\\
1.6748	0.0049\\
1.6755	0.0062\\
1.6761	0.0075\\
1.6789	0.0127\\
1.6796	0.0139\\
1.6817	0.0178\\
1.6824	0.019\\
1.6831	0.0203\\
1.6838	0.0215\\
1.6845	0.0228\\
1.6852	0.024\\
1.6859	0.0253\\
1.6873	0.0277\\
1.688	0.029\\
1.6894	0.0314\\
1.69	0.0326\\
1.6949	0.041\\
1.6956	0.0421\\
1.697	0.0445\\
1.6977	0.0456\\
1.6983	0.0466\\
1.6988	0.0475\\
1.6994	0.0485\\
1.7	0.0494\\
};
\addplot [color=black, forget plot]
  table[row sep=crcr]{%
0	1.05\\
0.001	1.05\\
0.002	1.0501\\
0.0051	1.0501\\
};
\addplot [color=black, forget plot]
  table[row sep=crcr]{%
0.0051	1.0501\\
0.0083	1.0501\\
0.0147	1.0497\\
0.0211	1.0489\\
0.0257	1.048\\
0.0303	1.047\\
0.0349	1.0458\\
0.0395	1.0443\\
0.0441	1.0427\\
0.0487	1.0408\\
0.0533	1.0387\\
0.0579	1.0365\\
0.0624	1.034\\
0.067	1.0313\\
0.0716	1.0284\\
0.0762	1.0253\\
0.0808	1.022\\
0.0854	1.0185\\
0.09	1.0148\\
0.0946	1.0108\\
0.0992	1.0067\\
0.1038	1.0023\\
0.1084	0.9978\\
0.113	0.993\\
0.1176	0.9881\\
0.1222	0.9829\\
0.1268	0.9775\\
0.1314	0.9719\\
0.136	0.9661\\
0.1406	0.9601\\
0.1452	0.9539\\
0.1497	0.9475\\
0.1543	0.9409\\
0.1589	0.934\\
0.1635	0.927\\
0.1681	0.9198\\
0.1727	0.9123\\
0.1773	0.9046\\
0.1819	0.8968\\
0.1865	0.8887\\
0.1871	0.8876\\
0.1877	0.8866\\
0.1889	0.8844\\
};
\addplot [color=black, forget plot]
  table[row sep=crcr]{%
0.1889	0.8844\\
0.1936	0.8758\\
0.1983	0.867\\
0.2031	0.8579\\
0.2078	0.8486\\
0.2125	0.8391\\
0.2172	0.8294\\
0.2219	0.8195\\
0.2267	0.8093\\
0.2314	0.799\\
0.2361	0.7884\\
0.2408	0.7775\\
0.2456	0.7665\\
0.2503	0.7553\\
0.255	0.7438\\
0.2597	0.7321\\
0.2644	0.7202\\
0.2692	0.7081\\
0.2739	0.6957\\
0.2786	0.6832\\
0.2833	0.6704\\
0.2881	0.6574\\
0.2928	0.6442\\
0.2975	0.6308\\
0.3022	0.6171\\
0.3069	0.6032\\
0.3117	0.5891\\
0.3164	0.5748\\
0.3211	0.5603\\
0.3258	0.5455\\
0.3306	0.5306\\
0.3353	0.5154\\
0.34	0.5\\
0.3447	0.4844\\
0.3494	0.4685\\
0.3542	0.4525\\
0.3589	0.4362\\
0.3636	0.4197\\
0.3683	0.403\\
0.3731	0.386\\
0.3778	0.3689\\
};
\addplot [color=black, forget plot]
  table[row sep=crcr]{%
0.3778	0.3689\\
0.3825	0.3515\\
0.3872	0.3339\\
0.3919	0.3161\\
0.3967	0.2981\\
0.4014	0.2798\\
0.4061	0.2613\\
0.4108	0.2427\\
0.4156	0.2238\\
0.4203	0.2046\\
0.425	0.1853\\
0.4297	0.1657\\
0.4344	0.1459\\
0.4392	0.1259\\
0.4439	0.1057\\
0.4486	0.0853\\
0.4533	0.0646\\
0.4569	0.0487\\
0.4606	0.0326\\
0.4642	0.0164\\
0.4678	-0\\
};
\addplot [color=black, forget plot]
  table[row sep=crcr]{%
0.4678	0\\
0.4678	0.0002\\
0.4679	0.0002\\
0.4679	0.0005\\
0.468	0.0007\\
0.4681	0.001\\
0.4682	0.0012\\
0.4685	0.0025\\
0.4689	0.0037\\
0.4693	0.005\\
0.4696	0.0062\\
0.4715	0.0124\\
0.4733	0.0186\\
0.4752	0.0248\\
0.477	0.0309\\
0.4795	0.0391\\
0.482	0.0472\\
0.4844	0.0552\\
0.4869	0.0632\\
0.4894	0.0711\\
0.4918	0.079\\
0.4943	0.0868\\
0.4993	0.1022\\
0.5017	0.1098\\
0.5042	0.1174\\
0.5067	0.1249\\
0.5091	0.1324\\
0.5116	0.1397\\
0.5141	0.1471\\
0.5166	0.1543\\
0.519	0.1615\\
0.5215	0.1687\\
0.524	0.1757\\
0.5264	0.1828\\
0.5339	0.2035\\
0.5363	0.2103\\
0.5388	0.217\\
0.5413	0.2236\\
0.5437	0.2302\\
0.5462	0.2368\\
0.5487	0.2433\\
0.5512	0.2497\\
0.5536	0.2561\\
0.5561	0.2624\\
0.5587	0.269\\
0.5614	0.2756\\
0.564	0.2822\\
0.5667	0.2886\\
};
\addplot [color=black, forget plot]
  table[row sep=crcr]{%
0.5667	0.2886\\
0.5714	0.3\\
0.5761	0.3112\\
0.5808	0.3221\\
0.5856	0.3329\\
0.5903	0.3434\\
0.595	0.3537\\
0.5997	0.3637\\
0.6044	0.3736\\
0.6092	0.3832\\
0.6139	0.3927\\
0.6186	0.4019\\
0.6233	0.4108\\
0.6281	0.4196\\
0.6328	0.4281\\
0.6375	0.4365\\
0.6422	0.4446\\
0.6469	0.4525\\
0.6517	0.4601\\
0.6564	0.4676\\
0.6611	0.4748\\
0.6658	0.4818\\
0.6706	0.4886\\
0.6753	0.4952\\
0.68	0.5015\\
0.6847	0.5077\\
0.6894	0.5136\\
0.6942	0.5193\\
0.6989	0.5248\\
0.7036	0.53\\
0.7083	0.5351\\
0.7131	0.5399\\
0.7178	0.5445\\
0.7225	0.5489\\
0.7272	0.5531\\
0.7319	0.557\\
0.7367	0.5607\\
0.7414	0.5642\\
0.7461	0.5675\\
0.7508	0.5706\\
0.7556	0.5735\\
};
\addplot [color=black, forget plot]
  table[row sep=crcr]{%
0.7556	0.5735\\
0.7585	0.5752\\
0.7615	0.5768\\
0.7645	0.5783\\
0.7675	0.5797\\
0.7722	0.5818\\
0.7769	0.5836\\
0.7816	0.5853\\
0.7864	0.5867\\
0.7911	0.5879\\
0.7958	0.5889\\
0.8005	0.5897\\
0.8052	0.5902\\
0.81	0.5906\\
0.8124	0.5907\\
0.8148	0.5907\\
};
\addplot [color=black, forget plot]
  table[row sep=crcr]{%
0.8148	0.5907\\
0.8174	0.5907\\
0.818	0.5906\\
0.8212	0.5905\\
0.8276	0.5899\\
0.8308	0.5894\\
0.8341	0.5889\\
0.8405	0.5875\\
0.8438	0.5866\\
0.847	0.5856\\
0.8503	0.5845\\
0.8535	0.5834\\
0.8567	0.5821\\
0.86	0.5807\\
0.8632	0.5792\\
0.8665	0.5776\\
0.8697	0.5759\\
0.8729	0.5741\\
0.8762	0.5722\\
0.8794	0.5702\\
0.8827	0.5681\\
0.8859	0.5659\\
0.8892	0.5636\\
0.8924	0.5612\\
0.8956	0.5587\\
0.8989	0.556\\
0.9021	0.5533\\
0.9054	0.5505\\
0.9086	0.5476\\
0.9118	0.5445\\
0.9151	0.5414\\
0.9183	0.5382\\
0.9216	0.5348\\
0.9248	0.5314\\
0.928	0.5278\\
0.9313	0.5242\\
0.9345	0.5204\\
0.937	0.5175\\
0.9395	0.5145\\
0.942	0.5114\\
0.9444	0.5083\\
};
\addplot [color=black, forget plot]
  table[row sep=crcr]{%
0.9444	0.5083\\
0.9492	0.5022\\
0.9539	0.4958\\
0.9586	0.4893\\
0.9633	0.4825\\
0.9681	0.4755\\
0.9728	0.4683\\
0.9775	0.4609\\
0.9822	0.4533\\
0.9869	0.4454\\
0.9917	0.4373\\
0.9964	0.429\\
1.0011	0.4205\\
1.0058	0.4117\\
1.0106	0.4028\\
1.0153	0.3936\\
1.02	0.3842\\
1.0247	0.3746\\
1.0294	0.3648\\
1.0342	0.3547\\
1.0389	0.3444\\
1.0436	0.334\\
1.0483	0.3232\\
1.0531	0.3123\\
1.0578	0.3012\\
1.0625	0.2898\\
1.0672	0.2782\\
1.0719	0.2664\\
1.0767	0.2544\\
1.0814	0.2422\\
1.0861	0.2297\\
1.0908	0.217\\
1.0956	0.2041\\
1.1003	0.191\\
1.105	0.1777\\
1.1097	0.1641\\
1.1144	0.1504\\
1.1192	0.1364\\
1.1239	0.1222\\
1.1286	0.1077\\
1.1333	0.0931\\
};
\addplot [color=black, forget plot]
  table[row sep=crcr]{%
1.1333	0.0931\\
1.1378	0.079\\
1.1393	0.0742\\
1.144	0.0591\\
1.1488	0.0437\\
1.1535	0.0281\\
1.1582	0.0123\\
1.1591	0.0093\\
1.1609	0.0031\\
1.1619	-0\\
};
\addplot [color=black, forget plot]
  table[row sep=crcr]{%
1.1619	0\\
1.1619	0.0002\\
1.162	0.0005\\
1.1621	0.0007\\
1.1622	0.001\\
1.1623	0.0012\\
1.1628	0.0025\\
1.1633	0.0037\\
1.1638	0.005\\
1.1643	0.0062\\
1.1668	0.0124\\
1.1692	0.0185\\
1.1717	0.0246\\
1.1741	0.0306\\
1.1781	0.0403\\
1.1821	0.0498\\
1.1862	0.0592\\
1.1902	0.0684\\
1.1942	0.0774\\
1.1982	0.0863\\
1.2022	0.095\\
1.2062	0.1036\\
1.2102	0.112\\
1.2142	0.1203\\
1.2182	0.1284\\
1.2222	0.1363\\
1.2263	0.1441\\
1.2303	0.1517\\
1.2343	0.1592\\
1.2383	0.1665\\
1.2423	0.1736\\
1.2463	0.1806\\
1.2503	0.1875\\
1.2583	0.2007\\
1.2623	0.207\\
1.2663	0.2132\\
1.2704	0.2193\\
1.2744	0.2252\\
1.2784	0.2309\\
1.2824	0.2365\\
1.2864	0.2419\\
1.2904	0.2472\\
1.2944	0.2523\\
1.2984	0.2572\\
1.3024	0.262\\
1.3064	0.2666\\
1.3104	0.2711\\
1.3145	0.2754\\
1.3185	0.2796\\
1.3194	0.2805\\
1.3203	0.2815\\
1.3213	0.2824\\
1.3222	0.2833\\
};
\addplot [color=black, forget plot]
  table[row sep=crcr]{%
1.3222	0.2833\\
1.3269	0.2878\\
1.3317	0.2921\\
1.3364	0.2962\\
1.3411	0.3001\\
1.3458	0.3037\\
1.3506	0.3071\\
1.3553	0.3104\\
1.36	0.3133\\
1.3647	0.3161\\
1.3694	0.3187\\
1.3742	0.321\\
1.3789	0.3231\\
1.3836	0.325\\
1.3883	0.3267\\
1.3931	0.3281\\
1.3978	0.3294\\
1.4025	0.3304\\
1.4072	0.3312\\
1.4119	0.3318\\
1.4167	0.3321\\
1.418	0.3322\\
1.4194	0.3322\\
1.4208	0.3323\\
1.4221	0.3323\\
};
\addplot [color=black, forget plot]
  table[row sep=crcr]{%
1.4221	0.3323\\
1.4234	0.3323\\
1.424	0.3322\\
1.4253	0.3322\\
1.4275	0.3321\\
1.4298	0.332\\
1.432	0.3318\\
1.4342	0.3315\\
1.4364	0.3313\\
1.4387	0.3309\\
1.4431	0.3301\\
1.4453	0.3296\\
1.4476	0.3291\\
1.452	0.3279\\
1.4542	0.3272\\
1.4565	0.3265\\
1.4609	0.3249\\
1.4631	0.324\\
1.4654	0.3231\\
1.4698	0.3211\\
1.472	0.32\\
1.4743	0.3189\\
1.4765	0.3178\\
1.4787	0.3166\\
1.4809	0.3153\\
1.4832	0.314\\
1.4898	0.3098\\
1.4921	0.3083\\
1.4987	0.3035\\
1.501	0.3018\\
1.5054	0.2982\\
1.5068	0.2971\\
1.5083	0.2959\\
1.5097	0.2947\\
1.5111	0.2934\\
};
\addplot [color=black, forget plot]
  table[row sep=crcr]{%
1.5111	0.2934\\
1.5156	0.2894\\
1.5201	0.2852\\
1.5245	0.2808\\
1.529	0.2762\\
1.5337	0.2712\\
1.5384	0.2659\\
1.5432	0.2604\\
1.5479	0.2547\\
1.5526	0.2488\\
1.5573	0.2426\\
1.562	0.2362\\
1.5668	0.2296\\
1.5715	0.2228\\
1.5762	0.2158\\
1.5809	0.2086\\
1.5857	0.2011\\
1.5904	0.1934\\
1.5951	0.1855\\
1.5998	0.1774\\
1.6045	0.169\\
1.6093	0.1605\\
1.614	0.1517\\
1.6187	0.1427\\
1.6234	0.1335\\
1.6282	0.124\\
1.6329	0.1144\\
1.6376	0.1045\\
1.6423	0.0944\\
1.647	0.0841\\
1.6518	0.0736\\
1.6565	0.0628\\
1.6612	0.0519\\
1.6659	0.0407\\
1.6707	0.0293\\
1.6754	0.0177\\
1.6801	0.0058\\
1.6807	0.0044\\
1.6812	0.0029\\
1.6818	0.0015\\
1.6824	-0\\
};
\addplot [color=black, forget plot]
  table[row sep=crcr]{%
1.6824	0\\
1.6824	0.0001\\
1.6825	0.0002\\
1.6826	0.0005\\
1.6828	0.0007\\
1.6829	0.001\\
1.683	0.0012\\
1.6835	0.002\\
1.6839	0.0029\\
1.6843	0.0037\\
1.6848	0.0045\\
1.6852	0.0054\\
1.6857	0.0062\\
1.6861	0.007\\
1.6865	0.0079\\
1.687	0.0087\\
1.6874	0.0095\\
1.6879	0.0103\\
1.6883	0.0111\\
1.6887	0.012\\
1.6892	0.0128\\
1.6896	0.0136\\
1.6901	0.0144\\
1.6909	0.016\\
1.6914	0.0168\\
1.6918	0.0176\\
1.6923	0.0184\\
1.6931	0.02\\
1.6936	0.0208\\
1.694	0.0216\\
1.6945	0.0224\\
1.6953	0.024\\
1.6958	0.0248\\
1.6962	0.0256\\
1.6967	0.0263\\
1.6975	0.0279\\
1.698	0.0287\\
1.6984	0.0294\\
1.6989	0.0302\\
1.6992	0.0307\\
1.6994	0.0312\\
1.7	0.0322\\
};
\addplot [color=black, forget plot]
  table[row sep=crcr]{%
0	0.95\\
0.001	0.95\\
0.002	0.9501\\
0.0051	0.9501\\
};
\addplot [color=black, forget plot]
  table[row sep=crcr]{%
0.0051	0.9501\\
0.0083	0.9501\\
0.0147	0.9497\\
0.0211	0.9489\\
0.0257	0.948\\
0.0303	0.947\\
0.0349	0.9458\\
0.0395	0.9443\\
0.0441	0.9427\\
0.0487	0.9408\\
0.0533	0.9387\\
0.0579	0.9365\\
0.0624	0.934\\
0.067	0.9313\\
0.0716	0.9284\\
0.0762	0.9253\\
0.0808	0.922\\
0.0854	0.9185\\
0.09	0.9148\\
0.0946	0.9108\\
0.0992	0.9067\\
0.1038	0.9023\\
0.1084	0.8978\\
0.113	0.893\\
0.1176	0.8881\\
0.1222	0.8829\\
0.1268	0.8775\\
0.1314	0.8719\\
0.136	0.8661\\
0.1406	0.8601\\
0.1452	0.8539\\
0.1497	0.8475\\
0.1543	0.8409\\
0.1589	0.834\\
0.1635	0.827\\
0.1681	0.8198\\
0.1727	0.8123\\
0.1773	0.8046\\
0.1819	0.7968\\
0.1865	0.7887\\
0.1871	0.7876\\
0.1877	0.7866\\
0.1889	0.7844\\
};
\addplot [color=black, forget plot]
  table[row sep=crcr]{%
0.1889	0.7844\\
0.1936	0.7758\\
0.1983	0.767\\
0.2031	0.7579\\
0.2078	0.7486\\
0.2125	0.7391\\
0.2172	0.7294\\
0.2219	0.7195\\
0.2267	0.7093\\
0.2314	0.699\\
0.2361	0.6884\\
0.2408	0.6775\\
0.2456	0.6665\\
0.2503	0.6553\\
0.255	0.6438\\
0.2597	0.6321\\
0.2644	0.6202\\
0.2692	0.6081\\
0.2739	0.5957\\
0.2786	0.5832\\
0.2833	0.5704\\
0.2881	0.5574\\
0.2928	0.5442\\
0.2975	0.5308\\
0.3022	0.5171\\
0.3069	0.5032\\
0.3117	0.4891\\
0.3164	0.4748\\
0.3211	0.4603\\
0.3258	0.4455\\
0.3306	0.4306\\
0.3353	0.4154\\
0.34	0.4\\
0.3447	0.3844\\
0.3494	0.3685\\
0.3542	0.3525\\
0.3589	0.3362\\
0.3636	0.3197\\
0.3683	0.303\\
0.3731	0.286\\
0.3778	0.2689\\
};
\addplot [color=black, forget plot]
  table[row sep=crcr]{%
0.3778	0.2689\\
0.3815	0.2553\\
0.3852	0.2416\\
0.3926	0.2138\\
0.3973	0.1957\\
0.402	0.1774\\
0.4067	0.1589\\
0.4114	0.1402\\
0.4162	0.1213\\
0.4209	0.1021\\
0.4256	0.0828\\
0.4303	0.0632\\
0.4341	0.0476\\
0.4378	0.0319\\
0.4415	0.016\\
0.4452	0\\
};
\addplot [color=black, forget plot]
  table[row sep=crcr]{%
0.4452	0\\
0.4452	0.0001\\
0.4453	0.0002\\
0.4454	0.0005\\
0.4454	0.0007\\
0.4455	0.001\\
0.4456	0.0012\\
0.446	0.0025\\
0.4464	0.0037\\
0.4468	0.005\\
0.4471	0.0062\\
0.4491	0.0124\\
0.451	0.0186\\
0.453	0.0248\\
0.4549	0.0309\\
0.4579	0.0404\\
0.461	0.0498\\
0.464	0.0591\\
0.467	0.0683\\
0.4701	0.0775\\
0.4731	0.0865\\
0.4762	0.0955\\
0.4822	0.1131\\
0.4853	0.1218\\
0.4883	0.1304\\
0.4913	0.1389\\
0.4944	0.1473\\
0.5004	0.1639\\
0.5035	0.172\\
0.5065	0.1801\\
0.5096	0.188\\
0.5126	0.1959\\
0.5156	0.2037\\
0.5187	0.2114\\
0.5217	0.219\\
0.5247	0.2265\\
0.5278	0.2339\\
0.5338	0.2485\\
0.5369	0.2556\\
0.5399	0.2627\\
0.5429	0.2696\\
0.546	0.2765\\
0.549	0.2833\\
0.5521	0.29\\
0.5551	0.2966\\
0.5581	0.3031\\
0.5612	0.3095\\
0.5642	0.3159\\
0.5648	0.3171\\
0.5654	0.3184\\
0.5661	0.3197\\
0.5667	0.3209\\
};
\addplot [color=black, forget plot]
  table[row sep=crcr]{%
0.5667	0.3209\\
0.5714	0.3305\\
0.5761	0.3398\\
0.5808	0.3489\\
0.5856	0.3578\\
0.5903	0.3665\\
0.595	0.375\\
0.5997	0.3832\\
0.6044	0.3912\\
0.6092	0.3991\\
0.6139	0.4066\\
0.6186	0.414\\
0.6233	0.4212\\
0.6281	0.4281\\
0.6328	0.4348\\
0.6375	0.4413\\
0.6422	0.4476\\
0.6469	0.4536\\
0.6517	0.4595\\
0.6564	0.4651\\
0.6611	0.4705\\
0.6658	0.4757\\
0.6706	0.4806\\
0.6753	0.4854\\
0.68	0.4899\\
0.6847	0.4942\\
0.6894	0.4983\\
0.6942	0.5022\\
0.6989	0.5058\\
0.7036	0.5092\\
0.7083	0.5124\\
0.7131	0.5154\\
0.7178	0.5182\\
0.7225	0.5208\\
0.7272	0.5231\\
0.7319	0.5252\\
0.7367	0.5271\\
0.7414	0.5288\\
0.7461	0.5303\\
0.7508	0.5315\\
0.7556	0.5325\\
};
\addplot [color=black, forget plot]
  table[row sep=crcr]{%
0.7556	0.5325\\
0.7565	0.5327\\
0.7585	0.5331\\
0.7595	0.5332\\
0.7635	0.5338\\
0.7674	0.5341\\
0.7714	0.5344\\
0.7753	0.5344\\
};
\addplot [color=black, forget plot]
  table[row sep=crcr]{%
0.7753	0.5344\\
0.7785	0.5344\\
0.7849	0.534\\
0.7913	0.5332\\
0.7955	0.5324\\
0.7998	0.5315\\
0.804	0.5304\\
0.8082	0.5291\\
0.8124	0.5277\\
0.8167	0.5261\\
0.8251	0.5223\\
0.8294	0.5201\\
0.8336	0.5178\\
0.8378	0.5153\\
0.842	0.5126\\
0.8463	0.5097\\
0.8505	0.5067\\
0.8547	0.5035\\
0.859	0.5001\\
0.8632	0.4966\\
0.8674	0.4928\\
0.8716	0.4889\\
0.8759	0.4848\\
0.8801	0.4806\\
0.8843	0.4761\\
0.8886	0.4715\\
0.8928	0.4667\\
0.897	0.4618\\
0.9012	0.4567\\
0.9055	0.4513\\
0.9097	0.4459\\
0.9139	0.4402\\
0.9182	0.4344\\
0.9266	0.4222\\
0.9311	0.4154\\
0.9355	0.4085\\
0.94	0.4014\\
0.9444	0.3941\\
};
\addplot [color=black, forget plot]
  table[row sep=crcr]{%
0.9444	0.3941\\
0.9492	0.3862\\
0.9539	0.378\\
0.9586	0.3696\\
0.9633	0.361\\
0.9681	0.3522\\
0.9728	0.3432\\
0.9775	0.3339\\
0.9822	0.3244\\
0.9869	0.3148\\
0.9917	0.3048\\
0.9964	0.2947\\
1.0011	0.2844\\
1.0058	0.2738\\
1.0106	0.263\\
1.0153	0.252\\
1.02	0.2408\\
1.0247	0.2293\\
1.0294	0.2177\\
1.0342	0.2058\\
1.0389	0.1937\\
1.0436	0.1814\\
1.0483	0.1688\\
1.0531	0.1561\\
1.0578	0.1431\\
1.0625	0.1299\\
1.0672	0.1165\\
1.0719	0.1028\\
1.0767	0.089\\
1.0814	0.0749\\
1.0861	0.0606\\
1.0908	0.0461\\
1.0956	0.0314\\
1.098	0.0236\\
1.1005	0.0158\\
1.1029	0.0079\\
1.1054	-0\\
};
\addplot [color=black, forget plot]
  table[row sep=crcr]{%
1.1054	0\\
1.1054	0.0001\\
1.1055	0.0002\\
1.1056	0.0005\\
1.1057	0.0007\\
1.1058	0.001\\
1.1059	0.0012\\
1.1064	0.0025\\
1.1069	0.0037\\
1.1074	0.005\\
1.108	0.0062\\
1.1087	0.0079\\
1.1094	0.0095\\
1.1108	0.0129\\
1.1115	0.0145\\
1.1122	0.0162\\
1.1129	0.0178\\
1.1135	0.0195\\
1.1149	0.0227\\
1.1156	0.0244\\
1.1233	0.042\\
1.124	0.0435\\
1.1254	0.0467\\
1.1261	0.0482\\
1.1268	0.0498\\
1.1275	0.0513\\
1.1282	0.0529\\
1.1296	0.0559\\
1.1303	0.0575\\
1.1311	0.0591\\
1.1318	0.0608\\
1.1326	0.0624\\
1.1333	0.064\\
};
\addplot [color=black, forget plot]
  table[row sep=crcr]{%
1.1333	0.064\\
1.1378	0.0736\\
1.1393	0.0767\\
1.144	0.0865\\
1.1487	0.0961\\
1.1535	0.1054\\
1.1582	0.1145\\
1.1629	0.1235\\
1.1676	0.1322\\
1.1724	0.1406\\
1.1771	0.1489\\
1.1818	0.1569\\
1.1865	0.1647\\
1.1912	0.1723\\
1.196	0.1797\\
1.2007	0.1869\\
1.2054	0.1938\\
1.2101	0.2006\\
1.2149	0.2071\\
1.2196	0.2134\\
1.2243	0.2194\\
1.229	0.2253\\
1.2337	0.2309\\
1.2385	0.2363\\
1.2432	0.2415\\
1.2479	0.2465\\
1.2526	0.2513\\
1.2574	0.2558\\
1.2621	0.2601\\
1.2668	0.2642\\
1.2715	0.2681\\
1.2762	0.2718\\
1.281	0.2752\\
1.2857	0.2784\\
1.2904	0.2814\\
1.2951	0.2842\\
1.2999	0.2868\\
1.3046	0.2891\\
1.3093	0.2913\\
1.3125	0.2926\\
1.3158	0.2938\\
1.319	0.295\\
1.3222	0.296\\
};
\addplot [color=black, forget plot]
  table[row sep=crcr]{%
1.3222	0.296\\
1.3238	0.2964\\
1.3253	0.2969\\
1.3269	0.2973\\
1.3284	0.2977\\
1.3331	0.2987\\
1.3378	0.2995\\
1.3426	0.3001\\
1.3473	0.3005\\
1.3487	0.3005\\
1.3501	0.3006\\
1.353	0.3006\\
};
\addplot [color=black, forget plot]
  table[row sep=crcr]{%
1.353	0.3006\\
1.3562	0.3006\\
1.3626	0.3002\\
1.369	0.2994\\
1.3729	0.2987\\
1.3769	0.2978\\
1.3808	0.2968\\
1.3848	0.2957\\
1.3887	0.2944\\
1.3927	0.2929\\
1.3966	0.2913\\
1.4006	0.2895\\
1.4045	0.2876\\
1.4085	0.2855\\
1.4125	0.2833\\
1.4164	0.2809\\
1.4204	0.2783\\
1.4243	0.2757\\
1.4283	0.2728\\
1.4322	0.2698\\
1.4362	0.2667\\
1.4401	0.2634\\
1.4441	0.2599\\
1.448	0.2563\\
1.452	0.2525\\
1.4559	0.2486\\
1.4599	0.2445\\
1.4639	0.2403\\
1.4678	0.2359\\
1.4718	0.2314\\
1.4757	0.2267\\
1.4797	0.2219\\
1.4836	0.2169\\
1.4876	0.2118\\
1.4915	0.2065\\
1.4955	0.201\\
1.4994	0.1955\\
1.5033	0.1898\\
1.5111	0.178\\
};
\addplot [color=black, forget plot]
  table[row sep=crcr]{%
1.5111	0.178\\
1.5158	0.1705\\
1.5206	0.1629\\
1.5253	0.155\\
1.53	0.1469\\
1.5347	0.1386\\
1.5394	0.1301\\
1.5442	0.1213\\
1.5489	0.1123\\
1.5536	0.1032\\
1.5583	0.0938\\
1.5631	0.0841\\
1.5678	0.0743\\
1.5725	0.0642\\
1.5772	0.0539\\
1.5819	0.0435\\
1.5867	0.0327\\
1.5901	0.0247\\
1.5936	0.0166\\
1.5971	0.0084\\
1.6005	-0\\
};
\addplot [color=black, forget plot]
  table[row sep=crcr]{%
1.6005	0\\
1.6006	0.0001\\
1.6006	0.0002\\
1.6008	0.0005\\
1.6009	0.0007\\
1.6011	0.001\\
1.6012	0.0012\\
1.6019	0.0025\\
1.6033	0.0049\\
1.604	0.0062\\
1.6064	0.0106\\
1.6139	0.0235\\
1.6164	0.0276\\
1.6189	0.0318\\
1.6214	0.0358\\
1.6238	0.0398\\
1.6288	0.0476\\
1.6338	0.0552\\
1.6363	0.0588\\
1.6388	0.0625\\
1.6413	0.066\\
1.6437	0.0695\\
1.6462	0.073\\
1.6487	0.0764\\
1.6537	0.083\\
1.6562	0.0862\\
1.6587	0.0893\\
1.6611	0.0924\\
1.6661	0.0984\\
1.6686	0.1013\\
1.6736	0.1069\\
1.6786	0.1123\\
1.681	0.1149\\
1.686	0.1199\\
1.6885	0.1223\\
1.6935	0.1269\\
1.6951	0.1284\\
1.6967	0.1298\\
1.6984	0.1313\\
1.7	0.1326\\
};
\addplot [color=black, forget plot]
  table[row sep=crcr]{%
0	0.95\\
0.0032	0.95\\
};
\addplot [color=black, forget plot]
  table[row sep=crcr]{%
0.0032	0.95\\
0.0064	0.95\\
0.0128	0.9496\\
0.0192	0.9488\\
0.0238	0.948\\
0.0285	0.9469\\
0.0331	0.9457\\
0.0377	0.9442\\
0.0424	0.9425\\
0.047	0.9406\\
0.0517	0.9385\\
0.0563	0.9362\\
0.061	0.9337\\
0.0656	0.9309\\
0.0702	0.928\\
0.0749	0.9248\\
0.0795	0.9215\\
0.0842	0.9179\\
0.0888	0.9141\\
0.0935	0.9101\\
0.0981	0.9058\\
0.1027	0.9014\\
0.1074	0.8968\\
0.112	0.8919\\
0.1167	0.8869\\
0.1213	0.8816\\
0.126	0.8761\\
0.1306	0.8704\\
0.1352	0.8645\\
0.1399	0.8584\\
0.1445	0.852\\
0.1492	0.8455\\
0.1538	0.8387\\
0.1585	0.8318\\
0.1631	0.8246\\
0.1677	0.8172\\
0.1724	0.8096\\
0.177	0.8018\\
0.1817	0.7938\\
0.1863	0.7855\\
0.187	0.7844\\
0.1882	0.782\\
0.1889	0.7809\\
};
\addplot [color=black, forget plot]
  table[row sep=crcr]{%
0.1889	0.7809\\
0.1936	0.7722\\
0.1983	0.7632\\
0.2031	0.7541\\
0.2078	0.7447\\
0.2125	0.7351\\
0.2172	0.7253\\
0.2219	0.7153\\
0.2267	0.705\\
0.2314	0.6946\\
0.2361	0.6839\\
0.2408	0.673\\
0.2456	0.6619\\
0.2503	0.6505\\
0.255	0.639\\
0.2597	0.6272\\
0.2644	0.6152\\
0.2692	0.603\\
0.2739	0.5906\\
0.2786	0.5779\\
0.2833	0.5651\\
0.2881	0.552\\
0.2928	0.5387\\
0.2975	0.5251\\
0.3022	0.5114\\
0.3069	0.4974\\
0.3117	0.4832\\
0.3164	0.4688\\
0.3211	0.4542\\
0.3258	0.4394\\
0.3306	0.4243\\
0.3353	0.4091\\
0.34	0.3936\\
0.3447	0.3779\\
0.3494	0.3619\\
0.3542	0.3458\\
0.3589	0.3294\\
0.3636	0.3128\\
0.3683	0.296\\
0.3731	0.279\\
0.3778	0.2617\\
};
\addplot [color=black, forget plot]
  table[row sep=crcr]{%
0.3778	0.2617\\
0.3814	0.2485\\
0.3849	0.2352\\
0.3885	0.2217\\
0.3921	0.2081\\
0.3968	0.19\\
0.4015	0.1717\\
0.4063	0.1531\\
0.411	0.1343\\
0.4157	0.1153\\
0.4204	0.0961\\
0.4251	0.0767\\
0.4299	0.057\\
0.4332	0.0429\\
0.4366	0.0287\\
0.4399	0.0144\\
0.4433	0\\
};
\addplot [color=black, forget plot]
  table[row sep=crcr]{%
0.4433	0\\
0.4433	0.0002\\
0.4434	0.0005\\
0.4435	0.0007\\
0.4436	0.001\\
0.4436	0.0012\\
0.444	0.0025\\
0.4444	0.0037\\
0.4448	0.005\\
0.4452	0.0062\\
0.4471	0.0124\\
0.4491	0.0186\\
0.451	0.0248\\
0.453	0.0309\\
0.456	0.0405\\
0.4591	0.0501\\
0.4622	0.0596\\
0.4684	0.0782\\
0.4715	0.0874\\
0.4745	0.0965\\
0.4776	0.1055\\
0.4807	0.1144\\
0.4838	0.1232\\
0.4869	0.1319\\
0.49	0.1405\\
0.4931	0.149\\
0.4961	0.1575\\
0.5023	0.1741\\
0.5085	0.1903\\
0.5116	0.1983\\
0.5147	0.2061\\
0.5177	0.2139\\
0.5208	0.2216\\
0.5239	0.2292\\
0.527	0.2367\\
0.5301	0.2441\\
0.5332	0.2514\\
0.5362	0.2586\\
0.5393	0.2658\\
0.5455	0.2798\\
0.5517	0.2934\\
0.5548	0.3\\
0.5578	0.3066\\
0.5609	0.3131\\
0.564	0.3194\\
0.5647	0.3208\\
0.5653	0.3222\\
0.566	0.3235\\
0.5667	0.3249\\
};
\addplot [color=black, forget plot]
  table[row sep=crcr]{%
0.5667	0.3249\\
0.5714	0.3343\\
0.5761	0.3436\\
0.5808	0.3526\\
0.5856	0.3614\\
0.5903	0.37\\
0.595	0.3784\\
0.5997	0.3865\\
0.6044	0.3945\\
0.6092	0.4022\\
0.6139	0.4097\\
0.6186	0.417\\
0.6233	0.424\\
0.6281	0.4309\\
0.6328	0.4375\\
0.6375	0.4439\\
0.6422	0.4501\\
0.6469	0.456\\
0.6517	0.4618\\
0.6564	0.4673\\
0.6611	0.4726\\
0.6658	0.4777\\
0.6706	0.4826\\
0.6753	0.4872\\
0.68	0.4917\\
0.6847	0.4959\\
0.6894	0.4999\\
0.6942	0.5036\\
0.6989	0.5072\\
0.7036	0.5105\\
0.7083	0.5137\\
0.7131	0.5166\\
0.7178	0.5193\\
0.7225	0.5217\\
0.7272	0.524\\
0.7319	0.526\\
0.7367	0.5278\\
0.7414	0.5294\\
0.7461	0.5308\\
0.7508	0.5319\\
0.7556	0.5328\\
};
\addplot [color=black, forget plot]
  table[row sep=crcr]{%
0.7556	0.5328\\
0.7564	0.533\\
0.7573	0.5331\\
0.7582	0.5333\\
0.7627	0.5338\\
0.7662	0.5342\\
0.7734	0.5344\\
};
\addplot [color=black, forget plot]
  table[row sep=crcr]{%
0.7734	0.5344\\
0.7765	0.5344\\
0.7829	0.534\\
0.7861	0.5336\\
0.7894	0.5331\\
0.7936	0.5324\\
0.7979	0.5314\\
0.8022	0.5303\\
0.8065	0.529\\
0.8107	0.5275\\
0.815	0.5259\\
0.8193	0.5241\\
0.8236	0.522\\
0.8278	0.5198\\
0.8321	0.5175\\
0.8364	0.5149\\
0.8407	0.5122\\
0.845	0.5093\\
0.8492	0.5062\\
0.8535	0.5029\\
0.8578	0.4994\\
0.8621	0.4958\\
0.8663	0.492\\
0.8706	0.488\\
0.8749	0.4838\\
0.8792	0.4795\\
0.8835	0.4749\\
0.8877	0.4702\\
0.892	0.4653\\
0.8963	0.4603\\
0.9006	0.455\\
0.9048	0.4496\\
0.9091	0.444\\
0.9134	0.4382\\
0.9177	0.4322\\
0.9219	0.4261\\
0.9262	0.4198\\
0.9308	0.4128\\
0.9353	0.4057\\
0.9399	0.3984\\
0.9444	0.3908\\
};
\addplot [color=black, forget plot]
  table[row sep=crcr]{%
0.9444	0.3908\\
0.9492	0.3828\\
0.9539	0.3745\\
0.9586	0.3661\\
0.9633	0.3574\\
0.9681	0.3485\\
0.9728	0.3393\\
0.9775	0.33\\
0.9822	0.3204\\
0.9869	0.3106\\
0.9917	0.3006\\
0.9964	0.2904\\
1.0011	0.28\\
1.0058	0.2693\\
1.0106	0.2584\\
1.0153	0.2473\\
1.02	0.236\\
1.0247	0.2245\\
1.0294	0.2127\\
1.0342	0.2007\\
1.0389	0.1886\\
1.0436	0.1761\\
1.0483	0.1635\\
1.0531	0.1507\\
1.0578	0.1376\\
1.0625	0.1243\\
1.0672	0.1108\\
1.0719	0.0971\\
1.0767	0.0831\\
1.0814	0.069\\
1.0861	0.0546\\
1.0908	0.04\\
1.0956	0.0252\\
1.0975	0.0189\\
1.0995	0.0127\\
1.1015	0.0064\\
1.1034	-0\\
};
\addplot [color=black, forget plot]
  table[row sep=crcr]{%
1.1034	0\\
1.1034	0.0001\\
1.1035	0.0001\\
1.1035	0.0002\\
1.1036	0.0005\\
1.1037	0.0007\\
1.1038	0.001\\
1.1039	0.0012\\
1.1044	0.0025\\
1.105	0.0037\\
1.1055	0.005\\
1.106	0.0062\\
1.1067	0.008\\
1.1075	0.0098\\
1.1082	0.0116\\
1.109	0.0133\\
1.1097	0.0151\\
1.1105	0.0169\\
1.1112	0.0186\\
1.112	0.0204\\
1.1127	0.0221\\
1.1135	0.0239\\
1.1142	0.0256\\
1.115	0.0274\\
1.1157	0.0291\\
1.1165	0.0308\\
1.1172	0.0325\\
1.118	0.0342\\
1.1187	0.036\\
1.1195	0.0377\\
1.1202	0.0393\\
1.1209	0.041\\
1.1217	0.0427\\
1.1224	0.0444\\
1.1232	0.0461\\
1.1239	0.0477\\
1.1247	0.0494\\
1.1254	0.0511\\
1.1262	0.0527\\
1.1269	0.0544\\
1.1277	0.056\\
1.1284	0.0576\\
1.1292	0.0593\\
1.1299	0.0609\\
1.1307	0.0625\\
1.1314	0.0641\\
1.1322	0.0657\\
1.1329	0.0673\\
1.133	0.0676\\
1.1333	0.0682\\
};
\addplot [color=black, forget plot]
  table[row sep=crcr]{%
1.1333	0.0682\\
1.1349	0.0717\\
1.1365	0.075\\
1.1382	0.0784\\
1.1398	0.0817\\
1.1445	0.0914\\
1.1492	0.1009\\
1.1539	0.1101\\
1.1586	0.1191\\
1.1634	0.1279\\
1.1681	0.1365\\
1.1728	0.1449\\
1.1775	0.153\\
1.1823	0.161\\
1.187	0.1687\\
1.1917	0.1762\\
1.1964	0.1834\\
1.2011	0.1905\\
1.2059	0.1973\\
1.2106	0.2039\\
1.2153	0.2103\\
1.22	0.2165\\
1.2248	0.2224\\
1.2295	0.2282\\
1.2342	0.2337\\
1.2389	0.239\\
1.2436	0.2441\\
1.2531	0.2536\\
1.2578	0.258\\
1.2625	0.2622\\
1.2673	0.2662\\
1.272	0.27\\
1.2767	0.2735\\
1.2814	0.2769\\
1.2861	0.28\\
1.2909	0.2829\\
1.2956	0.2855\\
1.3003	0.288\\
1.305	0.2902\\
1.3098	0.2923\\
1.3129	0.2935\\
1.316	0.2946\\
1.3191	0.2956\\
1.3222	0.2965\\
};
\addplot [color=black, forget plot]
  table[row sep=crcr]{%
1.3222	0.2965\\
1.3237	0.2969\\
1.3251	0.2973\\
1.3266	0.2977\\
1.328	0.298\\
1.3327	0.299\\
1.3374	0.2997\\
1.3422	0.3002\\
1.3469	0.3005\\
1.3479	0.3006\\
1.351	0.3006\\
};
\addplot [color=black, forget plot]
  table[row sep=crcr]{%
1.351	0.3006\\
1.3542	0.3006\\
1.3574	0.3004\\
1.3638	0.2998\\
1.367	0.2993\\
1.371	0.2986\\
1.375	0.2978\\
1.379	0.2968\\
1.383	0.2956\\
1.387	0.2942\\
1.391	0.2927\\
1.395	0.2911\\
1.399	0.2893\\
1.403	0.2873\\
1.407	0.2852\\
1.411	0.2829\\
1.415	0.2805\\
1.419	0.2779\\
1.423	0.2751\\
1.427	0.2722\\
1.431	0.2692\\
1.439	0.2626\\
1.443	0.259\\
1.447	0.2553\\
1.451	0.2515\\
1.4551	0.2475\\
1.4591	0.2433\\
1.4631	0.239\\
1.4671	0.2345\\
1.4711	0.2299\\
1.4751	0.2251\\
1.4791	0.2201\\
1.4831	0.215\\
1.4871	0.2098\\
1.4951	0.1988\\
1.4991	0.193\\
1.5031	0.1871\\
1.5071	0.1811\\
1.5111	0.1748\\
};
\addplot [color=black, forget plot]
  table[row sep=crcr]{%
1.5111	0.1748\\
1.5158	0.1673\\
1.5206	0.1596\\
1.5253	0.1516\\
1.53	0.1434\\
1.5347	0.135\\
1.5394	0.1264\\
1.5442	0.1176\\
1.5489	0.1085\\
1.5536	0.0992\\
1.5583	0.0897\\
1.5631	0.08\\
1.5678	0.0701\\
1.5725	0.0599\\
1.5772	0.0495\\
1.5819	0.039\\
1.5867	0.0282\\
1.5896	0.0212\\
1.5956	0.0072\\
1.5985	-0\\
};
\addplot [color=black, forget plot]
  table[row sep=crcr]{%
1.5985	0\\
1.5986	0.0001\\
1.5986	0.0002\\
1.5987	0.0002\\
1.5988	0.0005\\
1.5989	0.0007\\
1.5991	0.001\\
1.5992	0.0012\\
1.5999	0.0025\\
1.6013	0.0049\\
1.602	0.0062\\
1.6045	0.0107\\
1.607	0.0151\\
1.6096	0.0195\\
1.6146	0.0281\\
1.6172	0.0322\\
1.6197	0.0364\\
1.6223	0.0404\\
1.6273	0.0484\\
1.6299	0.0522\\
1.6349	0.0598\\
1.6375	0.0635\\
1.64	0.0671\\
1.6425	0.0706\\
1.6451	0.0741\\
1.6476	0.0776\\
1.6502	0.0809\\
1.6552	0.0875\\
1.6578	0.0907\\
1.6603	0.0938\\
1.6628	0.0968\\
1.6654	0.0998\\
1.6704	0.1056\\
1.673	0.1084\\
1.6755	0.1111\\
1.6781	0.1138\\
1.6831	0.119\\
1.6857	0.1215\\
1.6882	0.1239\\
1.6907	0.1262\\
1.6933	0.1285\\
1.695	0.13\\
1.6966	0.1315\\
1.7	0.1343\\
};
\addplot [color=black, forget plot]
  table[row sep=crcr]{%
0	0.9932\\
0.0008	0.9932\\
0.001	0.9933\\
0.005	0.9933\\
};
\addplot [color=black, forget plot]
  table[row sep=crcr]{%
0.005	0.9933\\
0.0082	0.9933\\
0.0146	0.9929\\
0.021	0.9921\\
0.0256	0.9913\\
0.0302	0.9902\\
0.0348	0.989\\
0.0394	0.9875\\
0.044	0.9859\\
0.0486	0.984\\
0.0532	0.982\\
0.0578	0.9797\\
0.0624	0.9772\\
0.067	0.9745\\
0.0716	0.9716\\
0.0762	0.9685\\
0.0808	0.9652\\
0.0854	0.9617\\
0.09	0.9579\\
0.0946	0.954\\
0.0992	0.9499\\
0.1038	0.9455\\
0.1084	0.941\\
0.113	0.9362\\
0.1176	0.9312\\
0.1222	0.9261\\
0.1267	0.9207\\
0.1313	0.9151\\
0.1359	0.9093\\
0.1405	0.9033\\
0.1451	0.8971\\
0.1497	0.8906\\
0.1543	0.884\\
0.1589	0.8772\\
0.1635	0.8701\\
0.1681	0.8629\\
0.1727	0.8554\\
0.1773	0.8478\\
0.1819	0.8399\\
0.1865	0.8318\\
0.1871	0.8307\\
0.1877	0.8297\\
0.1889	0.8275\\
};
\addplot [color=black, forget plot]
  table[row sep=crcr]{%
0.1889	0.8275\\
0.1936	0.8189\\
0.1983	0.8101\\
0.2031	0.801\\
0.2078	0.7917\\
0.2125	0.7822\\
0.2172	0.7725\\
0.2219	0.7625\\
0.2267	0.7524\\
0.2314	0.742\\
0.2361	0.7314\\
0.2408	0.7206\\
0.2456	0.7096\\
0.2503	0.6983\\
0.255	0.6868\\
0.2597	0.6752\\
0.2644	0.6632\\
0.2692	0.6511\\
0.2739	0.6388\\
0.2786	0.6262\\
0.2833	0.6134\\
0.2881	0.6004\\
0.2928	0.5872\\
0.2975	0.5738\\
0.3022	0.5601\\
0.3069	0.5462\\
0.3117	0.5321\\
0.3164	0.5178\\
0.3211	0.5033\\
0.3258	0.4885\\
0.3306	0.4736\\
0.3353	0.4584\\
0.34	0.443\\
0.3447	0.4273\\
0.3494	0.4115\\
0.3542	0.3954\\
0.3589	0.3791\\
0.3636	0.3626\\
0.3683	0.3459\\
0.3731	0.329\\
0.3778	0.3118\\
};
\addplot [color=black, forget plot]
  table[row sep=crcr]{%
0.3778	0.3118\\
0.3821	0.2961\\
0.3863	0.2801\\
0.3906	0.264\\
0.3949	0.2477\\
0.3996	0.2296\\
0.4044	0.2112\\
0.4091	0.1926\\
0.4138	0.1737\\
0.4185	0.1547\\
0.4232	0.1354\\
0.428	0.1159\\
0.4327	0.0962\\
0.4374	0.0763\\
0.4421	0.0562\\
0.4469	0.0358\\
0.4516	0.0152\\
0.4524	0.0114\\
0.4542	0.0038\\
0.455	-0\\
};
\addplot [color=black, forget plot]
  table[row sep=crcr]{%
0.455	0\\
0.4551	0.0001\\
0.4551	0.0002\\
0.4552	0.0005\\
0.4553	0.0007\\
0.4553	0.001\\
0.4554	0.0012\\
0.4558	0.0025\\
0.4562	0.0037\\
0.4565	0.005\\
0.4569	0.0062\\
0.4626	0.0248\\
0.4645	0.0309\\
0.4673	0.0399\\
0.4729	0.0575\\
0.4785	0.0749\\
0.4813	0.0834\\
0.484	0.0919\\
0.4868	0.1003\\
0.4924	0.1169\\
0.4952	0.1251\\
0.498	0.1332\\
0.5008	0.1412\\
0.5064	0.157\\
0.5092	0.1648\\
0.512	0.1725\\
0.5147	0.1802\\
0.5203	0.1952\\
0.5259	0.21\\
0.5315	0.2244\\
0.5371	0.2386\\
0.5399	0.2455\\
0.5426	0.2524\\
0.5454	0.2592\\
0.551	0.2726\\
0.5538	0.2792\\
0.5566	0.2857\\
0.5594	0.2921\\
0.5654	0.3056\\
0.5658	0.3066\\
0.5662	0.3075\\
0.5667	0.3085\\
};
\addplot [color=black, forget plot]
  table[row sep=crcr]{%
0.5667	0.3085\\
0.5714	0.3188\\
0.5761	0.329\\
0.5808	0.3389\\
0.5856	0.3486\\
0.5903	0.3581\\
0.595	0.3673\\
0.5997	0.3764\\
0.6044	0.3852\\
0.6092	0.3938\\
0.6139	0.4022\\
0.6186	0.4103\\
0.6233	0.4183\\
0.6281	0.426\\
0.6328	0.4335\\
0.6375	0.4408\\
0.6422	0.4479\\
0.6469	0.4547\\
0.6517	0.4614\\
0.6564	0.4678\\
0.6611	0.474\\
0.6658	0.48\\
0.6706	0.4857\\
0.6753	0.4913\\
0.68	0.4966\\
0.6847	0.5017\\
0.6894	0.5066\\
0.6942	0.5113\\
0.6989	0.5157\\
0.7036	0.5199\\
0.7083	0.524\\
0.7131	0.5277\\
0.7178	0.5313\\
0.7225	0.5347\\
0.7272	0.5378\\
0.7319	0.5407\\
0.7367	0.5434\\
0.7414	0.5459\\
0.7461	0.5482\\
0.7508	0.5502\\
0.7556	0.552\\
};
\addplot [color=black, forget plot]
  table[row sep=crcr]{%
0.7556	0.552\\
0.7574	0.5527\\
0.7593	0.5533\\
0.7611	0.5539\\
0.763	0.5545\\
0.7677	0.5557\\
0.7724	0.5568\\
0.7772	0.5576\\
0.7819	0.5582\\
0.7845	0.5584\\
0.7872	0.5586\\
0.7926	0.5588\\
};
\addplot [color=black, forget plot]
  table[row sep=crcr]{%
0.7926	0.5588\\
0.7929	0.5588\\
0.7931	0.5587\\
0.7958	0.5587\\
0.799	0.5586\\
0.8022	0.5583\\
0.8086	0.5575\\
0.8124	0.5568\\
0.8162	0.556\\
0.8199	0.5551\\
0.8237	0.554\\
0.8313	0.5514\\
0.8351	0.5499\\
0.8389	0.5482\\
0.8427	0.5464\\
0.8465	0.5445\\
0.8503	0.5424\\
0.8541	0.5402\\
0.8579	0.5378\\
0.8617	0.5353\\
0.8693	0.5299\\
0.8769	0.5239\\
0.8845	0.5173\\
0.8883	0.5138\\
0.8921	0.5102\\
0.8959	0.5064\\
0.8997	0.5025\\
0.9035	0.4984\\
0.9073	0.4942\\
0.9149	0.4854\\
0.9225	0.476\\
0.9263	0.4711\\
0.9301	0.466\\
0.9337	0.4611\\
0.9373	0.4561\\
0.9408	0.4509\\
0.9444	0.4456\\
};
\addplot [color=black, forget plot]
  table[row sep=crcr]{%
0.9444	0.4456\\
0.9492	0.4384\\
0.9539	0.4311\\
0.9586	0.4235\\
0.9633	0.4157\\
0.9681	0.4077\\
0.9728	0.3994\\
0.9775	0.391\\
0.9822	0.3823\\
0.9869	0.3734\\
0.9917	0.3643\\
0.9964	0.355\\
1.0011	0.3454\\
1.0058	0.3356\\
1.0106	0.3257\\
1.0153	0.3154\\
1.02	0.305\\
1.0247	0.2944\\
1.0294	0.2835\\
1.0342	0.2724\\
1.0389	0.2611\\
1.0436	0.2496\\
1.0483	0.2379\\
1.0531	0.2259\\
1.0578	0.2137\\
1.0625	0.2013\\
1.0672	0.1887\\
1.0719	0.1759\\
1.0767	0.1628\\
1.0814	0.1496\\
1.0861	0.1361\\
1.0908	0.1224\\
1.0956	0.1084\\
1.1003	0.0943\\
1.105	0.0799\\
1.1097	0.0653\\
1.1144	0.0505\\
1.1184	0.0381\\
1.1223	0.0256\\
1.1262	0.0129\\
1.1301	0\\
};
\addplot [color=black, forget plot]
  table[row sep=crcr]{%
1.1301	0\\
1.1301	0.0002\\
1.1302	0.0002\\
1.1302	0.0004\\
1.1306	0.0012\\
1.1306	0.0014\\
1.131	0.0022\\
1.131	0.0024\\
1.1315	0.0034\\
1.1315	0.0036\\
1.1319	0.0044\\
1.1319	0.0046\\
1.1324	0.0056\\
1.1324	0.0058\\
1.1328	0.0066\\
1.1328	0.0068\\
1.1331	0.0074\\
1.1331	0.0076\\
1.1332	0.0077\\
1.1333	0.0079\\
1.1333	0.0081\\
};
\addplot [color=black, forget plot]
  table[row sep=crcr]{%
1.1333	0.0081\\
1.1337	0.0089\\
1.1338	0.0093\\
1.134	0.0097\\
1.1356	0.0137\\
1.1365	0.0157\\
1.1373	0.0177\\
1.1414	0.0275\\
1.1455	0.0372\\
1.1497	0.0468\\
1.1538	0.0561\\
1.1585	0.0667\\
1.1632	0.077\\
1.168	0.087\\
1.1727	0.0969\\
1.1774	0.1065\\
1.1821	0.116\\
1.1868	0.1252\\
1.1916	0.1342\\
1.1963	0.1429\\
1.201	0.1515\\
1.2057	0.1598\\
1.2105	0.1679\\
1.2152	0.1758\\
1.2199	0.1835\\
1.2246	0.1909\\
1.2293	0.1982\\
1.2341	0.2052\\
1.2388	0.212\\
1.2435	0.2186\\
1.2482	0.2249\\
1.253	0.2311\\
1.2577	0.237\\
1.2624	0.2427\\
1.2671	0.2482\\
1.2718	0.2535\\
1.2766	0.2585\\
1.2813	0.2634\\
1.286	0.268\\
1.2907	0.2724\\
1.2955	0.2765\\
1.3002	0.2805\\
1.3049	0.2842\\
1.3092	0.2875\\
1.3136	0.2905\\
1.3179	0.2934\\
1.3222	0.2961\\
};
\addplot [color=black, forget plot]
  table[row sep=crcr]{%
1.3222	0.2961\\
1.3253	0.2978\\
1.3283	0.2995\\
1.3345	0.3027\\
1.3392	0.3048\\
1.3439	0.3067\\
1.3486	0.3084\\
1.3534	0.3099\\
1.3581	0.3112\\
1.3628	0.3123\\
1.3675	0.3131\\
1.3723	0.3137\\
1.375	0.314\\
1.3777	0.3142\\
1.3805	0.3143\\
1.3832	0.3143\\
};
\addplot [color=black, forget plot]
  table[row sep=crcr]{%
1.3832	0.3143\\
1.3858	0.3143\\
1.3864	0.3142\\
1.3896	0.3141\\
1.396	0.3135\\
1.4024	0.3125\\
1.4088	0.3111\\
1.4152	0.3093\\
1.4216	0.3071\\
1.428	0.3045\\
1.4344	0.3015\\
1.4376	0.2998\\
1.444	0.2962\\
1.4504	0.2922\\
1.4536	0.29\\
1.4567	0.2878\\
1.4631	0.283\\
1.4663	0.2804\\
1.4727	0.275\\
1.4791	0.2692\\
1.4823	0.2661\\
1.4887	0.2597\\
1.4951	0.2529\\
1.4983	0.2493\\
1.5015	0.2456\\
1.5063	0.24\\
1.5087	0.237\\
1.5111	0.2341\\
};
\addplot [color=black, forget plot]
  table[row sep=crcr]{%
1.5111	0.2341\\
1.5158	0.228\\
1.5206	0.2218\\
1.5253	0.2153\\
1.53	0.2086\\
1.5347	0.2017\\
1.5394	0.1946\\
1.5442	0.1872\\
1.5489	0.1796\\
1.5536	0.1719\\
1.5583	0.1639\\
1.5631	0.1556\\
1.5678	0.1472\\
1.5725	0.1385\\
1.5772	0.1297\\
1.5819	0.1206\\
1.5867	0.1112\\
1.5914	0.1017\\
1.5961	0.092\\
1.6008	0.082\\
1.6056	0.0718\\
1.6103	0.0614\\
1.615	0.0508\\
1.6197	0.0399\\
1.6244	0.0288\\
1.6274	0.0218\\
1.6304	0.0146\\
1.6334	0.0073\\
1.6363	-0\\
};
\addplot [color=black, forget plot]
  table[row sep=crcr]{%
1.6363	0\\
1.6364	0.0001\\
1.6364	0.0002\\
1.6366	0.0005\\
1.6367	0.0007\\
1.6369	0.001\\
1.637	0.0012\\
1.6377	0.0025\\
1.6383	0.0037\\
1.639	0.0049\\
1.6397	0.0062\\
1.6413	0.0091\\
1.6429	0.0119\\
1.6445	0.0148\\
1.646	0.0176\\
1.6492	0.0232\\
1.6508	0.0259\\
1.6524	0.0287\\
1.6556	0.0341\\
1.6636	0.0471\\
1.6651	0.0496\\
1.6683	0.0546\\
1.6763	0.0666\\
1.6811	0.0735\\
1.6827	0.0757\\
1.6842	0.078\\
1.6858	0.0802\\
1.6874	0.0823\\
1.689	0.0845\\
1.6954	0.0929\\
1.697	0.0949\\
1.6977	0.0959\\
1.6985	0.0968\\
1.6992	0.0977\\
1.7	0.0987\\
};
\addplot [color=black, forget plot]
  table[row sep=crcr]{%
0	0.9996\\
0.0018	0.9996\\
0.0054	0.9994\\
0.0073	0.9993\\
0.0091	0.9991\\
0.0139	0.9985\\
0.0186	0.9977\\
0.0233	0.9967\\
0.028	0.9954\\
0.0327	0.9939\\
0.0375	0.9922\\
0.0422	0.9903\\
0.0469	0.9882\\
0.0516	0.9858\\
0.0564	0.9833\\
0.0611	0.9805\\
0.0658	0.9775\\
0.0705	0.9742\\
0.0752	0.9708\\
0.08	0.9671\\
0.0847	0.9633\\
0.0894	0.9592\\
0.0941	0.9548\\
0.0989	0.9503\\
0.1036	0.9455\\
0.1083	0.9406\\
0.113	0.9354\\
0.1177	0.93\\
0.1225	0.9243\\
0.1272	0.9185\\
0.1319	0.9124\\
0.1366	0.9061\\
0.1414	0.8996\\
0.1461	0.8929\\
0.1508	0.8859\\
0.1555	0.8788\\
0.1602	0.8714\\
0.165	0.8638\\
0.1697	0.856\\
0.1744	0.8479\\
0.1791	0.8397\\
0.1816	0.8353\\
0.1864	0.8265\\
0.1889	0.8219\\
};
\addplot [color=black, forget plot]
  table[row sep=crcr]{%
0.1889	0.8219\\
0.1936	0.813\\
0.1983	0.8039\\
0.2031	0.7945\\
0.2078	0.7849\\
0.2125	0.7751\\
0.2172	0.7651\\
0.2219	0.7548\\
0.2267	0.7444\\
0.2314	0.7337\\
0.2361	0.7228\\
0.2408	0.7117\\
0.2456	0.7004\\
0.2503	0.6888\\
0.255	0.677\\
0.2597	0.665\\
0.2644	0.6528\\
0.2692	0.6404\\
0.2739	0.6278\\
0.2786	0.6149\\
0.2833	0.6018\\
0.2881	0.5885\\
0.2928	0.575\\
0.2975	0.5612\\
0.3022	0.5473\\
0.3069	0.5331\\
0.3117	0.5187\\
0.3164	0.5041\\
0.3211	0.4893\\
0.3258	0.4742\\
0.3306	0.4589\\
0.3353	0.4435\\
0.34	0.4277\\
0.3447	0.4118\\
0.3494	0.3957\\
0.3542	0.3793\\
0.3589	0.3627\\
0.3636	0.3459\\
0.3683	0.3289\\
0.3731	0.3117\\
0.3778	0.2942\\
};
\addplot [color=black, forget plot]
  table[row sep=crcr]{%
0.3778	0.2942\\
0.3818	0.2793\\
0.3857	0.2643\\
0.3897	0.2492\\
0.3937	0.2338\\
0.3984	0.2154\\
0.4031	0.1968\\
0.4078	0.1779\\
0.4126	0.1589\\
0.4173	0.1396\\
0.422	0.1201\\
0.4267	0.1003\\
0.4314	0.0804\\
0.4361	0.0606\\
0.4407	0.0406\\
0.4453	0.0204\\
0.45	-0\\
};
\addplot [color=black, forget plot]
  table[row sep=crcr]{%
0.45	0\\
0.45	0.0002\\
0.4501	0.0005\\
0.4502	0.0007\\
0.4503	0.001\\
0.4503	0.0012\\
0.4507	0.0025\\
0.4511	0.0037\\
0.4515	0.005\\
0.4519	0.0062\\
0.4537	0.0124\\
0.4575	0.0248\\
0.4594	0.0309\\
0.4623	0.0403\\
0.4653	0.0496\\
0.4682	0.0588\\
0.474	0.077\\
0.4798	0.0948\\
0.4828	0.1036\\
0.4857	0.1123\\
0.4915	0.1295\\
0.4973	0.1463\\
0.5003	0.1546\\
0.5032	0.1628\\
0.509	0.179\\
0.5148	0.1948\\
0.5178	0.2026\\
0.5207	0.2103\\
0.5265	0.2255\\
0.5323	0.2403\\
0.5353	0.2476\\
0.5382	0.2548\\
0.544	0.269\\
0.5469	0.2759\\
0.5499	0.2828\\
0.5528	0.2896\\
0.5557	0.2963\\
0.5615	0.3095\\
0.5644	0.3159\\
0.565	0.3171\\
0.5656	0.3184\\
0.5661	0.3196\\
0.5667	0.3208\\
};
\addplot [color=black, forget plot]
  table[row sep=crcr]{%
0.5667	0.3208\\
0.5714	0.331\\
0.5761	0.3409\\
0.5808	0.3506\\
0.5856	0.3602\\
0.5903	0.3695\\
0.595	0.3785\\
0.5997	0.3874\\
0.6044	0.396\\
0.6092	0.4044\\
0.6139	0.4126\\
0.6186	0.4206\\
0.6233	0.4284\\
0.6281	0.4359\\
0.6328	0.4433\\
0.6375	0.4504\\
0.6422	0.4573\\
0.6469	0.4639\\
0.6517	0.4704\\
0.6564	0.4766\\
0.6611	0.4826\\
0.6658	0.4884\\
0.6706	0.494\\
0.6753	0.4994\\
0.68	0.5045\\
0.6847	0.5094\\
0.6894	0.5141\\
0.6942	0.5186\\
0.6989	0.5229\\
0.7036	0.5269\\
0.7083	0.5307\\
0.7131	0.5343\\
0.7178	0.5377\\
0.7225	0.5409\\
0.7272	0.5438\\
0.7319	0.5466\\
0.7367	0.5491\\
0.7414	0.5514\\
0.7461	0.5535\\
0.7508	0.5553\\
0.7556	0.557\\
};
\addplot [color=black, forget plot]
  table[row sep=crcr]{%
0.7556	0.557\\
0.7572	0.5575\\
0.7589	0.558\\
0.7605	0.5585\\
0.7622	0.5589\\
0.7669	0.56\\
0.7716	0.5609\\
0.7764	0.5616\\
0.7811	0.562\\
0.7829	0.5622\\
0.7848	0.5622\\
0.7867	0.5623\\
0.7886	0.5623\\
};
\addplot [color=black, forget plot]
  table[row sep=crcr]{%
0.7886	0.5623\\
0.7918	0.5623\\
0.7982	0.5619\\
0.8046	0.5611\\
0.8085	0.5604\\
0.8124	0.5595\\
0.8163	0.5585\\
0.8202	0.5574\\
0.824	0.5561\\
0.8279	0.5547\\
0.8318	0.5531\\
0.8357	0.5514\\
0.8396	0.5495\\
0.8435	0.5475\\
0.8474	0.5453\\
0.8513	0.543\\
0.8552	0.5405\\
0.8591	0.5379\\
0.863	0.5351\\
0.8669	0.5322\\
0.8708	0.5291\\
0.8747	0.5259\\
0.8786	0.5225\\
0.8825	0.519\\
0.8864	0.5154\\
0.8942	0.5076\\
0.8981	0.5035\\
0.902	0.4992\\
0.9059	0.4948\\
0.9098	0.4902\\
0.9137	0.4855\\
0.9176	0.4807\\
0.9215	0.4757\\
0.9254	0.4705\\
0.9293	0.4652\\
0.9331	0.4599\\
0.9407	0.4489\\
0.9444	0.4431\\
};
\addplot [color=black, forget plot]
  table[row sep=crcr]{%
0.9444	0.4431\\
0.9492	0.4358\\
0.9539	0.4282\\
0.9586	0.4205\\
0.9633	0.4125\\
0.9681	0.4043\\
0.9728	0.3959\\
0.9775	0.3872\\
0.9822	0.3784\\
0.9869	0.3693\\
0.9917	0.36\\
0.9964	0.3505\\
1.0011	0.3407\\
1.0058	0.3308\\
1.0106	0.3206\\
1.0153	0.3102\\
1.02	0.2996\\
1.0247	0.2888\\
1.0294	0.2777\\
1.0342	0.2664\\
1.0389	0.255\\
1.0436	0.2432\\
1.0483	0.2313\\
1.0531	0.2192\\
1.0578	0.2068\\
1.0625	0.1942\\
1.0672	0.1814\\
1.0719	0.1684\\
1.0767	0.1552\\
1.0814	0.1417\\
1.0861	0.1281\\
1.0908	0.1142\\
1.0956	0.1\\
1.1003	0.0857\\
1.105	0.0712\\
1.1097	0.0564\\
1.1144	0.0414\\
1.1176	0.0312\\
1.1208	0.0209\\
1.124	0.0105\\
1.1272	-0\\
};
\addplot [color=black, forget plot]
  table[row sep=crcr]{%
1.1272	0\\
1.1272	0.0002\\
1.1273	0.0005\\
1.1274	0.0007\\
1.1275	0.001\\
1.1278	0.0016\\
1.1279	0.002\\
1.1281	0.0024\\
1.1283	0.0027\\
1.1284	0.0031\\
1.1286	0.0035\\
1.1287	0.0039\\
1.1289	0.0043\\
1.129	0.0047\\
1.1292	0.005\\
1.1293	0.0054\\
1.1295	0.0058\\
1.1296	0.0062\\
1.1298	0.0066\\
1.13	0.0069\\
1.1301	0.0073\\
1.1303	0.0077\\
1.1304	0.0081\\
1.1306	0.0085\\
1.1307	0.0088\\
1.1309	0.0092\\
1.131	0.0096\\
1.1312	0.01\\
1.1313	0.0104\\
1.1315	0.0107\\
1.1317	0.0111\\
1.1318	0.0115\\
1.132	0.0119\\
1.1321	0.0123\\
1.1323	0.0126\\
1.1324	0.013\\
1.1326	0.0134\\
1.1327	0.0138\\
1.1329	0.0141\\
1.133	0.0145\\
1.1332	0.0149\\
1.1332	0.015\\
1.1333	0.0151\\
1.1333	0.0152\\
};
\addplot [color=black, forget plot]
  table[row sep=crcr]{%
1.1333	0.0152\\
1.1336	0.016\\
1.134	0.0167\\
1.1346	0.0183\\
1.1362	0.0221\\
1.1377	0.0258\\
1.1393	0.0296\\
1.1409	0.0333\\
1.1456	0.0443\\
1.1503	0.0551\\
1.155	0.0657\\
1.1598	0.076\\
1.1645	0.0862\\
1.1692	0.0961\\
1.1739	0.1058\\
1.1787	0.1153\\
1.1834	0.1246\\
1.1881	0.1336\\
1.1928	0.1424\\
1.1975	0.1511\\
1.2023	0.1595\\
1.207	0.1676\\
1.2117	0.1756\\
1.2164	0.1833\\
1.2212	0.1908\\
1.2259	0.1981\\
1.2306	0.2052\\
1.2353	0.2121\\
1.24	0.2187\\
1.2448	0.2251\\
1.2495	0.2314\\
1.2542	0.2373\\
1.2589	0.2431\\
1.2637	0.2487\\
1.2684	0.254\\
1.2731	0.2591\\
1.2778	0.264\\
1.2825	0.2687\\
1.2873	0.2731\\
1.292	0.2774\\
1.2967	0.2814\\
1.3014	0.2852\\
1.3062	0.2888\\
1.3109	0.2921\\
1.3137	0.294\\
1.3166	0.2959\\
1.3222	0.2993\\
};
\addplot [color=black, forget plot]
  table[row sep=crcr]{%
1.3222	0.2993\\
1.3252	0.301\\
1.3281	0.3025\\
1.3311	0.304\\
1.3341	0.3054\\
1.3388	0.3075\\
1.3435	0.3094\\
1.3482	0.311\\
1.3529	0.3124\\
1.3577	0.3136\\
1.3624	0.3146\\
1.3671	0.3153\\
1.3718	0.3159\\
1.3741	0.3161\\
1.3765	0.3162\\
1.3788	0.3163\\
1.3811	0.3163\\
};
\addplot [color=black, forget plot]
  table[row sep=crcr]{%
1.3811	0.3163\\
1.3836	0.3163\\
1.3843	0.3162\\
1.3875	0.3161\\
1.3939	0.3155\\
1.4003	0.3145\\
1.4036	0.3138\\
1.4068	0.313\\
1.4101	0.3122\\
1.4133	0.3112\\
1.4166	0.3101\\
1.4198	0.3089\\
1.4231	0.3076\\
1.4263	0.3063\\
1.4296	0.3048\\
1.4328	0.3032\\
1.4361	0.3015\\
1.4393	0.2997\\
1.4426	0.2977\\
1.4458	0.2957\\
1.4491	0.2936\\
1.4523	0.2914\\
1.4556	0.2891\\
1.4588	0.2866\\
1.4621	0.2841\\
1.4653	0.2815\\
1.4686	0.2787\\
1.4718	0.2759\\
1.4784	0.2699\\
1.4816	0.2667\\
1.4849	0.2635\\
1.4881	0.2601\\
1.4914	0.2567\\
1.4946	0.2531\\
1.4979	0.2494\\
1.5011	0.2456\\
1.5036	0.2427\\
1.5111	0.2334\\
};
\addplot [color=black, forget plot]
  table[row sep=crcr]{%
1.5111	0.2334\\
1.5158	0.2272\\
1.5206	0.2209\\
1.5253	0.2143\\
1.53	0.2075\\
1.5347	0.2005\\
1.5394	0.1933\\
1.5442	0.1859\\
1.5489	0.1782\\
1.5536	0.1703\\
1.5583	0.1622\\
1.5631	0.1539\\
1.5678	0.1453\\
1.5725	0.1366\\
1.5772	0.1276\\
1.5819	0.1184\\
1.5867	0.109\\
1.5914	0.0994\\
1.5961	0.0895\\
1.6008	0.0794\\
1.6056	0.0692\\
1.6103	0.0587\\
1.615	0.0479\\
1.6197	0.037\\
1.6244	0.0258\\
1.6271	0.0195\\
1.6297	0.013\\
1.6324	0.0066\\
1.635	-0\\
};
\addplot [color=black, forget plot]
  table[row sep=crcr]{%
1.635	0\\
1.6351	0.0001\\
1.6351	0.0002\\
1.6353	0.0005\\
1.6354	0.0007\\
1.6355	0.001\\
1.6357	0.0012\\
1.6363	0.0025\\
1.6377	0.0049\\
1.6384	0.0062\\
1.64	0.0091\\
1.6416	0.0121\\
1.6432	0.015\\
1.6449	0.0179\\
1.6465	0.0208\\
1.6497	0.0264\\
1.6514	0.0292\\
1.653	0.032\\
1.6562	0.0374\\
1.6579	0.0401\\
1.6595	0.0427\\
1.6611	0.0454\\
1.6627	0.048\\
1.6644	0.0506\\
1.666	0.0531\\
1.6676	0.0557\\
1.6692	0.0582\\
1.6708	0.0606\\
1.6725	0.0631\\
1.6773	0.0703\\
1.679	0.0726\\
1.6838	0.0795\\
1.6855	0.0818\\
1.6887	0.0862\\
1.6903	0.0883\\
1.692	0.0905\\
1.6952	0.0947\\
1.7	0.1007\\
};
\addplot [color=black, forget plot]
  table[row sep=crcr]{%
0	0.981\\
0.0029	0.981\\
0.0043	0.9809\\
0.0073	0.9807\\
0.012	0.9802\\
0.0167	0.9795\\
0.0214	0.9785\\
0.0262	0.9774\\
0.0309	0.976\\
0.0356	0.9744\\
0.0403	0.9726\\
0.0451	0.9706\\
0.0498	0.9683\\
0.0545	0.9658\\
0.0592	0.9631\\
0.0639	0.9602\\
0.0687	0.9571\\
0.0734	0.9538\\
0.0781	0.9502\\
0.0828	0.9464\\
0.0876	0.9424\\
0.0923	0.9382\\
0.097	0.9338\\
0.1017	0.9291\\
0.1064	0.9242\\
0.1112	0.9191\\
0.1159	0.9138\\
0.1206	0.9083\\
0.1253	0.9025\\
0.1301	0.8966\\
0.1348	0.8904\\
0.1395	0.884\\
0.1442	0.8774\\
0.1489	0.8705\\
0.1537	0.8634\\
0.1584	0.8562\\
0.1631	0.8487\\
0.1678	0.8409\\
0.1726	0.833\\
0.1773	0.8248\\
0.1802	0.8197\\
0.1831	0.8145\\
0.1889	0.8039\\
};
\addplot [color=black, forget plot]
  table[row sep=crcr]{%
0.1889	0.8039\\
0.1936	0.7949\\
0.1983	0.7858\\
0.2031	0.7765\\
0.2078	0.7669\\
0.2125	0.7571\\
0.2172	0.7471\\
0.2219	0.7369\\
0.2267	0.7264\\
0.2314	0.7158\\
0.2361	0.7049\\
0.2408	0.6938\\
0.2456	0.6825\\
0.2503	0.6709\\
0.255	0.6592\\
0.2597	0.6472\\
0.2644	0.635\\
0.2692	0.6226\\
0.2739	0.6099\\
0.2786	0.5971\\
0.2833	0.584\\
0.2881	0.5707\\
0.2928	0.5572\\
0.2975	0.5435\\
0.3022	0.5296\\
0.3069	0.5154\\
0.3117	0.501\\
0.3164	0.4864\\
0.3211	0.4716\\
0.3258	0.4565\\
0.3306	0.4413\\
0.3353	0.4258\\
0.34	0.4101\\
0.3447	0.3942\\
0.3494	0.3781\\
0.3542	0.3617\\
0.3589	0.3451\\
0.3636	0.3284\\
0.3683	0.3113\\
0.3731	0.2941\\
0.3778	0.2767\\
};
\addplot [color=black, forget plot]
  table[row sep=crcr]{%
0.3778	0.2767\\
0.3815	0.2627\\
0.3853	0.2486\\
0.389	0.2344\\
0.3927	0.22\\
0.3975	0.2016\\
0.4022	0.183\\
0.4069	0.1643\\
0.4116	0.1452\\
0.4163	0.126\\
0.4211	0.1066\\
0.4258	0.0869\\
0.4305	0.067\\
0.4344	0.0505\\
0.4383	0.0338\\
0.4422	0.017\\
0.4461	0\\
};
\addplot [color=black, forget plot]
  table[row sep=crcr]{%
0.4461	0\\
0.4461	0.0002\\
0.4462	0.0005\\
0.4463	0.0007\\
0.4463	0.001\\
0.4464	0.0012\\
0.4468	0.0025\\
0.4472	0.0037\\
0.4476	0.005\\
0.4479	0.0062\\
0.4499	0.0124\\
0.4537	0.0248\\
0.4556	0.0309\\
0.4586	0.0405\\
0.4616	0.05\\
0.4646	0.0594\\
0.4676	0.0687\\
0.4707	0.078\\
0.4767	0.0962\\
0.4797	0.1052\\
0.4827	0.1141\\
0.4887	0.1315\\
0.4918	0.1402\\
0.4948	0.1487\\
0.5008	0.1655\\
0.5068	0.1819\\
0.5099	0.19\\
0.5129	0.198\\
0.5159	0.2059\\
0.5189	0.2137\\
0.5219	0.2214\\
0.5279	0.2366\\
0.531	0.244\\
0.534	0.2514\\
0.537	0.2587\\
0.54	0.2659\\
0.543	0.273\\
0.546	0.28\\
0.5491	0.2869\\
0.5551	0.3005\\
0.5611	0.3137\\
0.5641	0.3201\\
0.5648	0.3215\\
0.5654	0.3228\\
0.566	0.3242\\
0.5667	0.3255\\
};
\addplot [color=black, forget plot]
  table[row sep=crcr]{%
0.5667	0.3255\\
0.5714	0.3354\\
0.5761	0.345\\
0.5808	0.3544\\
0.5856	0.3636\\
0.5903	0.3725\\
0.595	0.3813\\
0.5997	0.3898\\
0.6044	0.3981\\
0.6092	0.4062\\
0.6139	0.4141\\
0.6186	0.4217\\
0.6233	0.4292\\
0.6281	0.4364\\
0.6328	0.4434\\
0.6375	0.4502\\
0.6422	0.4567\\
0.6469	0.4631\\
0.6517	0.4692\\
0.6564	0.4751\\
0.6611	0.4808\\
0.6658	0.4862\\
0.6706	0.4915\\
0.6753	0.4965\\
0.68	0.5013\\
0.6847	0.5059\\
0.6894	0.5103\\
0.6942	0.5144\\
0.6989	0.5184\\
0.7036	0.5221\\
0.7083	0.5256\\
0.7131	0.5289\\
0.7178	0.5319\\
0.7225	0.5348\\
0.7272	0.5374\\
0.7319	0.5398\\
0.7367	0.542\\
0.7414	0.544\\
0.7461	0.5457\\
0.7508	0.5472\\
0.7556	0.5485\\
};
\addplot [color=black, forget plot]
  table[row sep=crcr]{%
0.7556	0.5485\\
0.7569	0.5489\\
0.7595	0.5495\\
0.7608	0.5497\\
0.7655	0.5506\\
0.7702	0.5512\\
0.7749	0.5516\\
0.7797	0.5518\\
0.7815	0.5518\\
};
\addplot [color=black, forget plot]
  table[row sep=crcr]{%
0.7815	0.5518\\
0.7847	0.5518\\
0.7911	0.5514\\
0.7975	0.5506\\
0.8015	0.5499\\
0.8056	0.549\\
0.8097	0.5479\\
0.8138	0.5467\\
0.8178	0.5453\\
0.8219	0.5438\\
0.826	0.5421\\
0.8301	0.5403\\
0.8341	0.5382\\
0.8382	0.536\\
0.8423	0.5337\\
0.8464	0.5312\\
0.8504	0.5285\\
0.8545	0.5257\\
0.8586	0.5227\\
0.8627	0.5195\\
0.8667	0.5162\\
0.8708	0.5127\\
0.8749	0.509\\
0.879	0.5052\\
0.8871	0.4971\\
0.8912	0.4928\\
0.8953	0.4883\\
0.8993	0.4837\\
0.9034	0.4789\\
0.9075	0.474\\
0.9116	0.4688\\
0.9156	0.4636\\
0.9197	0.4581\\
0.9238	0.4525\\
0.9278	0.4467\\
0.932	0.4407\\
0.9361	0.4345\\
0.9403	0.4281\\
0.9444	0.4216\\
};
\addplot [color=black, forget plot]
  table[row sep=crcr]{%
0.9444	0.4216\\
0.9492	0.4139\\
0.9539	0.406\\
0.9586	0.3979\\
0.9633	0.3896\\
0.9681	0.3811\\
0.9728	0.3723\\
0.9775	0.3633\\
0.9822	0.3542\\
0.9869	0.3447\\
0.9917	0.3351\\
0.9964	0.3253\\
1.0011	0.3152\\
1.0058	0.3049\\
1.0106	0.2944\\
1.0153	0.2837\\
1.02	0.2728\\
1.0247	0.2616\\
1.0294	0.2502\\
1.0342	0.2386\\
1.0389	0.2268\\
1.0436	0.2148\\
1.0483	0.2025\\
1.0531	0.19\\
1.0578	0.1774\\
1.0625	0.1644\\
1.0672	0.1513\\
1.0719	0.138\\
1.0767	0.1244\\
1.0814	0.1106\\
1.0861	0.0966\\
1.0908	0.0824\\
1.0956	0.068\\
1.1003	0.0533\\
1.105	0.0384\\
1.1097	0.0233\\
1.1144	0.008\\
1.1151	0.006\\
1.1169	-0\\
};
\addplot [color=black, forget plot]
  table[row sep=crcr]{%
1.1169	0\\
1.1169	0.0002\\
1.117	0.0002\\
1.1171	0.0005\\
1.1172	0.0007\\
1.1173	0.001\\
1.1174	0.0012\\
1.1194	0.0062\\
1.1198	0.0073\\
1.1203	0.0083\\
1.1211	0.0103\\
1.1215	0.0112\\
1.1235	0.0162\\
1.124	0.0172\\
1.1248	0.0192\\
1.1252	0.0201\\
1.1268	0.0241\\
1.1272	0.025\\
1.1277	0.026\\
1.1281	0.027\\
1.1285	0.0279\\
1.1293	0.0299\\
1.1297	0.0308\\
1.1305	0.0328\\
1.1309	0.0337\\
1.1314	0.0347\\
1.1318	0.0356\\
1.1322	0.0366\\
1.1325	0.0373\\
1.1328	0.0379\\
1.133	0.0386\\
1.1333	0.0393\\
};
\addplot [color=black, forget plot]
  table[row sep=crcr]{%
1.1333	0.0393\\
1.1342	0.0412\\
1.135	0.0432\\
1.1359	0.0451\\
1.1368	0.0471\\
1.141	0.0567\\
1.1453	0.0662\\
1.1496	0.0754\\
1.1539	0.0845\\
1.1586	0.0944\\
1.1633	0.104\\
1.168	0.1134\\
1.1727	0.1225\\
1.1775	0.1315\\
1.1822	0.1402\\
1.1869	0.1488\\
1.1916	0.1571\\
1.1964	0.1651\\
1.2011	0.173\\
1.2058	0.1806\\
1.2105	0.1881\\
1.2152	0.1953\\
1.22	0.2023\\
1.2247	0.209\\
1.2294	0.2156\\
1.2341	0.2219\\
1.2389	0.228\\
1.2436	0.2339\\
1.2483	0.2396\\
1.253	0.2451\\
1.2577	0.2503\\
1.2625	0.2553\\
1.2672	0.2601\\
1.2719	0.2647\\
1.2766	0.2691\\
1.2814	0.2732\\
1.2861	0.2771\\
1.2908	0.2808\\
1.2955	0.2843\\
1.3002	0.2876\\
1.305	0.2906\\
1.3093	0.2932\\
1.3136	0.2956\\
1.3179	0.2979\\
1.3222	0.2999\\
};
\addplot [color=black, forget plot]
  table[row sep=crcr]{%
1.3222	0.2999\\
1.3245	0.301\\
1.3269	0.3019\\
1.3315	0.3037\\
1.3362	0.3053\\
1.341	0.3067\\
1.3457	0.3079\\
1.3504	0.3088\\
1.3549	0.3095\\
1.3594	0.31\\
1.3639	0.3103\\
1.3684	0.3104\\
};
\addplot [color=black, forget plot]
  table[row sep=crcr]{%
1.3684	0.3104\\
1.3716	0.3104\\
1.378	0.31\\
1.3844	0.3092\\
1.3916	0.3078\\
1.3951	0.3069\\
1.3987	0.3059\\
1.4023	0.3048\\
1.4058	0.3035\\
1.4094	0.3022\\
1.413	0.3007\\
1.4165	0.2991\\
1.4201	0.2973\\
1.4237	0.2954\\
1.4272	0.2934\\
1.4308	0.2913\\
1.4344	0.2891\\
1.4379	0.2867\\
1.4415	0.2842\\
1.4451	0.2816\\
1.4486	0.2789\\
1.4522	0.276\\
1.4558	0.273\\
1.4593	0.2699\\
1.4665	0.2633\\
1.4736	0.2562\\
1.4772	0.2524\\
1.4807	0.2486\\
1.4843	0.2446\\
1.4879	0.2404\\
1.4914	0.2362\\
1.495	0.2318\\
1.4986	0.2273\\
1.5017	0.2233\\
1.5048	0.2192\\
1.508	0.2149\\
1.5111	0.2106\\
};
\addplot [color=black, forget plot]
  table[row sep=crcr]{%
1.5111	0.2106\\
1.5158	0.2039\\
1.5206	0.1969\\
1.5253	0.1898\\
1.53	0.1824\\
1.5347	0.1748\\
1.5394	0.167\\
1.5442	0.159\\
1.5489	0.1507\\
1.5536	0.1422\\
1.5583	0.1336\\
1.5631	0.1246\\
1.5678	0.1155\\
1.5725	0.1062\\
1.5772	0.0966\\
1.5819	0.0868\\
1.5867	0.0768\\
1.5914	0.0666\\
1.5961	0.0562\\
1.6008	0.0455\\
1.6056	0.0346\\
1.6092	0.0262\\
1.6128	0.0176\\
1.6164	0.0089\\
1.62	-0\\
};
\addplot [color=black, forget plot]
  table[row sep=crcr]{%
1.62	0\\
1.62	0.0001\\
1.6201	0.0001\\
1.6201	0.0002\\
1.6203	0.0005\\
1.6204	0.0007\\
1.6205	0.001\\
1.6207	0.0012\\
1.6213	0.0025\\
1.6227	0.0049\\
1.6234	0.0062\\
1.6274	0.0134\\
1.6334	0.0239\\
1.6374	0.0307\\
1.6434	0.0406\\
1.6474	0.047\\
1.6534	0.0563\\
1.6594	0.0653\\
1.6614	0.0682\\
1.6634	0.071\\
1.6654	0.0739\\
1.6674	0.0767\\
1.6734	0.0848\\
1.6794	0.0926\\
1.6834	0.0976\\
1.6894	0.1048\\
1.6934	0.1094\\
1.6954	0.1116\\
1.6965	0.1129\\
1.6977	0.1142\\
1.6988	0.1154\\
1.7	0.1167\\
};
\addplot [color=black, forget plot]
  table[row sep=crcr]{%
0	1.0483\\
0.0011	1.0483\\
0.0016	1.0482\\
0.0028	1.0482\\
0.0058	1.048\\
0.0116	1.0474\\
0.0146	1.0469\\
0.0193	1.046\\
0.024	1.0449\\
0.0287	1.0436\\
0.0334	1.042\\
0.0382	1.0403\\
0.0429	1.0383\\
0.0476	1.0361\\
0.0523	1.0336\\
0.0571	1.031\\
0.0618	1.0281\\
0.0665	1.0251\\
0.0712	1.0218\\
0.0759	1.0182\\
0.0807	1.0145\\
0.0854	1.0106\\
0.0901	1.0064\\
0.0948	1.002\\
0.0996	0.9974\\
0.1043	0.9926\\
0.109	0.9875\\
0.1137	0.9822\\
0.1184	0.9768\\
0.1232	0.971\\
0.1279	0.9651\\
0.1326	0.959\\
0.1373	0.9526\\
0.1421	0.946\\
0.1468	0.9392\\
0.1515	0.9322\\
0.1562	0.925\\
0.1609	0.9175\\
0.1657	0.9099\\
0.1704	0.902\\
0.1751	0.8939\\
0.1798	0.8855\\
0.1846	0.877\\
0.1856	0.875\\
0.1878	0.871\\
0.1889	0.8689\\
};
\addplot [color=black, forget plot]
  table[row sep=crcr]{%
0.1889	0.8689\\
0.1936	0.86\\
0.1983	0.8508\\
0.2031	0.8414\\
0.2078	0.8318\\
0.2125	0.8219\\
0.2172	0.8119\\
0.2219	0.8016\\
0.2267	0.7911\\
0.2314	0.7804\\
0.2361	0.7694\\
0.2408	0.7583\\
0.2456	0.7469\\
0.2503	0.7353\\
0.255	0.7235\\
0.2597	0.7115\\
0.2644	0.6992\\
0.2692	0.6867\\
0.2739	0.6741\\
0.2786	0.6611\\
0.2833	0.648\\
0.2881	0.6347\\
0.2928	0.6211\\
0.2975	0.6073\\
0.3022	0.5933\\
0.3069	0.5791\\
0.3117	0.5647\\
0.3164	0.55\\
0.3211	0.5352\\
0.3258	0.5201\\
0.3306	0.5048\\
0.3353	0.4892\\
0.34	0.4735\\
0.3447	0.4575\\
0.3494	0.4413\\
0.3542	0.4249\\
0.3589	0.4083\\
0.3636	0.3914\\
0.3683	0.3744\\
0.3731	0.3571\\
0.3778	0.3396\\
};
\addplot [color=black, forget plot]
  table[row sep=crcr]{%
0.3778	0.3396\\
0.3824	0.3224\\
0.3869	0.3051\\
0.3915	0.2875\\
0.3961	0.2697\\
0.4008	0.2511\\
0.4055	0.2324\\
0.4102	0.2134\\
0.415	0.1941\\
0.4197	0.1747\\
0.4244	0.155\\
0.4291	0.1352\\
0.4339	0.1151\\
0.4386	0.0947\\
0.4433	0.0742\\
0.448	0.0535\\
0.4527	0.0325\\
0.4546	0.0244\\
0.4582	0.0082\\
0.46	-0\\
};
\addplot [color=black, forget plot]
  table[row sep=crcr]{%
0.46	0\\
0.46	0.0002\\
0.4601	0.0005\\
0.4602	0.0007\\
0.4602	0.001\\
0.4603	0.0012\\
0.4607	0.0025\\
0.4611	0.0037\\
0.4614	0.005\\
0.4618	0.0062\\
0.4636	0.0124\\
0.4655	0.0186\\
0.4673	0.0248\\
0.4692	0.0309\\
0.4718	0.0397\\
0.4745	0.0485\\
0.4799	0.0657\\
0.4825	0.0742\\
0.4852	0.0827\\
0.4879	0.091\\
0.4905	0.0994\\
0.4959	0.1158\\
0.4985	0.1239\\
0.5039	0.1399\\
0.5065	0.1477\\
0.5092	0.1556\\
0.5119	0.1633\\
0.5145	0.171\\
0.5172	0.1786\\
0.5199	0.1861\\
0.5225	0.1936\\
0.5252	0.201\\
0.5279	0.2083\\
0.5305	0.2156\\
0.5332	0.2228\\
0.5359	0.2299\\
0.5385	0.237\\
0.5412	0.244\\
0.5439	0.2509\\
0.5465	0.2577\\
0.5492	0.2645\\
0.5519	0.2712\\
0.5545	0.2778\\
0.5572	0.2844\\
0.5599	0.2909\\
0.5625	0.2973\\
0.5652	0.3037\\
0.5656	0.3045\\
0.5659	0.3054\\
0.5663	0.3062\\
0.5667	0.3071\\
};
\addplot [color=black, forget plot]
  table[row sep=crcr]{%
0.5667	0.3071\\
0.5714	0.3181\\
0.5761	0.3289\\
0.5808	0.3395\\
0.5856	0.3498\\
0.5903	0.36\\
0.595	0.3699\\
0.5997	0.3796\\
0.6044	0.389\\
0.6092	0.3983\\
0.6139	0.4073\\
0.6186	0.4162\\
0.6233	0.4248\\
0.6281	0.4332\\
0.6328	0.4413\\
0.6375	0.4493\\
0.6422	0.457\\
0.6469	0.4645\\
0.6517	0.4718\\
0.6564	0.4789\\
0.6611	0.4857\\
0.6658	0.4924\\
0.6706	0.4988\\
0.6753	0.505\\
0.68	0.5109\\
0.6847	0.5167\\
0.6894	0.5222\\
0.6942	0.5276\\
0.6989	0.5327\\
0.7036	0.5376\\
0.7083	0.5422\\
0.7131	0.5467\\
0.7178	0.5509\\
0.7225	0.5549\\
0.7272	0.5587\\
0.7319	0.5623\\
0.7367	0.5656\\
0.7414	0.5688\\
0.7461	0.5717\\
0.7508	0.5744\\
0.7556	0.5768\\
};
\addplot [color=black, forget plot]
  table[row sep=crcr]{%
0.7556	0.5768\\
0.7581	0.5781\\
0.7607	0.5793\\
0.7633	0.5804\\
0.7658	0.5815\\
0.7706	0.5833\\
0.7753	0.5848\\
0.78	0.5862\\
0.7847	0.5873\\
0.7894	0.5882\\
0.7942	0.5889\\
0.7989	0.5894\\
0.8036	0.5896\\
0.8044	0.5896\\
0.8051	0.5897\\
0.8067	0.5897\\
};
\addplot [color=black, forget plot]
  table[row sep=crcr]{%
0.8067	0.5897\\
0.8086	0.5897\\
0.8092	0.5896\\
0.8099	0.5896\\
0.8131	0.5895\\
0.8195	0.5889\\
0.8227	0.5884\\
0.8261	0.5878\\
0.8296	0.5871\\
0.833	0.5863\\
0.8365	0.5853\\
0.8399	0.5843\\
0.8434	0.5831\\
0.8468	0.5818\\
0.8502	0.5804\\
0.8537	0.5788\\
0.8571	0.5772\\
0.8606	0.5754\\
0.864	0.5736\\
0.8675	0.5716\\
0.8709	0.5695\\
0.8743	0.5672\\
0.8778	0.5649\\
0.8812	0.5624\\
0.8847	0.5598\\
0.8881	0.5571\\
0.8916	0.5543\\
0.895	0.5514\\
0.8985	0.5484\\
0.9019	0.5452\\
0.9053	0.5419\\
0.9088	0.5385\\
0.9122	0.535\\
0.9157	0.5314\\
0.9191	0.5277\\
0.9226	0.5238\\
0.9294	0.5158\\
0.9329	0.5116\\
0.9416	0.5005\\
0.9444	0.4966\\
};
\addplot [color=black, forget plot]
  table[row sep=crcr]{%
0.9444	0.4966\\
0.9492	0.4901\\
0.9539	0.4834\\
0.9586	0.4765\\
0.9633	0.4693\\
0.9681	0.462\\
0.9728	0.4544\\
0.9775	0.4466\\
0.9822	0.4385\\
0.9869	0.4303\\
0.9917	0.4218\\
0.9964	0.4132\\
1.0011	0.4043\\
1.0058	0.3952\\
1.0106	0.3858\\
1.0153	0.3763\\
1.02	0.3665\\
1.0247	0.3565\\
1.0294	0.3463\\
1.0342	0.3359\\
1.0389	0.3252\\
1.0436	0.3144\\
1.0483	0.3033\\
1.0531	0.292\\
1.0578	0.2804\\
1.0625	0.2687\\
1.0672	0.2567\\
1.0719	0.2446\\
1.0767	0.2322\\
1.0814	0.2195\\
1.0861	0.2067\\
1.0908	0.1937\\
1.0956	0.1804\\
1.1003	0.1669\\
1.105	0.1532\\
1.1097	0.1393\\
1.1144	0.1251\\
1.1192	0.1107\\
1.1239	0.0962\\
1.1286	0.0814\\
1.1333	0.0663\\
};
\addplot [color=black, forget plot]
  table[row sep=crcr]{%
1.1333	0.0663\\
1.1344	0.063\\
1.1354	0.0596\\
1.1365	0.0563\\
1.1375	0.0529\\
1.1415	0.0399\\
1.1455	0.0268\\
1.1494	0.0135\\
1.1534	-0\\
};
\addplot [color=black, forget plot]
  table[row sep=crcr]{%
1.1534	0\\
1.1534	0.0001\\
1.1535	0.0001\\
1.1535	0.0002\\
1.1536	0.0005\\
1.1537	0.0007\\
1.1538	0.001\\
1.1539	0.0012\\
1.1544	0.0025\\
1.1549	0.0037\\
1.1554	0.005\\
1.1559	0.0062\\
1.1583	0.0124\\
1.1608	0.0185\\
1.1632	0.0246\\
1.1657	0.0306\\
1.1699	0.0408\\
1.1741	0.0508\\
1.1784	0.0606\\
1.1826	0.0702\\
1.1868	0.0797\\
1.191	0.089\\
1.1952	0.0981\\
1.1995	0.1071\\
1.2037	0.1158\\
1.2079	0.1244\\
1.2121	0.1329\\
1.2163	0.1411\\
1.2206	0.1492\\
1.2248	0.1571\\
1.229	0.1648\\
1.2332	0.1724\\
1.2374	0.1797\\
1.2417	0.1869\\
1.2459	0.194\\
1.2501	0.2008\\
1.2543	0.2075\\
1.2585	0.214\\
1.2628	0.2203\\
1.267	0.2265\\
1.2712	0.2324\\
1.2754	0.2382\\
1.2796	0.2439\\
1.2839	0.2493\\
1.2881	0.2546\\
1.2923	0.2597\\
1.2965	0.2646\\
1.3007	0.2694\\
1.305	0.274\\
1.3092	0.2784\\
1.3134	0.2826\\
1.3176	0.2866\\
1.3188	0.2877\\
1.3199	0.2888\\
1.3211	0.2898\\
1.3222	0.2909\\
};
\addplot [color=black, forget plot]
  table[row sep=crcr]{%
1.3222	0.2909\\
1.3268	0.2949\\
1.3314	0.2987\\
1.336	0.3022\\
1.3406	0.3056\\
1.3453	0.3089\\
1.35	0.3119\\
1.3547	0.3148\\
1.3594	0.3174\\
1.3642	0.3198\\
1.3689	0.3219\\
1.3736	0.3239\\
1.3783	0.3256\\
1.3831	0.3272\\
1.3878	0.3285\\
1.3925	0.3295\\
1.3972	0.3304\\
1.4013	0.331\\
1.4053	0.3314\\
1.4094	0.3316\\
1.4135	0.3317\\
};
\addplot [color=black, forget plot]
  table[row sep=crcr]{%
1.4135	0.3317\\
1.416	0.3317\\
1.4167	0.3316\\
1.4215	0.3314\\
1.424	0.3311\\
1.4264	0.3309\\
1.4289	0.3305\\
1.4337	0.3297\\
1.4362	0.3292\\
1.4386	0.3286\\
1.4411	0.328\\
1.4435	0.3273\\
1.446	0.3265\\
1.4484	0.3257\\
1.4508	0.3248\\
1.4533	0.3239\\
1.4557	0.3229\\
1.4582	0.3219\\
1.4606	0.3208\\
1.463	0.3196\\
1.4655	0.3184\\
1.4679	0.3171\\
1.4704	0.3158\\
1.4752	0.313\\
1.4777	0.3115\\
1.4801	0.3099\\
1.4826	0.3083\\
1.485	0.3066\\
1.4875	0.3048\\
1.4923	0.3012\\
1.4948	0.2993\\
1.4972	0.2973\\
1.4997	0.2952\\
1.5021	0.2932\\
1.5045	0.291\\
1.5062	0.2895\\
1.5078	0.288\\
1.5095	0.2865\\
1.5111	0.2849\\
};
\addplot [color=black, forget plot]
  table[row sep=crcr]{%
1.5111	0.2849\\
1.5158	0.2803\\
1.5206	0.2754\\
1.5253	0.2704\\
1.53	0.2651\\
1.5347	0.2596\\
1.5394	0.2538\\
1.5442	0.2479\\
1.5489	0.2417\\
1.5536	0.2353\\
1.5583	0.2287\\
1.5631	0.2219\\
1.5678	0.2149\\
1.5725	0.2076\\
1.5772	0.2002\\
1.5819	0.1925\\
1.5867	0.1845\\
1.5914	0.1764\\
1.5961	0.1681\\
1.6008	0.1595\\
1.6056	0.1507\\
1.6103	0.1417\\
1.615	0.1325\\
1.6197	0.123\\
1.6244	0.1134\\
1.6292	0.1035\\
1.6339	0.0934\\
1.6386	0.083\\
1.6433	0.0725\\
1.6481	0.0617\\
1.6528	0.0508\\
1.6575	0.0396\\
1.6622	0.0282\\
1.665	0.0212\\
1.6679	0.0142\\
1.6707	0.0072\\
1.6735	-0\\
};
\addplot [color=black, forget plot]
  table[row sep=crcr]{%
1.6735	0\\
1.6735	0.0001\\
1.6736	0.0001\\
1.6736	0.0002\\
1.6737	0.0005\\
1.6739	0.0007\\
1.674	0.001\\
1.6741	0.0012\\
1.6748	0.0025\\
1.6754	0.0037\\
1.6761	0.0049\\
1.6768	0.0062\\
1.6774	0.0074\\
1.6781	0.0087\\
1.6787	0.0099\\
1.6794	0.0111\\
1.6801	0.0124\\
1.6807	0.0136\\
1.6821	0.016\\
1.6827	0.0172\\
1.6834	0.0184\\
1.684	0.0196\\
1.6854	0.022\\
1.686	0.0232\\
1.6874	0.0256\\
1.688	0.0267\\
1.6887	0.0279\\
1.6893	0.0291\\
1.69	0.0302\\
1.6907	0.0314\\
1.6913	0.0326\\
1.6927	0.0348\\
1.6933	0.036\\
1.694	0.0371\\
1.6946	0.0383\\
1.696	0.0405\\
1.6966	0.0416\\
1.698	0.0438\\
1.699	0.0456\\
1.7	0.0472\\
};
\addplot [color=black, forget plot]
  table[row sep=crcr]{%
0	0.95\\
0.0008	0.95\\
0.001	0.9499\\
0.0023	0.9499\\
0.0036	0.9498\\
0.0049	0.9496\\
0.0061	0.9495\\
0.0109	0.9489\\
0.0156	0.948\\
0.0203	0.947\\
0.025	0.9457\\
0.0298	0.9442\\
0.0345	0.9424\\
0.0392	0.9405\\
0.0439	0.9383\\
0.0486	0.936\\
0.0534	0.9334\\
0.0581	0.9305\\
0.0628	0.9275\\
0.0675	0.9243\\
0.0723	0.9208\\
0.077	0.9171\\
0.0817	0.9132\\
0.0864	0.909\\
0.0911	0.9047\\
0.0959	0.9001\\
0.1006	0.8953\\
0.1053	0.8903\\
0.11	0.8851\\
0.1148	0.8797\\
0.1195	0.874\\
0.1242	0.8681\\
0.1289	0.862\\
0.1336	0.8557\\
0.1384	0.8492\\
0.1431	0.8424\\
0.1478	0.8354\\
0.1525	0.8283\\
0.1573	0.8208\\
0.162	0.8132\\
0.1667	0.8054\\
0.1714	0.7973\\
0.1761	0.789\\
0.1793	0.7833\\
0.1825	0.7775\\
0.1857	0.7716\\
0.1889	0.7656\\
};
\addplot [color=black, forget plot]
  table[row sep=crcr]{%
0.1889	0.7656\\
0.1936	0.7565\\
0.1983	0.7471\\
0.2031	0.7376\\
0.2078	0.7279\\
0.2125	0.7179\\
0.2172	0.7077\\
0.2219	0.6973\\
0.2267	0.6867\\
0.2314	0.6758\\
0.2361	0.6647\\
0.2408	0.6535\\
0.2456	0.642\\
0.2503	0.6302\\
0.255	0.6183\\
0.2597	0.6061\\
0.2644	0.5938\\
0.2692	0.5812\\
0.2739	0.5684\\
0.2786	0.5553\\
0.2833	0.5421\\
0.2881	0.5286\\
0.2928	0.5149\\
0.2975	0.501\\
0.3022	0.4869\\
0.3069	0.4725\\
0.3117	0.458\\
0.3164	0.4432\\
0.3211	0.4282\\
0.3258	0.413\\
0.3306	0.3975\\
0.3353	0.3819\\
0.34	0.366\\
0.3447	0.3499\\
0.3494	0.3336\\
0.3542	0.317\\
0.3589	0.3003\\
0.3636	0.2833\\
0.3683	0.2661\\
0.3731	0.2487\\
0.3778	0.2311\\
};
\addplot [color=black, forget plot]
  table[row sep=crcr]{%
0.3778	0.2311\\
0.384	0.2077\\
0.3871	0.1958\\
0.3901	0.1839\\
0.3949	0.1655\\
0.3996	0.1468\\
0.4043	0.128\\
0.409	0.1089\\
0.4138	0.0896\\
0.4185	0.0701\\
0.4232	0.0504\\
0.4279	0.0304\\
0.4297	0.0229\\
0.4315	0.0153\\
0.4332	0.0077\\
0.435	-0\\
};
\addplot [color=black, forget plot]
  table[row sep=crcr]{%
0.435	0\\
0.435	0.0001\\
0.4351	0.0001\\
0.4351	0.0002\\
0.4352	0.0005\\
0.4352	0.0007\\
0.4353	0.001\\
0.4354	0.0012\\
0.4358	0.0025\\
0.4362	0.0037\\
0.4366	0.005\\
0.4369	0.0062\\
0.4389	0.0124\\
0.4408	0.0186\\
0.4428	0.0248\\
0.4447	0.0309\\
0.448	0.0412\\
0.4513	0.0514\\
0.4579	0.0714\\
0.4612	0.0813\\
0.4645	0.091\\
0.4677	0.1007\\
0.4743	0.1197\\
0.4809	0.1383\\
0.4842	0.1474\\
0.4875	0.1564\\
0.4908	0.1653\\
0.4941	0.1741\\
0.4974	0.1828\\
0.5007	0.1914\\
0.5039	0.1999\\
0.5072	0.2083\\
0.5138	0.2247\\
0.5171	0.2328\\
0.5237	0.2486\\
0.527	0.2563\\
0.5336	0.2715\\
0.5369	0.2789\\
0.5401	0.2862\\
0.5434	0.2934\\
0.5467	0.3005\\
0.55	0.3075\\
0.5533	0.3144\\
0.5566	0.3212\\
0.5599	0.3279\\
0.5632	0.3344\\
0.5641	0.3362\\
0.5649	0.3379\\
0.5667	0.3413\\
};
\addplot [color=black, forget plot]
  table[row sep=crcr]{%
0.5667	0.3413\\
0.5714	0.3504\\
0.5761	0.3592\\
0.5808	0.3679\\
0.5856	0.3763\\
0.5903	0.3845\\
0.595	0.3925\\
0.5997	0.4003\\
0.6044	0.4078\\
0.6092	0.4152\\
0.6139	0.4223\\
0.6186	0.4292\\
0.6233	0.4358\\
0.6281	0.4423\\
0.6328	0.4485\\
0.6375	0.4546\\
0.6422	0.4604\\
0.6469	0.466\\
0.6517	0.4713\\
0.6564	0.4765\\
0.6611	0.4814\\
0.6658	0.4861\\
0.6706	0.4906\\
0.6753	0.4949\\
0.68	0.4989\\
0.6847	0.5027\\
0.6894	0.5064\\
0.6942	0.5098\\
0.6989	0.5129\\
0.7036	0.5159\\
0.7083	0.5186\\
0.7131	0.5212\\
0.7178	0.5235\\
0.7225	0.5255\\
0.7272	0.5274\\
0.7319	0.5291\\
0.7367	0.5305\\
0.7414	0.5317\\
0.7461	0.5327\\
0.7508	0.5334\\
0.7556	0.534\\
};
\addplot [color=black, forget plot]
  table[row sep=crcr]{%
0.7556	0.534\\
0.756	0.534\\
0.7565	0.5341\\
0.757	0.5341\\
0.7575	0.5342\\
0.7613	0.5344\\
0.7651	0.5344\\
};
\addplot [color=black, forget plot]
  table[row sep=crcr]{%
0.7651	0.5344\\
0.7683	0.5344\\
0.7747	0.534\\
0.7811	0.5332\\
0.7856	0.5324\\
0.7901	0.5314\\
0.7946	0.5302\\
0.799	0.5288\\
0.8035	0.5272\\
0.808	0.5254\\
0.8125	0.5234\\
0.817	0.5213\\
0.8215	0.5189\\
0.8259	0.5163\\
0.8304	0.5135\\
0.8349	0.5106\\
0.8394	0.5074\\
0.8439	0.504\\
0.8484	0.5005\\
0.8528	0.4967\\
0.8573	0.4927\\
0.8618	0.4886\\
0.8663	0.4842\\
0.8708	0.4797\\
0.8753	0.4749\\
0.8797	0.47\\
0.8842	0.4649\\
0.8887	0.4595\\
0.8932	0.454\\
0.8977	0.4483\\
0.9022	0.4423\\
0.9066	0.4362\\
0.9111	0.4299\\
0.9156	0.4234\\
0.9201	0.4166\\
0.9246	0.4097\\
0.9291	0.4026\\
0.9335	0.3953\\
0.938	0.3878\\
0.9425	0.3801\\
0.943	0.3792\\
0.944	0.3776\\
0.9444	0.3767\\
};
\addplot [color=black, forget plot]
  table[row sep=crcr]{%
0.9444	0.3767\\
0.9492	0.3683\\
0.9539	0.3597\\
0.9586	0.3508\\
0.9633	0.3417\\
0.9681	0.3324\\
0.9728	0.3229\\
0.9775	0.3132\\
0.9822	0.3032\\
0.9869	0.2931\\
0.9917	0.2827\\
0.9964	0.2721\\
1.0011	0.2613\\
1.0058	0.2502\\
1.0106	0.239\\
1.0153	0.2275\\
1.02	0.2158\\
1.0247	0.2039\\
1.0294	0.1917\\
1.0342	0.1794\\
1.0389	0.1668\\
1.0436	0.154\\
1.0483	0.141\\
1.0531	0.1278\\
1.0578	0.1143\\
1.0625	0.1007\\
1.0672	0.0868\\
1.0767	0.0583\\
1.0813	0.0441\\
1.0859	0.0296\\
1.0906	0.0149\\
1.0952	-0\\
};
\addplot [color=black, forget plot]
  table[row sep=crcr]{%
1.0952	0\\
1.0952	0.0001\\
1.0953	0.0002\\
1.0954	0.0005\\
1.0955	0.0007\\
1.0956	0.001\\
1.0957	0.0012\\
1.0962	0.0025\\
1.0967	0.0037\\
1.0973	0.005\\
1.0978	0.0062\\
1.0987	0.0085\\
1.0997	0.0108\\
1.1006	0.013\\
1.1016	0.0153\\
1.1025	0.0175\\
1.1035	0.0198\\
1.1044	0.022\\
1.1054	0.0242\\
1.1063	0.0265\\
1.1083	0.0309\\
1.1092	0.033\\
1.1102	0.0352\\
1.1111	0.0374\\
1.1121	0.0396\\
1.113	0.0417\\
1.114	0.0439\\
1.1149	0.046\\
1.1159	0.0481\\
1.1168	0.0502\\
1.1178	0.0523\\
1.1187	0.0544\\
1.1197	0.0565\\
1.1206	0.0586\\
1.1226	0.0628\\
1.1235	0.0648\\
1.1245	0.0669\\
1.1254	0.0689\\
1.1264	0.0709\\
1.1273	0.0729\\
1.1283	0.0749\\
1.1292	0.0769\\
1.1302	0.0789\\
1.1311	0.0809\\
1.1324	0.0835\\
1.1327	0.0842\\
1.133	0.0848\\
1.1333	0.0855\\
};
\addplot [color=black, forget plot]
  table[row sep=crcr]{%
1.1333	0.0855\\
1.1354	0.0897\\
1.1375	0.094\\
1.1396	0.0982\\
1.1417	0.1023\\
1.1464	0.1115\\
1.1511	0.1205\\
1.1559	0.1293\\
1.1606	0.1378\\
1.1653	0.1461\\
1.17	0.1543\\
1.1747	0.1621\\
1.1795	0.1698\\
1.1842	0.1773\\
1.1889	0.1845\\
1.1936	0.1915\\
1.1984	0.1983\\
1.2031	0.2049\\
1.2078	0.2113\\
1.2125	0.2174\\
1.2172	0.2233\\
1.222	0.229\\
1.2267	0.2345\\
1.2314	0.2398\\
1.2361	0.2449\\
1.2409	0.2497\\
1.2456	0.2543\\
1.2503	0.2587\\
1.255	0.2629\\
1.2597	0.2668\\
1.2645	0.2706\\
1.2692	0.2741\\
1.2739	0.2774\\
1.2786	0.2805\\
1.2834	0.2833\\
1.2881	0.286\\
1.2928	0.2884\\
1.2975	0.2906\\
1.3022	0.2926\\
1.307	0.2943\\
1.3117	0.2959\\
1.3143	0.2967\\
1.317	0.2974\\
1.3222	0.2986\\
};
\addplot [color=black, forget plot]
  table[row sep=crcr]{%
1.3222	0.2986\\
1.3233	0.2988\\
1.3243	0.299\\
1.3253	0.2991\\
1.3264	0.2993\\
1.3305	0.2999\\
1.3346	0.3003\\
1.3387	0.3005\\
1.3428	0.3006\\
};
\addplot [color=black, forget plot]
  table[row sep=crcr]{%
1.3428	0.3006\\
1.346	0.3006\\
1.3524	0.3002\\
1.3588	0.2994\\
1.363	0.2986\\
1.3672	0.2977\\
1.3714	0.2966\\
1.3756	0.2953\\
1.3798	0.2939\\
1.384	0.2923\\
1.3882	0.2905\\
1.3924	0.2885\\
1.3966	0.2864\\
1.4009	0.2841\\
1.4051	0.2816\\
1.4093	0.2789\\
1.4135	0.2761\\
1.4177	0.2731\\
1.4219	0.2699\\
1.4261	0.2666\\
1.4303	0.263\\
1.4345	0.2593\\
1.4387	0.2555\\
1.4429	0.2514\\
1.4471	0.2472\\
1.4514	0.2428\\
1.4556	0.2382\\
1.4598	0.2335\\
1.464	0.2286\\
1.4682	0.2235\\
1.4724	0.2182\\
1.4766	0.2128\\
1.4808	0.2072\\
1.485	0.2014\\
1.4892	0.1954\\
1.4934	0.1893\\
1.4979	0.1827\\
1.5023	0.1758\\
1.5067	0.1688\\
1.5111	0.1616\\
};
\addplot [color=black, forget plot]
  table[row sep=crcr]{%
1.5111	0.1616\\
1.5158	0.1537\\
1.5206	0.1456\\
1.5253	0.1372\\
1.53	0.1287\\
1.5347	0.1199\\
1.5394	0.1109\\
1.5442	0.1017\\
1.5489	0.0922\\
1.5536	0.0826\\
1.5583	0.0727\\
1.5631	0.0626\\
1.5678	0.0523\\
1.5725	0.0418\\
1.5772	0.031\\
1.5819	0.02\\
1.5867	0.0089\\
1.5876	0.0067\\
1.5885	0.0044\\
1.5903	-0\\
};
\addplot [color=black, forget plot]
  table[row sep=crcr]{%
1.5903	0\\
1.5904	0.0001\\
1.5904	0.0002\\
1.5906	0.0005\\
1.5907	0.0007\\
1.5909	0.001\\
1.591	0.0012\\
1.5917	0.0025\\
1.5931	0.0049\\
1.5938	0.0062\\
1.5992	0.0158\\
1.602	0.0205\\
1.6047	0.0252\\
1.6075	0.0298\\
1.6102	0.0343\\
1.6129	0.0387\\
1.6157	0.043\\
1.6184	0.0473\\
1.6212	0.0515\\
1.6239	0.0556\\
1.6267	0.0597\\
1.6294	0.0637\\
1.6321	0.0676\\
1.6349	0.0714\\
1.6376	0.0752\\
1.6404	0.0788\\
1.6431	0.0825\\
1.6458	0.086\\
1.6486	0.0895\\
1.6513	0.0928\\
1.6541	0.0962\\
1.6568	0.0994\\
1.6596	0.1026\\
1.6623	0.1057\\
1.665	0.1087\\
1.6678	0.1116\\
1.6705	0.1145\\
1.6787	0.1227\\
1.6815	0.1253\\
1.6842	0.1278\\
1.687	0.1302\\
1.6897	0.1326\\
1.6925	0.1349\\
1.6943	0.1364\\
1.6962	0.1379\\
1.6981	0.1393\\
1.7	0.1408\\
};
\addplot [color=black, forget plot]
  table[row sep=crcr]{%
0	1.05\\
0.0008	1.05\\
0.001	1.0499\\
0.0023	1.0499\\
0.0036	1.0498\\
0.0049	1.0496\\
0.0061	1.0495\\
0.0109	1.0489\\
0.0156	1.048\\
0.0203	1.047\\
0.025	1.0457\\
0.0298	1.0442\\
0.0345	1.0424\\
0.0392	1.0405\\
0.0439	1.0383\\
0.0486	1.036\\
0.0534	1.0334\\
0.0581	1.0305\\
0.0628	1.0275\\
0.0675	1.0243\\
0.0723	1.0208\\
0.077	1.0171\\
0.0817	1.0132\\
0.0864	1.009\\
0.0911	1.0047\\
0.0959	1.0001\\
0.1006	0.9953\\
0.1053	0.9903\\
0.11	0.9851\\
0.1148	0.9797\\
0.1195	0.974\\
0.1242	0.9681\\
0.1289	0.962\\
0.1336	0.9557\\
0.1384	0.9492\\
0.1431	0.9424\\
0.1478	0.9354\\
0.1525	0.9283\\
0.1573	0.9208\\
0.162	0.9132\\
0.1667	0.9054\\
0.1714	0.8973\\
0.1761	0.889\\
0.1793	0.8833\\
0.1825	0.8775\\
0.1857	0.8716\\
0.1889	0.8656\\
};
\addplot [color=black, forget plot]
  table[row sep=crcr]{%
0.1889	0.8656\\
0.1936	0.8565\\
0.1983	0.8471\\
0.2031	0.8376\\
0.2078	0.8279\\
0.2125	0.8179\\
0.2172	0.8077\\
0.2219	0.7973\\
0.2267	0.7867\\
0.2314	0.7758\\
0.2361	0.7647\\
0.2408	0.7535\\
0.2456	0.742\\
0.2503	0.7302\\
0.255	0.7183\\
0.2597	0.7061\\
0.2644	0.6938\\
0.2692	0.6812\\
0.2739	0.6684\\
0.2786	0.6553\\
0.2833	0.6421\\
0.2881	0.6286\\
0.2928	0.6149\\
0.2975	0.601\\
0.3022	0.5869\\
0.3069	0.5725\\
0.3117	0.558\\
0.3164	0.5432\\
0.3211	0.5282\\
0.3258	0.513\\
0.3306	0.4975\\
0.3353	0.4819\\
0.34	0.466\\
0.3447	0.4499\\
0.3494	0.4336\\
0.3542	0.417\\
0.3589	0.4003\\
0.3636	0.3833\\
0.3683	0.3661\\
0.3731	0.3487\\
0.3778	0.3311\\
};
\addplot [color=black, forget plot]
  table[row sep=crcr]{%
0.3778	0.3311\\
0.3822	0.3144\\
0.3866	0.2974\\
0.3911	0.2803\\
0.3955	0.263\\
0.4002	0.2444\\
0.4049	0.2255\\
0.4097	0.2064\\
0.4144	0.187\\
0.4191	0.1675\\
0.4238	0.1477\\
0.4285	0.1278\\
0.4333	0.1076\\
0.438	0.0871\\
0.4427	0.0665\\
0.4474	0.0457\\
0.4522	0.0246\\
0.4535	0.0185\\
0.4549	0.0123\\
0.4562	0.0062\\
0.4576	-0\\
};
\addplot [color=black, forget plot]
  table[row sep=crcr]{%
0.4576	0\\
0.4576	0.0002\\
0.4577	0.0002\\
0.4577	0.0005\\
0.4578	0.0007\\
0.4579	0.001\\
0.458	0.0012\\
0.4583	0.0025\\
0.4587	0.0037\\
0.4591	0.005\\
0.4594	0.0062\\
0.4613	0.0124\\
0.4631	0.0186\\
0.465	0.0248\\
0.4668	0.0309\\
0.4695	0.0399\\
0.4723	0.0489\\
0.4777	0.0665\\
0.4804	0.0752\\
0.4832	0.0838\\
0.4859	0.0924\\
0.4886	0.1009\\
0.4914	0.1093\\
0.4968	0.1259\\
0.4995	0.1341\\
0.5023	0.1422\\
0.505	0.1503\\
0.5077	0.1583\\
0.5104	0.1662\\
0.5132	0.174\\
0.5159	0.1818\\
0.5186	0.1894\\
0.5213	0.1971\\
0.5241	0.2046\\
0.5268	0.2121\\
0.5295	0.2195\\
0.5323	0.2268\\
0.5377	0.2412\\
0.5404	0.2483\\
0.5432	0.2553\\
0.5459	0.2623\\
0.5486	0.2692\\
0.5513	0.276\\
0.5541	0.2827\\
0.5568	0.2894\\
0.5595	0.296\\
0.5622	0.3025\\
0.565	0.309\\
0.5658	0.311\\
0.5662	0.3119\\
0.5667	0.3129\\
};
\addplot [color=black, forget plot]
  table[row sep=crcr]{%
0.5667	0.3129\\
0.5714	0.3239\\
0.5761	0.3345\\
0.5808	0.345\\
0.5856	0.3553\\
0.5903	0.3653\\
0.595	0.3751\\
0.5997	0.3847\\
0.6044	0.3941\\
0.6092	0.4033\\
0.6139	0.4122\\
0.6186	0.421\\
0.6233	0.4295\\
0.6281	0.4378\\
0.6328	0.4458\\
0.6375	0.4537\\
0.6422	0.4613\\
0.6469	0.4687\\
0.6517	0.4759\\
0.6564	0.4829\\
0.6611	0.4897\\
0.6658	0.4962\\
0.6706	0.5025\\
0.6753	0.5086\\
0.68	0.5145\\
0.6847	0.5202\\
0.6894	0.5256\\
0.6942	0.5308\\
0.6989	0.5359\\
0.7036	0.5406\\
0.7083	0.5452\\
0.7131	0.5496\\
0.7178	0.5537\\
0.7225	0.5576\\
0.7272	0.5613\\
0.7319	0.5648\\
0.7367	0.568\\
0.7414	0.5711\\
0.7461	0.5739\\
0.7508	0.5765\\
0.7556	0.5789\\
};
\addplot [color=black, forget plot]
  table[row sep=crcr]{%
0.7556	0.5789\\
0.758	0.58\\
0.763	0.5822\\
0.7654	0.5832\\
0.7701	0.5849\\
0.7749	0.5863\\
0.7796	0.5876\\
0.7843	0.5887\\
0.789	0.5895\\
0.7938	0.5901\\
0.7985	0.5905\\
0.8032	0.5907\\
0.8046	0.5907\\
};
\addplot [color=black, forget plot]
  table[row sep=crcr]{%
0.8046	0.5907\\
0.8072	0.5907\\
0.8078	0.5906\\
0.811	0.5905\\
0.8174	0.5899\\
0.8206	0.5894\\
0.8241	0.5888\\
0.8276	0.5881\\
0.8311	0.5873\\
0.8346	0.5863\\
0.8381	0.5852\\
0.8416	0.584\\
0.8451	0.5827\\
0.8521	0.5797\\
0.8556	0.578\\
0.8591	0.5762\\
0.8661	0.5722\\
0.8696	0.57\\
0.8731	0.5677\\
0.8766	0.5653\\
0.8801	0.5628\\
0.8835	0.5602\\
0.887	0.5574\\
0.8905	0.5545\\
0.894	0.5515\\
0.8975	0.5484\\
0.9045	0.5418\\
0.908	0.5383\\
0.9115	0.5347\\
0.9185	0.5271\\
0.9255	0.5191\\
0.9325	0.5105\\
0.9355	0.5067\\
0.9415	0.4989\\
0.9444	0.4948\\
};
\addplot [color=black, forget plot]
  table[row sep=crcr]{%
0.9444	0.4948\\
0.9492	0.4882\\
0.9539	0.4814\\
0.9586	0.4744\\
0.9633	0.4672\\
0.9681	0.4597\\
0.9728	0.452\\
0.9775	0.4441\\
0.9822	0.436\\
0.9869	0.4277\\
0.9917	0.4191\\
0.9964	0.4103\\
1.0011	0.4013\\
1.0058	0.3921\\
1.0106	0.3827\\
1.0153	0.3731\\
1.02	0.3632\\
1.0247	0.3531\\
1.0294	0.3428\\
1.0342	0.3323\\
1.0389	0.3215\\
1.0436	0.3106\\
1.0483	0.2994\\
1.0531	0.288\\
1.0578	0.2764\\
1.0625	0.2645\\
1.0672	0.2525\\
1.0719	0.2402\\
1.0767	0.2277\\
1.0814	0.215\\
1.0861	0.2021\\
1.0908	0.1889\\
1.0956	0.1756\\
1.1003	0.162\\
1.105	0.1482\\
1.1097	0.1341\\
1.1144	0.1199\\
1.1192	0.1054\\
1.1239	0.0908\\
1.1286	0.0759\\
1.1333	0.0607\\
};
\addplot [color=black, forget plot]
  table[row sep=crcr]{%
1.1333	0.0607\\
1.1343	0.0577\\
1.1352	0.0546\\
1.1362	0.0515\\
1.1371	0.0485\\
1.1408	0.0365\\
1.1444	0.0245\\
1.148	0.0123\\
1.1517	0\\
};
\addplot [color=black, forget plot]
  table[row sep=crcr]{%
1.1517	0\\
1.1517	0.0002\\
1.1518	0.0005\\
1.1519	0.0007\\
1.152	0.001\\
1.1521	0.0012\\
1.1526	0.0025\\
1.1531	0.0037\\
1.1536	0.005\\
1.1541	0.0062\\
1.1566	0.0124\\
1.159	0.0185\\
1.1615	0.0246\\
1.1639	0.0306\\
1.1682	0.0409\\
1.1725	0.051\\
1.1767	0.0609\\
1.181	0.0707\\
1.1853	0.0802\\
1.1895	0.0896\\
1.1938	0.0989\\
1.198	0.1079\\
1.2023	0.1167\\
1.2066	0.1254\\
1.2108	0.1339\\
1.2151	0.1423\\
1.2194	0.1504\\
1.2236	0.1584\\
1.2322	0.1738\\
1.2364	0.1812\\
1.2407	0.1884\\
1.245	0.1955\\
1.2492	0.2024\\
1.2535	0.2091\\
1.2577	0.2157\\
1.262	0.222\\
1.2663	0.2282\\
1.2705	0.2342\\
1.2748	0.24\\
1.2791	0.2457\\
1.2833	0.2512\\
1.2919	0.2616\\
1.2961	0.2665\\
1.3004	0.2712\\
1.3047	0.2758\\
1.3089	0.2802\\
1.3132	0.2844\\
1.3174	0.2885\\
1.3198	0.2907\\
1.321	0.2917\\
1.3222	0.2928\\
};
\addplot [color=black, forget plot]
  table[row sep=crcr]{%
1.3222	0.2928\\
1.3267	0.2967\\
1.3312	0.3003\\
1.3357	0.3038\\
1.3402	0.3071\\
1.345	0.3103\\
1.3497	0.3133\\
1.3544	0.316\\
1.3591	0.3186\\
1.3639	0.3209\\
1.3686	0.3231\\
1.3733	0.3249\\
1.378	0.3266\\
1.3827	0.3281\\
1.3875	0.3293\\
1.3922	0.3304\\
1.3969	0.3312\\
1.4007	0.3316\\
1.4044	0.332\\
1.4082	0.3322\\
1.4119	0.3323\\
};
\addplot [color=black, forget plot]
  table[row sep=crcr]{%
1.4119	0.3323\\
1.4132	0.3323\\
1.4138	0.3322\\
1.4151	0.3322\\
1.4176	0.3321\\
1.4226	0.3317\\
1.425	0.3314\\
1.4275	0.3311\\
1.43	0.3307\\
1.435	0.3297\\
1.4374	0.3291\\
1.4424	0.3277\\
1.4474	0.3261\\
1.4498	0.3252\\
1.4523	0.3243\\
1.4548	0.3233\\
1.4573	0.3222\\
1.4598	0.321\\
1.4622	0.3199\\
1.4672	0.3173\\
1.4722	0.3145\\
1.4746	0.313\\
1.4796	0.3098\\
1.4846	0.3064\\
1.487	0.3046\\
1.492	0.3008\\
1.4945	0.2988\\
1.4969	0.2968\\
1.5019	0.2926\\
1.5044	0.2903\\
1.5061	0.2888\\
1.5077	0.2872\\
1.5111	0.284\\
};
\addplot [color=black, forget plot]
  table[row sep=crcr]{%
1.5111	0.284\\
1.5158	0.2793\\
1.5206	0.2744\\
1.5253	0.2692\\
1.53	0.2639\\
1.5347	0.2583\\
1.5394	0.2525\\
1.5442	0.2465\\
1.5489	0.2403\\
1.5536	0.2338\\
1.5583	0.2271\\
1.5631	0.2202\\
1.5678	0.2131\\
1.5725	0.2058\\
1.5772	0.1983\\
1.5819	0.1905\\
1.5867	0.1825\\
1.5914	0.1743\\
1.5961	0.1659\\
1.6008	0.1572\\
1.6056	0.1484\\
1.6103	0.1393\\
1.615	0.13\\
1.6197	0.1205\\
1.6244	0.1107\\
1.6292	0.1008\\
1.6339	0.0906\\
1.6386	0.0802\\
1.6433	0.0696\\
1.6481	0.0588\\
1.6528	0.0477\\
1.6575	0.0365\\
1.6622	0.025\\
1.6672	0.0126\\
1.6722	-0\\
};
\addplot [color=black, forget plot]
  table[row sep=crcr]{%
1.6722	0\\
1.6722	0.0001\\
1.6723	0.0001\\
1.6723	0.0002\\
1.6724	0.0005\\
1.6726	0.0007\\
1.6727	0.001\\
1.6728	0.0012\\
1.6735	0.0025\\
1.6741	0.0037\\
1.6748	0.0049\\
1.6755	0.0062\\
1.6761	0.0075\\
1.6789	0.0127\\
1.6796	0.0139\\
1.6817	0.0178\\
1.6824	0.019\\
1.6831	0.0203\\
1.6838	0.0215\\
1.6845	0.0228\\
1.6852	0.024\\
1.6859	0.0253\\
1.6873	0.0277\\
1.688	0.029\\
1.6894	0.0314\\
1.69	0.0326\\
1.6949	0.041\\
1.6956	0.0421\\
1.697	0.0445\\
1.6977	0.0456\\
1.6983	0.0466\\
1.6988	0.0475\\
1.6994	0.0485\\
1.7	0.0494\\
};
\addplot [color=black, forget plot]
  table[row sep=crcr]{%
0	1.05\\
0.001	1.05\\
0.002	1.0501\\
0.0051	1.0501\\
};
\addplot [color=black, forget plot]
  table[row sep=crcr]{%
0.0051	1.0501\\
0.0083	1.0501\\
0.0147	1.0497\\
0.0211	1.0489\\
0.0257	1.048\\
0.0303	1.047\\
0.0349	1.0458\\
0.0395	1.0443\\
0.0441	1.0427\\
0.0487	1.0408\\
0.0533	1.0387\\
0.0579	1.0365\\
0.0624	1.034\\
0.067	1.0313\\
0.0716	1.0284\\
0.0762	1.0253\\
0.0808	1.022\\
0.0854	1.0185\\
0.09	1.0148\\
0.0946	1.0108\\
0.0992	1.0067\\
0.1038	1.0023\\
0.1084	0.9978\\
0.113	0.993\\
0.1176	0.9881\\
0.1222	0.9829\\
0.1268	0.9775\\
0.1314	0.9719\\
0.136	0.9661\\
0.1406	0.9601\\
0.1452	0.9539\\
0.1497	0.9475\\
0.1543	0.9409\\
0.1589	0.934\\
0.1635	0.927\\
0.1681	0.9198\\
0.1727	0.9123\\
0.1773	0.9046\\
0.1819	0.8968\\
0.1865	0.8887\\
0.1871	0.8876\\
0.1877	0.8866\\
0.1889	0.8844\\
};
\addplot [color=black, forget plot]
  table[row sep=crcr]{%
0.1889	0.8844\\
0.1936	0.8758\\
0.1983	0.867\\
0.2031	0.8579\\
0.2078	0.8486\\
0.2125	0.8391\\
0.2172	0.8294\\
0.2219	0.8195\\
0.2267	0.8093\\
0.2314	0.799\\
0.2361	0.7884\\
0.2408	0.7775\\
0.2456	0.7665\\
0.2503	0.7553\\
0.255	0.7438\\
0.2597	0.7321\\
0.2644	0.7202\\
0.2692	0.7081\\
0.2739	0.6957\\
0.2786	0.6832\\
0.2833	0.6704\\
0.2881	0.6574\\
0.2928	0.6442\\
0.2975	0.6308\\
0.3022	0.6171\\
0.3069	0.6032\\
0.3117	0.5891\\
0.3164	0.5748\\
0.3211	0.5603\\
0.3258	0.5455\\
0.3306	0.5306\\
0.3353	0.5154\\
0.34	0.5\\
0.3447	0.4844\\
0.3494	0.4685\\
0.3542	0.4525\\
0.3589	0.4362\\
0.3636	0.4197\\
0.3683	0.403\\
0.3731	0.386\\
0.3778	0.3689\\
};
\addplot [color=black, forget plot]
  table[row sep=crcr]{%
0.3778	0.3689\\
0.3825	0.3515\\
0.3872	0.3339\\
0.3919	0.3161\\
0.3967	0.2981\\
0.4014	0.2798\\
0.4061	0.2613\\
0.4108	0.2427\\
0.4156	0.2238\\
0.4203	0.2046\\
0.425	0.1853\\
0.4297	0.1657\\
0.4344	0.1459\\
0.4392	0.1259\\
0.4439	0.1057\\
0.4486	0.0853\\
0.4533	0.0646\\
0.4569	0.0487\\
0.4606	0.0326\\
0.4642	0.0164\\
0.4678	-0\\
};
\addplot [color=black, forget plot]
  table[row sep=crcr]{%
0.4678	0\\
0.4678	0.0002\\
0.4679	0.0002\\
0.4679	0.0005\\
0.468	0.0007\\
0.4681	0.001\\
0.4682	0.0012\\
0.4685	0.0025\\
0.4689	0.0037\\
0.4693	0.005\\
0.4696	0.0062\\
0.4715	0.0124\\
0.4733	0.0186\\
0.4752	0.0248\\
0.477	0.0309\\
0.4795	0.0391\\
0.482	0.0472\\
0.4844	0.0552\\
0.4869	0.0632\\
0.4894	0.0711\\
0.4918	0.079\\
0.4943	0.0868\\
0.4993	0.1022\\
0.5017	0.1098\\
0.5042	0.1174\\
0.5067	0.1249\\
0.5091	0.1324\\
0.5116	0.1397\\
0.5141	0.1471\\
0.5166	0.1543\\
0.519	0.1615\\
0.5215	0.1687\\
0.524	0.1757\\
0.5264	0.1828\\
0.5339	0.2035\\
0.5363	0.2103\\
0.5388	0.217\\
0.5413	0.2236\\
0.5437	0.2302\\
0.5462	0.2368\\
0.5487	0.2433\\
0.5512	0.2497\\
0.5536	0.2561\\
0.5561	0.2624\\
0.5587	0.269\\
0.5614	0.2756\\
0.564	0.2822\\
0.5667	0.2886\\
};
\addplot [color=black, forget plot]
  table[row sep=crcr]{%
0.5667	0.2886\\
0.5714	0.3\\
0.5761	0.3112\\
0.5808	0.3221\\
0.5856	0.3329\\
0.5903	0.3434\\
0.595	0.3537\\
0.5997	0.3637\\
0.6044	0.3736\\
0.6092	0.3832\\
0.6139	0.3927\\
0.6186	0.4019\\
0.6233	0.4108\\
0.6281	0.4196\\
0.6328	0.4281\\
0.6375	0.4365\\
0.6422	0.4446\\
0.6469	0.4525\\
0.6517	0.4601\\
0.6564	0.4676\\
0.6611	0.4748\\
0.6658	0.4818\\
0.6706	0.4886\\
0.6753	0.4952\\
0.68	0.5015\\
0.6847	0.5077\\
0.6894	0.5136\\
0.6942	0.5193\\
0.6989	0.5248\\
0.7036	0.53\\
0.7083	0.5351\\
0.7131	0.5399\\
0.7178	0.5445\\
0.7225	0.5489\\
0.7272	0.5531\\
0.7319	0.557\\
0.7367	0.5607\\
0.7414	0.5642\\
0.7461	0.5675\\
0.7508	0.5706\\
0.7556	0.5735\\
};
\addplot [color=black, forget plot]
  table[row sep=crcr]{%
0.7556	0.5735\\
0.7585	0.5752\\
0.7615	0.5768\\
0.7645	0.5783\\
0.7675	0.5797\\
0.7722	0.5818\\
0.7769	0.5836\\
0.7816	0.5853\\
0.7864	0.5867\\
0.7911	0.5879\\
0.7958	0.5889\\
0.8005	0.5897\\
0.8052	0.5902\\
0.81	0.5906\\
0.8124	0.5907\\
0.8148	0.5907\\
};
\addplot [color=black, forget plot]
  table[row sep=crcr]{%
0.8148	0.5907\\
0.8174	0.5907\\
0.818	0.5906\\
0.8212	0.5905\\
0.8276	0.5899\\
0.8308	0.5894\\
0.8341	0.5889\\
0.8405	0.5875\\
0.8438	0.5866\\
0.847	0.5856\\
0.8503	0.5845\\
0.8535	0.5834\\
0.8567	0.5821\\
0.86	0.5807\\
0.8632	0.5792\\
0.8665	0.5776\\
0.8697	0.5759\\
0.8729	0.5741\\
0.8762	0.5722\\
0.8794	0.5702\\
0.8827	0.5681\\
0.8859	0.5659\\
0.8892	0.5636\\
0.8924	0.5612\\
0.8956	0.5587\\
0.8989	0.556\\
0.9021	0.5533\\
0.9054	0.5505\\
0.9086	0.5476\\
0.9118	0.5445\\
0.9151	0.5414\\
0.9183	0.5382\\
0.9216	0.5348\\
0.9248	0.5314\\
0.928	0.5278\\
0.9313	0.5242\\
0.9345	0.5204\\
0.937	0.5175\\
0.9395	0.5145\\
0.942	0.5114\\
0.9444	0.5083\\
};
\addplot [color=black, forget plot]
  table[row sep=crcr]{%
0.9444	0.5083\\
0.9492	0.5022\\
0.9539	0.4958\\
0.9586	0.4893\\
0.9633	0.4825\\
0.9681	0.4755\\
0.9728	0.4683\\
0.9775	0.4609\\
0.9822	0.4533\\
0.9869	0.4454\\
0.9917	0.4373\\
0.9964	0.429\\
1.0011	0.4205\\
1.0058	0.4117\\
1.0106	0.4028\\
1.0153	0.3936\\
1.02	0.3842\\
1.0247	0.3746\\
1.0294	0.3648\\
1.0342	0.3547\\
1.0389	0.3444\\
1.0436	0.334\\
1.0483	0.3232\\
1.0531	0.3123\\
1.0578	0.3012\\
1.0625	0.2898\\
1.0672	0.2782\\
1.0719	0.2664\\
1.0767	0.2544\\
1.0814	0.2422\\
1.0861	0.2297\\
1.0908	0.217\\
1.0956	0.2041\\
1.1003	0.191\\
1.105	0.1777\\
1.1097	0.1641\\
1.1144	0.1504\\
1.1192	0.1364\\
1.1239	0.1222\\
1.1286	0.1077\\
1.1333	0.0931\\
};
\addplot [color=black, forget plot]
  table[row sep=crcr]{%
1.1333	0.0931\\
1.1378	0.079\\
1.1393	0.0742\\
1.144	0.0591\\
1.1488	0.0437\\
1.1535	0.0281\\
1.1582	0.0123\\
1.1591	0.0093\\
1.1609	0.0031\\
1.1619	-0\\
};
\addplot [color=black, forget plot]
  table[row sep=crcr]{%
1.1619	0\\
1.1619	0.0002\\
1.162	0.0005\\
1.1621	0.0007\\
1.1622	0.001\\
1.1623	0.0012\\
1.1628	0.0025\\
1.1633	0.0037\\
1.1638	0.005\\
1.1643	0.0062\\
1.1668	0.0124\\
1.1692	0.0185\\
1.1717	0.0246\\
1.1741	0.0306\\
1.1781	0.0403\\
1.1821	0.0498\\
1.1862	0.0592\\
1.1902	0.0684\\
1.1942	0.0774\\
1.1982	0.0863\\
1.2022	0.095\\
1.2062	0.1036\\
1.2102	0.112\\
1.2142	0.1203\\
1.2182	0.1284\\
1.2222	0.1363\\
1.2263	0.1441\\
1.2303	0.1517\\
1.2343	0.1592\\
1.2383	0.1665\\
1.2423	0.1736\\
1.2463	0.1806\\
1.2503	0.1875\\
1.2583	0.2007\\
1.2623	0.207\\
1.2663	0.2132\\
1.2704	0.2193\\
1.2744	0.2252\\
1.2784	0.2309\\
1.2824	0.2365\\
1.2864	0.2419\\
1.2904	0.2472\\
1.2944	0.2523\\
1.2984	0.2572\\
1.3024	0.262\\
1.3064	0.2666\\
1.3104	0.2711\\
1.3145	0.2754\\
1.3185	0.2796\\
1.3194	0.2805\\
1.3203	0.2815\\
1.3213	0.2824\\
1.3222	0.2833\\
};
\addplot [color=black, forget plot]
  table[row sep=crcr]{%
1.3222	0.2833\\
1.3269	0.2878\\
1.3317	0.2921\\
1.3364	0.2962\\
1.3411	0.3001\\
1.3458	0.3037\\
1.3506	0.3071\\
1.3553	0.3104\\
1.36	0.3133\\
1.3647	0.3161\\
1.3694	0.3187\\
1.3742	0.321\\
1.3789	0.3231\\
1.3836	0.325\\
1.3883	0.3267\\
1.3931	0.3281\\
1.3978	0.3294\\
1.4025	0.3304\\
1.4072	0.3312\\
1.4119	0.3318\\
1.4167	0.3321\\
1.418	0.3322\\
1.4194	0.3322\\
1.4208	0.3323\\
1.4221	0.3323\\
};
\addplot [color=black, forget plot]
  table[row sep=crcr]{%
1.4221	0.3323\\
1.4234	0.3323\\
1.424	0.3322\\
1.4253	0.3322\\
1.4275	0.3321\\
1.4298	0.332\\
1.432	0.3318\\
1.4342	0.3315\\
1.4364	0.3313\\
1.4387	0.3309\\
1.4431	0.3301\\
1.4453	0.3296\\
1.4476	0.3291\\
1.452	0.3279\\
1.4542	0.3272\\
1.4565	0.3265\\
1.4609	0.3249\\
1.4631	0.324\\
1.4654	0.3231\\
1.4698	0.3211\\
1.472	0.32\\
1.4743	0.3189\\
1.4765	0.3178\\
1.4787	0.3166\\
1.4809	0.3153\\
1.4832	0.314\\
1.4898	0.3098\\
1.4921	0.3083\\
1.4987	0.3035\\
1.501	0.3018\\
1.5054	0.2982\\
1.5068	0.2971\\
1.5083	0.2959\\
1.5097	0.2947\\
1.5111	0.2934\\
};
\addplot [color=black, forget plot]
  table[row sep=crcr]{%
1.5111	0.2934\\
1.5156	0.2894\\
1.5201	0.2852\\
1.5245	0.2808\\
1.529	0.2762\\
1.5337	0.2712\\
1.5384	0.2659\\
1.5432	0.2604\\
1.5479	0.2547\\
1.5526	0.2488\\
1.5573	0.2426\\
1.562	0.2362\\
1.5668	0.2296\\
1.5715	0.2228\\
1.5762	0.2158\\
1.5809	0.2086\\
1.5857	0.2011\\
1.5904	0.1934\\
1.5951	0.1855\\
1.5998	0.1774\\
1.6045	0.169\\
1.6093	0.1605\\
1.614	0.1517\\
1.6187	0.1427\\
1.6234	0.1335\\
1.6282	0.124\\
1.6329	0.1144\\
1.6376	0.1045\\
1.6423	0.0944\\
1.647	0.0841\\
1.6518	0.0736\\
1.6565	0.0628\\
1.6612	0.0519\\
1.6659	0.0407\\
1.6707	0.0293\\
1.6754	0.0177\\
1.6801	0.0058\\
1.6807	0.0044\\
1.6812	0.0029\\
1.6818	0.0015\\
1.6824	-0\\
};
\addplot [color=black, forget plot]
  table[row sep=crcr]{%
1.6824	0\\
1.6824	0.0001\\
1.6825	0.0002\\
1.6826	0.0005\\
1.6828	0.0007\\
1.6829	0.001\\
1.683	0.0012\\
1.6835	0.002\\
1.6839	0.0029\\
1.6843	0.0037\\
1.6848	0.0045\\
1.6852	0.0054\\
1.6857	0.0062\\
1.6861	0.007\\
1.6865	0.0079\\
1.687	0.0087\\
1.6874	0.0095\\
1.6879	0.0103\\
1.6883	0.0111\\
1.6887	0.012\\
1.6892	0.0128\\
1.6896	0.0136\\
1.6901	0.0144\\
1.6909	0.016\\
1.6914	0.0168\\
1.6918	0.0176\\
1.6923	0.0184\\
1.6931	0.02\\
1.6936	0.0208\\
1.694	0.0216\\
1.6945	0.0224\\
1.6953	0.024\\
1.6958	0.0248\\
1.6962	0.0256\\
1.6967	0.0263\\
1.6975	0.0279\\
1.698	0.0287\\
1.6984	0.0294\\
1.6989	0.0302\\
1.6992	0.0307\\
1.6994	0.0312\\
1.7	0.0322\\
};
\addplot [color=black, forget plot]
  table[row sep=crcr]{%
0	0.95\\
0.001	0.95\\
0.002	0.9501\\
0.0051	0.9501\\
};
\addplot [color=black, forget plot]
  table[row sep=crcr]{%
0.0051	0.9501\\
0.0083	0.9501\\
0.0147	0.9497\\
0.0211	0.9489\\
0.0257	0.948\\
0.0303	0.947\\
0.0349	0.9458\\
0.0395	0.9443\\
0.0441	0.9427\\
0.0487	0.9408\\
0.0533	0.9387\\
0.0579	0.9365\\
0.0624	0.934\\
0.067	0.9313\\
0.0716	0.9284\\
0.0762	0.9253\\
0.0808	0.922\\
0.0854	0.9185\\
0.09	0.9148\\
0.0946	0.9108\\
0.0992	0.9067\\
0.1038	0.9023\\
0.1084	0.8978\\
0.113	0.893\\
0.1176	0.8881\\
0.1222	0.8829\\
0.1268	0.8775\\
0.1314	0.8719\\
0.136	0.8661\\
0.1406	0.8601\\
0.1452	0.8539\\
0.1497	0.8475\\
0.1543	0.8409\\
0.1589	0.834\\
0.1635	0.827\\
0.1681	0.8198\\
0.1727	0.8123\\
0.1773	0.8046\\
0.1819	0.7968\\
0.1865	0.7887\\
0.1871	0.7876\\
0.1877	0.7866\\
0.1889	0.7844\\
};
\addplot [color=black, forget plot]
  table[row sep=crcr]{%
0.1889	0.7844\\
0.1936	0.7758\\
0.1983	0.767\\
0.2031	0.7579\\
0.2078	0.7486\\
0.2125	0.7391\\
0.2172	0.7294\\
0.2219	0.7195\\
0.2267	0.7093\\
0.2314	0.699\\
0.2361	0.6884\\
0.2408	0.6775\\
0.2456	0.6665\\
0.2503	0.6553\\
0.255	0.6438\\
0.2597	0.6321\\
0.2644	0.6202\\
0.2692	0.6081\\
0.2739	0.5957\\
0.2786	0.5832\\
0.2833	0.5704\\
0.2881	0.5574\\
0.2928	0.5442\\
0.2975	0.5308\\
0.3022	0.5171\\
0.3069	0.5032\\
0.3117	0.4891\\
0.3164	0.4748\\
0.3211	0.4603\\
0.3258	0.4455\\
0.3306	0.4306\\
0.3353	0.4154\\
0.34	0.4\\
0.3447	0.3844\\
0.3494	0.3685\\
0.3542	0.3525\\
0.3589	0.3362\\
0.3636	0.3197\\
0.3683	0.303\\
0.3731	0.286\\
0.3778	0.2689\\
};
\addplot [color=black, forget plot]
  table[row sep=crcr]{%
0.3778	0.2689\\
0.3815	0.2553\\
0.3852	0.2416\\
0.3926	0.2138\\
0.3973	0.1957\\
0.402	0.1774\\
0.4067	0.1589\\
0.4114	0.1402\\
0.4162	0.1213\\
0.4209	0.1021\\
0.4256	0.0828\\
0.4303	0.0632\\
0.4341	0.0476\\
0.4378	0.0319\\
0.4415	0.016\\
0.4452	0\\
};
\addplot [color=black, forget plot]
  table[row sep=crcr]{%
0.4452	0\\
0.4452	0.0001\\
0.4453	0.0002\\
0.4454	0.0005\\
0.4454	0.0007\\
0.4455	0.001\\
0.4456	0.0012\\
0.446	0.0025\\
0.4464	0.0037\\
0.4468	0.005\\
0.4471	0.0062\\
0.4491	0.0124\\
0.451	0.0186\\
0.453	0.0248\\
0.4549	0.0309\\
0.4579	0.0404\\
0.461	0.0498\\
0.464	0.0591\\
0.467	0.0683\\
0.4701	0.0775\\
0.4731	0.0865\\
0.4762	0.0955\\
0.4822	0.1131\\
0.4853	0.1218\\
0.4883	0.1304\\
0.4913	0.1389\\
0.4944	0.1473\\
0.5004	0.1639\\
0.5035	0.172\\
0.5065	0.1801\\
0.5096	0.188\\
0.5126	0.1959\\
0.5156	0.2037\\
0.5187	0.2114\\
0.5217	0.219\\
0.5247	0.2265\\
0.5278	0.2339\\
0.5338	0.2485\\
0.5369	0.2556\\
0.5399	0.2627\\
0.5429	0.2696\\
0.546	0.2765\\
0.549	0.2833\\
0.5521	0.29\\
0.5551	0.2966\\
0.5581	0.3031\\
0.5612	0.3095\\
0.5642	0.3159\\
0.5648	0.3171\\
0.5654	0.3184\\
0.5661	0.3197\\
0.5667	0.3209\\
};
\addplot [color=black, forget plot]
  table[row sep=crcr]{%
0.5667	0.3209\\
0.5714	0.3305\\
0.5761	0.3398\\
0.5808	0.3489\\
0.5856	0.3578\\
0.5903	0.3665\\
0.595	0.375\\
0.5997	0.3832\\
0.6044	0.3912\\
0.6092	0.3991\\
0.6139	0.4066\\
0.6186	0.414\\
0.6233	0.4212\\
0.6281	0.4281\\
0.6328	0.4348\\
0.6375	0.4413\\
0.6422	0.4476\\
0.6469	0.4536\\
0.6517	0.4595\\
0.6564	0.4651\\
0.6611	0.4705\\
0.6658	0.4757\\
0.6706	0.4806\\
0.6753	0.4854\\
0.68	0.4899\\
0.6847	0.4942\\
0.6894	0.4983\\
0.6942	0.5022\\
0.6989	0.5058\\
0.7036	0.5092\\
0.7083	0.5124\\
0.7131	0.5154\\
0.7178	0.5182\\
0.7225	0.5208\\
0.7272	0.5231\\
0.7319	0.5252\\
0.7367	0.5271\\
0.7414	0.5288\\
0.7461	0.5303\\
0.7508	0.5315\\
0.7556	0.5325\\
};
\addplot [color=black, forget plot]
  table[row sep=crcr]{%
0.7556	0.5325\\
0.7565	0.5327\\
0.7585	0.5331\\
0.7595	0.5332\\
0.7635	0.5338\\
0.7674	0.5341\\
0.7714	0.5344\\
0.7753	0.5344\\
};
\addplot [color=black, forget plot]
  table[row sep=crcr]{%
0.7753	0.5344\\
0.7785	0.5344\\
0.7849	0.534\\
0.7913	0.5332\\
0.7955	0.5324\\
0.7998	0.5315\\
0.804	0.5304\\
0.8082	0.5291\\
0.8124	0.5277\\
0.8167	0.5261\\
0.8251	0.5223\\
0.8294	0.5201\\
0.8336	0.5178\\
0.8378	0.5153\\
0.842	0.5126\\
0.8463	0.5097\\
0.8505	0.5067\\
0.8547	0.5035\\
0.859	0.5001\\
0.8632	0.4966\\
0.8674	0.4928\\
0.8716	0.4889\\
0.8759	0.4848\\
0.8801	0.4806\\
0.8843	0.4761\\
0.8886	0.4715\\
0.8928	0.4667\\
0.897	0.4618\\
0.9012	0.4567\\
0.9055	0.4513\\
0.9097	0.4459\\
0.9139	0.4402\\
0.9182	0.4344\\
0.9266	0.4222\\
0.9311	0.4154\\
0.9355	0.4085\\
0.94	0.4014\\
0.9444	0.3941\\
};
\addplot [color=black, forget plot]
  table[row sep=crcr]{%
0.9444	0.3941\\
0.9492	0.3862\\
0.9539	0.378\\
0.9586	0.3696\\
0.9633	0.361\\
0.9681	0.3522\\
0.9728	0.3432\\
0.9775	0.3339\\
0.9822	0.3244\\
0.9869	0.3148\\
0.9917	0.3048\\
0.9964	0.2947\\
1.0011	0.2844\\
1.0058	0.2738\\
1.0106	0.263\\
1.0153	0.252\\
1.02	0.2408\\
1.0247	0.2293\\
1.0294	0.2177\\
1.0342	0.2058\\
1.0389	0.1937\\
1.0436	0.1814\\
1.0483	0.1688\\
1.0531	0.1561\\
1.0578	0.1431\\
1.0625	0.1299\\
1.0672	0.1165\\
1.0719	0.1028\\
1.0767	0.089\\
1.0814	0.0749\\
1.0861	0.0606\\
1.0908	0.0461\\
1.0956	0.0314\\
1.098	0.0236\\
1.1005	0.0158\\
1.1029	0.0079\\
1.1054	-0\\
};
\addplot [color=black, forget plot]
  table[row sep=crcr]{%
1.1054	0\\
1.1054	0.0001\\
1.1055	0.0002\\
1.1056	0.0005\\
1.1057	0.0007\\
1.1058	0.001\\
1.1059	0.0012\\
1.1064	0.0025\\
1.1069	0.0037\\
1.1074	0.005\\
1.108	0.0062\\
1.1087	0.0079\\
1.1094	0.0095\\
1.1108	0.0129\\
1.1115	0.0145\\
1.1122	0.0162\\
1.1129	0.0178\\
1.1135	0.0195\\
1.1149	0.0227\\
1.1156	0.0244\\
1.1233	0.042\\
1.124	0.0435\\
1.1254	0.0467\\
1.1261	0.0482\\
1.1268	0.0498\\
1.1275	0.0513\\
1.1282	0.0529\\
1.1296	0.0559\\
1.1303	0.0575\\
1.1311	0.0591\\
1.1318	0.0608\\
1.1326	0.0624\\
1.1333	0.064\\
};
\addplot [color=black, forget plot]
  table[row sep=crcr]{%
1.1333	0.064\\
1.1378	0.0736\\
1.1393	0.0767\\
1.144	0.0865\\
1.1487	0.0961\\
1.1535	0.1054\\
1.1582	0.1145\\
1.1629	0.1235\\
1.1676	0.1322\\
1.1724	0.1406\\
1.1771	0.1489\\
1.1818	0.1569\\
1.1865	0.1647\\
1.1912	0.1723\\
1.196	0.1797\\
1.2007	0.1869\\
1.2054	0.1938\\
1.2101	0.2006\\
1.2149	0.2071\\
1.2196	0.2134\\
1.2243	0.2194\\
1.229	0.2253\\
1.2337	0.2309\\
1.2385	0.2363\\
1.2432	0.2415\\
1.2479	0.2465\\
1.2526	0.2513\\
1.2574	0.2558\\
1.2621	0.2601\\
1.2668	0.2642\\
1.2715	0.2681\\
1.2762	0.2718\\
1.281	0.2752\\
1.2857	0.2784\\
1.2904	0.2814\\
1.2951	0.2842\\
1.2999	0.2868\\
1.3046	0.2891\\
1.3093	0.2913\\
1.3125	0.2926\\
1.3158	0.2938\\
1.319	0.295\\
1.3222	0.296\\
};
\addplot [color=black, forget plot]
  table[row sep=crcr]{%
1.3222	0.296\\
1.3238	0.2964\\
1.3253	0.2969\\
1.3269	0.2973\\
1.3284	0.2977\\
1.3331	0.2987\\
1.3378	0.2995\\
1.3426	0.3001\\
1.3473	0.3005\\
1.3487	0.3005\\
1.3501	0.3006\\
1.353	0.3006\\
};
\addplot [color=black, forget plot]
  table[row sep=crcr]{%
1.353	0.3006\\
1.3562	0.3006\\
1.3626	0.3002\\
1.369	0.2994\\
1.3729	0.2987\\
1.3769	0.2978\\
1.3808	0.2968\\
1.3848	0.2957\\
1.3887	0.2944\\
1.3927	0.2929\\
1.3966	0.2913\\
1.4006	0.2895\\
1.4045	0.2876\\
1.4085	0.2855\\
1.4125	0.2833\\
1.4164	0.2809\\
1.4204	0.2783\\
1.4243	0.2757\\
1.4283	0.2728\\
1.4322	0.2698\\
1.4362	0.2667\\
1.4401	0.2634\\
1.4441	0.2599\\
1.448	0.2563\\
1.452	0.2525\\
1.4559	0.2486\\
1.4599	0.2445\\
1.4639	0.2403\\
1.4678	0.2359\\
1.4718	0.2314\\
1.4757	0.2267\\
1.4797	0.2219\\
1.4836	0.2169\\
1.4876	0.2118\\
1.4915	0.2065\\
1.4955	0.201\\
1.4994	0.1955\\
1.5033	0.1898\\
1.5111	0.178\\
};
\addplot [color=black, forget plot]
  table[row sep=crcr]{%
1.5111	0.178\\
1.5158	0.1705\\
1.5206	0.1629\\
1.5253	0.155\\
1.53	0.1469\\
1.5347	0.1386\\
1.5394	0.1301\\
1.5442	0.1213\\
1.5489	0.1123\\
1.5536	0.1032\\
1.5583	0.0938\\
1.5631	0.0841\\
1.5678	0.0743\\
1.5725	0.0642\\
1.5772	0.0539\\
1.5819	0.0435\\
1.5867	0.0327\\
1.5901	0.0247\\
1.5936	0.0166\\
1.5971	0.0084\\
1.6005	-0\\
};
\addplot [color=black, forget plot]
  table[row sep=crcr]{%
1.6005	0\\
1.6006	0.0001\\
1.6006	0.0002\\
1.6008	0.0005\\
1.6009	0.0007\\
1.6011	0.001\\
1.6012	0.0012\\
1.6019	0.0025\\
1.6033	0.0049\\
1.604	0.0062\\
1.6064	0.0106\\
1.6139	0.0235\\
1.6164	0.0276\\
1.6189	0.0318\\
1.6214	0.0358\\
1.6238	0.0398\\
1.6288	0.0476\\
1.6338	0.0552\\
1.6363	0.0588\\
1.6388	0.0625\\
1.6413	0.066\\
1.6437	0.0695\\
1.6462	0.073\\
1.6487	0.0764\\
1.6537	0.083\\
1.6562	0.0862\\
1.6587	0.0893\\
1.6611	0.0924\\
1.6661	0.0984\\
1.6686	0.1013\\
1.6736	0.1069\\
1.6786	0.1123\\
1.681	0.1149\\
1.686	0.1199\\
1.6885	0.1223\\
1.6935	0.1269\\
1.6951	0.1284\\
1.6967	0.1298\\
1.6984	0.1313\\
1.7	0.1326\\
};
\addplot [color=black, forget plot]
  table[row sep=crcr]{%
0	0.95\\
0.0032	0.95\\
};
\addplot [color=black, forget plot]
  table[row sep=crcr]{%
0.0032	0.95\\
0.0064	0.95\\
0.0128	0.9496\\
0.0192	0.9488\\
0.0238	0.948\\
0.0285	0.9469\\
0.0331	0.9457\\
0.0377	0.9442\\
0.0424	0.9425\\
0.047	0.9406\\
0.0517	0.9385\\
0.0563	0.9362\\
0.061	0.9337\\
0.0656	0.9309\\
0.0702	0.928\\
0.0749	0.9248\\
0.0795	0.9215\\
0.0842	0.9179\\
0.0888	0.9141\\
0.0935	0.9101\\
0.0981	0.9058\\
0.1027	0.9014\\
0.1074	0.8968\\
0.112	0.8919\\
0.1167	0.8869\\
0.1213	0.8816\\
0.126	0.8761\\
0.1306	0.8704\\
0.1352	0.8645\\
0.1399	0.8584\\
0.1445	0.852\\
0.1492	0.8455\\
0.1538	0.8387\\
0.1585	0.8318\\
0.1631	0.8246\\
0.1677	0.8172\\
0.1724	0.8096\\
0.177	0.8018\\
0.1817	0.7938\\
0.1863	0.7855\\
0.187	0.7844\\
0.1882	0.782\\
0.1889	0.7809\\
};
\addplot [color=black, forget plot]
  table[row sep=crcr]{%
0.1889	0.7809\\
0.1936	0.7722\\
0.1983	0.7632\\
0.2031	0.7541\\
0.2078	0.7447\\
0.2125	0.7351\\
0.2172	0.7253\\
0.2219	0.7153\\
0.2267	0.705\\
0.2314	0.6946\\
0.2361	0.6839\\
0.2408	0.673\\
0.2456	0.6619\\
0.2503	0.6505\\
0.255	0.639\\
0.2597	0.6272\\
0.2644	0.6152\\
0.2692	0.603\\
0.2739	0.5906\\
0.2786	0.5779\\
0.2833	0.5651\\
0.2881	0.552\\
0.2928	0.5387\\
0.2975	0.5251\\
0.3022	0.5114\\
0.3069	0.4974\\
0.3117	0.4832\\
0.3164	0.4688\\
0.3211	0.4542\\
0.3258	0.4394\\
0.3306	0.4243\\
0.3353	0.4091\\
0.34	0.3936\\
0.3447	0.3779\\
0.3494	0.3619\\
0.3542	0.3458\\
0.3589	0.3294\\
0.3636	0.3128\\
0.3683	0.296\\
0.3731	0.279\\
0.3778	0.2617\\
};
\addplot [color=black, forget plot]
  table[row sep=crcr]{%
0.3778	0.2617\\
0.3814	0.2485\\
0.3849	0.2352\\
0.3885	0.2217\\
0.3921	0.2081\\
0.3968	0.19\\
0.4015	0.1717\\
0.4063	0.1531\\
0.411	0.1343\\
0.4157	0.1153\\
0.4204	0.0961\\
0.4251	0.0767\\
0.4299	0.057\\
0.4332	0.0429\\
0.4366	0.0287\\
0.4399	0.0144\\
0.4433	0\\
};
\addplot [color=black, forget plot]
  table[row sep=crcr]{%
0.4433	0\\
0.4433	0.0002\\
0.4434	0.0005\\
0.4435	0.0007\\
0.4436	0.001\\
0.4436	0.0012\\
0.444	0.0025\\
0.4444	0.0037\\
0.4448	0.005\\
0.4452	0.0062\\
0.4471	0.0124\\
0.4491	0.0186\\
0.451	0.0248\\
0.453	0.0309\\
0.456	0.0405\\
0.4591	0.0501\\
0.4622	0.0596\\
0.4684	0.0782\\
0.4715	0.0874\\
0.4745	0.0965\\
0.4776	0.1055\\
0.4807	0.1144\\
0.4838	0.1232\\
0.4869	0.1319\\
0.49	0.1405\\
0.4931	0.149\\
0.4961	0.1575\\
0.5023	0.1741\\
0.5085	0.1903\\
0.5116	0.1983\\
0.5147	0.2061\\
0.5177	0.2139\\
0.5208	0.2216\\
0.5239	0.2292\\
0.527	0.2367\\
0.5301	0.2441\\
0.5332	0.2514\\
0.5362	0.2586\\
0.5393	0.2658\\
0.5455	0.2798\\
0.5517	0.2934\\
0.5548	0.3\\
0.5578	0.3066\\
0.5609	0.3131\\
0.564	0.3194\\
0.5647	0.3208\\
0.5653	0.3222\\
0.566	0.3235\\
0.5667	0.3249\\
};
\addplot [color=black, forget plot]
  table[row sep=crcr]{%
0.5667	0.3249\\
0.5714	0.3343\\
0.5761	0.3436\\
0.5808	0.3526\\
0.5856	0.3614\\
0.5903	0.37\\
0.595	0.3784\\
0.5997	0.3865\\
0.6044	0.3945\\
0.6092	0.4022\\
0.6139	0.4097\\
0.6186	0.417\\
0.6233	0.424\\
0.6281	0.4309\\
0.6328	0.4375\\
0.6375	0.4439\\
0.6422	0.4501\\
0.6469	0.456\\
0.6517	0.4618\\
0.6564	0.4673\\
0.6611	0.4726\\
0.6658	0.4777\\
0.6706	0.4826\\
0.6753	0.4872\\
0.68	0.4917\\
0.6847	0.4959\\
0.6894	0.4999\\
0.6942	0.5036\\
0.6989	0.5072\\
0.7036	0.5105\\
0.7083	0.5137\\
0.7131	0.5166\\
0.7178	0.5193\\
0.7225	0.5217\\
0.7272	0.524\\
0.7319	0.526\\
0.7367	0.5278\\
0.7414	0.5294\\
0.7461	0.5308\\
0.7508	0.5319\\
0.7556	0.5328\\
};
\addplot [color=black, forget plot]
  table[row sep=crcr]{%
0.7556	0.5328\\
0.7564	0.533\\
0.7573	0.5331\\
0.7582	0.5333\\
0.7627	0.5338\\
0.7662	0.5342\\
0.7734	0.5344\\
};
\addplot [color=black, forget plot]
  table[row sep=crcr]{%
0.7734	0.5344\\
0.7765	0.5344\\
0.7829	0.534\\
0.7861	0.5336\\
0.7894	0.5331\\
0.7936	0.5324\\
0.7979	0.5314\\
0.8022	0.5303\\
0.8065	0.529\\
0.8107	0.5275\\
0.815	0.5259\\
0.8193	0.5241\\
0.8236	0.522\\
0.8278	0.5198\\
0.8321	0.5175\\
0.8364	0.5149\\
0.8407	0.5122\\
0.845	0.5093\\
0.8492	0.5062\\
0.8535	0.5029\\
0.8578	0.4994\\
0.8621	0.4958\\
0.8663	0.492\\
0.8706	0.488\\
0.8749	0.4838\\
0.8792	0.4795\\
0.8835	0.4749\\
0.8877	0.4702\\
0.892	0.4653\\
0.8963	0.4603\\
0.9006	0.455\\
0.9048	0.4496\\
0.9091	0.444\\
0.9134	0.4382\\
0.9177	0.4322\\
0.9219	0.4261\\
0.9262	0.4198\\
0.9308	0.4128\\
0.9353	0.4057\\
0.9399	0.3984\\
0.9444	0.3908\\
};
\addplot [color=black, forget plot]
  table[row sep=crcr]{%
0.9444	0.3908\\
0.9492	0.3828\\
0.9539	0.3745\\
0.9586	0.3661\\
0.9633	0.3574\\
0.9681	0.3485\\
0.9728	0.3393\\
0.9775	0.33\\
0.9822	0.3204\\
0.9869	0.3106\\
0.9917	0.3006\\
0.9964	0.2904\\
1.0011	0.28\\
1.0058	0.2693\\
1.0106	0.2584\\
1.0153	0.2473\\
1.02	0.236\\
1.0247	0.2245\\
1.0294	0.2127\\
1.0342	0.2007\\
1.0389	0.1886\\
1.0436	0.1761\\
1.0483	0.1635\\
1.0531	0.1507\\
1.0578	0.1376\\
1.0625	0.1243\\
1.0672	0.1108\\
1.0719	0.0971\\
1.0767	0.0831\\
1.0814	0.069\\
1.0861	0.0546\\
1.0908	0.04\\
1.0956	0.0252\\
1.0975	0.0189\\
1.0995	0.0127\\
1.1015	0.0064\\
1.1034	-0\\
};
\addplot [color=black, forget plot]
  table[row sep=crcr]{%
1.1034	0\\
1.1034	0.0001\\
1.1035	0.0001\\
1.1035	0.0002\\
1.1036	0.0005\\
1.1037	0.0007\\
1.1038	0.001\\
1.1039	0.0012\\
1.1044	0.0025\\
1.105	0.0037\\
1.1055	0.005\\
1.106	0.0062\\
1.1067	0.008\\
1.1075	0.0098\\
1.1082	0.0116\\
1.109	0.0133\\
1.1097	0.0151\\
1.1105	0.0169\\
1.1112	0.0186\\
1.112	0.0204\\
1.1127	0.0221\\
1.1135	0.0239\\
1.1142	0.0256\\
1.115	0.0274\\
1.1157	0.0291\\
1.1165	0.0308\\
1.1172	0.0325\\
1.118	0.0342\\
1.1187	0.036\\
1.1195	0.0377\\
1.1202	0.0393\\
1.1209	0.041\\
1.1217	0.0427\\
1.1224	0.0444\\
1.1232	0.0461\\
1.1239	0.0477\\
1.1247	0.0494\\
1.1254	0.0511\\
1.1262	0.0527\\
1.1269	0.0544\\
1.1277	0.056\\
1.1284	0.0576\\
1.1292	0.0593\\
1.1299	0.0609\\
1.1307	0.0625\\
1.1314	0.0641\\
1.1322	0.0657\\
1.1329	0.0673\\
1.133	0.0676\\
1.1333	0.0682\\
};
\addplot [color=black, forget plot]
  table[row sep=crcr]{%
1.1333	0.0682\\
1.1349	0.0717\\
1.1365	0.075\\
1.1382	0.0784\\
1.1398	0.0817\\
1.1445	0.0914\\
1.1492	0.1009\\
1.1539	0.1101\\
1.1586	0.1191\\
1.1634	0.1279\\
1.1681	0.1365\\
1.1728	0.1449\\
1.1775	0.153\\
1.1823	0.161\\
1.187	0.1687\\
1.1917	0.1762\\
1.1964	0.1834\\
1.2011	0.1905\\
1.2059	0.1973\\
1.2106	0.2039\\
1.2153	0.2103\\
1.22	0.2165\\
1.2248	0.2224\\
1.2295	0.2282\\
1.2342	0.2337\\
1.2389	0.239\\
1.2436	0.2441\\
1.2531	0.2536\\
1.2578	0.258\\
1.2625	0.2622\\
1.2673	0.2662\\
1.272	0.27\\
1.2767	0.2735\\
1.2814	0.2769\\
1.2861	0.28\\
1.2909	0.2829\\
1.2956	0.2855\\
1.3003	0.288\\
1.305	0.2902\\
1.3098	0.2923\\
1.3129	0.2935\\
1.316	0.2946\\
1.3191	0.2956\\
1.3222	0.2965\\
};
\addplot [color=black, forget plot]
  table[row sep=crcr]{%
1.3222	0.2965\\
1.3237	0.2969\\
1.3251	0.2973\\
1.3266	0.2977\\
1.328	0.298\\
1.3327	0.299\\
1.3374	0.2997\\
1.3422	0.3002\\
1.3469	0.3005\\
1.3479	0.3006\\
1.351	0.3006\\
};
\addplot [color=black, forget plot]
  table[row sep=crcr]{%
1.351	0.3006\\
1.3542	0.3006\\
1.3574	0.3004\\
1.3638	0.2998\\
1.367	0.2993\\
1.371	0.2986\\
1.375	0.2978\\
1.379	0.2968\\
1.383	0.2956\\
1.387	0.2942\\
1.391	0.2927\\
1.395	0.2911\\
1.399	0.2893\\
1.403	0.2873\\
1.407	0.2852\\
1.411	0.2829\\
1.415	0.2805\\
1.419	0.2779\\
1.423	0.2751\\
1.427	0.2722\\
1.431	0.2692\\
1.439	0.2626\\
1.443	0.259\\
1.447	0.2553\\
1.451	0.2515\\
1.4551	0.2475\\
1.4591	0.2433\\
1.4631	0.239\\
1.4671	0.2345\\
1.4711	0.2299\\
1.4751	0.2251\\
1.4791	0.2201\\
1.4831	0.215\\
1.4871	0.2098\\
1.4951	0.1988\\
1.4991	0.193\\
1.5031	0.1871\\
1.5071	0.1811\\
1.5111	0.1748\\
};
\addplot [color=black, forget plot]
  table[row sep=crcr]{%
1.5111	0.1748\\
1.5158	0.1673\\
1.5206	0.1596\\
1.5253	0.1516\\
1.53	0.1434\\
1.5347	0.135\\
1.5394	0.1264\\
1.5442	0.1176\\
1.5489	0.1085\\
1.5536	0.0992\\
1.5583	0.0897\\
1.5631	0.08\\
1.5678	0.0701\\
1.5725	0.0599\\
1.5772	0.0495\\
1.5819	0.039\\
1.5867	0.0282\\
1.5896	0.0212\\
1.5956	0.0072\\
1.5985	-0\\
};
\addplot [color=black, forget plot]
  table[row sep=crcr]{%
1.5985	0\\
1.5986	0.0001\\
1.5986	0.0002\\
1.5987	0.0002\\
1.5988	0.0005\\
1.5989	0.0007\\
1.5991	0.001\\
1.5992	0.0012\\
1.5999	0.0025\\
1.6013	0.0049\\
1.602	0.0062\\
1.6045	0.0107\\
1.607	0.0151\\
1.6096	0.0195\\
1.6146	0.0281\\
1.6172	0.0322\\
1.6197	0.0364\\
1.6223	0.0404\\
1.6273	0.0484\\
1.6299	0.0522\\
1.6349	0.0598\\
1.6375	0.0635\\
1.64	0.0671\\
1.6425	0.0706\\
1.6451	0.0741\\
1.6476	0.0776\\
1.6502	0.0809\\
1.6552	0.0875\\
1.6578	0.0907\\
1.6603	0.0938\\
1.6628	0.0968\\
1.6654	0.0998\\
1.6704	0.1056\\
1.673	0.1084\\
1.6755	0.1111\\
1.6781	0.1138\\
1.6831	0.119\\
1.6857	0.1215\\
1.6882	0.1239\\
1.6907	0.1262\\
1.6933	0.1285\\
1.695	0.13\\
1.6966	0.1315\\
1.7	0.1343\\
};
\addplot [color=black, forget plot]
  table[row sep=crcr]{%
0	0.9932\\
0.0008	0.9932\\
0.001	0.9933\\
0.005	0.9933\\
};
\addplot [color=black, forget plot]
  table[row sep=crcr]{%
0.005	0.9933\\
0.0082	0.9933\\
0.0146	0.9929\\
0.021	0.9921\\
0.0256	0.9913\\
0.0302	0.9902\\
0.0348	0.989\\
0.0394	0.9875\\
0.044	0.9859\\
0.0486	0.984\\
0.0532	0.982\\
0.0578	0.9797\\
0.0624	0.9772\\
0.067	0.9745\\
0.0716	0.9716\\
0.0762	0.9685\\
0.0808	0.9652\\
0.0854	0.9617\\
0.09	0.9579\\
0.0946	0.954\\
0.0992	0.9499\\
0.1038	0.9455\\
0.1084	0.941\\
0.113	0.9362\\
0.1176	0.9312\\
0.1222	0.9261\\
0.1267	0.9207\\
0.1313	0.9151\\
0.1359	0.9093\\
0.1405	0.9033\\
0.1451	0.8971\\
0.1497	0.8906\\
0.1543	0.884\\
0.1589	0.8772\\
0.1635	0.8701\\
0.1681	0.8629\\
0.1727	0.8554\\
0.1773	0.8478\\
0.1819	0.8399\\
0.1865	0.8318\\
0.1871	0.8307\\
0.1877	0.8297\\
0.1889	0.8275\\
};
\addplot [color=black, forget plot]
  table[row sep=crcr]{%
0.1889	0.8275\\
0.1936	0.8189\\
0.1983	0.8101\\
0.2031	0.801\\
0.2078	0.7917\\
0.2125	0.7822\\
0.2172	0.7725\\
0.2219	0.7625\\
0.2267	0.7524\\
0.2314	0.742\\
0.2361	0.7314\\
0.2408	0.7206\\
0.2456	0.7096\\
0.2503	0.6983\\
0.255	0.6868\\
0.2597	0.6752\\
0.2644	0.6632\\
0.2692	0.6511\\
0.2739	0.6388\\
0.2786	0.6262\\
0.2833	0.6134\\
0.2881	0.6004\\
0.2928	0.5872\\
0.2975	0.5738\\
0.3022	0.5601\\
0.3069	0.5462\\
0.3117	0.5321\\
0.3164	0.5178\\
0.3211	0.5033\\
0.3258	0.4885\\
0.3306	0.4736\\
0.3353	0.4584\\
0.34	0.443\\
0.3447	0.4273\\
0.3494	0.4115\\
0.3542	0.3954\\
0.3589	0.3791\\
0.3636	0.3626\\
0.3683	0.3459\\
0.3731	0.329\\
0.3778	0.3118\\
};
\addplot [color=black, forget plot]
  table[row sep=crcr]{%
0.3778	0.3118\\
0.3821	0.2961\\
0.3863	0.2801\\
0.3906	0.264\\
0.3949	0.2477\\
0.3996	0.2296\\
0.4044	0.2112\\
0.4091	0.1926\\
0.4138	0.1737\\
0.4185	0.1547\\
0.4232	0.1354\\
0.428	0.1159\\
0.4327	0.0962\\
0.4374	0.0763\\
0.4421	0.0562\\
0.4469	0.0358\\
0.4516	0.0152\\
0.4524	0.0114\\
0.4542	0.0038\\
0.455	-0\\
};
\addplot [color=black, forget plot]
  table[row sep=crcr]{%
0.455	0\\
0.4551	0.0001\\
0.4551	0.0002\\
0.4552	0.0005\\
0.4553	0.0007\\
0.4553	0.001\\
0.4554	0.0012\\
0.4558	0.0025\\
0.4562	0.0037\\
0.4565	0.005\\
0.4569	0.0062\\
0.4626	0.0248\\
0.4645	0.0309\\
0.4673	0.0399\\
0.4729	0.0575\\
0.4785	0.0749\\
0.4813	0.0834\\
0.484	0.0919\\
0.4868	0.1003\\
0.4924	0.1169\\
0.4952	0.1251\\
0.498	0.1332\\
0.5008	0.1412\\
0.5064	0.157\\
0.5092	0.1648\\
0.512	0.1725\\
0.5147	0.1802\\
0.5203	0.1952\\
0.5259	0.21\\
0.5315	0.2244\\
0.5371	0.2386\\
0.5399	0.2455\\
0.5426	0.2524\\
0.5454	0.2592\\
0.551	0.2726\\
0.5538	0.2792\\
0.5566	0.2857\\
0.5594	0.2921\\
0.5654	0.3056\\
0.5658	0.3066\\
0.5662	0.3075\\
0.5667	0.3085\\
};
\addplot [color=black, forget plot]
  table[row sep=crcr]{%
0.5667	0.3085\\
0.5714	0.3188\\
0.5761	0.329\\
0.5808	0.3389\\
0.5856	0.3486\\
0.5903	0.3581\\
0.595	0.3673\\
0.5997	0.3764\\
0.6044	0.3852\\
0.6092	0.3938\\
0.6139	0.4022\\
0.6186	0.4103\\
0.6233	0.4183\\
0.6281	0.426\\
0.6328	0.4335\\
0.6375	0.4408\\
0.6422	0.4479\\
0.6469	0.4547\\
0.6517	0.4614\\
0.6564	0.4678\\
0.6611	0.474\\
0.6658	0.48\\
0.6706	0.4857\\
0.6753	0.4913\\
0.68	0.4966\\
0.6847	0.5017\\
0.6894	0.5066\\
0.6942	0.5113\\
0.6989	0.5157\\
0.7036	0.5199\\
0.7083	0.524\\
0.7131	0.5277\\
0.7178	0.5313\\
0.7225	0.5347\\
0.7272	0.5378\\
0.7319	0.5407\\
0.7367	0.5434\\
0.7414	0.5459\\
0.7461	0.5482\\
0.7508	0.5502\\
0.7556	0.552\\
};
\addplot [color=black, forget plot]
  table[row sep=crcr]{%
0.7556	0.552\\
0.7574	0.5527\\
0.7593	0.5533\\
0.7611	0.5539\\
0.763	0.5545\\
0.7677	0.5557\\
0.7724	0.5568\\
0.7772	0.5576\\
0.7819	0.5582\\
0.7845	0.5584\\
0.7872	0.5586\\
0.7926	0.5588\\
};
\addplot [color=black, forget plot]
  table[row sep=crcr]{%
0.7926	0.5588\\
0.7929	0.5588\\
0.7931	0.5587\\
0.7958	0.5587\\
0.799	0.5586\\
0.8022	0.5583\\
0.8086	0.5575\\
0.8124	0.5568\\
0.8162	0.556\\
0.8199	0.5551\\
0.8237	0.554\\
0.8313	0.5514\\
0.8351	0.5499\\
0.8389	0.5482\\
0.8427	0.5464\\
0.8465	0.5445\\
0.8503	0.5424\\
0.8541	0.5402\\
0.8579	0.5378\\
0.8617	0.5353\\
0.8693	0.5299\\
0.8769	0.5239\\
0.8845	0.5173\\
0.8883	0.5138\\
0.8921	0.5102\\
0.8959	0.5064\\
0.8997	0.5025\\
0.9035	0.4984\\
0.9073	0.4942\\
0.9149	0.4854\\
0.9225	0.476\\
0.9263	0.4711\\
0.9301	0.466\\
0.9337	0.4611\\
0.9373	0.4561\\
0.9408	0.4509\\
0.9444	0.4456\\
};
\addplot [color=black, forget plot]
  table[row sep=crcr]{%
0.9444	0.4456\\
0.9492	0.4384\\
0.9539	0.4311\\
0.9586	0.4235\\
0.9633	0.4157\\
0.9681	0.4077\\
0.9728	0.3994\\
0.9775	0.391\\
0.9822	0.3823\\
0.9869	0.3734\\
0.9917	0.3643\\
0.9964	0.355\\
1.0011	0.3454\\
1.0058	0.3356\\
1.0106	0.3257\\
1.0153	0.3154\\
1.02	0.305\\
1.0247	0.2944\\
1.0294	0.2835\\
1.0342	0.2724\\
1.0389	0.2611\\
1.0436	0.2496\\
1.0483	0.2379\\
1.0531	0.2259\\
1.0578	0.2137\\
1.0625	0.2013\\
1.0672	0.1887\\
1.0719	0.1759\\
1.0767	0.1628\\
1.0814	0.1496\\
1.0861	0.1361\\
1.0908	0.1224\\
1.0956	0.1084\\
1.1003	0.0943\\
1.105	0.0799\\
1.1097	0.0653\\
1.1144	0.0505\\
1.1184	0.0381\\
1.1223	0.0256\\
1.1262	0.0129\\
1.1301	0\\
};
\addplot [color=black, forget plot]
  table[row sep=crcr]{%
1.1301	0\\
1.1301	0.0002\\
1.1302	0.0002\\
1.1302	0.0004\\
1.1306	0.0012\\
1.1306	0.0014\\
1.131	0.0022\\
1.131	0.0024\\
1.1315	0.0034\\
1.1315	0.0036\\
1.1319	0.0044\\
1.1319	0.0046\\
1.1324	0.0056\\
1.1324	0.0058\\
1.1328	0.0066\\
1.1328	0.0068\\
1.1331	0.0074\\
1.1331	0.0076\\
1.1332	0.0077\\
1.1333	0.0079\\
1.1333	0.0081\\
};
\addplot [color=black, forget plot]
  table[row sep=crcr]{%
1.1333	0.0081\\
1.1337	0.0089\\
1.1338	0.0093\\
1.134	0.0097\\
1.1356	0.0137\\
1.1365	0.0157\\
1.1373	0.0177\\
1.1414	0.0275\\
1.1455	0.0372\\
1.1497	0.0468\\
1.1538	0.0561\\
1.1585	0.0667\\
1.1632	0.077\\
1.168	0.087\\
1.1727	0.0969\\
1.1774	0.1065\\
1.1821	0.116\\
1.1868	0.1252\\
1.1916	0.1342\\
1.1963	0.1429\\
1.201	0.1515\\
1.2057	0.1598\\
1.2105	0.1679\\
1.2152	0.1758\\
1.2199	0.1835\\
1.2246	0.1909\\
1.2293	0.1982\\
1.2341	0.2052\\
1.2388	0.212\\
1.2435	0.2186\\
1.2482	0.2249\\
1.253	0.2311\\
1.2577	0.237\\
1.2624	0.2427\\
1.2671	0.2482\\
1.2718	0.2535\\
1.2766	0.2585\\
1.2813	0.2634\\
1.286	0.268\\
1.2907	0.2724\\
1.2955	0.2765\\
1.3002	0.2805\\
1.3049	0.2842\\
1.3092	0.2875\\
1.3136	0.2905\\
1.3179	0.2934\\
1.3222	0.2961\\
};
\addplot [color=black, forget plot]
  table[row sep=crcr]{%
1.3222	0.2961\\
1.3253	0.2978\\
1.3283	0.2995\\
1.3345	0.3027\\
1.3392	0.3048\\
1.3439	0.3067\\
1.3486	0.3084\\
1.3534	0.3099\\
1.3581	0.3112\\
1.3628	0.3123\\
1.3675	0.3131\\
1.3723	0.3137\\
1.375	0.314\\
1.3777	0.3142\\
1.3805	0.3143\\
1.3832	0.3143\\
};
\addplot [color=black, forget plot]
  table[row sep=crcr]{%
1.3832	0.3143\\
1.3858	0.3143\\
1.3864	0.3142\\
1.3896	0.3141\\
1.396	0.3135\\
1.4024	0.3125\\
1.4088	0.3111\\
1.4152	0.3093\\
1.4216	0.3071\\
1.428	0.3045\\
1.4344	0.3015\\
1.4376	0.2998\\
1.444	0.2962\\
1.4504	0.2922\\
1.4536	0.29\\
1.4567	0.2878\\
1.4631	0.283\\
1.4663	0.2804\\
1.4727	0.275\\
1.4791	0.2692\\
1.4823	0.2661\\
1.4887	0.2597\\
1.4951	0.2529\\
1.4983	0.2493\\
1.5015	0.2456\\
1.5063	0.24\\
1.5087	0.237\\
1.5111	0.2341\\
};
\addplot [color=black, forget plot]
  table[row sep=crcr]{%
1.5111	0.2341\\
1.5158	0.228\\
1.5206	0.2218\\
1.5253	0.2153\\
1.53	0.2086\\
1.5347	0.2017\\
1.5394	0.1946\\
1.5442	0.1872\\
1.5489	0.1796\\
1.5536	0.1719\\
1.5583	0.1639\\
1.5631	0.1556\\
1.5678	0.1472\\
1.5725	0.1385\\
1.5772	0.1297\\
1.5819	0.1206\\
1.5867	0.1112\\
1.5914	0.1017\\
1.5961	0.092\\
1.6008	0.082\\
1.6056	0.0718\\
1.6103	0.0614\\
1.615	0.0508\\
1.6197	0.0399\\
1.6244	0.0288\\
1.6274	0.0218\\
1.6304	0.0146\\
1.6334	0.0073\\
1.6363	-0\\
};
\addplot [color=black, forget plot]
  table[row sep=crcr]{%
1.6363	0\\
1.6364	0.0001\\
1.6364	0.0002\\
1.6366	0.0005\\
1.6367	0.0007\\
1.6369	0.001\\
1.637	0.0012\\
1.6377	0.0025\\
1.6383	0.0037\\
1.639	0.0049\\
1.6397	0.0062\\
1.6413	0.0091\\
1.6429	0.0119\\
1.6445	0.0148\\
1.646	0.0176\\
1.6492	0.0232\\
1.6508	0.0259\\
1.6524	0.0287\\
1.6556	0.0341\\
1.6636	0.0471\\
1.6651	0.0496\\
1.6683	0.0546\\
1.6763	0.0666\\
1.6811	0.0735\\
1.6827	0.0757\\
1.6842	0.078\\
1.6858	0.0802\\
1.6874	0.0823\\
1.689	0.0845\\
1.6954	0.0929\\
1.697	0.0949\\
1.6977	0.0959\\
1.6985	0.0968\\
1.6992	0.0977\\
1.7	0.0987\\
};
\addplot [color=black, forget plot]
  table[row sep=crcr]{%
0	0.9996\\
0.0018	0.9996\\
0.0054	0.9994\\
0.0073	0.9993\\
0.0091	0.9991\\
0.0139	0.9985\\
0.0186	0.9977\\
0.0233	0.9967\\
0.028	0.9954\\
0.0327	0.9939\\
0.0375	0.9922\\
0.0422	0.9903\\
0.0469	0.9882\\
0.0516	0.9858\\
0.0564	0.9833\\
0.0611	0.9805\\
0.0658	0.9775\\
0.0705	0.9742\\
0.0752	0.9708\\
0.08	0.9671\\
0.0847	0.9633\\
0.0894	0.9592\\
0.0941	0.9548\\
0.0989	0.9503\\
0.1036	0.9455\\
0.1083	0.9406\\
0.113	0.9354\\
0.1177	0.93\\
0.1225	0.9243\\
0.1272	0.9185\\
0.1319	0.9124\\
0.1366	0.9061\\
0.1414	0.8996\\
0.1461	0.8929\\
0.1508	0.8859\\
0.1555	0.8788\\
0.1602	0.8714\\
0.165	0.8638\\
0.1697	0.856\\
0.1744	0.8479\\
0.1791	0.8397\\
0.1816	0.8353\\
0.1864	0.8265\\
0.1889	0.8219\\
};
\addplot [color=black, forget plot]
  table[row sep=crcr]{%
0.1889	0.8219\\
0.1936	0.813\\
0.1983	0.8039\\
0.2031	0.7945\\
0.2078	0.7849\\
0.2125	0.7751\\
0.2172	0.7651\\
0.2219	0.7548\\
0.2267	0.7444\\
0.2314	0.7337\\
0.2361	0.7228\\
0.2408	0.7117\\
0.2456	0.7004\\
0.2503	0.6888\\
0.255	0.677\\
0.2597	0.665\\
0.2644	0.6528\\
0.2692	0.6404\\
0.2739	0.6278\\
0.2786	0.6149\\
0.2833	0.6018\\
0.2881	0.5885\\
0.2928	0.575\\
0.2975	0.5612\\
0.3022	0.5473\\
0.3069	0.5331\\
0.3117	0.5187\\
0.3164	0.5041\\
0.3211	0.4893\\
0.3258	0.4742\\
0.3306	0.4589\\
0.3353	0.4435\\
0.34	0.4277\\
0.3447	0.4118\\
0.3494	0.3957\\
0.3542	0.3793\\
0.3589	0.3627\\
0.3636	0.3459\\
0.3683	0.3289\\
0.3731	0.3117\\
0.3778	0.2942\\
};
\addplot [color=black, forget plot]
  table[row sep=crcr]{%
0.3778	0.2942\\
0.3818	0.2793\\
0.3857	0.2643\\
0.3897	0.2492\\
0.3937	0.2338\\
0.3984	0.2154\\
0.4031	0.1968\\
0.4078	0.1779\\
0.4126	0.1589\\
0.4173	0.1396\\
0.422	0.1201\\
0.4267	0.1003\\
0.4314	0.0804\\
0.4361	0.0606\\
0.4407	0.0406\\
0.4453	0.0204\\
0.45	-0\\
};
\addplot [color=black, forget plot]
  table[row sep=crcr]{%
0.45	0\\
0.45	0.0002\\
0.4501	0.0005\\
0.4502	0.0007\\
0.4503	0.001\\
0.4503	0.0012\\
0.4507	0.0025\\
0.4511	0.0037\\
0.4515	0.005\\
0.4519	0.0062\\
0.4537	0.0124\\
0.4575	0.0248\\
0.4594	0.0309\\
0.4623	0.0403\\
0.4653	0.0496\\
0.4682	0.0588\\
0.474	0.077\\
0.4798	0.0948\\
0.4828	0.1036\\
0.4857	0.1123\\
0.4915	0.1295\\
0.4973	0.1463\\
0.5003	0.1546\\
0.5032	0.1628\\
0.509	0.179\\
0.5148	0.1948\\
0.5178	0.2026\\
0.5207	0.2103\\
0.5265	0.2255\\
0.5323	0.2403\\
0.5353	0.2476\\
0.5382	0.2548\\
0.544	0.269\\
0.5469	0.2759\\
0.5499	0.2828\\
0.5528	0.2896\\
0.5557	0.2963\\
0.5615	0.3095\\
0.5644	0.3159\\
0.565	0.3171\\
0.5656	0.3184\\
0.5661	0.3196\\
0.5667	0.3208\\
};
\addplot [color=black, forget plot]
  table[row sep=crcr]{%
0.5667	0.3208\\
0.5714	0.331\\
0.5761	0.3409\\
0.5808	0.3506\\
0.5856	0.3602\\
0.5903	0.3695\\
0.595	0.3785\\
0.5997	0.3874\\
0.6044	0.396\\
0.6092	0.4044\\
0.6139	0.4126\\
0.6186	0.4206\\
0.6233	0.4284\\
0.6281	0.4359\\
0.6328	0.4433\\
0.6375	0.4504\\
0.6422	0.4573\\
0.6469	0.4639\\
0.6517	0.4704\\
0.6564	0.4766\\
0.6611	0.4826\\
0.6658	0.4884\\
0.6706	0.494\\
0.6753	0.4994\\
0.68	0.5045\\
0.6847	0.5094\\
0.6894	0.5141\\
0.6942	0.5186\\
0.6989	0.5229\\
0.7036	0.5269\\
0.7083	0.5307\\
0.7131	0.5343\\
0.7178	0.5377\\
0.7225	0.5409\\
0.7272	0.5438\\
0.7319	0.5466\\
0.7367	0.5491\\
0.7414	0.5514\\
0.7461	0.5535\\
0.7508	0.5553\\
0.7556	0.557\\
};
\addplot [color=black, forget plot]
  table[row sep=crcr]{%
0.7556	0.557\\
0.7572	0.5575\\
0.7589	0.558\\
0.7605	0.5585\\
0.7622	0.5589\\
0.7669	0.56\\
0.7716	0.5609\\
0.7764	0.5616\\
0.7811	0.562\\
0.7829	0.5622\\
0.7848	0.5622\\
0.7867	0.5623\\
0.7886	0.5623\\
};
\addplot [color=black, forget plot]
  table[row sep=crcr]{%
0.7886	0.5623\\
0.7918	0.5623\\
0.7982	0.5619\\
0.8046	0.5611\\
0.8085	0.5604\\
0.8124	0.5595\\
0.8163	0.5585\\
0.8202	0.5574\\
0.824	0.5561\\
0.8279	0.5547\\
0.8318	0.5531\\
0.8357	0.5514\\
0.8396	0.5495\\
0.8435	0.5475\\
0.8474	0.5453\\
0.8513	0.543\\
0.8552	0.5405\\
0.8591	0.5379\\
0.863	0.5351\\
0.8669	0.5322\\
0.8708	0.5291\\
0.8747	0.5259\\
0.8786	0.5225\\
0.8825	0.519\\
0.8864	0.5154\\
0.8942	0.5076\\
0.8981	0.5035\\
0.902	0.4992\\
0.9059	0.4948\\
0.9098	0.4902\\
0.9137	0.4855\\
0.9176	0.4807\\
0.9215	0.4757\\
0.9254	0.4705\\
0.9293	0.4652\\
0.9331	0.4599\\
0.9407	0.4489\\
0.9444	0.4431\\
};
\addplot [color=black, forget plot]
  table[row sep=crcr]{%
0.9444	0.4431\\
0.9492	0.4358\\
0.9539	0.4282\\
0.9586	0.4205\\
0.9633	0.4125\\
0.9681	0.4043\\
0.9728	0.3959\\
0.9775	0.3872\\
0.9822	0.3784\\
0.9869	0.3693\\
0.9917	0.36\\
0.9964	0.3505\\
1.0011	0.3407\\
1.0058	0.3308\\
1.0106	0.3206\\
1.0153	0.3102\\
1.02	0.2996\\
1.0247	0.2888\\
1.0294	0.2777\\
1.0342	0.2664\\
1.0389	0.255\\
1.0436	0.2432\\
1.0483	0.2313\\
1.0531	0.2192\\
1.0578	0.2068\\
1.0625	0.1942\\
1.0672	0.1814\\
1.0719	0.1684\\
1.0767	0.1552\\
1.0814	0.1417\\
1.0861	0.1281\\
1.0908	0.1142\\
1.0956	0.1\\
1.1003	0.0857\\
1.105	0.0712\\
1.1097	0.0564\\
1.1144	0.0414\\
1.1176	0.0312\\
1.1208	0.0209\\
1.124	0.0105\\
1.1272	-0\\
};
\addplot [color=black, forget plot]
  table[row sep=crcr]{%
1.1272	0\\
1.1272	0.0002\\
1.1273	0.0005\\
1.1274	0.0007\\
1.1275	0.001\\
1.1278	0.0016\\
1.1279	0.002\\
1.1281	0.0024\\
1.1283	0.0027\\
1.1284	0.0031\\
1.1286	0.0035\\
1.1287	0.0039\\
1.1289	0.0043\\
1.129	0.0047\\
1.1292	0.005\\
1.1293	0.0054\\
1.1295	0.0058\\
1.1296	0.0062\\
1.1298	0.0066\\
1.13	0.0069\\
1.1301	0.0073\\
1.1303	0.0077\\
1.1304	0.0081\\
1.1306	0.0085\\
1.1307	0.0088\\
1.1309	0.0092\\
1.131	0.0096\\
1.1312	0.01\\
1.1313	0.0104\\
1.1315	0.0107\\
1.1317	0.0111\\
1.1318	0.0115\\
1.132	0.0119\\
1.1321	0.0123\\
1.1323	0.0126\\
1.1324	0.013\\
1.1326	0.0134\\
1.1327	0.0138\\
1.1329	0.0141\\
1.133	0.0145\\
1.1332	0.0149\\
1.1332	0.015\\
1.1333	0.0151\\
1.1333	0.0152\\
};
\addplot [color=black, forget plot]
  table[row sep=crcr]{%
1.1333	0.0152\\
1.1336	0.016\\
1.134	0.0167\\
1.1346	0.0183\\
1.1362	0.0221\\
1.1377	0.0258\\
1.1393	0.0296\\
1.1409	0.0333\\
1.1456	0.0443\\
1.1503	0.0551\\
1.155	0.0657\\
1.1598	0.076\\
1.1645	0.0862\\
1.1692	0.0961\\
1.1739	0.1058\\
1.1787	0.1153\\
1.1834	0.1246\\
1.1881	0.1336\\
1.1928	0.1424\\
1.1975	0.1511\\
1.2023	0.1595\\
1.207	0.1676\\
1.2117	0.1756\\
1.2164	0.1833\\
1.2212	0.1908\\
1.2259	0.1981\\
1.2306	0.2052\\
1.2353	0.2121\\
1.24	0.2187\\
1.2448	0.2251\\
1.2495	0.2314\\
1.2542	0.2373\\
1.2589	0.2431\\
1.2637	0.2487\\
1.2684	0.254\\
1.2731	0.2591\\
1.2778	0.264\\
1.2825	0.2687\\
1.2873	0.2731\\
1.292	0.2774\\
1.2967	0.2814\\
1.3014	0.2852\\
1.3062	0.2888\\
1.3109	0.2921\\
1.3137	0.294\\
1.3166	0.2959\\
1.3222	0.2993\\
};
\addplot [color=black, forget plot]
  table[row sep=crcr]{%
1.3222	0.2993\\
1.3252	0.301\\
1.3281	0.3025\\
1.3311	0.304\\
1.3341	0.3054\\
1.3388	0.3075\\
1.3435	0.3094\\
1.3482	0.311\\
1.3529	0.3124\\
1.3577	0.3136\\
1.3624	0.3146\\
1.3671	0.3153\\
1.3718	0.3159\\
1.3741	0.3161\\
1.3765	0.3162\\
1.3788	0.3163\\
1.3811	0.3163\\
};
\addplot [color=black, forget plot]
  table[row sep=crcr]{%
1.3811	0.3163\\
1.3836	0.3163\\
1.3843	0.3162\\
1.3875	0.3161\\
1.3939	0.3155\\
1.4003	0.3145\\
1.4036	0.3138\\
1.4068	0.313\\
1.4101	0.3122\\
1.4133	0.3112\\
1.4166	0.3101\\
1.4198	0.3089\\
1.4231	0.3076\\
1.4263	0.3063\\
1.4296	0.3048\\
1.4328	0.3032\\
1.4361	0.3015\\
1.4393	0.2997\\
1.4426	0.2977\\
1.4458	0.2957\\
1.4491	0.2936\\
1.4523	0.2914\\
1.4556	0.2891\\
1.4588	0.2866\\
1.4621	0.2841\\
1.4653	0.2815\\
1.4686	0.2787\\
1.4718	0.2759\\
1.4784	0.2699\\
1.4816	0.2667\\
1.4849	0.2635\\
1.4881	0.2601\\
1.4914	0.2567\\
1.4946	0.2531\\
1.4979	0.2494\\
1.5011	0.2456\\
1.5036	0.2427\\
1.5111	0.2334\\
};
\addplot [color=black, forget plot]
  table[row sep=crcr]{%
1.5111	0.2334\\
1.5158	0.2272\\
1.5206	0.2209\\
1.5253	0.2143\\
1.53	0.2075\\
1.5347	0.2005\\
1.5394	0.1933\\
1.5442	0.1859\\
1.5489	0.1782\\
1.5536	0.1703\\
1.5583	0.1622\\
1.5631	0.1539\\
1.5678	0.1453\\
1.5725	0.1366\\
1.5772	0.1276\\
1.5819	0.1184\\
1.5867	0.109\\
1.5914	0.0994\\
1.5961	0.0895\\
1.6008	0.0794\\
1.6056	0.0692\\
1.6103	0.0587\\
1.615	0.0479\\
1.6197	0.037\\
1.6244	0.0258\\
1.6271	0.0195\\
1.6297	0.013\\
1.6324	0.0066\\
1.635	-0\\
};
\addplot [color=black, forget plot]
  table[row sep=crcr]{%
1.635	0\\
1.6351	0.0001\\
1.6351	0.0002\\
1.6353	0.0005\\
1.6354	0.0007\\
1.6355	0.001\\
1.6357	0.0012\\
1.6363	0.0025\\
1.6377	0.0049\\
1.6384	0.0062\\
1.64	0.0091\\
1.6416	0.0121\\
1.6432	0.015\\
1.6449	0.0179\\
1.6465	0.0208\\
1.6497	0.0264\\
1.6514	0.0292\\
1.653	0.032\\
1.6562	0.0374\\
1.6579	0.0401\\
1.6595	0.0427\\
1.6611	0.0454\\
1.6627	0.048\\
1.6644	0.0506\\
1.666	0.0531\\
1.6676	0.0557\\
1.6692	0.0582\\
1.6708	0.0606\\
1.6725	0.0631\\
1.6773	0.0703\\
1.679	0.0726\\
1.6838	0.0795\\
1.6855	0.0818\\
1.6887	0.0862\\
1.6903	0.0883\\
1.692	0.0905\\
1.6952	0.0947\\
1.7	0.1007\\
};
\addplot [color=black, forget plot]
  table[row sep=crcr]{%
0	0.981\\
0.0029	0.981\\
0.0043	0.9809\\
0.0073	0.9807\\
0.012	0.9802\\
0.0167	0.9795\\
0.0214	0.9785\\
0.0262	0.9774\\
0.0309	0.976\\
0.0356	0.9744\\
0.0403	0.9726\\
0.0451	0.9706\\
0.0498	0.9683\\
0.0545	0.9658\\
0.0592	0.9631\\
0.0639	0.9602\\
0.0687	0.9571\\
0.0734	0.9538\\
0.0781	0.9502\\
0.0828	0.9464\\
0.0876	0.9424\\
0.0923	0.9382\\
0.097	0.9338\\
0.1017	0.9291\\
0.1064	0.9242\\
0.1112	0.9191\\
0.1159	0.9138\\
0.1206	0.9083\\
0.1253	0.9025\\
0.1301	0.8966\\
0.1348	0.8904\\
0.1395	0.884\\
0.1442	0.8774\\
0.1489	0.8705\\
0.1537	0.8634\\
0.1584	0.8562\\
0.1631	0.8487\\
0.1678	0.8409\\
0.1726	0.833\\
0.1773	0.8248\\
0.1802	0.8197\\
0.1831	0.8145\\
0.1889	0.8039\\
};
\addplot [color=black, forget plot]
  table[row sep=crcr]{%
0.1889	0.8039\\
0.1936	0.7949\\
0.1983	0.7858\\
0.2031	0.7765\\
0.2078	0.7669\\
0.2125	0.7571\\
0.2172	0.7471\\
0.2219	0.7369\\
0.2267	0.7264\\
0.2314	0.7158\\
0.2361	0.7049\\
0.2408	0.6938\\
0.2456	0.6825\\
0.2503	0.6709\\
0.255	0.6592\\
0.2597	0.6472\\
0.2644	0.635\\
0.2692	0.6226\\
0.2739	0.6099\\
0.2786	0.5971\\
0.2833	0.584\\
0.2881	0.5707\\
0.2928	0.5572\\
0.2975	0.5435\\
0.3022	0.5296\\
0.3069	0.5154\\
0.3117	0.501\\
0.3164	0.4864\\
0.3211	0.4716\\
0.3258	0.4565\\
0.3306	0.4413\\
0.3353	0.4258\\
0.34	0.4101\\
0.3447	0.3942\\
0.3494	0.3781\\
0.3542	0.3617\\
0.3589	0.3451\\
0.3636	0.3284\\
0.3683	0.3113\\
0.3731	0.2941\\
0.3778	0.2767\\
};
\addplot [color=black, forget plot]
  table[row sep=crcr]{%
0.3778	0.2767\\
0.3815	0.2627\\
0.3853	0.2486\\
0.389	0.2344\\
0.3927	0.22\\
0.3975	0.2016\\
0.4022	0.183\\
0.4069	0.1643\\
0.4116	0.1452\\
0.4163	0.126\\
0.4211	0.1066\\
0.4258	0.0869\\
0.4305	0.067\\
0.4344	0.0505\\
0.4383	0.0338\\
0.4422	0.017\\
0.4461	0\\
};
\addplot [color=black, forget plot]
  table[row sep=crcr]{%
0.4461	0\\
0.4461	0.0002\\
0.4462	0.0005\\
0.4463	0.0007\\
0.4463	0.001\\
0.4464	0.0012\\
0.4468	0.0025\\
0.4472	0.0037\\
0.4476	0.005\\
0.4479	0.0062\\
0.4499	0.0124\\
0.4537	0.0248\\
0.4556	0.0309\\
0.4586	0.0405\\
0.4616	0.05\\
0.4646	0.0594\\
0.4676	0.0687\\
0.4707	0.078\\
0.4767	0.0962\\
0.4797	0.1052\\
0.4827	0.1141\\
0.4887	0.1315\\
0.4918	0.1402\\
0.4948	0.1487\\
0.5008	0.1655\\
0.5068	0.1819\\
0.5099	0.19\\
0.5129	0.198\\
0.5159	0.2059\\
0.5189	0.2137\\
0.5219	0.2214\\
0.5279	0.2366\\
0.531	0.244\\
0.534	0.2514\\
0.537	0.2587\\
0.54	0.2659\\
0.543	0.273\\
0.546	0.28\\
0.5491	0.2869\\
0.5551	0.3005\\
0.5611	0.3137\\
0.5641	0.3201\\
0.5648	0.3215\\
0.5654	0.3228\\
0.566	0.3242\\
0.5667	0.3255\\
};
\addplot [color=black, forget plot]
  table[row sep=crcr]{%
0.5667	0.3255\\
0.5714	0.3354\\
0.5761	0.345\\
0.5808	0.3544\\
0.5856	0.3636\\
0.5903	0.3725\\
0.595	0.3813\\
0.5997	0.3898\\
0.6044	0.3981\\
0.6092	0.4062\\
0.6139	0.4141\\
0.6186	0.4217\\
0.6233	0.4292\\
0.6281	0.4364\\
0.6328	0.4434\\
0.6375	0.4502\\
0.6422	0.4567\\
0.6469	0.4631\\
0.6517	0.4692\\
0.6564	0.4751\\
0.6611	0.4808\\
0.6658	0.4862\\
0.6706	0.4915\\
0.6753	0.4965\\
0.68	0.5013\\
0.6847	0.5059\\
0.6894	0.5103\\
0.6942	0.5144\\
0.6989	0.5184\\
0.7036	0.5221\\
0.7083	0.5256\\
0.7131	0.5289\\
0.7178	0.5319\\
0.7225	0.5348\\
0.7272	0.5374\\
0.7319	0.5398\\
0.7367	0.542\\
0.7414	0.544\\
0.7461	0.5457\\
0.7508	0.5472\\
0.7556	0.5485\\
};
\addplot [color=black, forget plot]
  table[row sep=crcr]{%
0.7556	0.5485\\
0.7569	0.5489\\
0.7595	0.5495\\
0.7608	0.5497\\
0.7655	0.5506\\
0.7702	0.5512\\
0.7749	0.5516\\
0.7797	0.5518\\
0.7815	0.5518\\
};
\addplot [color=black, forget plot]
  table[row sep=crcr]{%
0.7815	0.5518\\
0.7847	0.5518\\
0.7911	0.5514\\
0.7975	0.5506\\
0.8015	0.5499\\
0.8056	0.549\\
0.8097	0.5479\\
0.8138	0.5467\\
0.8178	0.5453\\
0.8219	0.5438\\
0.826	0.5421\\
0.8301	0.5403\\
0.8341	0.5382\\
0.8382	0.536\\
0.8423	0.5337\\
0.8464	0.5312\\
0.8504	0.5285\\
0.8545	0.5257\\
0.8586	0.5227\\
0.8627	0.5195\\
0.8667	0.5162\\
0.8708	0.5127\\
0.8749	0.509\\
0.879	0.5052\\
0.8871	0.4971\\
0.8912	0.4928\\
0.8953	0.4883\\
0.8993	0.4837\\
0.9034	0.4789\\
0.9075	0.474\\
0.9116	0.4688\\
0.9156	0.4636\\
0.9197	0.4581\\
0.9238	0.4525\\
0.9278	0.4467\\
0.932	0.4407\\
0.9361	0.4345\\
0.9403	0.4281\\
0.9444	0.4216\\
};
\addplot [color=black, forget plot]
  table[row sep=crcr]{%
0.9444	0.4216\\
0.9492	0.4139\\
0.9539	0.406\\
0.9586	0.3979\\
0.9633	0.3896\\
0.9681	0.3811\\
0.9728	0.3723\\
0.9775	0.3633\\
0.9822	0.3542\\
0.9869	0.3447\\
0.9917	0.3351\\
0.9964	0.3253\\
1.0011	0.3152\\
1.0058	0.3049\\
1.0106	0.2944\\
1.0153	0.2837\\
1.02	0.2728\\
1.0247	0.2616\\
1.0294	0.2502\\
1.0342	0.2386\\
1.0389	0.2268\\
1.0436	0.2148\\
1.0483	0.2025\\
1.0531	0.19\\
1.0578	0.1774\\
1.0625	0.1644\\
1.0672	0.1513\\
1.0719	0.138\\
1.0767	0.1244\\
1.0814	0.1106\\
1.0861	0.0966\\
1.0908	0.0824\\
1.0956	0.068\\
1.1003	0.0533\\
1.105	0.0384\\
1.1097	0.0233\\
1.1144	0.008\\
1.1151	0.006\\
1.1169	-0\\
};
\addplot [color=black, forget plot]
  table[row sep=crcr]{%
1.1169	0\\
1.1169	0.0002\\
1.117	0.0002\\
1.1171	0.0005\\
1.1172	0.0007\\
1.1173	0.001\\
1.1174	0.0012\\
1.1194	0.0062\\
1.1198	0.0073\\
1.1203	0.0083\\
1.1211	0.0103\\
1.1215	0.0112\\
1.1235	0.0162\\
1.124	0.0172\\
1.1248	0.0192\\
1.1252	0.0201\\
1.1268	0.0241\\
1.1272	0.025\\
1.1277	0.026\\
1.1281	0.027\\
1.1285	0.0279\\
1.1293	0.0299\\
1.1297	0.0308\\
1.1305	0.0328\\
1.1309	0.0337\\
1.1314	0.0347\\
1.1318	0.0356\\
1.1322	0.0366\\
1.1325	0.0373\\
1.1328	0.0379\\
1.133	0.0386\\
1.1333	0.0393\\
};
\addplot [color=black, forget plot]
  table[row sep=crcr]{%
1.1333	0.0393\\
1.1342	0.0412\\
1.135	0.0432\\
1.1359	0.0451\\
1.1368	0.0471\\
1.141	0.0567\\
1.1453	0.0662\\
1.1496	0.0754\\
1.1539	0.0845\\
1.1586	0.0944\\
1.1633	0.104\\
1.168	0.1134\\
1.1727	0.1225\\
1.1775	0.1315\\
1.1822	0.1402\\
1.1869	0.1488\\
1.1916	0.1571\\
1.1964	0.1651\\
1.2011	0.173\\
1.2058	0.1806\\
1.2105	0.1881\\
1.2152	0.1953\\
1.22	0.2023\\
1.2247	0.209\\
1.2294	0.2156\\
1.2341	0.2219\\
1.2389	0.228\\
1.2436	0.2339\\
1.2483	0.2396\\
1.253	0.2451\\
1.2577	0.2503\\
1.2625	0.2553\\
1.2672	0.2601\\
1.2719	0.2647\\
1.2766	0.2691\\
1.2814	0.2732\\
1.2861	0.2771\\
1.2908	0.2808\\
1.2955	0.2843\\
1.3002	0.2876\\
1.305	0.2906\\
1.3093	0.2932\\
1.3136	0.2956\\
1.3179	0.2979\\
1.3222	0.2999\\
};
\addplot [color=black, forget plot]
  table[row sep=crcr]{%
1.3222	0.2999\\
1.3245	0.301\\
1.3269	0.3019\\
1.3315	0.3037\\
1.3362	0.3053\\
1.341	0.3067\\
1.3457	0.3079\\
1.3504	0.3088\\
1.3549	0.3095\\
1.3594	0.31\\
1.3639	0.3103\\
1.3684	0.3104\\
};
\addplot [color=black, forget plot]
  table[row sep=crcr]{%
1.3684	0.3104\\
1.3716	0.3104\\
1.378	0.31\\
1.3844	0.3092\\
1.3916	0.3078\\
1.3951	0.3069\\
1.3987	0.3059\\
1.4023	0.3048\\
1.4058	0.3035\\
1.4094	0.3022\\
1.413	0.3007\\
1.4165	0.2991\\
1.4201	0.2973\\
1.4237	0.2954\\
1.4272	0.2934\\
1.4308	0.2913\\
1.4344	0.2891\\
1.4379	0.2867\\
1.4415	0.2842\\
1.4451	0.2816\\
1.4486	0.2789\\
1.4522	0.276\\
1.4558	0.273\\
1.4593	0.2699\\
1.4665	0.2633\\
1.4736	0.2562\\
1.4772	0.2524\\
1.4807	0.2486\\
1.4843	0.2446\\
1.4879	0.2404\\
1.4914	0.2362\\
1.495	0.2318\\
1.4986	0.2273\\
1.5017	0.2233\\
1.5048	0.2192\\
1.508	0.2149\\
1.5111	0.2106\\
};
\addplot [color=black, forget plot]
  table[row sep=crcr]{%
1.5111	0.2106\\
1.5158	0.2039\\
1.5206	0.1969\\
1.5253	0.1898\\
1.53	0.1824\\
1.5347	0.1748\\
1.5394	0.167\\
1.5442	0.159\\
1.5489	0.1507\\
1.5536	0.1422\\
1.5583	0.1336\\
1.5631	0.1246\\
1.5678	0.1155\\
1.5725	0.1062\\
1.5772	0.0966\\
1.5819	0.0868\\
1.5867	0.0768\\
1.5914	0.0666\\
1.5961	0.0562\\
1.6008	0.0455\\
1.6056	0.0346\\
1.6092	0.0262\\
1.6128	0.0176\\
1.6164	0.0089\\
1.62	-0\\
};
\addplot [color=black, forget plot]
  table[row sep=crcr]{%
1.62	0\\
1.62	0.0001\\
1.6201	0.0001\\
1.6201	0.0002\\
1.6203	0.0005\\
1.6204	0.0007\\
1.6205	0.001\\
1.6207	0.0012\\
1.6213	0.0025\\
1.6227	0.0049\\
1.6234	0.0062\\
1.6274	0.0134\\
1.6334	0.0239\\
1.6374	0.0307\\
1.6434	0.0406\\
1.6474	0.047\\
1.6534	0.0563\\
1.6594	0.0653\\
1.6614	0.0682\\
1.6634	0.071\\
1.6654	0.0739\\
1.6674	0.0767\\
1.6734	0.0848\\
1.6794	0.0926\\
1.6834	0.0976\\
1.6894	0.1048\\
1.6934	0.1094\\
1.6954	0.1116\\
1.6965	0.1129\\
1.6977	0.1142\\
1.6988	0.1154\\
1.7	0.1167\\
};
\addplot [color=black, forget plot]
  table[row sep=crcr]{%
0	1.0483\\
0.0011	1.0483\\
0.0016	1.0482\\
0.0028	1.0482\\
0.0058	1.048\\
0.0116	1.0474\\
0.0146	1.0469\\
0.0193	1.046\\
0.024	1.0449\\
0.0287	1.0436\\
0.0334	1.042\\
0.0382	1.0403\\
0.0429	1.0383\\
0.0476	1.0361\\
0.0523	1.0336\\
0.0571	1.031\\
0.0618	1.0281\\
0.0665	1.0251\\
0.0712	1.0218\\
0.0759	1.0182\\
0.0807	1.0145\\
0.0854	1.0106\\
0.0901	1.0064\\
0.0948	1.002\\
0.0996	0.9974\\
0.1043	0.9926\\
0.109	0.9875\\
0.1137	0.9822\\
0.1184	0.9768\\
0.1232	0.971\\
0.1279	0.9651\\
0.1326	0.959\\
0.1373	0.9526\\
0.1421	0.946\\
0.1468	0.9392\\
0.1515	0.9322\\
0.1562	0.925\\
0.1609	0.9175\\
0.1657	0.9099\\
0.1704	0.902\\
0.1751	0.8939\\
0.1798	0.8855\\
0.1846	0.877\\
0.1856	0.875\\
0.1878	0.871\\
0.1889	0.8689\\
};
\addplot [color=black, forget plot]
  table[row sep=crcr]{%
0.1889	0.8689\\
0.1936	0.86\\
0.1983	0.8508\\
0.2031	0.8414\\
0.2078	0.8318\\
0.2125	0.8219\\
0.2172	0.8119\\
0.2219	0.8016\\
0.2267	0.7911\\
0.2314	0.7804\\
0.2361	0.7694\\
0.2408	0.7583\\
0.2456	0.7469\\
0.2503	0.7353\\
0.255	0.7235\\
0.2597	0.7115\\
0.2644	0.6992\\
0.2692	0.6867\\
0.2739	0.6741\\
0.2786	0.6611\\
0.2833	0.648\\
0.2881	0.6347\\
0.2928	0.6211\\
0.2975	0.6073\\
0.3022	0.5933\\
0.3069	0.5791\\
0.3117	0.5647\\
0.3164	0.55\\
0.3211	0.5352\\
0.3258	0.5201\\
0.3306	0.5048\\
0.3353	0.4892\\
0.34	0.4735\\
0.3447	0.4575\\
0.3494	0.4413\\
0.3542	0.4249\\
0.3589	0.4083\\
0.3636	0.3914\\
0.3683	0.3744\\
0.3731	0.3571\\
0.3778	0.3396\\
};
\addplot [color=black, forget plot]
  table[row sep=crcr]{%
0.3778	0.3396\\
0.3824	0.3224\\
0.3869	0.3051\\
0.3915	0.2875\\
0.3961	0.2697\\
0.4008	0.2511\\
0.4055	0.2324\\
0.4102	0.2134\\
0.415	0.1941\\
0.4197	0.1747\\
0.4244	0.155\\
0.4291	0.1352\\
0.4339	0.1151\\
0.4386	0.0947\\
0.4433	0.0742\\
0.448	0.0535\\
0.4527	0.0325\\
0.4546	0.0244\\
0.4582	0.0082\\
0.46	-0\\
};
\addplot [color=black, forget plot]
  table[row sep=crcr]{%
0.46	0\\
0.46	0.0002\\
0.4601	0.0005\\
0.4602	0.0007\\
0.4602	0.001\\
0.4603	0.0012\\
0.4607	0.0025\\
0.4611	0.0037\\
0.4614	0.005\\
0.4618	0.0062\\
0.4636	0.0124\\
0.4655	0.0186\\
0.4673	0.0248\\
0.4692	0.0309\\
0.4718	0.0397\\
0.4745	0.0485\\
0.4799	0.0657\\
0.4825	0.0742\\
0.4852	0.0827\\
0.4879	0.091\\
0.4905	0.0994\\
0.4959	0.1158\\
0.4985	0.1239\\
0.5039	0.1399\\
0.5065	0.1477\\
0.5092	0.1556\\
0.5119	0.1633\\
0.5145	0.171\\
0.5172	0.1786\\
0.5199	0.1861\\
0.5225	0.1936\\
0.5252	0.201\\
0.5279	0.2083\\
0.5305	0.2156\\
0.5332	0.2228\\
0.5359	0.2299\\
0.5385	0.237\\
0.5412	0.244\\
0.5439	0.2509\\
0.5465	0.2577\\
0.5492	0.2645\\
0.5519	0.2712\\
0.5545	0.2778\\
0.5572	0.2844\\
0.5599	0.2909\\
0.5625	0.2973\\
0.5652	0.3037\\
0.5656	0.3045\\
0.5659	0.3054\\
0.5663	0.3062\\
0.5667	0.3071\\
};
\addplot [color=black, forget plot]
  table[row sep=crcr]{%
0.5667	0.3071\\
0.5714	0.3181\\
0.5761	0.3289\\
0.5808	0.3395\\
0.5856	0.3498\\
0.5903	0.36\\
0.595	0.3699\\
0.5997	0.3796\\
0.6044	0.389\\
0.6092	0.3983\\
0.6139	0.4073\\
0.6186	0.4162\\
0.6233	0.4248\\
0.6281	0.4332\\
0.6328	0.4413\\
0.6375	0.4493\\
0.6422	0.457\\
0.6469	0.4645\\
0.6517	0.4718\\
0.6564	0.4789\\
0.6611	0.4857\\
0.6658	0.4924\\
0.6706	0.4988\\
0.6753	0.505\\
0.68	0.5109\\
0.6847	0.5167\\
0.6894	0.5222\\
0.6942	0.5276\\
0.6989	0.5327\\
0.7036	0.5376\\
0.7083	0.5422\\
0.7131	0.5467\\
0.7178	0.5509\\
0.7225	0.5549\\
0.7272	0.5587\\
0.7319	0.5623\\
0.7367	0.5656\\
0.7414	0.5688\\
0.7461	0.5717\\
0.7508	0.5744\\
0.7556	0.5768\\
};
\addplot [color=black, forget plot]
  table[row sep=crcr]{%
0.7556	0.5768\\
0.7581	0.5781\\
0.7607	0.5793\\
0.7633	0.5804\\
0.7658	0.5815\\
0.7706	0.5833\\
0.7753	0.5848\\
0.78	0.5862\\
0.7847	0.5873\\
0.7894	0.5882\\
0.7942	0.5889\\
0.7989	0.5894\\
0.8036	0.5896\\
0.8044	0.5896\\
0.8051	0.5897\\
0.8067	0.5897\\
};
\addplot [color=black, forget plot]
  table[row sep=crcr]{%
0.8067	0.5897\\
0.8086	0.5897\\
0.8092	0.5896\\
0.8099	0.5896\\
0.8131	0.5895\\
0.8195	0.5889\\
0.8227	0.5884\\
0.8261	0.5878\\
0.8296	0.5871\\
0.833	0.5863\\
0.8365	0.5853\\
0.8399	0.5843\\
0.8434	0.5831\\
0.8468	0.5818\\
0.8502	0.5804\\
0.8537	0.5788\\
0.8571	0.5772\\
0.8606	0.5754\\
0.864	0.5736\\
0.8675	0.5716\\
0.8709	0.5695\\
0.8743	0.5672\\
0.8778	0.5649\\
0.8812	0.5624\\
0.8847	0.5598\\
0.8881	0.5571\\
0.8916	0.5543\\
0.895	0.5514\\
0.8985	0.5484\\
0.9019	0.5452\\
0.9053	0.5419\\
0.9088	0.5385\\
0.9122	0.535\\
0.9157	0.5314\\
0.9191	0.5277\\
0.9226	0.5238\\
0.9294	0.5158\\
0.9329	0.5116\\
0.9416	0.5005\\
0.9444	0.4966\\
};
\addplot [color=black, forget plot]
  table[row sep=crcr]{%
0.9444	0.4966\\
0.9492	0.4901\\
0.9539	0.4834\\
0.9586	0.4765\\
0.9633	0.4693\\
0.9681	0.462\\
0.9728	0.4544\\
0.9775	0.4466\\
0.9822	0.4385\\
0.9869	0.4303\\
0.9917	0.4218\\
0.9964	0.4132\\
1.0011	0.4043\\
1.0058	0.3952\\
1.0106	0.3858\\
1.0153	0.3763\\
1.02	0.3665\\
1.0247	0.3565\\
1.0294	0.3463\\
1.0342	0.3359\\
1.0389	0.3252\\
1.0436	0.3144\\
1.0483	0.3033\\
1.0531	0.292\\
1.0578	0.2804\\
1.0625	0.2687\\
1.0672	0.2567\\
1.0719	0.2446\\
1.0767	0.2322\\
1.0814	0.2195\\
1.0861	0.2067\\
1.0908	0.1937\\
1.0956	0.1804\\
1.1003	0.1669\\
1.105	0.1532\\
1.1097	0.1393\\
1.1144	0.1251\\
1.1192	0.1107\\
1.1239	0.0962\\
1.1286	0.0814\\
1.1333	0.0663\\
};
\addplot [color=black, forget plot]
  table[row sep=crcr]{%
1.1333	0.0663\\
1.1344	0.063\\
1.1354	0.0596\\
1.1365	0.0563\\
1.1375	0.0529\\
1.1415	0.0399\\
1.1455	0.0268\\
1.1494	0.0135\\
1.1534	-0\\
};
\addplot [color=black, forget plot]
  table[row sep=crcr]{%
1.1534	0\\
1.1534	0.0001\\
1.1535	0.0001\\
1.1535	0.0002\\
1.1536	0.0005\\
1.1537	0.0007\\
1.1538	0.001\\
1.1539	0.0012\\
1.1544	0.0025\\
1.1549	0.0037\\
1.1554	0.005\\
1.1559	0.0062\\
1.1583	0.0124\\
1.1608	0.0185\\
1.1632	0.0246\\
1.1657	0.0306\\
1.1699	0.0408\\
1.1741	0.0508\\
1.1784	0.0606\\
1.1826	0.0702\\
1.1868	0.0797\\
1.191	0.089\\
1.1952	0.0981\\
1.1995	0.1071\\
1.2037	0.1158\\
1.2079	0.1244\\
1.2121	0.1329\\
1.2163	0.1411\\
1.2206	0.1492\\
1.2248	0.1571\\
1.229	0.1648\\
1.2332	0.1724\\
1.2374	0.1797\\
1.2417	0.1869\\
1.2459	0.194\\
1.2501	0.2008\\
1.2543	0.2075\\
1.2585	0.214\\
1.2628	0.2203\\
1.267	0.2265\\
1.2712	0.2324\\
1.2754	0.2382\\
1.2796	0.2439\\
1.2839	0.2493\\
1.2881	0.2546\\
1.2923	0.2597\\
1.2965	0.2646\\
1.3007	0.2694\\
1.305	0.274\\
1.3092	0.2784\\
1.3134	0.2826\\
1.3176	0.2866\\
1.3188	0.2877\\
1.3199	0.2888\\
1.3211	0.2898\\
1.3222	0.2909\\
};
\addplot [color=black, forget plot]
  table[row sep=crcr]{%
1.3222	0.2909\\
1.3268	0.2949\\
1.3314	0.2987\\
1.336	0.3022\\
1.3406	0.3056\\
1.3453	0.3089\\
1.35	0.3119\\
1.3547	0.3148\\
1.3594	0.3174\\
1.3642	0.3198\\
1.3689	0.3219\\
1.3736	0.3239\\
1.3783	0.3256\\
1.3831	0.3272\\
1.3878	0.3285\\
1.3925	0.3295\\
1.3972	0.3304\\
1.4013	0.331\\
1.4053	0.3314\\
1.4094	0.3316\\
1.4135	0.3317\\
};
\addplot [color=black, forget plot]
  table[row sep=crcr]{%
1.4135	0.3317\\
1.416	0.3317\\
1.4167	0.3316\\
1.4215	0.3314\\
1.424	0.3311\\
1.4264	0.3309\\
1.4289	0.3305\\
1.4337	0.3297\\
1.4362	0.3292\\
1.4386	0.3286\\
1.4411	0.328\\
1.4435	0.3273\\
1.446	0.3265\\
1.4484	0.3257\\
1.4508	0.3248\\
1.4533	0.3239\\
1.4557	0.3229\\
1.4582	0.3219\\
1.4606	0.3208\\
1.463	0.3196\\
1.4655	0.3184\\
1.4679	0.3171\\
1.4704	0.3158\\
1.4752	0.313\\
1.4777	0.3115\\
1.4801	0.3099\\
1.4826	0.3083\\
1.485	0.3066\\
1.4875	0.3048\\
1.4923	0.3012\\
1.4948	0.2993\\
1.4972	0.2973\\
1.4997	0.2952\\
1.5021	0.2932\\
1.5045	0.291\\
1.5062	0.2895\\
1.5078	0.288\\
1.5095	0.2865\\
1.5111	0.2849\\
};
\addplot [color=black, forget plot]
  table[row sep=crcr]{%
1.5111	0.2849\\
1.5158	0.2803\\
1.5206	0.2754\\
1.5253	0.2704\\
1.53	0.2651\\
1.5347	0.2596\\
1.5394	0.2538\\
1.5442	0.2479\\
1.5489	0.2417\\
1.5536	0.2353\\
1.5583	0.2287\\
1.5631	0.2219\\
1.5678	0.2149\\
1.5725	0.2076\\
1.5772	0.2002\\
1.5819	0.1925\\
1.5867	0.1845\\
1.5914	0.1764\\
1.5961	0.1681\\
1.6008	0.1595\\
1.6056	0.1507\\
1.6103	0.1417\\
1.615	0.1325\\
1.6197	0.123\\
1.6244	0.1134\\
1.6292	0.1035\\
1.6339	0.0934\\
1.6386	0.083\\
1.6433	0.0725\\
1.6481	0.0617\\
1.6528	0.0508\\
1.6575	0.0396\\
1.6622	0.0282\\
1.665	0.0212\\
1.6679	0.0142\\
1.6707	0.0072\\
1.6735	-0\\
};
\addplot [color=black, forget plot]
  table[row sep=crcr]{%
1.6735	0\\
1.6735	0.0001\\
1.6736	0.0001\\
1.6736	0.0002\\
1.6737	0.0005\\
1.6739	0.0007\\
1.674	0.001\\
1.6741	0.0012\\
1.6748	0.0025\\
1.6754	0.0037\\
1.6761	0.0049\\
1.6768	0.0062\\
1.6774	0.0074\\
1.6781	0.0087\\
1.6787	0.0099\\
1.6794	0.0111\\
1.6801	0.0124\\
1.6807	0.0136\\
1.6821	0.016\\
1.6827	0.0172\\
1.6834	0.0184\\
1.684	0.0196\\
1.6854	0.022\\
1.686	0.0232\\
1.6874	0.0256\\
1.688	0.0267\\
1.6887	0.0279\\
1.6893	0.0291\\
1.69	0.0302\\
1.6907	0.0314\\
1.6913	0.0326\\
1.6927	0.0348\\
1.6933	0.036\\
1.694	0.0371\\
1.6946	0.0383\\
1.696	0.0405\\
1.6966	0.0416\\
1.698	0.0438\\
1.699	0.0456\\
1.7	0.0472\\
};
\addplot [color=black, forget plot]
  table[row sep=crcr]{%
0	0.9938\\
0.0016	0.9938\\
0.0025	0.9937\\
0.0034	0.9937\\
0.0043	0.9936\\
0.0087	0.9931\\
0.0132	0.9925\\
0.0177	0.9917\\
0.0221	0.9907\\
0.0269	0.9894\\
0.0316	0.9878\\
0.0363	0.9861\\
0.041	0.9841\\
0.0458	0.982\\
0.0505	0.9796\\
0.0552	0.977\\
0.0599	0.9741\\
0.0646	0.9711\\
0.0694	0.9678\\
0.0741	0.9643\\
0.0788	0.9606\\
0.0835	0.9567\\
0.0883	0.9526\\
0.093	0.9482\\
0.0977	0.9436\\
0.1024	0.9388\\
0.1071	0.9338\\
0.1119	0.9286\\
0.1166	0.9231\\
0.1213	0.9174\\
0.126	0.9115\\
0.1308	0.9054\\
0.1355	0.8991\\
0.1402	0.8925\\
0.1449	0.8858\\
0.1496	0.8788\\
0.1544	0.8716\\
0.1591	0.8641\\
0.1638	0.8565\\
0.1685	0.8486\\
0.1733	0.8406\\
0.1772	0.8337\\
0.1811	0.8267\\
0.185	0.8195\\
0.1889	0.8122\\
};
\addplot [color=black, forget plot]
  table[row sep=crcr]{%
0.1889	0.8122\\
0.1936	0.8032\\
0.1983	0.794\\
0.2031	0.7845\\
0.2078	0.7748\\
0.2125	0.7649\\
0.2172	0.7548\\
0.2219	0.7445\\
0.2267	0.7339\\
0.2314	0.7231\\
0.2361	0.7122\\
0.2408	0.7009\\
0.2456	0.6895\\
0.2503	0.6779\\
0.255	0.666\\
0.2597	0.6539\\
0.2644	0.6416\\
0.2692	0.6291\\
0.2739	0.6163\\
0.2786	0.6034\\
0.2833	0.5902\\
0.2881	0.5768\\
0.2928	0.5632\\
0.2975	0.5493\\
0.3022	0.5353\\
0.3069	0.521\\
0.3117	0.5065\\
0.3164	0.4918\\
0.3211	0.4769\\
0.3258	0.4617\\
0.3306	0.4463\\
0.3353	0.4308\\
0.34	0.415\\
0.3447	0.3989\\
0.3494	0.3827\\
0.3542	0.3662\\
0.3589	0.3495\\
0.3636	0.3326\\
0.3683	0.3155\\
0.3731	0.2982\\
0.3778	0.2806\\
};
\addplot [color=black, forget plot]
  table[row sep=crcr]{%
0.3778	0.2806\\
0.3815	0.2665\\
0.3853	0.2522\\
0.3891	0.2377\\
0.3929	0.2231\\
0.3976	0.2047\\
0.4023	0.186\\
0.407	0.1671\\
0.4117	0.1479\\
0.4165	0.1286\\
0.4212	0.109\\
0.4259	0.0892\\
0.4306	0.0692\\
0.4346	0.0521\\
0.4386	0.0349\\
0.4426	0.0175\\
0.4466	0\\
};
\addplot [color=black, forget plot]
  table[row sep=crcr]{%
0.4466	0\\
0.4466	0.0002\\
0.4467	0.0005\\
0.4468	0.0007\\
0.4469	0.001\\
0.447	0.0012\\
0.4473	0.0025\\
0.4477	0.0037\\
0.4481	0.005\\
0.4485	0.0062\\
0.4542	0.0248\\
0.4561	0.0309\\
0.4621	0.0501\\
0.4681	0.0689\\
0.4741	0.0873\\
0.4771	0.0964\\
0.4801	0.1054\\
0.4831	0.1143\\
0.4891	0.1319\\
0.4951	0.1491\\
0.5011	0.1659\\
0.5041	0.1742\\
0.5071	0.1824\\
0.5131	0.1986\\
0.5161	0.2065\\
0.5221	0.2221\\
0.5251	0.2298\\
0.5281	0.2374\\
0.5311	0.2449\\
0.5341	0.2523\\
0.5371	0.2596\\
0.5431	0.274\\
0.5491	0.288\\
0.5521	0.2949\\
0.5551	0.3017\\
0.5581	0.3084\\
0.5611	0.315\\
0.5641	0.3215\\
0.5648	0.3229\\
0.5654	0.3242\\
0.566	0.3256\\
0.5667	0.327\\
};
\addplot [color=black, forget plot]
  table[row sep=crcr]{%
0.5667	0.327\\
0.5714	0.3369\\
0.5761	0.3467\\
0.5808	0.3562\\
0.5856	0.3655\\
0.5903	0.3746\\
0.595	0.3835\\
0.5997	0.3921\\
0.6044	0.4006\\
0.6092	0.4088\\
0.6139	0.4168\\
0.6186	0.4246\\
0.6233	0.4321\\
0.6281	0.4395\\
0.6328	0.4466\\
0.6375	0.4535\\
0.6422	0.4602\\
0.6469	0.4667\\
0.6517	0.4729\\
0.6564	0.4789\\
0.6611	0.4848\\
0.6658	0.4903\\
0.6706	0.4957\\
0.6753	0.5009\\
0.68	0.5058\\
0.6847	0.5105\\
0.6894	0.515\\
0.6942	0.5193\\
0.6989	0.5234\\
0.7036	0.5272\\
0.7083	0.5308\\
0.7131	0.5342\\
0.7178	0.5374\\
0.7225	0.5404\\
0.7272	0.5431\\
0.7319	0.5457\\
0.7367	0.548\\
0.7414	0.5501\\
0.7461	0.5519\\
0.7508	0.5536\\
0.7556	0.555\\
};
\addplot [color=black, forget plot]
  table[row sep=crcr]{%
0.7556	0.555\\
0.7584	0.5558\\
0.7599	0.5562\\
0.7613	0.5565\\
0.766	0.5574\\
0.7708	0.5582\\
0.7755	0.5587\\
0.7802	0.559\\
0.7822	0.559\\
0.7832	0.5591\\
0.7842	0.5591\\
};
\addplot [color=black, forget plot]
  table[row sep=crcr]{%
0.7842	0.5591\\
0.7855	0.5591\\
0.7861	0.559\\
0.7874	0.559\\
0.7906	0.5589\\
0.797	0.5583\\
0.8002	0.5578\\
0.8042	0.5571\\
0.8082	0.5562\\
0.8122	0.5552\\
0.8162	0.554\\
0.8202	0.5527\\
0.8242	0.5512\\
0.8282	0.5495\\
0.8322	0.5477\\
0.8362	0.5458\\
0.8403	0.5436\\
0.8443	0.5414\\
0.8483	0.5389\\
0.8523	0.5363\\
0.8563	0.5336\\
0.8603	0.5307\\
0.8643	0.5276\\
0.8683	0.5244\\
0.8723	0.521\\
0.8763	0.5174\\
0.8803	0.5137\\
0.8843	0.5099\\
0.8883	0.5059\\
0.8923	0.5017\\
0.8963	0.4974\\
0.9003	0.4929\\
0.9044	0.4882\\
0.9084	0.4834\\
0.9124	0.4785\\
0.9164	0.4734\\
0.9204	0.4681\\
0.9244	0.4627\\
0.9284	0.4571\\
0.9324	0.4513\\
0.9364	0.4454\\
0.9404	0.4393\\
0.9444	0.4331\\
};
\addplot [color=black, forget plot]
  table[row sep=crcr]{%
0.9444	0.4331\\
0.9492	0.4256\\
0.9539	0.4178\\
0.9586	0.4098\\
0.9633	0.4017\\
0.9681	0.3933\\
0.9728	0.3846\\
0.9775	0.3758\\
0.9822	0.3667\\
0.9869	0.3574\\
0.9917	0.3479\\
0.9964	0.3382\\
1.0011	0.3283\\
1.0058	0.3181\\
1.0106	0.3077\\
1.0153	0.2971\\
1.02	0.2863\\
1.0247	0.2753\\
1.0294	0.264\\
1.0342	0.2526\\
1.0389	0.2409\\
1.0436	0.229\\
1.0483	0.2168\\
1.0531	0.2045\\
1.0578	0.1919\\
1.0625	0.1792\\
1.0672	0.1662\\
1.0719	0.1529\\
1.0767	0.1395\\
1.0814	0.1258\\
1.0861	0.112\\
1.0956	0.0835\\
1.1003	0.069\\
1.105	0.0543\\
1.1097	0.0393\\
1.1144	0.0241\\
1.1163	0.0181\\
1.1181	0.0121\\
1.12	0.0061\\
1.1218	-0\\
};
\addplot [color=black, forget plot]
  table[row sep=crcr]{%
1.1218	0\\
1.1218	0.0001\\
1.1219	0.0002\\
1.122	0.0005\\
1.1221	0.0007\\
1.1222	0.001\\
1.1223	0.0012\\
1.1232	0.0033\\
1.1234	0.0041\\
1.1255	0.009\\
1.1257	0.0097\\
1.1281	0.0153\\
1.1283	0.016\\
1.1304	0.0209\\
1.1306	0.0216\\
1.1315	0.0237\\
1.1318	0.0243\\
1.1327	0.0264\\
1.1328	0.0268\\
1.1332	0.0276\\
1.1333	0.028\\
};
\addplot [color=black, forget plot]
  table[row sep=crcr]{%
1.1333	0.028\\
1.1357	0.0336\\
1.1387	0.0405\\
1.1416	0.0473\\
1.1446	0.0541\\
1.1476	0.0607\\
1.1523	0.0712\\
1.157	0.0814\\
1.1617	0.0914\\
1.1665	0.1011\\
1.1712	0.1107\\
1.1759	0.12\\
1.1806	0.1291\\
1.1853	0.138\\
1.1901	0.1467\\
1.1948	0.1552\\
1.1995	0.1634\\
1.2042	0.1714\\
1.209	0.1792\\
1.2137	0.1868\\
1.2184	0.1942\\
1.2231	0.2013\\
1.2278	0.2082\\
1.2326	0.215\\
1.2373	0.2214\\
1.242	0.2277\\
1.2467	0.2338\\
1.2515	0.2396\\
1.2562	0.2452\\
1.2609	0.2506\\
1.2656	0.2558\\
1.2703	0.2607\\
1.2751	0.2655\\
1.2798	0.27\\
1.2845	0.2743\\
1.2892	0.2784\\
1.294	0.2823\\
1.2987	0.2859\\
1.3034	0.2893\\
1.3081	0.2925\\
1.3128	0.2955\\
1.3176	0.2983\\
1.3187	0.2989\\
1.3199	0.2996\\
1.3211	0.3002\\
1.3222	0.3008\\
};
\addplot [color=black, forget plot]
  table[row sep=crcr]{%
1.3222	0.3008\\
1.3249	0.3021\\
1.3275	0.3034\\
1.3302	0.3046\\
1.3328	0.3057\\
1.3376	0.3076\\
1.3423	0.3092\\
1.347	0.3106\\
1.3517	0.3118\\
1.3564	0.3128\\
1.3612	0.3135\\
1.3659	0.3141\\
1.3706	0.3144\\
1.3728	0.3144\\
1.3739	0.3145\\
1.375	0.3145\\
};
\addplot [color=black, forget plot]
  table[row sep=crcr]{%
1.375	0.3145\\
1.3769	0.3145\\
1.3776	0.3144\\
1.3782	0.3144\\
1.3814	0.3143\\
1.3878	0.3137\\
1.391	0.3132\\
1.3944	0.3126\\
1.3978	0.3119\\
1.4012	0.3111\\
1.4046	0.3102\\
1.4114	0.308\\
1.4148	0.3067\\
1.4182	0.3053\\
1.4216	0.3038\\
1.425	0.3022\\
1.4284	0.3005\\
1.4352	0.2967\\
1.4386	0.2946\\
1.442	0.2924\\
1.4454	0.2901\\
1.4488	0.2877\\
1.4523	0.2852\\
1.4557	0.2826\\
1.4625	0.277\\
1.4659	0.274\\
1.4693	0.2709\\
1.4727	0.2677\\
1.4761	0.2644\\
1.4829	0.2574\\
1.4897	0.25\\
1.4931	0.2461\\
1.4965	0.2421\\
1.4999	0.238\\
1.5027	0.2345\\
1.5083	0.2273\\
1.5111	0.2236\\
};
\addplot [color=black, forget plot]
  table[row sep=crcr]{%
1.5111	0.2236\\
1.5158	0.2172\\
1.5206	0.2106\\
1.5253	0.2037\\
1.53	0.1966\\
1.5347	0.1894\\
1.5394	0.1818\\
1.5442	0.1741\\
1.5489	0.1662\\
1.5536	0.158\\
1.5583	0.1496\\
1.5631	0.141\\
1.5678	0.1322\\
1.5725	0.1232\\
1.5772	0.1139\\
1.5819	0.1044\\
1.5867	0.0947\\
1.5914	0.0848\\
1.5961	0.0747\\
1.6008	0.0643\\
1.6056	0.0538\\
1.6103	0.043\\
1.615	0.032\\
1.6197	0.0207\\
1.6244	0.0093\\
1.6254	0.007\\
1.6263	0.0047\\
1.6273	0.0023\\
1.6282	-0\\
};
\addplot [color=black, forget plot]
  table[row sep=crcr]{%
1.6282	0\\
1.6282	0.0001\\
1.6283	0.0001\\
1.6283	0.0002\\
1.6285	0.0005\\
1.6286	0.0007\\
1.6287	0.001\\
1.6289	0.0012\\
1.6295	0.0025\\
1.6309	0.0049\\
1.6316	0.0062\\
1.6334	0.0094\\
1.6351	0.0127\\
1.6387	0.0191\\
1.6441	0.0284\\
1.6459	0.0314\\
1.6477	0.0345\\
1.6495	0.0374\\
1.6513	0.0404\\
1.6567	0.0491\\
1.6603	0.0547\\
1.6621	0.0574\\
1.6639	0.0602\\
1.6657	0.0629\\
1.6674	0.0655\\
1.6692	0.0682\\
1.6728	0.0734\\
1.6782	0.0809\\
1.68	0.0833\\
1.6818	0.0858\\
1.6836	0.0881\\
1.6854	0.0905\\
1.6908	0.0974\\
1.6944	0.1018\\
1.6962	0.1039\\
1.6971	0.1051\\
1.6981	0.1062\\
1.699	0.1073\\
1.7	0.1085\\
};
\end{axis}
\end{tikzpicture}%
%
% Note: Make sure that there are no empty lines within the document environment for standalone to properly crop the image
%
\end{document}